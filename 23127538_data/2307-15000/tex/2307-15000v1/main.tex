\documentclass[letterpaper,12pt]{article}
\usepackage{listings}
\usepackage{multirow}
\usepackage{float}
% \usepackage{physics}
\usepackage{xparse}
% \usepackage{ upgreek }
\usepackage{ amssymb }
\usepackage{amsmath}  % improve math presentation
\usepackage{graphicx} % takes care of graphic including machinery
\usepackage[margin=1in,letterpaper]{geometry} % decreases margins
\usepackage{cite} % takes care of citations
\usepackage[final]{hyperref} % adds hyper links inside the generated pdf file
\hypersetup{
	colorlinks=true,       % false: boxed links; true: colored links
	linkcolor=blue,        % color of internal links
	citecolor=blue,        % color of links to bibliography
	filecolor=magenta,     % color of file links
	urlcolor=blue         
}
\usepackage{blindtext}
\usepackage{array}% http://ctan.org/pkg/array
\usepackage{enumitem}
\usepackage{tikz}
\usepackage{graphicx,booktabs,array}
\usepackage[section]{placeins}
\newcommand\ddfrac[2]{\frac{\displaystyle #1}{\displaystyle #2}}
% \renewcommand \thesection{\Roman{section}}
% \renewcommand{\thesubsection}{\thesection.\Roman{subsection}}
\setlength{\abovedisplayskip}{0pt}
\setlength{\belowdisplayskip}{0pt}
\setlength{\abovedisplayshortskip}{0pt}
\setlength{\belowdisplayshortskip}{0pt}
\setlength{\intextsep}{5pt}
\setlength{\intextsep}{5pt}
%++++++++++++++++++++++++++++++++++++++++
\NewDocumentCommand{\codeword}{v}{%
\texttt{\textcolor{blue}{#1}}%
}
\lstset{language=C,keywordstyle={\bfseries \color{blue}}}
\begin{document}

\title{Radiation in a Moving Cavity}
\author{Refath Bari}
\date{\today}
\maketitle

\begin{abstract}
 A new formula for relativistic reflection from an inclined mirror is introduced. We use this formula to rigorously prove that a moving clock remains a clock. Numerical analysis of the long-term path of radiation in a cavity is also presented.\end{abstract}

\section{Introduction}
How does light reflect from a mirror? Simple: $\theta_i=\theta_r$. But how does light reflect from a mirror that is \textit{moving} at a constant velocity $v$? This is the well-known moving mirror problem, and it was originally solved by Einstein in his famous 1905 paper on Special Relativity \cite{15}. However, what if the mirror is \textit{inclined} at an angle $\theta$? This problem was recently solved by Gjurchinovski, starting from classical assumptions\cite{9}. In this paper, we introduce a new formula for relativistic reflection from an inclined mirror, derived from fully relativistic assumptions. Our formula is shown to reduce to three well-known relations: In the limit that the mirror is vertical ($\theta=90^{\circ}$), our equation reduces to Einstein's formula \cite{15}. In the limit that the mirror is stationary ($v=0$), our equation reduces to Euclid's formula \cite{maxOptics}. Lastly, our equation is in full agreement with Gjurchinovski's formula for inclined mirrors for $0^{\circ}\leq \theta \leq 90^{\circ}$\cite{2}. A corollary of our result, that a moving clock remains a clock, is subsequently proven in two ways: heuristically, via the postulates of special relativity and rigorously, via the relation derived in this paper. Finally, a neural network is trained to predict the long-term path of a beam of light in a cavity. These numerical results are presented, with the network achieving $53\%$ accuracy in determining the correct long-term optical path, significantly better than random chance ($\sim 33\%$). The calculations presented in this paper are suitable for a graduate course on Optics.

The paper is organized as follows. Section (\ref{Analysis}) focuses on the theoretical analysis of periodic optical paths. In particular, (\ref{HeuristicArgument}) gives a heuristic argument for the invariance of a periodic optical path between reference frames. (\ref{BariFormula}) presents a new formula for relativistic reflection from inclined mirrors. (\ref{RigorousArgument}) uses this formula to rigorously argue for the invariance of periodicity. Section (\ref{NumericalAnalysis}) focuses on the numerical analysis of periodic paths. (\ref{Algorithm}) discusses the architecture of the neural network. (\ref{NumericalResults}) presents the results of the neural network. Section (\ref{Discussion}) discusses the implications of our results in the context of the extensive existing literature. (\ref{MovingMirrorsDisc}) summarizes previous work on the moving mirrors problem. Finally, (\ref{PeriodicPathSignificance}) discusses the physical significance of a periodic optical path. 
\section{Theoretical Analysis} \label{Analysis}
% Figure environment removed
We propose the following thought experiment: consider an isosceles right-triangular train car with side length $l$ and a small hole in its hypotenuse (See Fig.\ref{fig:setup}). There is a passenger in the train car, as well as a stationary observer outside. There is a flashlight in the train, next to the passenger's head. The train's sides are constructed of idealized mirrors. The train is moving at a constant velocity $v$, appreciably close to $c$.

We are motivated to choose a triangular train for two reasons. First, so as to illuminate the intimate relationship between the problem of optical periodicity with the law of relativistic reflection -- while maintaining as simple of an analysis as possible. This opportunity is uniquely conferred by a cavity of a triangular shape. Second, we were inspired by a closely related classical mechanics problem, the Triangular Billiards Problem (See Section \ref{HeuristicArgument}). 

The passenger sees his own reflection in the mirror, by virtue of the periodic path taken by the beam of the light, as shown in Fig. \ref{fig:setup}. Furthermore, the passenger in the train car uses a radiometer to find that there is no radiation in his frame. However, the stationary observer's radiometer measures radiation leaking out of the triangular train. In fact, the observer predicts that the train emits blackbody radiation out of the small hole, with a blackbody temperature $T$. Clearly, this cannot be so, as it violates Einstein's demand that all reference frames must make the same physical predictions. Thus, a path that is periodic in the passenger's frame must also be periodic in the rest observer's frame. We proceed to prove this claim by two methods: heuristically via the postulates of Special Relativity and rigorously via the law of reflection. We briefly note that the physical significance of a periodic path is two-fold: First, an observer stationed at the initial position and direction of the periodic path will see their own reflection. Second, the only alternative to a periodic path is a fully ergodic path which fills the entire triangular cavity with radiation (See Fig. \ref{NeuralNet}). We now proceed to heuristically argue that a periodic path remains periodic, regardless of the inertial reference frame. In other words, there cannot be a conflict between reference frames wherein one observer records a periodic path, and another records an ergodic path.

\subsection{Heuristic Argument for Period Invariance}\label{HeuristicArgument}
% Figure environment removed
We designate the passenger (moving observer who sees train at rest) to be in frame $S$ and the bystander (rest observer who sees train moving) to be in frame $S'$. First, let us investigate the situation from the passenger's perspective. In Frame $S$, the passenger observes a periodic path because the light beam pursues the following trajectory. Note that points $A$, $B$, and $B'$ are the midpoints of their respective sides, as shown in Fig. \ref{fig:setup}.
\begin{enumerate}
    \item Light is emitted from $A$ and hits $O$ at $\theta_i=45^{\circ}$. 
    \item Light reflects off the hypotenuse at $\theta_r=45^{\circ}$ and hits $B$ along the normal.
    \item Light reflects from $B$ and hits $O$, reflecting at $45^{\circ}$ again and returning to $A$. 
\end{enumerate}

Now, let us investigate the situation from the stationary observer's perspective, in which the triangular train seems to be moving at a constant velocity $v$. In Frame $S'$, we will find that the observer does not see the same angles, but nevertheless records a periodic trajectory. The angle of incidence will change from $\theta_i=45^{\circ}$, since the lower leg is contracted to $\ell'$ along the direction of movement. We now have $\theta_i=\tan^{-1}\left(\frac{\ell'}{\ell} \right)=\tan^{-1}\left( \frac{1}{\gamma}\right)$.
To determine $\theta_r$, we employ a strategy similar to the classic light clock time dilation proof. The time and distance required for the photon to first hit the hypotenuse is given by 
\begin{align*}
&d_{p} =ct\ \mathbf{for\ photon}, \ d_{T} =vt+\frac{\ell '}{2} \ \mathbf{for\ triangle's\ hypotenuse}\\
&d_{p} =d_{T}\rightarrow ct=vt+\frac{\ell '}{2}\rightarrow t=\frac{\ell '}{2( c-v)}\rightarrow d=v\Delta t=\frac{\ell 'v}{2( c-v)}
\end{align*}

To calculate the reflected angle, we must first calculate the angle $\phi$ shown in Fig. \ref{reflectedPhi}. The photon travels along the hypotenuse to reach $B'$, which gives
$$\left(\frac{\ell }{2}\right)^{2} +( v\Delta t)^{2} =( c\Delta t)^{2}  \rightarrow ( \Delta t)^{2} =\frac{( \ell /2)^{2}}{\left( c^{2} -v^{2}\right)}\rightarrow \Delta t=\frac{( \ell /2)}{c\sqrt{1-\left(\frac{v}{c}\right)^{2}}} \rightarrow \therefore \Delta t=\frac{\ell /2}{c} \gamma $$
$$\phi =\tan^{-1}\left(\frac{\frac{v( \ell /2)}{c\sqrt{1-\left(\frac{v}{c}\right)^{2}}}}{( \ell /2)}\right) =\tan^{-1}\left(\frac{\frac{v( \ell /2)}{c\sqrt{1-\beta ^{2}}}}{( \ell /2)}\right) =\tan^{-1}\left(\frac{v( \ell /2)}{c( \ell /2)\sqrt{1-\beta ^{2}}}\right) \rightarrow \phi=\tan^{-1}( \beta \gamma )$$
We thus find that the reflected angle is
\begin{equation}
    \boxed{\theta _{r} =\frac{\pi }{2} -\tan^{-1}\left(\frac{1}{\gamma }\right) +\tan^{-1}( \beta \gamma )} \label{ThoughtEquation}
\end{equation}
This is in exact agreement with the formula obtained by Gjurchinovski (2004) \cite{2} up to the Jaffe Forward Reflection Limit of $\theta_r=90^{\circ}$, for a mirror inclined at $\phi=45^{\circ}$ in Frame $S$, with an angle of incidence $\theta_i=\tan^{-1}\left(\frac{1}{\gamma} \right)$: 
$$\theta_r =\frac{\pi }{2} -\tan^{-1}\left(\frac{1}{\gamma }\right) +\tan^{-1}( \beta \gamma )=\cos^{-1}\left(\frac{-2\frac{v}{c}\sin \phi +\left( 1+\left(\frac{v}{c}\sin \phi \right)^{2}\right)\cos \theta _{i}}{1-2\frac{v}{c}\sin \phi \cos \theta _{i} +\left(\frac{v}{c}\sin \phi \right)^{2}}\right)$$

We now briefly discuss where our selected example (right isosceles triangle in Frame $S$) falls in the phase space of all possible triangles. Fig. \ref{phaseSpace} depicts all configurations of triangles with periodic orbits. Fagnano demonstrated that all acute triangles have periodic orbits \cite{27}. Thus, $\alpha, \beta, \gamma < 90^{\circ}=\frac{90}{180}=0.5$ are shaded in blue in the phase space below. Holt \cite{31} and Galperin \cite{32} showed that for right triangles with angles that are rational multiples of $\pi$, all trajectories that start perpendicular to a leg will be periodic (except those which hit the vertices of the triangle). This is shown by a blue triangle enveloping Fagnano's blue region. We make similar demarcations for developments by Masur \cite{30}, Schwartz \cite{29}, and Tokarsky et. al. \cite{28}. The triangle of interest for this paper, an isosceles right triangle, is marked on the inscribed blue triangle by a red point. Transitioning from Frame $S$ to Frame $S'$ (stationary train to a moving train) transforms our red point to another point that \textit{remains }on the inscribed triangle, as we still have a right triangle, but one that is no longer isosceles due to length contraction. In other words, we have a right isoceles triangle in $S$, but only a right triangle in $S'$.
% Figure environment removed
\subsection{Relativistic Reflection from an Inclined Mirror}\label{BariFormula}
In the interest of a fully self-contained treatment, we now present an equation for the relativistic law of reflection for a moving mirror inclined at an angle $\theta$. This is the first derivation of its form, to the author's best knowledge. A similar approach has been employed by A. Gjurchinovski \cite{25}, but starting with non-relativistic assumptions, without an inclined mirror, and without leveraging Hamilton's optomechanical analogy \cite{34}.
% Figure environment removed
We have rotated our coordinate system such that the $x$ axis is along the normal to the mirror, and the $y$ axis is along the plane of the mirror. The initial and final momenta of the beam of light is given by $E=pc$:
\begin{gather*}
\text{Light}
\begin{cases}
  p_x: \frac{hf}{c}\cos{\alpha} \\
  p_y: -\frac{hf}{c}\sin{\alpha}
\end{cases}
\begin{cases}
  p'_x: -\frac{hf'}{c}\cos{\beta} \\
  p'_y: -\frac{hf'}{c}\sin{\beta}
\end{cases}
\end{gather*}
Likewise for the mirror of mass $M$ traveling at a constant velocity $v$, we have 
\begin{gather*}
\text{Mirror}
\begin{cases}
  P_x: Mv\gamma \sin\theta \\
  P_y: -Mv\gamma \cos\theta
\end{cases}
\begin{cases}
  P'_x: M'v'\gamma' \sin\theta \\
  P'_y: -M'v'\gamma' \cos\theta
\end{cases}
\end{gather*}
By the conservation of momentum and energy, we immediately have
\begin{align}
  &p_x+P_x=p'_x+P'_x \rightarrow   
  \frac{hf}{c}\cos{\alpha} + Mv\gamma \sin\theta = -\frac{hf'}{c}\cos{\beta} + M'v'\gamma' \sin\theta \label{eq:HorMomentumConservation}
  \\
  &p_y+P_y=p'_y+P'_y \rightarrow  
  -\frac{hf}{c}\sin{\alpha} + -Mv\gamma \cos\theta = -\frac{hf'}{c}\sin{\beta} + -M'v'\gamma' \cos\theta \label{eq:2}
  \\
  &E_i=E_f \rightarrow hf + \sqrt{(Mc^2)^2+(Pc)^2} = 
     hf' + \sqrt{(M'c^2)^2+(P'c)^2}
   \label{eq:EnergyConservation}
\end{align}
A billiard ball reflects as $\theta_i=\theta_r$ at a boundary precisely because the force exerted by the billiard table is normal to its surface. By the optomechanical analogy \cite{34}, we assume the contact force between the photon and mirror is also along the normal of the mirror, just as in the case for a billiard (i.e., no tangential forces). This implies that the vertical momentum of each the photon and the mirror remains invariant before and after the collision.  
\begin{align}
  &p_y=p'_y \rightarrow -\frac{hf}{c}\sin\alpha = -\frac{hf'}{c}\sin\beta \rightarrow     f\sin\alpha = f'\sin\beta \rightarrow f' = f \cdot \frac{\sin\alpha}{\sin\beta}
 \label{eq:firstFreq}
\end{align}
We seek an equation expressing the reflected angle as a function of the incident angle, velocity, and angle of inclination of the mirror: $\beta = \beta(\alpha, v, \theta)$. We will use \eqref{eq:EnergyConservation} to express $f'$ in terms of $f$. Setting this equal to \eqref{eq:firstFreq} will lead to cancelling $f'$ from both sides, resulting in the desired relation for $\beta$. We begin by rewriting \eqref{eq:EnergyConservation} in terms of the relativistic kinetic energy.
\begin{equation}
hf = hf' + \sqrt{(M'c^2)^2+(M'v'\gamma'c)^2} -Mc^2\gamma\label{eq:1}
\end{equation}
From conservation of momentum in the horizontal direction \eqref{eq:HorMomentumConservation}, we have
$$  p_x+P_x=p'_x+P'_x \rightarrow   
  \frac{hf}{c}\cos{\alpha} + Mv\gamma \sin\theta = -\frac{hf'}{c}\cos{\beta} + M'v'\gamma' \sin\theta
$$
Letting $\mathcal{K}=(f\cos\alpha+f'\cos\beta)$, we find that the final energy of the mirror is 
\begin{equation}
M'v'\gamma'=\frac{h}{c\sin\theta}(f\cos\alpha+f'\cos\beta)+Mv\gamma \rightarrow M'v'\gamma'c=\frac{h}{\sin\theta}\mathcal{K}+Mv\gamma c\label{eq:2}
\end{equation}
\begin{equation}
M'c^2=\frac{c}{v'\gamma'}\left\{\frac{h}{\sin\theta}(f\cos\alpha+f'\cos\beta)+Mv\gamma c\right\}\rightarrow M'c^2=\frac{c}{v'\gamma'}\left\{\frac{h}{\sin\theta}\mathcal{K}+Mv\gamma c \right\}\label{eq:3}
\end{equation}
Substituting \eqref{eq:2} and \eqref{eq:3} into the Conservation of Energy \eqref{eq:1}, we find
\begin{equation}
hf = hf' + \left(\frac{h}{\sin\theta}\mathcal{K} + Mv\gamma c\right)\sqrt{\left(\frac{c}{v'\gamma'}\right)^2 +1}-Mc^2\gamma\label{eq:4}
\end{equation}
Our aim is now to simplify that second term, in particular the radical. 
$$\sqrt{\left(\frac{c}{v'\gamma'} \right)^2 + 1}=\sqrt{\frac{c^2}{(v+\Delta v)^2}\left(1-\left[\frac{v+\Delta v}{c} \right]^2 \right)+1}=\sqrt{\frac{c^2}{(v+\Delta v)^2}-1+1}=\frac{c}{v+\Delta v}$$
Going back to our equation for the conservation of energy \eqref{eq:4}, we may now write
\begin{equation}hf = hf' + \left(\frac{h}{\sin\theta}\mathcal{K} + Mv\gamma c\right)\frac{c}{v+\Delta v}-Mc^2\gamma\label{eq:5}\end{equation}
% $$hf = hf' + \frac{hc}{\sin\theta(v+\Delta v)}\mathcal{K}+Mc^2\gamma \frac{v}{v+\Delta v}-Mc^2\gamma$$
\begin{equation}
hf = hf' + \frac{hc}{\sin\theta(v+\Delta v)}\mathcal{K}+Mc^2\gamma \left( \frac{-\Delta v}{v+\Delta v}\right) \label{eq:6}    
\end{equation}
We now substitute $\mathcal{K}$ into \eqref{eq:6}:
$$hf = hf' + \frac{hc}{\sin\theta(v+\Delta v)}(f\cos\alpha+f'\cos\beta)+Mc^2\gamma \left( \frac{-\Delta v}{v+\Delta v}\right)$$
Distributing out the right hand side, we have
$$hf = hf' + \frac{hc}{\sin\theta(v+\Delta v)}f\cos\alpha+
            \frac{hc}{\sin\theta(v+\Delta v)}f'\cos\beta-Mc^2\gamma \frac{\Delta v}{v+\Delta v}$$
Let's bring all the terms not involving $f'$ to the left hand side and cancel out a factor of $h$: 
$$f\left(1-\frac{c}{v+\Delta v}\frac{\cos\alpha}{\sin\theta}\right)+\frac{1}{h}Mc^2\gamma \frac{\Delta v}{v+\Delta v}=
f'\left(1+\frac{c}{v+\Delta v}\frac{\cos\beta}{\sin\theta} \right)$$
\begin{equation}
    f' = \frac{1}{1+\frac{c}{v+\Delta v}\frac{\cos\beta}{\sin\theta}}
\left\{f\left(1-\frac{c}{v+\Delta v}\frac{\cos\alpha}{\sin\theta}\right)+\frac{1}{h}Mc^2\gamma \frac{\Delta v}{v+\Delta v} \right\} \label{eq:7}
\end{equation}
This is with absolutely no simplifications or assumptions beyond the validity of the conservation of energy and momentum. We now invoke the simplification that the mirror is massless $M=0$ and the change in its velocity due to the photon is thus $\Delta v=0$. We thus find
\begin{equation}
f'=f \left(\frac{1-\frac{c}{v}\frac{\cos \alpha}{\sin \theta}} {1+\frac{c}{v}\frac{\cos \beta}{\sin \theta}}\right) \label{eq:secondFreq}
\end{equation}
We now set our two relations for $f'$ equal, \eqref{eq:firstFreq} and \eqref{eq:secondFreq}, which gives 
\begin{equation}
    \left(\frac{v\sin\theta-c\cos \alpha} {v\sin\theta+c\cos \beta}\right) = \frac{\sin\alpha}{\sin\beta} \label{eq:almostFinal}
\end{equation}
This is the desired relation: \eqref{eq:almostFinal} is the relativistic reflection law for a moving mirror inclined at an angle $\theta$. However, we must now perform three sanity checks: 
\begin{enumerate}
    \item Between $0\leq \theta \leq 90$ (horizontal and vertical mirror) and $0<v<c$, our formula should smoothly interpolate between the standard law of reflection and the relativistic law of reflection. Further, we expect it to conform to Gjurchinovski's relation for reflection off a moving, inclined mirror\footnote{Gjurchinovski derived this relation heuristically, by replacing the velocity of the inclined mirror in Einstein's formula with the horizontal velocity $v\cos\theta$. In this case, our axes are aligned such that $v_x=v\sin\theta$.} \cite{2}. $$\cos \beta=\frac{-2 \frac{(v\sin\theta)}{c}+\left(1+\frac{(v\sin\theta)^2}{c^2}\right) \cos \alpha}{1-2 \frac{(v\sin\theta)}{c} \cos \alpha+\frac{(v\sin\theta)^2}{c^2}}$$
    \item In the upper limit that $\theta=90^{\circ}$, we have a vertical moving mirror and our equation should reduce to Einstein's original relativistic reflection formula \cite{15}.
    \item In the lower limit that $v=0$, the mirror is not moving and our equation should reduce to Euclid's Law of Reflection \cite{maxOptics}.
\end{enumerate}
We begin by ensuring that the formula interpolates correctly between $0\leq\theta\leq90$, and agrees with Gjurchinovski's relation. Indeed, upon plotting them against each other, we find that they match exactly. The only caveat is that our formula seems agree with the right side of Gjurchinovski's relation. This is a trivial consequence of our coordinate setup and may be easily resolved by negating the left side of \eqref{eq:almostFinal}. Our final relativistic reflection law for inclined mirrors is thus 
\begin{equation}
    -\left(\frac{v\sin\theta-c\cos \alpha} {v\sin\theta+c\cos \beta}\right) = \frac{\sin\alpha}{\sin\beta} \label{eq:9}
\end{equation}
It is worth solving explicitly for the angle of reflection $\beta$, so that our formula matches the format of Einstein's and Gjurchinovski's. Letting $k=(v/c)\sin\theta$, we have
\begin{equation}
        -\left(\frac{(v/c)\sin\theta-\cos \alpha} {(v/c)\sin\theta+\cos \beta}\right) = \frac{\sin\alpha}{\sin\beta} \rightarrow     
        \frac{\cos\alpha-k}{\cos\beta+k}=\frac{\sin\alpha}{\sin\beta}
 \label{eq:9}
\end{equation}
$$ \sin(\beta-\alpha) = k(\sin \alpha + \sin \beta)=2k \sin \frac{\alpha+\beta}{2}\cos \frac{\beta-\alpha}{2} $$
Expanding by trigonometric relations, we have
$$ 2\sin \frac{\beta-\alpha}{2}\cos \frac{\beta-\alpha}{2} - 2k \sin \frac{\alpha+\beta}{2}\cos \frac{\beta-\alpha}{2} =0$$
$$ 2\cos \frac{\beta-\alpha}{2}(\sin \frac{\beta-\alpha}{2} - k \sin \frac{\alpha+\beta}{2}) =0 \rightarrow \sin \frac{\beta-\alpha}{2} - k \sin \frac{\alpha+\beta}{2}=0$$
Simplifying, we finally isolate the reflected angle $\beta$ to obtain
$$(k-1)\sin \frac{\beta}{2} \cos \frac{\alpha}{2}+(k+1)\cos \frac{\beta}{2} \sin \frac{\alpha}{2}=0 \rightarrow (k-1)\tan \frac{\beta}{2} +(k+1)\tan \frac{\alpha}{2}=0$$
\begin{equation}
    \boxed{\beta = 2\arctan\left(-\frac{(v/c)\sin\phi+1}{(v/c)\sin\phi-1}\tan\left(\frac{\alpha}{2}\right)\right)}\label{eq:result}
\end{equation}
This is our final result. We have shown that it agrees with Gjurchinovski. We now proceed to the second check, that it reduces to Einstein's Formula for a vertical mirror when $\phi=90$. As the formulas do not look alike, it is unclear whether they agree. However, once when plot the two functions, we observe exact agreement, as shown in Fig. \ref{BariComparison}. Finally, for our third check, does \eqref{eq:result} reduce to Euclid's Law of Reflection for $v=0$? In that case, 
$$\beta = 2\arctan\left(\tan\left(\frac{\alpha}{2}\right) \right) \rightarrow \alpha = \beta$$
We have thus demonstrated that \eqref{eq:result} agrees with the formulas of Euclid, Einstein, and Gjurchinovski. Furthermore, we have derived this formula from basic relativistic assumptions, in conjunction with the opto-mechanical analogy. We thus have a high confidence in the fidelity of our derived relation. 
% Figure environment removed
\subsection{Rigorous Argument for Period Invariance}\label{RigorousArgument}
We will now employ the relativistic law of reflection to verify whether we obtain the same $\theta_r$ calculated heuristically in Section \ref{HeuristicArgument}. In our thought experiment, $\theta_i=\arctan \frac{1}{\gamma}$. Upon substituting this into \eqref{eq:result}, we find that the two relations match exactly: 
\begin{equation}
    2\tan^{-1}\left(-\frac{\beta\cos\left(\tan^{-1}\frac{1}{\gamma}\right)+1}{\beta\cos\left(\tan^{-1}\frac{1}{\gamma}\right)-1}\tan\left(\frac{\tan^{-1}\frac{1}{\gamma}}{2}\right)\right)
\end{equation}
\begin{equation}
    2\tan^{-1}\left(\frac{\sqrt{2-\beta^{2}}+\beta}{\sqrt{2-\beta^{2}}-\beta}\cdot\frac{\frac{1}{\gamma}}{\sqrt{2-\beta^{2}}+1}\right) = \frac{\pi }{2} -\tan^{-1}\left(\frac{1}{\gamma }\right) +\tan^{-1}( \beta \gamma )
\end{equation}
Lest there be any doubt remaining regarding the agreement of the two formulas, we plot them against each other and find exact numerical agreement, as shown in Figure \ref{ThoughtAgree}.
% Figure environment removed
\section{Numerical Analysis}\label{NumericalAnalysis}
A program was created to simulate the long-term optical trajectory of a beam of light in a triangular cavity with angles $(\phi_1, \phi_2, \phi_3)$ and that is \textit{stationary} in our chosen reference frame. This simulation creates training data for the neural network. The motivation for creating the network is simple: to attempt to numerically discover patterns in periodic optical trajectories. The neural network, a long short term memory (LSTM) network, is then trained on this data and asked to predict the next 100 letters of a billiard word. The LSTM performs at $53\%$ accuracy. The neural network consists of a boilerplate architecture: two LSTM layers, two dropout layers, and a dense layer. Ten types of trajectories were chosen from the training dataset and the network was trained ten times separately for each of the ten runs.
\subsection{Parametrization Algorithm}\label{Algorithm}
The algorithm for the program is a simple parametrization procedure described below. The beam of light is initialized at a point $P_0=(P_{0x}, P_{0y})$ with velocity vector $V_0=(V_{0x},V_{0y})$. The triangle is initialized by its vertices $\{(x_0,y_0), (x_1,y_1), (x_2,y_2)\}$ and sides $\{A_1x+B_1y=C_1, A_2x+B_2y=C_2, A_3x+B_3y=C_3\}$.
Further, the light beam's trajectory is given by $L(t)=P_0+t*V_0=(V_{0x}t+P_{0x},V_{0y}t+P_{0y})$. Consider the first side of the triangle. Substituting the beam of light's coordinates, we find
$$A_1(V_{0x}t+P_{0x})+B_1(V_{0y}t+P_{0y})=C_1 \rightarrow A_1V_{0x}t+A_1P_{0x}+B_1V_{0y}t+B_1P_{0y}=C_1$$
$$t(A_1V_{0x}+B_1V_{0y})+A_1P_{0x}+B_1P_{0y}=C_1 \rightarrow \boxed{t_1=\frac{C-A_1P_{0x}-B_1P_{0y}}{A_1V_{0x}+B_1V_{0y}}}$$
Similarly for the other two sides, we calculate $t_2$ and $t_3$. The minimum of these times $\{t_1,t_2,t_3\}$ corresponds to the side that the billiard ball or photon will hit.
\subsection{LSTM Results \& Training Losses}\label{NumericalResults}
% Figure environment removed
The training results for representative periodic and ergodic paths are shown in Figure \ref{NeuralNet}. One would expect periodic training losses to drop faster than their ergodic counterparts, but this does not seem to occur, perhaps due to network memorization issues. 

The LSTM achieves a prediction accuracy of $53\%$, which is considerably better than random guessing. Its architecture is primarily a proof of concept to demonstrate that future neural networks may be capable of detecting conserved quantities in billiard systems. After increasing the size of the training data from 100 letters to 591 letters, the prediction accuracy improved significantly, from accurately predicting the orbit up to $10$ letters to $20$ letters. This suggests that improving the quality of the training set alone (i.e., by increasing the number, size, and diversity of words) would improve the network's accuracy. 

In addition, hyperparameter optimization may result in even better predictions. Adjusting the number of epochs seems to be key, as the network may begin memorizing if the number of epochs is too high, but performing poorly if too low. A variable learning rate (LR) may also prove beneficial, as it will decrease the LR as the training loss approaches a minimum. In addition, instead of retraining the network from scratch for every new orbit, it may be feasible to implement classes which help the network classify the orbit as either periodic or ergodic. However, this would run against the intended function of the network as a symbolic predictor of the orbit, not a classifier. A validation set in addition to the training set may streamline the process of testing the LSTM in the future. Future networks can address the possibility of analyzing the triangular table to find hidden symmetries, and thus find any conservation laws in the dynamical system. All these additional features are outside the scope of this paper, which seeks to merely advance the notion of an neural network analysis of optical periodic paths as a proof of concept.

\section{Discussion} \label{Discussion}
\subsection{Moving Mirrors Problem}\label{MovingMirrorsDisc}
We now supply a brief literature review of the moving mirrors problem. Bolotovskii and Stolyarov investigated reflection from a moving medium via Maxwell's Equations \cite{3}. Gjurchinovski investigated reflections via the Action Principle, but without the additional complications that the mirror is inclined at some angle $\theta$ and encloses the radiation in a cavity \cite{2}\cite{7}\cite{9}. Galli and Amiri considered the momentum exchange between the photon and mirror \cite{1}\cite{12}. Maesumi showed that relativistic reflection from a vertical mirror is equivalent to reflection from a stationary hyperbolic mirror \cite{8}.  
\subsection{Physical Significance of Periodic Path}\label{PeriodicPathSignificance}
We now briefly address the physical significance of a periodic orbit. First, it implies that an observer standing in the room at the initial position and direction of the incident light beam would see their own reflection. Second, the alternative is complete ergodicity in that the light beam would fill all of space and there would be no "shadows" which the light beam does not illuminate. In this paper, we posed the following thought experiment: If a triangular train is moving at a constant velocity $v$, would a stationary observer agree that the passenger inside can see his own reflection? In other words, would a path that is periodic inside the train remain periodic for a stationary observer outside? We have answered this question both heuristically and rigorously, via the postulates of special relativity and a new relation for relativistic reflection. We have also presented numerical results for a neural network which achieves $53\%$ accuracy in predicting the long-term optical path of a beam of light. 

\section{Acknowledgements}
The author thanks Richard Schwartz for suggesting a neural network to predict orbits. The author also thanks Javier Hasbun, Robert Young and Carl Mungan for illuminating discussions.
\bibliographystyle{unsrt}
\bibliography{refs}

\end{document}