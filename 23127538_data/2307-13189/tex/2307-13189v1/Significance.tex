\documentclass[aip,jcp,preprint]{revtex4-1} 
\usepackage{subcaption}
\usepackage{graphicx}
\usepackage{enumitem}

\usepackage{mathtools}
\usepackage{graphicx}   
\usepackage{color}
\usepackage{ulem}
\usepackage{caption}
\usepackage{subcaption}
\usepackage{gensymb}
\usepackage{bm}

\usepackage{algorithm}
\usepackage{algpseudocode}
\usepackage{algorithmicx}
\renewcommand{\algorithmicrequire}{\textbf{Input:}}
\renewcommand{\algorithmicensure}{\textbf{Output:}}

\usepackage{array}
\usepackage{booktabs}
\usepackage{multirow}
\newcommand{\head}[1]{\textnormal{\textbf{#1}}}
\newcommand{\normal}[1]{\multicolumn{1}{l}{#1}}
\usepackage{float}

\newcommand{\revone}{\textcolor{black}}
\newcommand{\revtwo}{\textcolor{black}}

\begin{document}

\section*{Significance}
Rare events in aqueous solutions, such as self-assembly, protein folding, and crystal nucleation, present theoretical and computational difficulties in describing solvent-mediated effects on long time scales. We use local molecular field theory to encode the long-range Coulomb interactions of complex aqueous systems in solute-solute correlations, capturing their detailed dielectric screening. Meanwhile, advanced sampling methods efficiently handle the dynamics controlled by short-range interactions. These two complementary approaches are combined to explore the collective response to solvent-mediated interactions in the nucleation of urea in water. We reveal the roles of short and long-range interactions in stabilizing the experimental primary form of urea in water and in the intricate nucleation mechanism, for which a deep-learning based method is employed to perform kinetic analysis.

\end{document}