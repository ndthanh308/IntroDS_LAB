\documentclass[aip,jcp,preprint]{revtex4-1} 
\usepackage{subcaption}
\usepackage{graphicx}
\usepackage{enumitem}

\usepackage{mathtools}
\usepackage{graphicx}   
\usepackage{color}
\usepackage{ulem}
\usepackage{caption}
\usepackage{subcaption}
\usepackage{gensymb}
\usepackage{bm}

\usepackage{algorithm}
\usepackage{algpseudocode}
\usepackage{algorithmicx}
\renewcommand{\algorithmicrequire}{\textbf{Input:}}
\renewcommand{\algorithmicensure}{\textbf{Output:}}

\usepackage{array}
\usepackage{booktabs}
\usepackage{multirow}
\newcommand{\head}[1]{\textnormal{\textbf{#1}}}
\newcommand{\normal}[1]{\multicolumn{1}{l}{#1}}
\renewcommand{\refname}{Supplementary References}
\usepackage{float}

\newcommand{\revone}{\textcolor{black}}
\newcommand{\revtwo}{\textcolor{black}}

\captionsetup[figure]{labelfont=bf,textfont=normalfont,justification=raggedright}
\usepackage{booktabs}
\newcommand{\ra}[1]{\renewcommand{\arraystretch}{#1}}

\cfoot{\thepage}

\renewcommand{\thepage}{S\arabic{page}}  
\renewcommand{\thesection}{S\arabic{section}}   
\renewcommand{\thetable}{S\arabic{table}}   
\renewcommand{\thefigure}{S\arabic{figure}}
\renewcommand{\theequation}{S\arabic{equation}}

\begin{document}
\title{\LARGE Supplementary Information}


 \author{Renjie Zhao}
 \affiliation{Chemical Physics Program and Institute for Physical Science and Technology, University of Maryland, College Park 20742, USA.}

\author{Ziyue Zou}
 \affiliation{Department of Chemistry and Biochemistry, University of Maryland, College Park 20742, USA.}


 \author{John D. Weeks*}
 \email{jdw@umd.edu}

\affiliation{Institute for Physical Science and Technology and Department of Chemistry and Biochemistry, University of Maryland, College Park 20742, USA.}

 \author{Pratyush Tiwary*}
 
 \email{ptiwary@umd.edu}
 \affiliation{Institute for Physical Science and Technology and Department of Chemistry and Biochemistry, University of Maryland, College Park 20742, USA.}


\maketitle

\section{Averaged intermolecular angles}
For an arbitrary solute molecule $i$, we have
\begin{equation}
\hat{\theta}(i)=\frac{\sum_{j}\sigma(r_{ij})\frac{1}{2}[(\pi-2\theta) \tanh\funcapply(5\theta-9.25)+\pi]}{\sum_{j}\sigma(r_{ij})},
\label{eq:angle}
\end{equation}
where $\theta$ is the intermolecular angle between characteristic vectors on molecule $i$ and molecule $j$, and $\sigma(r_{ij})$ is a switching function of intermolecular distance $r_{ij}$. In order to eliminate the mirror image symmetry, the hyperbolic tangent switching function is implemented in Eq.~\ref{eq:angle}. We then compute the mean of $\hat{\theta}(i)$ over the group of solutes to obtain the averaged intermolecular angles $\bar{\theta}_1$ and $\bar{\theta}_2$, for which the subscripts refer to the specific intermolecular angles defined in the context of pair orientational entropy. $\mu^2_{\theta_1}$ and $\mu^2_{\theta_2}$ are computed as the second moments of the $\hat{\theta}_1(i)$ and $\hat{\theta}_2(i)$ distributions.

\section{Coordination number}
We calculate the coordination number of urea molecules by the following continuous and differentiable expression,
\begin{equation}
c(i)=\sum_{j}\frac{1-(r_{ij}/r_c)^6}{1-(r_{ij}/r_c)^{12}},
\label{eq:CN}
\end{equation}
where $r_{ij}$ is the distance between reference sites of molecules $i$ and $j$, and $r_c$ is the cutoff. $N_{8+}$, $N_{11+}$, which are the populations of molecules with coordination numbers greater than 8 and 11, and the second moment of coordination numbers $\mu^2_c$ are derived from the $c(i)$ distribution.

\newpage
\section{Distributions of $N_{8+}$}
% Figure environment removed
In Fig. S1 are the distributions of $N_{8+}$ when the system transitions between the liquid-like basin and the primary crystal basin for the SS model (top), the GT model (middle), and the full model (bottom). Based on the criterion that the liquid-like configurations from WTmetaD simulations count toward the liquid state when their $N_{8+}$ values are within the three-sigma limits of the liquid state $N_{8+}$ distribution drawn from a 50 ns unbiased simulation and otherwise count toward the dense liquid state, the transitions between the liquid-like basin and the primary crystal basin take place directly from or to the liquid state in $8.7\%$ of the total 46 events for the SS model, $4.5\%$ of the total 44 events for the GT model, and $8.6\%$ of the total 58 events for the full model. The other transitions are intermediated by the dense liquid state. The results from the three models consistently indicate a predominant two-step nucleation mechanism. 


The distributions of $N_{8+}$ beyond the dense liquid state threshold ($\sim9.61$) do not necessarily take similar shapes for the three models. This is because there can exist one or multiple clusters in different configurations and $N_{8+}$ is not accurately proportional to the size of the cluster.



\end{document}