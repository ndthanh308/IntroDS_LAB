\documentclass[prb,twocolumn,aps]{revtex4-2}
\usepackage{graphicx,amsmath,xcolor}
\usepackage{hyperref}
\bibliographystyle{apsrev4-2}
\usepackage{subcaption}
\usepackage{float}
\usepackage{svg}
%\usepackage{biblatex}
\usepackage{silence}
\WarningFilter{revtex4-2}{Repair the float}
\usepackage{mathtools}
\usepackage[T1]{fontenc}
\DeclarePairedDelimiter{\bra}{\langle}{\rvert}%
\DeclarePairedDelimiter{\ket}{\lvert}{\rangle}%
\DeclarePairedDelimiterX\innerp[2]{\langle}{\rangle}{#1\delimsize\vert\mathopen{}#2}%
\DeclarePairedDelimiterX\braket[2]{\langle}{\rangle}{#1\delimsize\vert\mathopen{}#2}%
\DeclarePairedDelimiterX\braketOP[3]{\langle}{\rangle}{#1\,\delimsize\vert\,\mathopen{}#2\,\delimsize\vert\,\mathopen{}#3}%
\DeclarePairedDelimiterX\ketbra[2]{\lvert}{\rvert}{#1\delimsize\rangle\!\delimsize\langle#2}%
\DeclarePairedDelimiterX\outerp[2]{\lvert}{\rvert}{#1\delimsize\rangle\!\delimsize\langle#2}%
\DeclarePairedDelimiterX\projector[1]{\lvert}{\rvert}{#1\delimsize\rangle\!\delimsize\langle#1}%


\begin{document}

    \title{Assessing quantum dot SWAP gate fidelity using tensor network methods}
    \author{Jacob R. Taylor}
    \author{Nathan L. Foulk}
    \author{Sankar Das Sarma}
    \affiliation{Condensed Matter Theory Center and Joint Quantum Institute, Department of Physics, University of Maryland, College Park, Maryland 20742-4111 USA}
\begin{abstract}
%Temporary Abstract
    The SWAP gate facilitates the exchange of quantum states between qubits and is integral to quantum algorithms. 
    We utilize advanced tensor network methods to explore the fidelity for repeated SWAP operations on a system comprising 20 to 100 quantum dot spin qubits.
    We incorporate valley states, valley splitting, spin-valley coupling, Zeeman splitting, and crosstalk. 
    The fidelity of SWAP gates is largely unaffected by Zeeman splitting and valley splitting, except when these parameters come into resonance.
    In addition to confirming that fidelity is positively impacted by the larger exchange couplings $J_\text{SWAP}$ in terms of the residual exchange $J_0$ and that spin-valley coupling negatively impacts fidelity, we also show that for valley eigenstates, the fidelity remains independent of the valley phase, while for generic valley states some minor corrections arise.
    We also analyze the fidelity scaling for long qubit chains without valley effects, where crosstalk represents the only error source.
\end{abstract}
\maketitle 

\section{Introduction}
%Standard Swap gate is important stuff
A standard operation in quantum computing is the SWAP gate, which allows for the exchange (or ``swapping'') of the quantum states between two qubits.
The SWAP gate has important applications in quantum computing, such as in quantum error correction~\cite{gottesman1997stabilizer}, measurement schemes~\cite{barenco1997stabilization}, and quantum state engineering~\cite{nielsen2010quantum}.
The root $\sqrt{\text{SWAP}}$ is entangling and thus, in combination with arbitrary single-qubit gates, allows for the implementation of general unitary operations, sufficient for universal quantum computation~\cite{loss1998quantum}.
This fact, combined with the SWAP gate's ubiquity within quantum error correction, makes the creation of high fidelity SWAP gates essential for building a quantum computer on any platform. 


%Repeating summary of Isotope purification from the other paper. Talks about why Silicon is Good
Si-based quantum dot spin qubits have emerged as a promising candidate for realizing quantum computers due to their long coherence times~\cite{kobayashi2021engineering,Burkard2023}.
Both $^{28}$Si and $^{30}$Si are common spin-0 isotopes, and thus it is possible to remove the decoherence arising from nuclear spin noise by isotopic purification~\cite{Witzel2010}.
In addition, electrical gate operations implemented directly from the Heisenberg interaction between different qubits, such as SWAP gates, have durations on the order of 1 ns~\cite{medford2013quantum}.
These long coherence times arising from such isotopic purification, combined with the available short gate operation times, make Si especially well-suited for hosting qubits.
Most importantly, silicon-based qubits can be easily integrated into the existing semiconductor industry, allowing for more straightforward scalability. 
A silicon-based quantum computer platform could potentially host millions of qubits in a small chip similar to existing CMOS-based integrated circuits, and there has been spectacular recent progress in producing scalable multiqubit Si circuits~\cite{neyens2023,IntelSite}. 
It is therefore both timely and important to consider SWAP gates in large spin qubit systems.

Tensor networks provide a framework for representing many-body quantum states and operators in a computationally compact and efficient way, allowing for accurate simulations of large quantum systems intractable with direct methods~\cite{biamonte2017tensor}.
Acting on matrix product states (MPS) with tensor network time evolution methods such as TEBD~\cite{daley2004time} or TDVP~\cite{haegeman2011time} provides a technique to accurately approximate the time evolution of interacting quantum systems and can be used to investigate model-intrinsic sources of error.
Systems of hundreds of qubits that would be utterly intractable through direct simulation can often be represented efficiently by restricting one's system to an approximate low-entanglement subspace using tensor networks.



Here we seek to expand previous work into the effects of valley states on spin qubit devices by directly including them within our simulation. 
Previous work into the fidelity of sequences of SWAP gates on a spin qubit chain has looked into the effects of charge noise and dissipation on the fidelity of SWAP gates~\cite{foulk2022dissipation,throckmorton2020fidelity}. 
Such work did not directly include dynamical valley states or their interactions and was done only on small-scale systems due to computational constraints in exact diagonalization. 
We intend to use state-of-the-art methods within tensor networks to accurately model the effect of the initial valley state and different experimentally relevant parameters on the fidelity of a sequence of chained SWAP gates.
Our work can address tens to hundreds of spin qubits in contrast to all earlier works on the subject.

We first introduce our spin qubit model and describe the numerical methods used to perform the calculations. 
We then present the results of those numerical calculations. 
We demonstrate the effect of valley splitting, Zeeman splitting, and SWAP exchange strength on the fidelity of SWAP operations. 
We explore the effects of the spin-valley coupling and its phase on different initial valley states. 
We conclude with an investigation into the single gate fidelity scaling up to 100 spin qubits isolating the influence of crosstalk. 


%--------

\section{Model}

We label the basis states of our model by the valley and spin degrees of freedom. We consider the lowest two valley states, corresponding to $k=\pm z$, which we label $\ket{\pm}$, along with the spin states within the valley $\ket{\uparrow}$ and $\ket{\downarrow}$.
We consider exchange-coupled spin qubits~\cite{loss1998quantum}, where the spin states serve as our computational basis.

We model the system as a one-dimensional (1D) spin chain with both valley and spin degree of freedom using the following Hamiltonian:
\begin{multline}
H=\sum_{n=1}^{L-1} J_n \left(\boldsymbol{\sigma}_n\cdot \boldsymbol{\sigma}_{n+1}+1\right)\left(\boldsymbol{\tau}_n\cdot \boldsymbol{\tau}_{n+1}+1\right)+\\ h \sum_n^L \sigma_n^{z}+ \Delta \sum_n^L \tau_n^z+\\
\frac{\gamma_1}{2}\sum_n^L\left(\tau_n^x\sigma_n^x+\tau_n^y\sigma_n^y\right)+\frac{\gamma_2}{2}\sum_n^L \left(\tau_n^y\sigma_n^x-\tau_n^x\sigma_n^y\right)
\label{hamiltonian}
\end{multline}

where $L$ is the number of qubits in the spin chain, $h$ is the spin Zeeman splitting, $\Delta$ is the valley splitting and $\gamma_1$,$\gamma_2$ are the real and imaginary parts of the spin-valley coupling $\gamma=\gamma_1+i \gamma_2$.
$\boldsymbol{\sigma}_n=(\sigma_n^x,\sigma_n^y,\sigma_n^z)$ is the $n^\text{th}$ site Pauli vector in the spin basis, while $\boldsymbol{\tau}_n$ is the same but in the valley basis.

The role of the spin-valley coupling becomes clear in the matrix representation of the single-site Hamiltonian,
\begin{equation*}
    H_n = \begin{pmatrix} 
            h + \Delta & 0 & 0 & 0 \\
            0 & h - \Delta & \gamma & 0 \\
            0 & \gamma* & \Delta - h & 0 \\
            0 & 0 & 0 & -\Delta - h 
    \end{pmatrix}.
\end{equation*}
Coupling between the $\ket{-\downarrow}$ and $\ket{+\uparrow}$ states is assumed to be negligible.
It can be shown that $\gamma$ is proportional to the valley splitting, and we can reasonably assume $\frac{|\gamma|}{\Delta} \sim 1/500$~\cite{huang2014spin}.

We perform the SWAP operation using a $\frac{\pi}{4}$ pulse Heisenberg interaction between the swapping sites. 
Ideally, the exchange coefficient should be zero for non-swapping sites. 
In reality, this is never true. 
We set the exchange coefficient $J_n$ as follows:

\begin{equation}
J_n = 
\left\{
    \begin{array}{lr}
        J_0, & \text{if } n \neq l\\
        J_{\text{SWAP}}, & \text{if } n = l
    \end{array}
\right\},
\end{equation}

where the SWAP gate is between sites $l$ and $l+1$.
We assume $J_0$, $\gamma$, $\Delta$ and $h$ all to be site independent. These nonessential assumptions are easy to relax in our method if experimental information about their site dependence is available.

We take the initial state to be a product state between an initial spin state $\ket{\psi_i}_s$ and the valley state $\ket{\psi_i}_v$ up to a normalization factor,
\begin{equation}
\ket{\psi_i}_v \propto (1-\alpha)\ket{-...-}+\alpha\ket{-_x...-_x},
\label{eqn:initialstate}
\end{equation}
where
$$\ket{-_x}=\frac{1}{\sqrt{2}}\left(\ket{+}-\ket{-}\right).$$
%$$N=\frac{1}{\sqrt{\alpha ^2+(1-2 \alpha )^2}}$$

We define the total fidelity for our sequence of SWAP gates as:
\begin{equation}
F_{\text{tot}}=\text{Tr}_s\left[\text{Tr}_v[U\rho_i U^\dagger]\text{Tr}_v[R \rho_i R^\dagger]\right],
\label{eqn:fidelitymeasure}
\end{equation}
 where $R$ is an operator which performs the SWAP gate sequence with perfect fidelity, $U$ represents the actual SWAP gate
sequence with errors, and $\rho_i=\ketbra{\psi_i}{\psi_i}$ is the initial state of the system.
$\text{Tr}_s[...]$ and $\text{Tr}_v[...]$ are the partial trace operators over spin and valley degrees of freedom, respectively.
In our case, the SWAP sequence transports a spin state from one side of the spin chain to the other.
The transport is performed through swapping sites $1\leftrightarrow2$, then $2\leftrightarrow3$, and so forth until the first spin state is transported to the end of the chain.
We use the effective single gate fidelity taken as $F=(F_{tot})^{1/(L-1)}$ to be able to compare gate sequences of different lengths.






We perform the calculations using tensor network MPS methods.
Since each physical spin qubit consists of spin and valley states, we split each into two separate two-level sites in the tensor network representation.
We arrange the two-level tensors in two rows, with the top row representing the spin states and the bottom row representing the valley states.
To construct the tensor network Hilbert space, we order the sites such that the $2j-1$ and $2j$ MPS sites map to the $j$th physical qubit's spin and valley states, respectively.
The interweaving of the two states is ideal because it minimizes the number of long-range interactions necessary within the Hamiltonian, thus improving the efficiency of the MPS algorithm.

% Figure environment removed


The system is initialized by constructing the spin state MPS and the valley state MPS independently and then interspersing them with standard tensor network operations to build a single total MPS for the entire system.
In the case of the random spin state MPS, an initial MPS of bond dimension M=10 with complex matrix elements is generated.
We perform the time evolution of the total MPS using Time Evolving Block Decimation (TEBD)~\cite{daley2004time} with SWAP gate time $T=\frac{\pi/4}{J_\text{SWAP}}$.
We map the state to a rotating reference frame to correct for the background rotation caused by the external magnetic fields.
The simple mapping is achieved by applying a set of local unitary operations to all sites $U_r=\Pi_{n=1}^L \exp(i h \sigma^z_n T)$ when computing the fidelity.
We then convert the resulting MPS into a projector matrix product operator (MPO), which we use in Eq.~\ref{eqn:fidelitymeasure}.

\section{Calculations}
To assess the effects of different parameters on the fidelity of SWAP gates, we first look at the impact of the magnitude of $J_{\text{SWAP}}$ on the fidelity.
We would expect that the larger $J_{\text{SWAP}}$ is relative to $J_0$, the larger the fidelities of the SWAP gate would be since this would trivially make the SWAP gates faster. 
Our results are shown in Fig.~\ref{fig:JSwapFidelity}. 
Unsurprisingly this holds regardless of the initial spin basis state, though it is apparent that random, likely entangled states experience dramatically higher infidelity.

It is worth noting that in all plots, we write the relevant parameters $\Delta$, $h$, $J_{\text{SWAP}}$, $\gamma_1$ and $\gamma_2$ in units of $J_0$. 
In a recent experiment, the residual exchange was measured on average to about $J_0\approx 40$ kHz~\cite{Philips2022}. 


% Figure environment removed


Looking at the effects of $h$ and $\Delta$ in Fig.~\ref{fig:DvH}, we find a significant impact on the fidelity of the SWAP sequence only when in resonance ($h\approx \Delta$).
Large values of $J_{\text{SWAP}}$ broaden the $h \approx \Delta$ resonance regime, though they also suppress the infidelity peak in such a regime.

%second fidelity for delta vs. h
% Figure environment removed



% Figure environment removed



The magnitude $|\gamma|=\sqrt{\gamma_1^2+\gamma_2^2}$ of the spin valley coupling significantly impacts the fidelity of the SWAP gate.
As expected, a larger $\gamma$ is detrimental to the SWAP gates' reliability due to local spin-valley state mixing.
This result can be seen in Fig.~\ref{fig:SGGammaMag}. 

% Figure environment removed
We also examine the effect of different phases of the spin valley coupling $\theta=\text{arg}(\gamma)$.
What may seem initially surprising is that for the $\ket{-\ldots-}$ initial valley state ($\alpha = 0$), there is no valley phase dependence on fidelity, and only $|\gamma|$ has any effect.
If we initialize one or both of the spin or valley states to a Z-basis state, the effect of $\text{arg}(\gamma)$ reduces to a global phase.
However, there is a phase dependence when the valley state is initially $\ket{-_x\ldots -_x}$. 
To investigate this, we write the initial valley state in the $\alpha$ dependent form of Eq.~\ref{eqn:initialstate}, where the initial valley state is $\ket{-\ldots -}$ for $\alpha=0$ and $\ket{-_x\ldots -_x}$ for $\alpha=1$.
We show the SWAP fidelity for six random initial spin states at $\alpha=0$ and $\alpha=1$ in Fig.~\ref{fig:alpha0_and_alpha1}.
The effects of tuning $\alpha$ from 0 to 1 can be seen in Fig.~\ref{fig:SGFidelityAlpha}.
% Figure environment removed



% Figure environment removed
%Fix to alpha

To investigate the effects of crosstalk for even longer spin chains, we simulate lengths $L=[5,100]$ in Fig.~\ref{fig:Scaling}.
We find that the single gate fidelities drop to critically low values even for relatively large values of $J_\text{SWAP}$.
This is because the residual exchange $J_0$ crosstalk has more sites to entangle and more time overall to do so.

% Figure environment removed
%(Increase JSWAP max)
\section{Conclusion}
In conclusion, we investigate the effects of valley states on the fidelity of a sequence of SWAP gates in long quantum spin qubit systems (20-100 qubits).



Using state-of-the-art tensor network methods, we can examine the large-scale behavior of a complete model of a spin qubit chain, including both spin and valley, as well as the coupling between them. 
We calculate how the fidelity of repeated SWAP gates is directly affected by factors such as valley splitting, spin-valley coupling, Zeeman splitting, and crosstalk. 

We find that the fidelity is only weakly affected by Zeeman splitting and valley splitting except when brought into resonance.
We show that for Z-basis valley states, there is no dependence on the phase of the spin-valley coupling, and even when not in the Z-basis, the effect is small.
We also investigate the scaling of the fidelity spin chains due to the effects of crosstalk.


\section{Acknowledgement}
We thank Donovan Buterakos for helpful discussions and suggestions. This work was supported by the Laboratory for Physical Sciences.


\bibliography{mainbib}
\end{document}