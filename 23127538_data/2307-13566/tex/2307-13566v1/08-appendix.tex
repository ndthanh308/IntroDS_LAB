\newpage
\section{Appendix}

\subsection{Appropriate Reliance with Imperfect XAI}
\label{formulas}

\begin{table}[H]

\caption{Overview of the newly introduced metrics when considering imperfect explanations.}
    \begin{threeparttable}
\begin{tabular}{P{2cm} m{11cm} }
 \hline
$CSR_{IC}$ & Correct self-reliance for the case where the AI gives incorrect advice and a correct explanation is one when the initial human decision is correct and the final decision is correct. \\ \hline 
$OR_{IC}$ & Over-reliance for the case where the AI gives incorrect advice and a correct explanation is one when the initial human decision is correct and the final decision is correct. \\ \hline 
$CSR_{II}$ & Correct self-reliance for the case where the AI gives incorrect advice and an incorrect explanation is one when the initial human decision is correct and the final decision is correct. \\ \hline 
$OR_{II}$ & Over-reliance for the case where the AI gives incorrect advice and an incorrect explanation is one when the initial human decision is correct and the final decision is correct. \\ \hline 
$CAIR_{CC}$ & Correct AI reliance for the case where the AI gives correct advice and a correct explanation is one when the initial human decision is incorrect and the final decision is correct. \\ \hline 
$UR_{CC}$ & Under-reliance for the case where the AI gives correct advice and a correct explanation is one when the initial human decision is incorrect and the final decision is correct. \\ \hline 
$CAIR_{CI}$ & Correct AI reliance for the case where the AI gives correct advice and an incorrect explanation is one when the initial human decision is incorrect and the final decision is correct. \\ \hline 
$UR_{CI}$ & Under-reliance for the case where the AI gives correct advice and an incorrect explanation is one when the initial human decision is incorrect and the final decision is correct. \\ \hline 
\end{tabular}
    \begin{tablenotes}
        \item[1] A correct AI explanation correpsonds with the AI's advice, no matter if the advice is correct or incorrect. For a classification task this means the following: If the AI gives incorrect advice and the explanation is correct, the explanation aligns with the incorrectly predicted class.
    \end{tablenotes}
    \end{threeparttable}

\label{mod_anal_nle_rair}

\end{table}

Following the newly introduced dimension, the calculation for RAIR and RSR are adjusted to the following:
\begin{equation}
    RSR\ (Relative\ Self-Reliance) = \frac{\sum_{i=0}^{N}{(CSR_{IC, i} + CSR_{II, i})}}{\sum_{i=0}^{N}{IA_i}}
\end{equation}
\begin{equation}
    RAIR\ (Relative\ AI\ Reliance) = \frac{\sum_{i=0}^{N}{(CAIR_{CC, i} + CAIR_{CI, i})}}{\sum_{i=0}^{N}{CA_i}}
\end{equation}
\newpage
\subsection{Bird Identification Test}
\label{bird-test-details}

The bird identification test consists of images of six images: three ``easy'' common bird species and three ``hard'' bird species. The three ``easy'' common bird species were selected with the intention that most beginning birders would be familiar with them. For the ``easy'' common bird species, participants have to identify a \textit{Downy Woodpecker}, a \textit{Herring Gull}, and a \textit{Ruby-Throated Hummingbird}.  

For the ``hard'' bird species, participants have to identify a \textit{female Hooded Warbler}, a \textit{Blue-headed Vireo}, and a \textit{Chestnut-sided Warbler}. The \textit{female hooded warbler} is chosen because it looks significantly different than a male Hooded Warbler and requires a higher level of expertise to be able to correctly identify that. The \textit{Blue-headed Vireo} is chosen because it visually looks very similar to the \textit{Philadelphia Vireo}, again requiring a higher level of expertise to correctly identify that. Lastly, the \textit{Chestnut-sided Warbler} is chosen because there are several different species in the Warbler family, and they are easy for non-experts to mix up.

Below are figures showing the performance on the bird test based on the experts and non-experts grouping we do. 

% Figure environment removed

% \subsection{Dataset Details}
% \label{dataset}

% % Figure environment removed

% \katelyn{put more details here.}
\newpage
\subsection{Moderation Analyses}
\label{mod_appendix}
\begin{table}[htbp!]

\caption{Moderation analysis of correctness of natural language explanations on RAIR with the level of expertise and assertiveness as moderators (Since the level of expertise is a three-level categorical moderator, it is split up into Z1 --- non-assertive explanations in relation to all other values --- and Z2 --- assertive explanations in relation to all other values.}
    \begin{threeparttable}
\begin{tabular}{m{1.5cm} R{1.2cm} R{1.2cm} R{1.2cm} R{1.2cm} R{1.2cm} R{1.2cm}}
\\ \hline
 & coeff & ce & Z & p & LLCI & UCLI  \\ \hline \hline
const   & 1.26 & .34 & 3.67   & .00  & .59  & 1.94     \\\hline
corr  & .57 & .53 & 1.07 & .29  & -.48  & 1.62   \\\hline
exp  & -2.12 & .33 & -6.50 & .00  & -2.77  & -1.48  \\\hline
Z1 & -.46 & .39 & -1.16 & .24  & -1.23  & .31  \\\hline   
Z2 & .00 & .39  & .00 & 1.00  & -.76  & .76  \\\hline 
exp x corr & -1.00 & .51 & -1.96  & .05  & -1.99  & .00  \\\hline 
Z1 x corr & .46 & .59  & .77 & .44  & -.70  & 1.61  \\\hline 
Z2 x corr & .19 & .58 & .33  & .74  & -.95  & 1.34 \\\hline 
\end{tabular}
    \begin{tablenotes}
        \item[1] \textit{corr} --- \textit{correctness}; \textit{exp} --- \textit{level of expertise}
    \end{tablenotes}
    \end{threeparttable}
\label{mod_anal_nle_rair}

\end{table}

\begin{table}[htbp!]

\caption{Moderation analysis of correctness of example-based explanations on RAIR with level of expertise and assertiveness as moderators (Since level of expertise is a three-level categorical moderator, it is split up into Z1 --- non-assertive explanations in relation to all other values --- and Z2 --- assertive explanations in relation to all other values.}
    \begin{threeparttable}
\begin{tabular}{m{1.5cm} R{1.2cm} R{1.2cm} R{1.2cm} R{1.2cm} R{1.2cm} R{1.2cm}}
\\ \hline
 & coeff & ce & Z & p & LLCI & UCLI  \\ \hline \hline
const   & .43 & .32 & 1.35   & .17  & -.19  & 1.05     \\\hline
corr  & 1.02   & .49  & 2.08  & .04 & .06 & 1.99   \\\hline
exp  & -1.25  & .31  & -4.09  & .00 & -1.85 & -.65  \\\hline
Z1 & .26 & .36  & .73 & .47 & -.45  & .98  \\\hline   
Z2 & .00 & .37  & .00  & 1.00 & -.72 & .72  \\\hline 
exp x corr & -1.04 & .47 & -2.21  & .03  & -1.95  & -.12  \\\hline 
Z1 x corr & -.03  & .54  & -.06 & .95 & -1.08  & 1.02  \\\hline 
Z2 x corr & -.16  & .54  & -.29  & .77 & -1.22 & .90 \\\hline 
\end{tabular}

    \begin{tablenotes}
        \item[1] \textit{corr} --- \textit{correctness}; \textit{exp} --- \textit{level of expertise}
    \end{tablenotes}
    \end{threeparttable}

\label{mod_anal_ex_rair}

\end{table}

\begin{table}[htbp!]

\caption{Moderation analysis of correctness of natural language explanations on RSR with level of expertise and assertiveness as moderators (Since level of expertise is a three-level categorical moderator, it is split up into Z1 --- non-assertive explanations in relation to all other values --- and Z2 --- assertive explanations in relation to all other values.}

    \begin{threeparttable}
\begin{tabular}{m{1.5cm} R{1.2cm} R{1.2cm} R{1.2cm} R{1.2cm} R{1.2cm} R{1.2cm}}
\\ \hline
 & coeff & ce & Z & p & LLCI & UCLI  \\ \hline \hline
const   & -17.16   & 592.16  & -.03  & .98 & -1177.77 & 1143.45     \\\hline
corr  & 13.25   & 592.16  & .02  & .98 & -1147.37 & 1173.86   \\\hline
exp  & 15.89  & 592.16  & .03  & .98 & -1144.72 & 1176.50  \\\hline
Z1 & .14 & .52  & -.26  & .79 & -.89 & 1.16  \\\hline   
Z2 & -.31 & .56 & -.56  & .58 & -1.41 & .79  \\\hline 
exp x corr & -13.33  & 592.16  & -.02  & .98 & -1173.94 & 1147.29  \\\hline 
Z1 x corr & -.28  & .75  & -.37  & .71 & -1.75 & 1.19  \\\hline 
Z2 x corr & .90  & .74  & 1.20  & .23 & -.56 & 2.36 \\\hline 
\end{tabular}

    \begin{tablenotes}
        \item[1] \textit{corr} --- \textit{correctness}; \textit{exp} --- \textit{level of expertise}
    \end{tablenotes}
    \end{threeparttable}

\label{mod_anal_nle_rsr}

\end{table}

\begin{table}[htbp!]

\caption{Moderation analysis of correctness of example-based explanations on RSR with the level of expertise and assertiveness as moderators (Since level of expertise is a three-level categorical moderator, it is split up into Z1 --- non-assertive explanations in relation to all other values --- and Z2 --- assertive explanations in relation to all other values.}

    \begin{threeparttable}
\begin{tabular}{m{1.5cm} R{1.2cm} R{1.2cm} R{1.2cm} R{1.2cm} R{1.2cm} R{1.2cm}}
\\ \hline
 & coeff & ce & Z & p & LLCI & UCLI  \\ \hline \hline
const   & -3.60 & .76 & -4.74   & .00  & -5.09  & -2.11     \\\hline
corr  & -.44   & 1.29  & -.34  & .74 & -2.97 & 2.10   \\\hline
exp  & 3.25 & .74  & 4.39  & .00 & 1.80 & 4.70  \\\hline
Z1 & -.09 & .43  & -.22  & .83 & -.94 & .75  \\\hline   
Z2 & .09 & .43 & .21  & .83 & -.75 & .93  \\\hline 
exp x corr & -.73  & 1.28  & -.57  & .57 & -3.23 & 1.77  \\\hline 
Z1 x corr & -.45  & .75  & -.60  & .55 & -1.92 & 1.02  \\\hline 
Z2 x corr & -.43  & .72  & -.59  & .55 & -1.85 & .99 \\\hline 
\end{tabular}
    \begin{tablenotes}
        \item[1] \textit{corr} --- \textit{correctness}; \textit{exp} --- \textit{level of expertise}
    \end{tablenotes}
    \end{threeparttable}
\label{mod_anal_ex_rsr}

\end{table}


\newpage

\subsection{Human-AI Team Performance}
\label{hai-appendix}

% Figure environment removed

% Figure environment removed