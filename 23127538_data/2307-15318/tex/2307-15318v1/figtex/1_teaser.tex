%   % Figure environment removed

\begin{teaserfigure}
    \begin{minipage}[b]{1.0\linewidth}
        \begin{minipage}[b]{0.16\linewidth}
            \centering
            \centerline{% Figure removed}
            \centerline{(a) Input}\medskip
        \end{minipage}
        \hfill
        \begin{minipage}[b]{0.16\linewidth}
            \centering
            \centerline{% Figure removed}
            \centerline{(b) Kligler \etal}\medskip
        \end{minipage}
        \begin{minipage}[b]{0.16\linewidth}
            \centering
            \centerline{% Figure removed}
            \centerline{(c) DHAN }\medskip
        \end{minipage}
        \begin{minipage}[b]{0.16\linewidth}
            \centering
            \centerline{% Figure removed}
            \centerline{(d) SG-ShadowNet }\medskip
        \end{minipage}
        \hfill
        \begin{minipage}[b]{0.16\linewidth}
            \centering
            \centerline{% Figure removed}
            \centerline{(e) Ours}\medskip
        \end{minipage}
        \begin{minipage}[b]{0.16\linewidth}
            \centering
            \centerline{% Figure removed}
            \centerline{(f) Target}\medskip
        \end{minipage}
    \end{minipage}
    \caption{
    The figure illustrates the comparison between our method and several state-of-the-art methods in document shadows removal. It contains (a) input image, (b) traditional method, (c) and (d) learning-based methods, (e) ours and (f) target. The results presented in the figure demonstrate that our proposed method significantly outperforms other existing techniques for removing shadows from documents. 
    }
    \label{fig:teaser}
\end{teaserfigure}

