\section{Introduction}

The shape of human red blood cells is known to change depending on their volume, with normally discoid concave shapes becoming star-shaped or spherical in different physiological conditions. 
An improperly formed cell membrane in hereditary hemolytic anemias results in anomalous morphologies and reduced deformability. 
These factors contribute to conditions such as hereditary spherocytosis, sickle cell disease, and thalassemia, where irregularly-shaped cells appear. 
For example, in sickle cell anemia, a genetic mutation causes red blood cells to take on a sickle shape, which is crescent-shaped rather than discoid. 
 This shape can make it difficult for the cells to pass through small blood vessels, leading to reduced oxygen flow and a range of health complications~\cite{booker2020sickle}.
Close monitoring of blood samples is essential for the diagnosis, progression tracking, and severity estimation of hereditary hemolytic anemias~\cite{kazeminia2022anomaly}. 
However, identifying hallmark cells in a patient's sample is challenging, and the presence of a small number of anomalous cells does not necessarily indicate an underlying condition, making diagnosis even more complicated.
%
The \emph{manual} annotation of blood samples for supervised model training is challenging and expensive, manifesting in a large degree of intra-expert variability. This necessitates the use of~(weakly) supervised methods for supporting experts.
Multiple Instance Learning~(MIL) is one such supervised learning method that can automate the analysis of blood samples. 
MIL is a form of weakly-supervised learning in which each training example is a bag containing several instances. 
The goal is to learn a classifier that can accurately label new bags based on their contained instances~\cite{lu2020clinical}.

However, due to the difficulty of obtaining a large amount of training data for rare diseases, MIL models suffer from a high risk of overfitting~\cite{zhu2022murcl}. 
In this context, our experiments show that standard regularization techniques in deep learning, e.g., early stopping or $L_1$/$L_2$ regularization, turn out to be insufficient. 
Therefore, more domain-specific, expressive regularization constraints are necessary to address this challenge (see Fig. \ref{fig:overfit}).

% Figure environment removed

One such approach is to add a constraint that encourages the model to focus on the most informative instances in each bag. 
This can be achieved by incorporating attention mechanisms into the model, which allows the model to selectively focus on the most relevant instances \cite{Goodfellow-et-al-2016}.
Following the same intuition, Sadafi et al.~\cite{sadafi2020attention} could significantly improve MIL performance compared to basic MIL methods, i.e., using average or max-pooling techniques.
Another approach is to incorporate domain knowledge or prior information into the model, which can help to guide the learning process. 
For example, leveraging anomalous cells, Kazeminia et al.~\cite{kazeminia2022anomaly} proposed an anomaly detection-based pooling technique.
Attention mechanism as an influential part in both \cite{sadafi2020attention} and \cite{kazeminia2022anomaly} is a learning-based technique and is thus dependent on the amount of training data.
This fact restricts the robustness of both solutions as regularizing the learning scheme and leads to a limited attainable level of performance.

We here suggest imbuing the model with \emph{multi-scale characteristic shape information}, measured using geometrical--topological descriptors to overcome this issue. Specifically, working in a MIL setting, we propose to preserve geometrical--topological information of bags in the latent space when compared to the image space during training. 
Our results demonstrate that this serves as a powerful way to \emph{regularize models}, improving more than 3\% over the state-of-the-art for classifying rare anemia disorders.

