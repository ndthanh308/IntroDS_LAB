\section{Background}

% TODO: Also feel free to add more information about multiple instance learning here.

Our work is based on recent advances in topological machine learning.
We employ \emph{persistent homology}, a technique for calculating multi-scale geometrical--topological information from data.
Persistent homology uses a metric such as the Euclidean distance to calculate multi-scale shape information of point clouds, including information about connected components, cycles, and higher-dimensional voids.
Such information is collected in a set of \emph{persistence diagrams}, i.e., multi-scale topological descriptors~(see Fig.~\ref{fig:Overview}).
Despite its origins in computational topology, persistence diagrams can be shown to carry a large amount of geometrical information~\cite{Bubenik20a,Turkes22a}, making them a useful shape descriptor.
Recent work in computational topology showed that persistent homology can be integrated with deep-learning models, leading to a new class of hybrid models that are capable of capturing geometrical and topological aspects of data.
Such models have shown exceptional performance as regularization terms~\cite{Chen19a,moor2020topological,Vandaele22a,Waibel22a} in different applications.
The reader is referred to a recent survey for more details on the integration into modern machine-learning models~\cite{Hensel21}.

%%%%%%%%%%%%%%%%%%%%%%%%%%%%%%%%%%%%%%%%%%%%%%%%%%%%%%%%%%%%%%%%%%%%%%%%
% Figure environment removed
%%%%%%%%%%%%%%%%%%%%%%%%%%%%%%%%%%%%%%%%%%%%%%%%%%%%%%%%%%%%%%%%%%%%%%%%