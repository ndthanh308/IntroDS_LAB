%%%%%%%% ICML 2024 EXAMPLE LATEX SUBMISSION FILE %%%%%%%%%%%%%%%%%

\documentclass{article}

\usepackage[english]{babel}

% Recommended, but optional, packages for figures and better typesetting:
\usepackage{microtype}
\usepackage{graphicx}
% \usepackage{subfigure}
\usepackage{booktabs} % for professional tables

% \usepackage[table,xcdraw]{xcolor}
\usepackage{colortbl}

\usepackage[symbol]{footmisc}

% hyperref makes hyperlinks in the resulting PDF.
% If your build breaks (sometimes temporarily if a hyperlink spans a page)
% please comment out the following usepackage line and replace
% \usepackage{icml2024} with \usepackage[nohyperref]{icml2024} above.
\usepackage{hyperref}

\PassOptionsToPackage{sort}{natbib}

% Attempt to make hyperref and algorithmic work together better:
% \newcommand{\theHalgorithm}{\arabic{algorithm}}

% Use the following line for the initial blind version submitted for review:
\usepackage[accepted]{icml2024}

% If accepted, instead use the following line for the camera-ready submission:
% \usepackage[accepted]{icml2024}

% For theorems and such
\usepackage{amsmath}
\usepackage{amssymb}
\usepackage{mathtools}
\usepackage{amsthm}

\usepackage{babel}
\usepackage{microtype}

% if you use cleveref..
\usepackage[capitalize,noabbrev]{cleveref}

%%%%%%%%%%%%%%%%%%%%%%%%%%%%%%%%
% THEOREMS
%%%%%%%%%%%%%%%%%%%%%%%%%%%%%%%%
\theoremstyle{plain}
\newtheorem{theorem}{Theorem}[section]
\newtheorem{proposition}[theorem]{Proposition}
\newtheorem{lemma}[theorem]{Lemma}
\newtheorem{corollary}[theorem]{Corollary}
\theoremstyle{definition}
\newtheorem{definition}[theorem]{Definition}
\newtheorem{assumption}[theorem]{Assumption}
\theoremstyle{remark}
\newtheorem{remark}[theorem]{Remark}

% Todonotes is useful during development; simply uncomment the next line
%    and comment out the line below the next line to turn off comments
%\usepackage[disable,textsize=tiny]{todonotes}
\usepackage[textsize=tiny]{todonotes}


% The \icmltitle you define below is probably too long as a header.
% Therefore, a short form for the running title is supplied here:
\icmltitlerunning{Topologically Regularized Multiple Instance Learning to Harness Data Scarcity}


%%%%%%%%%%%%%%%%%%%%%%%% My packages %%%%%%%%%%%%%%%%%%%%%%%%%%%%%%%%%%%%%%

\usepackage[T1]{fontenc}
\usepackage{listings}
\lstset{language=Pascal}

\usepackage[normalem]{ulem}
\useunder{\uline}{\ul}{}
\usepackage{xcolor}
\usepackage[export]{adjustbox}
\usepackage{tabu}
\usepackage{subcaption}
% \usepackage[hidelinks]{hyperref}
\usepackage{stackengine}
\usepackage{pifont}
\newcommand{\cmark}{\ding{51}}%
\newcommand{\xmark}{\ding{55}}%

\usepackage{paralist}
\usepackage{tabularx}



\begin{document}

\twocolumn[
\icmltitle{Topologically Regularized Multiple Instance Learning to Harness Data Scarcity}

% It is OKAY to include author information, even for blind
% submissions: the style file will automatically remove it for you
% unless you've provided the [accepted] option to the icml2024
% package.

% List of affiliations: The first argument should be a (short)
% identifier you will use later to specify author affiliations
% Academic affiliations should list Department, University, City, Region, Country
% Industry affiliations should list Company, City, Region, Country

% You can specify symbols, otherwise they are numbered in order.
% Ideally, you should not use this facility. Affiliations will be numbered
% in order of appearance and this is the preferred way.
\icmlsetsymbol{equal}{$\dagger$}

\begin{icmlauthorlist}
\icmlauthor{Salome Kazeminia}{AIH,TUM}
\icmlauthor{Carsten Marr}{AIH,equal}
\icmlauthor{Bastian Rieck}{AIH,TUM,equal}
\end{icmlauthorlist}

\icmlaffiliation{AIH}{Institute of AI for Health, Helmholtz Zentrum München, German Research Center for Environmental Health, Neuherberg, Germany}
\icmlaffiliation{TUM}{TUM School of Computation, Information and Technology, Technical University of Munich, Munich, German}

\icmlcorrespondingauthor{Bastian Rieck}{bastian.rieck@helmholtz-munich.de}
\icmlcorrespondingauthor{Carsten Marr}{carsten.marr@helmholtz-munich.de}

% You may provide any keywords that you
% find helpful for describing your paper; these are used to populate
% the "keywords" metadata in the PDF but will not be shown in the document
\icmlkeywords{Machine Learning, ICML}

\vskip 0.3in
]

% this must go after the closing bracket ] following \twocolumn[ ...

% This command actually creates the footnote in the first column
% listing the affiliations and the copyright notice.
% The command takes one argument, which is text to display at the start of the footnote.
% The \icmlEqualContribution command is standard text for equal contribution.
% Remove it (just {}) if you do not need this facility.

%\printAffiliationsAndNotice{}  % leave blank if no need to mention equal contribution
\printAffiliationsAndNotice{\icmlEqualContribution} % otherwise use the standard text.

\begin{abstract}
In biomedical data analysis, Multiple Instance Learning (MIL) models have emerged as a powerful tool to classify patients' microscopy samples. 
%
However, the data-intensive requirement of these models poses a significant challenge in scenarios with scarce data availability, e.g., in rare diseases. 
%
We introduce a topological regularization term to MIL to mitigate this challenge.
%
It provides a shape-preserving inductive bias that compels the encoder to maintain the essential geometrical-topological structure of input bags during projection into latent space. 
%
This enhances the performance and generalization of the MIL classifier regardless of the aggregation function, particularly for scarce training data. 
%
The effectiveness of our method is confirmed through experiments across a range of datasets, showing an average enhancement of $2.8\%$ for MIL benchmarks, $15.3\%$ for synthetic MIL datasets, and $5.5\%$ for real-world biomedical datasets over the current state-of-the-art.
%
% We make our code publicly available at https://anonymous.4open.science/r/TR-MIL-4BB4/.
\end{abstract}


% !TEX root = ../AttackGraphBasedRiskAnalysis.tex
% !TEX spellcheck = en_US
% !TEX encoding = UTF-8 Unicode

\section{Introduction}\label{sec: intro}

Traditional cities are becoming smarter. 
One of the core smart city concepts is smart mobility, which has attracted considerable attention from security researchers due to the emergence of smart vehicles and V2X communication that have given rise to novel cybersecurity threats.

Over the last decade, several trends have contributed to the automotive and railway threat landscape. 
First, sophisticated features in smart vehicles come with a higher volume of lines of code, aggravating testability and auditing and increasing the likelihood and severity of vulnerabilities. 
Second, (wireless) communication interfaces in smart vehicles come with a higher volume of external peripheral devices that can connect to smart vehicles, hence increasing the attackers' access point options, and also with a higher volume of connections, hence increasing the risk of malicious interactions. 
Finally, a higher volume of connections between smart vehicles comes with a higher volume of exchanged data, which in most cases is personal and, therefore, immensely valuable. In other words, more data is generated and needs to be considered and protected.

Graphical security modeling is a widely-used and well-established approach for representing and analyzing threat landscapes that examine vulnerabilities of systems and organizations. 
One of the primary strengths of graphical security models is that they allow for the inclusion of user-friendly visual elements with formal semantics and algorithms, enabling both qualitative and quantitative analyses. 
Over the last couple of decades, security researchers have been progressively focusing on graphical security modeling, which has gradually evolved into a valuable tool for the assessment of risks in real-life systems, such as automotive and railway environments.

Threat landscapes include (1) malicious actions of an attacker, whose goal is to harm or damage one or more assets of a system or organization, and (2) countermeasures for either preventing or mitigating such malicious actions. 
The first \emph{tree-based approach} for graphical security modeling was the \emph{threat logic trees}, which was introduced by Weiss in 1991~\cite{weiss1991}, thereby motivating the development of several subsequent frameworks, such as attack trees, which are still considered one of the most important and favored tools for the assessment of risks to date.

In all tree-based approaches, the modeling process begins with identifying a feared event, which is shown as a root node, and continues with the refinement of the attack steps, resulting in a tree model.
However, tree structures are limited to only one path between a pair of nodes. 
In other words, with tree structures, each refined node can only have one parent node. 
This limitation is addressed by the \emph{directed acyclic graph (DAG) structure}, which enables refined nodes to have multiple parent nodes. 
As a result, DAG structures can provide a higher level of detail, but they can also come with a higher level of complexity, which can nevertheless be dealt with modularization, thereby allowing the model to be subdivided into loosely-coupled, independent, and interchangeable parts that can be studied individually and in parallel. 
Finally, while the one-to-many relationship between nodes in tree structures results in a linear analysis of the threat landscape, the many-to-many relationship between nodes in DAG structures can theoretically result in an exponential analysis.
However, the complexity is kept small in practice due to the acyclic structure, and the threat landscape analysis is eventually possible.

Ensuring the security of systems is not a static process that is over after going through once.
The conditions are constantly changing, on the one hand attackers and their capabilities are evolving, and on the other hand, systems themselves are being extended and evolving.
To effectively perform the necessary continuous security management, it is necessary to know not just the threat landscape but to be able to understand the consequences and impacts if attacks are performed successfully.
Hence, it is necessary to continuously perform a risk analysis to identify the potential exposure.
Nowadays, risk management is primarily done using large tables filled with a lot of information and use cases.
Large tables only offer limited visibility, as it is challenging to maintain a comprehensive overview of risks.
With numerous rows and columns, it becomes difficult to identify trends and patterns or prioritize risks effectively.
Furthermore, managing risk can be a tedious and time-consuming process.
Updating and maintaining tables with evolving risks and mitigation measures can require significant effort, especially when dealing with a complex system or multiple risk factors.
This gets even harder when dealing with large tables that often fail to provide the necessary context and connections between different risks.
Additionally, analyzing and interpreting data from large tables can be daunting. 
It may require specialized tools or skills to extract meaningful insights from the extensive amount of information presented in the table format.
Large tables may further lack the flexibility to accommodate changing risk scenarios or evolving requirements. 
Modifying or updating the table structure to incorporate new risks or factors can be cumbersome and may hinder agility in risk management.
With numerous cells and data entries, there is also an increased risk of errors, inaccuracies, or inconsistencies in the large table. 
These issues can undermine the reliability and integrity of the risk management process.

We propose a graphical solution for the risk management process to mitigate these disadvantages of tables.
A visual representation can enhance the understanding and communication of complex risk information and make it easier to identify patterns, trends, and relationships among risks, facilitating effective decision-making.
Complex risk data is further simplified by presenting it in a clear and concise manner.
Understanding  the relationships, dependencies, and interactions between various risk elements is necessary to understand the overall risk landscape.
Visual representations of the entire risk landscape provide this overview, allowing for the identification of interdependencies, hotspots, or areas of high vulnerability.
Graphical solutions can also aid in developing and evaluating risk mitigation strategies. 
By visually representing the potential consequences and effectiveness of different mitigation measures, decision-makers can make more informed choices and allocate resources more efficiently.
Furthermore, it allows for the exploration of different risk scenarios. 
By manipulating variables or parameters within the visual representation, it becomes possible to assess the potential impact of various risk factors and evaluate the effectiveness of different response strategies.
Additionally, as graphical solutions can be more adaptable to changing requirements and evolving risks, they allow for easier updates and modifications, enabling risk management processes to be more responsive and agile.

Consequently, we believe a graphical solution for the risk assessment process improves the maintenance of risk scenarios and facilitates accessibility to different stakeholders, including non-technical audiences.
However, the existing graphical solutions are momentarily used to describe the threat landscape.
Which, of course, is helpful for the risk management process but not sufficient to represent the entire risk management process.
Therefore, motivating us to define a new graphical method for risk assessment by extending existing graphical methods for depicting the threat landscape.
Besides ways to depict attack vectors, their probability, and countermeasures, our method includes a way to depict the consequences of attack vectors and the impact level, enabling us to calculate a risk value.

The remainder of the paper is structured as follows:
After the introduction,~\cref{sec: related work} discusses the related work.
Our definition of attack graphs is given in~\cref{sec: attack graphs}.
The necessary adjustments to use these attack graphs are presented in~\cref{sec: attack graph risk assessment}, including an example of how the risk assessment is performed in our project.
\cref{sec: applicability of attack graphs to risk management standards} validates our defined method by combing it with the risk assessment processes of ISO/SAE 21434~\cite{21434} and CLC/TS 50701~\cite{50701} respectively.
The scalability and practicality are evaluated in~\cref{sec: evaluation}.
Finally,~\cref{sec: conclusion} concludes this paper.	
\section{Background}

\paragraph{MIL Architectures.}
MIL architectures typically comprise three key components: an instance encoder, an aggregation function, and a classifier head (Figure \ref{fig:diagram}).
%
Given a collection of bags ${b_{1}, \dots, b_{M}}$, each bag contains a set of instances, represented as $X_{b_m} := \{x_{1}, \dots x_{n}\}$ with $n$ denoting the number of instances in the bag.
%
An instance encoder $f_{\theta}$ with parameters $\theta$ transfers instance data into a latent space, yielding feature vectors $z_i:=f_{\theta}(x_i)$. 
%
The aggregation function then creates a global representation of a bag $\zeta_{b_{m}}$ from these embedded instances. 
%
Finally, this bag representation is passed through a classifier head, which predicts the overall label of the bag.

\paragraph{Geometry \& Topology.}
Our work is based on recent advances in topological machine learning~\citep{Hensel21}, a nascent field that aims to leverage geometry and topology from data to elicit improved representations.
%
We employ \emph{persistent homology}, a technique for calculating multi-scale geometrical-topological information from data~\cite{edelsbrunner2009computational}.
%
Persistent homology considers data to be a point cloud (Figure \ref{fig:diagram}, using a metric~(e.g., Euclidean distance) to assess its multi-scale shape information. 
%
This includes topological information like connected components, cycles, and higher-dimensional voids in addition to geometrical information like curvature or convexity~\citep{Bubenik20a, Turkes22a}.
%
Such information is collected in a set of \emph{persistence diagrams}, i.e., multi-scale topological descriptors.
%
These descriptors are calculated by approximating the data in terms of a simplicial complex, i.e., a generalized graph, typically based on distance functions like the Euclidean distance.
%
Recent work proved that persistent homology can be integrated with deep learning models, leading to a new class of hybrid models that are capable of capturing topological aspects of data.
%
Such models have shown exceptional performance as regularization terms in different applications~\citep{Chen19a,Vandaele22a,Waibel22a}.
\section{Related Work}

MIL aggregation functions can take simple forms, such as max or average pooling, or more complex forms, like attention-based pooling, as commonly used in various MIL models in the literature~\citep{ilse2018attention, li2021dual, shao2021transmil, kazeminia2022anomaly, zhang2022dtfd, zhao2023generalized}.
%
Simple aggregation functions are computationally efficient and easy to implement but may overlook instance-level information, whereas complex forms offer a more detailed understanding of relevant instances at the cost of increased computational complexity.
%
Attention-based pooling amplifies the learning signal for important instances, ensuring improved classification performance, even when only a subset of positive instances is distinguished. 
%
However, in scenarios where instance-level performance is crucial, the attention-based pooling approach may lack the necessary reliability, as it does not uniformly enhance expression across all instances.
%
To overcome this limitation, \citet{du2023rgmil} recently introduced a regressor-guided aggregator that removes learnable attention parameters and rectifies the inference process, thereby enhancing the direct passage of the learning signal through the encoder. 
%
Although this state-of-the-art approach significantly refines instance-level representation and enhances overall MIL performance, it still faces challenges when dealing with scarcity of training data. 
%
In such scenarios, the regressor-guided aggregator may struggle to accurately capture and represent the nuanced variations within the data, leading to potential limitations in model generalization and reliability. 
%
This is particularly evident in settings where the data lacks diversity or sufficient examples of certain classes, making it difficult for the model to learn and generalize effectively.
%
To address these limitations, we introduce topological regularization, establishing a more robust and rational inductive bias that enhances the model's overall performance and generalization.

% \paragraph{Anemia classification.}
% %
% The diagnosis of anemia relies on the minority of red blood cells in a patient's blood sample that shows morphological features associated with the disease. 
% %
% Anemia disorders lead to various aberrant shapes such as sickle-shaped (SCD), crumpled or perforated (thalassemia), star-shaped (Xero), or even spherical (HS) cells. 
% %
% These deformations can manifest with varying degrees of severity and in different proportions, while it's also possible for other cell types unrelated to Anemia conditions to coexist.
% %
% Detecting the hallmark cells indicative of anemia poses a significant challenge due to substantial variability in expert opinions. 
% %
% Furthermore, a few atypical cells do not indicate an anemia condition, complicating the diagnostic process.
% %
% This makes the \emph{manual} annotation of blood samples for supervised model training a laborious and costly endeavor~\citep{kazeminia2022anomaly}. 
% %
% In the absence of cell-level annotations, MIL was employed in this context.
% %
% In the context of MIL, individual instances (cells) are organized within bags (blood samples), and a singular label is assigned to represent the anemia type presented in the entire bag~\citep{lu2020clinical}.
% %
% \citet{kazeminia2022anomaly} as the state-of-the-art in this MIL application proposes the concept of anomaly score added to attention aggregation to facilitate detection of negative instances encountering their Mahalanobis distance to the fitted Gaussian mixture model on negative instances distribution.
% %
% Although this method demonstrated state-of-the-art performance in anemia classification, its effectiveness in estimating anomaly scores is constrained by the encoder's ability to map negative instances in close proximity, with no direct learning signal guiding this process.
% %
% Even with a stronger learning signal for a negative instance, having only a limited training data, the encoder is in danger of overfitting and can not generalize well on unseen instances of diversities.
% % 

\section{Method}

\looseness -1 As warm-up, consider an example with deterministic agents, i.e., when $p_{t,j}$ is a Dirac measure on a specific action $\vx_{t,j}$. 
Suppose it was practically feasible to draw the action suggested by every agent and observe the corresponding reward vector \smash{$\vr_t = (r_{t,j})_{j=1}^M$}.
In this case, model selection becomes a full-information online optimization problem, and we can design a minimax optimal algorithm as follows. We assign a probability distribution $\vq_{t} = (q_{t,j})_{j=1}^M$ to the models, and update it such that the overall average return \smash{$\sum_{t=1}^n \vq^\top_{t} \vr_t$} is competitive to the best agent's average return \smash{$\sum_{t=1}^n r_{t,j^\star}$}. 
At every step, we update \smash{$q_{t+1,j} \propto \exp(\sum_{s=1}^{t} r_{s,j})$}, since such exponential weighting is known to lead to an optimal solution for this classic online learning problem  \citep{cesa2006prediction}.
In our setting however, the agents are stochastic, and we do not have access to the full $\vr_t$ vector.


We propose the \textsc{\textbf{A}}nytime \textsc{\textbf{Exp}}onential weighting algorithm based on \textsc{\textbf{L}}asso reward estimates (\alexp),  summarized in \cref{alg:lassoexp}.
  At step $t$ we first sample an agent $j_t$, and then sample an action $\vx_t$ according to the agent's policy $p_{t, j_t}$. 
Let $\Delta_M$ denote the $M$-dimensional probability simplex.
We maintain a probability distribution $\vq_{t} \in \Delta_M$ over the agents, and update it sequentially as we accumulate evidence on the performance of each agent. 
Ideally, we would have adjusted $q_{t,j}$ according to the average return of model $j$, that is,  $\E_{\vx \sim p_{t,j}} r(\vx)$. However, since $r$ is unknown, we estimate the average reward with some $\hat r_{t,j}$. We then update $\vq_{t}$ for the next step via,
\[
q_{t+1, j} = \frac{\exp(\eta_t \sum_{s=1}^{t}\hat r_{s,j})}{\sum_{i=1}^M \exp(\eta_t \sum_{s=1}^t\hat r_{s,i})}
\]
for all $j = 1, \dots, M$, where $\eta_t$ is the learning rate, and controls the sensitivity of the updates. 
This rule allows us to imitate the full-information example that we mentioned above. By utilizing $\hat r_{t,j}$ and hallucinating feedback from all agents, we can reduce the probability of selecting a badly performing agent, without ever having sampled them (c.f. \cref{fig:prob_curves}). 
 It remains to design the estimator $\hat r_{t,j}$.
 We concatenate all feature maps, and, knowing that many features are redundant, use a sparsity inducing estimator over the resulting coefficients vector. Mainly, let $\vphi(\vx) \coloneqq (\vphi_1(\vx), \dots, \vphi_M(\vx))$, and $\vtheta = (\vtheta_1, \dots, \vtheta_M) \in \sR^{Md}$ be the concatenated coefficients vector. We then solve
\begin{equation}\label{eq:glasso} 
    \hat \vtheta_t = \argmin_{\vtheta \in \sR^{Md}} \calL(\vtheta; H_t, \lambda_t) = \argmin_{\vtheta \in \sR^{Md}} \frac{1}{t} \norm{ \vy_t- \Phi_t\vtheta}^2_2 + 2\lambda_t\sum_{j=1}^M  \norm{\vtheta_j}_2
\end{equation}
where $\Phi_t = [\vphi^\top(\vx_s)]_{s\leq t} \in \sR^{t \times Md}$ is the feature matrix, $\vy_t \in \sR^t$ is the concatenated reward vector, and $\lambda_t$ is an adaptive regularization parameter. 
Problem \eqref{eq:glasso} is the online variant of the group Lasso \citep{lounici2011oracle}, and enforced sparsity at group level. Therefore, the sub-vectors \smash{$\hat\vtheta_{t,j} \in \sR^{d}$} that correspond to redundant feature maps are expected to be $\boldsymbol{0}$, i.e. the null vector. 
We then estimate the average return of each model by simply taking an expectation \smash{$\hat r_{t,j} = \E_{\vx \sim p_{t+1,j} } [\hat\vtheta_t^\top\vphi(\vx)]$}. This quantity is the average return of the agent's policy $p_{t+1,j}$, according to our Lasso estimator.
In \cref{sec:properties} we explain why the particular choice of Lasso is crucial for obtaining a $\log M$ rate for the regret. \looseness -1

For action selection, with probability $\gamma_t$, we sample agent $j$ with probability $q_{t,j}$ and draw $\vx_t \sim p_{t,j}$ as per suggestion of the agent. With probability $1-\gamma_t$, we choose the action according to some exploratory distribution $\pi \in \calM(\calX)$ which aims to sample informative actions.
This can be any design where $\mathrm{supp}(\pi) = \calX$. 
We mix $p_{t,j}$ with $\pi$, to collect sufficiently diverse data for model selection. We are not restricting the agents' policy, and therefore can not rely on them to explore adequately. 
In \cref{thm:regret_main}, we choose a decreasing sequence of $(\gamma_t)_{t\geq 1}$ the probabilities of exploration at step $t\geq 1$, since less exploration will be required as data accumulates.
To conclude, the action selection policy of \alexp is formally described as the mixture
\vspaceequation
\[
p_t(\vx) = \gamma_t \pi(\vx) + (1-\gamma_t) \sum_{j=1}^M q_{t,j}p_{t,j}(\vx).
\]

\begin{algorithm}[t]
\caption{\alexp \label{alg:lassoexp}}
\begin{algorithmic}
\State Inputs: $\pi, (\gamma_t,\eta_t,\lambda_t)$ for $t \geq 1$
\State Let $\vq_1 \leftarrow \text{Unif}(M)$ and initialize base agents $(p_{1,1}, \dots, p_{1,M})$.
\For{$t \geq 1$}
\State Draw $\alpha_t\sim \mathrm{Bernoulli}(\gamma_t)$. \Comment{Decide to explore or exploit}
\If{$\alpha_t = 1$}
\State Choose action $\vx_t$ randomly according to $\pi$. \Comment{Explore}
\Else 
\State Draw $j_t\sim \vq_{t}$.  \Comment{Select an agent}
\State Draw $\vx_t \sim p_{t,j_t}$. \Comment{Select the action suggested by the agent}
\EndIf
\State Observe $y_t = r(\vx_t) + \varepsilon_t$. \Comment{Receive reward}
\State $H_t = H_{t-1} \cup \{(\vx_t, y_t)\}$. \Comment{Append history}
\State $\hat \vtheta_{t} \leftarrow \argmin \calL(\vtheta; H_t, \lambda_t)$. \Comment{Update the parameter estimate}
\State Report $H_t$ to all agents, and get updated policies $p_{t+1,1}, \dots, p_{t+1,M}$. \Comment{Update agents}
\State Update estimated average return of every agent
\[
\hat r_{t,j} \leftarrow \E_{\vx \sim p_{t+1,j}} [\hat \vtheta_{t}^\top\vphi(\vx)], \quad  j=1, \dots, M
\]
\State Update agent probabilities
\[
\vspaceequation
q_{t+1,j} \leftarrow \frac{\exp(\eta_t \sum_{s=1}^t\hat r_{s,j})}{\sum_{i=1}^M \exp(\eta_t \sum_{s=1}^t\hat r_{s,i})}
\vspaceequation
\]
\EndFor
\end{algorithmic}
\end{algorithm}
\vspacefigure

\section{Experiments}
%
We evaluate TR-MIL on different datasets: MIL benchmarks, synthetic MIL datasets, and a real-world biomedical dataset for anemia classification.
%
\subsection{MIL Benchmarks}
%
\begin{table*}[]
    \caption{Topological regularization improves the classification performance of RGMIL (SOTA) on MIL benchmarks. 
    Among previous methods, we specifically reimplemented RGMIL for our analysis. Other results (gray) are collected from papers proposed by \citet{ilse2018attention} (APMIL and GAPMIL), \citet{yan2018deep} (DPMIL), \citet{li2021dual} (DSMIL), \citet{huang2022bag} (BDRMIL), and \citet{du2023rgmil} (RGMIL).}
    \label{tab:benchmark}
    \renewcommand{\arraystretch}{1.2}
    \begin{tabularx}{\textwidth}{lXXXXX}
    \toprule
        Method               
        & MUSK1 
        & MUSK2
        & FOX 
        & TIGER
        & ELEPHANT
        \\
    \midrule
        \textcolor{gray}{APMIL (2018)}            
        & \textcolor{gray}{0.892$\pm$0.040}
        & \textcolor{gray}{0.858$\pm$0.048}
        & \textcolor{gray}{0.615$\pm$0.043}
        & \textcolor{gray}{0.839$\pm$0.022}
        & \textcolor{gray}{0.868$\pm$0.022}
        \\
        \textcolor{gray}{GAPMIL (2018)}        
        & \textcolor{gray}{0.900$\pm$0.050}
        & \textcolor{gray}{0.863$\pm$0.042}
        & \textcolor{gray}{0.603$\pm$0.029}
        & \textcolor{gray}{0.845$\pm$0.018}
        & \textcolor{gray}{0.857$\pm$0.027}
        \\
        \textcolor{gray}{DPMIL (2018)}              
        & \textcolor{gray}{0.907$\pm$0.036}
        & \textcolor{gray}{0.926$\pm$0.043}
        & \textcolor{gray}{0.655$\pm$0.052}
        & \textcolor{gray}{0.897$\pm$0.028}
        & \textcolor{gray}{0.894$\pm$0.030}
        \\
        \textcolor{gray}{DSMIL (2021)}     
        & \textcolor{gray}{0.932$\pm$0.023}
        & \textcolor{gray}{0.930$\pm$0.020}
        & \textcolor{gray}{0.729$\pm$0.018}
        & \textcolor{gray}{0.869$\pm$0.008}
        & \textcolor{gray}{0.925$\pm$0.007}
        \\
        \textcolor{gray}{BDRMIL (2022)}                 
        & \textcolor{gray}{0.926$\pm$0.079}
        & \textcolor{gray}{0.905$\pm$0.092}
        & \textcolor{gray}{0.629$\pm$0.110}
        & \textcolor{gray}{0.869$\pm$0.066}
        & \textcolor{gray}{0.908$\pm$0.054}
        \\
        \midrule
        RGMIL (2023)           
        & 0.940$\pm$0.070
        & 0.920$\pm$0.106
        & 0.714$\pm$0.107
        & 0.842$\pm$0.088
        & 0.915$\pm$0.042
        \\
        TR-RGMIL (ours)      
        & \textbf{0.946$\pm$0.078}
        & \textbf{0.970$\pm$0.042}
        & \textbf{0.747$\pm$0.054}
        & \textbf{0.961$\pm$0.040}
        & \textbf{0.941$\pm$0.054}   
        \\
        \hline
    \end{tabularx}
\end{table*}


We evaluate TR-MIL on five classic MIL benchmark datasets. 
%
These include three image-based datasets (FOX, TIGER, and ELEPHANT), each comprising $200$ bags, introduced by \citet{dietterich1997solving}. 
%
For these datasets, instead of actual images, we only have extracted features from tiled image patches (instances) representing parts of an image. 
%
Additionally, we employ MUSK1 and MUSK2 datasets, introduced by \citet{andrews2002support}, which contain data on $92$ and $102$ molecules, respectively. 
%
In these datasets, each molecule is represented by a bag of instances, with each instance corresponding to a different molecular conformation.
%
The number of instances per bag ranges from as few as $1$ to as many as $1044$, providing a comprehensive assessment of our model's adaptability and robustness across different scales of data representation.
%

We utilized an identical encoder architecture to clarify and ensure an equitable comparison with the existing state-of-the-art MIL method (RGMIL). 
%
This architecture includes $2$ linear layers with a ReLU activation, projecting input features into a $512$-dimensional space for both layers. 
%
The primary modification in our setup is integrating a topological signature calculator into the input and instance encoder.
%

The original RGMIL model used $231$ features for FOX, TIGER, and ELEPHANT datasets and $167$ features for MUSK1 and MUSK2 datasets, including a last feature representing the repeated label of the bag for each instance. 
%
However, in our re-implementation, we followed the standard benchmark settings of $230$ and $166$ features for the respective datasets to align with previous works and provide a comprehensive comparison (see Table \ref{table:bm_mil_model}).
%
We run both the RGMIL and TR-RGMIL models $5$ times, applying $10$-fold cross-validation and reporting the average optimal performance of the model during the training.
%
When using this instance feature vector, we observed a decline in RGMIL's performance, with TR-RGMIL still outperforming all other methods in all five standard MIL benchmarks (Table \ref{tab:benchmark}).

Our experiments reveal that the RGMIL model is prone to overfitting if no topological regularization is being used.
%
This phenomenon is characterized by the MIL classifier exhibiting its best performance during the initial epochs of training.
%
Topological regularization effectively mitigates this overfitting issue (Figure \ref{fig:Benchmarks_LC}). 
%






\subsection{Synthetic Datasets}
%
% Figure environment removed

% Figure environment removed

For evaluating the robustness of TR-MIL framework across varied MIL problem definitions, including instance image complexity, number of training bags, and bag sizes, we draw on the methods outlined by \citet{ilse2018attention}. 
%
To consider the inherent complexity of instance images, we create two synthetic datasets: the first comprising MNIST images as instances (MIL-MNIST), and the second bags of Fashion-MNIST images~\citep{xiao2017fashion} as instances (MIL-FashionMNIST), providing a more challenging scenario with complex visual data.  
%
In MIL-MNIST, the digit ``9'' is considered a positive instance, while all other digits are considered negative instances. 
%
In MIL-FashionMNIST, the label ``Dress'' is considered a positive instance, while other labels showcase negative instances. 
%
We construct distinct training datasets containing a total number of $10$, $14$, $20$, $50$, $100$, and $200$ bags to evaluate the influence of the quantity of training data. 
%
Additionally, we explore different amounts of instances per bag, sampling them from Gaussian distributions with mean and standard deviations defined as $(10, 2)$, $(50, 10)$, and $(100, 20)$, respectively. 
%
Positive bags are defined as those containing at least one positive instance, accounting for up to $20\%$ of the instances within the bag. 

\paragraph{Models.}
%
We use a deep instance encoder architecture introduced by \citet{ilse2018attention} (see Table \ref{table:cv_mil_model} for details).
%
It consists of two convolutional layers with a kernel size of $5$, a stride of $1$, and ReLU activation functions. 
%
These layers generate $20$ and $500$ feature maps, respectively. 
This is followed by a fully-connected layer. 
%
The output from this encoder is a $500$-dimensional feature vector, which then undergoes further processing in the aggregation function. 
%
The attention network comprises two linear layers, resulting in a final output dimension of $128$ followed by $1$.
%
The topological signature of input instances is calculated on image space and latent space, applying pixel-vise Euclidean distance of instance images and latent feature vectors (Figure \ref{fig:diagram}.

\paragraph{Results.}
%
We evaluate the effectiveness of topological regularization on three aggregation functions in MIL, max pooling, average pooling, attention-based pooling, which serves as the baseline for numerous studies in the field~\citep{ilse2018attention}, in addition to the regressor-guided pooling technique, recognized as the state-of-the-art~\citep{du2023rgmil}.
%
% For the evaluation metric, we choose the F1-score over accuracy because it provides more informative insights by simultaneously accounting for both false positives and false negatives.
%
We analyze the average F1-score and its standard deviation for different numbers of training bags (Figure \ref{fig:Mnist_Fmnist}) and bag sizes (Figure \ref{fig:Mnist_Fmnist_detailed}) over five runs on MIL-MNIST and MIL-FashionMNIST datasets. 
%
Without topological regularization, models trained with few training bags perform poorly, akin to random guessing, due to overfitting (Learning curves are shown in Figure \ref{fig:Mnist_LC}). 
%
Adding topological regularization provides a reasonable complexity for the encoder to resolve overfitting and lets the encoder learn a more meaningful latent representation of data.
%
Consequently, it improves the MIL model performance across both datasets.
%
Notably, topological regularization narrows the performance gap between basic aggregations of max pooling and average pooling compared to advanced techniques of attention and regressor-guided pooling.
%
This demonstrates the crucial role of accounting for a bag's topological structure in enhancing MIL classification, surpassing the impact of the aggregation function, as evidenced in our toy experiment.
%




\subsection{Anemia Classification}

\begin{table*}[t]
    \caption{Topological regularization improves classification performance for all pooling strategies for Anemia classification. We apply it to different MIL methods with average/max pooling, attention-based pooling~\cite{sadafi2020attention}, and anomaly-aware pooling~\cite{kazeminia2022anomaly}. Numbers show the average classification performance along with the standard deviation from  3 cross-validation and 3 runs. Best performance is indicated by bold text. Additionally, for each pooling method, we compare the classification performance without (\xmark) and with (\cmark) topological regularization, and the winner is underlined for clarity.}
    \label{tab:performance}
    \resizebox{\textwidth}{!}{%
    \renewcommand{\arraystretch}{1.2}
    \begin{tabular}{lcccccccc}
        \toprule
        & \multicolumn{2}{c}{Average pooling}                                            
        & \multicolumn{2}{c}{Anomaly pooling}                        
        & \multicolumn{2}{c}{Attention pooling}
        & \multicolumn{2}{c}{Max pooling}                  
        \\ \midrule
        \multicolumn{1}{c}{\begin{tabular}[c]{@{}l@{}}Topological 
        \\regularization\end{tabular}} 
        & \xmark
        & \multicolumn{1}{c}{\cmark}                        
        & \xmark
        & \multicolumn{1}{c}{\cmark}               
        & \xmark
        & \multicolumn{1}{c}{\cmark}               
        & \xmark
        & \cmark               
        \\ \midrule
        \multicolumn{1}{l}{Accuracy}                                                              
        & \multicolumn{1}{c}{72.25$\pm$7.0} 
        & \multicolumn{1}{c}{{\ul \textbf{81.29$\pm$2.5}}} 
        & \multicolumn{1}{c}{77.85$\pm$3.7} 
        & \multicolumn{1}{c}{{\ul 79.50$\pm$1.2}} 
        & \multicolumn{1}{c}{73.72$\pm$3.8} 
        & \multicolumn{1}{c}{{\ul 77.76$\pm$1.6}} 
        & \multicolumn{1}{c}{64.33$\pm$5.8} 
        & {\ul 71.44$\pm$5.6} 
        \\
        \multicolumn{1}{l}{F1-Score}                                                              
        & \multicolumn{1}{c}{70.47$\pm$7.4} 
        & \multicolumn{1}{c}{{\ul \textbf{80.28$\pm$3.1}}} 
        & \multicolumn{1}{c}{76.69$\pm$4.0} 
        & \multicolumn{1}{c}{{\ul 77.01$\pm$1.8}} 
        & \multicolumn{1}{c}{72.38$\pm$3.8} 
        & \multicolumn{1}{c}{{\ul 74.69$\pm$1.6}} 
        & \multicolumn{1}{c}{62.96$\pm$5.0}   
        & {\ul 68.77$\pm$5.4} 
        \\
        \multicolumn{1}{l}{AUROC}                                                                 
        & \multicolumn{1}{c}{89.88$\pm$2.7} & \multicolumn{1}{c}{{\ul \textbf{93.72$\pm$4.4}}} 
        & \multicolumn{1}{c}{89.05$\pm$4.3} & \multicolumn{1}{c}{{\ul 90.89$\pm$2.5}} 
        & \multicolumn{1}{c}{91.58$\pm$3.0} & \multicolumn{1}{c}{{\ul 91.88$\pm$2.5}} 
        & \multicolumn{1}{c}{84.83$\pm$2.8} & {\ul 89.73$\pm$3.1} 
        \\
        \multicolumn{1}{l}{Recall}                                                                 
        & \multicolumn{1}{c}{59.68$\pm$7.8} & \multicolumn{1}{c}{{\ul \textbf{65.12$\pm$5.0}}} 
        & \multicolumn{1}{c}{63.24$\pm$3.2} & \multicolumn{1}{c}{{\ul 65.99$\pm$3.3}} 
        & \multicolumn{1}{c}{59.31$\pm$6.4} & \multicolumn{1}{c}{{\ul 60.42$\pm$2.3}} 
        & \multicolumn{1}{c}{52.77$\pm$8.6} & {\ul 53.75$\pm$4.7} 
        \\
        \multicolumn{1}{l}{Precision}                                                                 
        & \multicolumn{1}{c}{61.77$\pm$7.1} & \multicolumn{1}{c}{{\ul \textbf{79.06$\pm$12.0}}} 
        & \multicolumn{1}{c}{67.36$\pm$4.5} & \multicolumn{1}{c}{{\ul 69.67$\pm$4.6}} 
        & \multicolumn{1}{c}{64.89$\pm$4.9} & \multicolumn{1}{c}{{\ul 73.24$\pm$7.9}} 
        & \multicolumn{1}{c}{52.86$\pm$9.6} & {\ul 63.43$\pm$7.5} 
        \\ \bottomrule
    \end{tabular}%
    }
\end{table*}

The diagnosis of anemia relies on presence of the minority red blood cells in a patient's blood sample that shows morphological features associated with the disease. 
%
Anemia disorders lead to various aberrant shapes such as sickle-shaped (SCD), crumpled or perforated (thalassemia), star-shaped (Xero), or even spherical (HS) cells. 
%
These deformations can manifest with varying degrees of severity and in different proportions, while it is also possible for other cell types unrelated to anemia conditions to coexist.
%
Detecting the hallmark cells indicative of anemia poses a significant challenge due to substantial variability in expert opinions. 
%
% Furthermore, a few atypical cells do not indicate an anemia condition, complicating the diagnostic process.
%
This makes the \emph{manual} annotation of blood samples for supervised model training a laborious and costly endeavor~\citep{kazeminia2022anomaly}. 
%
Lacking cell-level annotations, MIL is used in this context by treating cells as instances and blood samples as bags, with anemia types assigned to each blood sample~\citep{lu2020clinical}.
%


This dataset consists of $521$ microscopy images of blood samples obtained from patients who underwent various treatments at different times. 
%
Each sample comprises $4$ to $12$ images, each containing $12$ to $45$ cells.
%
The data is distributed among five classes, i.e.,
%
\begin{inparaenum}[(i)]
\item Sickle Cell Disease (SCD) with $13$ patients and $170$ samples, 
\item Thalassemia with only $3$ patients and $25$ samples,
\item Hereditary Xerocytosis with $9$ patients and $56$ samples,
\item Hereditary Spherocytosis (HS) with $13$ patients and $89$, as well as
\item healthy control group consisting of $33$ individuals and $181$ samples.
\end{inparaenum}
%
Given the rarity of disease samples in anemia, the dataset for the $5$-class Anemia classification task is exhibiting data scarcity of training bags, making its classification challenging.
%
Following previous research, we implement a patient-centric approach by dividing the dataset into three equivalent folds. 
%
This division allocates two folds for training and reserves one for test. 
%

In this application, \citet{kazeminia2022anomaly} introduces the state-of-the-art MIL approach, introducing anomaly scores derived from the Mahalanobis distance to a Gaussian mixture model for detecting negative instances in anemia classification. 
%
However, the effectiveness of this method is limited by the encoder's capacity to map negative instances without a direct learning signal.
%

In addition to data scarcity, the anemia dataset introduces an inherent ambiguity within the dataset:
Blood samples may contain a low deformed cell ratio that falls below a specified threshold to identify a disorder.
%
Such data introduces a significant challenge for attention-based and anomaly-aware pooling techniques, as they do not conform to the fundamental assumptions of these mechanisms.
% 
Consequently, it is not just the presence of positive instances that is critical, but also their \emph{proportion} in the data. 
%
We anticipate that geometry and topology can capture this nuanced information, helping us overcome this challenge.
%
Consistent with prior experiments, for a fair comparison, we apply topological regularization to this architectures~\citep{kazeminia2022anomaly}.
%
In previous works, the instance encoder contains $3$ convolutions followed by $2$ ReLu and Tanh activation functions, followed by $2$ linear layers to obtain a latent representation of instances in a $500$-dimensional space. 
%
The instance encoder's input is $4 \times 4 \times 256$ features extracted by a frozen encoder trained in a cell segmentation network. 
%
However, in our experiments, we capture the topological signature of each bag directly from the image data space and the $500$-dimensional latent space. We posit that features extracted by the segmentation model, irrespective of the cell type, may lack crucial shape information and thus potentially manipulate the topology of the bag (see Table \ref{table:mil_rbc_model}).

Following common practice in medical and biomedical evaluations, we employ five standard evaluation metrics: Accuracy, F1-Score, Area Under the Receiver Operating Characteristic Curve (AUROC), Precision, and Recall.
%
All metrics are macro-weighted because of the dataset imbalance.

\paragraph{Results.}

% Figure environment removed

% Figure environment removed

%
\cref{tab:performance} shows that topological regularization improves the performance of MIL models using \emph{all} aggregation functions, resulting in higher mean performance and often resulting in reduced variance.
%
Notably, topologically regularized MIL with \emph{average pooling} surpasses other aggregation schemes.
%
This aligns with our findings from experiments on synthetic datasets, where we observe that topological regularization particularly narrowing the gap between performance of the MIL employing different aggregation functions.
%
The inherent ambiguity in the anemia dataset for MIL suggests that enhancing instance projection in latent space via average pooling is more effective than attention pooling, as it better captures the ratio of positive instances. 
%
Without topological regularization, scarce training data impede the instance encoder from generating meaningful, generalizable latent representations. 
%
However, integrating topological inductive bias into the latent space mitigates these challenges, significantly improving model performance.

\paragraph{Instance-level analysis.}
%
We evaluate the influence of topological regularization on the instance-level explanation of the anomaly-aware MIL approach. 
%
Figure \ref{fig:interpretation} shows anomaly scores achieved with and without topological regularization. 
%
Without topological regularization, we observe a notable inconsistency: the anomaly detector assigns different anomaly scores to visually similar instances.
%
This inconsistency is mitigated when topological regularization is applied.
%
This is an important aspect of our analysis, revealing a challenge in the model's ability to evaluate similar data points uniformly and demonstrating the effectiveness of topological regularization in enhancing the model's explainability.
%
Further illustrating this point, we visualize the distance matrix of instances within a bag in the input space and compare them with their corresponding matrices in the latent space in scenarios with and without topological regularization in Figure \ref{fig:dists_umaps}.
%
This figure shows that MIL with topological regularization better preserves the distances between bag instances in the latent space projection compared to anomaly-aware MIL without regularization.
%
The figure indicates that, without topological regularization, only a few instances are projected far from the majority, elucidating the observed inconsistency in anomaly scores for deformed shapes.
%

In addressing potential inquiries regarding our choice of topological regularization over a distance-preservation-based loss, it's noteworthy to emphasize the distinct advantages of our approach. 
%
Topological regularization loss is particularly robust against noise and highly effective in high-dimensional spaces. 
%
It exhibits scale invariance, a critical feature that enables the preservation of the distance pattern of instances within a bag. 
%
This level of distance pattern preservation might not be as effectively achieved with a regularization loss focused solely on distance preservation. 
%
This aspect underscores the strategic advantage of our chosen method, confirming the efficacy of topological regularization in maintaining the integrity of instance relationships in the latent space, thereby enhancing both the model's performance and its explainability.



\section{Conclusion}
We introduce a novel event-based approach for background image reconstruction in the presence of dynamic occlusions.
It leverages the complementary nature of event camera and frames to reconstruct true scene information instead of hallucinating occluded areas as done by image inpainting approaches.
Specifically, our proposed data-driven approach reconstructs the background image using only one occluded image and events.
The high temporal resolution of the events provides our method additional information on the relative intensity changes between the foreground and background, making it robust to dense occlusions. %
To evaluate our approach, we present the first large-scale dataset recorded in the real world containing over $230$ challenging scenes with synchronized events, occluded images, and groundtruth images.
Our method achieves an improvement of 3dB in PSNR over state-of-the-art frame-based and event-based methods on both synthetic and real datasets.
We will release our synthetic and recorded dataset representing the first datasets for background image reconstruction using events and images in the presence of dynamic occlusions.
We believe that our proposed method and dataset lay the foundation for future research.


% \clearpage

\section{Impact Statement}
This paper presents work with the primary goal of advancing the field of Machine Learning in healthcare. 
We have carefully considered the ethical implications of our methodology and application and have not identified any negative social impact that must be specifically highlighted here.

\section*{Acknowledgments}

We gratefully acknowledge Anna Bogdanova, Asya Makhro, and Ario Sadafi for providing the biomedical dataset used in this research. Their efforts in collecting and sharing the data have been instrumental in advancing our understanding of this field.
% Helmholtz
The Helmholtz Association supports the present contribution under the joint research school “Munich School for Data Science - MUDS”.
% Carsten funding
C.M.\ has received funding from the European Research Council (ERC) under the European Union’s Horizon 2020 research and innovation program (Grant Agreement No. 866411).
%
B.R.\ is supported by the Bavarian state government with
funds from the \emph{Hightech Agenda Bavaria}.



% \bibliography{ref}
\begin{thebibliography}{33}
\providecommand{\natexlab}[1]{#1}
\providecommand{\url}[1]{\texttt{#1}}
\expandafter\ifx\csname urlstyle\endcsname\relax
  \providecommand{\doi}[1]{doi: #1}\else
  \providecommand{\doi}{doi: \begingroup \urlstyle{rm}\Url}\fi

\bibitem[Andrews et~al.(2002)Andrews, Tsochantaridis, and Hofmann]{andrews2002support}
Andrews, S., Tsochantaridis, I., and Hofmann, T.
\newblock Support vector machines for multiple-instance learning.
\newblock \emph{Advances in neural information processing systems}, 15, 2002.

\bibitem[Bubenik et~al.(2020)Bubenik, Hull, Patel, and Whittle]{Bubenik20a}
Bubenik, P., Hull, M., Patel, D., and Whittle, B.
\newblock Persistent homology detects curvature.
\newblock \emph{Inverse Problems}, 36\penalty0 (2):\penalty0 025008, 2020.

\bibitem[Chazal et~al.(2009)Chazal, Cohen-Steiner, Guibas, Mémoli, and Oudot]{Chazal09a}
Chazal, F., Cohen-Steiner, D., Guibas, L.~J., Mémoli, F., and Oudot, S.~Y.
\newblock Gromov--{H}ausdorff stable signatures for shapes using persistence.
\newblock \emph{Computer Graphics Forum}, 28\penalty0 (5):\penalty0 1393--1403, 2009.

\bibitem[Chen et~al.(2019)Chen, Ni, Bai, and Wang]{Chen19a}
Chen, C., Ni, X., Bai, Q., and Wang, Y.
\newblock A topological regularizer for classifiers via persistent homology.
\newblock In \emph{International Conference on Artificial Intelligence and Statistics}, pp.\  2573--2582, 2019.

\bibitem[Chen et~al.(2022)Chen, Chen, Li, Chen, Trister, Krishnan, and Mahmood]{chen2022scaling}
Chen, R.~J., Chen, C., Li, Y., Chen, T.~Y., Trister, A.~D., Krishnan, R.~G., and Mahmood, F.
\newblock Scaling vision transformers to gigapixel images via hierarchical self-supervised learning.
\newblock In \emph{Proceedings of the IEEE/CVF Conference on Computer Vision and Pattern Recognition}, pp.\  16144--16155, 2022.

\bibitem[Cormen et~al.(2022)Cormen, Leiserson, Rivest, and Stein]{cormen2022introduction}
Cormen, T.~H., Leiserson, C.~E., Rivest, R.~L., and Stein, C.
\newblock \emph{Introduction to algorithms}.
\newblock MIT press, 2022.

\bibitem[Dietterich et~al.(1997)Dietterich, Lathrop, and Lozano-P{\'e}rez]{dietterich1997solving}
Dietterich, T.~G., Lathrop, R.~H., and Lozano-P{\'e}rez, T.
\newblock Solving the multiple instance problem with axis-parallel rectangles.
\newblock \emph{Artificial intelligence}, 89\penalty0 (1-2):\penalty0 31--71, 1997.

\bibitem[Du et~al.(2023)Du, Mao, Zhang, Gou, Jiao, and Xiong]{du2023rgmil}
Du, Z., Mao, S., Zhang, Y., Gou, S., Jiao, L., and Xiong, L.
\newblock Rgmil: Guide your multiple-instance learning model with regressor.
\newblock In \emph{Thirty-seventh Conference on Neural Information Processing Systems}, 2023.

\bibitem[Edelsbrunner \& Harer(2009)Edelsbrunner and Harer]{edelsbrunner2009computational}
Edelsbrunner, H. and Harer, J.
\newblock Computational topology: An introduction.
\newblock \emph{American Mathematical Society}, 9\penalty0 (2):\penalty0 117--138, 2009.

\bibitem[Goyal \& Bengio(2022)Goyal and Bengio]{goyal2022inductive}
Goyal, A. and Bengio, Y.
\newblock Inductive biases for deep learning of higher-level cognition.
\newblock \emph{Proceedings of the Royal Society A}, 478\penalty0 (2266):\penalty0 20210068, 2022.

\bibitem[Hehr et~al.(2023)Hehr, Sadafi, Matek, Lienemann, Pohlkamp, Haferlach, Spiekermann, and Marr]{hehr2023explainable}
Hehr, M., Sadafi, A., Matek, C., Lienemann, P., Pohlkamp, C., Haferlach, T., Spiekermann, K., and Marr, C.
\newblock Explainable ai identifies diagnostic cells of genetic aml subtypes.
\newblock \emph{PLOS Digital Health}, 2\penalty0 (3):\penalty0 e0000187, 2023.

\bibitem[Hensel et~al.(2021)Hensel, Moor, and Rieck]{Hensel21}
Hensel, F., Moor, M., and Rieck, B.
\newblock A survey of topological machine learning methods.
\newblock \emph{Frontiers in Artificial Intelligence}, 4, 2021.

\bibitem[Horn et~al.(2022)Horn, {De Brouwer}, Moor, Moreau, Rieck, and Borgwardt]{Horn22a}
Horn, M., {De Brouwer}, E., Moor, M., Moreau, Y., Rieck, B., and Borgwardt, K.
\newblock Topological graph neural networks.
\newblock In \emph{International Conference on Learning Representations~(ICLR)}, 2022.
\newblock URL \url{https://openreview.net/forum?id=oxxUMeFwEHd}.

\bibitem[Huang et~al.(2022)Huang, Liu, Jin, and Mu]{huang2022bag}
Huang, S., Liu, Z., Jin, W., and Mu, Y.
\newblock Bag dissimilarity regularized multi-instance learning.
\newblock \emph{Pattern Recognition}, 126:\penalty0 108583, 2022.

\bibitem[Ilse et~al.(2018)Ilse, Tomczak, and Welling]{ilse2018attention}
Ilse, M., Tomczak, J., and Welling, M.
\newblock Attention-based deep multiple instance learning.
\newblock In \emph{International conference on machine learning}, pp.\  2127--2136. PMLR, 2018.

\bibitem[Kazeminia et~al.(2022)Kazeminia, Sadafi, Makhro, Bogdanova, Albarqouni, and Marr]{kazeminia2022anomaly}
Kazeminia, S., Sadafi, A., Makhro, A., Bogdanova, A., Albarqouni, S., and Marr, C.
\newblock Anomaly-aware multiple instance learning for rare anemia disorder classification.
\newblock In \emph{25th International Conference on Medical Image Computing and Computer Assisted Intervention~(MICCAI)}, pp.\  341--350. Springer, 2022.

\bibitem[Li et~al.(2021)Li, Li, and Eliceiri]{li2021dual}
Li, B., Li, Y., and Eliceiri, K.~W.
\newblock Dual-stream multiple instance learning network for whole slide image classification with self-supervised contrastive learning.
\newblock In \emph{Proceedings of the IEEE/CVF conference on computer vision and pattern recognition}, pp.\  14318--14328, 2021.

\bibitem[Lu et~al.(2020)Lu, Han, Liu, Niu, Zhang, Li, Zhou, and Zhang]{lu2020clinical}
Lu, C., Han, B., Liu, Y., Niu, G., Zhang, R., Li, E., Zhou, Y., and Zhang, S.
\newblock Clinical-grade computational pathology using weakly supervised deep learning on whole slide images.
\newblock \emph{Nature Medicine}, 26\penalty0 (9):\penalty0 1301--1309, 2020.

\bibitem[Moor et~al.(2020)Moor, Horn, Rieck, and Borgwardt]{moor2020topological}
Moor, M., Horn, M., Rieck, B., and Borgwardt, K.
\newblock Topological autoencoders.
\newblock In \emph{International Conference on Machine Learning}, pp.\  7045--7054, 2020.

\bibitem[Sadafi et~al.(2020)Sadafi, Makhro, Bogdanova, Navab, Peng, Albarqouni, and Marr]{sadafi2020attention}
Sadafi, A., Makhro, A., Bogdanova, A., Navab, N., Peng, T., Albarqouni, S., and Marr, C.
\newblock Attention based multiple instance learning for classification of blood cell disorders.
\newblock In \emph{23rd International Conference on Medical Image Computing and Computer Assisted Intervention~(MICCAI)}, pp.\  246--256. Springer, 2020.

\bibitem[Shao et~al.(2021)Shao, Bian, Chen, Wang, Zhang, Ji, et~al.]{shao2021transmil}
Shao, Z., Bian, H., Chen, Y., Wang, Y., Zhang, J., Ji, X., et~al.
\newblock Transmil: Transformer based correlated multiple instance learning for whole slide image classification.
\newblock \emph{Advances in neural information processing systems}, 34:\penalty0 2136--2147, 2021.

\bibitem[Sheehy(2013)]{Sheehy13a}
Sheehy, D.~R.
\newblock Linear-size approximations to the vietoris--rips filtration.
\newblock \emph{Discrete {\&} Computational Geometry}, 49\penalty0 (4):\penalty0 778--796, 2013.
\newblock \doi{10.1007/s00454-013-9513-1}.

\bibitem[Sheehy(2014)]{sheehy2014persistent}
Sheehy, D.~R.
\newblock The persistent homology of distance functions under random projection.
\newblock In \emph{Proceedings of the thirtieth annual symposium on Computational geometry}, pp.\  328--334, 2014.

\bibitem[Turkes et~al.(2022)Turkes, Montufar, and Otter]{Turkes22a}
Turkes, R., Montufar, G., and Otter, N.
\newblock On the effectiveness of persistent homology.
\newblock In \emph{Advances in Neural Information Processing Systems}, 2022.

\bibitem[Vandaele et~al.(2022)Vandaele, Kang, Lijffijt, Bie, and Saeys]{Vandaele22a}
Vandaele, R., Kang, B., Lijffijt, J., Bie, T.~D., and Saeys, Y.
\newblock Topologically regularized data embeddings.
\newblock In \emph{International Conference on Learning Representations}, 2022.

\bibitem[von Rohrscheidt \& Rieck(2023)von Rohrscheidt and Rieck]{vonRohrscheidt23a}
von Rohrscheidt, J. and Rieck, B.
\newblock Topological singularity detection at multiple scales.
\newblock In Krause, A., Brunskill, E., Cho, K., Engelhardt, B., Sabato, S., and Scarlett, J. (eds.), \emph{Proceedings of the 40th International Conference on Machine Learning~(ICML)}, number 202 in Proceedings of Machine Learning Research, pp.\  35175--35197. PMLR, 2023.

\bibitem[Wagner et~al.(2021)Wagner, Solomon, and Bendich]{wagner2021improving}
Wagner, A., Solomon, E., and Bendich, P.
\newblock Improving metric dimensionality reduction with distributed topology, 2021.
\newblock arXiv:2106.07613.

\bibitem[Wagner et~al.(2023)Wagner, Reisenb{\"u}chler, West, Niehues, Zhu, Foersch, Veldhuizen, Quirke, Grabsch, van~den Brandt, et~al.]{wagner2023transformer}
Wagner, S.~J., Reisenb{\"u}chler, D., West, N.~P., Niehues, J.~M., Zhu, J., Foersch, S., Veldhuizen, G.~P., Quirke, P., Grabsch, H.~I., van~den Brandt, P.~A., et~al.
\newblock Transformer-based biomarker prediction from colorectal cancer histology: A large-scale multicentric study.
\newblock \emph{Cancer Cell}, 41\penalty0 (9):\penalty0 1650--1661, 2023.

\bibitem[Waibel et~al.(2022)Waibel, Atwell, Meier, Marr, and Rieck]{Waibel22a}
Waibel, D. J.~E., Atwell, S., Meier, M., Marr, C., and Rieck, B.
\newblock Capturing shape information with multi-scale topological loss terms for {3D} reconstruction.
\newblock In \emph{25th International Conference on Medical Image Computing and Computer Assisted Intervention~(MICCAI)}, pp.\  150--159, 2022.

\bibitem[Xiao et~al.(2017)Xiao, Rasul, and Vollgraf]{xiao2017fashion}
Xiao, H., Rasul, K., and Vollgraf, R.
\newblock Fashion-mnist: a novel image dataset for benchmarking machine learning algorithms.
\newblock \emph{arXiv preprint arXiv:1708.07747}, 2017.

\bibitem[Yan et~al.(2018)Yan, Wang, Guo, Fang, Liu, and Huang]{yan2018deep}
Yan, Y., Wang, X., Guo, X., Fang, J., Liu, W., and Huang, J.
\newblock Deep multi-instance learning with dynamic pooling.
\newblock In \emph{Asian Conference on Machine Learning}, pp.\  662--677. PMLR, 2018.

\bibitem[Zhang et~al.(2022)Zhang, Meng, Zhao, Qiao, Yang, Coupland, and Zheng]{zhang2022dtfd}
Zhang, H., Meng, Y., Zhao, Y., Qiao, Y., Yang, X., Coupland, S.~E., and Zheng, Y.
\newblock Dtfd-mil: Double-tier feature distillation multiple instance learning for histopathology whole slide image classification.
\newblock In \emph{Proceedings of the IEEE/CVF Conference on Computer Vision and Pattern Recognition}, pp.\  18802--18812, 2022.

\bibitem[Zhao et~al.(2023)Zhao, Yuan, Hao, and Wen]{zhao2023generalized}
Zhao, L., Yuan, L., Hao, K., and Wen, X.
\newblock Generalized attention-based deep multi-instance learning.
\newblock \emph{Multimedia Systems}, 29\penalty0 (1):\penalty0 275--287, 2023.

\end{thebibliography}

\bibliographystyle{icml2024}


%%%%%%%%%%%%%%%%%%%%%%%%%%%%%%%%%%%%%%%%%%%%%%%%%%%%%%%%%%%%%%%%%%%%%%%%%%%%%%%
%%%%%%%%%%%%%%%%%%%%%%%%%%%%%%%%%%%%%%%%%%%%%%%%%%%%%%%%%%%%%%%%%%%%%%%%%%%%%%%
% APPENDIX
%%%%%%%%%%%%%%%%%%%%%%%%%%%%%%%%%%%%%%%%%%%%%%%%%%%%%%%%%%%%%%%%%%%%%%%%%%%%%%%
%%%%%%%%%%%%%%%%%%%%%%%%%%%%%%%%%%%%%%%%%%%%%%%%%%%%%%%%%%%%%%%%%%%%%%%%%%%%%%%
\newpage
\appendix
\onecolumn
\section{Appendix}
\section{Supplementary Proofs}\label{sec:concentration}

% \begin{lemma}\label{lem:concentration}
% For a $d$-dimensional random vector $\textbf{u} \sim \mathcal{N}(0,\sigma^2)$ and $\frac{b}{\sigma} < 1$, 
% \begin{equation}\label{eq:chi1}
%     \underset{\textbf{u} \sim \mathcal{N}(0,\sigma^2)}{\mathbb{P}}[||\textbf{u}||_2 \leq  \sqrt{d} b] \leq e^{\frac{d}{2}}\frac{b^d}{\sigma^d}
% \end{equation}
% \end{lemma}
% \begin{proof}
% We can also write the LHS as
% \begin{equation}\label{eq:chi2}
%         \underset{\textbf{u} \sim \mathcal{N}(0,\sigma^2)}{\mathbb{P}}\left[{\left\lVert \frac{\textbf{u}}{\sigma}\right\rVert}_2^2 \leq  \frac{d b^2}{\sigma^2}\right]
% \end{equation}
% Now, for $t<0$,
% \begin{equation}\label{eq:chi2}
%         \underset{\textbf{u} \sim \mathcal{N}(0,\sigma^2)}{\mathbb{P}}\left[{\left\lVert \frac{\textbf{u}}{\sigma}\right\rVert}_2^2 \leq  \frac{d b^2}{\sigma^2}\right] = \underset{\textbf{u} \sim \mathcal{N}(0,\sigma^2)}{\mathbb{P}}\left[e^{t{\left\lVert \frac{\textbf{u}}{\sigma}\right\rVert}_2^2 } \geq  e^{t\frac{d b^2}{\sigma^2}}\right]
% \end{equation}
% Applying Markov's inequality $\left(\mathbb{P}[X\geq a]\leq \frac{\mathbb{E}[X]}{a}\right)$), we get
% \begin{equation}
%     \underset{\textbf{u} \sim \mathcal{N}(0,\sigma^2)}{\mathbb{P}}\left[e^{t{\left\lVert \frac{\textbf{u}}{\sigma}\right\rVert}_2^2 } \geq  e^{t\frac{d b^2}{\sigma^2}}\right] \leq \frac{\mathbb{E}\left[e^{t{\left\lVert \frac{\textbf{u}}{\sigma}\right\rVert}_2^2 }\right]}{e^{t\frac{d b^2}{\sigma^2}}}
% \end{equation}

% If $\textbf{u} \sim \mathcal{N}(0,\sigma^2)$, then $\frac{\textbf{u}}{\sigma} \sim \mathcal{N}(0,1)$. And $\frac{||\textbf{u}||_2^2}{\sigma^2} \sim \mathcal{X}^2_n$ follows the chi-square distribution. Using the moment generating function for chi-square distribution in Eq. 21 and combining with Eq. 20, we get
% \begin{equation}
% \begin{split}
%     \underset{\textbf{u} \sim \mathcal{N}(0,\sigma^2)}{\mathbb{P}}\left[{\left\lVert \frac{\textbf{u}}{\sigma}\right\rVert}_2^2 \leq  \frac{d b^2}{\sigma^2}\right] &\leq \frac{{(1-2t)}^{\frac{-d}{2}}}{e^{t\frac{d b^2}{\sigma^2}}} \\
%     &= \exp\left(\frac{-d}{2}\log{(1-2t)} - \frac{td b^2}{\sigma^2}\right)\\
% \end{split}
% \end{equation}
% To get a tight upper bound, we can replace a value of $t$ that minimizes the RHS i.e. $t = \frac{1}{2}\left[1-\frac{\sigma^2}{b^2}\right]$. Since, $t<0$, we need $\sigma > b$. Using this, we get 
% \begin{equation}
% \begin{split}
%     \underset{\textbf{u} \sim \mathcal{N}(0,\sigma^2)}{\mathbb{P}}\left[{\left\lVert \frac{\textbf{u}}{\sigma}\right\rVert}_2^2 \leq  \frac{d b^2}{\sigma^2}\right] &\leq \exp\left( \frac{d}{2}\log\left( \frac{b^2}{\sigma^2} \right) - \frac{db^2}{2\sigma^2} + \frac{d}{2} \right)\\
%     &\leq \exp\left( \frac{d}{2}\log\left( \frac{b^2}{\sigma^2} \right) + \frac{d}{2} \right)\\
%     &= e^{\frac{d}{2}} \frac{b^d}{\sigma^d}
% \end{split}
% \end{equation}
% \end{proof}

\begin{lemma}
Let $G$ be a $k \times d$ random matrix with rows $\sigma g^i \sim \mathcal{N}(0, \mathbf{I_d}\sigma^2) ~\forall 1 \leq i \leq k$. Then, for any unit vector $v \in \mathbb{R}^d$,
\begin{equation*}
    P[|\|Gv\|^2 - 1| > \epsilon] \leq 2 exp\left(-\left(k + \dfrac{\epsilon + 1}{2 \sigma^2}\right)\right)
\end{equation*}
\label{lemma:concn_msur}
\end{lemma}

\begin{proof} Note that by rotational invariance of Gaussians, $Gv \stackrel{D}{=} Ge^1$, where $e^1$ is the standard basis vector. This implies that $\|Gv\|^2 \stackrel{D}{=} \|Ge^1\|^2 \stackrel{D}{=} \sigma^2 \chi_k^2$, where $\chi_k^2$ is a chi-square random variable with $k$-degrees of freedom. Then, by Chernoff's bounding method:

\begin{equation*}
\begin{split}
    P[|\|Gv\|^2 - 1| > & \epsilon] = P[|\sigma^2 \chi_k^2 - 1| > \epsilon] \\ \\
    &\leq 2 \inf\limits_{t>0}e^{-\epsilon t}\mathbb{E}[e^{t(\sigma^2\chi^2-1)}] \\
    &\leq 2 \inf_{t>0} e^{-\epsilon t - t} \mathbb{E}[e^{t\sigma^2 \chi^2}] \\
    & \leq 2 \inf_{t>0} e^{-\epsilon t - t}(1-2\sigma^2 t)^{\frac{-k}{2}} \\
    & \leq 2 \inf_{t > 0} e^{-\epsilon t - t - \frac{k}{2} \log(1-2\sigma^2 t)} \\
    & \leq 2 e^{- \frac{\epsilon - k \sigma^2 + 1}{2\sigma^2} - \frac{k}{2}\log(\frac{\sigma^2 k}{\epsilon + 1})} \\
    & \leq 2 e^{- \frac{(\epsilon + 1 -\sigma^2 k)^2}{2\sigma^2(1+\epsilon)}} \\
    & \leq 2e^{- \frac{(\epsilon+1)^2 - 2\sigma^2 k(\epsilon+1)}{2\sigma^2 (1+\epsilon)}} \\
    & \leq 2e^{- \frac{\epsilon+1 - 2\sigma^2 k}{2\sigma^2}} \\
    & \leq 2e^{-k}e^{-\frac{\epsilon+1}{2\sigma^2}} \\
    & \leq 2e^{-k - \frac{\epsilon+1}{2\sigma^2}}
\end{split}
\end{equation*}
\end{proof}

\subsubsection{Proof for Theorem~\ref{lemma:min_var_grad}}\label{app:proof_min_grad_var}
Let $\nabla_\mathbf{x}$ be the true gradient of $\mathbf{x}$ for the classifier's loss, and $G$ be a matrix of rows $g_1, \cdots, g_k \sim \mathcal{N}(0, \mathbf{I_d}\beta^2)$. Then, the norm of estimated gradient $G\cdot\nabla_\mathbf{x} $ is bounded in probability by:
\begin{equation*}
\begin{split}
    \mathbb{P}[(1-\epsilon)\|\nabla_\mathbf{x}\| \leq \|G\cdot\nabla_\mathbf{x}\| & \leq (1+\epsilon) \|\nabla_\mathbf{x}\|] \geq  \\
     & 1 - 2\cdot exp{\left(- k - \dfrac{1 + \epsilon}{2\beta^2}\right)}
\end{split} 
\end{equation*}
where $0 \leq \epsilon \leq 1$ is the estimation error.

\begin{proof}
    \begin{equation*}
    \begin{split}
   &  P[(1-\epsilon)\|\nabla_\mathbf{x}\| \leq \|G\nabla_\mathbf{x}\| \leq (1+\epsilon)\|\nabla_\mathbf{x}\|] \\
   & = P[(1-\epsilon)^2\|\nabla_\mathbf{x}\|^2 \leq \|G\nabla_\mathbf{x}\|^2 \leq (1+\epsilon)^2\|\nabla_\mathbf{x}\|^2] \\
   & \geq P\left[1-\epsilon \leq \dfrac{\|G \nabla_\mathbf{x}\|^2}{\|\nabla_\mathbf{x}\|^2} \leq 1 + \epsilon \right] \\
   & = P\left[\left|\dfrac{\|G \nabla_\mathbf{x}\|^2}{\|\nabla_\mathbf{x}\|^2} - 1\right| \leq  \epsilon \right] \\
   &= P\left[\left|\left\|G\dfrac{ \nabla_\mathbf{x}}{\|\nabla_\mathbf{x}\|}\right\|^2 - 1\right| \leq  \epsilon \right] \\
   & \geq 1 - 2e^{-k - \frac{\epsilon+1}{2\beta^2}}
    \end{split}
    \end{equation*}
    Where the last step is by Lemma 1.
\end{proof}
%%%%%%%%%%%%%%%%%%%%%%%%%%%%%%%%%%%%%%%%%%%%%%%%%%%%%%%%%%%%%%%%%%%%%%%%%%%%%%%
%%%%%%%%%%%%%%%%%%%%%%%%%%%%%%%%%%%%%%%%%%%%%%%%%%%%%%%%%%%%%%%%%%%%%%%%%%%%%%%


\end{document}


% This document was modified from the file originally made available by
% Pat Langley and Andrea Danyluk for ICML-2K. This version was created
% by Iain Murray in 2018, and modified by Alexandre Bouchard in
% 2019 and 2021 and by Csaba Szepesvari, Gang Niu and Sivan Sabato in 2022.
% Modified again in 2023 and 2024 by Sivan Sabato and Jonathan Scarlett.
% Previous contributors include Dan Roy, Lise Getoor and Tobias
% Scheffer, which was slightly modified from the 2010 version by
% Thorsten Joachims & Johannes Fuernkranz, slightly modified from the
% 2009 version by Kiri Wagstaff and Sam Roweis's 2008 version, which is
% slightly modified from Prasad Tadepalli's 2007 version which is a
% lightly changed version of the previous year's version by Andrew
% Moore, which was in turn edited from those of Kristian Kersting and
% Codrina Lauth. Alex Smola contributed to the algorithmic style files.
