\section{Related Work}

MIL aggregation functions can take simple forms, such as max or average pooling, or more complex forms, like attention-based pooling, as commonly used in various MIL models in the literature~\citep{ilse2018attention, li2021dual, shao2021transmil, kazeminia2022anomaly, zhang2022dtfd, zhao2023generalized}.
%
Simple aggregation functions are computationally efficient and easy to implement but may overlook instance-level information, whereas complex forms offer a more detailed understanding of relevant instances at the cost of increased computational complexity.
%
Attention-based pooling amplifies the learning signal for important instances, ensuring improved classification performance, even when only a subset of positive instances is distinguished. 
%
However, in scenarios where instance-level performance is crucial, the attention-based pooling approach may lack the necessary reliability, as it does not uniformly enhance expression across all instances.
%
To overcome this limitation, \citet{du2023rgmil} recently introduced a regressor-guided aggregator that removes learnable attention parameters and rectifies the inference process, thereby enhancing the direct passage of the learning signal through the encoder. 
%
Although this state-of-the-art approach significantly refines instance-level representation and enhances overall MIL performance, it still faces challenges when dealing with scarcity of training data. 
%
In such scenarios, the regressor-guided aggregator may struggle to accurately capture and represent the nuanced variations within the data, leading to potential limitations in model generalization and reliability. 
%
This is particularly evident in settings where the data lacks diversity or sufficient examples of certain classes, making it difficult for the model to learn and generalize effectively.
%
To address these limitations, we introduce topological regularization, establishing a more robust and rational inductive bias that enhances the model's overall performance and generalization.

% \paragraph{Anemia classification.}
% %
% The diagnosis of anemia relies on the minority of red blood cells in a patient's blood sample that shows morphological features associated with the disease. 
% %
% Anemia disorders lead to various aberrant shapes such as sickle-shaped (SCD), crumpled or perforated (thalassemia), star-shaped (Xero), or even spherical (HS) cells. 
% %
% These deformations can manifest with varying degrees of severity and in different proportions, while it's also possible for other cell types unrelated to Anemia conditions to coexist.
% %
% Detecting the hallmark cells indicative of anemia poses a significant challenge due to substantial variability in expert opinions. 
% %
% Furthermore, a few atypical cells do not indicate an anemia condition, complicating the diagnostic process.
% %
% This makes the \emph{manual} annotation of blood samples for supervised model training a laborious and costly endeavor~\citep{kazeminia2022anomaly}. 
% %
% In the absence of cell-level annotations, MIL was employed in this context.
% %
% In the context of MIL, individual instances (cells) are organized within bags (blood samples), and a singular label is assigned to represent the anemia type presented in the entire bag~\citep{lu2020clinical}.
% %
% \citet{kazeminia2022anomaly} as the state-of-the-art in this MIL application proposes the concept of anomaly score added to attention aggregation to facilitate detection of negative instances encountering their Mahalanobis distance to the fitted Gaussian mixture model on negative instances distribution.
% %
% Although this method demonstrated state-of-the-art performance in anemia classification, its effectiveness in estimating anomaly scores is constrained by the encoder's ability to map negative instances in close proximity, with no direct learning signal guiding this process.
% %
% Even with a stronger learning signal for a negative instance, having only a limited training data, the encoder is in danger of overfitting and can not generalize well on unseen instances of diversities.
% % 
