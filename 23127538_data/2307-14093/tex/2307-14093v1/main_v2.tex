\documentclass[reprint,fleqn,10pt]{wlscirep}
\usepackage[utf8]{inputenc}
\usepackage[T1]{fontenc}
\usepackage{bm}
\usepackage{multicol}
%\usepackage{caption}
\usepackage{subcaption}
\captionsetup[subfigure]{font={bf,small}, justification=raggedright,singlelinecheck=false}
\usepackage{enumitem}
\usepackage{tcolorbox}
% \subcaptionsetup[subfigure]{font={bf,small}, justification=raggedright,singlelinecheck=false}
\newcommand{\bb}{{\bm b}}
\newcommand{\bj}{{\bm j}}
\newcommand{\bu}{{\bm u}}
\newcommand{\bx}{{\bm x}}
\newcommand{\br}{{\bm r}}
\newcommand{\bomega}{{\bm \omega}}
\newcommand{\bQ}{{\bm Q}}
\newcommand{\lamb}{{\bm u\times\bm \omega}}
\newcommand{\bnabla}{{\bm \nabla}}
\newcommand{\bxiQdivQ}{{\xi \bm Q(\bm \nabla\cdot \bm Q)}}
\usepackage[switch]{lineno} 
\usepackage{float}
\usepackage{graphicx}
\usepackage{comment}
\usepackage{enumerate}
\usepackage{amsmath,nccmath}
\usepackage{xcolor}
\usepackage{mathrsfs}
\usepackage{dcolumn}% Align table 

\title{Universal relaxation of turbulent binary fluids}

% Universal turbulent relaxation by vanishing nonlinear transfer: A quiescent pressure-balance relaxed states 

\author[1]{Nandita Pan}
\author[1,*]{Supratik Banerjee}
\author[1]{Arijit Halder}
\affil[1]{Department of Physics, Indian Institute of Technology, Kanpur, INDIA, 208016}
\affil[*]{e-mail: sbanerjee@iitk.ac.in}


\begin{document}
\twocolumn[
  \begin{@twocolumnfalse}

\begin{abstract}
Upon quenching the forcing, a turbulent system tends to attain the state of stable equilibrium through the process of turbulent relaxation. Such relaxation in binary fluids is of surmount interest for both fundamental science understanding and industrial applications. A systematic investigation of the same has been carried out, for the first time, using direct numerical simulations of Cahn-Hilliard-Navier-Stokes equations. With the help of a thorough scanning, the bulk of each fluid and its interface are found to relax in a different way. However, using the principle of vanishing nonlinear transfer, we propose a convincing, universal pathway of obtaining the turbulent relaxed states for both the bulk and the interface which attain a relaxed state when the turbulent cascades of the inviscid invariants are suppressed. Interestingly, the relaxation of the bulk turns up to be subtly different from the turbulent relaxation of a single hydrodynamic fluid and the interface relaxation is found to follow a Helmholtz-like pressure-balanced condition. 
\end{abstract}

\flushbottom
\maketitle
   \end{@twocolumnfalse}
]
\thispagestyle{empty}

Binary fluids encompass a wide range of two-component systems consisting of  natural oil-water mixtures, cosmetic fluids, active binary suspensions \textit{etc}\cite{Stalidis1990, Erucar2016,tiribocchi2015, Scarbolo2015, Cates2018}. Above a critical temperature ($T_c$), they exist as a homogeneous phase-mixed  emulsion which finds its prolific usage starting from industrial applications (\textit{e.g.,} food, chemical, pharmaceutical products \textit{etc.}) to droplet dynamics in oceanic and atmospheric turbulence \cite{Pascual2021,chan2021,mazzitelli2003turbulent, serizawa1975,Bailey1993, narsimhan2019guidelines,Shenoy2015}. Below $T_c$, binary fluids tend to attain a stable phase separated state (with minimum free energy) through coarsening dynamics or spinodal decomposition  \cite{Bray1995, hohenberg1977, chaikin1995,Koga1993}. Such coarsening can be arrested by the presence of a driven turbulent flow which fragments these separated domains by virtue of its enhanced mixing properties thus leading to a non-equilibrium steady emulsion at large times\cite{berti2005,perlekar2014,perlekar2019, Pan2022,Mukherjee2019}. Upon the quenching of the turbulence drive, the system is expected to relax towards the equilibrium phase-separated state. A thorough understanding of this turbulent relaxation process is therefore pivotal in controlling the state of emulsion in a binary fluid system. 

Albeit largely studied, a pristine characterization of the turbulent relaxed states in neutral fluids and plasmas has been a matter of long-standing debate. While a Beltrami-Taylor type force-free relaxed state has been observed in cosmic plasmas \cite{Chandrasekhar1958, Woltjer1958a,Taylor1974}, a clear signature of Grad-Shafranov type pressure-balanced state has been found in incompressible fluids and plasmas with moderate plasma-$\beta$ \cite{Zhu1995, Sato1996,Kraichnan1988}. 
Although the aligned states are explained using the principle of selective decay, a general theory accounting for the non-vanishing pressure gradient was lacking until recently where a universal mechanism for fluids and plasma relaxation has been proposed based on the suppression of turbulent cascades of inviscid invariants \cite{Banerjee2023}.

Relaxation in binary fluids has been investigated under various situations including the transition of the binary fluid above and below the critical temperature\cite{Koga1993}, the dielectric relaxation of polar binary mixtures\cite{Baba1969}, the long-time glasslike relaxation\cite{Benzi2011}, secondary relaxation in supercooled binary fluids\cite{Harbola2003}, relaxation of optically heated binary colloids etc\cite{Araki2022}. Despite substantial practical importance,  turbulent relaxation in binary fluids has not been explored to date. Interestingly, unlike an ordinary fluid, a binary fluid system is governed by the Cahn-Hilliard-Navier-Stokes (CHNS) equations which collectively describe the evolution of both the bulk of the individual fluids and their mutual interface. When the driving force is switched off, both the bulk and the interfacial regions are expected to go through turbulent relaxation. Our principal objective includes the obtention and complete characterization of such relaxed states. In particular, (i) whether 
such a state will be a force-free or a pressure-balanced state, (ii) whether there exists a universal way to describe the relaxation both in the bulk of each fluid and their interfaces, and finally (iii) whether the bulk of individual fluids relaxes similar to an ordinary hydrodynamic fluid. These questions are fundamental from the perspective of both complex fluid dynamics and industrial applications. 

In this Article, we address these questions in order and show that the turbulent relaxation of a binary fluid is not exactly similar to that of a single hydrodynamic fluid or plasma. Further, the way the bulk relaxes is significantly different from that of the interface. However, the relaxation of both the bulk and the interface can be universally described using the principle of vanishing nonlinear transfer (PVNLT)\cite{Banerjee2023}. Finally, we show that the turbulent relaxation of each component fluid deviates from the relaxation of a single fluid where a pressure balanced relaxation is obtained before the fluid ceases to flow.  
    
% Figure environment removed
 


\section*{The CHNS model}
\label{model_invariants}
As mentioned in the introduction, the evolution of binary fluids is described by the CHNS equations\cite{berti2005,Pan2022}
 \begin{linenomath}
 \begin{align}
     \partial_{t} \bu &=  \lamb -   \xi \bQ (\bm{\nabla} \cdot \bQ) - \bm{\nabla}P + \nu \nabla^2 \bu + \bm{f} \label{Eu},\\
     \partial_{t} \phi &=  -\bu\cdot \bQ + \mathcal{M} \nabla^2 \mu , \label{EQ}
 \end{align}
\end{linenomath}
where $\bu (\bm{x}, t)$ is the velocity of the center of mass fluid, $\bomega = \bnabla \times \bu$ the vorticity field, $\bQ (\bm{x}, t)=\bm{\nabla}\phi$ the composition gradient field, where the composition field $\phi (\bm{x}, t)$ represents the local normalized density difference of the two fluids, $P = p + u^2/2 + \phi \mu - \Gamma$ is the total pressure, with $p$ being the fluid pressure, $\mu (=\delta \mathcal{F}/\delta \phi)$ the chemical potential derived from a free energy functional \begin{linenomath}$$\mathcal{F} = \int \left[ \frac{a}{2} \phi^2 + \frac{b}{4} \phi^4 + \xi (\bm{\nabla} \phi )^2\right]
d\tau$$ \end{linenomath} and $\Gamma$ the homogeneous part of the free energy. Here, we perform direct numerical simulations of CHNS equations (Methods). The entire analysis is based on the runs NS ($512^3$ hydrodynamic simulation
) and CHNS2 ($512^3$ binary fluid simulation).
To ensure that the system remains below the critical temperature $T_c$, we have taken $a=-b=-1$. Unlike a passive scalar field\cite{Yaglom1949}, the composition field $\phi$ provides feedback $-\bxiQdivQ$ to the Navier-Stokes equation due to the presence of two-fluid interfaces \cite{Cates2018},  where the parameter $\xi$ is associated with the width of the interface and is chosen in order to numerically resolve the same (Methods). The parameters $\nu$ and $\mathcal{M}$ are the fluid viscosity and the mobility coefficient respectively. Eqs.~\eqref{Eu} and \eqref{EQ} are supplemented by the incompressibility condition $\bnabla\cdot\bu=0$. 
% Figure environment removed

The large-scale stirring force $\bm{f}$ is applied to yield fully developed turbulence which subsequently leads to a phase-arrested steady emulsion state (left panel of Fig.\ref{fig:phi_color}). Upon quenching the forcing ($\bm{f} = \bm{0}$), the emulsion undergoes turbulent relaxation by domain coarsening (middle and right panel of Fig.\ref{fig:phi_color}) to ultimately attain an equilibrium phase-separated state. The whole binary fluid system can be decomposed into bulk and interface according to whether $|\bQ|=0$ or $|\bQ| \neq 0$ respectively (Fig.\ref{fig:Bulk_interface_19}). In the following, we shall systematically study and characterize the turbulent relaxation of both bulk and interface separately. 


\section*{Relaxation of bulk and interface}
A system is expected to reduce its nonlinearities during turbulent relaxation \cite{Kraichnan1988, Stribling1991, Servidio2008, Matthaeus2008}. For a hydrodynamic (HD) fluid ($\bQ=\bm{0}$), the relaxed state is obtained by vanishing the net nonlinear contribution in the Navier-Stokes (NS) equations to have an alignment between $\lamb$ and $\bm{\nabla} P$ rather than a pure Beltrami-Taylor type $\bu$-$\bomega$ alignment\cite{Kraichnan1988}. For a binary fluid, we have additional nonlinear feedback $\xi\bQ\left(\bm{\nabla}\cdot\bQ\right)$ in the momentum evolution equation and it is therefore natural to study the evolution of the individual nonlinear contribution and the combination of them during the relaxation. We start by looking into the time evolutions of the averaged nonlinear terms for both the bulk and the interface before and after the quenching of the driving force in Fig.\ref{fig:Average_timeS}.  

In the bulk, we found that all $\langle|\lamb|\rangle$, $\langle|\bnabla P|\rangle$  and $\langle|\lamb-\bnabla P|\rangle$  damp out during relaxation. In addition, $\bQ$ vanishes identically in the bulk region thus leading to zero contribution from the nonlinear term $\bxiQdivQ$ to the corresponding relaxed states. In the interfacial region, again $\langle|\lamb|\rangle$ decays to zero at large times while $\bQ$ does not vanish due to the presence of a chemical potential gradient across the interface. However, unlike the bulk, $\langle|\bnabla P|\rangle$ does not fall off during relaxation. 
In particular, $\langle|\xi\bQ(\bnabla\cdot\bQ)|\rangle$ and $\langle|\bnabla P|\rangle$ are found to attain the same value at large times with $\langle|-\xi\bQ(\bnabla\cdot\bQ)-\bnabla P|\rangle$ dying out quickly indicating a balance between the feedback term and the total pressure gradient. 
Despite the apparent discrepancy in the turbulent relaxation of the bulk and the interface, it is clearly observed that the quantity $\langle|\lamb-\bxiQdivQ-\bm{\nabla}P|\rangle$
decreases to zero both for the bulk and the interfacial relaxation. In the following, we shall probe this particular aspect in more details to universally characterize the turbulent relaxation of both bulk and interface. 
In order to analyze and compare the relaxation-decay of different quantities, we study the evolution of the probability density function (PDF) of different nonlinear contributions at earlier stages of relaxation ($t_{nl} = 16, 18$ and $20$ and the forcing is withdrawn just after $t_{nl} = 16$) where the decay rates were pronounced. 
Such a study convincingly characterizes the relaxed states at every point of the flow field in terms of the local field variables. 

From Fig.\ref{fig:Fig2_a}, looking into the early stage dynamics of bulk relaxation, we observe that the peaks of the PDFs of all $|\lamb|$, $|\bnabla P| (\equiv |-\bxiQdivQ - \bnabla P|$ in bulk) and $|\lamb - \bnabla P|$ shift towards zero during the relaxation. The dominant decay of the total nonlinear term ($|\lamb - \bnabla P|$ ) also persists in the later stages of relaxation (as it is clear for $t_{nl} = 32$ and $50 $ in Fig.\ref{fig:Fig3_a}). This indicates a plausible balance between $\lamb$ and $\bnabla P$ as observed in the turbulent relaxation of an ordinary HD fluid \cite{Kraichnan1988}. However, a scrupulous investigation indicates a slight discrepancy between the PDFs of $|\lamb|$ and $|\bnabla P|$ which will be addressed later. 

% Figure environment removed

For the interfacial region, a similar early-stage analysis reveals that the peak of the PDF of $|\lamb|$ shifts towards zero with time whereas no such tendency is observed for $|\lamb - \bnabla P|$ (Fig.~\ref{fig:Fig2_b}). Instead of the balance between $\lamb$ and $\bnabla P$, a clear signature of balance between $-\bxiQdivQ$ and $\bnabla P$ is found from the PDF of $|-\bxiQdivQ - \bnabla P|$. This is completely in accordance with the behaviour of the average values of absolute pressure gradient and $|\bxiQdivQ|$ in Fig.\ref{fig:Average_timeS}. 


In later times, this balance becomes more prominent, e.g., at $t_{nl} = 50$, PDFs of $|\bnabla P|$ and $|\bxiQdivQ|$ completely overlap thereby confirming strong equality between the two vectors at every  point on the interface (Fig.\ref{fig:Fig3_b}). The aforesaid equality thus indicates a pressure-balanced relaxed state on the interface. In addition, both $|\lamb|$ and $|- \bxiQdivQ - \bnabla P|$ die off during the relaxation and so does the total nonlinear term $|\lamb - \bxiQdivQ - \bnabla P|$. In fact, similar to the bulk, the total nonlinear term remains the fastest decaying quantity (light blue curves in Fig.\ref{fig:Fig3_a} and \ref{fig:Fig3_b}), thus providing a universal characterization of the turbulent relaxation of a binary mixture.
% Figure environment removed

% Figure environment removed


\section*{Linking bulk and interface relaxation: a universal theory}
Despite markedly distinct relaxation processes, the bulk and the interfacial relaxations are characterized by the suppression of a common decaying nonlinear quantity. In the following, we provide a theoretical justification for such universal relaxation using PVNLT\cite{Banerjee2023}. According to this principle, the relaxed states are obtained when the average scale-to-scale transfer of a cascading invariant identically vanishes at each scale within the inertial range\footnote{The range of length scales free from macroscopic and microscopic effects and is principally governed by the nonlinearity.}. 


Incompressible CHNS system allows two ideal invariants- (i) the total energy $E \ [=\int  (u^2 + \xi Q^2)/2 \ d\tau]$ and (ii) the scalar energy $S \ (= \int \phi^2/2\ d \tau)$. Inside the inertial range, the average scale-to-scale transfer of $E$ and $S$ can be written in terms of two-point ($\bx$ and $\bx'$) correlators  as\cite{Pan2022} 
\begin{linenomath}
\begin{align}
    \langle\mathcal{F}_{tr}^E\rangle
&= \left\langle \bu^{\prime} \cdot \left[  \bu \times \bomega - \xi \bQ (\bm{\nabla}\cdot\bQ) - \bm{\nabla} P \right]\right. \nonumber\\
&\left.+
 \bu\cdot [ \bu^{\prime} \times \bomega^{\prime} - \xi \bQ^{\prime}(\bm{\nabla}^{\prime}\cdot\bQ^\prime) -\bm{\nabla}^{\prime} P^{\prime}]\right.\nonumber\\  
 &\left.- \xi \bQ^{\prime}\cdot\bm{\nabla}(\bu\cdot\bQ)-\xi\bQ \cdot\bm{\nabla}^{\prime}(\bu^{\prime} \cdot \bQ^{\prime})\right\rangle,\\
 \langle\mathcal{F}_{tr}^S\rangle &=  \langle \phi^{\prime} (\bu\cdot\bQ) + \phi ( \bu^{\prime}\cdot\bQ^{\prime}) \rangle,\label{Ftrs_Sp}
\end{align}
\end{linenomath}
where $\langle\cdot\rangle$ denotes the ensemble average (equivalent to space average in homogeneous turbulence). Relaxation essentially implies the vanishing of $\langle\mathcal{F}_{tr}^E\rangle$ and $\langle\mathcal{F}_{tr}^S\rangle$ at each scale of the inertial range. For the case where $\bu$ may or may not be zero, the vanishing of $\langle\mathcal{F}_{tr}^E\rangle$ provides
\begin{linenomath}
\begin{align}
\bu \times \bomega - \xi \bQ (\bm{\nabla}\cdot\bQ)-  \bm{\nabla}P &=  \bm{\nabla} \Phi, \label{PVNLT_Ep} \\
\bm{\nabla}(\bu \cdot \bQ ) &= \bnabla\times\bm{A} \label{PVNLT_Ep1},
\end{align}
\end{linenomath}
where $\Phi$ and $\bm{A}$ are an arbitrary scalar and vector fields respectively.
Similarly, for vanishing $\langle\mathcal{F}_{tr}^S\rangle$ and non-vanishing $\phi$, one obtains 
\begin{linenomath}
\begin{equation}
\hspace{2.5cm}    \bu \cdot \bQ = 0 \label{PVNLT_Sp1}.
\end{equation}
\end{linenomath}
 
Inside the bulk $\bQ = \bm{0}$ and hence Eq.~\eqref{PVNLT_Sp1} is trivially satisfied similar to Eq.~\eqref{PVNLT_Ep1}, if $\bm{A}$ is chosen to be irrotational. On the interface, where none of $\bu$ and $\bQ$ vanishes, Eqs.~\eqref{PVNLT_Ep1} and \eqref{PVNLT_Sp1} require point-wise perpendicularity of $\bu$ and $\bQ$. However, the distribution of the angle between $\bu$ and $\bQ$ shows a broader spread instead of a sharp peak at $90^{\circ}$ (Fig.\ref{fig:uQ}). This apparent puzzle can be reconciled by means of a careful inspection of \eqref{Ftrs_Sp} which shows $\langle\mathcal{F}_{tr}^S\rangle = - \langle \phi^{\prime} \bu^{\prime} \cdot\bQ + \phi  \bu\cdot\bQ^{\prime} \rangle$ for homogeneous turbulence.
For $\langle\mathcal{F}_{tr}^S\rangle$ to vanish at all scales inside the inertial range, at least one of $\bu$ and $\bQ$ must vanish.
Relaxation of the interface therefore necessarily requires $\bu$ to vanish identically. 


Choosing $\bnabla \Phi = \bm{0}$ in Eq.~\eqref{PVNLT_Ep}, one recovers the universal relaxation of both the bulk and the interface as described in Fig.\ref{fig:Fig3}. In particular, 
\begin{linenomath}
\begin{align}
\hspace{-2cm}\text{for bulk ($\bQ =\bm{0}$)} &: \,\, \bu\times \bomega = \bnabla P  
    \quad \text{and}  \label{Bulkp} \\
\hspace{-0.5cm}    \text{for relaxed interface ($\bu =\bm{0} $)} &:\,\,
\xi \bQ (\bm{\nabla}\cdot\bQ) =  -\bm{\nabla}P. \label{PVNLT_Interface2}
\end{align}
\end{linenomath}
By taking the curl on both sides of Eq.~\eqref{PVNLT_Interface2} and using $\bnabla \times \bQ = \bm{0}$, we obtain the following Helmholtz-like condition
\begin{linenomath}
\begin{equation}
\bQ \times \bnabla (\bnabla \cdot \bQ) = \bm{0} \implies \nabla^2 \bQ = \lambda \bQ\implies \nabla^2\phi=\lambda\phi + C, \label{Helmholtz}
\end{equation}
\end{linenomath}
where $\lambda$ and $C$ are arbitrary constants. 
This relation is clearly verified in Fig.\ref{fig:Helmholtz_condition} whence the value of $\lambda$ is found to be nearly $-2815$. 


 % Figure environment removed
\section*{Bulk relaxation vs. single fluid relaxation} 
A close inspection of the Fig.\ref{fig:Fig3_a} (third panel) shows that
although the peak of the PDF of $|\lamb - \bnabla P|$ shifts rapidly towards zero during relaxation, a non-negligible gap is visible between the individual PDFs for $|\lamb|$ and $|\bnabla P|$  even at large times. The possibility of a non-trivial balance between the two can therefore be discarded. In addition, the alignment between the two is also found to reduce significantly during the relaxation (Fig.\ref{fig:Bulk_HD}). The only possibility that the Eq.~\eqref{Bulkp} holds, therefore, requires both sides of the equation to vanish identically at relaxation. This is in direct contrast with what is observed for an ordinary fluid, where the alignment between $\lamb$ and $\bnabla P$ increases in the course of relaxation (Fig.\ref{fig:Bulk_HD}). 
Interestingly, for the interface, a non-trivial balance between the feedback term and the total pressure gradient is meticulously obeyed as the PDFs of $|\bxiQdivQ|$ and $|\bnabla P|$ are found to overlap in Fig.\ref{fig:Fig3_b} (inset of the third panel). 

\section*{Discussion}
In this work, we have systematically investigated the turbulent relaxation of a binary fluid and prescribed a universal way to describe both the bulk and the interfacial relaxation. We have found a clear signature of a pressure-balanced relaxed state for the interface whereas the bulk is found to relax only when it attains a static state. This is a stark difference with respect to the turbulent relaxation of an ordinary hydrodynamic fluid where a balance between pressure gradient and the nonlinear term takes place before the fluid comes to rest. For a binary fluid system, this discrepancy can be attributed to the conservation of scalar energy which identically vanishes at every point of a single fluid. Despite this difference, the relaxation of both the bulk and the interface can be interpreted as a tendency of suppression of the total nonlinear term $\lamb - \bxiQdivQ - \bnabla P$.

In addition, the pressure-balanced relaxation on the interface can be reduced to a Helmholtz-like condition as it is seen in Eq.~\eqref{Helmholtz}. This kind 
of topographic relaxation is also observed in various other systems e.g. strongly rotating two-dimensional flows \cite{bretherton1976}, gravitational relaxation of mantle diapirs of Venus \cite{Janes1995}, formation of anti-cyclonic eddies over oceanic basins \cite{solodoch2021} and in particular, where the flow is governed by a scalar field e.g., the stream function in 2D HD and the magnitude of the magnetic vector potential in 2D MHD turbulence\cite{Banerjee2023}.

The present study entirely implements the principle of vanishing nonlinear transfer to find the relaxed states of a turbulent binary mixture under critical temperature. Note that, the principle of selective decay cannot give the correct relaxed states as obtained in Eqs.~\eqref{Bulkp} and \eqref{PVNLT_Interface2} as, by construction, it cannot capture a finite pressure gradient in the turbulent relaxed states. 

The current 
study can be extended to investigate turbulent relaxation of more complex binary fluid systems $e.g.,$ biological systems such as dilute bacterial suspensions\cite{tiribocchi2015}, phytoplankton suspensions\cite{peters2000} and synthetic active colloids\cite{campbell2019}, etc.,  which exerts extra feedback to the Navier-stokes equation owing to its activity \cite{Pan2022, Cates2018}. Broadly, this will open up future research possibilities of controlling and engineering different active systems from biologically and chemically relevant perspectives. 



 
%\bibliography{main}

\begin{thebibliography}{10}
\urlstyle{rm}
\expandafter\ifx\csname url\endcsname\relax
  \def\url#1{\texttt{#1}}\fi
\expandafter\ifx\csname urlprefix\endcsname\relax\def\urlprefix{URL }\fi
\expandafter\ifx\csname doiprefix\endcsname\relax\def\doiprefix{DOI: }\fi
\providecommand{\bibinfo}[2]{#2}
\providecommand{\eprint}[2][]{\url{#2}}

\bibitem{Stalidis1990}
\bibinfo{author}{Stalidis, G.}, \bibinfo{author}{Avranas, A.} \&
  \bibinfo{author}{Jannakoudakis, D.}
\newblock \bibinfo{journal}{\bibinfo{title}{Interfacial properties and
  stability of oil-in-water emulsions stabilized with binary mixtures of
  surfactants}}.
\newblock {\emph{\JournalTitle{Journal of Colloid and Interface Science}}}
  \textbf{\bibinfo{volume}{135}}, \bibinfo{pages}{313--324},
  \doiprefix\url{https://doi.org/10.1016/0021-9797(90)90002-6}
  (\bibinfo{year}{1990}).

\bibitem{Erucar2016}
\bibinfo{author}{Erucar, I.} \& \bibinfo{author}{Keskin, S.}
\newblock \bibinfo{journal}{\bibinfo{title}{Efficient storage of drug and
  cosmetic molecules in biocompatible metal organic frameworks: A molecular
  simulation study}}.
\newblock {\emph{\JournalTitle{Industrial \& Engineering Chemistry Research}}}
  \textbf{\bibinfo{volume}{55}}, \bibinfo{pages}{1929--1939},
  \doiprefix\url{10.1021/acs.iecr.5b04556} (\bibinfo{year}{2016}).
\newblock \eprint{https://doi.org/10.1021/acs.iecr.5b04556}.

\bibitem{tiribocchi2015}
\bibinfo{author}{Tiribocchi, A.}, \bibinfo{author}{Wittkowski, R.},
  \bibinfo{author}{Marenduzzo, D.} \& \bibinfo{author}{Cates, M.~E.}
\newblock \bibinfo{journal}{\bibinfo{title}{Active model h: scalar active
  matter in a momentum-conserving fluid}}.
\newblock {\emph{\JournalTitle{Physical review letters}}}
  \textbf{\bibinfo{volume}{115}}, \bibinfo{pages}{188302}
  (\bibinfo{year}{2015}).

\bibitem{Scarbolo2015}
\bibinfo{author}{Scarbolo, L.}, \bibinfo{author}{Bianco, F.} \&
  \bibinfo{author}{Soldati, A.}
\newblock \bibinfo{journal}{\bibinfo{title}{Coalescence and breakup of large
  droplets in turbulent channel flow}}.
\newblock {\emph{\JournalTitle{Physics of Fluids}}}
  \textbf{\bibinfo{volume}{27}}, \bibinfo{pages}{073302},
  \doiprefix\url{10.1063/1.4923424} (\bibinfo{year}{2015}).
\newblock \eprint{https://doi.org/10.1063/1.4923424}.

\bibitem{Cates2018}
\bibinfo{author}{Cates, M.~E.} \& \bibinfo{author}{Tjhung, E.}
\newblock \bibinfo{journal}{\bibinfo{title}{Theories of binary fluid mixtures:
  from phase-separation kinetics to active emulsions}}.
\newblock {\emph{\JournalTitle{Journal of Fluid Mechanics}}}
  \textbf{\bibinfo{volume}{836}}, \doiprefix\url{10.1017/jfm.2017.832}
  (\bibinfo{year}{2018}).

\bibitem{Pascual2021}
\bibinfo{author}{Pascual, M.}, \bibinfo{author}{Poquet, A.},
  \bibinfo{author}{Vilquin, A.} \& \bibinfo{author}{Jullien, M.-C.}
\newblock \bibinfo{journal}{\bibinfo{title}{Phase separation of an ionic liquid
  mixture assisted by a temperature gradient}}.
\newblock {\emph{\JournalTitle{Phys. Rev. Fluids}}}
  \textbf{\bibinfo{volume}{6}}, \bibinfo{pages}{024001},
  \doiprefix\url{10.1103/PhysRevFluids.6.024001} (\bibinfo{year}{2021}).

\bibitem{chan2021}
\bibinfo{author}{Chan, W. H.~R.}, \bibinfo{author}{Johnson, P.~L.},
  \bibinfo{author}{Moin, P.} \& \bibinfo{author}{Urzay, J.}
\newblock \bibinfo{journal}{\bibinfo{title}{The turbulent bubble break-up
  cascade. part 2. numerical simulations of breaking waves}}.
\newblock {\emph{\JournalTitle{Journal of Fluid Mechanics}}}
  \textbf{\bibinfo{volume}{912}}, \bibinfo{pages}{A43},
  \doiprefix\url{10.1017/jfm.2020.1084} (\bibinfo{year}{2021}).

\bibitem{mazzitelli2003turbulent}
\bibinfo{author}{Mazzitelli, I.}
\newblock \emph{\bibinfo{title}{Turbulent bubbly flow}}
  (\bibinfo{publisher}{Citeseer}, \bibinfo{year}{2003}).

\bibitem{serizawa1975}
\bibinfo{author}{Serizawa, A.}, \bibinfo{author}{Kataoka, I.} \&
  \bibinfo{author}{Michiyoshi, I.}
\newblock \bibinfo{journal}{\bibinfo{title}{Turbulence structure of air-water
  bubbly flow—ii. local properties}}.
\newblock {\emph{\JournalTitle{International Journal of Multiphase Flow}}}
  \textbf{\bibinfo{volume}{2}}, \bibinfo{pages}{235--246},
  \doiprefix\url{https://doi.org/10.1016/0301-9322(75)90012-9}
  (\bibinfo{year}{1975}).

\bibitem{Bailey1993}
\bibinfo{author}{Bailey, A.~E.} \& \bibinfo{author}{Cannell, D.~S.}
\newblock \bibinfo{journal}{\bibinfo{title}{Spinodal decomposition in a binary
  fluid}}.
\newblock {\emph{\JournalTitle{Phys. Rev. Lett.}}}
  \textbf{\bibinfo{volume}{70}}, \bibinfo{pages}{2110--2113},
  \doiprefix\url{10.1103/PhysRevLett.70.2110} (\bibinfo{year}{1993}).

\bibitem{narsimhan2019guidelines}
\bibinfo{author}{Narsimhan, G.}, \bibinfo{author}{Wang, Z.} \&
  \bibinfo{author}{Xiang, N.}
\newblock \bibinfo{journal}{\bibinfo{title}{Guidelines for processing
  emulsion-based foods}}.
\newblock {\emph{\JournalTitle{Food emulsifiers and their applications}}}
  \bibinfo{pages}{435--501} (\bibinfo{year}{2019}).

\bibitem{Shenoy2015}
\bibinfo{author}{Shenoy, P.} \emph{et~al.}
\newblock \bibinfo{journal}{\bibinfo{title}{Dry mixing of food powders: Effect
  of water content and composition on mixture quality of binary mixtures}}.
\newblock {\emph{\JournalTitle{Journal of Food Engineering}}}
  \textbf{\bibinfo{volume}{149}}, \bibinfo{pages}{229--236},
  \doiprefix\url{https://doi.org/10.1016/j.jfoodeng.2014.10.019}
  (\bibinfo{year}{2015}).

\bibitem{Bray1995}
\bibinfo{author}{Bray, A.~J.}
\newblock \bibinfo{journal}{\bibinfo{title}{Theory of phase-ordering
  kinetics}}.
\newblock {\emph{\JournalTitle{Advances in Physics}}}
  \textbf{\bibinfo{volume}{51}}, \bibinfo{pages}{481--587},
  \doiprefix\url{10.1080/00018730110117433} (\bibinfo{year}{2002}).

\bibitem{hohenberg1977}
\bibinfo{author}{Hohenberg, P.~C.} \& \bibinfo{author}{Halperin, B.~I.}
\newblock \bibinfo{journal}{\bibinfo{title}{Theory of dynamic critical
  phenomena}}.
\newblock {\emph{\JournalTitle{Rev. Mod. Phys.}}}
  \textbf{\bibinfo{volume}{49}}, \bibinfo{pages}{435--479},
  \doiprefix\url{10.1103/RevModPhys.49.435} (\bibinfo{year}{1977}).

\bibitem{chaikin1995}
\bibinfo{author}{Chaikin, P.} \& \bibinfo{author}{Lubensky, T.}
\newblock \bibinfo{journal}{\bibinfo{title}{Cambridge university press; new
  york}}.
\newblock {\emph{\JournalTitle{Principles of Condensed Matter Physics.[Google
  Scholar]}}}  (\bibinfo{year}{1995}).

\bibitem{Koga1993}
\bibinfo{author}{Koga, T.} \& \bibinfo{author}{Kawasaki, K.}
\newblock \bibinfo{journal}{\bibinfo{title}{Late stage dynamics of spinodal
  decomposition in binary fluid mixtures}}.
\newblock {\emph{\JournalTitle{Physica A: Statistical Mechanics and its
  Applications}}} \textbf{\bibinfo{volume}{196}}, \bibinfo{pages}{389--415},
  \doiprefix\url{https://doi.org/10.1016/0378-4371(93)90204-H}
  (\bibinfo{year}{1993}).

\bibitem{berti2005}
\bibinfo{author}{Berti, S.}, \bibinfo{author}{Boffetta, G.},
  \bibinfo{author}{Cencini, M.} \& \bibinfo{author}{Vulpiani, A.}
\newblock \bibinfo{journal}{\bibinfo{title}{Turbulence and coarsening in active
  and passive binary mixtures}}.
\newblock {\emph{\JournalTitle{Physical review letters}}}
  \textbf{\bibinfo{volume}{95}}, \bibinfo{pages}{224501},
  \doiprefix\url{10.1103/PhysRevLett.95.224501} (\bibinfo{year}{2005}).

\bibitem{perlekar2014}
\bibinfo{author}{Perlekar, P.}, \bibinfo{author}{Benzi, R.},
  \bibinfo{author}{Clercx, H.~J.}, \bibinfo{author}{Nelson, D.~R.} \&
  \bibinfo{author}{Toschi, F.}
\newblock \bibinfo{journal}{\bibinfo{title}{Spinodal decomposition in
  homogeneous and isotropic turbulence}}.
\newblock {\emph{\JournalTitle{Physical review letters}}}
  \textbf{\bibinfo{volume}{112}}, \bibinfo{pages}{014502}
  (\bibinfo{year}{2014}).

\bibitem{perlekar2019}
\bibinfo{author}{Perlekar, P.}
\newblock \bibinfo{journal}{\bibinfo{title}{Kinetic energy spectra and flux in
  turbulent phase-separating symmetric binary-fluid mixtures}}.
\newblock {\emph{\JournalTitle{Journal of Fluid Mechanics}}}
  \textbf{\bibinfo{volume}{873}}, \bibinfo{pages}{459--474},
  \doiprefix\url{10.1017/jfm.2019.425} (\bibinfo{year}{2019}).

\bibitem{Pan2022}
\bibinfo{author}{Pan, N.} \& \bibinfo{author}{Banerjee, S.}
\newblock \bibinfo{journal}{\bibinfo{title}{Exact relations for energy transfer
  in simple and active binary fluid turbulence}}.
\newblock {\emph{\JournalTitle{Phys. Rev. E}}} \textbf{\bibinfo{volume}{106}},
  \bibinfo{pages}{025104}, \doiprefix\url{10.1103/PhysRevE.106.025104}
  (\bibinfo{year}{2022}).

\bibitem{Mukherjee2019}
\bibinfo{author}{Mukherjee, S.}, \bibinfo{author}{Safdari, A.},
  \bibinfo{author}{Shardt, O.}, \bibinfo{author}{Kenjereš, S.} \&
  \bibinfo{author}{Van~den Akker, H. E.~A.}
\newblock \bibinfo{journal}{\bibinfo{title}{Droplet–turbulence interactions
  and quasi-equilibrium dynamics in turbulent emulsions}}.
\newblock {\emph{\JournalTitle{Journal of Fluid Mechanics}}}
  \textbf{\bibinfo{volume}{878}}, \bibinfo{pages}{221–276},
  \doiprefix\url{10.1017/jfm.2019.654} (\bibinfo{year}{2019}).

\bibitem{Chandrasekhar1958}
\bibinfo{author}{Chandrasekhar, S.} \& \bibinfo{author}{Woltjer, L.}
\newblock \bibinfo{journal}{\bibinfo{title}{On force-free magnetic fields}}.
\newblock {\emph{\JournalTitle{Proceedings of the National Academy of
  Sciences}}} \textbf{\bibinfo{volume}{44}}, \bibinfo{pages}{285--289},
  \doiprefix\url{10.1073/pnas.44.4.285} (\bibinfo{year}{1958}).
\newblock \eprint{https://www.pnas.org/doi/pdf/10.1073/pnas.44.4.285}.

\bibitem{Woltjer1958a}
\bibinfo{author}{Woltjer, L.}
\newblock \bibinfo{journal}{\bibinfo{title}{A theorem on force-free magnetic
  fields}}.
\newblock {\emph{\JournalTitle{Proceedings of the National Academy of
  Sciences}}} \textbf{\bibinfo{volume}{44}}, \bibinfo{pages}{489--491},
  \doiprefix\url{10.1073/pnas.44.6.489} (\bibinfo{year}{1958}).

\bibitem{Taylor1974}
\bibinfo{author}{Taylor, J.~B.}
\newblock \bibinfo{journal}{\bibinfo{title}{Relaxation of toroidal plasma and
  generation of reverse magnetic fields}}.
\newblock {\emph{\JournalTitle{Phys. Rev. Lett.}}}
  \textbf{\bibinfo{volume}{33}}, \bibinfo{pages}{1139--1141},
  \doiprefix\url{10.1103/PhysRevLett.33.1139} (\bibinfo{year}{1974}).

\bibitem{Zhu1995}
\bibinfo{author}{Zhu, S.~P.}, \bibinfo{author}{Horiuchi, R.},
  \bibinfo{author}{Sato, T.} \& \bibinfo{author}{ComplexitySimulationGroup}.
\newblock \bibinfo{journal}{\bibinfo{title}{Non-taylor magnetohydrodynamic
  self-organization}}.
\newblock {\emph{\JournalTitle{Phys. Rev. E}}} \textbf{\bibinfo{volume}{51}},
  \bibinfo{pages}{6047--6054} (\bibinfo{year}{1995}).

\bibitem{Sato1996}
\bibinfo{author}{Sato, T.}
\newblock \bibinfo{journal}{\bibinfo{title}{Complexity in plasma: From
  self‐organization to geodynamo}}.
\newblock {\emph{\JournalTitle{Physics of Plasmas}}}
  \textbf{\bibinfo{volume}{3}}, \bibinfo{pages}{2135--2142},
  \doiprefix\url{10.1063/1.871666} (\bibinfo{year}{1996}).

\bibitem{Kraichnan1988}
\bibinfo{author}{Kraichnan, R.~H.} \& \bibinfo{author}{Panda, R.}
\newblock \bibinfo{journal}{\bibinfo{title}{Depression of nonlinearity in
  decaying isotropic turbulence}}.
\newblock {\emph{\JournalTitle{The Physics of Fluids}}}
  \textbf{\bibinfo{volume}{31}}, \bibinfo{pages}{2395--2397},
  \doiprefix\url{10.1063/1.866591} (\bibinfo{year}{1988}).

\bibitem{Banerjee2023}
\bibinfo{author}{Banerjee, S.}, \bibinfo{author}{Halder, A.} \&
  \bibinfo{author}{Pan, N.}
\newblock \bibinfo{journal}{\bibinfo{title}{Universal turbulent relaxation of
  fluids and plasmas by the principle of vanishing nonlinear transfers}}.
\newblock {\emph{\JournalTitle{Phys. Rev. E}}} \textbf{\bibinfo{volume}{107}},
  \bibinfo{pages}{L043201}, \doiprefix\url{10.1103/PhysRevE.107.L043201}
  (\bibinfo{year}{2023}).

\bibitem{Baba1969}
\bibinfo{author}{Baba, K.}, \bibinfo{author}{Fujimura, T.} \&
  \bibinfo{author}{Kamiyoshi, K.}
\newblock \bibinfo{journal}{\bibinfo{title}{Dielectric relaxation of binary
  liquid mixtures}}.
\newblock {\emph{\JournalTitle{The Journal of Physical Chemistry}}}
  \textbf{\bibinfo{volume}{73}}, \bibinfo{pages}{1146--1147},
  \doiprefix\url{10.1021/j100724a066} (\bibinfo{year}{1969}).
\newblock \eprint{https://doi.org/10.1021/j100724a066}.

\bibitem{Benzi2011}
\bibinfo{author}{Benzi, R.}, \bibinfo{author}{Sbragaglia, M.},
  \bibinfo{author}{Bernaschi, M.} \& \bibinfo{author}{Succi, S.}
\newblock \bibinfo{journal}{\bibinfo{title}{Phase-field model of long-time
  glasslike relaxation in binary fluid mixtures}}.
\newblock {\emph{\JournalTitle{Phys. Rev. Lett.}}}
  \textbf{\bibinfo{volume}{106}}, \bibinfo{pages}{164501},
  \doiprefix\url{10.1103/PhysRevLett.106.164501} (\bibinfo{year}{2011}).

\bibitem{Harbola2003}
\bibinfo{author}{Harbola, U.} \& \bibinfo{author}{Das, S.~P.}
\newblock \bibinfo{journal}{\bibinfo{title}{Secondary relaxation in a
  supercooled binary mixture}}.
\newblock {\emph{\JournalTitle{International Journal of Modern Physics B}}}
  \textbf{\bibinfo{volume}{17}}, \bibinfo{pages}{2395--2415},
  \doiprefix\url{10.1142/S0217979203018260} (\bibinfo{year}{2003}).
\newblock \eprint{https://doi.org/10.1142/S0217979203018260}.

\bibitem{Araki2022}
\bibinfo{author}{Araki, T.}, \bibinfo{author}{Gomez-Solano, J.~R.} \&
  \bibinfo{author}{Macio\l{}ek, A.}
\newblock \bibinfo{journal}{\bibinfo{title}{Relaxation to steady states of a
  binary liquid mixture around an optically heated colloid}}.
\newblock {\emph{\JournalTitle{Phys. Rev. E}}} \textbf{\bibinfo{volume}{105}},
  \bibinfo{pages}{014123}, \doiprefix\url{10.1103/PhysRevE.105.014123}
  (\bibinfo{year}{2022}).

\bibitem{Yaglom1949}
\bibinfo{author}{Yaglom, A.~M.}
\newblock \bibinfo{title}{On the local structures of temperature field in a
  turbulent flow}.
\newblock In \emph{\bibinfo{booktitle}{Dokl. Akad. Nauk SSSR A}},
  vol.~\bibinfo{volume}{69}, \bibinfo{pages}{743--46} (\bibinfo{year}{1949}).

\bibitem{Stribling1991}
\bibinfo{author}{Stribling, T.} \& \bibinfo{author}{Matthaeus, W.~H.}
\newblock \bibinfo{journal}{\bibinfo{title}{Relaxation processes in a
  low‐order three‐dimensional magnetohydrodynamics model}}.
\newblock {\emph{\JournalTitle{Physics of Fluids B: Plasma Physics}}}
  \textbf{\bibinfo{volume}{3}}, \bibinfo{pages}{1848--1864},
  \doiprefix\url{10.1063/1.859654} (\bibinfo{year}{1991}).
\newblock \eprint{https://doi.org/10.1063/1.859654}.

\bibitem{Servidio2008}
\bibinfo{author}{Servidio, S.}, \bibinfo{author}{Matthaeus, W.~H.} \&
  \bibinfo{author}{Dmitruk, P.}
\newblock \bibinfo{journal}{\bibinfo{title}{Depression of nonlinearity in
  decaying isotropic mhd turbulence}}.
\newblock {\emph{\JournalTitle{Phys. Rev. Lett.}}}
  \textbf{\bibinfo{volume}{100}}, \bibinfo{pages}{095005},
  \doiprefix\url{10.1103/PhysRevLett.100.095005} (\bibinfo{year}{2008}).

\bibitem{Matthaeus2008}
\bibinfo{author}{Matthaeus, W.~H.}, \bibinfo{author}{Pouquet, A.},
  \bibinfo{author}{Mininni, P.~D.}, \bibinfo{author}{Dmitruk, P.} \&
  \bibinfo{author}{Breech, B.}
\newblock \bibinfo{journal}{\bibinfo{title}{Rapid alignment of velocity and
  magnetic field in magnetohydrodynamic turbulence}}.
\newblock {\emph{\JournalTitle{Phys. Rev. Lett.}}}
  \textbf{\bibinfo{volume}{100}}, \bibinfo{pages}{085003},
  \doiprefix\url{10.1103/PhysRevLett.100.085003} (\bibinfo{year}{2008}).

\bibitem{bretherton1976}
\bibinfo{author}{Bretherton, F.~P.} \& \bibinfo{author}{Haidvogel, D.~B.}
\newblock \bibinfo{journal}{\bibinfo{title}{Two-dimensional turbulence above
  topography}}.
\newblock {\emph{\JournalTitle{Journal of Fluid Mechanics}}}
  \textbf{\bibinfo{volume}{78}}, \bibinfo{pages}{129–154},
  \doiprefix\url{10.1017/S002211207600236X} (\bibinfo{year}{1976}).

\bibitem{Janes1995}
\bibinfo{author}{Janes, D.~M.} \& \bibinfo{author}{Squyres, S.~W.}
\newblock \bibinfo{journal}{\bibinfo{title}{Viscoelastic relaxation of
  topographic highs on venus to produce coronae}}.
\newblock {\emph{\JournalTitle{Journal of Geophysical Research: Planets}}}
  \textbf{\bibinfo{volume}{100}}, \bibinfo{pages}{21173--21187},
  \doiprefix\url{https://doi.org/10.1029/95JE01748} (\bibinfo{year}{1995}).
\newblock
  \eprint{https://agupubs.onlinelibrary.wiley.com/doi/pdf/10.1029/95JE01748}.

\bibitem{solodoch2021}
\bibinfo{author}{Solodoch, A.}, \bibinfo{author}{Stewart, A.~L.} \&
  \bibinfo{author}{McWilliams, J.~C.}
\newblock \bibinfo{journal}{\bibinfo{title}{Formation of anticyclones above
  topographic depressions}}.
\newblock {\emph{\JournalTitle{Journal of Physical Oceanography}}}
  \textbf{\bibinfo{volume}{51}}, \bibinfo{pages}{207--228}
  (\bibinfo{year}{2021}).

\bibitem{peters2000}
\bibinfo{author}{Peters, F.} \& \bibinfo{author}{Marras{\'e}, C.}
\newblock \bibinfo{journal}{\bibinfo{title}{Effects of turbulence on plankton:
  an overview of experimental evidence and some theoretical considerations}}.
\newblock {\emph{\JournalTitle{Marine Ecology Progress Series}}}
  \textbf{\bibinfo{volume}{205}}, \bibinfo{pages}{291--306}
  (\bibinfo{year}{2000}).

\bibitem{campbell2019}
\bibinfo{author}{Campbell, A.~I.}, \bibinfo{author}{Ebbens, S.~J.},
  \bibinfo{author}{Illien, P.} \& \bibinfo{author}{Golestanian, R.}
\newblock \bibinfo{journal}{\bibinfo{title}{Experimental observation of flow
  fields around active janus spheres}}.
\newblock {\emph{\JournalTitle{Nature communications}}}
  \textbf{\bibinfo{volume}{10}}, \bibinfo{pages}{3952} (\bibinfo{year}{2019}).

\bibitem{canuto2007spectral}
\bibinfo{author}{Canuto, C.}, \bibinfo{author}{Hussaini, M.},
  \bibinfo{author}{Quarteroni, A.} \& \bibinfo{author}{Zang, T.}
\newblock \emph{\bibinfo{title}{Spectral Methods: Evolution to Complex
  Geometries and Applications to Fluid Dynamics}}.
\newblock Scientific Computation (\bibinfo{publisher}{Springer Berlin
  Heidelberg}, \bibinfo{year}{2007}).

\bibitem{Orszag_1971}
\bibinfo{author}{Orszag, S.~A.}
\newblock \bibinfo{journal}{\bibinfo{title}{On the elimination of aliasing in
  finite-difference schemes by filtering high-wavenumber components}}.
\newblock {\emph{\JournalTitle{Journal of Atmospheric Sciences}}}
  \textbf{\bibinfo{volume}{28}}, \bibinfo{pages}{1074 -- 1074},
  \doiprefix\url{10.1175/1520-0469(1971)028<1074:OTEOAI>2.0.CO;2}
  (\bibinfo{year}{1971}).

\bibitem{Mortensen2016HighPerformance}
\bibinfo{author}{Mortensen, M.} \& \bibinfo{author}{Langtangen, H.~P.}
\newblock \bibinfo{journal}{\bibinfo{title}{High performance python for direct
  numerical simulations of turbulent flows}}.
\newblock {\emph{\JournalTitle{Computer Physics Communications}}}
  \textbf{\bibinfo{volume}{203}}, \bibinfo{pages}{53--65},
  \doiprefix\url{https://doi.org/10.1016/j.cpc.2016.02.005}
  (\bibinfo{year}{2016}).

\bibitem{Hinze1955}
\bibinfo{author}{Hinze, J.~O.}
\newblock \bibinfo{journal}{\bibinfo{title}{Fundamentals of the hydrodynamic
  mechanism of splitting in dispersion processes}}.
\newblock {\emph{\JournalTitle{AIChE Journal}}} \textbf{\bibinfo{volume}{1}},
  \bibinfo{pages}{289--295},
  \doiprefix\url{https://doi.org/10.1002/aic.690010303} (\bibinfo{year}{1955}).
\newblock
  \eprint{https://aiche.onlinelibrary.wiley.com/doi/pdf/10.1002/aic.690010303}.

\bibitem{Eyre1998}
\bibinfo{author}{Eyre, D.~J.}
\newblock \bibinfo{journal}{\bibinfo{title}{Unconditionally gradient stable
  time marching the cahn-hilliard equation}}.
\newblock {\emph{\JournalTitle{MRS Online Proceedings Library}}}
  \textbf{\bibinfo{volume}{529}}, \bibinfo{pages}{39},
  \doiprefix\url{10.1557/PROC-529-39} (\bibinfo{year}{1998}).

\bibitem{Yoon2020}
\bibinfo{author}{Yoon, S.} \emph{et~al.}
\newblock \bibinfo{journal}{\bibinfo{title}{Fourier-spectral method for the
  phase-field equations}}.
\newblock {\emph{\JournalTitle{Mathematics}}} \textbf{\bibinfo{volume}{8}},
  \doiprefix\url{10.3390/math8081385} (\bibinfo{year}{2020}).

\bibitem{Li2021}
\bibinfo{author}{Li, M.} \& \bibinfo{author}{Xu, C.}
\newblock \bibinfo{journal}{\bibinfo{title}{New efficient time-stepping schemes
  for the navier–stokes–cahn–hilliard equations}}.
\newblock {\emph{\JournalTitle{Computers and Fluids}}}
  \textbf{\bibinfo{volume}{231}}, \bibinfo{pages}{105174},
  \doiprefix\url{https://doi.org/10.1016/j.compfluid.2021.105174}
  (\bibinfo{year}{2021}).

\end{thebibliography}
\section*{Methods}

\subsection*{Simulation details}
Eqs.~\eqref{Eu} and \eqref{EQ} are simulated using pseudo-spectral method in three dimensions in a 2$\pi$-periodic box with $N$ grid points in each direction. Due to the presence of a cubic non-linearity in chemical potential, a $N/2$-dealiasing method is employed \cite{canuto2007spectral,Orszag_1971}.
\begin{table}[ht]\label{Table1}
\centering
\begin{tabular}{|l|l|l|l|l|l|}
\hline
 & $N$ & $\nu (\times 10 ^{-3})$ & $\xi (\times 10 ^{-3})$ & $\mathcal{M}$ & $Re$ \\
\hline
NS & $512$  & $0.6$ & - & - & 3000\\
\hline
CHNS1 & $256$  & $2$ & $1.0$ & 0.01 & 453\\
\hline
CHNS2 & $512$ & $0.6$ & $0.3$  & 0.01 & 1225 \\
\hline
\end{tabular}
\caption{\label{tab} NS and CHNS stand for single-fluid Navier-Stokes and Cahn-Hiliard-Navier-Stokes simulations respectively.}
\end{table}
The system is initialized below $T_c$ from rest ($\bu = \bm{0})$ in a phase-mixed state with a uniform distribution of $\phi$ with $-0.05\leq\phi(\bm{x},0)\leq 0.05$. The simulation code is parallelized using a python-based message-passing interface (MPI) scheme\cite{Mortensen2016HighPerformance}.

In the absence of any forcing, the two fluids will eventually phase-separate in regions with $\phi = \pm 1$ (determined from the free energy functional). 
% Figure environment removed
The arrest of phase separation is achieved through turbulence by stirring the momentum evolution equation \eqref{Eu} with a large-scale non-helical forcing $\bm{f}$,
\begin{linenomath}
\begin{align}
 \bm{f}= &f_0 sin(k_f x)cos(k_f y)cos(k_f z)\hat{x}\nonumber \\ 
 &- f_0 cos(k_f x )sin(k_f y)cos(k_f z)\hat{y},\nonumber 
\end{align}
\end{linenomath}
where $k_f$ and $f_0$ are the forcing  wavenumber and amplitude respectively. For the current study, we choose $k_f = 2$, $f_0 = 0.5$. The typical domain size of such a turbulent phase-arrested binary fluid system is determined by the Hinze-criterion \cite{Hinze1955}. The system is evolved in time with an unconditionally stable  explicit-implicit scheme \cite{Eyre1998, Yoon2020, Li2021}, until it reaches a statistical steady state  (Fig.\ref{fig:diss}). The parameter $\xi $ is proportional to the square of the width of the two-fluid interface. Ideally, the interface is very sharp $i.e.,$ the width of the interface (the length over which $\phi$ changes from 0.9 to -0.9) is negligibly small. However, to have a numerically well-resolved interface, it is chosen to cover at least six grid points \cite{berti2005,perlekar2019}. Other parameters $\nu$ and $\mathcal{M}$ are set in order to have a reasonably high Reynolds number ($R_e$) turbulence (Table \ref{tab}). 


As an initial step, we performed the simulation with $256^3$ grid points with appropriate parameters given in Table \ref{tab}. Main findings of our study is found to remain unchanged with respect to the analysis carried out with $512^3$ grid points (mentioned in the main text). For the ordinary HD fluid simulation, we employ the same non-helical forcing $\bm{f}$ and a standard fourth order Runge-Kutta scheme for time-integration.
\subsection*{Characterization of bulk and interface}
The whole binary fluid has been divided into into bulk and interfacial regions. Ideally, $\bm{Q}$ should be zero within bulk as it consists only of single fluid phase of the two-fluid mixture. However, due to numerical limitations and for computational convenience, we choose $|\bm{Q}|\leq 1.2$ for bulk region. Similarly, the interface is a sharp-transition region between the two-fluid ($|\bm{Q}|>>1)$), however for our study the interface is defined for $|\bm{Q}|\geq 21$ (Fig.\ref{fig:Bulk_interface_19}). Although $\bQ$ is not identically zero inside bulk, it does not effect the analysis as the value of $-\bxiQdivQ$ is found to be negligibly small (Fig.\ref{fig:Average_timeS}).  
\subsection*{Probability density functions (PDFs)}
Below the critical temperature, binary fluid always minimizes the interfacial regions in order to minimize the free energy $\mathcal{F}$. During relaxation, this leads to the increase and decrease of bulk and interface points respectively.
Note that, here a standard definition of PDF has been employed where the number of counts in each bin has been divided by the total count and the corresponding bin size. The simulations have been carried out upto $t_{nl} = 60$ (Fig.\ref{fig:phi_color}) and the tendency of the universal relaxation remains unchanged as seen from the Fig.\ref{fig:Bulk_inter_timS} below. However, for the sake of visual clarity, the PDFs in the main text are plotted for three different nonlinear times upto $t_{nl} = 50$. 
% Figure environment removed

\section*{Acknowledgements}
The authors acknowledge useful discussions with Anando G. Chatterjee. The simulation code is developed by the first author following the parallelization scheme provided in ref \cite{Mortensen2016HighPerformance}. The simulations are performed using the support and resources provided by PARAM Sanganak under the National Supercomputing Mission, Government of India at the Indian Institute of Technology, Kanpur. S.B. acknowledges DST-INSPIRE faculty Research Grant No. DST/PHY/2017514 and CEFIPRA Research Grant No. 6104-01. 
\section*{Author contributions}
S.B. and N.P. conceived the idea and designed the study. N.P. performed the numerical simulation. S.B., N.P. and A.H. analysed and interpreted the results and contributed to the composition of the manuscript.

% \section*{Supplementary information}
% \subsection{Selective decay theory}
% Selective decay theory (SDT) has been traditionally invoked in order to explain the relaxed states of turbulent fluids and plasmas. Relaxed states from SDT are obtained by varying the fast decaying invariants keeping other slowly decaying invariants constant. We start by comparing decay rates of inviscid invariants mentioned before-
% \begin{align}
%     d_t E &\equiv \int \left[-\nu \omega^2 + \xi \mathcal{M} \bQ \cdot \nabla^2 \bm{\nabla} \mu \right] d \tau, \label{ddE} \\
%     d_t S &\equiv \int \left[ \mathcal{M} \phi \nabla^2 \mu \right]d \tau. \label{ddS}
%  \end{align}
% From Eqs.~\eqref{ddE} and \eqref{ddS}, $E$ is expected to decay faster than  $S$ as it contains a positive definite . Therefore, we vary
% \begin{equation}
% \hspace{2cm} E-\lambda_1 S \label{variation}    
% \end{equation}
% with respect to $\bu$ and $\phi$ respectively. Variation of Eq.~\eqref{variation} with respect to $\bu$ gives $\bm{u} = 0$ and with $\phi$ one obtains 
% \begin{align}
%    \int \left[ 2\xi \bQ \cdot \bnabla \delta \phi - 2\lambda_{1}\delta \phi \phi \right] d \tau  &= 0, \nonumber\\  
%    \int \left[ \xi \bnabla \cdot (\bQ \delta \phi) - \xi (\bnabla \cdot \bQ )\delta\phi - \lambda_{1} \phi \delta\phi \right]  d \tau  &= 0.
% \end{align}
% Again, applying the Gauss-divergence theorem with periodic or vanishing boundary conditions and for arbitrary $\delta \phi$, the following relaxed state is obtained
% \begin{equation}
% \hspace{2cm}\nabla^2 \phi = - \frac{\lambda_1}{\xi}\phi.
% \end{equation}

% \subsection{Topographical relaxation of 2D HD and MHD fluids}
% The following double-curl relaxed states are found in HD 
% \begin{equation}
%     \bnabla \times (\bnabla \times \bu)  = -\lambda_u\bu,  \label{2DHD_topography}
% \end{equation}
% and MHD
% \begin{equation}
%     \bnabla \times (\bnabla \times \bm{b}) =  \lambda_b \bm{b},\label{2DMHD_topography}
% \end{equation}
% where $\bu$ and $\bm{b}$ are velocity and magnetic field, $\lambda_u$ and $\lambda_b$ are arbitrary constants \cite{Banerjee2023}. Owing to the divergence free property of the fields, the above states can further be reduced to  
% \begin{align}
%     \nabla^2 \bu =- \lambda_u  \bu \implies \nabla^2 \psi =- \lambda_u  \psi,
%     \label{2DHD_topography1}\\
%    \nabla^2 \bm{b} = -\lambda_b \bm{b} \implies \nabla^2 a = -\lambda_b a, \label{2DMHD_topography1}
% \end{align}
% where $\psi$ and $a$ are stream function and vector potential for the velocity and magnetic field respectively. 


\end{document}

xxxxxxxxxxxxxxxxxxxxxxxxxxxxxxxxxxxxxxxxxxxxxxxxxxxxxxxxxxxxxxxxxxxxxxxxxxxxxxxxxxxxxxxxxxxxxxxxxxxxxxxxxxxxxxxxxxxxxxxxxxxxxxxxxxxxxxxxxxxxxxxxxxxxxxxxxxxxxxxxxxxxxxxxxxxxxxxxxxxxxxxxxxxxxxxxxxxxxxxxxxxxxxxxxxxxxxxxxxxxxxxxxxxxxxxxxxxxxxxxxxxxxxxxxxxxxxxxxxxxxxxxxxxxxxxxxxxxxxxxxxxxxxxxxxxxxxxxxxxxxxxxxxxxxxxxxxxxxxxxxxxxxxxxxxxxxxxxxxxxxxxxxxxxxxxxxxxxxxxxxxxxxxxxxxxxxxxxxxxxxxxxxxxxxxxxx

\section*{An atypical relaxation of the bulk}
Before reaching a quiescent state with zero flow, as predicted by PVNLT, $\lamb = \bnabla P$ is expected to hold at each point in the bulk during relaxation. Accordingly, their individual PDFs should have significant overlap at large times. 
In Fig.~\ref{fig:Fig3_a} (third panel), we see that
although the peak of the PDF of $|\lamb - \bnabla P|$ shifts rapidly towards zero at later times along with the increase in the peak count, the individual PDFs for $|\lamb|$ and $|\bnabla P|$ differ significantly even at $t_{nl} = 50$. The pressure-balanced state hinted earlier from the average behaviour (Fig.~\ref{fig:Average_timeS}) and from the early-stage dynamics (Fig.~\ref{fig:Fig2_a}) therefore does not hold true. This observation is further corroborated from the PDF of cosine of the angle between the two (see Fig.~\ref{fig:Angle_Bulk_interface_uCw_gradP}b). The decrease of the peak value from $+1$, after the quenching at $t_{nl}=16$, clearly signifies a lack of equality between $\lamb$ and $\bnabla P$. Evidently, both the PDFs (Fig.~\ref{fig:Fig3}) and the angle plot (Fig.~\ref{fig:Angle_Bulk_interface_uCw_gradP}b) support the fact that viscous dissipation rather than the equality of $\lamb$ and $\bnabla P$ is responsible for the observed shift of $|\lamb - \bnabla P|$ towards zero. Further to check whether an $\bu$-$\bomega$ alignment holds, we plot the corresponding angle in Fig.~\ref{fig:Angle_Bulk_interface_uCw_gradP}a. The peak around $\pm 1$ does not deviate appreciably with time, thereby negating the possibility of the aforesaid alignment. This fact also confirms that the decay of $\lamb$ at large times is solely due to the viscous dissipation instead of an alignment of the two vectors. In conclusion, the condition \eqref{Bulkp} can only be satisfied for $\bu =\bm{0}$.

\section*{Pressure-balanced relaxation of the interface}
Similar to the bulk, the dissipative decay of $\lamb$ persists in the interfacial region. A clear deviation from an $\bu$-$\bomega$ alignment is observed during the relaxation process as the angle between $\bu$ and $\bomega$ peaks around zero rather than $\pm 1$ (see Fig.~\ref{fig:Angle_Bulk_interface_uCw_gradP}c). Thus as argued in the earlier section by PVNLT, vanishing of $\langle\mathcal{F}^{tr}_S\rangle$ at all scales is only possible for $\bu = \bm{0}$, justifying a quiescent state for both bulk and interface. Additionally, a strong support of condition \eqref{PVNLT_Interface2} is found in Fig.~\ref{fig:Fig3_b}. The peaks of the PDFs of $|\bxiQdivQ|$ and $|\bnabla P|$ almost overlaps, suggesting a pointwise alignment of the two. This fact is further validated in the PDF of the cosine of the angle between the feedback term and pressure gradient (Fig.~\ref{fig:Angle_Bulk_interface_uCw_gradP}d). As relaxation proceeds, the peak near $+1$ sharply increases, signifying an equality of the two vectors. To summarize, both PVNLT and numerical findings ascertain that the relaxed state of a turbulent binary fluid is a state with zero flow supporting a finite pressure-gradient on the two-fluid interfaces.

 



% % Figure environment removed


\bibliography{main}


\iffalse
% Figure environment removed
% Figure environment removed
% Figure environment removed
\fi

Whether such a steady state would be a Beltrami-Taylor type force-free state or a Grad-Shafranov type pressure-balanced state, is a matter of long-standing debate. A force-free aligned state has been observed in cosmic plasmas whereas a clear signature of pressure-balanced state has been found in incompressible fluids and plasmas with moderate plasma-$\beta$. Although, the aligned states can easily be explained by the principle of selective decay, an unambiguous theoretical explanation for the relaxed states with finite pressure gradient is still lacking. 
Besides a primitive variational approach, some systematic attempts including the principle of minimum entropy production rate, the principle of minimum dissipation rate etc. are made
to account for the non-vanishing pressure gradient \cite{Grad1958, Hamieri1987, Dasgupta1998}. However, the aforementioned methods suffer from several limitations which have been taken care of in a recently proposed method of universal relaxation \cite{Banerjee2022}.  

\textcolor{red}{Relaxed state of a two-component fluid mixture has been widely studied from the perspective of equilibrium statistical physics. However, recent advancements in the effect of turbulence in a binary fluid mixture have illuminated the possibility of studying such phenomena from the viewpoint of non-equilibrium statistical physics. In case of neutral fluids, several studies have shown that the relaxed state is a balance between the Lamb vector and pressure gradient \textit{i.e.}, $\lamb=\bnabla p$, whereas for plasmas the balance is obtained as $\bj\times\bb=\bnabla p$. Whether such universal pressure-balanced states can be obtained in a turbulent binary fluid undergoing quench, is interesting to study not only from the perspective of non-equilibrium theoretical physics but also has important consequences in practical applications like formation of emulsion, droplet dynamics in ocean and atmosphere etc.}  
A fully developed turbulent flow is sustained owing to an external forcing mainly applied at large scales. At large times, a non-equilibrium steady state is achieved due to the mutual balance between the average forcing and the dissipation. When the forcing is quenched, the flow tends to attain an equilibrium steady state through the process of turbulent relaxation. In case of neutral fluids, such relaxed states are found as a non-static constraint condition $\lamb=\bm{\nabla}p$, whereas for plasmas the state becomes $\bj\times\bb=\bnabla p$ considering low plasma-$\beta$.  





Although the turbulent relaxation of fluids and plasmas is a widely studied subject, the same has not been accomplished for other complex systems. One example of such a complex system are binary fluids. 
In recent times, a plethora of studies have been carried out in turbulent binary fluid model to explore its wide range of applications ranging from emulsions in industrial research to the study of droplet dynamics in the oceanic and atmospheric studies\cite{Mukherjee2019}.



\iffalse
Relaxation of systems undergoing turbulent motions is an old and widely studied subject. Starting from the force-free nature of cosmic plasmas, the relaxed state of a turbulent system has been traditionally explained in terms of the alignment between several dynamical variables. However, further analytical and numerical investigations have revealed the possibility of finding such relaxed states with a non-vanishing pressure gradient. In particular, such states have been obtained for charged fluids whose bulk motion can be neglected \textit{i.e.} low-$\beta$ plasmas. Outside plasmas, existence of such force-balanced states are not immediate and the physical mechanism responsible for obtaining such states is also not clear-cut. One example of such a system where turbulent relaxation has not been studied is binary fluids. In recent times, binary fluids and the corresponding turbulence have garnered considerable attention from different domains of scientific research. Apart from studies based on basic features like turbulent cascade, universal scaling laws, domain coarsening etc., several studies have also highlighted the importance of binary fluid turbulence (BFT) in the area of industrial research \cite{Pan2022}.   

All the studies focused on BFT have been carried out by treating the whole system as a single component. However, a binary fluid system can be additionally decomposed into bulks and interfaces, and it is interesting to study the dynamics of them individually. To the best of our knowledge, such a dedicated study has not been performed yet.
In this paper, for the first time, we have carried out a detailed study on the mechanism of binary fluid relaxation. In addition, a systematic study has been performed by investigating the dynamics of bulk and interface individually during the relaxation. We physically interpret our results using well-known `selective decay principle', as well as using recently proposed `principle of vanishing nonlinear transfer' \cite{Banerjee2022}. 

Nevertheless, several competing yet incomplete theories \textit{e.g.}, selective decay, minimum entropy production rate etc., have been proposed in order to explain such relaxed configurations. The selective decay principle, being the most popular among them, explains such states as the final product of a constrained minimization of some quantities, which may or may not be an inviscid invariant of the system. One major drawback of the selective decay principle is its inability to find relaxed states with a finite pressure gradients which are generally observed in numerical simulations.


In recent times, the effect of turbulence in a binary fluid system is an active area of research. Several aspects of binary fluid turbulence (BFT hereinafter) such as universal scaling laws, dynamics of coarsening length etc., have been widely studied both analytically and numerically. However, one aspect of BFT which is not generally studied is the area of BFT relaxation. Although it is well-known that a binary fluid mixture tends to phase separate in the absence of turbulent forcing, however the underlying mechanism for a such a relaxation is not well-known. More importantly, how the different components of the system behave while going from a highly nonlinear chaotic regime to a more quiescent one is an important area of study. In this paper, for the first time, we try to address these questions by performing a detailed numerical study of a binary fluid system undergoing a turbulent quench.


Turbulence is characterized by highly nonlinear flow regimes which are ubiquitous in nature. Fully developed turbulence consists of a wide range of interacting time and length scales. In general, any inviscid invariant (energy or helicities) of the flow, injected in the large scales, is expected to cascade nonlinearly from one scale to other until it dissipates at the so called Kolmogorov scale. The flux rates of such invariants are expected to remain constant over a range of length scales which are far from both injection and dissipation. If the injection is abruptly quenched, the system tries self-organize itself through the process of turbulent relaxation. Historically, such self-organizing relaxation processes have been studied in the context of astrophysical plasmas where an alignment between the magnetic and the current density field was observed. Aforementioned Beltrami-Taylor (BT) states have been physically explained in terms of the selective decay principle where a quickly decaying quantity is varied for fixed values of certain inviscid invariants. Selective decay principle has been invoked to explain relaxed states of incompressible hydrodynamic (HD) and magnetohydrodynamic (MHD) flows as well as for two-fluid plasmas. However, numerically, such aligned states are also found to support finite pressure gradients which are not captured by the principle of selective decay. In order to explain such states, a principle akin to selective decay has been proposed where the relaxed states are found by minimizing systems entropy production rate. Instead of using such extremization principles, another interesting method of finding such aligned states was proposed for non-ideal MHD by using Cauchy-Schwartz inequality.   
Recently, a new principle, namely the principle of vanishing nonlinear transfer (PVNLT) has been proposed to provide a more general description of such relaxation processes. Rather than extremizing some esoteric quantities, PVNLT defines the relaxed state as the condition where the nonlinear scale-to-scale transfer rate of all inviscid invariants simultaneously vanishes. Relaxed state solutions obtained through PVNLT are not only able to capture finite pressure gradients but they also reduce to well-known BT aligned states under appropriate limits (\textit{e.g.} low plasma-$\beta$).  
Relaxed states for a scalar field in a turbulent flow have been rarely studied. A scalar field can be active or passive depending on whether it exerts feedback to the momentum equation or not. If the scalar field is active, and has a linear feedback force density, the total energy (kinetic + scalar) is an inviscid invariant and is expected to cascade nonlinearly. For a binary fluid system, the form of the feedback force is much more complicated. Binary fluids are two component systems that consist of two simple immiscible fluids \textit{e.g.} oil and water to more complicated mixtures where one component can be active. In general, below a critical temperature, the components of a binary fluid tend to phase separate. However, the separation can be arrested through turbulence owing to its mixing properties. Recently, the cascade rate of energy has been studied for such a turbulent binary fluid using two-point statistics assuming homogeneity \cite{Pan2022}. In this paper, we study the relaxation process in binary fluid turbulence (BFT) using PVNLT. The relaxed state solutions are obtained by setting two-point correlators of inviscid invariants to zero.\fi


%%%%%%%%%%%%%%%%%%%%%%%%%%%%%%%%%%%%%%%%%%%%%%%%%%%%%%%%%%%%%%%%
\subsection{Principle of vanishing nonlinear transfer}
From Eqs.~\eqref{Eu}--\eqref{EQ}, the time evolution of the correlators are given by
\begin{align}
\partial_t{\cal R}_E &= \langle\mathcal{F}_{tr}^E\rangle + \langle f_c^E\rangle + \langle d_c^E\rangle,\, \text{and}\\
\partial_t{\cal R}_{S} &= \langle\mathcal{F}_{tr}^{S}\rangle + \langle f_c^{S}\rangle + \langle d_c^{S}\rangle, \end{align}
where,
\begin{align}
\langle\mathcal{F}_{tr}^E\rangle
&= \left\langle \bu^{\prime} \cdot \left(  \bu \times \bomega - \xi \bQ \bm{\nabla}\cdot\bQ - \bm{\nabla} P \right)\right. \nonumber\\
&\left.+
 \bu\cdot ( \bu^{\prime} \times \bomega^{\prime} - \xi \bQ^{\prime}\bm{\nabla}^{\prime}\cdot\bQ^\prime -\bm{\nabla}^{\prime} P^{\prime})\right.\nonumber\\  
 &\left.- \xi \bQ^{\prime}\cdot\bm{\nabla}(\bu\cdot\bQ)-\xi\bQ \cdot\bm{\nabla}^{\prime}(\bu^{\prime} \cdot \bQ^{\prime})\right\rangle,   
\label{Ftrs_E} \\
 \langle\mathcal{F}_{tr}^S\rangle &=  \langle \phi^{\prime} (\bu\cdot\bQ) + \phi ( \bu^{\prime}\cdot\bQ^{\prime}) \rangle, \label{Ftrs_S}
\end{align}
with $\langle f_c^E\rangle$ and $\langle f_c^S\rangle$ being the respective two-point forcing contributions 
and $\langle d_c^E\rangle $ and $\langle d_c^S\rangle$ being the respective two-point dissipation contributions. When the forcing is quenched, the system attains a state of relaxation where $\langle \mathcal{F}_{tr}^E\rangle = \langle \mathcal{F}_{tr}^S\rangle= \langle = 0$ and one then simultaneously obtains
\begin{gather}
    \bu \times \bomega - \xi \bQ (\bm{\nabla}\cdot\bQ)- \bm{\nabla}P =  \bm{\nabla} \Phi \label{PVNLT_E1}, \text{and} \\
    \bu \cdot \bQ = 0, \label{PVNLT_S1}\end{gather}
where $\Phi$ is an arbitrary scalar field.
%%%%%%%%%%%%%%%%%%%%%%%%%%%%%%%%%%%%%%%%%%%%%


@@@%%%%%%%%%%%%%%%%%%%%%%@@@@@@@@@@@@@@@
\section*{A possible universal theory of relaxation}
Note that from Eq.~\eqref{Ftrs_E} we obtain another condition $\bnabla (\bu \cdot \bQ) = \bnabla \times \bm{A}$, where $\bm{A}$ is an arbitrary vector field. However, from Eq.~\eqref{PVNLT_S} one can see that this condition is trivially satisfied. As one can expect, the relaxed state of a binary fluid is characterized by the minimization of the interface between the two fluids. This is achieved when the two fluids are entirely phase separated. The aforesaid conditions should therefore be satisfied both inside the bulk of each fluid and also on the interface. Let us start with the condition Eq.~\eqref{PVNLT_S}. Apparently it means that $\bu$ and $\bQ$ are perpendicular at relaxed state if none of the two is a null vector. However, a careful inspection of the Eq.~\eqref{Ftrs_S} shows $\langle\mathcal{F}_{tr}^S\rangle = - \langle \phi^{\prime} \bu^{\prime} \cdot\bQ + \phi  \bu\cdot\bQ^{\prime} \rangle$ for homogeneous turbulence. A vanishing of $\langle\mathcal{F}_{tr}^S\rangle$ at all scales inside the inertial range therefore requires the vanishing of at least one of $\bu$ and $\bQ$. Inside the bulk of a single fluid, this condition is satisfied even for non-zero $\bu$ as $\bQ =0$. However, on the interface, both $\phi$ and $\bQ$ are non-zero and hence $\bu$ must vanish at the relaxed state. This conclusion obviously makes the Eq.~\eqref{PVNLT_Q} to satisfy trivially both in the bulk and on the interface when the system reaches the relaxed state. In the bulk region, $\bQ =0$ and using the Eq.~\eqref{PVNLT_E} one can directly recover the traditional relaxed state $\bu \times \bomega = \bm{\nabla}P $ if we set the term $\bm{\nabla} \Phi$ to  zero. On the interface, $\bu =0$ and the relaxation condition reduces to 
$ \bnabla \times \bQ \left( \bnabla \cdot \bQ \right) = 0 \implies \nabla^2 \bQ = \lambda \bQ$, where $\lambda$ is the alignment parameter.
%%%%%%%%%%%%%%%%%%%%%%%%%%%%%%%%%%%%%%%%%%%%%%%%%%%%%
%%%%%%%%%%%%%%%%%%%%%%%%%%%%%%%%%%%%%%%%%%%%%%%%%%%%


% % Figure environment removed

\section*{An atypical relaxation of single fluid components of binary fluid }
Till now, we have looked into the average behavior of various quantities and obtained signatures of both BT-type force-free and PB-type relaxed states in binary fluids. According to PVNLT, any such relaxation condition must hold at each space point. Hence, to check the validity of the different apparently possible relaxed states at each fluid point, we plot the PDFs of various nonlinear terms at different times (Fig.~\ref{fig:Fig2} and \ref{fig:Fig3}). 
In this section, we shall examine the possible relaxed states of the bulk fluid. 

Looking into the early stage dynamics from Fig.~\ref{fig:Fig2} (a), we observe that the peak of PDFs of both $|\lamb|$ and $|\lamb - \bnabla P|$ shifts towards zero just after the relaxation and the later one is more peaked compared to previous indicating towards a balance between $\lamb$ and $\bnabla P$ \cite{Kraichnan1988}. However, to see whether this shift is due to equality between $\lamb$ and $\bnabla P$, we plotted the PDFs of both the terms for later stages of relaxation (see Fig.~\ref{fig:Fig3_a}). In case of balance between $\lamb$ and $\bnabla$ at each bulk fluid point, their PDFs should have a significant overlap at later stages. however, we see that although the peak of the PDF of $|\lamb - \bnabla P|$ shifts rapidly towards zero at later times ($t_{nl} = 50 $), the PDFs of $|\lamb|$ and $|\bnabla P|$ differ significantly. We further plotted the probability density distribution (PDF) of the cosine of the angle between $\lamb$ and $\bnabla P$ in Fig.~\ref{fig:PDF of angles} (b). The cosine of angle ($\theta$) between any two vectors $\bm{A}$ and $\bm{B}$ is defined as follows -
\begin{equation}
    \cos \angle (\bm{A},\bm{B}) \equiv \cos \theta = \frac{\bm{A}\cdot \bm{B}}{|\bm{A}||\bm{B}|} \label{cos_angle} 
\end{equation}
If the vectors $\lamb$ and $\bnabla P$ tend to  equalize during relaxation, then the PDF of $\cos \angle (\lamb,\bnabla P)$ would peak more to the bins nearest to $1$ at later times. From Fig.~\ref{fig:PDF of angles} (b), we also observe that the peak decrease with relaxation, implying that the pressure-balanced condition \eqref{Bulk} does not hold during relaxation. 

Now we look for the possibility of force-free relaxation. Again, from the early stage dynamics, we see that the peak of the PDF of $|\lamb|$ 
shifts towards zero with time. To distinguish whether this is due to the viscous dissipation or due to the $\bu$-$\bomega$ alignment, in Fig.~\ref{fig:PDF of angles}(a) we plot the probability density (PDF) of cosine of the angle between $\bu$ and $\bomega$. 
For the $\bu$-$\bomega$ alignment the PDF of $\cos \angle (\bu, \bomega)$ should peak to the bins nearest  to $\pm 1$. Again, from the figure, we see that the peaks do not deviate much from the fully-developed turbulent state. Hence, also removes the possibility of the force-free relaxation of the bulk relaxation. Hence, the condition \eqref{Bulk} would only satisfy at each fluid point only when $\bu = 0$.

From the above analysis, we conclude that the single fluid does not relax like an HD fluid.


%\section*{A pressure-balanced relaxed state of the binary fluid interface}
%Similarly, in the interface 
 

% In bulk, $|\bm{Q}|<< 1$ (see \ref{Numerical_simulations} for details), hence $\xi |\bm{Q}\left(\bm{\nabla}\cdot \bm{Q}\right)|$ can be neglected and the only non-linear terms, $\lamb$ and $\bnabla P$ are important here. We see that after quenching, the mode of the HS of both quantities shifts toward zero. This signifies that both quantities decay due to viscous dissipation (Fig.~\ref{fig:TimeS_All_Nl_Hist_BulkInterface62}a). However, the decay is in such a manner that the absolute values of $\lamb$ and $\bnabla P$ balance each other, $i.e.,$ $|\lamb| \sim |\bnabla P|$ as time increases. This is also evident from the HS of $|\lamb - \bxiQdivQ- \bnabla P|\sim |\lamb - \bnabla P|$ (see Fig.~\ref{fig:TimeS_All_Nl_Hist_BulkInterface62}a at $t_{nl} = 70$). Also, on average, the same balance holds \textit{i.e.},$\langle|\lamb|\rangle \sim \langle |\bnabla P|\rangle $ (see the left panel of Fig.~\ref{fig:Avg_Nl_BulkInterface}). Such pressure-balance relaxation is also observed in the earlier works of the single fluid relaxation \cite{Kraichnan1988,Shtilman1989}. 


% In the interfacial region ($|\bm{Q}|>>1$), the non-linear term $\xi \bm{Q}\left(\bm{\nabla}\cdot \bm{Q}\right)$ plays a significant role. In addition, since $\bnabla P$ contains terms involving $\bm{Q}$ \cite{Pan2022}, it is non-negligible in the interface. However, $\lamb$ decays with time as in bulk due to viscous dissipation and is negligible compare to the others $i.e.,$ $|\lamb| << |\bxiQdivQ|$ and $|\lamb| << |\bnabla P|$ at later times (see the HS in Fig.~\ref{fig:TimeS_All_Nl_Hist_BulkInterface62}b). Here also we observe a relaxation via pressure-balance which take place due to the balance of $|\bnabla P|$ with $|\bxiQdivQ|$. This fact clearly manifests from the HS in Fig.~\ref{fig:TimeS_All_Nl_Hist_BulkInterface62}b at $t_{nl} = 40$ and $70$ and also from the average behaviour of $|\bnabla P|$ and $|\bxiQdivQ|$ with time (see right panel of Fig.~\ref{fig:Avg_Nl_BulkInterface}). This kind of pressure-balanced relaxation is seen in the solar-wind in the low plasma-$\beta$ limit \cite{She1991}. 


% it  RMS value of $\lamb$ is negligible compare to the RMS values of $\xi \bm{Q}\bm{\nabla}\cdot \bm{Q}$ and $\bnabla P$ in the interface as it again decays due to dissipation. 
%%%%%%%%%%%%%%%%%%%%%%%%%%%%%%%%%%%%%%%%%%%%%%%%%%%%%%%%%
%%%%%%%%%%%%%%%%%%%%%%%%%%%%%%%%%%%%%%%%%%%%%%%%%%%%%%%%

\section*{PVNLT}
Inside the inertial zone, i.e., a range of length scales free from both forcing and dissipation, such nonlinearities are responsible for the scale-to-scale transfer of inviscid invariants with a constant flux rate. Therefore, a reduction of nonlinearities essentially dictates the vanishing of scale-to-scale transfer of inviscid invariants. Based on these general ideas, a new principle, namely the principle of vanishing nonlinear transfer (PVNLT) has been recently proposed \cite{Banerjee2023}. According to PVNLT, the relaxed states are the trivial steady states achieved at relaxation when $\partial_t\mathcal{R}_M$ vanishes, where $\mathcal{R}_M$ is the two-point symmetric correlator  corresponding to the inviscid invariant $M$. The incompressible CHNS system permits two inviscid invariants, namely, the scalar energy $S \ (= \int \phi^2/2\ d \tau)$ and the total energy $E \ [=\int  (u^2 + \xi Q^2)/2 \ d\tau]$\cite{Pan2022}. In the inviscid limit, the invariance of $E$ and $S$ can be readily shown by writing their evolution equations in the local conservative forms as
\begin{align}
d_t S &= -\int \bnabla \cdot \left[ \frac{\phi^2}{2} \bu\right]d \tau , \label{dS}\\
    d_t E &= - \int \bnabla \cdot \left[P \bu + \xi (\bu \cdot \bQ) \bQ\right]d \tau , \label{dE} 
    \end{align}
followed by the application of the Gauss divergence theorem along with periodic or vanishing boundary conditions. $E$ and $S$ are statistically conserved (see Fig.~\ref{fig:diss}). 



For incompressible BFT system, we therefore construct the two-point correlators $\mathcal{R}_S = \left\langle \phi\phi^\prime \right\rangle/2$ and  $\mathcal{R}_E= \left\langle \bu\cdot\bu^\prime+\xi\bQ\cdot\bQ^\prime \right\rangle/2 $ 
 for $S$ and $E$ respectively. The primed and unprimed quantities represent field variables at position  $\bm{x}+\bm{r}$ and $\bm{x}$ respectively. Vanishing $\partial_t \mathcal{R}_E$ and $\partial_t \mathcal{R}_S$ one obtains (see methods for detailed derivation)
\begin{align}
  \bu \times \bomega - \xi \bQ (\bm{\nabla}\cdot\bQ)- & \bm{\nabla}P =  \bm{\nabla} \Phi \label{PVNLT_E}, \\
  \bu \cdot \bQ = 0 ,\;\;\text{or }\;\;& \bu = 0 \label{PVNLT_S}
\end{align}
where $\Phi$ is an arbitrary scalar field. 
One of our primary objectives was to see whether we are getting a pressure-balanced relaxed state (hereafter, PB states) obtained in \eqref{PVNLT_E} or whether this state ultimately reduces to a force-free relaxed state as obtained in incompressible low-$\beta$ plasmas. To explore different possibilities we begin our investigation by looking into the average behavior of different nonlinear terms present in condition \eqref{PVNLT_E}. Inside bulk, \eqref{PVNLT_E} reduces to $\bu\times \bomega - \bnabla P = \bnabla \Phi$ and condition \eqref{PVNLT_S} is trivially satisfied. Primitively, one can expect the bulk to relax as a simple hydrodynamic (HD) fluid. Several numerical studies have shown that a HD fluid relaxes either by a BT type $\bu$-$\bomega$ alignment or by a balance between $\lamb$ and $\bnabla P$ \cite{Pelz1985,Kraichnan1988,Banerjee2023}. Hence, for the moment, it is convenient to take $\bnabla \Phi = \bm{0}$ in bulk. The relaxation condition in bulk reduces to 
\begin{equation}
    \lamb - \bnabla P = 0 \label{Bulk}
\end{equation}
For the two-fluid interface, no such single fluid analogy can be drawn and $\bnabla \Phi$ cannot be set to zero priorly. In bulk, comparing the average values of $|\lamb|$ and $|\bnabla P|$ (Fig.~\ref{fig:Average_timeS}), we found that both $|\lamb|$ and $|\bnabla P|$ and their difference decay at large times due to viscous dissipation. In addition, both the terms apparently decay in a similar manner which might be due to the possible alignment between them. At the interface, $|\lamb|$ again decays to zero at large times. However, $\bQ$ always remains finite because of the presence of a chemical potential gradient across the interface. Hence, the nonlinear terms containing composition field $\bQ$  i.e., $\xi\bQ(\bnabla\cdot\bQ)$ and $\bnabla P$ remain finite at later times. Interestingly, these two on average approaches the same value at large times. From \eqref{PVNLT_E} this behavior can be justified by the absence of $\bnabla \Phi$ and a pressure balance relaxation in the interfacial region  (Fig.~\ref{fig:Average_timeS}). Hence, we have obtained the possibility of PB-type relaxation in both regimes. Apart from this, the decay of $|\lamb|$ in bulk could possibly be due to the $\bu$-$\bomega$ alignment which could also lead to a force-free relaxation.  For the interface, a force-free condition would imply $ \lamb - \bxiQdivQ = 0$, which is clearly not possible since the pressure gradient is always non-negligible here (see Fig.~\ref{fig:Average_timeS}). We shall examine all the above possibilities in detail in the next section. 
%%%%%%%%%%%%%%%%%%%%%%%%%%%%%%%%%%%%%%%%%%%%%%%%%%%%%%%%%%%%%%%%%%%%%%%%%%%%%%%%%%%%%%%%%%%%%%%%%%%%%

o\textcolor{purple}{\textbf{Points discussed so far}\\
1. Bulk: For CHNS, from the average behavior we see that both $\lamb$ and $\bnabla P$ decrease with time.\\
2. Now, the question is whether the fall is only due to dissipation or in addition there is a tendency of the two vectors to align with time so that the condition $\lamb - \bnabla P = 0$ satisfies nontrivially.\\
3. From, Fig.~\ref{fig:Angle_Bulk_interface_uCw_gradP}, we see that the alignment decrease with time from the initial peak. Hence, the above condition satisfies only trivially i.e., when $\bu = \bm{0}$. \\
4. However, note that for the corresponding HD case, from Fig.~\ref{fig:HD_angle} we see that the corresponding peak increases with time implying that along with dissipation there is a tendency for the alignment of two vectors for incompressible HD. \\
5. Next, we also observe that both in bulk and in the interface, $|\lamb|$ decreases with time. Again, to check whether, in addition to viscous dissipation, the decay is due to $\bu$-$\bomega$ alignment, we plot the cosine of the angle between the two in Fig.~\ref{fig:Angle_Bulk_interface}\\.
6. From Fig.~\ref{fig:Angle_Bulk_interface}, we see that neither bulk nor interface has a tendency toward $\bu$-$\bomega$ alignment. Instead, in the interfacial region, they tend to dissalign.}
%%%%%%%%%%%%%%%%%%%%%%%%%%%%%%%%%%%%%%%%%%%%%%%%%%%%%%%%%%%%%%%%%%%%%%%%%%%%%%%%%%%%%%%%%%%%%%%%%%%%%%%