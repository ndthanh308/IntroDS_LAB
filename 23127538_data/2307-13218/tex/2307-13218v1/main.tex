\documentclass{article}
\usepackage[utf8]{inputenc}
% \usepackage[bottom = 1.5in, top = 1.5in, total={6.6in, 9in}]{geometry}
\usepackage{libertine}
\usepackage{graphicx}
\usepackage{epigraph}
\usepackage{comment}
\usepackage[table]{xcolor}
\usepackage{physics}
\usepackage{microtype}
 \usepackage[labelfont=bf]{caption}
\usepackage{multirow}
\usepackage{amsmath}
\usepackage{amssymb}
\usepackage[style=authoryear-icomp, bibstyle=authortitle,labelyear=true,backend=biber,isbn=false,url=false,doi=false,eprint=false]{biblatex} 
\addbibresource{bib.bib}
%\newcommand{\x}[1]{{\color{blue}#1}}
%\newcommand{\y}[1]{{\color{purple}#1}}
%\newcommand{\ec}[1]{{\color{red}#1}}
\newcommand{\x}[1]{{#1}}
\newcommand{\y}[1]{{#1}}
\newcommand{\ec}[1]{{#1}}

\title{Decoherence, Branching, and the Born Rule \\ in a Mixed-State Everettian Multiverse\thanks{This work is fully collaborative; the authors are listed anti-alphabetically.}}
\author{Eugene Y. S. Chua\thanks{Division of the Humanities and Social Sciences, California Institute of Technology. Website: www.eugenechua.com. Email: eugene.chua@cantab.net} \, and Eddy Keming Chen\thanks{Department of Philosophy, University of California, San Diego. Website: www.eddykemingchen.net. Email: eddykemingchen@ucsd.edu}}
\date{\today}

\begin{document}

\maketitle

\begin{abstract}
  \noindent  In Everettian quantum mechanics, justifications for the Born rule appeal to self-locating uncertainty \y{or} decision theory. Such justifications have focused exclusively on a pure-state Everettian multiverse, represented by a wave function. Recent works in quantum foundations suggest that it is viable to consider a mixed-state Everettian multiverse, represented by a (mixed-state) density matrix. Here, we develop the conceptual foundations for decoherence and branching in a mixed-state multiverse, and extend the standard Everettian justifications for the Born rule to this setting. This extended framework provides a unification of `classical' and `quantum' probabilities, and additional theoretical benefits, for the Everettian picture.
\end{abstract}

\begingroup
\tableofcontents
\endgroup

% \section*{To-Do}

% \begin{itemize}
%     \item Rewrite introduction and Abstract to emphasize decoherence and branching (\y{Done; Eddy})
%     \item Reframe 2.1 as section 2: Decoherence, emphasis on branching. \y{Done; Eddy}.  
%     \item Add complete references; Eddy to do. 
%     \item Merge 2.2 into 3: Born Rule, and streamline (\ec{Done, Eugene, streamlining to be continued})
%     \item Leave out Appendix A + references to it (\ec{Done; Eugene})
%     \item Cut down the pure state, argue that it's the same. In the pure state case there's nothing special or new. What about mixed states case. (\ec{Not Done, but not sure if including concrete elaboration of the Sebens-Carroll might be better, Eugene})
%     \item Maybe include the appendix B in footnote as exercise for reader.  (Not Done, Eugene)
%      \item Erasure and its generalization. \y{Eddy, done}. 
%     \item Discussion -- rework / be careful with classical and quantum probability unification.[Eddy, done ]
%     \item Add discussion of Simon Saunders's branching counting thing as a footnote. Eddy, done. 
%     \item Discussion -- concluding paragraph/sentence: given DMR\textsubscript{E}'s equivalence... 
%     \item merge 3.2, and 5, into 4: cut out pure WFR version of Sebens-Carroll -- unified section 4 (\ec{Merging done, Eugene)}
% \end{itemize}

\section{Introduction}

Everettian quantum mechanics (EQM) is a minimalist interpretation of quantum mechanics with some counter-intuitive features (\cite{sep-qm-manyworlds, sep-qm-everett}). Instead of attempting to collapse the quantum state or adding extra variables to obtain a definite outcome for each experiment, it proposes to take unitary quantum mechanics as fundamental and replace our single-world ontology with a multiverse, where every possible outcome of an experiment is realized in some branch (a parallel world). Hence it is also sometimes called the `many-worlds' interpretation. 

There are two main issues with EQM, one metaphysical and the other epistemological. The metaphysical issue concerns the ontology of EQM. How do we obtain the appearance of a classical world, with definite records and observers, from the quantum state? A much discussed solution appeals to decoherence, with its ability to suppress interference and give rise to an ``emergent multiverse'' (\cite{Wallace2012}). The universal quantum state evolves into one with many branches, each representing an emergent (quasi-)classical world. 

The epistemological issue concerns the understanding of probability in EQM. A key postulate of quantum mechanics, and a crucial element of its empirical confirmation, is the Born rule: the probability of observing a certain outcome is given by the squared amplitude of the quantum state. How should we  make sense of this probability when every measurement outcome occurs on some branch of the Everettian multiverse, and what justifies the interpretation of the squared amplitutdes as probabilities? There are several responses to the probability issue. The Deutsch-Wallace program understands probability in terms of the betting preferences of agents within the multiverse, which uses a decision-theoretic representation theorem to prove that the agent's credences must satisfy the Born rule, on pain of irrationality (e.g. \cite{deutsch1999}, \cite{Wallace2012}). The Sebens-Carroll (\citeyear{sebenscarroll2016}) and McQueen-Vaidman (\citeyear{mcqueenvaidman2018})  programs understand probability in terms of self-locating uncertainty of a localized agent on some branch, employing certain epistemic principles -- such as ``separability'' or ``symmetry'' -- to prove that the agent's self-locating uncertainty must satisfy the Born rule. 

Promising as they may be, these defenses and justifications of EQM have an apparent limitation. They focus exclusively on the case of a universal pure state, where the quantum state of the multiverse is represented by a wave function. Defenders of EQM, like many other realist interpreters, regard the universal pure state as representing something objective and mind-independent. However, recent works in quantum foundations (\cite{durr2005role, maroney2005density, allori2013predictions, Wallace2012, chen2018, robertson2022search}) suggest that the above approach to realism, based on the wave function, is not the only possibility for realism about the quantum state. It's also viable -- and in some circumstances even more theoretically attractive -- to take a realist stance based on the density matrix (\cite{chen2018}). On this view, we can associate (possibly mixed-state) density matrices, rather than (necessarily pure-state) wave functions, to isolated systems and even to the entire universe. While density matrices are conventionally used to represent ignorance about some underlying wave function or the external environment, \textit{it's also possible to regard density matrices as fundamental}. On the new picture, the universe as a whole can be aptly represented by a fundamental density matrix evolving unitarily according to the von Neumann equation. In contrast, on the standard picture, it is represented as a wave function evolving unitarily according to the Schrodinger equation. If the fundamental density matrix in this new realist picture is mathematically the same as that of the ``ignorance'' density matrix in the standard picture, the two theories will be empirically equivalent, since they make the same statistical predictions for all experiments.  

All wave functions correspond to some pure-state density matrices, but not all density matrices have corresponding wave functions. Thus, realism based on the density matrix allows for more quantum states than realism based on the wave function. The former is also compatible with a theoretically attractive package -- the Wentaculus -- which provides a unified explanation for quantum phenomena and the thermodynamic arrow of time (\cite{chen2018valia}, \cite{chen2018}, \cite{chen2018HU}, \cite{chen2018NV}). Following \textcite{chen2018, chen2019realism}, we call this new picture \textit{Density Matrix Realism} (DMR) and the old one \textit{Wave Function Realism} (WFR). We denote the Everettian versions of DMR and WFR as DMR\textsubscript{E} and WFR\textsubscript{E} respectively. (Note that this is a wider conception of quantum state realism than that of \textcite{AlbertEQM} and \textcite{ney2021world}.)

It is an open question whether standard arguments for branching and the Born rule generalize from WFR\textsubscript{E} to DMR\textsubscript{E}. If there's no generalization available, then WFR\textsubscript{E} might still be preferable to DMR\textsubscript{E}, since the former -- not the latter -- solves the problems of ontology and probability.  
In this paper, we argue that the standard justifications for branching and the Born rule in WFR\textsubscript{E} can be extended to DMR\textsubscript{E}. In particular, we show how the three aforementioned programs for justifying the Born rule -- the Sebens-Carroll program (\S3.1), the McQueen-Vaidman program (\S3.2), and the Deutsch-Wallace program (\S3.3) -- do not depend crucially on WFR\textsubscript{E}, but can also generalize readily to DMR\textsubscript{E}. 

% Just as various programs have justified the Born rule for pure states by appealing only to the quantum state, together with the dynamical laws of quantum mechanics, decoherence, and some epistemic principles, we show that the same can be done for mixed states and DMR\textsubscript{E}. We argue that none of the postulates in previous justifications of the Born rule essentially require a universal pure state: we can divide the (possibly-mixed) quantum state into equal-amplitude branches, to which we may then apply the same epistemic principles. For concreteness, we show how the Sebens-Carroll derivation of the Born rule for universal pure states extends to the case of universal mixed states. (The extended proof uses the same Epistemic Separability Principle that \cite{sebenscarroll2016} appeals to, and inherits the same conceptual issues of whether it has a rational justification.)

This project has several conceptual payoffs. 
First, it requires us to clarify the ontological structure of the multiverse \y{and the requirements of decoherence}. As it turns out, branching requires decoherence but decoherence does not require a universal pure state.  The story of decoherence applies both to pure and mixed states, which has been underappreciated in the literature. 

Second, with the access to a larger state space, Everettians can explore new theoretical possibilities that are naturally suggested by DMR. For example, DMR\textsubscript{E} provides the basis for a unified account of probability that may be absent on WFR\textsubscript{E}. On WFR\textsubscript{E}, without knowing what the universal wave function is, we may assign a density matrix $\rho$ to represent our epistemic state. The probabilities we extract from $\rho$ range over various possible candidate multiverses. As such, it is not interpreted as self-locating uncertainty or betting preferences of agents within a multiverse, and must be treated as a distinct source of probability (e.g. statistical mechanical / classical probability of possible initial conditions). In contrast, DMR\textsubscript{E} allows us to regard $\rho$ as representing the actual fundamental quantum state of the multiverse. We have the option to posit just one source of probability, corresponding to the weights associated with branches of the actual mixed-state multiverse. 

Finally, our results establish DMR\textsubscript{E} \x{as a viable version of EQM} and a viable rival to WFR\textsubscript{E}, by showing the former capable of tackling EQM's issues of ontology and probability, via the same resources as the latter. This leads to a case of in-principle empirical underdetermination: \textit{by Everettians' own lights},  there is an open question which version of EQM is the correct one. 


\section{Decoherence and Branching}

\ec{In this section, we suggest that decoherence and branching work essentially the same way in DMR\textsubscript{E} as in WFR\textsubscript{E}.}  

\subsection{Decoherence for Wave Function Realism}

We start with a brief review of how decoherence and branching work in WFR\textsubscript{E}. (Here we mostly follow \textcite{schlosshauer2007decoherence}.)

Consider a universal pure state $\Psi$ describing a system $S$ interacting with the environment $E$.  Given a system $S$ in a microscopic superposition of states $S_n$: 
\begin{equation}
    |S\rangle = \sum_{n} c_n |S_n\rangle
\end{equation}
interacting with $E$, at some time $t$ after the interaction, the universal state will become a macroscopic superposition: 
\begin{equation}\label{universal}
    |S+E\rangle = \sum_{n}c_n |S_n\rangle |E_n(t)\rangle
\end{equation}
\ec{where $|E_n(t)\rangle$ is the macroscopic `pointer state' associated with $S_n$.} For a simple case, consider $n = 2$. The density matrix describing the measurement outcomes on $S$, $\rho_S$ is: 
    \begin{equation}\label{subsystem}
    \begin{split}
      \rho_S = Tr_E (\rho_{S+E}) & = Tr_E|S+E\rangle \langle S+E| \\
    & = |a|^2|S_1\rangle \langle S_1| + |b|^2|S_2\rangle \langle S_2| \\
    & + ab^*|S_1\rangle \langle S_2| \langle  E_2(t)|E_1(t)\rangle + a^*b |S_2\rangle \langle S_1| \langle E_1(t)|E_2(t)\rangle  
    \end{split}
    \end{equation}
The last two terms represent the interference between the two macroscopically superposed states, and depends partly on $\langle E_1(t)|E_2(t)\rangle$ and $\langle E_2(t)|E_1(t)\rangle$. Generically, $\langle E_i(t)|E_j(t)\rangle$ quantifies the difference between two states of the environment. Due to innumerable interactions within its parts, the states of the environment are assumed to become rapidly approximately orthogonal under Schrödinger evolution, such that
\begin{equation}
    \langle E_i(t)|E_j(t)\rangle \propto e^{-t/\tau_d}, i \neq j
\end{equation}
holds, where $\tau_d$ is the characteristic decoherence timescale to be empirically determined for specific systems. Over time, $\langle E_1(t)|E_2(t)\rangle$ and $\langle E_2(t)|E_1(t)\rangle$ approach zero, so that: 
\begin{equation}
    \rho_S \approx |a|^2|S_1\rangle \langle S_1| + |b|^2|S_2\rangle \langle S_2|
\end{equation}
In other words, any measurement on the system $S$, entangled with $E$, effectively ignores the quantum interference between the macroscopically superposed component states. \y{Notice that the interference terms show up only at the subsystem level (\ref{subsystem}) and not explicitly at the universal level (\ref{universal}). In \S2.2 we shall see that although the story of decoherence applies the same way in DMR,  ``interference terms" also show up at the universal level, even though, as it is still the case, they make almost no contribution to the evolution of the component states.}

 We'll now use the position representation for both wave functions and density matrices, as it explicates the above story of decoherence and its approximate nature in a more concrete fashion. However, once we have a concrete grasp of decoherence in hand, we'll use the bra-ket notation for generalizing the Born rule to DMR\textsubscript{E}. 


Let us start with an example.  Before measurement, at $t_1$, the subsystem is about to be measured in one of two distinct spatial regions: $A$ and $B$;  the environment -- including the measurement apparatus -- is in a ``ready'' state to measure the location of the subsystem. Suppose the (pure) universal quantum state is: 
\begin{equation}\label{pure1}
 \Psi_1^{t_1}(x, y) = \frac{1}{\sqrt{2}} \bigg[A(x)\phi^{ready}(y) + B(x)\phi^{ready}(y) \bigg]
\end{equation}
with $A(x)$ the wave function of the subsystem located in region $A$ and $\phi^{ready}(y)$ the wave function of the environment ready to measure the subsystem; likewise for $B(x)$. Note that $y$ includes an enormous number of degrees of freedom since the environment, which includes the measurement apparatus, is a macroscopic system. In contrast, $x$ may only have a few degrees of freedom. 
By $t_2$, a measurement has been carried out. Now, the universal quantum state has unitarily evolved into:
 \begin{equation}\label{pure2}
 \Psi_1^{t_2}(x, y) = \frac{1}{\sqrt{2}} \bigg[A(x)\phi^A(y) + B(x)\phi^B(y) \bigg]
\end{equation}
with $\phi^A(y)$ the wave function of the environment that indicates the subsystem is in region $A$, and similarly for $\phi^B(y)$. Following the standard assumptions about measurement outcomes and decoherence, we stipulate that $\phi^A(y)$ and $\phi^B(y)$ have macroscopically disjoint supports, in the sense that their supports are not only disjoint but also contain macroscopic differences, such as different orientations of the pointer. We are then allowed to ignore the interference, so that, after measurement at $t_2$, there are only two branches of the wave function:  
\begin{itemize}
    \item a branch with the pointer reading ``$A$,'' represented by the pure state $A(x)\phi^A(y)$, with branch weight $\frac{1}{2}$. 
    \item a branch with the pointer reading ``$B$,'' represented by the pure state $B(x)\phi^B(y)$, with branch weight $\frac{1}{2}$. 
\end{itemize}
In short, the wave functions representing the two branches start out concentrated in nearby regions in configuration space, but after measurement their supports in configuration space move much farther apart into distinct macrostates (macrostates with different pointer readings) with negligible interference, and will move even further in the future. \y{Following Wallace (2012, p.88), we may say that ```branching' (relative to a given basis) is just the absence of interference.'' Future developments of branch A will (almost) entirely depend on $A(x)\phi^A(y)$ and receive (almost) no influence from $B(x)\phi^B(y)$.} 


\subsection{Decoherence for Density Matrix Realism -- Pure States}

We now show that the above story can also be told in DMR\textsubscript{E}. 

%NOTE: write equation (4) with density matrices, simplifying notations by labeling c.t. for generic cross-terms. c.t. becomes I when we take partial trace. Add CT to every equation about the universal density matrix later in the paper, and add approximation relation to the reduced states by appeal to decoherence. 

To set up the simple case of decoherence for DMR\textsubscript{E} and a fundamental quantum state as (possibly mixed) density matrix, let us first show how decoherence will work out for DMR\textsubscript{E} in the case of a pure density matrix. Let us first consider two possible \textit{pure} universal density matrices at $t_1$ before measurement:
\begin{equation}\label{rho2t1}
    \rho_2^{t_1} = \Psi_2^{t_1}(x, y) \Psi_2^{t_1*}(x', y')
\end{equation}
\begin{equation}\label{rho3t1}
    \rho_3^{t_1} = \Psi_3^{t_1}(x, y) \Psi_3^{t_1*}(x', y')
\end{equation}
corresponding to two universal wave functions respectively: 
\begin{equation}\label{pure3}
 \Psi_2^{t_1}(x, y) = \frac{1}{\sqrt{2}} \bigg[A(x)\phi^{ready}(y) + B(x)\phi^{ready}(y)\bigg]
\end{equation}
\begin{equation}\label{pure4}
 \Psi_3^{t_1}(x, y) = \frac{1}{\sqrt{2}} \bigg[A(x-\delta)\phi^{ready}(y) + C(x)\phi^{ready}(y)\bigg]
\end{equation}
with $\delta$ a microscopic position displacement that leaves the measurement macrostate invariant, so that the measurement apparatus will record $A(x-\delta)$ as ``$A$'' just as it does for $A(x)$. In other words, the microscopic differences between $A(x)$ and $A(x-\delta)$ will not be registered by the measurement devices. 

At $t_2$, the subsystem has undergone a measurement procedure. The above density matrices have unitarily evolved into these:
\begin{equation}\label{rho2t2}
    \rho_2^{t_2} = \Psi_2^{t_2}(x, y) \Psi_2^{t_2*}(x', y')
\end{equation}
\begin{equation}\label{rho3t2}
    \rho_3^{t_2} = \Psi_3^{t_2}(x, y) \Psi_3^{t_2*}(x', y')
\end{equation}
corresponding to the respective wave functions:
\begin{equation}\label{pure5}
 \Psi_2^{t_2}(x, y) = \frac{1}{\sqrt{2}} \bigg[A(x)\phi^{A}(y) + B(x)\phi^{B}(y)\bigg]
\end{equation}
\begin{equation}\label{pure6}
 \Psi_3^{t_2}(x, y) = \frac{1}{\sqrt{2}} \bigg[A(x-\delta)\phi^{A}(y) + C(x)\phi^{C}(y)\bigg]
\end{equation}
If the universe \textit{were} in either of the states (\ref{rho2t1}) or (\ref{rho3t1}), then its quantum state will be given by (\ref{rho2t2}) or (\ref{rho3t2}) respectively after unitary time evolution. For this case, essentially the same story for decoherence as WFR\textsubscript{E} can be given for DMR\textsubscript{E}. 

\y{However, there is an interesting feature of the density-matrix representation. In DMR\textsubscript{E}, even when the universal quantum state is pure, the ``interference terms'' will be formally present at the universal level after expansion into component states. For example, let us first expand the universal density matrix represented by (\ref{rho2t2}):
\begin{equation}\label{impure13}
\begin{aligned}
   \rho_2^{t_2} = \frac{1}{2}\Psi_2^{t_2}(x, y) \Psi_2^{t_2\ast}(x', y')  = &  \frac{1}{2}\bigg[ A(x)A^{\ast}(x')\phi^{A}(y)\phi^{A\ast}(y') \\
    & + A(x)B^{\ast}(x')\phi^{A}(y)\phi^{B\ast}(y') \\
    & + B(x)A^{\ast}(x')\phi^{B}(y)\phi^{A\ast}(y') \\
    & + B(x)B^{\ast}(x')\phi^{B}(y)\phi^{B\ast}(y') \bigg] 
\end{aligned}
\end{equation}
The second and the third terms in the middle will become almost zero only after taking the partial trace over the environmental degrees of freedom \x{$y$}. Note that such terms show up in the universal density matrix (even when it is pure) but do not appear in the wave function representation (\ref{pure5}), an interesting difference worth keeping in mind. Let us group such terms together and call them \textit{c.t.} for ``cross terms.'' Since they make negligible contributions to locally measurable quantities,  the cross terms are irrelevant to the emergent branching structure.
Rewriting (\ref{impure13}), we have: 
\begin{equation}\label{impure10}
\begin{aligned}
    \rho_2^{t_2} = \Psi_2^{t_2}(x, y) \Psi_2^{t_2\ast}(x', y') = & \frac{1}{2}\bigg[A(x)A^{\ast}(x')\phi^{A}(y)\phi^{A\ast}(y') \\
    & + B(x)B^{\ast}(x')\phi^{B}(y)\phi^{B\ast}(y')\bigg] + c.t.
\end{aligned}
\end{equation}
 In this case, the universal density matrix $\rho_2^{t_2}$ will have two branches: 
\begin{itemize}
    \item a branch with the pointer reading ``$A$,'' represented by the pure density matrix $A(x)A^*(x')\phi^{A}(y)\phi^{A*}(y')$, with branch weight $\frac{1}{2}$. 
    \item a branch with the pointer reading ``$B$,'' represented by the pure density matrix $B(x)B^*(x')\phi^{B}(y)\phi^{B*}(y')$, with branch weight $\frac{1}{2}$. 
\end{itemize}
Future developments of the first branch will (almost) entirely depend on $A(x)A^*(x')\phi^{A}(y)\phi^{A*}(y')$ and receive (almost) no contribution from $B(x)B^*(x')\phi^{B}(y)\phi^{B*}(y')$ or the cross terms. And similarly for the second branch.  }

For the other universal density matrix, we have:
\begin{equation}\label{impure100}
\begin{aligned}
    \rho_3^{t_2} =  \Psi_3^{t_2}(x, y) \Psi_3^{t_2\ast}(x', y')  = &  \frac{1}{2} \bigg[A(x-\delta)A^{\ast}(x'-\delta)\phi^{A}(y)\phi^{A\ast}(y') \\ & + C(x)C^{\ast}(x')\phi^{C}(y)\phi^{C\ast}(y')\bigg] + c.t.
\end{aligned}
\end{equation}
 In this case, the universal density matrix $\rho_3^{t_2}$ will have two branches: 
\begin{itemize}
    \item a branch with the pointer reading ``$A$,'' represented by the pure density matrix $A(x-\delta)A^{\ast}(x'-\delta)\phi^{A}(y)\phi^{A\ast}(y')$, with branch weight $\frac{1}{2}$. 
    \item a branch with the pointer reading ``$C$,'' represented by the pure density matrix $C(x)C^{\ast}(x')\phi^{C}(y)\phi^{C\ast}(y')$, with branch weight $\frac{1}{2}$. 
\end{itemize}
Future developments of the first branch will (almost) entirely depend on $A(x-\delta)A^{\ast}(x'-\delta)\phi^{A}(y)\phi^{A\ast}(y')$ and receive (almost) no contribution from $C(x)C^{\ast}(x')\phi^{C}(y)\phi^{C\ast}(y')$ or the cross terms. And similarly for the second branch. 


\subsection{Decoherence for Density Matrix Realism -- Mixed States}

Now, consider the case when the fundamental universal quantum state is an \textit{impure} (mixed-state) density matrix, the equal mixture of $\rho_2^{t_1}$ and $\rho_3^{t_1}$:
\begin{equation}\label{impure7}
\begin{aligned}
    \rho^{t_1} (x,y,x',y') & = \frac{1}{2} \rho_2^{t_1} + \frac{1}{2} \rho_3^{t_1} \\
        & = \frac{1}{2} \Psi_2^{t_1}(x, y) \Psi_2^{t_1\ast}(x', y') + \frac{1}{2} \Psi_3^{t_1}(x, y) \Psi_3^{t_1\ast}(x', y')
\end{aligned}
\end{equation}
What is the story for decoherence here? As before, the subsystem has undergone a measurement procedure by $t_2$. The universal quantum state has unitarily evolved into an equal mixture of $\rho_2^{t_2}$ and $\rho_3^{t_2}$:
\begin{equation}\label{impure8}
\begin{aligned}
 \rho^{t_2}(x,y,x',y') & = \frac{1}{2} \rho_2^{t_2} + \frac{1}{2} \rho_3^{t_2} \\
 & = \frac{1}{2} \Psi_2^{t_2}(x, y) \Psi_2^{t_2\ast}(x', y') + \frac{1}{2} \Psi_3^{t_2}(x, y) \Psi_3^{t_2\ast}(x', y')
 \end{aligned}
\end{equation}
The first term on the right hand side of (\ref{impure8}) can be expanded as follows: 
\begin{equation}\label{impure9}
\begin{aligned}
    \frac{1}{2}\rho_2^{t_2} = \frac{1}{2} \Psi_2^{t_2}(x, y) \Psi_2^{t_2\ast}(x', y') = & \frac{1}{4}\bigg[A(x)A^{\ast}(x')\phi^{A}(y)\phi^{A\ast}(y') \\
    & + B(x)B^{\ast}(x')\phi^{B}(y)\phi^{B\ast}(y')\bigg] + c.t.
\end{aligned}
\end{equation}
Similarly, the second term on the right hand side of equation (\ref{impure8}) is: 
\begin{equation}\label{impure11}
\begin{aligned}
    \frac{1}{2}\rho_3^{t_2} = \frac{1}{2} \Psi_3^{t_2}(x, y) \Psi_3^{t_2\ast}(x', y')  = &  \frac{1}{4} \bigg[A(x-\delta)A^{\ast}(x'-\delta)\phi^{A}(y)\phi^{A\ast}(y') \\ & + C(x)C^{\ast}(x')\phi^{C}(y)\phi^{C\ast}(y')\bigg] + c.t.
\end{aligned}
\end{equation}
Hence, putting (\ref{impure9}) and (\ref{impure11}) together, we see that the universal impure density matrix  after measurement at $t_2$ is:
\begin{equation}\label{impure12}
\begin{aligned}
    \rho^{t_2} = & \frac{1}{4} \bigg[A(x)A^{\ast}(x')\phi^{A}(y)\phi^{A\ast}(y') \\
    & + A(x-\delta)A^{\ast}(x'-\delta)\phi^{A}(y)\phi^{A\ast}(y') \\
    & + B(x)B^{\ast}(x')\phi^{B}(y)\phi^{B\ast}(y') \\
    & + C(x)C^{\ast}(x')\phi^{C}(y)\phi^{C\ast}(y') \bigg] + c.t.
\end{aligned}
\end{equation}
In this case, we can say that there are \textit{not} four branches for the universal density matrix (\ref{impure12}), but \textit{three}:

\begin{itemize}
    \item A branch with the pointer reading ``$A$,'' represented by that part of the density matrix with the mixed state: 
    \begin{equation*}\begin{aligned}
        & \frac{1}{4} A(x)A^{\ast}(x')\phi^{A}(y)\phi^{A\ast}(y')+ \frac{1}{4}A(x-\delta)A^{\ast}(x'-\delta)\phi^{A}(y)\phi^{A\ast}(y') \\   & = \frac{1}{2}\bigg[\frac{1}{2} A(x)A^{\ast}(x')+\frac{1}{2} A(x-\delta)A^{\ast}(x'-\delta)\bigg]\phi^{A}(y)\phi^{A\ast}(y') 
    \end{aligned}\end{equation*}
    with overall branch weight $\frac{1}{2}$. The subsystem density matrix is a mixed state while the environmental density matrix is a pure state. (Of course, this case involves some idealization, as the environmental density matrix in general is also mixed rather than pure.) %For example, the pure state is macroscopically indistinguishable from the mixed state  $\frac{1}{2}\phi^{A}(y)\phi^{A\ast}(y') + \frac{1}{2}\phi^{A}(y-\delta')\phi^{A\ast}(y'-\delta')$.   
    \item A branch with the pointer reading ``$B$,'' represented by that part of the density matrix with the pure state:
    \begin{equation*}
    \frac{1}{4}B(x)B^{\ast}(x')\phi^{B}(y)\phi^{B\ast}(y')
    \end{equation*}
    with branch weight $\frac{1}{4}$. 
    \item A branch with the pointer reading ``$C$,'' represented by that part of the density matrix with the pure state: \begin{equation*}\frac{1}{4}C(x)C^{\ast}(x')\phi^{C}(y)\phi^{C\ast}(y')\end{equation*}
    with branch weight $\frac{1}{4}$. 
\end{itemize}
Compared to the two multiverses (\ref{impure10} and \ref{impure100}) considered in the previous section, this multiverse is more expansive. It contains all the branches of the previous two. 

 \x{Decoherence is approximate, and so is branching of the Everettian multiverse. They both somewhat depend on how we define macrostates and what counts as macroscopically indistinguishable. For example, with a more fine-grained partition of macrostates, we may as well count the multiverse (\ref{impure12}) as having four branches instead of three.}   While this example is a toy model, we believe it's representative of how branching occurs in a mixed-state multiverse, and is in line with the general Everettian stance on branching and decoherence. 

One worry we want to immediately defuse is the thought that the branching involved in pure states somehow differs from the branching of mixed states, since mixed states can be decomposed non-uniquely into different mixtures of pure states. But that is not a problem for the multiverse -- the set of quasi-classical emergent worlds -- to emerge from a fundamental mixed state. Recall that there's also a preferred basis problem for WFR\textsubscript{E}. There, decoherence is used to justify the choice of a particular basis (modulo the artificial precision of the macrostates and pointer basis).  Given a particular measurement setup, there's usually a well-defined pointer basis. Associated with this well-defined pointer basis is a specific choice of macrostates (e.g. $A$, $B$ or $C$). These determine what the  emergent branches are. The same story about decoherence holds in DMR\textsubscript{E}: nothing in this story essentially depends on there being pure states. There is no more \x{mystery} to branching in DMR\textsubscript{E} than in WFR\textsubscript{E}.

% In general $\rho$ can be decomposed differently into different (and different number of) branches. However, in general, the environment and measuring apparatus will determine the approximate branching structure. For instance, if we were only measuring $\rho$ in terms of $x$-spin, then we should expect the universe to branch into two equally weighted branches corresponding to $\lvert \uparrow_x \rangle$ and $\lvert \downarrow_x \rangle$ respectively, \x{since only in the $x$-measurement pointer basis will the two parts of the quantum states have macroscopically disjoint support.}

% %The same is true for doing experiments in a mixed-state multiverse. For conceptual clarity, consider \y{an example where a system} is in the maximally mixed spin state $I/2$, represented in its diagonalized form as: 

% %\begin{equation*}
% %\rho = \begin{bmatrix}
% %\frac{1}{2} & 0 \\
% %0 & \frac{1}{2} 
% %\end{bmatrix}
% %\end{equation*}

% %\noindent It's true that $\rho$ can generally be decomposed in many different ways. We'll use the bra-ket formalism here (and here on out) for simplicity of presentation and clarity. Let $\lvert \uparrow_x \rangle$ be the `ket' associated with the electron being $x$-spin-up, and $\lvert \downarrow_x \rangle$ be the `ket' associated with the electron being $x$-spin-down. $\lvert \uparrow_z \rangle$ and $\lvert \downarrow_z \rangle$ are associated with the electron having $z$-spin-up and $z$-spin-down respectively. $\rho$ can be decomposed non-uniquely into:

% %\begin{equation}
% %\begin{aligned}
% %     \rho & = \frac{1}{2} \lvert \uparrow_x \rangle \langle \uparrow_x \rvert + \frac{1}{2} \lvert \downarrow_x \rangle \langle \downarrow_x \rvert  \\
% %     & = \frac{1}{2} \lvert \uparrow_z \rangle \langle \uparrow_z \rvert + \frac{1}{2} \lvert  \downarrow_z  \rangle \langle \downarrow_z \rvert \\
%      & = \frac{1}{3}\lvert  \uparrow_z \rangle \langle  \uparrow_z \rvert + \frac{1}{3}\lvert  \downarrow_z  \rangle \langle \downarrow_z \rvert + \frac{1}{6}\lvert \uparrow_x \rangle \langle \uparrow_x \rvert + \frac{1}{6} \lvert \downarrow_x \rangle \langle \downarrow_x \rvert \\
%      & = ...
% \end{aligned}
% \end{equation}

\section{The Born Rule}

\ec{We now turn to the promised task of generalizing standard derivations of the Born rule from WFR\textsubscript{E} to DMR\textsubscript{E}.}

% In this section, we'll focus on three programs -- the Sebens-Carroll program, the McQueen-Vaidman program, and the Deutsch-Wallace program -- from WFR\textsubscript{E} to DMR\textsubscript{E}.

The motivation for deriving the Born rule derives from the fact that WFR\textsubscript{E} runs into a \textit{problem of probability}: given that every branch of the wave function exists, how can one make sense of quantum-mechanical probabilities? The Born rule tells us that the squared-amplitudes associated with each branch should be interpreted as the probabilities of outcomes on that branch. Intuitively, for an outcome to occur with some probability (that isn't 1) is for it to possibly not occur. In WFR\textsubscript{E}, though, \textit{every} branch -- every possible outcome of a measurement -- always obtains. How, then, can we defend the Born rule and the probabilities it prescribes?  

Consider Alice, an experimenter, performing an $x$-spin measurement on an electron prepared in $\lvert\downarrow_z\rangle$, a state of equal superposition of $x$-spin-up and $x$-spin-down:
\begin{equation}
\lvert\downarrow_z\rangle = \frac{1}{\sqrt{2}}\bigg[\lvert\uparrow_x\rangle - \lvert\downarrow_x\rangle\bigg]
\end{equation}
Then, pre-measurement, the universal wave function is in the `ready' state $R$: 
\begin{equation}
    \Psi_R = \lvert\downarrow_z\rangle \lvert R_A \rangle \lvert R_{D_1} \rangle \lvert R_{E} \rangle
\end{equation}
where Alice, $A$, is in the `ready' state $\lvert R_A \rangle$, the measurement device $D_1$ in the state $\lvert R_{D_1} \rangle$ is `ready' to display one of two measurement outcomes $\{\uparrow, \downarrow\}$ , and the environment $E$ (everything else) is `ready' for the measurement by being in the state $\lvert R_{E} \rangle$. After measurement, $\Psi_R$ unitarily evolves into $\Psi_M$:
\begin{equation}
\Psi_M = \frac{1}{\sqrt{2}}\bigg[\lvert\uparrow_x\rangle \lvert {x_\uparrow}_A \rangle \lvert \uparrow_{D_1} \rangle \lvert \uparrow_{E} \rangle - \lvert\downarrow_x\rangle \lvert {x_\downarrow}_A \rangle \lvert \downarrow_{D_1} \rangle \lvert \downarrow_{E} \rangle \bigg]
\end{equation}
with two outcomes with equal amplitudes $\frac{1}{\sqrt{2}}$. The Born rule prescribes that the probability of each outcome occurring is the squared-amplitude (or weight) associated with that outcome, i.e. $(\frac{1}{\sqrt 2})^2 = \frac{1}{2}$. But in WFR\textsubscript{E}, both branches exist -- they're equally real, and (a copy of) Alice exists on each branch, observing both measurement outcomes. How can one make sense of the weights being probabilities?

Much work has been done by defenders of WFR\textsubscript{E} to justify the Born rule. There are three programs we wish to discuss: the Sebens-Carroll program, the McQueen-Vaidman program, and the Deutsch-Wallace program. Each proposes a rational justification of Born rule probabilities by appealing to certain epistemic principles. 

Here, we'll show that \textit{if} we accept any of these programs, as Everettian defenders of Born rule probabilities do, then DMR\textsubscript{E} can accomplish what WFR\textsubscript{E} does: by the Everettian's own lights, DMR\textsubscript{E} provides the same justifications as WFR\textsubscript{E} for using squared-amplitude branch weights as bona fide probabilities and, hence, the Born rule. 

\ec{We emphasize that we're \textit{not} committing ourselves here to any of these programs. They remain debated, and we won't join the fray. (See e.g. \cite{Vaidman2020} for discussion.) We simply note that these programs implicitly assume WFR\textsubscript{E} by assuming that a system's quantum state is given by a pure-state wave function. If this assumption is necessary for defending the Born rule, then DMR\textsubscript{E} is dead in the water: DMR\textsubscript{E} assumes that the possible fundamental quantum states of the universe are not only pure density matrices (associated with pure-state wave functions) but also \textit{impure} density matrices (associated with arbitrary mixtures of pure states).} On the contrary, if these justifications do \textit{not} necessarily depend on pure-state wave functions but generalize to mixed-state density matrices, then DMR\textsubscript{E} seems to be conceptually on a par with WFR\textsubscript{E}. This supports the view that DMR\textsubscript{E} is a viable competitor to WFR\textsubscript{E}. 

Furthermore, we won't discuss the well-known problems of circularity surrounding the use of decoherence in the Everettian justification of probability, as we don't offer any new solutions to them. In any case, these problems will affect WFR\textsubscript{E} just as much as DMR\textsubscript{E}. (See \textcite{Baker2007} and \textcite{kent2010}.)

\subsection{The Sebens-Carroll Program}

\textcite{sebenscarroll2016} proposes a strategy for justifying the Born rule for Alice above: the probabilities ascribed to each branch by the Born rule are to be interpreted as Alice's \textit{self-locating} uncertainty as to which branch they're located in, post-measurement but before they observe the measurement outcome. In their words:
\begin{quote}
    If we assume that the experimenter knows the relevant information about the wave function, it's unclear what the agent might be uncertain of before a measurement is made. They know that every outcome will occur and that they will have a successor who sees each possible result. [...] We must answer the question: What can one assign probabilities to? Our answer will be that agents performing measurements pass through a period of self-locating uncertainty, in which they can assign probabilities to being one of several identical copies, each on a different branch of the wave function. (\cite{sebenscarroll2016}, 33)
\end{quote}
To elaborate, their strategy relies on (i) the fact that Alice knows the universal wave function, (ii) has undergone branching due to some measurement having been performed, but (iii) may not be able to discern which branch they're on \textit{prior to observing the measurement outcome} due to each copy of Alice, post-branching, having qualitatively identical internal states as each other. For Sebens and Carroll, ``two agents are in the same internal qualitative state if they have identical current evidence: the patterns of colors in their visual fields are identical, they recall the same apparent memories, they both feel equally hungry, etc." (2018, 36)

In this `post-measurement pre-observation' period, as they call it, the universal wave function is:
\begin{equation}
\Psi_{P} = \frac{1}{\sqrt{2}}\bigg[\lvert\uparrow_x\rangle \lvert R_{A} \rangle \lvert \uparrow_{D_1} \rangle \lvert \uparrow_{E} \rangle + \lvert\downarrow_x\rangle \lvert R_{A} \rangle \lvert \downarrow_{D_1} \rangle \lvert \downarrow_{E} \rangle \bigg]
\end{equation}
That is, while the measurement has been performed and branching has occurred resulting in two branches associated with $\lvert\uparrow_x\rangle$ and $\lvert\downarrow_x\rangle$ respectively, Alice remains in the `ready' state because they have not observed the measurement outcome yet. They thus have self-locating uncertainty -- subjective credences -- as to which branch of the wave function they might be in.

\textcite{sebenscarroll2016} then proposes an intuitive epistemic principle with which they justify Alice's use of the Born rule, where the probabilities are now interpreted in terms of \textit{subjective credences}:
\begin{quote}
    \textbf{Epistemic Separability Principle (ESP):} Suppose that universe $U$ contains within it a set of subsystems $S$ such that every agent in an internally qualitatively identical state to agent $A$ is located in some subsystem which is an element of $S$. The probability that $A$ ought to assign to being located in a particular subsystem $S$ given that they're in $U$ is identical in any possible universe which also contains subsystems $S$ in the same exact states (and does not contain any copies of the agent in an internally qualitatively identical state that are not located in $S$).
    \begin{equation}
    P(X \mid U) = P(X \mid S)
    \end{equation}
\end{quote}
where $P(A\mid B)$ is the conditional probability of $A$ given $B$. An agent, when ascribing credences to each branch, should restrict attention only to those subsystems containing copies of themselves which are internally qualitatively identical. Given ESP, agents ought to ignore everything outside of those subsystems of concern because they are irrelevant to the agent's consideration of credences. In quantum mechanics, the standard way to do that is to construct a density matrix for e.g. $\Psi_P$ and then trace out the irrelevant degrees of freedom, ending up with the relevant reduced density matrix. Supposing that the subsystem which an agent, such as Alice, cares about is the subsystem containing (copies of) Alice and the $n$ measurement outcomes $O_n$ on $D_1$, then the reduced density matrix of interest for Alice is simply $\rho^{AD_1}$ and the credences should be assigned according to:
\begin{equation}
    P(O_n \mid \Psi_P) = P(O_n \mid \rho^{AD_1})
\end{equation}
While ESP seems tame, it does heavy-lifting in their argument for the Born rule. Crucially, it allows agents to consider different possible multiverses in which the universal wave function could have been different: the environment beyond the subsystems of interest for agents, such as $\rho^{AD_1}$, could have contained different possible configurations which leave the subsystems unchanged. As we'll see, such modal considerations can be used to construct constraints on an agent's self-locating uncertainty about being in a particular branch. This rationally prescribes certain numerical credences for the agent being in a specific branch of the wave function, allowing them to derive the Born rule.

\subsubsection{Generalizing the Sebens-Carroll Program to DMR\textsubscript{E}}

\ec{We'll now show precisely how this strategy, which depends on ESP and self-locating uncertainty, can be generalized to DMR\textsubscript{E}. This demonstrates that their strategy does not depend on WFR\textsubscript{E}; they can be employed by DMR\textsubscript{E} too.} We should also regard the branch weights, such as the $\frac{1}{4}$-weight of the $B$ branch discussed in \S2.3, as self-locating probabilities. 

% We chose the Sebens-Carroll program as it relies on a single simple epistemic principle, ESP, and the generalized result for DMR\textsubscript{E} can be shown in a concrete yet conceptually clear way. Furthermore, to our knowledge, this generalization has not been done before. Later, we'll sketch how the other two programs can be generalized for DMR\textsubscript{E} as well. }

% In what follows, we'll show how the Sebens-Carroll program's epistemic justification for using the Born rule probabilities -- interpreted as subjective self-locating uncertainty -- can be given not just in WFR\textsubscript{E}, but also in DMR\textsubscript{E}, using ESP. We show that the weights associated with each branch of the fundamental universal density matrix ought to be interpreted as the subjective self-locating probability for an observer that they're located in that branch. We first recreate the derivation for the same base case of pure states as Sebens \& Carroll did, but in terms of DMR\textsubscript{E}, then show that the proof can be readily generalized to arbitrary mixed states.}

\subsubsection*{Case 1: Pure states}

Let's start with the simple case of a pure state. Here, the strategy is essentially identical to Seben \& Carroll's for WFR\textsubscript{E} via pure \textit{wave functions}, except it's in terms of DMR\textsubscript{E} via pure \textit{density matrices}. 

An observer, Alice, is about to make a $z$-spin measurement of some subsystem, say, an electron prepared in the $x$-spin-down state $\lvert\downarrow_x\rangle$. $D_1$ is in the state $\lvert R_{D_1} \rangle$, ready to show the measurement outcome. Alice is in the state $\lvert R_A \rangle$, ready to observe the measurement outcome. The rest of the universe, i.e. the environment, is also in the `ready' state $\lvert R_{E} \rangle$. 

Given ESP, Alice's self-locating uncertainty should only depend on the subsystems containing Alice and $D_1$: Alice is considering their self-location uncertainty due to branching, occurring as a result of spin measurement, and $D_1$ is the only subsystem showing the outcome of that measurement. Everything else is irrelevant.  

This means that Alice could also consider a second display, $D_2$, likewise in the ready state, represented by $\lvert R_{D_2} \rangle$. The set-up of $D_2$ is irrelevant to Alice's self-locating uncertainty since Alice will only observe $D_1$, and so they can entertain the possibility of $D_2$ being set up in different configurations without affecting their considerations about self-locating uncertainty concerning measurements on the electron and the outcomes of those measurements displayed on $D_1$.  (Why they would do this will become apparent later.)

Pre-measurement, given the above, the fundamental quantum state of the universe, given by a pure density matrix, can be represented by the `ready' state $\rho_R$: 
\begin{equation}
    \rho_R =    \lvert R_{E} \rangle  \lvert R_{D_2} \rangle \lvert R_{D_1}  \rangle \lvert R_A \rangle \lvert\downarrow_x\rangle \langle\downarrow_x\rvert \langle R_A \rvert \langle R_{D_1} \rvert \langle R_{D_2} \rvert \langle R_{E} \rvert 
\end{equation}
$D_1$ displays the measurement outcome, with two possible outputs $\{\uparrow, \downarrow\}$. That is, it displays $\uparrow$ if the electron was measured to be in the z-spin up direction ($\uparrow_z$), and $\downarrow$ if the electron was measured to be in the z-spin down direction ($\downarrow_z$). Alice can also suppose that $D_2$, too, has two possible outputs $\{\heartsuit, \diamondsuit\}$ correlated in some way with the outcomes of $D_1$. Again, given ESP, the set-up of $D_2$ per se should be irrelevant to their assignment of self-locating uncertainty given their observation of the measurement outcomes given by $D_1$. Two possible set-ups for $D_2$ can be considered:
\begin{itemize}
    \item \textbf{Set-up $\alpha$}: $D_2$ displays $\heartsuit$ if $D_1$ displays $\uparrow$, and $\diamondsuit$ if the $D_1$ displays $\downarrow$. 
    \item \textbf{Set-up $\beta$}: $D_2$ displays $\diamondsuit$ if $D_1$ displays $\uparrow$, and $\heartsuit$ if the $D_1$ displays $\downarrow$.
\end{itemize}
To set up for our next case, and to make the correlations between the various displays clear for each set-up, we can also write the set-ups as per \textbf{Table 1}.

\begin{table}[h]
\centering
\begin{tabular}{| p{2cm}|p{1cm}|p{1cm}||p{1cm}|p{1cm}|  }
\cline{2-5}
\multicolumn{1}{c|}{} & \multicolumn{4}{|c|}{\textbf{Set-up}} \\
\cline{2-5}
\multicolumn{1}{c|}{} & \multicolumn{2}{|c||}{$\alpha$} & \multicolumn{2}{|c|}{$\beta$} \\
\hline
\hfil Electron & \hfil $\uparrow_z$ & \hfil $\downarrow_z$ & \hfil  $\uparrow_z$  & \hfil $\downarrow_z$ \\
\hline
\hfil $D_1$ & \hfil $\uparrow$ & \hfil $\downarrow$ & \hfil $\uparrow$ & \hfil $\downarrow$ \\
\hline
\hfil $D_2$ & \hfil $\heartsuit$ & \hfil $\diamondsuit$ & \hfil $\diamondsuit$ & \hfil $\heartsuit$ \\
\hline
\end{tabular}
\captionsetup{width=8.2cm}
\caption{Two possible set-ups $\alpha$ and $\beta$, with the only difference (for now) being two possible choices of display output set-ups for $D_2$.} 
\label{tab:fig1}
\end{table}
We suppose that Alice has access to the quantum state, the dynamical laws, and can consider these possible set-ups, but is not immediately aware of, nor affected by, the measurement outcome. At this point, they can consider their self-locating uncertainty. Post-measurement pre-observation of $D_1$, Alice can consider one possibility: $\rho_R$, given set-up $\alpha$, unitarily evolves into
\begin{equation}
\begin{aligned}
    \rho_\alpha = & \frac{1}{2} \bigg[      \lvert \uparrow_{\alpha E} \rangle \lvert \heartsuit_{D_2} \rangle   \lvert \uparrow_{D_1} \rangle  \lvert R_A \rangle \lvert\uparrow_z\rangle \langle\uparrow_z\rvert \langle R_A \rvert \langle \uparrow_{D_1} \rvert \langle \heartsuit_{D_2} \rvert \langle \uparrow_{\alpha E} \rvert \\ 
    & + \lvert \downarrow_{\alpha E} \rangle \lvert \diamondsuit_{D_2} \rangle  \lvert \downarrow_{D_1} \rangle  \lvert R_A \rangle  \lvert \downarrow_z\rangle   \lvert  \langle\downarrow_z\rvert \langle R_A \rvert \langle \downarrow_{D_1} \rvert \langle \diamondsuit_{D_2} \rvert \langle \downarrow_{\alpha E} \rvert \bigg]  \y{+ c.t.}
\end{aligned}
\end{equation}
\noindent For notational convenience, we rewrite $\rho_\alpha$ as: 
\begin{equation}\label{rhoalpha}
\begin{aligned}
    \rho_\alpha = & \frac{1}{2} \bigg[ \rho_{\uparrow_z} \rho_{R_A} \rho_{\uparrow_{D_1}} \rho_{\heartsuit_{D_2}} \rho_{\uparrow_{\alpha E}} \\ 
    & + \rho_{\downarrow_z} \rho_{R_A} \rho_{\downarrow_{D_1}} \rho_{\diamondsuit_{D_2}} \rho_{\downarrow_{\alpha E}}  \bigg]  \y{+ c.t.}
\end{aligned}
\end{equation}
\noindent But they could have considered set-up $\beta$ instead. If that were the case, $\rho_R$ would have instead unitarily evolved into 
\begin{equation}\label{rhobeta}
\begin{aligned}
    \rho_\beta = & \frac{1}{2} \bigg[ \rho_{\uparrow_z} \rho_{R_A} \rho_{\uparrow_{D_1}} \rho_{\diamondsuit_{D_2}} \rho_{\uparrow_{\beta E}} \\ 
    & + \rho_{\downarrow_z} \rho_{R_A} \rho_{\downarrow_{D_1}} \rho_{\heartsuit_{D_2}} \rho_{\downarrow_{\beta E}}  \bigg]
     \y{+ c.t.}
\end{aligned}
\end{equation}
To emphasize, these are \textit{not} the only configurations $D_2$ can have, but rather two \textit{possible} set-ups that are available to use for our derivation. We're allowed to consider these configurations since ESP asks Alice to restrict their attention only to the subsystems containing $A$ and $D_1$; possible changes in everything else which do not affect $A$ and $D_1$ per se can be entertained without affecting Alice's considerations about their self-locating uncertainty. 

In particular, note that when we trace out these irrelevant degrees of freedom (by the lights of ESP) from $\rho_\alpha$ and $\rho_\beta$ and restrict attention only to Alice and $D_1$, \ec{decoherence and branching becomes apparent (as per \S2).} The resultant reduced density matrices are equivalent:
\begin{equation}\label{ESPrhoAD1}
\begin{aligned}
    \rho^{AD_1} = \rho_\alpha^{AD_1} & = \rho_\beta^{AD_1} \\
    & \ec{\approx} \frac{1}{2}\bigg[\rho_{R_A} \rho_{\uparrow_{D_1}} + \rho_{R_A} \rho_{\downarrow_{D_1}}\bigg]
\end{aligned} 
\end{equation}
Two branches emerge as a result of decoherence -- associated with definite measurement outcomes for states $\uparrow_z$ and $\downarrow_z$. We can associate each branch of $\rho_{\alpha}$ and $\rho_{\beta}$ with each column of \textbf{Table 1}.
Now, the goal is to show, given ESP, that Alice ought to take each branch of $\rho^{AD_1}$ to be equiprobable: Alice ought to assign credences $P(\uparrow \;\mid \rho_\alpha) = P(\downarrow \;\mid \rho_\alpha) = 1/2$.

Since $\rho_\alpha^{AD_1} = \rho_\beta^{AD_1}$ as per (\ref{ESPrhoAD1}):
\begin{equation}\label{purestateproof1}
    P(\uparrow \; \mid \rho_\alpha) = P(\uparrow \; \mid \rho_\beta)
\end{equation}
We can also use ESP to restrict attention to subsystems containing (different possible) $D_2$ and Alice, if we wanted to consider Alice's self-locating uncertainty over being in a branch correlated with an outcome of $D_2$. By tracing out $D_1$, the electron state, and the rest of the environment, we see that
\begin{equation}\label{ESPrhoAD2}
\begin{aligned}
    \rho^{AD_2} = \rho_\alpha^{AD_2} & = \rho_\beta^{AD_2} \\
    & \ec{\approx} \frac{1}{2}\bigg[\rho_{R_A} \rho_{\heartsuit_{D_2}} + \rho_{R_A} \rho_{\diamondsuit_{D_2}}\bigg]
\end{aligned}
\end{equation}
Hence, given ESP, the credences that Alice ascribes to outcomes of $D_2$, e.g. $\diamondsuit$, in both set-ups $\alpha$ and $\beta$ should also be identical:
\begin{equation}\label{purestateproof2}
    P(\diamondsuit \mid \rho_\alpha) = P(\diamondsuit \mid \rho_\beta)
\end{equation}
Note from \textbf{Table \ref{tab:fig1}}, or from (\ref{rhoalpha}), that the $\downarrow$-branch \textit{just is} the $\diamondsuit$-branch in a universe with the quantum state $\rho_\alpha$, that is, a universe where set-up $\alpha$ was implemented. So Alice's self-locating uncertainty about being in the $\downarrow$-branch must be the same as that for being in the $\diamondsuit$-branch. Alice knows this \textit{same-branch relationship} since, \textit{ex hypothesi}, they have access to the quantum state. Hence, they can use this to conclude that:
\begin{equation}\label{purestateproof3}
    P(\downarrow \; \mid \rho_\alpha) = P(\diamondsuit \mid \rho_\alpha)
\end{equation}
\noindent Likewise, for a universe with the quantum state $\rho_\beta$, Alice can observe that the $\uparrow$-branch \textit{just is} the $\diamondsuit$-branch. Hence:
\begin{equation}\label{purestateproof4}
    P(\uparrow \; \mid \rho_\beta) = P(\diamondsuit \mid \rho_\beta)
\end{equation}
\noindent Therefore, putting (\ref{purestateproof1}), (\ref{purestateproof2}), (\ref{purestateproof3}), and (\ref{purestateproof4}) together, we see that:
\begin{equation}
\begin{aligned}\label{purestateproofresult}
    & 1. \; P(\uparrow \; \mid \rho_\alpha) = P(\uparrow \; \mid \rho_\beta) \;\;\;\;\;\; \text{from } (\ref{purestateproof1})\\
    & 2. \;  P(\uparrow \; \mid \rho_\beta) = P(\diamondsuit \mid \rho_\beta) \;\;\;\;\; \text{from } (\ref{purestateproof4})\\
    & 3. \;  P(\diamondsuit \mid \rho_\beta) = P(\diamondsuit \mid \rho_\alpha) \;\;\;\; \text{from } (\ref{purestateproof2}) \\
    & 4. \;  P(\diamondsuit \mid \rho_\alpha) = P(\downarrow \; \mid \rho_\alpha) \;\;\;\;\; \text{from } (\ref{purestateproof3}) \\
    & \therefore P(\uparrow \; \mid \rho_\alpha) = P(\downarrow \; \mid \rho_\alpha)
\end{aligned}
\end{equation}
From (\ref{purestateproofresult}), we see that considerations of the probabilities prescribed by ESP require Alice to assign equal credences, when considering self-locating uncertainty, to both the $\uparrow$-branch and the $\downarrow$-branch. This uniquely determines their credences for being in either branch to be equal to that branch's weight, $1/2$. 

This thus vindicates the Born rule for DMR\textsubscript{E} for the simple case of \textit{equal-weight} superpositions represented by pure-state density matrices.  \textcite{sebenscarroll2016} has already shown that the strategy works also for \textit{un}equal superpositions. We won't rehearse their argument here, but our discussion of \textbf{Case 2} will be instructive, and our strategy there -- and the general strategy sketched in \S3.1.2 -- will also apply to quantum states in unequal superpositions.

% For $N$ number of equally weighted branches, we simply consider displays $D_1$ and $D_2$ with $N$ different possible outcomes, and use the same sort of considerations as before -- alternating between using ESP and same-branch relationships -- to consider two different set-ups -- two different correlations between the outcomes of $D_1$ and $D_2$ -- via ESP. 

\subsubsection*{Case 2: Mixed states}

Suppose instead a mixture of two pure density matrices. Would Sebens \& Carroll's proof work then? To our knowledge, no one has proven that their derivation of the Born rule generalizes to the case of mixed states and hence to DMR\textsubscript{E}. Here, we establish exactly this claim, given our account of mixed state decoherence in \S2.3. We show this explicitly for one case, and provide an algorithm for generalizing this to arbitrary density matrices. 

Let $\rho_{\downarrow_z}$ be the pure density matrix representing an electron in the $\downarrow_z$ state, and let $\rho_{\downarrow_x}$ be the pure density matrix representing an electron in the $\downarrow_x$ state. Then, suppose Alice is in a universe in the mixed state: 
\begin{equation}
    \rho_{R^\prime} = \frac{1}{2}\bigg(\rho_{\downarrow_z} \rho_{R_A} \rho_{R_{D_1}} \rho_{R_{E}} + \rho_{\downarrow_x} \rho_{R_A} \rho_{R_{D_1}} \rho_{R_{E}}\bigg)
\end{equation}
Depending on the environment, especially the measurement device being used, $\rho_{R^\prime}$ will evolve differently given DMR\textsubscript{E}, just as with WFR\textsubscript{E}. 

Suppose we made a measurement for $x$-spin.\footnote{We could also have made a measurement for $z$-spin, in which case completely analogous results follow: the density matrix gives rise to a different branching structure.} Then $\rho_{R^\prime}$ will unitarily evolve into: \begin{equation}
    \rho_{M^\prime} = \frac{1}{4}\rho_{\uparrow_x} \rho_{R_A} \rho_{\uparrow_{D_1}} \rho_{\uparrow_{E}} +  \frac{3}{4}\rho_{\downarrow_x} \rho_{R_A} \rho_{\downarrow_{D_1}} \rho_{\downarrow_{E}} \ec{+ c.t.}
\end{equation}
Now, to determine Alice's self-locating uncertainty over the possible branches of $\rho_{M^\prime}$, we consider two possible scenarios, $\mu$ and $\nu$, in which additional displays in the environment, $D_2$ with associated outputs $\{\heartsuit, \diamondsuit\}$, $D_3$ with $\{\clubsuit, \spadesuit\}$, and $D_4$ with $\{\cross, \star\}$, may display results:
\begin{table}[h]
\centering
\begin{tabular}{|p{2cm}|p{0.5cm}|p{0.5cm}|p{0.5cm}|p{0.5cm}||p{0.5cm}|p{0.5cm}|p{0.5cm}|p{0.5cm}|  }
\cline{2-9}
\multicolumn{1}{c|}{} & \multicolumn{8}{|c|}{\textbf{Set-up}} \\
\cline{2-9}
\multicolumn{1}{c|}{} & \multicolumn{4}{|c||}{$\mu$} & \multicolumn{4}{|c|}{$\nu$} \\
\hline
\hfil Electron & \hfil $\uparrow_x$ & \hfil $\downarrow_x$ & \hfil  $\downarrow_x$  & \hfil $\downarrow_x$ & \hfil $\uparrow_x$ & \hfil $\downarrow_x$ & \hfil  $\downarrow_x$  & \hfil $\downarrow_x$ \\
\hline
\hfil $D_1$ & \hfil $\uparrow$ & \hfil $\downarrow$ & \hfil $\downarrow$ & \hfil $\downarrow$ & \hfil $\uparrow$ & \hfil $\downarrow$ & \hfil $\downarrow$ & \hfil $\downarrow$ \\
\hline
\hfil $D_2$ & \hfil $\diamondsuit$ & \hfil $\heartsuit$ & \hfil $\diamondsuit$ & \hfil $\diamondsuit$ & \hfil $\heartsuit$ & \hfil $\diamondsuit$ & \hfil $\diamondsuit$ & \hfil $\diamondsuit$ \\
\hline
\hfil $D_3$ & \hfil $\clubsuit$ & \hfil $\clubsuit$ & \hfil $\spadesuit$ & \hfil $\clubsuit$ & \hfil $\spadesuit$ & \hfil $\clubsuit$ & \hfil $\clubsuit$ & \hfil $\clubsuit$ \\
\hline
\hfil $D_4$ & \hfil $\star$ & \hfil $\star$ & \hfil $\star$ & \hfil $\cross$ & \hfil $\cross$ & \hfil $\star$ & \hfil $\star$ & \hfil $\star$ \\
\hline
\end{tabular}
\captionsetup{width=9.8cm}
\caption{Two possible set-ups $\mu$ and $\nu$, corresponding to two possible choices of display output set-ups for $D_2$, $D_3$, and $D_4$.} 
\label{tab:fig2}
\end{table}

\noindent There are many physically possible ways to achieve the above correlations by performing transformations on the environment (\cite[p.46]{sebenscarroll2016}). For example, someone could conditionally measure a second particle upon observing $D_1$'s display, and $D_2$ could conditionally display the outcome of that measurement instead. For instance, in set-up $\mu$, $D_2$/$D_3$/$D_4$ could display $\diamondsuit$/$\clubsuit$/$\star$ if $D_1$ displays $\uparrow$. If $D_1$ displays $\downarrow$, then the other displays will display the result of some measurement on a second particle which yields three distinct outcomes, only showing $\spadesuit$/$\cross$/$\heartsuit$ on one of the outcomes. On set-up $\nu$, $D_2$, $D_3$ and $D_4$ might just output $\heartsuit$/$\spadesuit$/$\cross$ if $D_1$ displays $\uparrow$, and $\diamondsuit$/$\clubsuit$/$\star$ if $D_1$ displays $\downarrow$. 

As with \textbf{Case 1}, we can associate each column of a set-up in \textbf{Table 2} with a decohered branch. Note that $\uparrow$, $\heartsuit$, $\spadesuit$, and $\cross$ \textit{each uniquely picks out a branch} in set-up $\mu$, and that these four symbols also \textit{all pick out the same branch} in $\nu$. As with \textbf{Case 1}, we'll use these facts to derive the Born rule probabilities from the universal density matrix.

Corresponding to each possible set-up, $\rho_{M^\prime}$ could have unitarily evolved into two possible states: 
\begin{equation}
\begin{aligned}
    \rho_\mu &  =  \frac{1}{4}  \rho_{\uparrow_x} \rho_{R_A} \rho_{\uparrow_{D_1}} \rho_{\diamondsuit_{D_2}} \rho_{\clubsuit_{D_3}} \rho_{\star_{D_4}} \rho_{\mu1_E} \\
    & + \frac{1}{4} \rho_{\downarrow_x} \rho_{R_A} \rho_{\downarrow_{D_1}} \rho_{\heartsuit_{D_2}} \rho_{\clubsuit_{D_3}} \rho_{\star_{D_4}} \rho_{\mu2_E} \\
    & + \frac{1}{4} \rho_{\downarrow_x} \rho_{R_A} \rho_{\downarrow_{D_1}} \rho_{\diamondsuit_{D_2}} \rho_{\spadesuit_{D_3}} \rho_{\star_{D_4}} \rho_{\mu3_E} \\
    & + \frac{1}{4} \rho_{\downarrow_x} \rho_{R_A} \rho_{\downarrow_{D_1}} \rho_{\diamondsuit_{D_2}} \rho_{\clubsuit_{D_3}} \rho_{\cross_{D_4}} \rho_{\mu4_E} \\
    & \ec{+ c.t.}
\end{aligned}
\end{equation}
Or:
\begin{equation}
\begin{aligned}
    \rho_\nu & =  \frac{1}{4} \rho_{\uparrow_x} \rho_{R_A} \rho_{\uparrow_{D_1}} \rho_{\heartsuit_{D_2}} \rho_{\spadesuit_{D_3}} \rho_{\cross_{D_4}} \rho_{\mu1_E} \\
    & + \frac{1}{4} \rho_{\downarrow_x} \rho_{R_A} \rho_{\downarrow_{D_1}} \rho_{\diamondsuit_{D_2}} \rho_{\clubsuit_{D_3}} \rho_{\star_{D_4}} \rho_{\mu2_E} \\
    & + \frac{1}{4} \rho_{\downarrow_x} \rho_{R_A} \rho_{\downarrow_{D_1}} \rho_{\diamondsuit_{D_2}} \rho_{\clubsuit_{D_3}} \rho_{\star_{D_4}} \rho_{\mu3_E} \\
    & + \frac{1}{4} \rho_{\downarrow_x} \rho_{R_A} \rho_{\downarrow_{D_1}} \rho_{\diamondsuit_{D_2}} \rho_{\clubsuit_{D_3}} \rho_{\star_{D_4}} \rho_{\mu4_E} \\
    & \ec{+ c.t.}
\end{aligned}
\end{equation}
We're now able to derive our main result. Using ESP again, we can see that:
\begin{equation}\label{eq:47}
    P(\uparrow \; \mid \mu) =  P(\uparrow \; \mid \nu)
\end{equation}
\begin{equation}\label{eq:48}
    P(\heartsuit\mid \mu) =  P(\heartsuit \mid \nu)
\end{equation}
Furthermore, by scrutinizing $\rho_\mu$ and $\rho_\nu$, or by consulting \textbf{Table 2}, we see that the $\uparrow$-branch just is the $\heartsuit$-branch in $\rho_\nu$ and hence:
\begin{equation}\label{eq:49}
    P(\heartsuit \mid \nu) =  P(\uparrow \; \mid \nu)
\end{equation}
So now we've established the equivalence of probabilities for the $\uparrow$-branch and $\heartsuit$-branch in $\rho_\mu$. We do the same for the two remaining branches using the same strategy of using ESP and consulting \textbf{Table 2} to observe same-branch relationships between the symbols. We end up with: \begin{equation}\label{eq:50}
   P(\spadesuit \mid \mu) =  P(\spadesuit \mid \nu) = P(\uparrow \; \mid \nu)
\end{equation}
\begin{equation}\label{eq:51}
    P(\cross \mid \mu) =  P(\cross \mid \nu) = P(\uparrow \; \mid \nu)
\end{equation}
Hence, from \eqref{eq:47} to \eqref{eq:51}: 
\begin{equation}\label{eq:52}
    P(\uparrow \; \mid \mu) = P(\heartsuit \mid \mu) = P(\spadesuit \mid \mu) = P(\cross \mid \mu)
\end{equation}
Since the $\uparrow$-branch, $\heartsuit$-branch, $\spadesuit$-branch, and $\cross$-branch exhaust the branches of $\rho_\mu$ and are mutually exclusive after decoherence, and since they're equiprobable from \eqref{eq:52}, Alice should assign equal credences to being in any of the branches. That is: 
\begin{equation}
    P(\uparrow \; \mid \mu) = P(\heartsuit \mid \mu) = P(\spadesuit \mid \mu) = P(\cross \mid \mu) = \frac{1}{4}
\end{equation}
Consulting \textbf{Table 2} again reveals that the $\heartsuit$-branch, $\spadesuit$-branch, and $\cross$-branch are all $\downarrow$-branches. Since they're approximately mutually exclusive (per decoherence) and exhaust all possible branches in which $\downarrow$ shows up in $\rho_\mu$, we have: 
\begin{equation}
    P(\downarrow \; \mid \mu) = (P(\heartsuit \mid \mu) + P(\spadesuit \mid \mu) + P(\cross \mid \mu)
\end{equation}
Hence, if Alice ought to assign equal credences of $1/4$ each to $P(\heartsuit \mid \mu)$, $P(\spadesuit \mid \mu)$, and $P(\cross \mid \mu)$, then:
\begin{equation}
    P(\downarrow \; \mid \mu) = \frac{3}{4}
\end{equation}
But since Alice rationally ought to assign 1/4 to $P(\uparrow \; \mid \mu)$ and 3/4 to $P(\downarrow \; \mid \mu)$, Alice rationally ought to follow the Born rule! So we've provided a rational justification for the Born rule in DMR\textsubscript{E} for mixed states, as promised. $\square$

\subsubsection{General Strategy}

Sebens and Carroll's strategy exploits the fact that there are many physically possible and convenient set-ups, i.e. possible environments, such that one can write down the quantum state as a sum of equal-amplitude, and hence equally weighted, branches. They discuss this general proof obliquely in the appendix section, but we think their strategy can be explicated much clearly, especially through the sort of schematic tables we've used thus far. 

For a quantum state that one wishes to split into $N$ equal-amplitude branches, one considers, beyond $D_0$ which displays the original measurement outcome, $N$ further displays in the environment, each of which displays two outputs $\{\star_N, \star_N^\prime\}$. The agent then considers cases in which the $\star_N^\prime$ symbols only show up \textit{once} for the $N^{th}$ display. Per ESP, one may unitarily transform these displays (and systems whose measurement outcomes they represent) in many physically possible ways without affecting one's self-location uncertainty regarding the subsystem containing the agent and $D_0$. Then, the Sebens-Carroll strategy for setting up the two set-ups can be made more explicit in terms of two simple strategies:
\begin{itemize}
    \item \textbf{Diagonalization}: the first set-up diagonalizes, for the $N^{th}$ display, the $\star_N^\prime$ symbol (as seen in \textbf{Table 3} as set-up $\alpha$).
    \item \textbf{Same-Branch}: the second set-up considers a possibility in which the $\star_N^\prime$ symbols all show up in the same branch (seen in \textbf{Table 3} as set-up $\beta$). 
\end{itemize}
Suppose Alice is considering a measurement on some system with $k$ unequally weighted outcomes, $O_k$, such that $D_1$ displays $\{``1", ``2", ... ``k"$\}. Given this set-up, one can consider a schematic table as per \textbf{Table 3}.
\begin{table}[h]
\centering
\begin{tabular}{|p{1.4cm}|p{0.6cm}|p{0.6cm}|p{0.6cm}|p{0.6cm}|p{0.6cm}||p{0.6cm}|p{0.6cm}|p{0.6cm}|p{0.6cm}|p{0.6cm}|}
\cline{2-11}
\multicolumn{1}{c|}{} & \multicolumn{10}{|c|}{\textbf{Set-up}} \\
\cline{2-11}
\multicolumn{1}{c|}{} & \multicolumn{5}{|c||}{$\alpha$} & \multicolumn{5}{|c|}{$\beta$} \\
\hline
\hfil System & \hfil $O_1$ & \hfil $O_1$ & \hfil $O_2$  & \hfil \ldots & \hfil $O_k$ & \hfil $O_1$ & \hfil $O_1$ & \hfil $O_2$  & \hfil \ldots & \hfil $O_k$ \\
\hline
\hfil $D_0$ & \hfil $``1"$ & \hfil $``1"$ & \hfil $``2"$  & \hfil \ldots & \hfil $``k"$ & \hfil $``1"$ & \hfil $``1"$ & \hfil $``2"$  & \hfil \ldots & \hfil $``k"$ \\
\hline
\hfil $D_1$ & \hfil \cellcolor{gray!10}$\star_1^\prime$ & \hfil $\star_1$ & \hfil $\star_1$  & \hfil $\star_1$ & \hfil $\star_1$ & \hfil \cellcolor{gray!10}$\star_1^\prime$ & \hfil $\star_1$ & \hfil $\star_1$ & \hfil $\star_1$ & \hfil $\star_1$\\
\hline
\hfil \ldots & \hfil $\star_2$ & \hfil \cellcolor{gray!10}$\star_2^\prime$  & \hfil $\star_2$  & \hfil $\star_2$ & \hfil $\star_2$ & \cellcolor{gray!10}\hfil $\star_2^\prime$ & \hfil $\star_2$ & \hfil$\star_2$ & \hfil$\star_2$ & \hfil $\star_2$\\
\hline
\hfil \ldots & \hfil \ldots & \hfil \ldots & \hfil \cellcolor{gray!10}\ldots & \hfil \ldots & \hfil\ldots & \hfil \cellcolor{gray!10}\ldots & \hfil \ldots & \hfil \ldots & \hfil \ldots & \hfil \ldots\\
\hline 
\hfil \ldots & \hfil  \ldots  & \hfil \ldots  & \hfil \ldots  & \hfil \cellcolor{gray!10}\ldots  & \hfil \ldots  & \hfil \cellcolor{gray!10}\ldots  & \hfil \ldots & \hfil \ldots  & \hfil \ldots  & \hfil \ldots \\
\hline
\hfil $D_{N}$ & \hfil $\star_{N}$ & \hfil $\star_{N}$ & \hfil $\star_{N}$ & \hfil $\star_{N}$ & \hfil \cellcolor{gray!10} $\star_N^\prime$ & \hfil \cellcolor{gray!10} $\star_N^\prime$ & \hfil $\star_{N}$ & \hfil $\star_{N}$ & \hfil $\star_{N}$ & \hfil $\star_{N}$ \\
\hline
\end{tabular}
\captionsetup{width=11cm}
\caption{Two possible set-ups $\alpha$ and $\beta$, with two possible choices of display output set-ups for each $D_1, D_2, ...,D_N$, with two possible outputs $\star_N$ and $\star^\prime_N$ each. Generally there can be many equal-amplitude branches with the same outcomes (e.g. $O_1$). } 
\label{tab:fig5}
\end{table}

Again, one can generically treat each column of each set-up as an equal-amplitude weighted branch of some possible universal quantum state with set-up $\alpha$ or $\beta$ respectively. Then, one simply uses ESP to judge that, for each $N$:
\begin{equation}
\begin{aligned}
    P(\star_N^\prime \mid \alpha) = P(\star_N^\prime \mid \beta) 
\end{aligned}
\end{equation}
and consider the same-branch relationships between the symbols in $\rho_\beta$:
\begin{equation}
    P(\star_1^\prime \mid \beta) = P(\star_2^\prime \mid \beta) = ... = P(\star_N^\prime \mid \beta)
\end{equation}
This straightforwardly entails 
\begin{equation}
    P(\star_1^\prime \mid \alpha) = P(\star_2^\prime \mid \alpha) = ... = P(\star_N^\prime \mid \alpha)
\end{equation}
which entails that an agent ought to assign equal credences that they might be located on each of the $N^{th}$ equally weighted branches. But this just is the Born rule! (One simple last step involves counting how many branches correspond to the outcomes $O_k$ of interest, and summing up each of the 1/$N$ credences for each branch, in order to get an agent's rational self-locating uncertainty about whether they're in the $O_k$ branch.)

We note that this just is the strategy provided by \textcite{sebenscarroll2016}. However, we hope to have made the reasoning behind the set-ups conceptually clearer by explicitly stating \textbf{Diagonalization} and \textbf{Same-Branch} as the principles for choosing the appropriate set-ups $\alpha$ and $\beta$. We also hope to have shown that the Sebens-Carroll program readily generalizes to DMR\textsubscript{E} without issue. DMR\textsubscript{E} can thus employ the same arguments as WFR\textsubscript{E} for justifying the Born rule in EQM. 

\subsection{The McQueen-Vaidman Program}

Similar to the Sebens-Carroll program, \textcite{mcqueenvaidman2018} also proposes an interpretation of the Born rule in terms of self-locating uncertainty. \x{This follows earlier attempts initiated by e.g. \textcite{Vaidman1998} and \textcite{Tappenden2011}.} McQueen and Vaidman's setup depends on the fiction of a sleeping pill, which induces the same post-measurement pre-observation uncertainty as the Sebens-Carroll program:
\begin{quote}
    The experimenter performs the experiment without looking at the result; she instead arranges to be put to sleep with a sleeping pill and taken to room $A$ if the result was $a$, and room $B$ if the result was $b$. The rooms are identical from the inside. So when each of the experimenter's descendants [post-branching copies] wakes up, they will be uncertain as to which room they're in, and therefore uncertain as to which result, $a$ or $b$, obtains in their own world. The question: What is the probability for result $a$? makes sense for them. It's not a question about what happened, it's a question about their self-location. The descendants might know everything relevant regarding the wavefunction of the universe, but still be ignorant about who they are. The descendants are in states of self-location uncertainty. (2018, 2)
\end{quote}
However, instead of relying on ESP, they rely on three physical principles:
\begin{itemize}
    \item \textbf{Symmetry:} Symmetric situations should be assigned equal probabilities.
    \item \textbf{No-FTL:} Faster-than-light signaling is impossible; the probability of finding a particle in some location with some state cannot be influenced by actions occurring remotely. 
    \item \textbf{Locality:} The probability of finding a particle somewhere in some state depends only on that particle's quantum state. 
\end{itemize}
The idea is simple. Start off with the base case of perfect symmetry. Consider a particle described by a subsystem wave function with an equal superposition of $N$ very well-localized and remote wave-packets $\lvert L_N \rangle$ each corresponding to the particle being found at the $N$\textsuperscript{th} location. $N$ identical measurement apparatus are set up at each of $N$ identically built space-stations, each containing an agent $A_N$ which, for all practical purposes, are identical to one another. Each space-station is located on the circumference of a perfect circle such that the particle has $N$-fold spherical symmetry. For instance, if $N = 3$, then the particle's subsystem wave function is described by:
\begin{equation}\label{symmetric-vaidman}
    \Psi_S = \frac{1}{\sqrt{3}}\bigg[\lvert L_1 \rangle + \lvert L_2 \rangle + \lvert L_3 \rangle \bigg]
\end{equation}
Given this situation, each agent -- well aware of the symmetry of the situation -- is put in a sleeping pill situation: they're put to sleep before measurement in a `ready' room, and then moved to a `found' room -- stipulated to be internally identical as the `ready' room -- if the particle is found by the measurement apparatus in their space-station. The measurement then takes place.

Now, given WFR\textsubscript{E}, the agent branches into two copies, one remaining in the `ready' room, and one moving to the `found' room. What is the self-locating uncertainty they should ascribe to being in the `found' room? Agent $A_1$ knows that if they're in the `found' room, then the other agents ($A_2$, $A_3$...) are in the `ready' room. But they also know this is true for each other agent: if $A_2$ is in the `found' room, then the others are in the `ready' room, and likewise for $A_3$, $A_4$... $A_N$. This exhausts all the possibilities given the form of the particle's wave function. Since each outcome is symmetric given the set-up, under \textbf{Symmetry}, each agent should rationally assign each outcome the same credence. The unique way to assign each outcome a probability is to assign each outcome $\frac{1}{N}$. This corresponds to the squared-amplitude weights of each outcome, and so vindicates the Born rule for this specific symmetric case.  

For asymmetric cases, we keep the same symmetric set-up as before. However,  \textbf{No-FTL} and \textbf{Locality} ensures that changes to the wave-packets at $L_2$, $L_3$, ... of a wave function with $N$ remote wave-packets do not influence the wave-packet at $L_1$ in terms of the credences $A_1$ ought to assign to their local measurement outcomes. Then, even if the wave function in question evolves from e.g. (\ref{symmetric-vaidman}) to: 
\begin{equation}
    \frac{1}{\sqrt{3}}\lvert L_1 \rangle + \frac{2}{\sqrt{3}}\lvert ? \rangle
\end{equation}
by e.g. unitarily transforming the $ \lvert L_2 \rangle$ and $\lvert L_3 \rangle $ wave packets into some arbitrary state $\lvert ? \rangle$, the agent $A_1$ at $L_1$ should assign probabilities \textit{as though} they were in the symmetric case since they have no access to the information that the transformation took place. That is, they should assign 1/3 to the outcome that they find the particle in $L_1$ (and hence end up in the `found' room after awakening from the sleeping pill). Note that this is very similar in spirit to ESP: the agent restricts attention to local matters of facts and assigns credences based on that.

The McQueen-Vaidman general strategy for justifying the Born rule for asymmetric cases, then, is to entertain the possibility that any asymmetric state can be obtained from a symmetric state of something like (\ref{symmetric-vaidman}) and then impose \textbf{No-FTL} and \textbf{Locality} so that an agent $A_N$ at $L_N$ can treat their situation in both asymmetric and symmetric cases the same way, by using \textbf{Symmetry}. (For a more thorough statement, see \textcite{mcqueenvaidman2018}).

\subsubsection{Generalizing the McQueen-Vaidman Program to DMR\textsubscript{E}}

\ec{Like the Sebens-Carroll program, we believe that the McQueen-Vaidman program does not require WFR\textsubscript{E}.} 

To begin, the principles of \textbf{Symmetry}, \textbf{No-FTL}, and \textbf{Locality} do not turn on the quantum state's purity. Furthermore, it seems to us that an agent living in a mixed-state multiverse will equally be able to entertain questions about their self-locating uncertainty with the same strategy proposed by McQueen \& Vaidman.

Here's how the story will go for DMR\textsubscript{E} for the simplest base case of perfect symmetry. We work out the $N = 3$ case as McQueen and Vaidman (2018) does. Consider, again, $3$ identical measurement apparatus, $M_1$, $M_2$, $M_3$, set up at each of $3$ identically built space-stations, each containing an agent $A_1$, $A_2$, $A_3$ respectively, which, for all practical purposes, are identical to one another. Each space-station is located at locations $1$, $2$, or $3$, on the circumference of a perfect circle such that the particle has $3$-fold spherical symmetry. The agents, measurement devices, and environment $E$ are all in the ready state $R$. There is a particle described by a (reduced) density matrix constructed from an equal superposition of $3$ very well-localized and remote wave-packets $\lvert L_1 \rangle$, $\lvert L_2 \rangle$, and $\lvert L_3 \rangle$, corresponding to the particle being found at locations 1, 2, and 3 respectively. So, the universal density matrix, in the ready state, $\rho_R$ is described by:
\begin{equation}
    \rho_R = \lvert \psi\rangle \langle \psi \rvert
\end{equation}
\begin{equation}
    \lvert \psi \rangle = \frac{1}{\sqrt{3}}\bigg[\lvert L_1 \rangle + \lvert L_2 \rangle + \lvert L_3 \rangle \bigg] \lvert R_{A_1} \rangle \lvert R_{A_2} \rangle \lvert R_{A_3} \rangle  \lvert R_{M_1} \rangle  \lvert R_{M_2} \rangle \lvert R_{M_3} \rangle \lvert R_{E} \rangle
\end{equation}
Once again, each agent -- well aware of the symmetry of the situation -- is put in a sleeping pill situation: they're put to sleep before measurement in a `ready' room, and then moved to a `found' room -- stipulated to be internally identical as the `ready' room -- if the particle is found by the measurement apparatus in their space-station. 

Now, given DMR\textsubscript{E}, measurement-induced decoherence occurs just like in WFR\textsubscript{E}. The post-measurement universal density matrix evolves into:
\begin{equation}
\begin{aligned}
    \rho_P = & \frac{1}{3}\bigg[\rho_{L_1} \rho_{R_{A_1}} \rho_{R_{A_2}} \rho_{R_{A_3}} \rho_{\checkmark_{M_1}} \rho_{R_{M_2}} \rho_{R_{M_3}} \rho_{{E_1}} \\
     & + \rho_{L_2} \rho_{R_{A_1}} \rho_{R_{A_2}} \rho_{R_{A_3}} \rho_{R_{M_1}} \rho_{\checkmark_{M_2}} \rho_{R_{M_3}} \rho_{{E_2}} \\
     & + \rho_{L_3} \rho_{R_{A_1}} \rho_{R_{A_2}} \rho_{R_{A_3}} \rho_{R_{M_1}} \rho_{R_{M_2}} \rho_{R_{\checkmark_3}} \rho_{{E_3}} \bigg] \y{+ c.t.}
\end{aligned}
\end{equation}
\ec{and the reduced density matrix corresponding to the subsystems of interest $S$ is}:
\begin{equation}
\begin{aligned}
    \rho_S \approx & \frac{1}{3}\bigg[\rho_{L_1} \rho_{R_{A_1}} \rho_{R_{A_2}} \rho_{R_{A_3}} \rho_{\checkmark_{M_1}} \rho_{R_{M_2}} \rho_{R_{M_3}} \\
     & + \rho_{L_2} \rho_{R_{A_1}} \rho_{R_{A_2}} \rho_{R_{A_3}} \rho_{R_{M_1}} \rho_{\checkmark_{M_2}} \rho_{R_{M_3}} \\
     & + \rho_{L_3} \rho_{R_{A_1}} \rho_{R_{A_2}} \rho_{R_{A_3}} \rho_{R_{M_1}} \rho_{R_{M_2}} \rho_{R_{\checkmark_3}} \bigg]
\end{aligned}
\end{equation}
Post-measurement pre-observation, each agent, remaining in the R state because they have yet to find out whether they're in the `found' or `ready' room, may ask: What is the self-locating uncertainty they should ascribe to being in the `found' room? Agent $A_1$ knows that if they're in the `found' room, then the other agents ($A_2$, $A_3$...) are in the `ready' room. But they also know this is true for each other agent: if $A_2$ is in the `found' room, then the others are in the `ready' room, and likewise for $A_3$, $A_4$... $A_N$. This exhausts all the possibilities given the form of the particle's wave function. Since each outcome is symmetric given the set-up, under \textbf{Symmetry}, each agent should rationally assign each outcome the same credence. The unique way to assign each outcome a probability is to assign each outcome $1/3$. This corresponds to the squared-amplitude weights of each outcome, and so vindicates the Born rule for the symmetric case.  

Furthermore, each agent may, even in DMR\textsubscript{E}, employ \textbf{No-FTL} and \textbf{Locality}, to entertain the possibility that any asymmetric case can be transformed unitarily (remotely) into the symmetric case without the agent's knowledge -- the same strategy for WFR\textsubscript{E} generalizes to DMR\textsubscript{E} as well. 

In short, we think that the McQueen-Vaidman program does not depend essentially on WFR\textsubscript{E}, and can be readily generalized to DMR\textsubscript{E} without issue. 

\subsection{The Deutsch-Wallace Program}

\ec{Finally, we turn to the Deutsch-Wallace program.} In contrast to the previous two, this program provides a justification of the Born rule in WFR\textsubscript{E} by appealing \textit{not} to self-locating uncertainty, but to rational choice theory. A rational agent betting on outcomes of measurements for some wave function ought to bet in such a way that the credences they have over these outcomes are governed by the squared-amplitudes of the wave function. \textcite{deutsch1999} provided one of the earliest proofs for this result. However, the most refined result is due to \textcite{Wallace2012}, who proves a representation theorem to this effect given certain axioms of rational choice and assumptions about the structure of quantum bets. 
    
The decision problem can be summarized schematically as such: a system's state space -- its Hilbert space -- can be decomposed into various macrostates $\pi$, with their fineness (i.e. size) determined by decoherence and the environment. Any system in a macrostate $\pi$ is compatible with a set of unitary transformations, which are understood as acts on the system by an agent (for instance, measurement). These acts lead to outcomes in the form of the system ending up in different macrostates on different branches as a result of the unitary transformations. Agents are then asked how they would place monetary bets on these outcomes, on which they will collect rewards after the act is performed; that is, agents are asked to state their preferences for bets on these outcomes. Now, the question is this: what credences should agents rationally assign to these outcomes? 

Wallace assumes a set of four `richness' axioms on the structure of the set of possible bets,\footnote{They are called Reward Availability, Branching Availability, Erasure, and Problem Continuity respectively.} as well as a set of six `rationality' axioms on the structure of the agent's rational preferences on pairs of bets.\footnote{They are called Ordering, Diachronic Consistency, Macrostate Indifference, Branching Indifference, State Supervenience, Solution-Continuity, respectively.} The first two are general axioms of rationality, while the latter four are `Everettian' rationality axioms proposed by Wallace. We won't go into detail stating the axioms, except for one (which we'll discuss in the following section). The interested reader is invited to read Wallace (2012, \S5). 

Wallace (2012, 172) proves the following theorem with the above set-up: 
\begin{quote}
    \textbf{Born Rule Theorem}: There is a utility function on the set of rewards, unique up to positive affine transformations, such that one act is preferred to another if and only if its expected utility, calculated with respect to this utility function and to the quantum-mechanical weights of each reward, is higher.
\end{quote}
That is, the rational agent ought to place bets on the outcomes of quantum bets using credences as assigned by the Born rule through the squared-amplitudes associated with each outcome (i.e. branch) of the act. This then vindicates the Born rule in WFR\textsubscript{E} as a matter of rational decision-making (provided one accepts all the axioms employed in Wallace's proof as a matter of rationality).

\subsubsection{Generalizing the Deutsch-Wallace Program to DMR\textsubscript{E}}


With the exception of a richness axiom, nothing in the Deutsch-Wallace program turns on the purity of the universal quantum state or that of agent's branch. In fact, one can rewrite Wallace's axioms and replace any mention of the wave function $\psi$ with a density matrix $\rho$, and use the expectation values of various observables given by the state $\rho$ instead of $\psi$. The exception is the richness axiom called Erasure. It can be stated as follows: 

\begin{description}
\item[Erasure] Given a pair of states $\psi\in E$ and $\phi\in F$ in the same reward, there's an act $U$ available at $E$ and an act $V$ available at $F$ such that $U\psi=V\phi$. 
\end{description}
It's a crucial axiom for proving the Equivalence Lemma on the way to the Born Rule Theorem. We use a simple example from Wallace (2012, pp.172-73) to illustrate this axiom. Suppose we have two acts that lead to two states:

\begin{equation}
\text{A: } \alpha \lvert + \rangle \vert \text{reward} \rangle + \beta \lvert - \rangle \vert \text{no reward} \rangle
\end{equation}

\begin{equation}
\text{B: } \alpha \lvert + \rangle \vert \text{no reward} \rangle + \beta \lvert - \rangle \vert \text{reward} \rangle
\end{equation}
In order to prove that agents should be indifferent between acts A and B since they assign the same weight to $\vert \text{reward} \rangle$ and $\vert \text{no reward} \rangle$ (an instance of the Equivalence Lemma), Wallace appeals to Erasure. By Erasure, the state space for acts is rich enough such that there always exist acts available in the reward branch of A  and the reward branch of B to unitarily transform their quantum states into the same one, and the same is true for the no-reward branches of A and B. These acts produce:

\begin{equation}
\text{A-plus-erasure: } \alpha \lvert 0 \rangle \vert \text{reward} \rangle + \beta \lvert 0' \rangle \vert \text{no reward} \rangle
\end{equation}

\begin{equation}
\text{B-plus-erasure: } \beta \lvert 0 \rangle \vert \text{reward} \rangle + \alpha \lvert 0' \rangle \vert \text{no reward} \rangle
\end{equation}
By Wallace's rationality axioms (Branching Indifference and Diachronic Consistency), the agent's future selves should be indifferent between A-plus-erasure and A, and between B-plus-erasure and B. In the simplest case where $\alpha=\beta$, the quantum states produced by A-plus-erasure and B-plus-erasure are the same, and the agent should be indifferent between them (by State Supervenience). Hence, she should be indifferent between A and B (by Transitivity). 
%At an abstract level, Erasure plays a similar role as the assumption about the richness of the possible setups in the Sebens-Carroll program; they allow us to use symmetry reasoning to derive the Born rule. 

The argument above does not go through on DMR\textsubscript{E}. Erasure, as stated, is false for arbitrary density matrices. Acts, defined by Wallace, must be unitary transformations. Acting on states, such transformations preserve their degree of ``mixedness.'' While this is fine if we only consider pure states (all of which have the same degree of mixedness, namely zero), different density matrices can have different degrees of mixedness.  If we are given a pair consisting in a pure state (like A) and a mixed state, no unitary transformations can map them to the same state. 
The space of available acts, restricted to unitary transformations, will not be rich enough to satisfy Erasure on DMR\textsubscript{E}. We need to modify the argument or the assumptions.  

The problem can be solved in two ways, by suitable revisions of Erasure. First, we may include non-unitary transformations in the space of available acts. Since there will be non-unitary transformations that relate two density matrices of different mixedness, the larger space will provide the needed erasure acts to take two arbitrary quantum states to the same one. The proof for the Equivalence Lemma would go through as before. However, one might worry that making use of non-unitrary transformations violates the spirit of EQM, as the theory assumes that the time-evolution of the quantum state of the multiverse is unitary. Fortunately, the conflict is only apparent, because decision-theoretic acts need not correspond to actual time-evolutions. In the decision-theoretic framework, we're merely considering possible multiverses (one of them may be the actual multiverse). Given the determinism of the fundamental dynamical laws, different multiverses (represented by different quantum states and at some time $t$) must have evolved from different initial conditions, only one of which is actual. EQM only requires that each initial condition $\rho_0$ unitarily evolves into some quantum state $\rho_t$, but it does not require that any two possible quantum states  $\rho_t$ and $\rho'_t$ can be unitarily transformed into the same quantum state $\rho''$. The latter is  not a consequence of the unitary dynamics but an optional property that may fail to hold. 

%In other words, it's unclear to us why acts are restricted to unitary transformations. We have not found any argument for this restriction. It might be due to the emphasis of unitary time evolutions in EQM, but it's unrelated to the connections among possible contemplatable acts.  Without this assumption, there's no contradiction between Erasure and DMR\textsubscript{E}.  
 
Second, if (for whatever reason) we prefer representing acts only as unitary transformations, we have the option to consider the following revised version of Erasure:
\begin{description}
\item[Erasure$_{\rho}$] Given a pair of states $\rho_1\in E$ and $\rho_2\in F$ in the same reward and of the same mixedness, there's an act $U$ available at $E$ and an act $V$ available at $F$ such that $U\rho_1=V\rho_2$. 
\end{description}
 This revision of Erasure resolves the conflict discussed earlier, since density matrices of the same mixedness can be unitarily transformed into the same density matrix. The crucial question now becomes: what is the appropriate measure of mixedness? The standard measure is von Neumann entropy $S_{\text{vN}}(\rho)=-tr(\rho \text{ log } \rho)$. However, there's a more fine-grained measure---the spectrum of the density matrix, i.e. its set of eigenvalues. Let $\sigma(\rho)=\{\lambda_i\}$ be the set of eigenvalues associated with density matrix $\rho$, then  $S_{\text{vN}}(\rho)$ is a function of $\sigma(\rho)$, since $S_{\text{vN}}(\rho)=-\sum_i \lambda_i \text{ log } \lambda_i$ (\cite{nielsen2010quantum}, p.510). The spectrum is the right measure of mixedness for Erasure$_{\rho}$, because unitary transformations preserve eigenvalues. Any two density matrices with the same spectrum can be unitarily transformed into the same density matrix (with that spectrum). The revised version of Erasure, with the requirement of equivalent spectrum, can be satisfied for density matrices.\footnote{Many thanks to David Wallace for discussions about this point.} 

With Erasure$_{\rho}$, we can formulate many Born rule theorems, one for each choice of the spectrum. For example, density matrices with the spectrum consisting of the singleton set of $\{1\}$ are the pure states, and the associated Born rule theorem is the same as proved by Wallace (2012, \S5.7). Moreover, with suitable replacements of $\psi$ with $\rho$, a Born rule theorem holds for every other choice of the spectrum. Hence, in DMR\textsubscript{E}, we can prove the Born rule theorem ``sector by sector'' and eventually cover all the density matrices associated with a Hilbert space.  This establishes a general version of the Born rule theorem covering both mixed states and pure states, for which the original version proved (for pure states) becomes a special case. 

In summary, the Deutsch-Wallace program can be generalized to DMR\textsubscript{E}. The conflict between DMR\textsubscript{E} and Erasure can be addressed in at least two ways.   The second option respects the desire to represent acts with only unitary transformations. For that reason, it may be the more conservative extension of the decision-theoretic strategy.  
 


\section{Discussion}

In the previous sections, we've generalized the standard defenses for the Born rule to DMR\textsubscript{E}. In addition to answering a technical question that has been neglected in the literature, we take our results to have several conceptual implications. 

First, in order to set up the stage for the generalized arguments, we were required to contemplate, without presupposing a universal pure state, the ontological structure of the Everettian multiverse. For EQM to allow both WFR and DMR, the story about decoherence and branching should apply to both without prejudice. As we've seen, that is indeed the case. This leads us to see that the essence of the Everettian story about the emergence of a branching multiverse is not a universal wave function that gives rise to many branches represented by wave functions, but a universal density matrix (which can be pure or mixed) that gives rise to many branches represented by density matrices.  According to the perspective of DMR\textsubscript{E}, a pure-state multiverse is a very special case. 

Second, with DMR\textsubscript{E}, Everettians can explore new theoretical possibilities of DMR. For example, we can consider a unified treatment of `classical' and `quantum' probabilities in EQM. In WFR\textsubscript{E}, there are two sources of probabilities: the quantum probability of finding ourselves in a particular branch \y{(or betting preferences in the decision-theoretic framework)}, associated with the weight of the branch in the multiverse, and the classical probability of the particular multiverse, associated with a density matrix representing our ignorance of the underlying universal pure state. Their justifications are very different. \y{The latter is not understood in terms of self-locating uncertainties or betting preferences. Instead, it may have a statistical mechanical origin, corresponding to a probability distribution over initial universal quantum states, the so-called \textit{Statistical Postulate} (\cite{albert2000time}).}    In DMR\textsubscript{E}, however, the two \textit{can} be reduced to a single notion of probability, that of finding ourselves in a particular branch \y{(or betting preferences given the actual quantum state)}, albeit in a more expansive multiverse. Whichever $\rho$ is used by defenders of WFR\textsubscript{E} to represent their ignorance of the fundamental pure state of the multiverse, defenders of DMR\textsubscript{E} can regard that $\rho$ as the fundamental mixed state. Insofar as classical and quantum probabilities in EQM can be reduced to a single source, they also can be justified in the same way. 

A theory on which we can apply this strategy is the Everettian Wentaculus (\cite{chen2018}, \cite{chen2022strong}). This version of DMR\textsubscript{E} proposes a simple and unique choice of the initial density matrix of the multiverse (as a version of the Past Hypothesis) and regards it as the only nomological possibility. As a matter of physical laws, the history of the Everettian multiverse could not have been different. There is no longer a choice of the fundamental density matrix, beyond the choice of the physical law, because the actual one is nomologically necessary. It is an instance of ``strong determinism.'' Both classical (statistical mechanical) and quantum probabilities can be understood as branch weights of the Everettian Wentaculus multiverse, represented by a mixed-state density matrix. \y{With the possibility of a unified treatment of probabilities (among other things), the generalization from WFR\textsubscript{E} to DMR\textsubscript{E} is theoretically attractive.}\footnote{\y{Two remarks here: (1) \textcite{saunders2021} has proposed that we can understand quantum probabilities in terms of ``branch-counting.'' His considerations are analogous to the counting arguments of Boltzmann in the foundations of statistical mechanics. When applied to the Everettian Wentaculus, Saunders's proposal, if correct, would allow us to justify both classical and quantum probabilities by counting branches in a natural way. (2) For more discussions about the elimination of classical probability in the Wentaculus theories and other theoretical advantages, see \textcite{chen2018valia, chen2018}.  For two other proposals of eliminating the Statistical Postulate, see Albert (2000, \S7) and Wallace (2012, \S9).}  } 


Finally, we've derived the Born rule in DMR\textsubscript{E} in essentially the same ways as in WFR\textsubscript{E}, by appealing to  the same epistemic principles (separability, symmetry, decision theoretic axioms) and metaphysical foundations (decoherence and branching). The two theories are empirically equivalent, not just in a mathematical sense, but also conceptually. They give us the same empirical predictions, not just in terms of equal probabilities of measurement outcomes, but also the same kind of probabilities (self-locating uncertainties or betting preferences).  We suggest that Everettians, by their own lights, should regard DMR\textsubscript{E} as a genuine rival to WFR\textsubscript{E}. Everettians, then, face the question which version of EQM they should accept. What can be the grounds for deciding?  It cannot be settled by experiments because of the empirical equivalence. It cannot be based on the insistence that the universal quantum state must be pure, because that would beg the question. It cannot be based on the incompleteness of justifications for DMR\textsubscript{E}, for the solutions to the ontology problem and the probability problem in WFR\textsubscript{E} readily extend to DMR\textsubscript{E}.\footnote{This adds an interesting wrinkle to the debate about scientific realism and the issues raised by \textcite{wallace2022}. Even if Wallace is right  that EQM is the only way to make sense of why sky is blue, there is a further question about which version of EQM is correct. Quantum mechanics still leads to in-principle empirical underdetermination.}


\section{Conclusion}

We suggest that the Everettian understanding of decoherence and branching, as well as the justifications for the Born rule, apply to both WFR\textsubscript{E} and DMR\textsubscript{E}. \y{Hence, the theoretical benefits of DMR are available on EQM. Another consequence is that} Everettians face a choice between two types of theories, one allowing only pure states for the multiverse and the other allowing mixed states also. The choice will not be based on different understandings of the branching structure or the Born rule, as the Everettian justifications equally apply in both theories, but must involve some other theoretical considerations. In any case, the availability of different versions of EQM is an interesting example of empirical underdetermination. Its implications and possible resolutions are questions we leave for future work. 


\section*{Acknowledgements}
For helpful feedback, we thank Jefferey Barrett, Charles Sebens, Kelvin McQueen, Katie Robertson, Simon Saunders, Tony Short, Karim Th\'ebault, David Wallace, and the participants at the 2023 Workshop on Relational Clocks, Decoherence, and the Arrow of Time at the University of Bristol, and the 2022 California Quantum Interpretation Network Conference at Chapman University. 

% \section*{Appendix A -- Deriving the Born Rule for 3 Equally Weighted Branches of a Pure Density Matrix.}

% Prior to measurement, suppose that the universal quantum state is the density matrix $\rho_R$ associated with the following pure state: 
% \begin{equation}
%     \lvert \psi \rangle = \frac{1}{\sqrt{3}}\bigg[\lvert X \rangle + \lvert Y \rangle + \lvert Z \rangle \bigg]\lvert R_A \rangle \lvert R_{D_1} \rangle \lvert R_{D_2} \rangle \lvert{R_E} \rangle
% \end{equation}
% where $X$, $Y$, and $Z$ are three macroscopically distinct states, such that:
% \begin{equation}
%     \rho_R = \lvert \psi \rangle \langle \psi \rvert
% \end{equation}
% Now, we consider $D_1$ with three possible outputs $\{``X", ``Y", ``Z"\}$, and $D_2$ with three possible outputs $\{\cross, \dagger, \star\}$. Consider two possible set-ups for $D_2$ in \textbf{Table 4}.

% \begin{table}[h]
% \centering
% \begin{tabular}{| p{2cm}|p{1cm}|p{1cm}|p{1cm}||p{1cm}||p{1cm}|p{1cm}|  }
% \cline{2-7}
% \multicolumn{1}{c|}{} & \multicolumn{6}{|c|}{\textbf{Set-up}} \\
% \cline{2-7}
% \multicolumn{1}{c|}{} & \multicolumn{3}{|c||}{$\alpha$} & \multicolumn{3}{|c|}{$\beta$} \\
% \hline
% \hfil System & \hfil $X$ & \hfil $Y$ & \hfil  $Z$  & \hfil $X$ & \hfil $Y$ & \hfil  $Z$ \\
% \hline
% \hfil $D_1$ & \hfil $``X"$ & \hfil $``Y"$ & \hfil  $``Z"$  & \hfil $``X"$ & \hfil $``Y"$ & \hfil  $``Z"$ \\
% \hline
% \hfil $D_2$ & \hfil $\cross$ & \hfil $\dagger$ & \hfil $\star$ & \hfil $\star$ & \hfil $\cross$ & \hfil $\dagger$ \\
% \hline
% \end{tabular}
% \captionsetup{width=11cm}
% \caption{Two possible set-ups $\alpha$ and $\beta$, with two possible choices of display output set-ups for $D_2$, with three possible outputs for $D_1$ and $D_2$.} 
% \label{tab:fig4}
% \end{table}
% \noindent After measurement, after decoherence, as with before, $\rho_R$ unitarily evolves into a state with three equally weighted branches associated with the three macroscopically distinct states $X$, $Y$, and $Z$. As we can already see from \textbf{Table 3}, given set-up $\alpha$:
% \begin{equation}
% \begin{aligned}
%      \rho_\alpha = & \frac{1}{3} \bigg[\rho_X \rho_{R_A} \rho_{``X"_{D_1}} \rho_{\cross_{D_2}} \rho_{X_{\alpha E}} \\
%      & + \rho_Y \rho_{R_A} \rho_{``Y"_{D_1}} \rho_{\dagger_{D_2}} \rho_{Y_{\alpha E}} \\
%      & + \rho_Z \rho_{R_A} \rho_{``Z"_{D_1}} \rho_{\star_{D_2}} \rho_{Z_{\alpha E}} \bigg]
%      \end{aligned}
% \end{equation}
% Given set-up $\beta$:
% \begin{equation}
% \begin{aligned}
%      \rho_\beta = & \frac{1}{3} \bigg[\rho_X \rho_{R_A} \rho_{``X"_{D_1}} \rho_{\star_{D_2}} \rho_{X_{\alpha E}} \\
%      & + \rho_Y \rho_{R_A} \rho_{``Y"_{D_1}} \rho_{\cross_{D_2}} \rho_{Y_{\alpha E}} \\
%      & + \rho_Z \rho_{R_A} \rho_{``Z"_{D_1}} \rho_{\dagger_{D_2}} \rho_{Z_{\alpha E}} \bigg]
%      \end{aligned}
% \end{equation}
% Then the same considerations can proceed as for the case of two equal branches, but now we must show that $P(``X"\mid \rho_\alpha) = P(``Y" \mid \rho_\alpha) = P(``Z" \mid \rho_\alpha) = 1/3$. We can do this by showing pair-wise equivalence. 

% We'll show first that $P(``X"\mid \rho_\alpha) = P(``Z" \mid \rho_\alpha)$. Consider, again by using ESP, that:
% \begin{equation}
%     P(``X"\mid \rho_\alpha) = P(``X"\mid \rho_\beta)
% \end{equation}
% Then, consider by consulting \textbf{Table 3} that the $``X"$-branch just is the $\star$-branch in $\rho_\beta$. Hence:
% \begin{equation}
%     P(``X"\mid \rho_\beta) = P(\star \mid \rho_\beta)
% \end{equation}
% By using ESP, we see that:
% \begin{equation}
%     P(\star \mid \rho_\beta) = P(\star \mid \rho_\alpha)
% \end{equation}
% Finally, we can see, by consulting \textbf{Table 3}, that the $\star$-branch just is the $``Z"$-branch in $\rho_\alpha$, and that: 
% \begin{equation}
%    P(\star \mid \rho_\alpha) = P(``Z"\mid \rho_\alpha)
% \end{equation}
% And hence:
% \begin{equation}
%      P(``X"\mid \rho_\alpha) = P(``Z"\mid \rho_\alpha)
% \end{equation}
% Now we'll show that $P(``X"\mid \rho_\alpha) = P(``Y" \mid \rho_\alpha)$. Again with ESP, we can see that:
% \begin{equation}
%     P(``Y"\mid \rho_\alpha) = P(``Y"\mid \rho_\beta)
% \end{equation}
% From \textbf{Table 3} we can see that:
% \begin{equation}
%     P(``Y"\mid \rho_\beta) = P(\cross \mid \rho_\beta)
% \end{equation}
% Using ESP one last time, we see that:
% \begin{equation}
%     P(\cross \mid \rho_\beta) = P(\cross \mid \rho_\alpha) 
% \end{equation}
% But note, also from \textbf{Table 3}, that the $``X"$-branch just is the $\cross$-branch in $\rho_\alpha$. Hence:
% \begin{equation}
%     P(\cross \mid \rho_\alpha) = P(``X" \mid \rho_\alpha)
% \end{equation}
% Hence it follows that:
% \begin{equation}
%     P(``Y"\mid \rho_\alpha) = P(``X" \mid \rho_\alpha)
% \end{equation}
% Putting the two results together, we see that:
% \begin{equation}
%     P(``X" \mid \rho_\alpha) = P(``Y"\mid \rho_\alpha) = P(``Z"\mid \rho_\alpha)
% \end{equation}
% Since these are all the possibilities -- all the branches -- given the universal density matrix, and they must be equal to each other, an agent considering their self-locating uncertainty should assign equal probabilities to each: 1/3. But this is just the Born rule prescription, and so we have shown what we sought to prove. $\square$

% In general, for a universal density matrix with $N$ equally weighted branches, simply use $N$ distinct possible outputs for both $D_1$ and $D_2$, and perform the above algorithm in order to derive the Born rule probabilities for that quantum state.

% \section*{Appendix A -- A worked-out example: deriving the Born rule for 2 unequally weighted branches of a density matrix}

% Post-measurement pre-observation, suppose that the reduced density matrix for the subsystem containing Alice and a display device $D_0$, displaying the measurement outcome outputs $\{\uparrow, \downarrow\}$, is: 
% \begin{equation}
%     \rho^{AD_1} = \frac{2}{5}\rho_{\uparrow_{D_0}}\rho_{R_A} + \frac{3}{5}\rho_{\downarrow_{D_0}}\rho_{R_A}
% \end{equation}
% Alice considers 5 additional displays with the following outputs:
% \begin{equation}
% \begin{aligned}
%      & D_2 = \{\heartsuit, \diamondsuit\}\\
%      & D_3 = \{\spadesuit, \clubsuit\}\\
%      & D_4 = \{\cross, \star\}\\
%      & D_5 = \{\dagger, \ddagger \} \\
%      & D_6 = \{\bigtriangleup, \bigtriangledown \}
% \end{aligned}
% \end{equation}
% They then considers two cases in which the five displays only displays $\heartsuit, \spadesuit, \cross, \dagger, \bigtriangleup$ respectively on one of the branches and not any others. The following correlations and possible set-ups using the Sebens-Carroll strategy, and hence two possible universal density matrices corresponding to set-ups $\alpha$ and $\beta$, are shown in \textbf{Table 4}.
% \begin{table}[h]
% \centering
% \begin{tabular}{|p{1.5cm}|p{0.5cm}|p{0.5cm}|p{0.5cm}|p{0.5cm}|p{0.5cm}||p{0.5cm}|p{0.5cm}|p{0.5cm}|p{0.5cm}|p{0.5cm}|}
% \cline{2-11}
% \multicolumn{1}{c|}{} & \multicolumn{10}{|c|}{\textbf{Set-up}} \\
% \cline{2-11}
% \multicolumn{1}{c|}{} & \multicolumn{5}{|c||}{$\alpha$} & \multicolumn{5}{|c|}{$\beta$} \\
% \hline
% \hfil System & \hfil $\uparrow_x$ & \hfil $\uparrow_x$ & \hfil $\downarrow_x$  & \hfil $\downarrow_x$ & \hfil $\downarrow_x$ & \hfil $\uparrow_x$ & \hfil $\uparrow_x$ & \hfil $\downarrow_x$ & \hfil $\downarrow_x$ & \hfil $\downarrow_x$ \\
% \hline
% \hfil $D_1$ & \hfil $\uparrow$ & \hfil $\uparrow$ & \hfil $\downarrow$  & \hfil $\downarrow$ & \hfil $\downarrow$ & \hfil $\uparrow$ & \hfil $\uparrow$ & \hfil $\downarrow$& \hfil $\downarrow$ & \hfil $\downarrow$\\
% \hline
% \hfil $D_2$ & \hfil \cellcolor{gray!10}$\heartsuit$ & \hfil $\diamondsuit$ & \hfil $\diamondsuit$  & \hfil $\diamondsuit$ & \hfil $\diamondsuit$ & \hfil \cellcolor{gray!10}$\heartsuit$ & \hfil $\diamondsuit$ & \hfil $\diamondsuit$  & \hfil $\diamondsuit$ & \hfil $\diamondsuit$\\
% \hline
% \hfil $D_3$ & \hfil $\clubsuit$ & \hfil \cellcolor{gray!10}$\spadesuit$  & \hfil $\clubsuit$  & \hfil $\clubsuit$ & \hfil $\clubsuit$ & \cellcolor{gray!10}\hfil $\spadesuit$  & \hfil $\clubsuit$ & \hfil $\clubsuit$  & \hfil $\clubsuit$ & \hfil $\clubsuit$\\
% \hline
% \hfil $D_4$ & \hfil $\star$ & \hfil $\star$ & \hfil \cellcolor{gray!10} $\cross$  & \hfil $\star$ & \hfil $\star$ & \hfil \cellcolor{gray!10}$\cross$ & \hfil $\star$ & \hfil $\star$& \hfil $\star$ & \hfil $\star$\\
% \hline
% \hfil $D_5$ & \hfil $\ddagger$ & \hfil $\ddagger$ & \hfil $\ddagger$  & \hfil \cellcolor{gray!10}$\dagger$ & \hfil $\ddagger$ & \hfil \cellcolor{gray!10}$\dagger$ & \hfil $\ddagger$ & \hfil $\ddagger$ & \hfil $\ddagger$ & \hfil $\ddagger$\\
% \hline
% \hfil $D_6$ & \hfil $\bigtriangledown$ & \hfil $\bigtriangledown$ & \hfil $\bigtriangledown$ & \hfil $\bigtriangledown$ & \hfil \cellcolor{gray!10}$\bigtriangleup$ & \hfil \cellcolor{gray!10}$\bigtriangleup$ & \hfil $\bigtriangledown$ & \hfil $\bigtriangledown$ & \hfil $\bigtriangledown$ & \hfil $\bigtriangledown$ \\
% \hline
% \end{tabular}
% \captionsetup{width=11cm}
% \caption{Two possible set-ups $\alpha$ and $\beta$, with two possible choices of display output set-ups for $D_2-D_6$, with two possible outputs each.} 
% \label{tab:fig5}
% \end{table}

% \noindent Again these set-ups may be assumed to be achieved by some possible transformations to the environment. As we can see here, set-up $\alpha$ and set-up $\beta$ employs the same \textbf{Diagonalization} and \textbf{Same-Branch} strategies respectively as discussed in $\S3.2$. Each column of \textbf{Table 4} may be taken to represent an equal-amplitude weighted branch of the universal density matrix (conditional on set-ups $\alpha$ and $\beta$ respectively). From here, the same strategy of iteratively using ESP and same-branch relationships is applied again in a straightforward fashion. From the following:
% \begin{equation}
%     P(\heartsuit\mid \alpha) =  P(\heartsuit \mid \beta)
% \end{equation}
% \begin{equation}
%     P(\spadesuit\mid \alpha) =  P(\spadesuit \mid \beta)
% \end{equation}
% \begin{equation}
%     P(\cross \mid \alpha) =  P(\cross \mid \beta)
% \end{equation}
% \begin{equation}
%     P(\dagger \mid \alpha) =  P(\dagger \mid \beta)
% \end{equation}
% \begin{equation}
%     P(\bigtriangleup\mid \alpha) =  P(\bigtriangleup \mid \beta)
% \end{equation}
% \begin{equation}
%     P(\heartsuit\mid \beta) =  P(\spadesuit \mid \beta) = P(\cross \mid \beta) =  P(\dagger \mid \beta) = P(\bigtriangleup \mid \beta)
% \end{equation}
% We deduce that:
% \begin{equation}
%     P(\heartsuit\mid \alpha) =  P(\spadesuit \mid \alpha) = P(\cross \mid \alpha) =  P(\dagger \mid \alpha) = P(\bigtriangleup \mid \alpha)
% \end{equation}
% That is, Alice ought to assign equal self-locating uncertainty to being in each branch of $\rho_\alpha$ as every other branch: 1/5. Now, by recognizing that the $\heartsuit$-branch and $\spadesuit$-branch are the $\uparrow$-branch, and the $\cross$-branch, $\dagger$-branch, and $\bigtriangleup$-branch are the $\downarrow$-branch, we can then see that:
% \begin{equation}
% \begin{aligned}
% & P(\uparrow \mid \alpha) = \frac{2}{5}\\
% & P(\downarrow \mid \alpha) = \frac{3}{5}
% \end{aligned}
% \end{equation}
% which is precisely what the Born rule prescribes. $\square$




%\section*{Plan}

%\begin{itemize}
%    \item Introduction
%    \begin{itemize}
 %       \item Emphasize unification of probabilities as self-locating probabilities 
  %      \item Emphasize generalization as a matter of formal proof, plus clarification of Sebens/Carroll approach
   %     \item bypasses appeals to ignorance or lack of information + principle of indifference in justifying probabilities in quantum stat mech
    %\end{itemize}
    
    %\item Motivation
    %\begin{itemize}
     %   \item DMR empirically equivalent to WFR
      %  \item metaphysically possible for universe to begin in a mixed state. 
       % \item Probabilistic interpretation of empirical equivalence (unitary equivalence) -- assign meaning to probability to branch weights
        %\item open question; how to do it? 
        %\item Status: everyone focus on provng born rule from universal pure state
        %\item straightforward way to extend existing results to DMR, also clarify how Sebens proof works
        %\item How the proof works, and why we can generalize
    %\end{itemize}
    
    %\item Conceptual foundations
    %\begin{itemize}
     %   \item DMR - motivation for why DMR
      %  \item emergence of pure states from mixed states 
       % \item Branching and decoherence
        %\begin{itemize}
         %   \item non-unique decomposition of mixed states, just like non-unique decomposition of wave functions
          %  \item no need to decompose W into $\psi$, just into branches of mixed states.
        %\end{itemize}
       % \item ESP and symmetry -- why intuitive
        %\begin{itemize}
         %   \item ESP is more general than assumed
          %  \item suggests ESP is more robust? or vulnerable?
        %\end{itemize}
    %\end{itemize}
    
     % \item Main results
   % \begin{itemize}
   %     \item Sebens-Carroll
   %     \begin{itemize}
    %        \item Start with the pure superposition of $z_{down}$ and $z_{up}$
    %        \item Go to our case, equal mixture of $z_{down}$ and $x_{up}$ -- but unequal weights determined by decoherence and environment, but before observation after measurement
    %        \item mention detectors when setting up the initial mixed state
    %    \end{itemize}
    %    \item McQueen-Vaidman
   %      \item Deutsch?
     %\end{itemize}
    
    % \item Discussions and implications
    % \begin{itemize}
     %    \item Unification of two kinds of probabilities in Everett -- same source -- self-locating uncertainty constrained by epistemic norms (symmetry of ESP)
     %    \item Justifies and simplifies reduced density matrix talk -- no switching between states and density matrices.
     %    \item Fewer reasons for Everettians not to choose W-Everett.
    %     \begin{itemize}
    %         \item On Wentaculus DMR, part of the unificatory strategy
    %         \item disagreement over further unificatory issues
   %      \end{itemize}
   %      \item Validity of the self-locating strategy depends on the extendability to DMR
    %     \item possibility (metaphysical) for universe starting in mixed state, so strategy should apply to it as well
    %     \item empirical equivalence itself requires probabilities to make sense
     %    \item Circularity? same problems. Sebens and Carroll's defense? \item epistemic norms not self-evident?  %problems of probability from possibility issue, maybe less worrisome here because theory guides the weights. 
   %  \end{itemize}
 %\end{itemize}


\printbibliography

  
\end{document}
