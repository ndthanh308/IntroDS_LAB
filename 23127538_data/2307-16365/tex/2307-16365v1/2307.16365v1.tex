\documentclass[10pt]{article}
\usepackage{enumitem}
\usepackage{amsfonts}
\usepackage{amsmath}
\usepackage{graphicx}
\usepackage{epstopdf}
\usepackage{fancyhdr}
\usepackage{amsthm}
\usepackage{geometry}
\geometry{left=2.4cm,right=2.4cm,top=2.4cm,bottom=2.4cm}
\usepackage{subfigure}
\usepackage[figuresright]{rotating}
\usepackage{caption}
\usepackage{bm}
%\usepackage{txfonts}
\usepackage{amssymb}
\usepackage{amscd}
\usepackage{float}
\usepackage{color}
%\usepackage{floatrow}
\usepackage{tabularx}
\usepackage{diagbox}
\usepackage{tabularx}
\usepackage{mathtools}
\usepackage{multicol}
\setlength\columnsep{25pt}
\usepackage{multirow}
\usepackage{makecell}
\usepackage[justification=centering]{caption}
\usepackage{rotfloat}
\usepackage{longtable}
\allowdisplaybreaks[4]
\renewcommand{\theequation}{\thesection.\arabic{equation}}
\renewcommand{\thetable}{\thesection.\arabic{table}}
\renewcommand{\thefigure}{\thesection.\arabic{figure}}
\newtheorem{theorem}{Theorem}[section]
\newtheorem{orollary}{Corollary}[section]
\newtheorem{lemma}{Lemma}[section]
\newtheorem{problem}{Problem}[section]
\newtheorem{assumption}{Assumption}[section]
\newtheorem{remark}{Remark}[section]
\newtheorem{proposition}{Proposition}[section]
\newtheorem{definition}{Definition}[section]
\newtheorem{example}{Example}[section]
\newcounter{nextauthor}
\setcounter{nextauthor}{1}
\def\mathrm{\mbox}
\def\P{\stackrel{o}P}
\numberwithin{remark}{section}
\renewcommand{\theremark}{\thesection.\arabic{remark}}
\renewcommand{\baselinestretch}{1.2}
 \textheight 9.3in
 \textwidth 6.6in
 \topmargin -15mm
 \oddsidemargin 2mm
 \evensidemargin 2mm
 \parskip 1mm
 %\parindent 2em
%\allowdisplaybreaks[4]
\begin{document}
\title{{\bf Consumption and portfolio optimization solvable problems with recursive preferences}\thanks{This work was supported by the National Natural Science Foundation of China (12171339),  the grant from Chongqing Technology and Business University (2356004) and the Fundamental Research Funds for the Central Universities (2682023CX071).}}
\author{Jian-hao Kang$^a$, Zhun Gou$^b$ and Nan-jing Huang$^c$ \thanks{Corresponding author: nanjinghuang@hotmail.com; njhuang@scu.edu.cn}\\
{\small a. School of Mathematics, Southwest Jiaotong University, Chengdu, Sichuan 610031, P.R. China}\\
{\small b. College of Mathematics and Statistics, Chongqing Technology and Business University,}\\
{\small Chongqing 400067, P.R. China}\\
{\small c. Department of Mathematics, Sichuan University, Chengdu, Sichuan 610064, P.R. China} }
\date{}
\maketitle \vspace*{-9mm}
\begin{abstract}
\noindent This paper considers consumption and portfolio optimization problems with recursive preferences in both infinite and finite time regions. Specially, the financial market consists of a risk-free asset and a risky asset that follows a general stochastic volatility process. By using Bellman's dynamic programming principle, the Hamilton-Jacobi-Bellman (HJB) equation is derived for characterizing the optimal consumption-investment strategy and the corresponding value function. Based on the conjecture of the exponential-polynomial form of the value function, we prove that, when the order of the polynomial $n\leq2$,  the HJB equation has an analytical solution if the investor with unit elasticity of intertemporal substitution (EIS) and an approximate solution otherwise.
\\ \ \\
\noindent {\bf Keywords}: Stochastic volatility; consumption and investment; recursive utility; HJB equation.
\\ \ \\
\noindent \textbf{AMS Subject Classification:}  93E20, 91G10, 91G80, 60H30.
\end{abstract}

\section{Introduction}
Since the seminal work by Merton \cite{Merton1971}, the investment problems under the expected utility maximization criterion have obtained much attention from researchers. However, the assumption of constant volatility for risky assets was adopted in this pioneering work. Numerous empirical studies documented that the stochastic volatility model can capture actual situations better than the constant volatility model, such as describing the phenomena of the volatility clustering and the heavy tailedness of return distributions \cite{Cui2017, He2021}. Recently, Cheng and Escobar-Anel \cite{Cheng2023} revealed a largest class of stochastic volatility processes solvable in closed form within expected utility theory for a hyperbolic absolute risk aversion investor. Although the investment strategies under expected utility theory are very appealing, consumption is essential in people's daily lives and the decisions made for consumption would naturally have an effect on the optimality of the investment strategies \cite{Iftimie2023}. Therefore, it is desirable to investigate the investment problems under the stochastic volatility model by taking consumption into consideration \cite{Kang2021, Liu2007, Zhang2016}.

On the other hand, in the category of utility functions, the recursive utility that can separate the relative risk aversion from the EIS has been one of hot choices to capture the investor's consumption and investment preferences \cite{Epstein1989, Kreps1978}. As a continuous time limit of recursive utility, stochastic differential utility (SDU) introduced by Duffine and Epstein \cite{Duffie1992} receives more and more attention. For example, Schroder and Skiadas \cite{Schroder1999} developed the utility gradient (or martingale) approach to considering portfolio and consumption problem with SDU in finite time region. Chacko and Viceira \cite{Chacko2005} applied Bellman's dynamic programming principle to study optimal portfolio choice and consumption under stochastic volatility with SDU in infinite time region. Kraft et al. \cite{Kraft2017} analyzed continuous-time optimal consumption and investment with Epstein-Zin recursive preferences in incomplete markets and finite time region. It is worth mentioning that the utility gradient approach and the dynamic programming approach are two main methods to solve the optimal consumption-investment problems with SDU. Specially, solving the optimal consumption-investment problems with SDU by the dynamic programming approach can be changed into solving the corresponding HJB equations, which may not be tractable under stochastic volatility models.

The present paper is thus devoted to studying consumption and portfolio optimization problems with SDU in a financial market governed by a general stochastic volatility model within both infinite and finite time regions and to providing solvability analysis of these problems by the dynamic programming approach. The main contributions of this paper are threefold. First, a more general model setting is established, which seems more realistic and includes some models in the aforementioned literature \cite{Chacko2005, Duffie1992} as special cases. For instance, speaking of strategies, this paper considers the optimal consumption-investment, while \cite{Cheng2023} only studies the optimal investment. Speaking of the stochastic volatility and investment interval, we adopt a general stochastic volatility model within both infinite and finite investment regions, while \cite{Chacko2005} uses a specific stochastic volatility model only within infinite investment region. Second, the method of changing control is used, which is helpful in revealing solvable models with more complex expressions for the drift and diffusion terms of the risky asset return and makes it convenient to give solvability analysis of the HJB equations under the setting of the general stochastic volatility model at the same time. Third, some new results of the optimal consumption-investment with SDU are obtained. With the conjecture of the exponential-polynomial form of the value function, we verify that, when the order of the polynomial $n\leq2$, the HJB equation has an analytical solution if the investor with unit EIS and an approximate solution otherwise.  We also prove that the conjecture does not work in solving the HJB equation when the order of the polynomial $n>3$.

The rest of this paper is organized as follows. Section 2 introduces the financial market and related assumptions. Section 3 formulates the consumption and portfolio problems with recursive preferences. Section 4 provides the solvability analysis of consumption and portfolio problems. Section 5 gives some conclusions and appendices contain the proofs.

\section{Preliminaries and assumptions}
Let $(\Omega,\mathcal{F},\mathbb{F},\mathbb{P})$ be a complete filtered probability space equipped with a natural filtration $\mathbb{F}=\{\mathcal{F}_{t}\}_{t\geq0}$ satisfying the usual conditions and a physical probability measure $\mathbb{P}$, where $\mathbb{F}$ is generated by Brownian motions $W^{S}_{t}$ and $W^{\nu}_{t}$ that will be defined later.

We consider a frictionless financial market consisting of a risk-free asset and a risky asset that can be traded continuously. Specifically, the price $M_{t}$ of the risk-free asset follows
\begin{equation}\label{eq1}
  \mathrm{d}M_{t}=rM_{t}\mathrm{d}t, \quad M_{0}=1,
\end{equation}
where $r$ is a positive constant that denotes the risk-free interest rate. Moreover, inspired by the studies \cite{Cheng2023, Kraft2017}, we assume that the price $S_{t}$ of the risky asset satisfies the following general model
\begin{equation}\label{eq2}
\frac{\mathrm{d}S_{t}}{S_{t}}=\left[r+\eta(\nu_{t})G(\nu_{t},t)\right]\mathrm{d}t+G(\nu_{t},t)\mathrm{d}W^{S}_{t}, \quad S_{0}=s_{0}>0
\end{equation}
with a state variable $\nu_{t}$ evolving as follows
\begin{equation}\label{eq3}
\mathrm{d}\nu_{t}=m_{1}(\nu_{t})\mathrm{d}t+m_{2}(\nu_{t})\mathrm{d}W^{\nu}_{t}, \quad \nu_{0}=\overline{\nu}>0,
\end{equation}
where $\eta(\nu_{t})$ represents the market price of risk, $G(\nu_{t},t)$ denotes the volatility of $S_{t}$,  and $m_{1}(\nu_{t})$ and $m_{2}(\nu_{t})$ are the drift and volatility of $\nu_{t}$. The correlation between $S_{t}$ and $\nu_{t}$ can be captured by $W^{S}_{t}$ and $W^{\nu}_{t}$ via the parameter $\rho\in[-1,1]$. Therefore, we can write $\mathrm{d}W^{S}_{t}=\rho\mathrm{d}W^{\nu}_{t}+\sqrt{1-\rho^{2}}\mathrm{d}W^{\perp}_{t}$, where $W^{\perp}_{t}$ is another standard Brownian motion independent of $W^{\nu}_{t}$.

Following the work \cite{Cheng2023}, we now make the following assumptions:
\begin{itemize}
  \item[(A.1)] All the coefficients of equations \eqref{eq2} and \eqref{eq3} are progressively measurable with respect to $\{\mathcal{F}_{t}\}_{t\geq0}$.
  \item[(A.2)] To guarantee the uniqueness of the solution to \eqref{eq2}, we assume (see, for example, \cite{Korn2001, Kraft2005})
\begin{equation}\label{eq4}
\int_{0}^{\infty}\left(|\eta(\nu_{t})G(\nu_{t},t)|+G^{2}(\nu_{t},t)\right)\mathrm{d}t<\infty\quad a.s.
\end{equation}
  \item[(A.3)] To obtain the existence of the solution to \eqref{eq3}, the growth condition on the coefficients of \eqref{eq3} holds. That is,
\begin{equation}\label{eq5}
m_{1}^{2}(x_{0})+m_{2}^{2}(x_{0})\leq K(1+x^{2}_{0})
\end{equation}
for $x_{0}\in\mathbb{R}$ and $|x_{0}|\leq K_{1}$ with positive constants $K$ and $K_{1}$.
  \item[(A.4)] To ensure the uniqueness of the solution to \eqref{eq3}, the Yamada-Watanabe condition (see, for example, Theorem 4 in \cite{Yamada1971}) holds. That is, there exist real-valued functions $\widetilde{f}(x)$ and $g(x)$ defined on $[0, K_{2})$ with $ K_{2}>0$ such that
\begin{align}\label{eq6}
&|m_{1}(x_{0})-m_{1}(y_{0})|\leq \widetilde{f}(|x_{0}-y_{0}|), \nonumber\\
&|m_{2}(x_{0})-m_{2}(y_{0})|\leq g(|x_{0}-y_{0}|)
\end{align}
for all $x_{0},y_{0}\in\mathbb{R}$ with $|x_{0}-y_{0}|<K_{2}$, where $\widetilde{f}$ and $g$ are continuous, positive and increasing with $\widetilde{f}(0)=g(0)=0$, and $\widetilde{f}(x)$ and $g^{2}(x)x^{-1}$ are concave and satisfy
\begin{equation}\label{eq7}
\int_{0+}\left[\widetilde{f}(x)+g^{2}(x)x^{-1}\right]^{-1}\mathrm{d}x=\infty.
\end{equation}
\end{itemize}




\section{Problem formulation}
In this section, we consider a rational investor who can invest in the aforementioned financial market. We denote by $\pi_{t}$ the proportion of wealth invested in the risky asset and $(1-\pi_{t})$ the remaining proportion of wealth invested in the risk-free asset. Moreover, the investor can consume at an instantaneous rate $c_{t}$ at time $t$. Therefore, given the initial wealth value $X_{0}$, the wealth dynamics of the investor is described by the following stochastic differential equation (SDE)
\begin{equation}\label{eq8}
\mathrm{d}X_{t}=\left[r+\pi_{t}\eta(\nu_{t})G(\nu_{t},t)\right]X_{t}\mathrm{d}t-c_{t}\mathrm{d}t
+\pi_{t}G(\nu_{t},t)X_{t}\mathrm{d}W^{S}_{t}.
\end{equation}

Now, we assume that the preferences of the investor are captured by recursive utility functions, which are also known as stochastic differential utilities in continuous time. Due to \cite{Chacko2005, Duffie1992}, the investor's preference can be given by
\begin{equation}\label{eq9}
J_{t}=\mathbb{E}_{t}\left[\int^{\infty}_{t}f(c_{s},J_{s})\mathrm{d}s\right].
\end{equation}
Here, $\mathbb{E}_{t}$ represents the $\mathcal{F}_{t}$-conditional expectation with respect to the measure $\mathbb{P}$ and $f(c,J)$ denotes the normalized aggregator of consumption and continuation value that takes the form
\begin{equation}\label{eq10}
f(c,J)=\beta(1-\frac{1}{\phi})^{-1}(1-\gamma)J\left[\left(\frac{c}{((1-\gamma)J)^{\frac{1}{1-\gamma}}}\right)^{1-\frac{1}{\phi}}-1\right],
\end{equation}
where $\beta>0$ denotes the investor's discount rate, $\gamma>0$, $\gamma\neq1$ and $\phi>0$, $\phi\neq1$, and $\gamma$ represents the relative risk aversion coefficient and $\phi$ is the EIS parameter. When $\phi=1$, the aggregator $f(c,J)$ takes the form
\begin{equation}\label{eq11}
f(c,J)=\beta(1-\gamma)J\left[\ln c-\frac{1}{1-\gamma}\ln ((1-\gamma)J)\right].
\end{equation}
If $\phi=\frac{1}{\gamma}$, $f$ takes the form of the power utility, and further takes the form of the logarithmic utility if $\phi=\gamma=1$.

Next, given the wealth dynamics \eqref{eq8} and the recursive preference \eqref{eq10} or \eqref{eq11}, the investor aims to choose a consumption-investment strategy $(c_{t},\pi_{t})$ to maximize the recursive utility
\begin{align}\label{eq12}
\sup\limits_{(c_{t},\pi_{t})\in \widetilde{\Pi}_{1}} \; \mathbb{E}_{t}\left[\int^{\infty}_{t}f(c_{s},J_{s})\mathrm{d}s\right],
\end{align}
where $\widetilde{\Pi}_{1}$ denotes the set of admissible strategies as defined below.
\begin{definition}\label{definition3.1}
A consumption-investment strategy $(c_{t},\pi_{t})$ of \eqref{eq12} is admissible if the following conditions are satisfied:
\begin{itemize}
\item[(i)] $c_{t}$ and $\pi_{t}$ are $\mathcal{F}_{t}$-progressively measurable processes and $c_{t}\geq0$;
\item[(ii)] for any initial value $(t, x, \nu)\in [0,\infty) \times \mathbb{R}^{+} \times \mathbb{R}$, the SDE $\eqref{eq8}$ for $\{X_{s}\}_{s\geq t}$ with $X_{t}=x$ and $\nu_{t}=\nu$ admits a pathwise unique positive solution;
\item[(iii)] the necessary integrability conditions for the conditional expectation $\mathbb{E}_{t}$ in \eqref{eq12} to be well-defined hold.
\end{itemize}
\end{definition}

For the sake of convenience, we denote a new control variable by
\begin{align}\label{eq13}
\psi_{t}=\pi_{t}G(\nu_{t},t)
\end{align}
and the corresponding admissible set by $\Pi_{1}$. Therefore, the wealth process \eqref{eq8} and problem \eqref{eq12} in term of the new control variable can be rewritten as
\begin{align}\label{eq14}
\mathrm{d}X_{t}=\left[r+\eta(\nu_{t})\psi_{t}\right]X_{t}\mathrm{d}t-c_{t}\mathrm{d}t+\psi_{t}X_{t}\mathrm{d}W^{S}_{t}
\end{align}
and
\begin{align}\label{eq15}
\sup\limits_{(c_{t},\psi_{t})\in \Pi_{1}} \; \mathbb{E}_{t}\left[\int^{\infty}_{t}f(c_{s},J_{s})\mathrm{d}s\right],
\end{align}
respectively. It seems that the equation \eqref{eq14} under $\psi_{t}$ looks simpler than \eqref{eq8}. In addition, if the problem \eqref{eq15} is solved in term of $(c_{t}, \psi_{t})$, then we can obtain the solution to the original problem \eqref{eq12}.

When the investment period is a finite interval, we follow \cite{Kraft2017} to rewrite the investor's preference in \eqref{eq9} as
\begin{equation}\label{eq16}
J_{t}=\mathbb{E}_{t}\left[\int^{T}_{t}f(c_{s},J_{s})\mathrm{d}s+U(X_{T})\right],
\end{equation}
where $T>0$ denotes the terminal of investment interval, $U:(0,\infty)\rightarrow\mathbb{R}$ with $U(x)=\varepsilon^{1-\gamma}\frac{x^{1-\gamma}}{1-\gamma}$ is a constant relative risk aversion utility function for bequest, and where $\varepsilon\in(0,\infty)$ is a weight factor. Then the investor wants to adopt the control variable $(c_{t},\psi_{t})$ to maximize the preference \eqref{eq16} as follows
\begin{align}\label{eq17}
\sup\limits_{(c_{t},\psi_{t})\in \Pi_{2}} \; \mathbb{E}_{t}\left[\int^{T}_{t}f(c_{s},J_{s})\mathrm{d}s+U(X_{T})\right].
\end{align}
Here, $\Pi_{2}$ denotes the set of admissible strategies, which can be defined similarly as Definition \ref{definition3.1} by replacing $\pi_{t}$ with $\psi_{t}$ and by changing the conditional expectation $\mathbb{E}_{t}$ in \eqref{eq12} with the conditional expectation $\mathbb{E}_{t}$ in \eqref{eq17}.


\section{Optimal consumption and investment solvable problems}
In this section, we will adopt Bellman's dynamic programming principle to solve problems \eqref{eq15} and \eqref{eq17}, respectively.


\subsection{The case of infinite time-horizon}
 First of all, we consider the case of the infinite time interval. When the investor has unit EIS, which implies the aggregator $f(c,J)$ takes the form \eqref{eq11} in the problem \eqref{eq15}, it follows from \cite{Chacko2005} that the corresponding HJB equation can be given by
\begin{equation}\label{eq18}
\begin{aligned}
\sup\limits_{c\in(0,\infty),\psi\in\mathbb{R}} \;&\left\{\left[r+\psi\eta(\nu)\right]x\omega_{x}-c\omega_{x}+\frac{1}{2}\psi^{2}x^{2}\omega_{xx}+m_{1}(\nu)\omega_{\nu}+\frac{1}{2}m^{2}_{2}(\nu)\omega_{\nu\nu}\right.\\
&\left.\quad+x\psi\rho m_{2}(\nu)\omega_{x\nu}+\beta(1-\gamma)\omega\left[\ln c-\frac{1}{1-\gamma}\ln ((1-\gamma)\omega)\right]\right\}=0,
\end{aligned}
\end{equation}
where $\omega$ denotes the value function of the problem \eqref{eq15} when taking the form \eqref{eq11}, and $\omega_{x}$, $\omega_{\nu}$, $\omega_{xx}$, $\omega_{\nu\nu}$ and $\omega_{x\nu}$ represent the first and second partial derivatives of $\omega$ with respect to $x$ and $\nu$. By differentiating the expression inside the bracket of $\eqref{eq18}$ with respect to $c$ and $\psi$, respectively, one can obtain
\begin{equation}\label{eq19}
   \left\{ \begin{aligned}
   &c^{*}_{t}=\frac{\beta(1-\gamma)\omega}{\omega_{x}},\\
   &\psi^{*}_{t}=-\frac{\eta(\nu)\omega_{x}+\rho m_{2}(\nu)\omega_{x\nu}}{x\omega_{xx}}.
  \end{aligned}\right.
\end{equation}
We make an ansatz that $\omega(x,\nu)=\frac{x^{1-\gamma}}{1-\gamma}h(\nu)^{1-\gamma}$ for some deterministic function $h(\nu)$. Substituting $\eqref{eq19}$ and the ansatz for $\omega(x,\nu)$ into $\eqref{eq18}$ yields the following partial differential equation (PDE)
\begin{align}\label{eq20}
  r&-\beta+\frac{1}{2\gamma}\eta^{2}(\nu)+\beta\left(\ln\beta-\ln h\right)+\left[m_{1}(\nu)+\frac{\eta(\nu)}{\gamma}\rho m_{2}(\nu)(1-\gamma)\right]\frac{h_{\nu}}{h}\nonumber\\
  &+\frac{ 1}{2}m^{2}_{2}(\nu)\left[\frac{\rho^{2}(1-\gamma)^{2}}{\gamma}-\gamma\right]\frac{h^{2}_{\nu}}{h^{2}}+\frac{ 1}{2}m^{2}_{2}(\nu)\frac{h_{\nu\nu}}{h}=0.
\end{align}

In what follows, we will make a general ansatz for $h$ as an exponential-polynomial form and prove that when the order of the polynomial is higher than 2, the problem \eqref{eq15} with $f(c,J)$ given by \eqref{eq11} is unsolvable under this ansatz.
\begin{theorem}\label{theorem4.1}
Consider the problem \eqref{eq15} when $\phi=1$. If $h(\nu)$ is conjectured in the exponential-polynomial form such that
\begin{align}\label{eq21}
\omega(x,\nu)=\frac{x^{1-\gamma}}{1-\gamma}h(\nu)^{1-\gamma}=\frac{x^{1-\gamma}}{1-\gamma}\exp\left\{(1-\gamma)\left(A_{0}+\sum\limits_{k=1}^{n}\frac{1}{k}A_{k}\nu^{k}\right)\right\},
\end{align}
where $A_{k}$ for all $k=0,\cdot\cdot\cdot,n$ are constants, then the above ansatz is not useful in solving the PDE \eqref{eq20} when the order $n$ of the polynomial is higher than 2.
\end{theorem}
\begin{proof}
See Appendix A.
\end{proof}

Next, we consider the case of $\phi\neq1$. That is, the aggregator $f(c,J)$ takes the form \eqref{eq10} in the problem \eqref{eq15}. Similarly, the corresponding HJB equation can be given as follows
\begin{equation}\label{eq24}
\begin{aligned}
\sup\limits_{c\in(0,\infty),\psi\in\mathbb{R}} \;&\left\{\left[r+\psi\eta(\nu)\right]x\omega_{x}-c\omega_{x}+\frac{1}{2}\psi^{2}x^{2}\omega_{xx}+m_{1}(\nu)\omega_{\nu}+\frac{1}{2}m^{2}_{2}(\nu)\omega_{\nu\nu}\right.\\
&\left.\quad+x\psi\rho m_{2}(\nu)\omega_{x\nu}+\beta(1-\frac{1}{\phi})^{-1}(1-\gamma)\omega\left[\left(\frac{c}{((1-\gamma)\omega)^{\frac{1}{1-\gamma}}}\right)^{1-\frac{1}{\phi}}-1\right]\right\}=0
\end{aligned}
\end{equation}
and the candidate optimal consumption-investment strategy satisfies
\begin{equation}\label{eq25}
   \left\{ \begin{aligned}
   &c^{*}_{t}=\left[\frac{\omega_{x}}{\beta(1-\gamma)\omega}((1-\gamma)\omega)^{\frac{1-\frac{1}{\phi}}{1-\gamma}}\right]^{-\phi},\\
   &\psi^{*}_{t}=-\frac{\eta(\nu)\omega_{x}+\rho m_{2}(\nu)\omega_{x\nu}}{x\omega_{xx}}.
  \end{aligned}\right.
\end{equation}
We make an ansatz that $\omega(x,\nu)=\frac{x^{1-\gamma}}{1-\gamma}h(\nu)^{-\frac{1-\gamma}{1-\phi}}$ for some deterministic function $h(\nu)$. By substituting $\eqref{eq25}$ and the above ansatz for $\omega(x,\nu)$ into $\eqref{eq24}$, we can derive that
\begin{align}\label{eq26}
  r&-\beta^{\phi}h^{-1}+\frac{1}{2\gamma}\eta^{2}(\nu)+\frac{\beta\phi}{\phi-1}\left(\beta^{\phi-1}h^{-1}-1\right)-\frac{1}{1-\phi}\left[m_{1}(\nu)+\frac{\eta(\nu)}{\gamma}\rho m_{2}(\nu)(1-\gamma)\right]\frac{h_{\nu}}{h}\nonumber\\
  &+\frac{ 1}{2(1-\phi)^{2}}m^{2}_{2}(\nu)\left[2-\phi-\gamma+\frac{\rho^{2}(1-\gamma)^{2}}{\gamma}\right]\frac{h^{2}_{\nu}}{h^{2}}-\frac{ 1}{2(1-\phi)}m^{2}_{2}(\nu)\frac{h_{\nu\nu}}{h}=0.
\end{align}
Since the equation \eqref{eq26} involves the term $\beta^{\phi}h^{-1}$, it usually does not admit an exact analytical solution. In the following, we will adopt the log-linear approximation method used in \cite{Campbell2004} to obtain an approximate solution. Inserting the conjecture of $\omega(x,\nu)$ into \eqref{eq25} yields
$
c^{*}_{t}=\beta^{\phi}h^{-1}x,
$
which shows that
$$
\beta^{\phi}h^{-1}=\frac{c^{*}_{t}}{x}=\exp\{\ln c^{*}_{t}-\ln x\}:=\exp\{\overline{c}^{*}_{t}-\overline{x}\},
$$
where $\overline{c}^{*}_{t}=\ln c^{*}_{t}$ and $\overline{x}=\ln x$. By the first-order Taylor expansion of $\beta^{\phi}h^{-1}$ around the expected value of the log consumption-wealth ratio $\mathbb{E}(\overline{c}^{*}_{t}-\overline{x})$, one has
\begin{align}\label{eq27}
   \beta^{\phi}h^{-1}\approx&\exp\{\mathbb{E}(\overline{c}^{*}_{t}-\overline{x})\}
   +\exp\{\mathbb{E}(\overline{c}^{*}_{t}-\overline{x})\}\left[\overline{c}^{*}_{t}-\overline{x}-\mathbb{E}(\overline{c}^{*}_{t}-\overline{x})\right]
   =\zeta_{1}+\zeta_{2}(\overline{c}^{*}_{t}-\overline{x}).
\end{align}
Here, $\zeta_{1}=\exp\{\mathbb{E}(\overline{c}^{*}_{t}-\overline{x})\}\left[1-\mathbb{E}(\overline{c}^{*}_{t}-\overline{x})\right]$ and $\zeta_{2}=\exp\{\mathbb{E}(\overline{c}^{*}_{t}-\overline{x})\}$. Moreover, one can obtain that
\begin{align}\label{eq28}
\beta^{\phi}h^{-1}\approx\zeta_{1}+\zeta_{2}(\ln c^{*}_{t}-\ln x)=\zeta_{1}+\zeta_{2}(\phi \ln\beta-\ln h).
\end{align}

By making a general ansatz for $h$ as an exponential-polynomial form, we will show that when the order of the polynomial is higher than 2, the problem \eqref{eq15} with $f(c,J)$ given by \eqref{eq10} is approximately unsolvable under this ansatz.
\begin{theorem}\label{theorem4.2}
Consider the problem \eqref{eq15} when $\phi\neq1$. If $h(\nu)$ is conjectured in the exponential-polynomial form such that
\begin{align}\label{eq29}
\omega(x,\nu)=\frac{x^{1-\gamma}}{1-\gamma}h(\nu)^{-\frac{1-\gamma}{1-\phi}}=\frac{x^{1-\gamma}}{1-\gamma}\exp\left\{-\frac{1-\gamma}{1-\phi}
\left(A_{0}+\sum\limits_{k=1}^{n}\frac{1}{k}A_{k}\nu^{k}\right)\right\},
\end{align}
where $A_{k}$ for all $k=0,\cdot\cdot\cdot,n$ are constants, then the above ansatz is not useful in providing an approximate solution to the PDE \eqref{eq26} when the order $n$ of the polynomial is higher than 2.
\end{theorem}
\begin{proof}
See Appendix B.
\end{proof}

\subsection{The case of finite time-horizon}
Now, we will concentrate on the case of the finite time interval. If the investor has unit EIS, which means the aggregator $f(c,J)$ takes the form \eqref{eq11} in the problem \eqref{eq17}, then it follows from \cite{Kraft2017} that the corresponding HJB equation satisfies
\begin{equation}\label{eq32}
\begin{aligned}
\sup\limits_{c\in(0,\infty),\psi\in\mathbb{R}} \;&\left\{\omega_{t}+\left[r+\psi\eta(\nu)\right]x\omega_{x}-c\omega_{x}+\frac{1}{2}\psi^{2}x^{2}\omega_{xx}+m_{1}(\nu)\omega_{\nu}+\frac{1}{2}m^{2}_{2}(\nu)\omega_{\nu\nu}\right.\\
&\left.\quad+x\psi\rho m_{2}(\nu)\omega_{x\nu}+\beta(1-\gamma)\omega\left[\ln c-\frac{1}{1-\gamma}\ln ((1-\gamma)\omega)\right]\right\}=0
\end{aligned}
\end{equation}
with the boundary condition $\omega(T,x,\nu)=\varepsilon^{1-\gamma}\frac{x^{1-\gamma}}{1-\gamma}$. Solving the maximization problem \eqref{eq32} with respect to $c$ and $\psi$, we have the following candidate optimal consumption-investment strategy
\begin{equation}\label{eq33}
   \left\{ \begin{aligned}
   &c^{*}_{t}=\frac{\beta(1-\gamma)\omega}{\omega_{x}},\\
   &\psi^{*}_{t}=-\frac{\eta(\nu)\omega_{x}+\rho m_{2}(\nu)\omega_{x\nu}}{x\omega_{xx}}.
  \end{aligned}\right.
\end{equation}
Moreover, we conjecture a solution of the form $\omega(t,x,\nu)=\frac{x^{1-\gamma}}{1-\gamma}h(t,\nu)^{1-\gamma}$ for some deterministic function $h(t,\nu)$. Substituting $\eqref{eq33}$ and the ansatz for $\omega(t,x,\nu)$ into $\eqref{eq32}$ yields the following PDE
\begin{align}\label{eq34}
  h_{t}&+\left[r-\beta+\frac{1}{2\gamma}\eta^{2}(\nu)+\beta\left(\ln\beta-\ln h\right)\right]h+\left[m_{1}(\nu)+\frac{\eta(\nu)}{\gamma}\rho m_{2}(\nu)(1-\gamma)\right]h_{\nu}\nonumber\\
  &+\frac{ 1}{2}m^{2}_{2}(\nu)\left[\frac{\rho^{2}(1-\gamma)^{2}}{\gamma}-\gamma\right]\frac{h^{2}_{\nu}}{h}+\frac{ 1}{2}m^{2}_{2}(\nu)h_{\nu\nu}=0.
\end{align}

In the following, we will assume a general ansatz for $h$ as an exponential-polynomial form and show that when the order of the polynomial is higher than 2, the problem \eqref{eq17} with $f(c,J)$ given by \eqref{eq11} is unsolvable under this ansatz.
\begin{theorem}\label{theorem4.3}
Consider the problem \eqref{eq17} when $\phi=1$. If $h(t,\nu)$ is conjectured in the exponential-polynomial form such that
\begin{align}\label{eq35}
\omega(t,x,\nu)=\frac{x^{1-\gamma}}{1-\gamma}h(t,\nu)^{1-\gamma}=\frac{x^{1-\gamma}}{1-\gamma}\exp\left\{(1-\gamma)\left(A_{0}(T-t)+
\sum\limits_{k=1}^{n}\frac{1}{k}A_{k}(T-t)\nu^{k}\right)\right\},
\end{align}
where $A_{k}(t)$ for all $k=0,\cdot\cdot\cdot,n$ are functions of $t$ with $A_{0}(0)=\ln \varepsilon$ and $A_{k}(0)=0\;(k=1,\cdot\cdot\cdot,n)$, then the above ansatz is not useful in solving the PDE \eqref{eq34} when the order $n$ of the polynomial is higher than 2.
\end{theorem}
\begin{proof}
See Appendix C.
\end{proof}


Then, we study the case of $\phi\neq1$. That is, the aggregator $f(c,J)$ takes the form \eqref{eq10} in the problem \eqref{eq17}. Following \cite{Kraft2017}, the corresponding HJB equation is
\begin{equation}\label{eq37}
\begin{aligned}
\sup\limits_{c\in(0,\infty),\psi\in\mathbb{R}} \;&\left\{\omega_{t}+\left[r+\psi\eta(\nu)\right]x\omega_{x}-c\omega_{x}+\frac{1}{2}\psi^{2}x^{2}\omega_{xx}+m_{1}(\nu)\omega_{\nu}+\frac{1}{2}m^{2}_{2}(\nu)\omega_{\nu\nu}\right.\\
&\left.\quad+x\psi\rho m_{2}(\nu)\omega_{x\nu}+\beta(1-\frac{1}{\phi})^{-1}(1-\gamma)\omega\left[\left(\frac{c}{((1-\gamma)\omega)^{\frac{1}{1-\gamma}}}\right)^{1-\frac{1}{\phi}}-1\right]\right\}=0.
\end{aligned}
\end{equation}
Moreover, we can derive that the candidate optimal consumption-investment strategy follows
\begin{equation}\label{eq38}
   \left\{ \begin{aligned}
   &c^{*}_{t}=\left[\frac{\omega_{x}}{\beta(1-\gamma)\omega}((1-\gamma)\omega)^{\frac{1-\frac{1}{\phi}}{1-\gamma}}\right]^{-\phi},\\
   &\psi^{*}_{t}=-\frac{\eta(\nu)\omega_{x}+\rho m_{2}(\nu)\omega_{x\nu}}{x\omega_{xx}}.
  \end{aligned}\right.
\end{equation}
We assume $\omega(t,x,\nu)=\frac{x^{1-\gamma}}{1-\gamma}h(t,\nu)^{-\frac{1-\gamma}{1-\phi}}$ for some deterministic function $h(t,\nu)$. Substituting $\eqref{eq38}$ and the above form of $\omega(t,x,\nu)$ into \eqref{eq37} leads to
\begin{align}\label{eq39}
  -\frac{h_{t}}{(1-\phi)h}&+r-\beta^{\phi}h^{-1}+\frac{1}{2\gamma}\eta^{2}(\nu)+\frac{\beta\phi}{\phi-1}\left(\beta^{\phi-1}h^{-1}-1\right)-\frac{1}{1-\phi}\left[m_{1}(\nu)+\frac{\eta(\nu)}{\gamma}\rho m_{2}(\nu)(1-\gamma)\right]\frac{h_{\nu}}{h}\nonumber\\
  &+\frac{ 1}{2(1-\phi)^{2}}m^{2}_{2}(\nu)\left[2-\phi-\gamma+\frac{\rho^{2}(1-\gamma)^{2}}{\gamma}\right]\frac{h^{2}_{\nu}}{h^{2}}-\frac{ 1}{2(1-\phi)}m^{2}_{2}(\nu)\frac{h_{\nu\nu}}{h}=0.
\end{align}
In addition, we have $c^{*}_{t}=\beta^{\phi}h^{-1}x$. Similar to the solvability analysis of the PDE \eqref{eq26}, it follows from \eqref{eq28} that \eqref{eq39} can be approximated by the following formula
\begin{align}\label{eq40}
  -&\frac{h_{t}}{(1-\phi)h}+r-\zeta_{1}-\zeta_{2}(\phi \ln\beta-\ln h)+\frac{1}{2\gamma}\eta^{2}(\nu)\nonumber\\
  &+\frac{\beta\phi}{\phi-1}\left(\beta^{-1}(\zeta_{1}+\zeta_{2}(\phi \ln\beta-\ln h))-1\right)-\frac{1}{1-\phi}\left[m_{1}(\nu)+\frac{\eta(\nu)}{\gamma}\rho m_{2}(\nu)(1-\gamma)\right]\frac{h_{\nu}}{h}\nonumber\\
  &+\frac{ 1}{2(1-\phi)^{2}}m^{2}_{2}(\nu)\left[2-\phi-\gamma+\frac{\rho^{2}(1-\gamma)^{2}}{\gamma}\right]\frac{h^{2}_{\nu}}{h^{2}}-\frac{ 1}{2(1-\phi)}m^{2}_{2}(\nu)\frac{h_{\nu\nu}}{h}=0.
\end{align}
By assuming that $h$ takes a form of exponential-polynomial, we will discuss that when the order of the polynomial is higher than 2, the PDE \eqref{eq40} is unsolvable and thus the problem \eqref{eq17} with $f(c,J)$ given by \eqref{eq10} is approximately unsolvable under this ansatz.
\begin{theorem}\label{theorem4.4}
Consider the problem \eqref{eq17} when $\phi\neq1$. If $h(t,\nu)$ is conjectured in the exponential-polynomial form such that
\begin{align}\label{eq41}
\omega(t,x,\nu)=\frac{x^{1-\gamma}}{1-\gamma}h(t,\nu)^{-\frac{1-\gamma}{1-\phi}}=\frac{x^{1-\gamma}}{1-\gamma}\exp\left\{-\frac{1-\gamma}{1-\phi}\left(A_{0}(T-t)+
\sum\limits_{k=1}^{n}\frac{1}{k}A_{k}(T-t)\nu^{k}\right)\right\},
\end{align}
where $A_{k}(t)$ for all $k=0,\cdot\cdot\cdot,n$ are functions of $t$ with $A_{0}(0)=(\phi-1)\ln \varepsilon$ and $A_{k}(0)=0\;(k=1,\cdot\cdot\cdot,n)$, then the above ansatz is not useful in providing an analytical solution to the PDE \eqref{eq40} when the order $n$ of the polynomial is higher than 2.
\end{theorem}
\begin{proof}
See Appendix D.
\end{proof}




\section{Conclusions}
This paper was devoted to investigating consumption and portfolio optimization problems under the general stochastic volatility model with recursive preferences in both infinite and finite time regions. In both cases of time regions, we further considered the investor with unit EIS and general EIS. By the dynamic programming approach, the optimization problems were changed into the solvability analysis of the corresponding HJB equations. Since the HJB equations usually do not have analytical solutions when the investor takes general EIS, we turned to find approximate solutions through the log-linear approximation method. By the conjecture of the exponential-polynomial form of the value function of the optimization problems, we proved that, when the order of the polynomial $n\leq2$, the HJB equation exists an analytical solution if the investor with unit EIS and an approximation solution otherwise.

Without further essential difficulties, our studies also work when considering multiple risky assets and multiple state variable as well as ambiguity-aversion within the framework of \cite{Maenhout2004}. It is also interesting to derive the expressions for analytical or approximate solutions under specific stochastic volatility models. We hope to present relevant study in the future.

\section*{Appendices}
\appendix
\renewcommand{\appendixname}{Appendix~\Alph{section}}
\section{Proof of Theorem \ref{theorem4.1}}
\begin{proof}
We will first discuss the cases of constant, linear and quadratic, then we will prove that the cubic and higher order cases do not work. Note that we will also point out the solvability of the PDE \eqref{eq20} in terms of $\eta(\nu)$, $m_{1}(\nu)$ and $m_{2}(\nu)$.

\noindent\textbf{Exponential-constant:} Assume that $h(\nu)=\exp\{A_{0}\}$.

Substituting the partial derivatives of $h$ into \eqref{eq20} yields
\begin{equation*}
  r-\beta+\frac{1}{2\gamma}\eta^{2}(\nu)+\beta(\ln\beta-A_{0})=0,
\end{equation*}
which implies that the PDE \eqref{eq20} is solvable for a constant $\eta$ and any feasible $m_{1}$ and $m_{2}$.

\noindent\textbf{Exponential-linear:} Assume that $h(\nu)=\exp\{A_{0}+A_{1}\nu\}$.

Substituting the partial derivatives of $h$ into \eqref{eq20} leads to
\begin{align*}
  r&-\beta+\frac{1}{2\gamma}\eta^{2}(\nu)+\beta(\ln\beta-A_{0}-A_{1}\nu)+\left[m_{1}(\nu)+\frac{\eta(\nu)}{\gamma}\rho m_{2}(\nu)(1-\gamma)\right]A_{1}\\
  &+\frac{ 1}{2}m^{2}_{2}(\nu)\left[1-\gamma+\frac{\rho^{2}}{\gamma}(1-\gamma)^{2}\right]A_{1}^{2}=0,
\end{align*}
which implies that the PDE \eqref{eq20} is solvable when $\eta^{2}(\nu)$, $m_{1}(\nu)$, $\eta(\nu) m_{2}(\nu)$ and $m^{2}_{2}(\nu)$ are linear in $\nu$.

\noindent\textbf{Exponential-quadratic:} Assume that $h(\nu)=\exp\{A_{0}+A_{1}\nu+\frac{1}{2}A_{2}\nu^{2}\}$.

Similarly, we can obtain that
\begin{align*}
  r&-\beta+\frac{1}{2\gamma}\eta^{2}(\nu)+\beta(\ln\beta-A_{0}-A_{1}\nu-\frac{1}{2}A_{2}\nu^{2})+\left[m_{1}(\nu)+\frac{\eta(\nu)}{\gamma}\rho m_{2}(\nu)(1-\gamma)\right](A_{1}+A_{2}\nu)\\
  &+\frac{ 1}{2}m^{2}_{2}(\nu)\left[(1-\gamma)(A_{1}+A_{2}\nu)^{2}+A_{2}\right]
  +\frac{\rho^{2}}{2\gamma}m^{2}_{2}(\nu)(1-\gamma)^{2}(A_{1}+A_{2}\nu)^{2}=0,
\end{align*}
which means that the PDE \eqref{eq20} is solvable when $\eta^{2}(\nu)$ is quadratic in $\nu$, $m_{1}(\nu)$ and $\eta(\nu) m_{2}(\nu)$ is linear in $\nu$, and $m^{2}_{2}(\nu)$ is a constant.

\noindent\textbf{Exponential-cubic:} Assume that $h(\nu)=\exp\{A_{0}+A_{1}\nu+\frac{1}{2}A_{2}\nu^{2}+\frac{1}{3}A_{3}\nu^{3}\}$.

Inserting the partial derivatives of $h$ into \eqref{eq20}, one has
\begin{align}\label{eq22}
  r&-\beta+\frac{1}{2\gamma}\eta^{2}(\nu)+\beta(\ln\beta-A_{0}-A_{1}\nu-\frac{1}{2}A_{2}\nu^{2}-\frac{1}{3}A_{3}\nu^{3})+\left[m_{1}(\nu)+\frac{\eta(\nu)}{\gamma}\rho m_{2}(\nu)(1-\gamma)\right](A_{1}+A_{2}\nu+A_{3}\nu^{2})\nonumber\\
  &+\frac{ 1}{2}m^{2}_{2}(\nu)\left[(1-\gamma)(A_{1}+A_{2}\nu+A_{3}\nu^{2})^{2}+A_{2}+2A_{3}\nu\right]
  +\frac{\rho^{2}}{2\gamma}m^{2}_{2}(\nu)(1-\gamma)^{2}(A_{1}+A_{2}\nu+A_{3}\nu^{2})^{2}=0.
\end{align}
Note that the above equation has terms involving $\nu^{4}$. For this equation to be solvable, one needs to cancel the terms involving $\nu^{4}$. It implies that
\begin{equation}\label{eq23}
  1-\gamma+\frac{\rho^{2}}{\gamma}(1-\gamma)^{2}=0
\end{equation}
should be satisfied, which is equivalent to $\rho^{2}=\frac{\gamma}{\gamma-1}$ due to $\gamma\neq1$. If $\gamma>1$, then $\rho^{2}>1$, which contradicts the fact that $\rho\in[-1,1]$. If $0<\gamma<1$, then $\rho^{2}<0$, which is impossible. Therefore, the terms involving $\nu^{4}$ in \eqref{eq22} cannot be matched. Thus, the conjecture of the exponential-cubic form of $h$ cannot solve the problem \eqref{eq15} when $\phi=1$.

\noindent\textbf{Exponential-nth:} Assume that $h(\nu)=\exp\{A_{0}+\sum\limits_{k=1}^{n}\frac{1}{k}A_{k}\nu^{k}\}$.

By substituting the partial derivatives of $h$ into \eqref{eq20}, we have
\begin{align*}
  r&-\beta+\frac{1}{2\gamma}\eta^{2}(\nu)+\beta\left(\ln\beta-A_{0}-\sum\limits_{k=1}^{n}\frac{1}{k}A_{k}\nu^{k}\right)+\left[m_{1}(\nu)+\frac{\eta(\nu)}{\gamma}\rho m_{2}(\nu)(1-\gamma)\right]\sum\limits_{k=1}^{n}A_{k}\nu^{k-1}\nonumber\\
  &+\frac{ 1}{2}m^{2}_{2}(\nu)\left[(1-\gamma)\left(\sum\limits_{k=1}^{n}A_{k}\nu^{k-1}\right)^{2}+\sum\limits_{k=2}^{n}(k-1)A_{k}\nu^{k-2}\right]
  +\frac{\rho^{2}}{2\gamma}m^{2}_{2}(\nu)(1-\gamma)^{2}\left(\sum\limits_{k=1}^{n}A_{k}\nu^{k-1}\right)^{2}=0,
\end{align*}
which involves $\nu^{n+1},\cdot\cdot\cdot,\nu^{2n-2}$ for $n\geq3$. These terms cannot be matched unless $1-\gamma+\frac{\rho^{2}}{\gamma}(1-\gamma)^{2}=0$. It follows from the previous analysis that the equation \eqref{eq23} does not hold. Therefore, when $n\geq3$, the conjecture of the exponential-polynomial form of $h$ cannot solve the problem \eqref{eq15} when $\phi=1$.
\end{proof}






\section{Proof of Theorem \ref{theorem4.2}}
\begin{proof}
Similar to the proof of Theorem \ref{theorem4.1}, we will first study the cases of constant, linear and quadratic, and then verify that the cubic and higher order cases do not work. By the way, we will state out the solvability of approximate solutions of the PDE \eqref{eq26} in terms of $\eta(\nu)$, $m_{1}(\nu)$ and $m_{2}(\nu)$.

\noindent\textbf{Exponential-constant:} Assume that $h(\nu)=\exp\{A_{0}\}$.

Substituting the partial derivatives of $h$ and \eqref{eq28} into \eqref{eq26} yields
\begin{equation*}
  r-\zeta_{1}-\zeta_{2}(\phi \ln\beta-A_{0})+\frac{1}{2\gamma}\eta^{2}(\nu)+\frac{\beta\phi}{\phi-1}\left[\frac{1}{\beta}\bigg(\zeta_{1}+\zeta_{2}(\phi \ln\beta-A_{0})\bigg)-1\right]=0,
\end{equation*}
which implies that the PDE \eqref{eq26} can be approximately solvable for a constant $\eta$ and any feasible $m_{1}$ and $m_{2}$.

\noindent\textbf{Exponential-linear:} Assume that $h(\nu)=\exp\{A_{0}+A_{1}\nu\}$.

Substituting the partial derivatives of $h$ and \eqref{eq28} into \eqref{eq26} leads to
\begin{align*}
  r&-\zeta_{1}-\zeta_{2}(\phi \ln\beta-A_{0}-A_{1}\nu)+\frac{1}{2\gamma}\eta^{2}(\nu)+\frac{\beta\phi}{\phi-1}\left[\frac{1}{\beta}\bigg(\zeta_{1}+\zeta_{2}(\phi \ln\beta-A_{0}-A_{1}\nu)\bigg)-1\right]\\
  &-\frac{1}{1-\phi}\left[m_{1}(\nu)+\frac{\eta(\nu)}{\gamma}\rho m_{2}(\nu)(1-\gamma)\right]A_{1}\nonumber\\
  &+\frac{ 1}{2(1-\phi)^{2}}m^{2}_{2}(\nu)\left[2-\phi-\gamma+\frac{\rho^{2}(1-\gamma)^{2}}{\gamma}\right]A_{1}^{2}-\frac{ 1}{2(1-\phi)}m^{2}_{2}(\nu)A_{1}^{2}=0,
\end{align*}
which implies that the PDE \eqref{eq26} can be approximately solvable when $\eta^{2}(\nu)$, $m_{1}(\nu)$, $\eta(\nu) m_{2}(\nu)$ and $m^{2}_{2}(\nu)$ are linear in $\nu$.

\noindent\textbf{Exponential-quadratic:} Assume that $h(\nu)=\exp\{A_{0}+A_{1}\nu+\frac{1}{2}A_{2}\nu^{2}\}$.

Similarly, one can obtain that
\begin{align*}
  r&-\zeta_{1}-\zeta_{2}(\phi \ln\beta-A_{0}-A_{1}\nu-\frac{1}{2}A_{2}\nu^{2})+\frac{\beta\phi}{\phi-1}\left[\frac{1}{\beta}\bigg(\zeta_{1}+\zeta_{2}(\phi \ln\beta-A_{0}-A_{1}\nu-\frac{1}{2}A_{2}\nu^{2})\bigg)-1\right]\\
  &+\frac{1}{2\gamma}\eta^{2}(\nu)-\frac{1}{1-\phi}\left[m_{1}(\nu)+\frac{\eta(\nu)}{\gamma}\rho m_{2}(\nu)(1-\gamma)\right](A_{1}+A_{2} \nu)\nonumber\\
  &+\frac{ 1}{2(1-\phi)^{2}}m^{2}_{2}(\nu)\left[2-\phi-\gamma+\frac{\rho^{2}(1-\gamma)^{2}}{\gamma}\right](A_{1}+A_{2} \nu)^{2}-\frac{ 1}{2(1-\phi)}m^{2}_{2}(\nu)\left[(A_{1}+A_{2} \nu)^{2}+A_{2}\right]=0,
\end{align*}
which means that the PDE \eqref{eq26} can be approximately solvable when $\eta^{2}(\nu)$ is quadratic in $\nu$, $m_{1}(\nu)$ and $\eta(\nu) m_{2}(\nu)$ is linear in $\nu$, and $m^{2}_{2}(\nu)$ is a constant.

\noindent\textbf{Exponential-cubic:} Assume that $h(\nu)=\exp\{A_{0}+A_{1}\nu+\frac{1}{2}A_{2}\nu^{2}+\frac{1}{3}A_{3}\nu^{3}\}$.

Inserting the partial derivatives of $h$ and \eqref{eq28} into \eqref{eq26}, we can derive that
\begin{align}\label{eq30}
  r&-\zeta_{1}-\zeta_{2}\phi \ln\beta+\zeta_{2}(A_{0}+A_{1}\nu+\frac{1}{2}A_{2}\nu^{2}+\frac{1}{3}A_{3}\nu^{3})+\frac{1}{2\gamma}\eta^{2}(\nu)\nonumber\\
  &+\frac{\beta\phi}{\phi-1}\left[\frac{1}{\beta}\bigg(\zeta_{1}+\zeta_{2}(\phi \ln\beta-A_{0}-A_{1}\nu-\frac{1}{2}A_{2}\nu^{2}-\frac{1}{3}A_{3}\nu^{3})\bigg)-1\right]\nonumber\\
  &-\frac{1}{1-\phi}\left[m_{1}(\nu)+\frac{\eta(\nu)}{\gamma}\rho m_{2}(\nu)(1-\gamma)\right](A_{1}+A_{2} \nu+A_{3} \nu^{2})\nonumber\\
  &+\frac{ 1}{2(1-\phi)^{2}}m^{2}_{2}(\nu)
  \left[2-\phi-\gamma+\frac{\rho^{2}(1-\gamma)^{2}}{\gamma}\right](A_{1}+A_{2} \nu+A_{3} \nu^{2})^{2}\nonumber\\
  &-\frac{ 1}{2(1-\phi)}m^{2}_{2}(\nu)\left[(A_{1}+A_{2} \nu+A_{3} \nu^{2})^{2}+A_{2}+2A_{3} \nu\right]=0.
\end{align}
We can see that the above equation has terms involving $\nu^{4}$. If this equation can be solvable, one has to cancel the terms involving $\nu^{4}$. It implies that
\begin{equation}\label{eq31}
  \frac{ 1}{(1-\phi)^{2}}\left[2-\phi-\gamma+\frac{\rho^{2}(1-\gamma)^{2}}{\gamma}\right]-\frac{ 1}{1-\phi}=0
\end{equation}
should be satisfied. It is easy to derive that the equation \eqref{eq31} is equivalent to $\rho^{2}=\frac{\gamma}{\gamma-1}$, which does not hold by the proof of Theorem \ref{theorem4.1}. Therefore, the terms involving $\nu^{4}$ in \eqref{eq30} cannot be matched. Thus, the conjecture of the exponential-cubic form of $h$ cannot approximately solve the problem \eqref{eq15} when $\phi\neq1$.

\noindent\textbf{Exponential-nth:} Assume that $h(\nu)=\exp\{A_{0}+\sum\limits_{k=1}^{n}\frac{1}{k}A_{k}\nu^{k}\}$.

By substituting the partial derivatives of $h$ and \eqref{eq28} into \eqref{eq26}, we have
\begin{align*}
  r&-\zeta_{1}-\zeta_{2}\left(\phi \ln\beta-A_{0}-\sum\limits_{k=1}^{n}\frac{1}{k}A_{k}\nu^{k}\right)+\frac{\beta\phi}{\phi-1}\left[\frac{1}{\beta}\bigg(\zeta_{1}+\zeta_{2}(\phi \ln\beta-A_{0}-\sum\limits_{k=1}^{n}\frac{1}{k}A_{k}\nu^{k})\bigg)-1\right]\\
  &+\frac{1}{2\gamma}\eta^{2}(\nu)-\frac{1}{1-\phi}\left[m_{1}(\nu)+\frac{\eta(\nu)}{\gamma}\rho m_{2}(\nu)(1-\gamma)\right]\sum\limits_{k=1}^{n}A_{k}\nu^{k-1}\nonumber\\
  &+\frac{ 1}{2(1-\phi)^{2}}m^{2}_{2}(\nu)\left[2-\phi-\gamma+\frac{\rho^{2}(1-\gamma)^{2}}{\gamma}\right]\left(\sum\limits_{k=1}^{n}A_{k}\nu^{k-1}\right)^{2}\\
  &-\frac{ 1}{2(1-\phi)}m^{2}_{2}(\nu)\left[\left(\sum\limits_{k=1}^{n}A_{k}\nu^{k-1}\right)^{2}+\sum\limits_{k=2}^{n}(k-1)A_{k}\nu^{k-2}\right]=0,
\end{align*}
which involves $\nu^{n+1},\cdot\cdot\cdot,\nu^{2n-2}$ for $n\geq3$. Since the equation \eqref{eq31} does not hold, these terms cannot be matched. Therefore, given $n\geq3$, the conjecture of the exponential-polynomial form of $h$ cannot approximately solve the problem \eqref{eq15} when $\phi\neq1$.
\end{proof}


\section{Proof of Theorem \ref{theorem4.3}}

\begin{proof}
We will first investigate the constant, linear and quadratic cases and show the solvability of the PDE \eqref{eq34} in terms of $\eta(\nu)$, $m_{1}(\nu)$ and $m_{2}(\nu)$. Then we will show that the conjecture does not work in solving the PDE \eqref{eq34} under the cases of the cubic and higher order.

\noindent\textbf{Exponential-constant:} Assume that $h(t,\nu)=\exp\{A_{0}(T-t)\}$.

Substituting the partial derivatives of $h$ into \eqref{eq34} leads to
\begin{equation*}
  -A'_{0}+r-\beta+\frac{1}{2\gamma}\eta^{2}(\nu)+\beta(\ln\beta-A_{0})=0,
\end{equation*}
where $A'_{0}$ denotes the derivative of $A_{0}(t)$ with respect to $t$. Similar expressions will be used later when there is no ambiguity. Therefore, the PDE \eqref{eq34} is solvable for a constant $\eta$ and any feasible $m_{1}$ and $m_{2}$.

\noindent\textbf{Exponential-linear:} Assume that $h(t,\nu)=\exp\{A_{0}(T-t)+A_{1}(T-t)\nu\}$.

Inserting the partial derivatives of $h$ into \eqref{eq34} yields
\begin{align*}
   -A'_{0}&-A'_{1}\nu+r-\beta+\frac{1}{2\gamma}\eta^{2}(\nu)+\beta(\ln\beta-A_{0}-A_{1}\nu)+\left[m_{1}(\nu)+\frac{\eta(\nu)}{\gamma}\rho m_{2}(\nu)(1-\gamma)\right]A_{1}\\
  &+\frac{ 1}{2}m^{2}_{2}(\nu)\left[1-\gamma+\frac{\rho^{2}}{\gamma}(1-\gamma)^{2}\right]A_{1}^{2}=0,
\end{align*}
which shows that the PDE \eqref{eq34} is solvable when $\eta^{2}(\nu)$, $m_{1}(\nu)$, $\eta(\nu) m_{2}(\nu)$ and $m^{2}_{2}(\nu)$ are linear in $\nu$.

\noindent\textbf{Exponential-quadratic:} Assume that $h(t,\nu)=\exp\{A_{0}(T-t)+A_{1}(T-t)\nu+\frac{1}{2}A_{2}(T-t)\nu^{2}\}$.

Similarly, one has
\begin{align*}
  -A'_{0}&-A'_{1}\nu-\frac{1}{2}A'_{2}\nu^{2}+r-\beta+\frac{1}{2\gamma}\eta^{2}(\nu)+\beta(\ln\beta-A_{0}-A_{1}\nu-\frac{1}{2}A_{2}\nu^{2})
  +\left[m_{1}(\nu)+\frac{\eta(\nu)}{\gamma}\rho m_{2}(\nu)(1-\gamma)\right](A_{1}+A_{2}\nu)\\
  &+\frac{ 1}{2}m^{2}_{2}(\nu)\left[(1-\gamma)(A_{1}+A_{2}\nu)^{2}+A_{2}\right]
  +\frac{\rho^{2}}{2\gamma}m^{2}_{2}(\nu)(1-\gamma)^{2}(A_{1}+A_{2}\nu)^{2}=0,
\end{align*}
which means that the PDE \eqref{eq34} is solvable when $\eta^{2}(\nu)$ is quadratic in $\nu$, $m_{1}(\nu)$ and $\eta(\nu) m_{2}(\nu)$ is linear in $\nu$, and $m^{2}_{2}(\nu)$ is a constant.

\noindent\textbf{Exponential-cubic:} Assume that $h(t,\nu)=\exp\{A_{0}(T-t)+A_{1}(T-t)\nu+\frac{1}{2}A_{2}(T-t)\nu^{2}+\frac{1}{3}A_{3}(T-t)\nu^{3}\}$.

By using the partial derivatives of $h$ and the equation \eqref{eq34}, we have
\begin{align}\label{eq36}
  -A'_{0}&-A'_{1}\nu-\frac{1}{2}A'_{2}\nu^{2}-\frac{1}{3}A'_{3}\nu^{3}+r-\beta+\frac{1}{2\gamma}\eta^{2}(\nu)
  +\beta(\ln\beta-A_{0}-A_{1}\nu-\frac{1}{2}A_{2}\nu^{2}-\frac{1}{3}A_{3}\nu^{3})\nonumber\\
  &+\left[m_{1}(\nu)+\frac{\eta(\nu)}{\gamma}\rho m_{2}(\nu)(1-\gamma)\right](A_{1}+A_{2}\nu+A_{3}\nu^{2})+\frac{ 1}{2}m^{2}_{2}(\nu)\nonumber\\
  &\times\left[(1-\gamma)(A_{1}+A_{2}\nu+A_{3}\nu^{2})^{2}+A_{2}+2A_{3}\nu\right]
  +\frac{\rho^{2}}{2\gamma}m^{2}_{2}(\nu)(1-\gamma)^{2}(A_{1}+A_{2}\nu+A_{3}\nu^{2})^{2}=0.
\end{align}
Similar to the proof of Theorem \ref{theorem4.1}, we can know that the terms involving $\nu^{4}$ in \eqref{eq36} cannot be matched and then the conjecture of the exponential-cubic form of $h$ cannot solve the problem \eqref{eq17} when $\phi=1$.

\noindent\textbf{Exponential-nth:} Assume that $h(t,\nu)=\exp\{A_{0}(T-t)+\sum\limits_{k=1}^{n}\frac{1}{k}A_{k}(T-t)\nu^{k}\}$.

By substituting the partial derivatives of $h$ into \eqref{eq34}, we can derive that
\begin{align*}
  -A'_{0}&-\sum\limits_{k=1}^{n}\frac{1}{k}A'_{k}\nu^{k}+r-\beta+\frac{1}{2\gamma}\eta^{2}(\nu)
  +\beta\left(\ln\beta-A_{0}-\sum\limits_{k=1}^{n}\frac{1}{k}A_{k}\nu^{k}\right)
  +\left[m_{1}(\nu)+\frac{\eta(\nu)}{\gamma}\rho m_{2}(\nu)(1-\gamma)\right]\sum\limits_{k=1}^{n}A_{k}\nu^{k-1}\nonumber\\
  &+\frac{ 1}{2}m^{2}_{2}(\nu)\left[(1-\gamma)\left(\sum\limits_{k=1}^{n}A_{k}\nu^{k-1}\right)^{2}+\sum\limits_{k=2}^{n}(k-1)A_{k}\nu^{k-2}\right]
  +\frac{\rho^{2}}{2\gamma}m^{2}_{2}(\nu)(1-\gamma)^{2}\left(\sum\limits_{k=1}^{n}A_{k}\nu^{k-1}\right)^{2}=0,
\end{align*}
which involves $\nu^{n+1},\cdot\cdot\cdot,\nu^{2n-2}$ for $n\geq3$. It follows from the proof of Theorem \ref{theorem4.1} that these terms cannot be matched. Thus, when $n\geq3$, the exponential-polynomial form of $h$ cannot solve the problem \eqref{eq17} with $\phi=1$.
\end{proof}



\section{Proof of Theorem \ref{theorem4.4}}
\begin{proof}
In the following proof process, we first consider the solvability of the PDE \eqref{eq40} in terms of $\eta(\nu)$, $m_{1}(\nu)$ and $m_{2}(\nu)$ under the cases of constant, linear and quadratic. Then we show that the conjecture of $h$ does not work under the cases of the cubic and higher order.

\noindent\textbf{Exponential-constant:} Assume that $h(t,\nu)=\exp\{A_{0}(T-t)\}$.

Substituting the partial derivatives of $h$ into \eqref{eq40} yields
\begin{equation*}
  \frac{A'_{0}}{1-\phi}+r-\zeta_{1}-\zeta_{2}(\phi \ln\beta-A_{0})+\frac{1}{2\gamma}\eta^{2}(\nu)+\frac{\beta\phi}{\phi-1}\left[\frac{1}{\beta}\bigg(\zeta_{1}+\zeta_{2}(\phi \ln\beta-A_{0})\bigg)-1\right]=0,
\end{equation*}
which implies that the PDE \eqref{eq40} can be solvable for a constant $\eta$ and any feasible $m_{1}$ and $m_{2}$.

\noindent\textbf{Exponential-linear:} Assume that $h(t,\nu)=\exp\{A_{0}(T-t)+A_{1}(T-t)\nu\}$.

Similarly, one can obtain that
\begin{align*}
  \frac{1}{1-\phi}&(A'_{0}+A'_{1}\nu)+r-\zeta_{1}-\zeta_{2}(\phi \ln\beta-A_{0}-A_{1}\nu)+\frac{1}{2\gamma}\eta^{2}(\nu)\\
  &+\frac{\beta\phi}{\phi-1}\left[\frac{1}{\beta}\bigg(\zeta_{1}+\zeta_{2}(\phi \ln\beta-A_{0}-A_{1}\nu)\bigg)-1\right]
  -\frac{1}{1-\phi}\left[m_{1}(\nu)+\frac{\eta(\nu)}{\gamma}\rho m_{2}(\nu)(1-\gamma)\right]A_{1}\nonumber\\
  &+\frac{ 1}{2(1-\phi)^{2}}m^{2}_{2}(\nu)\left[2-\phi-\gamma+\frac{\rho^{2}(1-\gamma)^{2}}{\gamma}\right]A_{1}^{2}-\frac{ 1}{2(1-\phi)}m^{2}_{2}(\nu)A_{1}^{2}=0,
\end{align*}
which shows that the PDE \eqref{eq40} can be solvable when $\eta^{2}(\nu)$, $m_{1}(\nu)$, $\eta(\nu) m_{2}(\nu)$ and $m^{2}_{2}(\nu)$ are linear in $\nu$.

\noindent\textbf{Exponential-quadratic:} Assume that $h(t,\nu)=\exp\{A_{0}(T-t)+A_{1}(T-t)\nu+\frac{1}{2}A_{2}(T-t)\nu^{2}\}$.

Inserting the partial derivatives of $h$ into \eqref{eq40} leads to
\begin{align*}
  \frac{1}{1-\phi}&(A'_{0}+A'_{1}\nu+\frac{1}{2}A'_{2}\nu^{2})+r-\zeta_{1}-\zeta_{2}(\phi \ln\beta-A_{0}-A_{1}\nu-\frac{1}{2}A_{2}\nu^{2})\\
  &+\frac{\beta\phi}{\phi-1}\left[\frac{1}{\beta}\bigg(\zeta_{1}+\zeta_{2}(\phi \ln\beta-A_{0}-A_{1}\nu-\frac{1}{2}A_{2}\nu^{2})\bigg)-1\right]\\
  &+\frac{1}{2\gamma}\eta^{2}(\nu)-\frac{1}{1-\phi}\left[m_{1}(\nu)+\frac{\eta(\nu)}{\gamma}\rho m_{2}(\nu)(1-\gamma)\right](A_{1}+A_{2} \nu)\nonumber\\
  &+\frac{ 1}{2(1-\phi)^{2}}m^{2}_{2}(\nu)\left[2-\phi-\gamma+\frac{\rho^{2}(1-\gamma)^{2}}{\gamma}\right](A_{1}+A_{2} \nu)^{2}-\frac{ 1}{2(1-\phi)}m^{2}_{2}(\nu)\left[(A_{1}+A_{2} \nu)^{2}+A_{2}\right]=0,
\end{align*}
which means that the PDE \eqref{eq40} can be solvable when $\eta^{2}(\nu)$ is quadratic in $\nu$, $m_{1}(\nu)$ and $\eta(\nu) m_{2}(\nu)$ is linear in $\nu$, and $m^{2}_{2}(\nu)$ is a constant.

\noindent\textbf{Exponential-cubic:} Assume that $h(t,\nu)=\exp\{A_{0}(T-t)+A_{1}(T-t)\nu+\frac{1}{2}A_{2}(T-t)\nu^{2}+\frac{1}{3}A_{3}(T-t)\nu^{3}\}$.

By substituting the partial derivatives of $h$ into \eqref{eq40}, we can derive that
\begin{align}\label{eq42}
  \frac{1}{1-\phi}&(A'_{0}+A'_{1}\nu+\frac{1}{2}A'_{2}\nu^{2}+\frac{1}{3}A'_{3}\nu^{3})+r-\zeta_{1}-\zeta_{2}\phi \ln\beta+\zeta_{2}(A_{0}+A_{1}\nu+\frac{1}{2}A_{2}\nu^{2}+\frac{1}{3}A_{3}\nu^{3})\nonumber\\
  &+\frac{1}{2\gamma}\eta^{2}(\nu)+\frac{\beta\phi}{\phi-1}\left[\frac{1}{\beta}\bigg(\zeta_{1}+\zeta_{2}(\phi \ln\beta-A_{0}-A_{1}\nu-\frac{1}{2}A_{2}\nu^{2}-\frac{1}{3}A_{3}\nu^{3})\bigg)-1\right]\nonumber\\
  &-\frac{1}{1-\phi}\left[m_{1}(\nu)+\frac{\eta(\nu)}{\gamma}\rho m_{2}(\nu)(1-\gamma)\right](A_{1}+A_{2} \nu+A_{3} \nu^{2})\nonumber\\
  &+\frac{ 1}{2(1-\phi)^{2}}m^{2}_{2}(\nu)
  \left[2-\phi-\gamma+\frac{\rho^{2}(1-\gamma)^{2}}{\gamma}\right](A_{1}+A_{2} \nu+A_{3} \nu^{2})^{2}\nonumber\\
  &-\frac{ 1}{2(1-\phi)}m^{2}_{2}(\nu)\left[(A_{1}+A_{2} \nu+A_{3} \nu^{2})^{2}+A_{2}+2A_{3} \nu\right]=0,
\end{align}
which has terms involving $\nu^{4}$. By the proof of Theorem \ref{theorem4.2}, the terms involving $\nu^{4}$ in \eqref{eq42} cannot be matched and thus the exponential-cubic form of $h$ cannot approximately solve the problem \eqref{eq17} when $\phi\neq1$.

\noindent\textbf{Exponential-nth:} Assume that $h(t,\nu)=\exp\{A_{0}(T-t)+\sum\limits_{k=1}^{n}\frac{1}{k}A_{k}(T-t)\nu^{k}\}$.

Substituting the partial derivatives of $h$ into \eqref{eq40}, we have
\begin{align*}
  \frac{1}{1-\phi}&(A'_{0}+\sum\limits_{k=1}^{n}\frac{1}{k}A'_{k}\nu^{k})+r-\zeta_{1}-\zeta_{2}\left(\phi \ln\beta-A_{0}-\sum\limits_{k=1}^{n}\frac{1}{k}A_{k}\nu^{k}\right)\\
  &+\frac{\beta\phi}{\phi-1}\left[\frac{1}{\beta}\bigg(\zeta_{1}+\zeta_{2}(\phi \ln\beta-A_{0}-\sum\limits_{k=1}^{n}\frac{1}{k}A_{k}\nu^{k})\bigg)-1\right]\\
  &+\frac{1}{2\gamma}\eta^{2}(\nu)-\frac{1}{1-\phi}\left[m_{1}(\nu)+\frac{\eta(\nu)}{\gamma}\rho m_{2}(\nu)(1-\gamma)\right]\sum\limits_{k=1}^{n}A_{k}\nu^{k-1}\nonumber\\
  &+\frac{ 1}{2(1-\phi)^{2}}m^{2}_{2}(\nu)\left[2-\phi-\gamma+\frac{\rho^{2}(1-\gamma)^{2}}{\gamma}\right]\left(\sum\limits_{k=1}^{n}A_{k}\nu^{k-1}\right)^{2}\\
  &-\frac{ 1}{2(1-\phi)}m^{2}_{2}(\nu)\left[\left(\sum\limits_{k=1}^{n}A_{k}\nu^{k-1}\right)^{2}+\sum\limits_{k=2}^{n}(k-1)A_{k}\nu^{k-2}\right]=0,
\end{align*}
which involves $\nu^{n+1},\cdot\cdot\cdot,\nu^{2n-2}$ for $n\geq3$. It follows from the proof of Theorem \ref{theorem4.2} that these terms cannot be matched. Thus, when $n\geq3$, the conjecture of the exponential-polynomial form of $h$ cannot approximately solve the problem \eqref{eq17} with $\phi\neq1$.
\end{proof}




\begin{thebibliography}{99}

\bibitem{Campbell2004}
J.Y. Campbell, G. Chacko, J. Rodriguez and L.M. Viceira,
Strategic asset allocation in a continuous-time VAR model,
{\em Journal of Economic Dynamics and Control}, {\bf28(11)}:2195-2214, 2004.

\bibitem{Chacko2005}
G. Chacko and L.M. Viceira,
Dynamic consumption and portfolio choice with stochastic volatility in incomplete markets,
{\em The Review of Financial Studies}, {\bf18(4)}:1369-1402, 2005.

\bibitem{Cheng2023}
Y. Cheng and M. Escobar-Anel,
A class of portfolio optimization solvable problems,
{\em Finance Research Letters}, {\bf52}:103373, 2023.

\bibitem{Cui2017}
Z. Cui, J.L. Kirkby and D. Nguyen,
A general framework for discretely sampled realized variance derivatives in stochastic volatility models with jumps,
{\em European Journal of Operational Research}, {\bf262(1)}:381-400, 2017.

\bibitem{Duffie1992}
D. Duffie and L.G. Epstein,
Stochastic differential utility,
{\em Econometrica}, {\bf60(2)}:353-394, 1992.

\bibitem{Epstein1989}
L. Epstein and S. Zin,
Substitution, risk aversion and the temporal behavior of consumption and asset returns: A theoretical framework,
{\em Econometrica}, {\bf57}:937-969, 1989.

\bibitem{He2021}
X.J. He and W. Chen,
 A closed-form pricing formula for European options under a new stochastic volatility model with a stochastic long-term mean,
{\em Mathematics and Financial Economics}, {\bf15}:381-396, 2021.

\bibitem{Iftimie2023}
B. Iftimie,
A robust investment-consumption optimization problem in a switching regime interest rate setting,
{\em Journal of Global Optimization}, {\bf86}:713-739, 2023.

\bibitem{Kang2021}
J.H. Kang, M.H. Wang and N.J. Huang, Equilibrium strategy for mean-variance-utility portfolio selection under Heston's SV model,
{\em Journal of Computational and Applied Mathematics}, {\bf392}:113490, 2021.

\bibitem{Korn2001}
R. Korn and E. Korn,
{\em Option Pricing and Portfolio Optimization: Modern Methods of Financial Mathematics},
American Mathematical Society, Providence, 2001.

\bibitem{Kraft2005}
H. Kraft,
Optimal portfolios and Heston's stochastic volatility model: an explicit solution for power utility,
{\em Quantitative Finance}, {\bf5(3)}:303-313, 2005.


\bibitem{Kraft2017}
H. Kraft, T. Seiferling and F.T. Seifried,
Optimal consumption and investment with Epstein-Zin recursive utility,
{\em Finance and Stochastics}, {\bf21}:187-226, 2017.


\bibitem{Kreps1978}
D.M. Kreps and E.L. Porteus,
Temporal resolution of uncertainty and dynamic choice theory,
{\em Econometrica}, {\bf46}:185-200, 1978.

\bibitem{Liu2007}
J. Liu,
Portfolio selection in stochastic environments,
{\em The Review of Financial Studies}, {\bf20(1)}:1-39, 2007.

\bibitem{Maenhout2004}
P.J. Maenhout,
Robust portfolio rules and asset pricing,
{\em The Review of Financial Studies}, {\bf17(4)}:951-983, 2004.

\bibitem{Merton1971}
R. Merton,
Lifetime portfolio selection under uncertainty: the continuous-time case,
{\em The Review of Economics and Statistics}, {\bf51}:247-257, 1969.


\bibitem{Schroder1999}
M. Schroder and C. Skiadas,
Optimal consumption and portfolio selection with stochastic differential utility,
{\em Journal of Economic Theory}, {\bf89(1)}:68-126, 1999.



\bibitem{Yamada1971}
T. Yamada and S. Watanabe,
On the uniqueness of solutions of stochastic differential equations II,
{\em Journal of Mathematics of Kyoto University}, {\bf11(3)}:553-563, 1971.

\bibitem{Zhang2016}
Q. Zhang and L. Ge,
Optimal strategies for asset allocation and consumption under stochastic volatility,
{\em Applied Mathematics Letters}, {\bf58}:69-73, 2016.



\end{thebibliography}
\end{document}
