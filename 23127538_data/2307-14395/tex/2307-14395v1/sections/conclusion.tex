\section{Discussion and future work}
PDE-Net++ is a newly proposed hybrid neural network architecture that incorporates physics prior with black-box models.
Comparisons of accuracy and stability have exhibited that even partly embedding the governing equations into the network can significantly improve the accuracy of predictions. Besides, inspired by the trainable difference operators in the existing works, two additional modules named TFDL and TDDL are firstly introduced. All the experiment results have shown that these two modules are able to stablize the roll-out process effectively.

Current architecture can be applied to any other prediction tasks for spatio-temporal dynamics as long as partial knowledge of the underlying PDEs is provided, although we have to emphasize that necessary modifications for the parameterizations of the feasible schemes are needed. For instance, in deep-learning based weather prediction tasks, till now, only black-box models such as U-Net \cite{Weyn2021modifiedDLWP}, ResNet \cite{Rasp2021ResNet,Clare2021ResNetProbability}, GNN \cite{keisler2022GNNs,lam2022graphcast}, and Transformer \cite{Guibas2021AFNO,pathak2022fourcastnet,Liu2021SwinTransformer,bi2022pangu} have been tested. Since the dynamical core has been studied for decades by experts, with the aid of PDE-Net++, existing state-of-the-art models are believed to be able to achieve better performance.