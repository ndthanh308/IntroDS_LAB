\section{Non extreme amenability}
Recall that the \emph{extended mapping class group} $\MCG(\Sigma)^\pm$ is defined similarly as $MCG(\Sigma)$ but using the group of all automorphisms $\mathrm{Homeo}(\Sigma)$ in the stead of the orientation preserving one. The result below is already known following several basic facts about mapping class groups:

\begin{prop}\label{prop_non-arch}
If the surface $\Sigma$ is of complexity at least two, then the mapping class group $\MCG(\Sigma)$ and the extended mapping class group $\MCG^\pm(\Sigma)$ are non-archimedean Polish groups.
\end{prop}
\begin{proof}
Let $\mathcal{C}(\Sigma)$ be the curve graph of the surface $\Sigma$, where vertices are isotopic classes of essential simple closed curves and an edge is attached to two vertices if the two classes have disjoint representatives. It is well-known that for a surface $\Sigma$ with complexity at least two, of finite or infinite type, we have the isomorphism $\MCG^\pm(\Sigma)\simeq\mathrm{Aut}(\mathcal{C}(\Sigma))$ between topological groups \cite{ivanov1997automorphisms,luo2000automorphisms,bavard2020isomorphisms}. Note that the vertices in $\mathcal{C}(\Sigma)$ are countable. The graph $\mathcal{C}(\Sigma)$ is actually a countable relational first-order structure. Such an automorphism group is closed subgroup of $S_\infty$ and thus a non-archimedean Polish group (see for example \cite[Part I, \S9.B(7)]{kechris2012classical}). Hence $\MCG^\pm(\Sigma)$ is a non-archimedean Polish group. To show that $\MCG(\Sigma)$ is also one, it suffices to demonstrate that $\MCG(\Sigma)$ is closed in $\MCG^\pm(\Sigma)$. Indeed, given a convergent sequence of homeomorphisms $\Sigma\to\Sigma$ in $\mathrm{Homeo}(\Sigma)$, if they are all orientation preserving, then so will be their limit.
\end{proof}

To further discuss the extreme amenability of a non-archimedean Polish group, we need to introduce the following notions. Let $G$ be a group acting on a space $X$ and $Y$ be a subspace of $X$. We call the subgroup $G_{(Y)}=\{g\in G: gy=y,\ \forall y\in Y\}$ of $G$ the \textbf{pointwise stabiliser} of $Y$. Similarly, the subgroup $G_Y=\{g\in G: gY=Y\}$ is called the \textbf{setwise stabiliser} of $Y$ in $G$. 

The following observation is a natural consequence after KPT correspondence (see \cite[Proposition 4.3]{kechris2005fraisse}), which can find its root in \cite{glasner2002minimal}.
\begin{lem}\label{lem_stab}
If $G$ is an extremely amenable non-archimedean Polish group acting continuously on the discrete space $\mathbb{N}$, then $G_{(F)}=G_F$ for any finite subset $F\subset \mathbb{N}$.
\end{lem}

The end space $\End(\Sigma)$ is a totally disconnected, separable, metrisable topological space (and thus a closed subset of Cantor set). Among the ends, there are \emph{non planary ends}, of which the collection is denoted $\End_\infty(\Sigma)$, in the sense that every neighbourhood in $\Sigma$ of a non planary end has at least genus $1$. The non planary ends $\End_\infty(\Sigma)$ form a closed subset of $\End(\Sigma)$.

\begin{rem}
There is a slight \emph{abus de langage} here. The neighbourhood mentioned above resides in the end compactification of $\Sigma$ instead of $\End(\Sigma)$, but its intersection with $\Sigma$ becomes a subsurface in $\Sigma$ and can still be regarded as a ``neighbourhood'' of an end.
\end{rem}

Let $D,D'$ be two subsets of $\End(\Sigma)$. We say that $D$ and $D'$ are \emph{isomorphic} if $D$ is homeomorphic to $D'$ and $D\cap \End_\infty(\Sigma)$ is homeomorphic $D'\cap \End_\infty(\Sigma)$ simultaneously.

Mann and Rafi give a way to order the ends of a surface by their similarity \cite{mann2020large}. Let $x,y\in \End(\Sigma)$ be two ends. Then we write $x\preccurlyeq y$ if every clopen neighbourhood of $y$ contains a clopen subset that is isomorphic to a neighbourhood of $x$. Two ends are said \emph{equivalent} if both $x\preccurlyeq y$ and $y\preccurlyeq x$. They are said \emph{non-comparable} if neither case happens.

It is worth noticing that each compact neighbourhood of an end of $\Sigma$ corresponds to one of its clopen neighbourhoods $D$ in $\End(\Sigma)$ and this compact neighbourhood can be chosen to be the end compactification of a subsurface in $\Sigma$ whose ends are the union of $D$ with an additional isolated point.

Since the group $\mathrm{Homeo}^+(\Sigma)$ acts on the surface $\Sigma$ continuously, this action has a natural continuous extension onto the end compactification of $\Sigma$. By taking the quotient, it is not hard to see that $\MCG(\Sigma)$ acts continuously by homoemorphisms on $\End(\Sigma)$ and $\End_\infty(\Sigma)$ is an invariant subspace of this group action. Moreover, from the definition, this action preserves the ordering given above, namely $gx\preccurlyeq gy$ for any $g\in\MCG(\Sigma)$ if and only if $x\preccurlyeq y$. In particular, if there exists $g\in\MCG(\Sigma)$ such that $gx=y$, then necessarily $x\preccurlyeq y$. Also, if $\Sigma$ is a surface with genus $0$, then $\MCG(\Sigma)$ is the homeomorphism group of $\End(\Sigma)$.

Adopting the terminology from \cite{farb2011primer}, the \textbf{topological type} of a simple closed curve $c$ on $\Sigma$ is $\Sigma\setminus c$. Two curves $c$ and $c'$ on $\Sigma$ are said to \emph{have the same topological type} if there is a homeomorphism between $\Sigma\setminus c$ and $\Sigma \setminus c'$.

\begin{rem}
Since no distinction will be needed here, in the sequel, a curve on the surface can mean either a topological embedding of $S^1$ or its isotopy class, depending on the context. 
\end{rem}

An argument of the renowned \emph{Alexander method} (see for example \cite[\S 2.3]{farb2011primer} and \cite{hernandez2019alexander}) is that given two distinct simple closed curves (up to isotopy) of the same topological type, one can always find a non-trivial mapping class sending one curve to the other. Moreover, this mapping class $g$ can be chosen to have finite order if the surface $\Sigma$ is of finite type. This is not true in general for infinite-type surfaces, but only some special cases will be needed for our purpose here and similar arguments yield the same result for these special cases:
\begin{lem}\label{lem_non_ea_1}
Let $\Sigma$ be a surface with genus $0$. Suppose that there exists two disjoint simple closed curves $c,c'$ on $\Sigma$ with the same topological type. Assume that $c$ cuts $\End(\Sigma)=E\sqcup N$ and $c'$ cuts $\End(\Sigma)=E'\sqcup N'$ in a way that $E\simeq E'$ and that $E,E'$ are disjoint. Then there exists a non-trivial mapping class $\phi\in \MCG(\Sigma)$ such that $\phi(c)=c'$ and $\phi^2=\Id$.
\end{lem}
\begin{proof}
Figure \ref{fig:mickey} depicts how such a surface $\Sigma$ looks like. The simplest case is when $E$ and $E'$ reduces to singleton and $\Sigma$ is a twice-punctured sphere. Now the desired $\phi\in\MCG(\Sigma)$ is just the symmetry sending $E$ to $E'$. More precisely, cut $\Sigma$ into three parts $S\sqcup S' \sqcup \big(\Sigma\setminus (S\cup S')\big)$ along $c$ and $c'$ so that $\partial S=c$ and $\partial S'=c'$. By Richards' theorem, there is a homeomorphism $\varphi\colon S\to S'$ sending $\partial S$ to $\partial S'$. Take a symmetry $\sigma\in\MCG(\Sigma\setminus(S\cup S'))$ that interchanges the position of $c$ and $c'$: $\sigma$ is the continuous extension by identity of the symmetry $\overline{\sigma}\in\MCG(S_{0,3})$, where $S_{0,3}\subset \Sigma\setminus(S\cup S')$ is a subsurface of finite type with two of its ends being $\partial S=c$ and $\partial S'=c'$. Now $\phi$ is piecewise defined by $\sigma$ on $\Sigma\setminus(S\cup S')$, $\varphi$ on $S$ and $\varphi^{-1}$ on $S'$ (up to isotopy).
% Figure environment removed
\end{proof}

Let us first deal with the (non) extreme amenability of $\MCG(\Sigma)$ for surfaces $\Sigma$ with genus $0$ by using the results above.
\begin{lem}\label{lem_non_comparable}
Let $\Sigma$ be an orientable surface with genus $0$ and complexity at least $2$. If $\MCG(\Sigma)$ is extremely amenable, then any two distinct ends in $\End(\Sigma)$ are non-comparable.
\end{lem}
\begin{proof}
Suppose for contradiction that $\MCG(\Sigma)$ is extremely amenable but there exist two distinct $x,y\in \End(\Sigma)$ such that $x\preccurlyeq y$. Since $\End(\Sigma)$ is metrisable and \emph{a fortiori} Hausdorff, we can take a clopen neighbourhood $N$ of $y$ so that $x\notin N$. By definition, inside of $N$, there exists a homeomorphic copy of a clopen neighbourhood of $x$ but excludes $x$. This implies the existence of $x'\in \End(\Sigma)$ which is distinct from $x$ but equivalent to $x$. Moreover, one can take a clopen neighbourhood $D$ of $x$ and a clopen neighbourhood $D'$ of $x'$ in the way that $D\simeq D'$ but $D\cap D'=\emptyset$. We should note that $\End(\Sigma)\setminus D$ is also homeomorphic to $\End(\Sigma)\setminus D'$. Now associate $D$ to a subsurface with boundary $S\subset \Sigma$ as described above so that $\End(S)=D\sqcup\{\ast\}$ and find a subsurface $S'\subset \Sigma$ likewise for $D'$. By Richards' theorem, the curves $\partial S$ and $\partial S'$ have the same topological type and satisfies the hypothesis of Lemma \ref{lem_non_ea_1}. But Lemma \ref{lem_non_ea_1} indicates that there exists a mapping class $\phi\in\MCG(\Sigma)$ such that if $F=\{\partial S,\partial S'\}$, then $\phi\in \MCG(\Sigma)_F\setminus \MCG(\Sigma)_{(F)}$. This yields a contradiction to Lemma \ref{lem_stab}.
\end{proof}

Now one can show the following proposition:
\begin{prop}\label{prop_genus_0}
Let $\Sigma$ be an orientable surface with genus $0$. Then $\MCG(\Sigma)$ is extremely amenable if and only if $\Sigma$ is a sphere or a once-punctured sphere, in which case $\MCG(\Sigma)$ is the trivial group.
\end{prop}
\begin{proof}
If the surface has complexity less than $2$, then it has non-trivial discrete group as mapping class group unless it is a sphere or a once-punctured sphere. So by the virtue of Lemma \ref{lem_non_comparable}, it remains to show that there is no surface with genus $0$, complexity at least $2$ and pairwise non-comparable ends unless it is a sphere or a once-punctured sphere, where the statement is satisfied vacuously. Indeed, let $\Sigma$ be a such surface. Assume that there exist distinct $x,y\in\End(\Sigma)$ and an element $g\in\MCG(\Sigma)$ so that $gx=y$. Then necessarily $x\preccurlyeq y$ as remarked above, which contradicts to the non-comparing assumption. Hence the action of $\MCG(\Sigma)$ on $\End(\Sigma)$ must be trivial. But in this case, the group $\MCG(\Sigma)$ is the homeomorphism group of $\End(\Sigma)$. As a result, $\MCG(\Sigma)$ is trivial and this can only occur when $\Sigma$ is a sphere or a once-punctured sphere.
\end{proof}

Although it is readily seen that the mapping class group of a torus is not extremely amenable as it is a non-trivial discrete group, the following observation is worth mentioning:
\begin{lem}\label{lem_torus}
A torus has two non-isotopic simple closed curves $c,c'$ with the same topological type and there is a mapping class $\phi\in \MCG(\Sigma)$ such that $\phi(c)=c'$, $\phi(c')=c$ and $\phi^4=\Id$.
\end{lem}
\begin{proof}
It has two non-isotopic simple closed curves that are non-separating, \emph{i.e.} a curve $c$ so that $\Sigma\setminus c$ is still connected. See Figure \ref{fig:torus}. Now identify $\mathbb{T}^2=\mathbb{R}^2/\mathbb{Z}^2$. Then the element 
\[\phi\coloneqq\begin{pmatrix}0 & -1\\ 1& 0\end{pmatrix}\in \mathrm{SL}_2(\mathbb{Z})\simeq \MCG(\mathbb{T}^2)\]
is the desired mapping class.
\end{proof}
% Figure environment removed

Adapting the arguments of Lemma \ref{lem_torus}, one can now prove Theorem \ref{thm1}:

\begin{proof}[Proof of Theorem \ref{thm1}]
If the surface $\Sigma$ has genus $0$, then the extreme amenability of $\MCG(\Sigma)$ is determined by Proposition \ref{prop_genus_0}. For the torus, the mapping class group is not extremely amenable since it is a non-trivial discrete group. Suppose now that the surface $\Sigma$ has complexity at least $2$ and non-zero genus. If $\Sigma$ contains an essential (sub)surface that is homeomorphic to a once-punctured torus $S_{1,1}$, \emph{i.e.} the genus $g(\Sigma)\geq 2$ or $g(\Sigma)=1$ and $\End(\Sigma)\neq \emptyset$, then the same arguments for Lemma \ref{lem_torus} \emph{verbatim} yields two distinct curves $c,c'$ on $\Sigma$ and a mapping class $\phi\in\MCG(\Sigma)$ of order $4$ interchanges the position of these two curves. To be rigorous, let $c,c'$ be two distinct non-separating simple closed curves on $S_{1,1}\subset \Sigma$. By cutting $\Sigma$ into $S\cup S_{1,1}$ along a simple closed curve encircling all ends of $\Sigma$, one can define
\[\phi\coloneqq\begin{pmatrix}0 & -1\\ 1& 0\end{pmatrix}\in \mathrm{SL}_2(\mathbb{Z})\simeq \MCG(S_{1,1}),\]
which will further yield a mapping class $\widetilde{\phi}\in \MCG(\Sigma)$ via taking an extension by identity of $\overline{\phi}\in\MCG(S_{1,1})$. Note that if $F=\{c,c'\}$, then $\widetilde{\phi}\in \MCG(\Sigma)_{F}\setminus \MCG(\Sigma)_{(F)}$. Hence by Lemma \ref{lem_stab}, $\MCG(\Sigma)$ is not extremely amenable.
\end{proof}

Finally, there are still some details in Theorem \ref{thm2} that need further clarifications.
\begin{lem}\label{lem_dense}
Let $G$ be a topological group and $H$ be a dense subgroup of $G$. Then $G$ is extremely amenable if and only if $H$ is.
\end{lem}
\begin{proof}
Note that a topological group $G$ is extremely amenable if and only if it admits a $G$-invariant multiplicative mean over $\mathrm{RUCB}(G)$, the space of right-uniformly continuous functions on $G$, or equivalently the $G$-action on its \emph{Samuel compactification} $\sigma G$ has a fixed point (see for example \cite[\S 1.1]{pestov2006dynamics}). But if $H$ is a dense subgroup of $G$, then $\mathrm{RUCB}(H)\simeq \mathrm{RUCB}(G)$ and any continuous $H$-action on the multiplicative means of $\mathrm{RUCB}(G)$ can be extended continuously to a $G$-action on it. Conversely, the restriction of a continuous $G$-action on $H$ yields an $H$-action. Hence $H$ and $G$ can only be simultaneously extremely amenable.
\end{proof}

\begin{proof}[Proof of Theorem \ref{thm2}]
For any subgroup $G<\MCG(\Sigma)$ with complexity $c(\Sigma)>2$ that contains a periodic mapping class $\phi$ of order $n>1$, it is clear that $\phi$ belongs to $\overline{G}_{F}$ but not $\overline{G}_{(F)}$, for some curve $c\in \mathcal{C}(\Sigma)$ and a finite collection $F=\{c,\phi(c), \cdots, \phi^{n-1}(c)\}$. Hence the closure of such subgroup $G$ in $\MCG(\Sigma)$ is not extremely amenable. It follows from Lemma \ref{lem_dense} that $G$ itself is neither extremely amenable.
\end{proof}
