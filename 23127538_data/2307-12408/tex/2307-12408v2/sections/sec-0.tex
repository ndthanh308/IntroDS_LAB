\section{Introduction}
Let $\Sigma$ be an orientable surface. A theorem of Richards \cite{richards1963classification} asserts that it is characterised by the following parameters: genus and the space of ends. Let $g(\Sigma)$ be genus of the surface $\Sigma$ and $n(\Sigma)$ be the cardinality of its ends. The complexity of the surface is defined as $c(\Sigma)\coloneqq 3g(\Sigma)+n(\Sigma)-3$. Such a surface is said to have \emph{finite type} if $c(\Sigma)<\infty$, otherwise it is of \emph{infinite type}.

Define the mapping class group of $\Sigma$ by $\MCG(\Sigma)\coloneqq\mathrm{Homeo}^+(\Sigma)/\{\mathrm{isotopy}\}$, where $\mathrm{Homeo}^+(\Sigma)$ is the group of orientation preserving automorphisms of the surface, equipped with compact-open topology, and $\MCG(\Sigma)$ is carrying the quotient topology. With this topology, the group $\MCG(\Sigma)$ is a \emph{non-archimedean Polish group} (see Proposition \ref{prop_non-arch}), or equivalently it is a closed subgroup of $S_\infty$, the symmetric group of $\mathbb{N}$, with respect to the poitwise convergence topology (see for example \cite[\S 1.5]{becker1996descriptive}). A mapping class group is \emph{big} if the underlying surface is of infinite type.

From a model theoretic aspect, a non-archimedean Polish group can always be realised as the automorphism group of some countable first-order relational structure. In particular, one may ask the following question:
\begin{ques}\label{q1}
Given a countable first-order relational structure $\mathcal{F}$, how can one detect if its automorphism group $\mathrm{Aut}(\mathcal{F})$ can be realised as the (extended) mapping class group of an orientable surface?
\end{ques}

This paper tends to give a partial answer to Question \ref{q1}: the mapping class groups of all but finitely many orientable surfaces can never be the automorphism group of a countable first-order relational structure $\mathcal{F}$ such that $\mathrm{Age}(\mathcal{F})$ has Ramsey property.

Recall that a topological group $G$ is said to have \emph{fixed point on compacta property} or \emph{extremely amenable} if every continuous $G$-action on a compact Hausdorff space admits a fixed point, or equivalently the universal minimal $G$-flow $M(G)$ reduces to a singleton. However, it is worth noticing that, other than the trivial group, no locally compact groups are extremely amenable \cite{veech1977topological}, thus \emph{a fortiori} no discrete ones \cite{ellis1960universal}.

In a celebrated paper \cite{kechris2005fraisse}, Kechris, Pestov and Todor\v{c}evi\'{c} develop a surprising correspondence (\emph{abbrv.} KPT correspondence) between model theory, combinatorics and topological dynamics: if $\mathcal{F}$ is a structure with universe $\mathbb{N}$, then the non-archimedean Polish group $\mathrm{Aut}(\mathcal{F})$ is extremely amenable if and only if the \emph{age} $\mathrm{Age}(\mathcal{F})$ has Ramsey property. So it is a natural question to ask for a non-archimedean Polish group arising in the study of a geometry object whether or not it is extremely amenable. In particular, it is worth knowing whether big mapping class groups are extremely amenable or not. One shall notice that big mapping class groups are not locally compact \cite[Theorem 4.2]{aramayona2020big}.

\begin{thm}\label{thm1}
Let $\Sigma$ be an orientable surface of finite or infinite topological type. Then $\MCG(\Sigma)$ is not extremely amenable unless $\Sigma$ is a sphere or a once-punctured sphere, in which cases the mapping class groups are trivial.
\end{thm}

Denote by $\End(\Sigma)$ the end space of a surface $\Sigma$. It is a compact space with a natural continuous $\MCG(\Sigma)$-action. In many cases, this action is fixed-point free, which will witness the non extreme amenability of the mapping class group. But this is not always the case. For example, the \emph{Loch Ness monster} surface has infinite genus but only one end and the action of its mapping class group on the end space is trivial. Other non trivial examples are (non self-similar) surfaces with a unique maximal end \cite{mann2022results}.

The proof of Theorem \ref{thm1} mainly relies on the description of extremely amenable non-archimedean Polish groups provided in \cite{kechris2005fraisse} and can be divided into the discussion of two cases, depending on whether the surface is zero-genus or not. When the surface $\Sigma$ has zero genus, the Mann-Rafi ordering \cite{mann2020large} yields two disjoint curves of the same topological type unless $\Sigma$ is a sphere or a once-punctured sphere. In this case, the existence of these two curves further indicates that $\MCG(\Sigma)$ is not extremely amenable. If $g(\Sigma)>0$, then one may also find another pair of curves with the same topological type, not necessarily disjoint, of which the existence also implies the non extreme amenability of $\MCG(\Sigma)$.

Moreover, the proof of Theorem \ref{thm1} remains valid for closed subgroup of $\MCG(\Sigma)$ containing a non-trivial mapping class with finite orbit, including pure mapping class groups $\PMCG(\Sigma)$, \emph{i.e.} the subgroup in $\MCG(\Sigma)$ that fixes pointwise every ends, and the closure of compactly supported mapping class group $\MCG_c(\Sigma)$. 
\begin{thm}\label{thm2}
Let $\Sigma$ be an orientable surface of finite or infinite type. If $G<\MCG(\Sigma)$ is a subgroup containing a mapping class $\phi\in G$ such that for some simple closed curve $c$ on $\Sigma$, the orbit $\{\phi^n(c):n\in\mathbb{Z}\}$ is finite, then $G$ is not extremely amenable. In particular, the groups $\PMCG(\Sigma)$ and $\MCG_c(\Sigma)$ of a surface $\Sigma$ with $g(\Sigma)\geq 1$ are not extremely amenable.
\end{thm}

Note that the non extreme amenability of $\PMCG(\Sigma)$ can also be shown in a way that one constructs a $\PMCG(\Sigma)$-action on the circle $S^1$ without fixed-point from the homomorphisms built in \cite{aramayona2020first}. But this method cannot be adapted to $\MCG_c(\Sigma)$.