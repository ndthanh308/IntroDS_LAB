\documentclass[12pt,a4paper,reqno]{amsart}
\usepackage{xypic,a4wide,amsmath,amssymb,amsthm}
\usepackage[utf8x]{inputenc}
%opening

\usepackage[T1]{fontenc}
%\usepackage[english, french]{babel}


\usepackage{amssymb}
\usepackage{amsmath}

%url,xspace, smfthm

\usepackage[dvips]{graphicx}
%\usepackage{graphicx}
\usepackage{array}



\title[\ ]{Quantum   Painlev\'e  II  Lax Pair and Quantum (Matrix) analogues of Classical  Painlev\'e  II equation }

\author{Irfan Mahmood}
\address{
College of Science, Shanghai University, China and  CHEP, University of the Punjab, 54590 Lahore,  Pakistan} \email{mahirfan@yahoo.com} \urladdr{}

%\keywords{ Convolution product, Resurgent algebra}

%\subjclass{34M37, 30Hxx}




%\date{Received date / Revised version date}
% The correct dates will be entered by the editor

\setlength{\textheight}{220mm} \setlength{\textwidth}{140mm}
%\setlength{\baselineskip}{13pt}
%\setlength{\parindent}{10pt}
%\setlength{\parskip}{5pt plus 2pt
\begin{document}


\maketitle

\begin{abstract}
%% Text of abstract
 In this article, a new  quantum Painlev\'e II  Lax pair is presented which explicitly   involves the Planck constant $\hbar$ and an arbitrary field variable $v$ so these two objects make this new pair different from Flaschka-Newell Painlev\'e II  Lax pair. It is shown that the compatibility of quantum Painlev\'e II  Lax pair simultaneously yields a   quantum Painlev\'e II equation and a quantum commutation relation between field variable $v $ and independent  variable $z$. It is also manifested that with different choices of arbitrary field variable  system reduces to Non-comuutative  Painlev\'e II,  derivative matrix  Painlev\'e II equation  and to its classical analogue. Further, the gauge equivalence of quantum Painlev\'e II  Lax pair is derived whose compatibility condition gives quantum p34 equation which reduces to its classical analogue under classical limit as  $\hbar \rightarrow 0 $.
 \end{abstract}
\section{Introduction}

 Among the Painlev\'e  equations   \cite{PP}  the classical Painlev\'e   second (PII) equation is an  only one parametric  equation  therefore it can be regarded  as a primary object in hierarchy of  Painlev\'e   equations to understand  their  algebraic and geometrical aspects which vary by changing of parametric values. In theory of Painlev\'e  equations number of remarkable developments on PII equation have been achieved for example, very initial  results concerning its solutions  have been studied  in \cite{YAB, VOR, HAR}  where it has been proved that the PII equation possesses rational solutions  for integer values of parameter $\alpha$ and expressed in terms of Yablonski Vorob’ev polynomials  by\cite{YAB, VOR}. Where as for the  half odd-integers  values of parameter $\alpha$ , PII admits  Airy's type solutions \cite{HAR} and  also  owns the Backlund transformation that relates two its different solutions  through the  parameter  $\alpha$.  Subsequently, Flaschka and Newell \cite{FN}   expressed
  the rational solutions of PII equation as the logarithmic derivatives of determinants. In addition to that  Kajiwara and Ohta \cite{KO}  generalised  PII solutions in devisme polynomial determinant and as well as in terms of Hankel determinant. \\
 Painlev\'e equations are also  regarded as completely integrable equations as mentioned by  \cite{r5, r11, r12}  that they admit  linear representations, possess Hamiltonian structures. and obeyed the Painlev\'e test.


One of the  very interesting aspects of these equations is their appearance as ordinary differential reduction of some integrable systems,for example in \cite{r4, MA}  has been shown the the ODE reduction of the KdV equation is Painlev\'e II (PII) equation. The classical PII equation 
 \begin{equation}\label{CPII} 
 u^{''} = 2 u^{3}-zu +  \alpha\\
  \end{equation} 
  among the six  Painlev\'e  equation is an only one  parametric system and has been regarded as simplest model to understand Painlev\'e  transcends in background of parameters. \\
 In the context of derivation of its various analogues, very initially   its derivative matrix version presented in \cite{OS} as the dimensional reduction of matrix mKdv equation through the scale transformation and subsequently another  direct matrix (quantum) version studied \cite{NH, NGR} with its partner equation P34 that involves Planck constant and  gives the sense of quantization of PII  equation which does not conatain the Planck constant explicitly.  After that its most advance version as Non-commutative (NC) analogue presented by Retakh and Rubtsove  \cite{7}  which possesses anti-commutation term between field variable $u(z; \alpha)$ and independent varible $z$ but does not carry explicit expression to manifest commutation relation between these variables. Subsequently its Darboux solutions with its non-commutative Toda equation for $n=1$ derived in  \cite{MIRFAN, IM}  in terms of quasideterminants  \cite{GelRet}. \\
  In this article, a new  quantum Painlev\'e II  Lax Pair is presented that directly involves the Planck constant $\hbar$ and an arbitrary field variable $v$ so these two objects make this new pair different from Flaschka-Newell Painlev\'e II  Lax pair. It is shown that the compatibility of quantum Painlev\'e II  Lax Pair simultaneously yields a   quantum Painlev\'e II equation and a quantum commutation relation between field variable $v $ and independent  variable $z$. it is also manifested with different choices of arbitrary field variable  system reduces to Non-comuutative  Painlev\'e II,  derivative matrix  Painlev\'e II equation  and to its classical analogue. Further, the gauge equivalence of quantum Painlev\'e II  Lax pair is derived whose compatibility condition gives quantum p34 equation which reduces to its classical analogue under classical limit as  $\hbar \rightarrow 0 $.

 
    \section{Different Analogues of Classical PII equation}
  This section encloses a brief  review on various analogues of Classical PII equation as its matrix  and non-commutative   versions.
    \subsection{Classical PII equation}
    The classical PII equation  (\ref{CPII}) initially  was proposed by P. Painlev\'e as one of the member of six Painlev\'e  equation whose solutions possess parametric dependence expect PI equation, here classical means field variable $u(z; \alpha)$ and variable $z$ are scalars. 
The classical PII equation is integrable as its possesses linear representation \cite{FN}  and arises from the compatibility of following linear system
\begin{equation}\label{LPII} 
\Psi_{z}=U(z;\lambda)\Psi  , \; \; \;   \Psi_{\lambda}=V(z;\lambda)\Psi  
\end{equation}

 with matrices $U$ and $V$  as
\begin{equation}\label{GLP} 
\left\{
\begin{array}{lr}
U =  -i \lambda \sigma_{3}  + u \sigma_{1}  \\
A= -i (4 \lambda^{2} +z + 2u^{2} ) \sigma_{3} +(4 \lambda u - \frac{\alpha}{\lambda} ) \sigma_{1} - 2v \sigma_{2}
\end{array}
\right.
 \end{equation}
 here $\Psi$ is arbitrary two component column vector and    $\sigma_{j} $ are the Pauli spin matrices,
  $\sigma_{1} =  \begin{pmatrix}
0 & 1 \\
1 & 0
\end{pmatrix}$, 
   $\sigma_{2} = \begin{pmatrix}
0 & -i \\
i & 0
\end{pmatrix} $,   $\sigma_{3} = \begin{pmatrix}
1 & 0 \\
0 & -1
\end{pmatrix} $
where the matrices $U$ and $V$ are called the Flaschka-Newell Lax pair. \\
\textbf{Remark 1.1.}{Gauge Equivalence of Flaschka-Newell Lax Pair }\\
The compatibility of  following linear system 
\begin{equation}
    \frac{\partial \Psi}{\partial \eta}=A \Psi,\hspace{1cm}\frac{\partial \Psi}{\partial z}=B \Psi
\end{equation}
with matrices
\begin{equation}
    A=\begin{pmatrix}
        2u+\frac{\alpha+1/2}{2\eta}& 2i \eta+i q\\
        2i+\frac{i \sigma}{\eta}&-2u-\frac{\alpha+1/2}{2\eta}
    \end{pmatrix}
\end{equation}
\begin{equation}
    B=\begin{pmatrix}
        u&i\eta \\
        i &-u
    \end{pmatrix}
\end{equation}
 gives rise to following set of equations
\begin{equation}\label{cspII} 
\left\{
\begin{array}{lr}
 q^{'}= 2qu -\alpha+\frac{1}{2} \\
 r^{'}= -2ru + \alpha+\frac{1}{2}\\
u^{'}= \frac{1}{2}(q -r)
\end{array}
\right.
 \end{equation} 
 where $\eta=\lambda^2$ and  matrices $A$ and $B$  \cite{AAAV} are the gauge equivalence of Flschka-Newel Lax pair. On eliminating $u$ this can be shown that $r$ and $q$ satisfy P34 equation respectively. 
 \begin{equation}\label{q34} 
 r_{zz}=   \frac{r^{2}_{z}}{r}+2r^{2}-zr- \frac{1}{2r}(\alpha +  \frac{1}{2} )^{2}
\end{equation}
and 
 \begin{equation}\label{p34} 
q_{zz}=   \frac{q^{2}_{z}}{q}+2q^{2}-zq- \frac{1}{2q}(\alpha -  \frac{1}{2} )^{2}.
\end{equation} 
 \subsection{Derivative Matrix PII equation}
In theory of integrable systems it has been found  that the symmetry reductions  of various  integrable systems resulting the  ordinary differential equations which are Painlev\'e  equations.  In context of ordinary reduction of integrable systems in  non-commutative  settings its has been shown by Olver and Sokolov \cite{OS} that the symmetry reduction of Matrix mKdV equation
\begin{equation}
    v_t=v_{xxx}+3[v,v_{xx}]_{-}-6vv_xv
\end{equation}
with transformation
\begin{equation}
    v(x,t)=u(z)t^{-1/3}
\end{equation}
gives rise to ODE as follow
\begin{equation}
    u^{'''}=3u^{''}u-3uu^{''}+6uu^{'}u-\frac{1}{3}u-\frac{1}{3}zu^{'}
\end{equation}
which is derivative matrix PII equation that reduces to its classical analogue in scalar case.  
   \subsection{Matrix PII equation}
   The  Matrix (Quantum) analogue  of the classical Painlev\'e II equation derived in  \cite{NH, NGR}   with help Painlev\'e II  symmetric form (\ref{cspII}) writting  nonabelian form as below
   \begin{equation}\label{spII} 
\left\{
\begin{array}{lr}
 q^{'}= uq+ qu -\alpha+\frac{1}{2} \\
 r^{'}= -ru-ur + \alpha+\frac{1}{2}\\
u^{'}= \frac{1}{2}( q -r)
\end{array}
\right.
 \end{equation} 
 where $u$ satisfies matrix  Painlev\'e II equation
  $ u^{''}=2u^3-zu+\alpha_1-\alpha_0$ 
 and $q$, $r$ are the solutions of P34 equation which explicitly involve Planck constant, for example for $q$  the quantum P34 equation  can be obtained as follow
\begin{equation}
    q^{''}=\frac{1}{2}q^{'}q^{-1}q^{'}-4q^2+2zq-\frac{1}{2}(\alpha_1^2-\hbar^2)q^{-1}
\end{equation}
The fields in non-abalian Painlev\'e II  symmetric form  subjected to quantum commutation relations as 
  \begin{equation}\label{qcr} 
[r,q]_{-}=2 \hbar u, \hspace{0.5cm} [u,q]_{-}=   [u,r]_{-}=\hbar 
 \end{equation}
 under the  affine  Weyl group symmetry of type $ A^{1}_{l}$.
Here field variables $u$,$q$ and $r$ are matrices and variable $z$ treated as scalar commuting object with field variables, therefore mathematical forms  of classical PII equation and f Matrix PII equation look similar.
   \subsection{Non-commutative PII equation}
 In the context of extension of classical  Painlev\'e equations to non-commutative spaces, a very initial achievement in this  direction was obtained by V. Retakh and V. Roubtsov in \cite{7} where non-commutative analogue of classical Painlev\'e II equtaion obtained through non-abalian Painlev\'e II  symmetric form connected to  noncommutative Toda chain. The non-commutative Painlev\'e II equation derived in folowing form
 
 \begin{equation}\label{VRVR} 
  u^{''} = 2 u^{3}- 2[z,u]_{+} + 4 ( \beta + \frac{1}{2} ) 
 \end{equation} 
 
here $[z,u]_{+}$ is the anti-commutation relation between field variable $u$ and variable $z$ which gives the pure sense of non-commutativity but still here we do not have the explicit commutation relation between field variable $u$ and variable $z$ in non-commutative settings.
\section{Quantum PII  linear system}
This section includes the presentation of new Lax pair, Quantum Painlev\'e II  Lax Pair, which directly involves the Planck constant and an arbitrary field variable. In subsequent proposition$2.1$,  it is shown that the compatibility of Quantum Painlev\'e II  Lax Pair simultaneously yields   Matrix Quantum Painlev\'e II equation and quantum commutation relation between field variable $u(z; \alpha) $ and variable $z$. Further it is elaborated with different choices of arbitrary field varible the system reduces to Non-comuutative quantum Painlev\'e II,  derivative matrix  Painlev\'e II equation \cite{OS} and to its classical analogue.   The Quantum Painlev\'e II  Lax Pair incorporates two addition objects which make this new pair different from Flaschka-Newell Lax pair, as the arbitrary field variable $v(z)$ and Planck constant explicitly. The arbitrary field can be chosed in four different ways and the resulting linear system consistants with Painlev\'e II  equation equation under classical limit, briefly  four cases are as   (i) if the arbirary field variable is take as $v=u^{'}$ , we get non-comuutative quantum Painlev\'e II equation with quantum commutation relation $zu-uz=-\frac{i}{2} \hbar \int u dz$ , (ii) for arbitrary field variable $v=u$ the system gives rise to derivative matrix Painlev\'e II  equation with quantum commutation relation $zu-uz=-\frac{i}{2} \hbar u$. , (iii) under the classical limit with $v(z)=u(z; \alpha)$ as scalar we obtain a new classical Painlev\'e II  Lax Pair, one the member  of that pair possesses additional term which makes it different from  Flaschka-Newell  Painlev\'e II  Lax pair and its compatibility consistants with  classical Painlev\'e II equation  (\ref{CPII})
, (iv) under the classical limit with $v(z)=0$ the Quantum Painlev\'e II  Lax Pair reduces to Flaschka-Newell  Painlev\'e II  Lax pair .


\subsection*{Proposition 2.1}
The compatibility  condition of following linear system
\begin{equation} \label{QPL}
    \Psi^{'}=P\Psi,\hspace{1cm}\Psi_\lambda=Q\Psi
\end{equation}
with matrices
 \begin{equation}\label{QPM} 
\left\{
\begin{array}{lr}
  P=u\sigma_1-i \lambda \sigma_3+4vI \\
    Q=-(4 i \lambda^2+i z+2u^2)\sigma_3+(4 \lambda u-\frac{\alpha}{\lambda})\sigma_1-(2u^{'}-i \hbar)\sigma_2
\end{array}
\right.
 \end{equation} 
 simultaneously yields 
 \begin{equation}\label{QMPII} 
\left\{
\begin{array}{lr}
   u^{''}=2u^3-\frac{1}{2}[z,u]_{+}+4[v,u^{'}]_{-}+\alpha\\
    zv-vz=-\frac{i}{2} \hbar u
\end{array}
\right.
 \end{equation} 
here $I$ is identity matrix of order $2$ and $\hbar$ is Planck constant, $v(z)$ is arbitrary field variable.\\
\textbf{Proof:}\\
This can be shown that  from linear  system (\ref{QPL}) we can calculate $  ( \Psi^{'} )_\lambda =   ( \Psi_\lambda)^{'} $  in following form
\begin{equation}\label{ZC4} 
 Q_{z}-P_{\lambda}= [P,Q]_{-}.
\end{equation} 
 We can easily  evaluate the values for  $ Q_{z}$, $P_{\lambda}$ and $[P,Q]_{-} = PQ -QP$ from the linear system (\ref{QPL}) as follow
\begin{equation}\label{V1} 
 Q_{z} = -i(  2u^{'} u + 2u u^{'}  +1 )\sigma_{3} -2u^{''} \sigma_{2} + 4 \lambda u^{'}\sigma_{1} 
\end{equation} 

\begin{equation}\label{V2} 
 P_{\lambda} =  -i \sigma_{3}  
\end{equation} 
and now
\begin{equation}\label{V3} 
 Q_{z}- P_{\lambda} = \begin{pmatrix}
-2i [u ,u^{'} ]_{+}  & 4 \lambda u^{'} + 2iu^{''}\\
4 \lambda u^{'} -2iu^{''} & 2i [u ,u^{'} ]_{+}
\end{pmatrix}  
\end{equation}
\begin{equation}\label{V4} 
[ P,Q ]_{-} =   \begin{pmatrix}
 i [z, v]_{-} -2i  [u, u^{'}]_{+}  -\frac{1}{2}\hbar u  &  \delta^{+}  \\
  \delta^{-}  &   -i [z, v]_{-} +2i  [u, u^{'}]_{+}  +\frac{1}{2}\hbar u  
\end{pmatrix}
\end{equation} 

where
\[   \delta^{+} = 4\lambda u^{'}+ 4iu^{3}+i[z,u]_{+} + 2i \alpha  + 2i [v, u^{'}]_{-}- 2i \lambda   \hbar.\]

and 
\[  \delta^{-} = 4\lambda u^{'} -4iu^{3}- i[z,u]_{+} - 2i \alpha  -2i \hbar [v, u^{'}]_{-}- 2i\lambda   \hbar.\]
now  substituting the values of $ L_{z}- P_{\lambda} $ and $[ P,L ]_{-} $from (\ref{V3}) and (\ref{V4}) into zero curvature condition equation  (\ref{ZC4}) and after some simplification, then equating the corresponding elements of resulting matrices on both side, we get

  
\begin{equation}\label{L4} 
 [ z,v] = -\frac{1}{2}i  \hbar u
\end{equation} 
and
\begin{equation}\label{L5} 
u^{''} = 2u^{3} -\frac{1}{2}[z,u]_{+} +  \alpha  + [v, u^{'}]_{-}- \lambda   \hbar
\end{equation} 
\begin{equation}\label{L6} 
u^{''} = 2u^{3}-\frac{1}{2}[z,u]_{+} +  \alpha  + [v, u^{'}]_{-}+ \lambda   \hbar.
\end{equation} 
Now adding  (\ref{L5}) and (\ref{L6}) we obtain 
\begin{equation}\label{L7} 
u^{''} = 2u^{3}-\frac{1}{2}[z,u]_{+}   + [v, u^{'}]_{-} +  \alpha
\end{equation} 
\subsection{Case-i}  {Taking $v=u^{'}$ }\\
With the choice of $v=u^{'}$ Quantum Matrix Painlev\'e II  (\ref{QMPII})  reduces to the following form 
 \begin{equation}\label{QPII} 
\left\{
\begin{array}{lr}
   u^{''}=2u^3-\frac{1}{2}[z,u]_{+}+\alpha\\
    zu-uz=-\frac{i}{2} \hbar \int u dz
\end{array}
\right.
 \end{equation} 
 the last expression in above equation (\ref{QPII})  shows the quantum commutation relation between independent variable $z$ and field variable $u$. 
\subsection{Case-ii} {Matrix field $v=u$  Derivative matrix PII equation:}\\
Taking  derivation  of Quantum matrix PII equation  (\ref{QMPII}) with respect to $z$, we gat  
\begin{equation}
    u^{'''}=(2u^3)^{'}- \frac{1}{2}[2uz-\frac{i}{2}\hbar u]^{'}+4[u,u^{'}]_{-}^{'}
\end{equation}
\begin{equation}
    u^{'''}=2u^2 u^{'}+2u^{'}u^2+2uu^{'}u-\frac{1}{2}[2u+2zu^{'}-\frac{i}{2}\hbar u^{'}]+4[u,u^{''}]_{-}
\end{equation}
or
\begin{equation}
    u^{'''}=2u^2u^{'}+2u^{'}u^2+2uu^{'}u-u-(z-\frac{i}{4}\hbar )u^{'}+4[u,u^{''}]_{-}
\end{equation}
now introducing  new field variable $\nu(x)=u(z)$  where  $x=z-\frac{i}{4}\hbar $  in above expression, we obtain
\begin{equation}
    \nu^{'''}=2\nu^2\nu^{'}+2\nu^{'}\nu^2+2\nu \nu^{'}\nu-\nu-x\nu^{'}+4[\nu,\nu^{''}]_{-}
\end{equation}
above equation is not exactly but  similar to derivative matrix Painlev\'e II equation \cite{OS} obtained dimensional reduction of matrix mKdV equation which differs by two additional terms  $2\nu^2\nu^{'}+2\nu^{'}\nu^2$ but under the classical limit both coincide. 

\subsection{Case-iii} {Quantum PII Lax pair  (\ref{QPM})\\  With arbitrary field variable  $v=0$ and  $\hbar \rightarrow 0 $} the compatibility condition of resulting Lax pair (Flaschk-Newell Pair) still yields the non-commutative analogue \cite{7} of standard classical Painlev\'e II equation  (\ref{CPII}) without explicit quantum commutation relation between $z$ and $u$. Here this has been demonstrated  that Flaschk-Newell Pair appears as  case of our newly presented  Quantum Painlev\'e II Lax pair  (\ref{QPM}).  
\subsection{Case-iv}  {under the classical limit as  $\hbar \rightarrow 0 $: }\\
Under the classical limit  $\hbar \rightarrow 0 $ as  the qunatum commutation relation vanishes and above system (\ref{QMPII})  reduces to its classical analogue  and compatibility condition  of pair  (\ref{QPM}) under this limit still consistants for the classical Painlev\'e II  equation,  where as the additional term $v=u^{'}$ at diagonal of $P$ makes that pair   different  from  Flschka-Newell Lax Pair in classical case, if we take $v=0$ that pair exactly reduces to    Flschka-Newell Lax Pair.
  \section{Gauge Equivalence of Quantum PII Lax Pair }
  \subsubsection{\textbf{Proposition 1.1}}
 The compatibility of gauge equivalent  Quantum  Painlev\'e II Lax  $\tilde{P}= GPG^{-1}$ and $\tilde{Q}= GQG^{-1}$ 
 \begin{equation}\label{PQT} 
\left\{
\begin{array}{lr}
\tilde{P}= u \sigma_3 -i \lambda  \sigma_2+4u I\\
\tilde{Q}= (4 \lambda u-\frac{\alpha}{\lambda})\sigma_3 -(4i \lambda^2+ \frac{1}{4}\hbar)\sigma_2+2pI_{+} - 2qI_{-} 
\end{array}
\right.
 \end{equation} 
 produces quantum non-ablian  set of three equation
      \begin{equation}\label{qspII} 
\left\{
\begin{array}{lr}
 p^{'}= vp-pv+ up+ pu - i\frac{1}{4}\hbar u  - \alpha+\frac{1}{2} \\
 q^{'}= qv-vq-uq -qu + i\frac{1}{4}\hbar u  + \alpha+\frac{1}{2} \\\\
u^{'}= \frac{2}{2}( p -q)
\end{array}
\right.
 \end{equation} 
 where  $p=u^2+ u^{'}+ \frac{z}{2}$ , $q=u^2- u^{'}+ \frac{z}{2}$ and   $ G= \frac{1}{ \sqrt{2}}\begin{pmatrix} 
-i & -i \\
-1 & 1
\end{pmatrix}$, 
  $ I_{+}= \begin{pmatrix}
0 & 1 \\
0 & 0
\end{pmatrix} $,  $I_{-}= \begin{pmatrix}
0 & 0 \\
-1 & 0
\end{pmatrix} $\\
\textbf{Proof:}\\
It is  straight forward to construct $\tilde{P}$ and $\tilde{Q}$ from  (\ref{QPM}) under  gauge transformations   $\tilde{P}= GPG^{-1}$ and $\tilde{Q}= GQG^{-1}$.\\

  The compatibility of linear system $\Psi^{'}=\tilde{P} \Psi,\hspace{1cm}\Psi_\lambda= \tilde{Q} \Psi$  gives rise to  the zero-curvature condition as $\tilde{Q} -\tilde{P}_{\lambda}= [\tilde{P},\tilde{Q} ]_{-}$   which implies with help of matrices (\ref{PQT}) the set of equations (\ref{qspII}) as the quantum non-abelain analogue of  (\ref{cspII}).\\
 Now from system (\ref{cspII}) first equation with arbitrary field $v=u^{'}$ can be written 
 \begin{equation}\label{qpII} 
p^{'}= 2up - i\frac{1}{4}\hbar u  - \alpha+\frac{1}{2} \\
 \end{equation} 
or 
 \begin{equation}\label{fqspII} 
p^{'}= 2u(p - \frac{\beta}{2} )  - \delta 
 \end{equation}
 here $\beta =  i\frac{1}{4}\hbar $, $\delta= \alpha-\frac{1}{2}$ and if we take $\mathbf{ p} =p - \beta$ then above expression can be written as
  \begin{equation}\label{fqpII} 
u= \mathbf{p^{'} }\mathbf{p^{-1} } + \delta \mathbf{p^{-1} } 
 \end{equation}
 with 
    \begin{equation}\label{fp} 
    \mathbf{p}=  u^{2} +  u^{'}+\frac{z}{2} - \frac{\beta}{2} 
 \end{equation}
   
 It is straight forward to calculate  $u^{'}$ and  $u^{2}$  from expression (\ref{fqpII}) in following form 
  \begin{equation}\label{uup} 
\left\{
\begin{array}{lr}
 u^{'}= -\frac{1}{2}\mathbf{p^{'} }\mathbf{p^{-1} }\mathbf{p^{'} }\mathbf{p^{-1} }+ \frac{\delta }{2}\mathbf{p^{-1} }\mathbf{p^{'} }\mathbf{p^{-1} }\\
  u^{2}= \frac{1}{4}\mathbf{p^{'} }\mathbf{p^{-1} }\mathbf{p^{'} }\mathbf{p^{-1} }+ \frac{\delta }{4}\mathbf{p^{'} }\mathbf{p^{-2} }+\frac{\delta }{4}\mathbf{p^{-1} }\mathbf{p^{'} }\mathbf{p^{-1} }+\frac{\delta^{2} }{4}\mathbf{p^{-2} }
 \end{array}
\right.
 \end{equation} 
 Now substituting the values of $u^{'}$ and  $u^{2}$ into  expression  (\ref{fp}) and then after doing some simplification 
  \begin{equation}\label{uup34} 
 \mathbf{p}^{''} = -\frac{1}{2}\mathbf{{p^{'} }^{2} } + 2\mathbf{p^{2} }- \frac{\delta^{2} }{2}\mathbf{p^{-1} } - (z-\beta ) \mathbf{p } .
   \end{equation} 
 Here during simplification  the non-commutative definition of  logarithmic derivative  \cite{7}  is  used as  $\frac{d}{dz} ln \mathbf{p}= \mathbf{p^{'} }\mathbf{p^{-1} } $ or $\frac{d}{dz} ln \mathbf{p}= \mathbf{p^{-1} } \mathbf{p^{'} }$.
The above can be regraded as Non-abelian quantum P34 equation for  $\mathbf{p}$ which involves Planck constant with power $+1$ as $ \hbar $ which rescues its to be negligible as compare to  $ \hbar^{2} $  where as in  \cite{NH, NGR}  quantum P34 incorporates $ \hbar^{2} $  that is much smaller   then $ \hbar $  and  can be assumed  negligible as compare to $ \hbar $. Therefore the presence of Planck constant as $ \hbar $ in  (\ref{uup34}) strong  quantized version  of P34 equation as compare to the P34 equation possesses  Planck constant with higher positive powers as  $ \hbar^{2} $.   \\
It is straight  forward to see that under the classical limit as  $\hbar \rightarrow 0 $ quantum  P34 equation  (\ref{uup34})  reduces to its classical analogue obtained from  system (\ref{cspII}) which arises from compatibility  Flaschka-Newell  gauge equivalent Lax pair.

 \section{Conclusion}
In this work a new  quantum Painlev\'e II  Lax Pair has been  presented that directly involves the Planck constant $\hbar$ and an arbitrary field variable $v$. Its has been shown that compatibility condition produces quantum Painlev\'e II equation and a quantum commutation relation between field variable $v $ and independent  variable $z$ simultaneously where as its gauge equivalent pair generates quantum p34 equation, this has also been manifested  all these calculated results coincide to their classical analogues under $\hbar \rightarrow 0 $ as in classical case. For further motivation, it seems quite interesting to construct the quantum matrix analogue of classical mKdV equation $\And$ KdV equation from presented quantum p34 equation through reverse scale transformations. More interestingly to investigate the pure quantum analogue of nonlinear system of equations associated to Toda chain at $n=1$ with quantum commutation relations  and the obtained results for higher values of $n$ with the help of  quantum Painlev\'e II setting presented in this paper.


 \section*{Acknowledgement}
This research work has been completed as the part of Belt and Road Young Scientist project sponsored by Science and technology commission of Shanghai at college of science, Shanghai University with project, No. 20590742900. I am thankful to Shanghai University on providing me facilities and also my sincere thanks to the Punjab University 54590, Lahore for its partial financial support 
to complete this project at Shanghai University, China.


\noindent
\begin{thebibliography}{99}
 \bibitem{PP} P. Painlev\'e,  ``Sur les Equations Differentielles du Second Ordre et d’Ordre Superieur, dont l’Interable Generale est Uniforme'', Acta. Math., 25 (1902)1-86.
  \bibitem{YAB}Yablonskii AI. On rational solutions of the second Painleve equations. Vesti Akad Nauk BSSR, Ser Fiz Tkh Nauk 1959;3:30–5 [in Russian].
 \bibitem{VOR} Vorob’ev AP. On rational solutions of the second Painleve equations. Differen Equat 1965;1:79–81 [in Russian].
 \bibitem{HAR}H. Airault, ``Rational solutions of Painlev\'e equations'', Stud. Appl. Math. 61 (1979), 31-53.
   \bibitem{FN}H. Flaschka, A.C. Newell, Comm. Math. Phys. 76 (1980) 65–116.
   \bibitem{AAAV} A. S. Fokas, A. R. Its, A. A. Kapaev and V. Y. Novokshenov, Painlev\'e Transcendents: The Riemann-Hilbert Approach, The American Mathematical Society, (2006), ISSN 0076-5376; v.128.
  \bibitem{KO}Kajiwara K and Ohta Y 1996 Determinant structure of the rational solutions for the Painlev\'e II equation
J. Math. Phys. 37 4693–704
\bibitem{r5}A. N. W. Hone, Painlev\'e Test, Singularity Structure and Integrability, Lect. Notes Phys. \textbf{ 767} , 245-277 (2009).
\bibitem{r11}K. Okamoto, in: R. Conte (Ed.), The Painlev\'e property, One century later, CRM Series in Mathematical Physics, Springer, Berlin, (1999) 735-787.
\bibitem{r12}S. P. Balandin, V. V. Sokolov, On the Painlev\'e test for non-abelian equations, Physics letters, \textbf{A246} (1998) 267-272.
\bibitem{r4}N. Joshi, The second Painlev\'e hierarchy and the stationarty KdV hierarchy, Publ. RIMS, Kyoto Univ. \textbf{ 40}(2004) 1039-1061.
\bibitem{MA}N. Joshi, M. Mazzocco, Existence and uniqueness of tri-tronqu\'ee solutions of the second Painlev\'e hierarchy, Nonlinearity \textbf{ 16}(2003) 427--439. 
\bibitem{OS} P. J. Olver and V.  V. Sokolov, Integrable Evolution Equations on Associative Algebras, Commun. Math. Phys. \textbf{ 193}(1998) 245 – 268 
\bibitem{NH} H. Nagoya, Quantum Painlev\'e systems of type Al , Internat. J. Math. 15 (2004), 1007–1031,math.QA/0402281.
\bibitem{NGR} H. Nagoya, B. Grammaticos and A. Ramani, Quantum Painlev\'e equations: from Continuous to discrete, SIGMA \textbf{ 4} (2008), no. 051, 9 pages.
\bibitem{7}V. Retakh, V. Rubtsov, Noncommutative Toda chain , Hankel quasideterminants and Painlev\'e II equation, J. Phys. A: Math. Theor. \textbf{ 43}   (2010)  505204 (13pp)
\bibitem{MIRFAN} M. Irfan, Lax pair representation and Darboux transformation of noncommutative Painlev\'e's second equation,Journal of Geometry and Physics, \textbf{ 62} (2012) 15751582.
\bibitem{IM}  Irfan Mahmood, Quasideterminant solutions of NC Painlev\'e II equation with the Toda solution at $n=1$ as a seed solution in its Darboux transformation,  Journal of geometry and physics 95 (2015) 127-136.
\bibitem{GelRet} I. Gelfand, S. Gelfand V. Retakh, R. L. Wilson, Quasideterminants , Advances in Mathematics \textbf{ 193}  (2005) 56-141.


 \end{thebibliography}



%\bibliography{<your-bib-database>}



\end{document}

