\section{\texttt{Prometheus}: Output and Validation}
\label{sec:validation}

% Figure environment removed

\texttt{Prometheus} outputs to \texttt{Parquet}~\cite{parquet_docs} files that include two primary fields—\texttt{photons} and \texttt{mc\_truth}.
\texttt{photons} contains information on photons that produce hits in user-defined detection regions.
This includes photon arrival time, OM identification numbers, OM position, and the final-state particle that produced the photon. 
Fig. 1 shows event displays for various detectors generated using the information in \texttt{photons}.
\texttt{mc\_truth} includes information on the injection like the interaction vertex; the initial neutrino type, energy, and direction; and the final state types, energies, directions, and parent particles.
Users may also save the configuration file (see Sec.~\ref{sec:examples}) as a \texttt{json} file. 
This allows the user to resimulate events using the same parameters, which is useful for comparing the same event across multiple detector geometries.

The information stored in \texttt{mc\_truth} allows us to compute effective areas when combined with the weights from \texttt{LeptonWeighter}. 
Since it mainly depends on the physics implemented in \texttt{Prometheus}---such as neutrino-nucleon interactions, lepton range, and photon propagation---effective area serves as a reliable indicator of our code's performance when all simulation steps are properly integrated.
We can then validate our simulation by comparing these effective areas to published data.
Fig.~\ref{fig:effa} compares different experiments published effective areas to our estimation using \texttt{Prometheus} simulations. 
It is worth noting that the calculations for effective area rely on detector-specific cuts and OM response, to which we have limited or no access. 
We can therefore expect differences of $\mathcal{O}(10\%)$ from these missing detector details.

% Figure environment removed