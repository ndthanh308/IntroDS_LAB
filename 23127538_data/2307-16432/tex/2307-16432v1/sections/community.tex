\section{Community Contributions}
\label{sec:community}

% Figure environment removed

% Figure environment removed
\label{fig:angular_resolution}

Developing a published reconstruction is often a long-term effort internal to an experiment, meaning methods are slow to implement and difficult to compare.
\texttt{Prometheus} remedies this by allowing users to readily generate the large data sets necessary to test the feasibility and relative performance of different reconstruction methods.
In this section we highlight two submissions at ICRC 2023 that utilize \texttt{Prometheus} in such a way.
The first of these works is a software-focused effort that looks to improve the speed of first reconstructions at detectors and the second is a hardware-focused effort concerning GPU alternatives for low power computing. 

Ref.~\cite{Yu:2023pos}, proposes sparse-submanifold convolutions (SSCNNs) as an alternative to the convolutional neural network (CNN) and traditional trigger-level event reconstructions currently used in neutrino telescopes. 
Their directional and energy algorithm is trained on a data set of 412892 events and tested on a further 50000 events, cut from a set of 3 million generated by \texttt{Prometheus}. 
As seen in Fig.~\ref{fig:event_rates}, the SSCNN is capable of running at speeds comparable to the neutrino telescope trigger rates.
In Fig. 5, it achieves a median angular resolution below $4^\circ$ for the highest-energy trigger-level events, which matches or outperforms currently employed reconstructions.
Their work could be applied to improve on-site reconstructions and filtering for notable events at any detector. 

Ref.~\cite{Jin:2023}, is another \texttt{Prometheus} user submission that explores hardware accelerators to improving event reconstruction. 
Specifically, they look at how a Tensor Processing Unit (TPU) compatible algorithm could lower energy consumption while running comparable operations to the current GPU-based ones. 
Their model reaches a median angular resolution around $5^\circ$ above $10^3$ GeV. 
Meanwhile, approximate peak total power consumption drops from 100W on the best performing GPU (Apple M1 Pro chip) down to 3W on the Google Edge TPU.
The portion of the power consumption directed to the ML accelerator component drops from 15W to 2W.
This submission illustrates \texttt{Prometheus}'s utility in testing more speculative ideas and providing proof-of-concepts to encourage further work, particularly in the direction of machine learning.

The findings of both works are generic to any ice or water neutrino telescope.
Their models are trained and tested on example detector geometries rather than the geometry of any existing or proposed detector.
Since \texttt{Prometheus} is fully open-source, we hope it will facilitate not only easier iteration on these techniques but also greater collaboration in sharing and adapting them.

