\section{Examples}
\label{sec:examples}

In this section we will walk through producing a simulation of $\nu_{\mu}$ charged-current interactions in an ice-based detector.
This example will use mostly default values for injection parameters to show the essential steps in running a simulation, after which we will show how the user can set these parameters. 

In this example, we set the number of events to simulate, the detector geometry file, and---using the final state particles---the event type.
With just this, we can simulate events with \texttt{Prometheus}!

%
\begin{lstlisting}[language=python,upquote=true]
import prometheus
from prometheus import config, Prometheus

config["run"]["nevents"] = 100
geofile = f"{resource_dir}/geofiles/demo_ice.geo"
config["detector"]["geo file"] = geofile
injection_config["simulation"]["final state 1"] = "MuMinus"
injection_config["simulation"]["final state 2"] = "Hadrons"

p = Prometheus(config)
p.sim()
\end{lstlisting}
%

As briefly shown here, the \texttt{config} dictionary is our primary interface for configuring \texttt{Prometheus}. 
Key parameters not shown here include the output directory; random state seed; and injection parameters like injection angle and energy.
Information on a detector's medium, either ice or water, is stored in its geometry file.
The \texttt{geofiles} directory in the GitHub repository has geometry files for all of the detectors in Fig. 2. 
Again, this only scratches the surface of \texttt{Prometheus}'s capabilities.
For a more thorough description of all the options and features available see Ref.~\cite{Lazar:2023prm}, which has more in-depth examples for $\nu_{\mu}$ charged-current events in ice and $\bar\nu_{e}$ neutral-current events in water as well as directions on constructing a detector, weighting events, and getting event rates.
