\section{Introduction}
\label{sec:intro}

% Figure environment removed

The global network of neutrino telescopes, defined as gigaton-scale neutrino detectors, has allowed us observe the Universe in new ways. 
The subset of this network of telescope deployed on ice or water includes the IceCube Neutrino Observatory~\cite{IceCube:2016zyt} near the South Pole, proposed detectors ORCA and ARCA~\cite{KM3Net:2016zxf} in the Mediterranean Sea (KM3NeT collaboration), and Baikal-GVD in Lake Baikal, Russia~\cite{Avrorin:2015wba} (BDUNT collaboration). 
Additionally, new experiments like P-ONE~\cite{P-ONE:2020ljt} off the coast of Vancouver and TRIDENT~\cite{Ye:2022vbk} in the South China Sea are underway, as well as an expansion for the IceCube Observatory~\cite{Ishihara:2019aao, IceCube-Gen2:2020qha}.

These telescopes all share many technological features.
Each of these detectors operate by detecting Cherenkov photons emitted by neutrino interaction byproducts, and as such follow the same general simulation chain illustrated in Fig. 1. 
The only proprietary step is the final detector response, which occurs after an optical module (OM) has detected a photon. 
Yet for many years we have lacked a simulation framework that takes advantage of this similarity.
Existing packages individually cover one or two of these common steps, but until now there has been no easy way to simulate a particle from injection to photon propagation. 

\texttt{Prometheus}~\cite{Lazar:2023prm} looks to correct this by providing an integrated framework to simulate these common steps for arbitrary detectors in ice and water, using a combination of publicly available packages and those newly developed for this work.
Neutrino injection is handled by \texttt{LeptonInjector}\cite{IceCube:2020tcq}, an event generation recently developed by the IceCube Collaboration.
Taus and muons are then propagated by \texttt{PROPOSAL}~\cite{Koehne:2013gpa}. 
Light yield simulation and photon propagation in ice relies on \texttt{PPC}~\cite{chirkin2022kpl}, while in water these steps are covered by \texttt{Fennel}~\cite{fennel2022@github} and \texttt{Hyperion}, respectively.

\texttt{Prometheus}'s flexibility allows one to optimize detector configurations for specific physics goals, while the common format allows one to develop reconstruction techniques that may be applied across different experiments.
With the recent explosion in machine-learning research, it is now more important than ever that we are able to rapidly implement and test new ideas without relying on tools and data that may be proprietary to their experiments. 

The rest of this article is organized as follows. 
In Sec.~\ref{sec:validation} we outline the format of \texttt{Prometheus}'s output and validate against published results. 
In Sec.~\ref{sec:community} we present community work that employs \texttt{Prometheus}. 
In Sec.~\ref{sec:examples} we provide a short example running \texttt{Prometheus} code. 
Finally, in Sec.~\ref{sec:conclusion} we conclude and offer our future outlook.