\section{Conclusion}
\label{sec:conclusion}

In this submission we have introduced \texttt{Prometheus} as an open-source software package for simulating  neutrino telescopes. 
We have provided a brief example for simulating events in an ice-based detector and highlighted two current works that employ \texttt{Prometheus}. 
\texttt{Prometheus}'s flexibility of input for detector geometry and injection parameters allows it to handle simulation for the full range of existing and proposed telescopes in both water and ice.

As we have demonstrated in the highlighted community contributions, \texttt{Prometheus} facilitates the implementation of new ideas without the need for proprietary data, or for data on a scale not yet available.
Via its particular application to machine-learning models, \texttt{Prometheus} can be a key piece in accelerating the development of faster, more efficient reconstructions for all detectors.

Finally, it is our hope that \texttt{Prometheus} opens the door for greater collaboration within the community. 
By encouraging the sharing of methods and simulated data sets, we hope work done by any one effort more quickly and easily becomes progress for every group in the global neutrino telescope network.
