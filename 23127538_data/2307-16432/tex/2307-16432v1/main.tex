% Please make sure you insert your
% data according to the instructions in PoSauthmanual.pdf
\documentclass[a4paper,11pt]{article}
\usepackage[utf8x]{inputenc}
\usepackage{amscd}
\usepackage{amsmath}
%\usepackage{amssymb}
\usepackage{amsthm}

\usepackage[colorinlistoftodos]{todonotes}

\usepackage{thmtools}
\usepackage{thm-restate}
\usepackage{mathtools}
\usepackage[full]{complexity}
\usepackage{longtable}

%\usepackage[usenames,dvipsnames]{xcolor}
\usepackage{xcolor}
% % tables
 \usepackage{array}

\usepackage{bbm}
\usepackage{comment}
\usepackage{enumerate}
\usepackage{floatrow}


\usepackage{parallel,enumitem}

\usepackage{xspace}
\usepackage{paralist}
\usepackage{xifthen}
\usepackage{url}
\usepackage{csquotes}
% \usepackage{graphicx}
\usepackage{wrapfig}
\usepackage{multirow}
\usepackage[binary-units=true]{siunitx}

\usepackage{tikz}
\usetikzlibrary{trees,decorations,arrows,automata,shadows,positioning,plotmarks,backgrounds,shapes}
\usetikzlibrary{calc,matrix,fit,petri,decorations.markings,decorations.pathmorphing,patterns,intersections,decorations.text}
\usepackage{pgfplots}
\usepackage{pgfplotstable}

\tikzstyle{mystate}=[state,inner sep=3pt,minimum size=20pt,line width=0.2mm]
\tikzstyle{fstate}=[state,accepting,inner sep=2pt,minimum size=3pt]
\tikzstyle{istate}=[state,initial,inner sep=2pt,minimum size=3pt]
\tikzstyle{mysquare}=[inner sep=3pt,minimum size=15pt,line width=0.2mm]
\tikzstyle{fmysquare}=[inner sep=3pt,minimum size=15pt,line width=0.5mm,accepting]
\newcommand{\SFSAutomatEdge}[5]{\path[->](#1) edge[#4,line width=0.2mm] node[#5] {\ensuremath{#2}} (#3);}
\usepackage{subcaption}
\usepackage{tabularx}
\usepackage{booktabs}
\usepackage{xfrac}

\usepackage{etoc}
\etocsettocdepth{3}

% \usepackage{minitoc}

% \usepackage{titletoc}
% 
% \newcommand\DoToC{%
%   \startcontents
%   \printcontents{}{2}{\textbf{Contents}\vskip3pt\hrule\vskip5pt}
%   \vskip3pt\hrule\vskip5pt
% }

% New commands
\newcommand{\incr}{\,\mathrm{d}}
\newcommand{\set}[1]{\{#1\}}
\newcommand{\diff}[2]{\frac{\mathrm{d}{#1}}{\mathrm{d}{#2}}}
\newcommand{\pdiff}[2]{\frac{\partial{#1}}{\partial{#2}}}
\newcommand{\ndiff}[3][]{\frac{\mathrm{d}^{#1}{#2}}{\mathrm{d}{#3}^{#1}}}
\newcommand{\npdiff}[3][]{\frac{\partial^{#1}{#2}}{\partial{#3}^{#1}}}
\newcommand{\R}{\mathbb R}

% New environments
\newtheorem{remark}{Remark}

\title{Prometheus: An Open-Source Neutrino Telescope Simulation}
 \ShortTitle{Prometheus: An Open-Source Neutrino Telescope Simulation}

\author*[a]{David Kim for the Prometheus authors}
% \author[d]{Stephan Meighen-Berger}
% \author[e]{Christian Haack}
% \author[b]{Santiago Giner}
% \author[b]{Carlos A. Arg\"{u}elles}

\affiliation[a]{
    Department of Physics, Cornell University, Ithaca 14853, NY, United States
    }

% \affiliation[b]{
%     Department of Physics and Laboratory for Particle Physics and Cosmology, Cambridge 02138, MA, United States
%     }
% \affiliation[c]{
%     Department of Physics and Wisconsin IceCube Particle Astrophysics Center University of Wisconsin–Madison, Madison 53703, WI, United States
%     }

% \affiliation[d]{School of Physics, The University of Melbourne, Melbourne, Melbourne 3010, Victoria, Australia}

% \affiliation[e]{Physik-department, Technische Universität München, München, D-85748 Garching, Germany}



% Uncomment \forColl{coll.name} below to add "for the XXX Collaboration" to the authors list. 
% In this case, you also have to uncomment the lines after "%Full authors list" below and include the full authors list,
% \forColl{FADSFas} % W/O "Collaboration"

\emailAdd{dsk265@cornell.edu}

\abstract{
The soon-to-be-realized, global network of neutrino telescopes will allow new opportunities for collaboration between detectors.
While each detector is distinct, they share the same underlying physical processes and detection principles.
The full simulation chain for these telescopes is typically proprietary which limits the opportunity for joint studies. This means there is no consistent framework for simulating multiple detectors.
To overcome these challenges, we introduce Prometheus, an open-source simulation tool for neutrino telescopes.
Prometheus simulates neutrino injection and final state and photon propagation in both ice and water.
It also supports user-supplied injection and detector specifications.
In this contribution, we will introduce the software; show its runtime performance; and highlight successes in reproducing simulation results from multiple ice- and water-based observatories.
}

\ConferenceLogo{PoS_ICRC2023_logo.pdf}

\FullConference{%
The 38th International Cosmic Ray Conference (ICRC2023)\\
  26 July -- 3 August, 2023\\
  Nagoya, Japan}

%% \tableofcontents

\begin{document}
\maketitle

\section{Introduction}
Current quantum hardware is unable to carry out universal quantum computations due to the buildup of errors that occur during the computation. 
The magnitude of the individual error is currently above the value that the Threshold Theorem requires in order to kick-start quantum error correction and fault-tolerant quantum computation~\cite[Section 10.6]{nielsen_chuang_2010}. 
Although the experimentally achieved fidelity rates are promising and the error bounds are inching closer to the required threshold, we will have to work for the foreseeable future with quantum hardware with errors that build-up during the computation.  This implies that we can only do a limited number of steps before the output of the computation has become completely uncorrelated with the intended one.

For fault-tolerant quantum computing, we repeat four steps: 
1) We apply a number of single and two-qubit quantum gates, in parallel whenever possible; 
2) We perform a syndrome measurement on a subset of the qubits; 
3) We perform fast classical computations to determine which errors have occurred and how to correct them; 
and, 4) We apply correction terms based on the classical computations.
We then repeat these four steps with a next sequence of gates. 
These four steps are essential to fault-tolerant quantum computing. 


The starting point of this work is to use the four steps outlined above, not to carry out error correction and fault-tolerant computation, but to enhance short, constant-depth, {\em uncorrected} quantum circuits that perform single qubit gates and {\em nearest-neighbor} two qubit gates. 
Since in the long run we will have to implement error-correction and fault-tolerant computation anyhow, and this is done by such a four-step process, why not make other use of this architecture? Moreover, on some of the quantum hardware platforms, these operations are already in place.
Embracing this idea we naturally arrive at the question: what is the computational power of \textit{low-depth} quantum-classical circuits organized as in the four steps outlined above? 
We thus investigate circuits that execute a small, ideally constant, number of stages, where at each stage we may apply, in parallel, single qubit gates and {\em nearest-neighbor} two qubit gates, followed by measurements, followed by low-depth classical computations of which the outcome can control quantum gates in later stages. 
It is not clear, at first, whether such circuits, especially with constant depth, can do anything remotely useful. 
But we will see that this is indeed the case: many quantum computations can be done by such circuits in constant depth. 
By parallelizing quantum computations in this way, we improve the overall computational capabilities of these circuits, as we do not incur errors on qubits that are idle, simply because qubits are not idle for a very long time. 
Furthermore, reducing the depth of quantum circuits, at the cost of increasing width, allows the circuit to be run faster even if errors occur.

The first usage of such a four-step layout, not to do error correction, but to perform computations, can be found in the paradigm of measurement-based quantum computing~\cite{gottesman1999demonstrating,raussendorf2001one,jozsa2006introduction,clark2007generalised}: 
A universal form of quantum computing where a quantum state is prepared and operations are performed by measuring qubits in different bases, depending on previous measurements and intermediate measurements.

\citeauthor{PhamSvore2013} were the first to formalize the four-step protocol for performing computations~\cite{PhamSvore2013}. They included specific hardware topologies by considering two-dimensional graphs for imposing constraints on qubit interactions. In their model, they develop circuits for particularly useful multi-qubit gates, including specifying costs in the width, number of qubits, depth, number of concurrent time steps, size, and total number of non-Identity operations.
As a result, they find an algorithm that factors integers in polylogarithmic depth.
\citeauthor{Browne:2011} showed that the main tool in the work by \citeauthor{PhamSvore2013}, the fan-out gate, can also be replaced by additional log-depth classical computations in the measurement-based quantum computing setting~\cite{Browne:2011}.

More recently, \citeauthor{Cirac:2021} introduced a scheme to implement unitary operations involving quantum circuits combined with Local Operations and Classical Communication ($\mathsf{LOCC}$) channels: $\mathsf{LOCC}$-assisted quantum circuits~\cite{Cirac:2021}. Similarly to the four-step scheme we just described, they allow for a short depth circuit to be run on the qubits, followed by one round of $\mathsf{LOCC}$, in which ancilla qubits are measured and local unitaries are applied based on the measurement outcomes. They show that in this model any 1D transitionally invariant matrix-product state (MPS) with fixed bond dimension is in the same phase of matter as the trivial state. Similar ideas can be found in~\cite{TVV_NonAbelianTopologicalOrder_2022, tantivasadakarn2021long}.

In this work, we introduce a new model, called \textit{Local Alternating Quantum-Classical Computations} ($\LAQCC$). In this model we alternate between running quantum circuits (constrained by locality), ending in the measurement of a subset of qubits, and fast classical computations based on the measurement results. The outcome of the classical computations are then used to control future quantum circuits. We allow for flexibility in this model, by giving different constraints to the power of both the quantum circuits and the classical circuits as well as the number of alternations between them. 
Most attention will be given to $\LAQCC$ containing quantum circuits of constant depth, classical circuits of logarithmic depth and at most a constant number of alternations between them. 
Any circuit constructed in this model is considered to be of constant depth. 
We restrict ourselves to logarithmic depth classical computations, as this is the first natural and non-trivial extension beyond constant-depth classical computations. 
Constant-depth classical computations do however also have an equivalent constant-depth quantum implementation.

The definition of $\LAQCC$ sharpens the original definition of \citeauthor{PhamSvore2013} by adding constraints to the intermediate classical computations. This allows us to bound the power of $\LAQCC$ from above. 

The main result of \citeauthor{Cirac:2021}, that 1D translational invariant MPS with fixed bond dimension can be prepared by $\mathsf{LOCC}$-assisted circuits, relies on local symmetries of the MPS. These symmetries allow them to prepare local states (on a constant number of qubits) and glue them together by doing one round of the appropriate entangling measurement and corrections, after which they run a round of local unitaries to get the desired result. This general scheme for preparing states that exhibit an MPS description with the appropriate local symmetries requires only geometrically local unitaries and one round of measurement and corrections an therefore is accessible in $\LAQCC$. Studying different local symmetries, known as Symmetry Protected Topological (SPT) phases of matter, to find measurement-based constant depth circuits for states is a broad ongoing field of research~\cite{TVV_NonAbelianTopologicalOrder_2022, tantivasadakarn2021long, smith2023deterministic}. 
All these schemes have a $\LAQCC$ implementation.

%$\LAQCC$-circuits also exist for general schemes of preparing local states, based on the local tensors, and gluing them together using one round of entangled measurement and corrections, based on the local symmetry. 
%The main result of \citeauthor{Cirac:2021}, that 1D translational invariant MPS with fixed bond dimension can be prepared by $\mathsf{LOCC}$-assisted circuits, relies heavily on local symmetries of the MPS and as a result also has an equivalent $\LAQCC$ implementation. 
%The corrections applied after the measurement round are local unitaries depending on the local symmetries of the MPS. 

 

%This general scheme of preparing local states, based on the local tensors, and gluing it together by doing one round of entangled measurement and corrections, based on the local symmetry, is accessible in $\LAQCC$.
Note however that \citeauthor{Cirac:2021} also suggest a circuit for the $W$-state.
This circuit uses sequentially and dependent measurement-based corrections of the ancilla qubits. 
These dependent measurements translate to sequential alternations between the quantum and classical circuits and therefore increase the total depth to linear depth, exceeding the constant-depth constraints imposed by $\LAQCC$-circuits. 

We study the power of the $\LAQCC$ model with respect to state preparation, showing that even with only constant quantum-depth and logarithmic classical depth it remains possible to prepare states with long-range entanglement.
Another surprising result is that it is unlikely that $\LAQCC$ circuits are classically simulatable. We show that any instantaneous quantum polynomial-time (IQP) circuit~\cite{Bremner2010,Shepherd2009} has an $\LAQCC$ implementation.
Classical simulation of IQP circuits implies the collapse of the polynomial hierarchy to the third level, which is not believed to be true~\cite{Bremner2017}. Therefore, we expect that $\LAQCC$ circuits are unlikely to be classically simulatable. We bound the power of $\LAQCC$ by showing that it is contained in $\QNC^1$, the class of polynomial-size, log-depth circuits.

Next, we also study the power that intermediate classical calculations can add to quantum computations, by considering a new model that alternates between polynomially many polynomial-depth quantum circuits and unbounded classical computations
We study this model by doing a complexity theoretical analysis, where we draw inspiration from the notions of complexity given by \citeauthor{RosenthalYuen:2022}, \citeauthor{MetgerYuen:2023}, and \citeauthor{Aaronson:2004}.
All three complexity notions are based on the notion of state preparation, instead of more traditional definition of complexity such as the decidability of a computational problem. 
The first two consider classes based on sequences of quantum states preparable by a polynomial-sized quantum circuit, where the circuits are uniformly generated by a computational class, for instance, the class $\mathsf{PSPACE}$, which results in the complexity class $\mathsf{StatePSPACE}$~\cite{RosenthalYuen:2022,MetgerYuen:2023}.
The third notion considers a relative complexity, where the complexity is measured between two given states, and is measured by the number of gates, from a given gate-set, required to transform one state in another state~\cite{Aaronson:2004}. 
For our definition of state preparation complexity, we drop the uniformity constraint from~\cite{RosenthalYuen:2022,MetgerYuen:2023} and define a class as $\mathsf{StateX}$, which refers to states preparable by circuits of type $\mathsf{X}$. 
As an example, if $\mathsf{X} = \QNC^0$, this results in the class $\mathsf{StateQNC^0}$, which is the set of states preparable from the $\ket{0}^n$ state by poly-size constant-depth circuits. 
This notion is similar to the relative complexity from~\cite{Aaronson:2004}, where one state is the  $\ket{0}^n$ state and instead of counting the number of gates we consider the set of states preparable by a fixed number of gates. Using this notion of complexity we show that any state preparable by an $\LAQCC^*$ circuit is also preparable by a $\mathsf{PostQPoly}$ circuit, the class of circuits of polynomial depth with an additional post-selection gate. 

All Clifford circuits have a constant-depth $\LAQCC$ implementation, implying that any stabilizer state can be implemented by a constant-depth $\LAQCC$ circuit, see Section~\ref{sec:clifford_circuits} for a proof of this statement. 
Efficient circuits for stabilizer states have been known already through measurement-based quantum computing. Therefore this paper focuses on the preparation of non-stabilizer states, and as a surprising result we find novel constant-depth protocols for four very natural classes of non-stabilizer states.
Despite the extensive research into these four classes of non-stabilizer states and the many applications of them, no efficient constant- or low-depth state preparation protocols are known yet. We specifically consider these four classes as they are all often used as initial states in other algorithms.

The first state is a uniform superposition over an arbitrary number of states. 
This state finds applications in many quantum algorithms, as they often start with a uniform superposition over multiple states. 
This superposition is often achieved by applying Hadamard gates to every qubit due to its simplicity to prepare. 
Yet, the analysis of many algorithms, such as Shor's algorithm~\cite{Shor:1997}, would benefit from a different initial superposition. 
The circuit to prepare the uniform superposition over an arbitrary number of states uses an exact version of Grover search as a subroutine, that turns a probabilistic circuit, with a known constant probability of success, into a deterministic circuit. 
We use the circuit for preparing a uniform superposition over an arbitrary number of states as a subroutine in the next two quantum state preparation protocols. 

The second state is the $W$-state, the uniform superposition over all computational basis states of Hamming-weight~$1$, a natural long-ranged entangled state that displays a fundamentally nonequivalent type of entanglement from the Greenberger–Horne–Zeilinger state~\cite{WState:2000}, for which $\LAQCC$-type constant-depth circuits were previously known~\cite{PhamSvore2013, Cirac:2021}. 
The $W$-state is often used as benchmark for new quantum hardware~\cite{Haffner2005,Neeley2010,GarciaPerez:2021}. 
A novel way to prepare the $W$-state therefore gives a new way to benchmark different quantum devices with each other. 
A circuit for preparing the $W$-state was given in~\cite{Cirac:2021}, but this implementation requires sequentially alternating measurements followed by local unitaries, which in the $\LAQCC$ model is not considered to be of constant depth. 
We improve this protocol by giving an $\LAQCC$ implementation of the $W$-state, based on a compress-uncompress method that links the one-hot and binary encoding of integers.

The third state considered is the Dicke state, a generalization of the $W$-state, a superposition over all computational basis states with Hamming-weight $k$~\cite{Dicke:1954}. 
Dicke states have relevance in various practical settings.
For instance, for quantum game theory~\cite{zdemir2007}, quantum storage~\cite{Bacon_Compress:2006,Plesch:2010}, quantum error correction~\cite{ouyang2014permutation}, quantum metrology~\cite{toth2012multipartite}, and quantum networking~\cite{prevedel2009experimental}. 
Dicke states have been used as a starting state for variational optimization algorithms, most notably Quantum Alternating Operator Ansatz (QAOA)~\cite{Hadfield2019}, to find solutions to problems such as Maximum k-vertex Cover~\cite{Brandhofer2022,cook2020quantum}.
The ground states of physical Hamiltonians describing one-dimensional chains tend to show a resemblance to Dicke states such as states resulting from the Bethe ansatz, making them an ideal starting state when investigating the ground state behavior of these Hamiltonians~\cite{TDL_BetheAnsatzDerivation:2010,B_ExcitedStateQuantumPhaseTransitions:2013,DickeTransitions:2021}. 
For instance, the algorithm by \citeauthor{van2021preparing}, who give an algorithm to prepare the Bethe ansatz eigenstates of the spin-1/2 XXZ spin chain, starts by first preparing a Dicke state~\cite{van2021preparing}. 
A Dicke-state preparation protocol based on the compress-uncompress methodology used in the $W$-state furthermore finds applications in entanglement distillation, where the entanglement of a large state is concentrated on only a few qubits. 
Efficient deterministic circuits for preparing Dicke states have been proposed by \citeauthor{bartschi2019deterministic}~\cite{bartschi2019deterministic, bartschi2022deterministic_short_depth}. 
They provide a quantum circuit of depth $\mathO(k \log(\frac{n}{k}))$, allowing arbitrary connectivity, to prepare a Dicke state, which they conjecture to be optimal when $k$ is constant. 
In this work, we provide a constant-depth $\LAQCC$ circuit below their conjectured bound already for constant $k$. 
However, this does not directly disprove their conjecture, as we allow for intermediate measurements and classical computations. 
More significantly, we even construct constant-depth $\LAQCC$ circuits for $k = \mathO(\sqrt{n})$ greatly improving their bound.
This construction extends the compress-uncompress method for the $W$-state combined with additional subroutines. 

We continue with a log-depth state preparation protocol for the Dicke-state for arbitrary $k$. 
This protocol implements an efficient transformation between the factoradic number representation and the combinatorial number representation of a positive integer. 
The combinatorial number representation relates directly to the Dicke state. 
The provided efficient transformation between number representation systems might be of independent interest. 

We conclude by modifying our protocol for preparing a Dicke-state to a protocol that prepares quantum many-body scar states in constant-depth. 
These states have low entanglement and longer coherence times than states with similar energy density.
These characteristics make many-body scar states interesting to analyze and relevant within physics.
Many-body scar states appear for instance in the AKLT model~\cite{AKLT:1987,MRBAR:2018,MRB:2018} and different spin models~\cite{SI:2019,MOBFR:2020}.
Known methods for preparing these states have polynomial-depth~\cite{Gustafson:2023}, whereas our circuit has constant depth. 

% We conclude by studying the power that intermediate classical calculations can add to quantum computations. 
% In this study, we define a new model that relaxes constant-depth quantum circuits to polynomial depth quantum circuits, log-depth classical calculations to unbounded classical computations and a constant number of alternations to a polynomial number of alternations. 
% We call this model $\LAQCC^*$. 
% We study this model by doing a complexity theoretical analysis, where we draw inspiration from the notions of complexity given by \citeauthor{RosenthalYuen:2022}, \citeauthor{MetgerYuen:2023}, and \citeauthor{Aaronson:2004}.
% All three complexity notions are based on the notion of state preparation, instead of more traditional definition of complexity such as the decidability of a computational problem. 
% The first two consider classes based on sequences of quantum states preparable by a polynomial-sized quantum circuit, where the circuits are uniformly generated by a computational class, for instance, the class $\mathsf{PSPACE}$, which results in the complexity class $\mathsf{StatePSPACE}$~\cite{RosenthalYuen:2022,MetgerYuen:2023}.
% The third notion considers a relative complexity, where the complexity is measured between two given states, and is measured by the number of gates, from a given gate-set, required to transform one state in another state~\cite{Aaronson:2004}. 
% For our definition of state preparation complexity, we drop the uniformity constraint from~\cite{RosenthalYuen:2022,MetgerYuen:2023} and define a class as $\mathsf{StateX}$, which refers to states preparable by circuits of type $\mathsf{X}$. 
% As an example, if $\mathsf{X} = \QNC^0$, this results in the class $\mathsf{StateQNC^0}$, which is the set of states preparable from the $\ket{0}^n$ state by poly-size constant-depth circuits. 
% This notion is similar to the relative complexity from~\cite{Aaronson:2004}, where one state is the  $\ket{0}^n$ state and instead of counting the number of gates we consider the set of states preparable by a fixed number of gates. Using this notion of complexity we show that any state preparable by an $\LAQCC^*$ circuit is also preparable by a $\mathsf{PostQPoly}$ circuit, the class of circuits of polynomial depth with an additional post-selection gate. 

\paragraph{Summary of results}
\begin{itemize}
    \item We give a new definition of a computational model that captures the power of the four step process: applying a constant number of layers of one- and two-qubit gates; performing a syndrome measurement; perform a fast classical computation determining corrections; apply corrections. We call this model \emph{Local Alternating Quantum Classical Computations}, or $\LAQCC$ for short. In this model we bound the allowed quantum operations, intermediate classical calculations, and number of rounds separately. In Section~\ref{sec:LAQCC_model} we define this model and give a list of operations based on results from literature contained in this computational model. In some of these operations we explicitly use that we allow for multiple, but at most constant, rounds  of corrections.
    \item  We show show that there exist $\LAQCC$ circuits that can not be weakly simulated in Section~\ref{sec:IQP_in_LAQCC}. We further show that for every $\LAQCC$ circuit there exists a $\QNC^1$ circuit simulating it perfectly, in Section~\ref{sec:LAQCC_in_QNC1}.
    \item We introduce a new type computational complexity for preparing states and show that the extension of $\LAQCC$ where we allow a polynomial number of rounds and unbounded classical computation, is contained in $\mathsf{PostQPoly}$, the class of polynomial circuits with post-selection, in Section~\ref{sec:Complexity results}.
    \item We show a protocol to prepare the uniform superposition state of size $q$ in $\LAQCC$ using $\mathO(\ceil{\log_2(q)}^2)$ qubits in Section~\ref{sec:superposition_modulo_q}. 
    \item We show a protocol to prepare the $W_n$ state in $\LAQCC$ using $\mathO(n\log(n))$ qubits in Section~\ref{sec:W_state_in_LAQCC}.
    \item We show two ways of preparing the Dicke-$(n,k)$ state. The first method is in $\LAQCC$, works up to $k = \mathO(\sqrt{n})$, uses $\mathO(n^2\log(n))$ qubits, and is found in Section~\ref{sec:dicke:small_k}. The second method is in $\LAQCC\text{-}\mathsf{LOG}$ (an extension of $\LAQCC$ allowing for logarithmic number of alterations instead of constant), works for any $k$, uses $\mathO(\text{poly}(n))$ qubits, and is found in Section~\ref{sec:Dicke_in_LAQCC_LOG}. 
    \item We extend on our $\LAQCC$ method of generating Dicke-$(n,k)$ states for $k = \mathO(\sqrt{n})$ and show a protocol to generate many-body scar states for a particular Hamiltonian in $\LAQCC$ (Section~\ref{sec:many_body_scar}). 
\end{itemize}
Summarized in a table, we provide the following state generation protocols:
\begin{table}[htb]
\centering
\begin{tabular}{l|l|l|l}
\textbf{State description} & \textbf{Width} & \textbf{Depth} & \textbf{Implementation}\\
\hline 
Uniform superposition mod $q$: $\frac{1}{\sqrt{q}} \sum_{i = 0}^{q-1}\ket{i}$ & $\mathO(\ceil{\log^2 q})$ & $\mathO(1)$ & Section~\ref{sec:superposition_modulo_q}\\

$W$-state: $\frac{1}{\sqrt{n}}\sum_{i = 0}^{n-1}\ket{e_i}$ & $\mathO(n \log n)$ & $\mathO(1)$ & Section~\ref{sec:W_state_in_LAQCC}\\

Dicke-$(n,k)$, $k = \mathO(\sqrt{n})$: $\binom{n}{k}^{-1/2}\sum_{x \in \{0,1\}^n: |x| = k} \ket{x}$ &  $\mathO(n^2\log n)$ & $\mathO(1)$ 
&Section~\ref{sec:dicke:small_k}\\

Dicke-$(n,k)$: $\binom{n}{k}^{-1/2}\sum_{x \in \{0,1\}^n: |x| = k} \ket{x}$ & $\mathO(\text{poly}(n))$ & $\mathO(\log n)$ &Section~\ref{sec:Dicke_in_LAQCC_LOG}\\

QMBS: $\ket{S_k} = \frac{1}{k! \sqrt{\mathcal N(n,k)}}(Q^\dagger)^k \ket{\Omega}$ &  $\mathO(n^2\log n)$ & $\mathO(1)$  &  Section~\ref{sec:many_body_scar}
\end{tabular}
\caption{Summary of state preparation protocols given in this paper.}
\label{tab:sate_prep}
\end{table}
In the entry for the quantum many-body scar state $Q$ denotes the raising operator and $\mathcal N(n,k)=\binom{n-k-1}{k}$. 
Section~\ref{sec:many_body_scar} will provide more details on the variables and the implementation. 

\paragraph{Organization of the paper}
\noindent We first introduce relevant preliminaries in Section~\ref{sec:preliminaries}. 
In Section~\ref{sec:LAQCC_model} we formally define the class of Local Alternating Quantum-Classical Computations ($\LAQCC$). We also show that any Clifford circuit can be implemented in constant depth $\LAQCC$ (a result based on a result from measurement-based quantum computing~\cite{jozsa2006introduction}). 
This result allows us to give many useful multi-qubit gates and routines in Section~\ref{sec:gates_created_in_LAQCC}. 
Beyond that we show that constant depth $\LAQCC$ circuits are contained in $\QNC^1$ and that any $\mathsf{IQP}$ circuit has an $\LAQCC$ implementation.
We conclude this section with an analysis of a more powerful instantiation of $\LAQCC$ and show an inclusion with respect to the class $\mathsf{PostQPoly}$, which is the class of circuits of polynomial depth with one additional post-selection gate. 
In Section~\ref{sec:state_prep_in_LAQCC} we give $\LAQCC$ circuit implementations for preparing the uniform superposition over an arbitrary number of states, the $W$-state and the Dicke state up to $k = \mathO(\sqrt{n})$. We furthermore give a log-depth circuit implementation for preparing the Dicke state for any $k$. We conclude by showing a $\LAQCC$ circuit for generating many body scar states of a particular type of Hamiltonian.



\section{Applicability to Optimal Control}

\subsection{Problem formulation and optimality conditions}
The following is a generic optimal control problem formulation written in terms of \emph{generalized inequalities}, denoted by "$\preceq$" as in \cite{Boyd2004}. This choice is motivated by the potential applicability of the new method to convex optimal control problems.
\begin{prob} \label{prob:optimal-control}
Determine $\bm{x}(t)$, $\bm{u}(t)$, $t_0$ and $t_f$ such that
\begin{equation} \label{eq:ocp-cost}
        \Psi_0 \big( t_0, \bm{x}(t_0)\big) + \Psi_f \big( t_f, \bm{x}(t_f)\big)
                          + \int_{t_0}^{t_f} h \bigl(t, \bm{x}(t), \bm{u}(t) \bigr) \dl{t}
\end{equation}
is minimized, subject to
\begin{gather}
    \dot{\bm{x}} (t) = \bm{f} \bigl(t, \bm{x}(t), \bm{u}(t) \bigr) \label{eq:dynamics} \,, \\
    \bm{g} \bigl(t, \bm{x}(t), \bm{u}(t) \bigr)  \preceq_{\mathcal{G}} \bm{0} \,, \label{eq:path-constraints} \\
    \bm{\phi}_0 \bigl(t_0, \bm{x}(t_0) \bigr) \preceq_{\Phi_0} \bm{0} \,, \quad 
    \bm{\phi}_f \bigl(t_f, \bm{x}(t_f) \bigr) \preceq_{\Phi_f} \bm{0} \,, \label{eq:endpoint-constraints} \\
    \forall t \in \closedinterval{t_0}{t_f} \,,
\end{gather}
where $\bm{x}(t) \in \mathbb{R}^{n_x}$ and $\bm{u}(t) \in \mathbb{R}^{n_u}$ denote, respectively, the state and the control variables of the dynamic system in \eqref{eq:dynamics}.
Moreover, the constraint sets $\mathcal{G} \subseteq \mathbb{R}^{n_g}$, $\Phi_0 \subseteq \mathbb{R}^{n_{\phi,0}}$ and $\Phi_f \subseteq \mathbb{R}^{n_{\phi,f}}$ are arbitrary intersections of \emph{proper cones}, see \cite{Boyd2004}, and $\Psi_0(\cdot)$, $\Psi_f(\cdot)$, $h(\cdot)$, $\bm{f}(\cdot)$, $\bm{g}(\cdot)$, $\bm{\phi}_0(\cdot)$, and $\bm{\phi}_f(\cdot)$ are arbitrary twice-differentiable functions characterized by the following mappings:
\begin{alignat}{2}
    \Psi_0 &: \mathbb{R} \times \mathbb{R}^{n_x} &&\mapsto \mathbb{R} \,, \\
        \Psi_f &: \mathbb{R} \times \mathbb{R}^{n_x} &&\mapsto \mathbb{R} \,, \\
        h &: \mathbb{R} \times \mathbb{R}^{n_x} \times \mathbb{R}^{n_u} &&\mapsto \mathbb{R} \,, \\
        \bm{f} &: \mathbb{R} \times \mathbb{R}^{n_x} \times \mathbb{R}^{n_u} &&\mapsto \mathbb{R}^{n_x} \,, \\
        \bm{g} &: \mathbb{R} \times \mathbb{R}^{n_x} \times \mathbb{R}^{n_u} &&\mapsto \mathbb{R}^{n_g} \,, \\
        \bm{\phi}_0 &: \mathbb{R} \times \mathbb{R}^{n_x} &&\mapsto \mathbb{R}^{n_{\phi,0}} \,, \\
        \bm{\phi}_f &: \mathbb{R} \times \mathbb{R}^{n_x} &&\mapsto \mathbb{R}^{n_{\phi,f}} \,,
    \end{alignat}
where the integers $n_x$, $n_u$, $n_g$, $n_{\phi,0}$ and $n_{\phi,f}$ are non-negative and serve as arbitrary dimension specifiers.
\end{prob}

\begin{defn} \label{defn:dual-cone}
    Consider a cone $K \subseteq \mathbb{R}^{n_k}$ for some integer $n_k$. The \emph{dual cone} of $K$ is such that, see \cite{Boyd2004},
    \begin{equation}
        K^{\bm{\star}} = \{ \bm{y} \in \mathbb{R}^{n_k} : \mathbf{v}^\intercal \bm{y} \geq 0, \: \forall \mathbf{v} \in K \} \,.
    \end{equation}
\end{defn}

By introducing $\bm{\lambda}(t) \in \mathbb{R}^{n_x}$, hereinafter referred to as \emph{costate vector},
as well as 
$\bm{\mu}(t) \in \mathbb{R}^{n_g}$, 
$\bm{\nu}_0 \in \mathbb{R}^{n_{\phi,0}}$ and 
$\bm{\nu}_f \in \mathbb{R}^{n_{\phi,f}}$, 
one may define the Hamiltonian as:
\begin{multline} \label{eq:hamiltonian}
    \mathcal{H}( t, \bm{x}, \bm{u}, \bm{\lambda}, \bm{\mu}) = h ( t, \bm{x}, \bm{u} ) + \bm{\lambda}^\intercal \bm{f}( t, \bm{x}, \bm{u} ) \\
    + \bm{\mu}^\intercal \bm{g}( t, \bm{x}, \bm{u} ) \,,
\end{multline}
such that, in addition to \eqref{eq:path-constraints} and \eqref{eq:endpoint-constraints}, the solution to Problem \ref{prob:optimal-control} satisfies the following conditions
\begin{align}
    \nabla_{\bm{\lambda}} \mathcal{H}( t, \bm{x}, \bm{u}, \bm{\lambda}, \bm{\mu} ) - \dot{\bm{x}}^\intercal &= \bm{0}^\intercal \,, \label{eq:condition-lambda} \\
    \nabla_{\bm{x}} \mathcal{H}( t, \bm{x}, \bm{u}, \bm{\lambda}, \bm{\mu} ) + \dot{\bm{\lambda}}^\intercal &= \bm{0}^\intercal \,, \label{eq:condition-state} \\
    \nabla_{\bm{u}} \mathcal{H}( t, \bm{x}, \bm{u}, \bm{\lambda}, \bm{\mu} ) &= \bm{0}^\intercal \label{eq:condition-control} \,, \\
    \bm{\mu}^\intercal \bm{g}( t, \bm{x}, \bm{u} ) &= 0 \,, \label{eq:condition-slack-g} \\ 
    \bm{\nu}_0^\intercal \bm{\phi}_0 \bigl(t_0, \bm{x}(t_0) \bigr) &= 0 \,, \label{eq:condition-slack-nu-0} \\ 
    \bm{\nu}_f^\intercal \bm{\phi}_f \bigl(t_f, \bm{x}(t_f) \bigr) &= 0 \,, \label{eq:condition-slack-nu-f} \\
    \bm{\mu}(t) &\succeq_{\mathcal{G}^{\bm{\star}}} \bm{0} \\
    \bm{\nu}_0 &\succeq_{\Phi_0^{\bm{\star}}} \bm{0} \\
    \bm{\nu}_f &\succeq_{\Phi_f^{\bm{\star}}} \bm{0} \\
    \nabla_{\bm{x}} \Psi_0 \bigl(t_0, \bm{x}(t_0) \bigr) + \bm{\nu}_0^\intercal \nabla_{\bm{x}} \bm{\phi}_0 \bigl(t_0, \bm{x}(t_0) \bigr) &= -\bm{\lambda}^\intercal (t_0) \,, \label{eq:condition-state-0} \\
    \nabla_{\bm{x}} \Psi_f \bigl(t_f, \bm{x}(t_f) \bigr) + \bm{\nu}_f^\intercal \nabla_{\bm{x}} \bm{\phi}_f \bigl(t_f, \bm{x}(t_f) \bigr) &= \bm{\lambda}^\intercal (t_f) \,, \label{eq:condition-state-f} \\ 
    \diffp{\Psi_0}{t_0} \bigl(t_0, \bm{x}(t_0) \bigr) + \bm{\nu}_0^\intercal \diffp{\bm{\phi}_0}{t_0} \bigl(t_0, \bm{x}(t_0) \bigr) &= \mathcal{H}(t_0) \,,  \label{eq:condition-t0}\\
    \diffp{\Psi_f}{t_f} \bigl(t_f, \bm{x}(t_f) \bigr) + \bm{\nu}_f^\intercal \diffp{\bm{\phi}_f}{t_f} \bigl(t_f, \bm{x}(t_f) \bigr) &= - \mathcal{H}(t_f) \,, \label{eq:condition-tf}
\end{align}
where the operator $\nabla_{\mathbf{x}}$ yields the row vector of partial derivatives of the associated function with respect to the components of vector $\mathbf{x}$, and the short-hand notation $\mathcal{H}(t_0)$ stands for $\mathcal{H} \bigl( t_0, \bm{x}(t_0), \bm{u}(t_0), \bm{\lambda}(t_0), \bm{\mu}(t_0) \bigr)$. The analogous is true for $\mathcal{H}(t_f)$.
Furthermore, $\mathcal{G}^{\bm{\star}}$, $\Phi_0^{\bm{\star}}$ and $\Phi_f^{\bm{\star}}$ denote the dual cones of $\mathcal{G}$, $\Phi_0$ and $\Phi_f$, respectively, see Definition \ref{defn:dual-cone}.
In equations \eqref{eq:hamiltonian}, \eqref{eq:condition-lambda}, \eqref{eq:condition-state}, \eqref{eq:condition-control} and \eqref{eq:condition-slack-g} the parametrization of variables $\bm{x}(t)$, $\bm{u}(t)$, $\bm{\lambda}(t)$ and $\bm{\mu}(t)$ with respect to $t$ has been omitted for simplicity of notation.
Equations \eqref{eq:condition-lambda} through \eqref{eq:condition-tf} are called \emph{first order optimality conditions}.
For the formulation of the Hamiltonian and the derivation of optimality conditions see, \emph{e.g.}, \cite{Bryson1975,Kirk2004,Boyd2004}.

\subsection{Algebraic transcription and constraint enforcement}
To employ the new method we discretize Problem \ref{prob:optimal-control} according to the following strategy.
\begin{description}
    \item[State Variables]{Sampled according to $\mathcal{T}_N \cup \{\tau_\xid\}$, with $\tau_\xid$ given by Corollary \ref{cor:final}. Forming a set of $N+1$ points that are indexed with $\setS$.}
    \item[Control Variables]{Sampled according to $\mathcal{T}_N$. Forming a set of $N$ points that are indexed with $\setC$.}
\end{description}
Please recall that $\mathcal{T}_N$ is the set of roots of the Lobatto polynomial defined in \eqref{eq:set-lobatto-roots}, and that the index sets $\setS$ and $\setC$ are defined in \eqref{eq:set-s} and \eqref{eq:set-c}, respectively.

Note that by discretizing the problem we reduce the dimensionality of the search space to a finite set, therefore, there is always an intrinsic error present between discrete and analytic variables.
Regardless, for the sake of simplicity and intuitiveness, we shall use identical variable naming for respective quantities.
% Note that by introducing discretization, the solution obtained will have an intrinsic error. Regardless, for simplicity, we use identical notation for the decision variables \lambda, \mu, x, u.
% In order to construct a finite dimensional problem that can be solved with the new method we discretize Problem \ref{prob:optimal-control} according to the following strategy.


Given the domain of interest $\closedinterval{t_0}{t_f}$ and the normalized domain $\tau \in \closedinterval{-1}{1}$, a \emph{one-to-one} relationship can be established through the linear mapping $t : \closedinterval{-1}{1} \mapsto \closedinterval{t_0}{t_f}$, defined as:
\begin{equation} \label{eq:domain-mapping}
    t(\tau) = \frac{t_f-t_0}{2} \tau + \frac{t_f+t_0}{2} \,.
\end{equation}

% Algebraic path constraints.
In this way, we are able to enforce the path constraints specified in \eqref{eq:path-constraints} as:
\begin{equation} \label{eq:discrete-path-constraints}
    \bm{g}( t_k, \bm{x}_k, \bm{u}_k ) \preceq_{\mathcal{G}} \bm{0} \,, \quad k \in \setC \,,
\end{equation}
where $t_k = t( \tau_k )$ according to \eqref{eq:domain-mapping}, $\bm{x}_k$ and $\bm{u}_k$ denote $\bm{x}(t_k)$ and $\bm{u}(t_k)$, respectively, and $\tau_k \in \mathcal{T}_N$ with $k \in \setC$.

% Definite integral
Moreover, recalling the Gaussian quadrature from \eqref{eq:gauss-quadrature} and the domain mapping in \eqref{eq:domain-mapping}, one may approximate the definite integral of Problem \ref{prob:optimal-control} with the following algebraic form, see, \emph{e.g.}, \cite{Garg2011,Patterson2014,Garrido2021,GarridoMScThesis}:
\begin{equation} \label{eq:algebraic-integral}
    \int_{t_0}^{t_f} h \bigl(t, \bm{x}(t), \bm{u}(t) \bigr) \dl{t} \approx \frac{t_f - t_0}{2} \sum_{k \in \setC} w_k h( t_k, \bm{x}_k, \bm{u}_k ) \,.
\end{equation}

% Equations of motion
Finally, the differential constraints expressed in \eqref{eq:dynamics} are enforced with the following algebraic construction: 
\begin{equation} \label{eq:diff-scheme}
    \bm{f}(t_k, \bm{x}_k, \bm{u}_k) - \frac{2}{t_f-t_0} \sum_{i \in \setS} \bm{D}_{ki} \bm{x}_i = \bm{0} \,, \quad k \in \setC \,,
\end{equation}
where $\bm{D}$ denotes the differentiation matrix of the new method, see Theorem \ref{thm:new-matrix-is-inversion-ready}.


\subsection{Karush–Kuhn–Tucker conditions}
In this section, let $1$ and $N$ be the indices associated with the initial and final domain endpoints, respectively.
In this way, we may define a \emph{KKT Hamiltonian} as:
\begin{multline} \label{eq:kkt-hamiltonian}
    \tilde{H}( t_0, t_f, \bm{x}_k, \bm{u}_k, \bm{\lambda}_k, \bm{\mu}_k, \bm{\nu}_0, \bm{\nu}_f ) = \\
    \tfrac{t_f-t_0}{2} w_k \big(
    h_k
    + \bm{\lambda}_k^\intercal \bm{f}_k 
    + \bm{\mu}_k^\intercal \bm{g}_k 
    \big) \\
    + \big( \Psi_0(t_0, \bm{x}_1) + \bm{\nu}_0^\intercal \bm{\phi}_0(t_0, \bm{x}_1) \big) \delta_{1k} \\
    + \big( \Psi_f(t_f, \bm{x}_N) + \bm{\nu}_f^\intercal \bm{\phi}_f(t_f, \bm{x}_N) \big) \delta_{Nk} \,,
\end{multline}
where
$\bm{\lambda}_k$ and $\bm{\mu}_k$ denote $\bm{\lambda}(t_k)$ and $\bm{\mu}(t_k)$, respectively, with $k \in \setC$.
Moreover, $\delta_{ik}$ denotes the Kronecker delta defined in \eqref{eq:lagrange-kronecker-delta},
and the short-hand $\mathbf{f}_k = \mathbf{f}\bigl( t_k, \bm{x}(t_k), \bm{u}(t_k) \bigr)$ is used for functions $h$, $\bm{f}$ and $\bm{g}$ of Problem \ref{prob:optimal-control}.

To simplify notation, we introduce the following short-hand for the KKT Hamiltonian function:
\begin{equation}
    \tilde{H}_k = \tilde{H}( t_0, t_f, \bm{x}_k, \bm{u}_k, \bm{\lambda}_k, \bm{\mu}_k, \bm{\nu}_0, \bm{\nu}_f ) \,,
\end{equation}
where the initial and final domain endpoints, $t_0$ and $t_f$, as well as the multipliers, $\bm{\nu}_0$ and $\bm{\nu}_f$, do not change for $k \in \setC$.

Using the KKT Hamiltonian, we construct an \emph{augmented cost function} as follows:
\begin{multline} \label{eq:augmented-cost}
    J( t_0, t_f, \bm{x}_j, \bm{u}_m, \bm{\lambda}_m, \bm{\mu}_m, \bm{\nu}_0, \bm{\nu}_f ) = \\
    \sum_{k \in \setC} \Bigl\{ \tilde{H}_k - w_k \bm{\lambda}_k^\intercal \sum_{i \in \setS} \bm{D}_{ki} \bm{x}_i \Bigr\}\,, \\ \quad j \in \setS, \: m \in \setC,
\end{multline}
In this way, the augmented cost in \eqref{eq:augmented-cost} is analogous to the Lagrangian function in the context of constrained parametric optimization with respect to the transcription strategy expressed in \eqref{eq:algebraic-integral}, \eqref{eq:discrete-path-constraints}, and \eqref{eq:diff-scheme}, see, \emph{e.g.}, \cite{Bertsekas2016,Boyd2004,Garg2011}.

As an intermediate step towards the KKT conditions, we highlight the gradient of the augmented cost with respect to the discrete state variables below:
\begin{equation} \label{eq:kkt-x-intermediate}
    \nabla_{\bm{x}_j} J(\cdot) = \sum_{k \in \setC} \bigl\{ \nabla_{\bm{x}_k} \tilde{H}_k \delta_{kj} - w_k \bm{\lambda}_k^\intercal \bm{D}_{kj} \bigr\} \,, \quad j \in \setS \,.
\end{equation}
Equation \eqref{eq:kkt-x-intermediate} exhibits a \emph{left} multiplication between the costate samples, $\bm{\lambda}_k$ with $k \in \setC$, and the differentiation matrix $\bm{D}$.
To convert this into a \emph{right} multiplication and obtain a linear combination in column space, we define an auxiliary matrix $\dualD \in \mathbb{R}^{N \times N}$ as:
\begin{equation} \label{eq:dual-matrix}
    \dualD_{ki} = \frac{\delta_{ki}}{w_k}( \delta_{Nk} - \delta_{1k} ) - \frac{w_i}{w_k} \bm{D}_{ik} \,, \quad k, i \in \setC\,,
\end{equation}
where $w_k$ are the quadrature weights obtained from \eqref{eq:weights}.
In this way, the KKT conditions, which are obtained by setting the gradient of $J(\cdot)$ with respect to each variable to zero, are expressed as follows:
\begin{align}
    \nabla_{\bm{\lambda}_k} \tilde{H}_k - w_k \sum_{i \in \setS} \bm{D}_{ki} \bm{x}_i^\intercal &= \bm{0}^\intercal \,, \label{eq:kkt-l} \\
    \nabla_{\bm{x}_k} \tilde{H}_k + w_k \sum_{i \in \setC} \dualD_{ki} \bm{\lambda}_i^\intercal &= \bm{\lambda}_N^\intercal \delta_{Nk} - \bm{\lambda}_1^\intercal \delta_{1k} \,, \label{eq:kkt-x} \\
    \sum_{i \in \setC} w_i \bm{\lambda}_i \bm{D}_{i\xid} &= \bm{0} \,, \label{eq:kkt-x-sigma} \\
    \nabla_{\bm{u}_k} \tilde{H}_k &= \bm{0}^\intercal \,, \label{eq:kkt-u} \\
    \bm{\mu}_k^\intercal \bm{g}_k &= 0 \,, \\ 
    \bm{\nu}_0^\intercal \bm{\phi}_0 (t_0, \bm{x}_1 ) &= 0 \,, \\ 
    \bm{\nu}_f^\intercal \bm{\phi}_f (t_f, \bm{x}_N ) &= 0 \,, \\
    \sum_{i \in \setC} \diffp{\tilde{H}_i}{t_0} = \sum_{i \in \setC} \diffp{\tilde{H}_i}{t_f} &= 0 \,,
\end{align}
with $k \in \setC$, see, \emph{e.g.}, \cite{Ross2003,Benson2006,Fahroo2008,Garg2011}.
Note that \eqref{eq:kkt-x} is a consequence of \eqref{eq:kkt-x-intermediate} for $j \in \setC$.
Similarly, \eqref{eq:kkt-x-sigma} is a consequence of \eqref{eq:kkt-x-intermediate} for $j = \xid$.

\begin{prop} \label{prop:costate-degree}
    Condition \eqref{eq:kkt-x-sigma} can be satisfied if the discrete samples of the costate variables, $\bm{\lambda}_i$ with $i \in \setC$, are interpolatable with a polynomial of degree no greater than $N-2$.
\end{prop}
\begin{proof}[Proof of Theorem \ref{prop:costate-degree}]
    If the samples of the costate variables, $\bm{\lambda}_i$ with $i \in \setC$, are interpolatable with a polynomial $\bm{\lambda}(\tau)$ of degree no greater than $N-2$, the Gauss-Lobatto quadrature rule imposed in condition \eqref{eq:kkt-x-sigma} is exact. Therefore, the following holds:
    \begin{equation}
        \sum_{i \in \setC} w_i \bm{\lambda}_i \bm{D}_{i\xid} = \int_{-1}^{1} \bm{\lambda}(\tau) \dot{l}_{\xid} (\tau) \dl \tau \,.
    \end{equation}
    Integrating by parts and rewriting the integral term with a Gaussian quadrature rule yields
    \begin{equation}
        \bm{\lambda}(\tau_N) l_{\xid} (\tau_N) - \bm{\lambda}(\tau_1) l_{\xid} (\tau_1) - \sum_{i \in \setC} w_i \dot{\bm{\lambda}}_i l_{\xid}(\tau_i) \,,
    \end{equation}
    which is identically zero due to \eqref{eq:lagrange-kronecker-delta}. Therefore, condition \eqref{eq:kkt-x-sigma} is satisfied.
\end{proof}

\begin{thm} \label{thm:dual-matrix-is-diff-matrix}
    Matrix $\dualD \in \mathbb{R}^{N\times N}$ specified in \eqref{eq:dual-matrix} is a differentiation matrix over the set of abscissas $\mathcal{T}_N$ with order of accuracy $N-2$, according to Definition \ref{defn:differentiation-matrix}.
\end{thm}
\begin{proof}[Proof of Theorem \ref{thm:dual-matrix-is-diff-matrix}]
    Consider a polynomial $p(\tau)$ of degree $N_p$, with $\tau \in \closedinterval{-1}{1}$ sampled at the nodes $\tau_i \in \mathcal{T}_N$, with $i \in \setC$. We apply the matrix $\dualD$ to the samples $p(\tau_i)$, as follows:
    \begin{multline} \label{eq:dual-D-with-poly}
        \sum_{i \in \setC} \dualD_{ki} p(\tau_i) = \frac{\delta_{Nk} - \delta_{1k}}{w_k} \sum_{i \in \setC} \delta_{ki} p(\tau_i) \\ 
        - \frac{1}{w_k} \sum_{i \in \setC} w_i \dot{l}_k (\tau_i) p(\tau_i)
    \end{multline}
    where $l_k(\cdot)$ denotes the Lagrange interpolating polynomial of index $k$ defined in \eqref{eq:poly-interp}. As a result, the polynomial $\dot{l}_k(\cdot)$ has degree $N-1$. In this context, let $N_p$ be such that $(N-1) + N_p \leq 2N - 3$ leading to $N_p \leq N - 2$, then the Gaussian quadrature rule in \eqref{eq:dual-D-with-poly} is exact, allowing us to write the right-hand-side of \eqref{eq:dual-D-with-poly} as:
    \begin{equation}
         \frac{\delta_{Nk} - \delta_{1k}}{w_k} p(\tau_k) - \frac{1}{w_k} \int_{-1}^{1} \dot{l}_k (\tau) p(\tau) \dl \tau \,.
    \end{equation}
    Performing integration by parts yields
    \begin{multline}
        \frac{\delta_{Nk} - \delta_{1k}}{w_k} p(\tau_k) +
        \frac{1}{w_k} \int_{-1}^{1} l_k (\tau) \dot{p}(\tau) \dl \tau \\
         - \frac{1}{w_k}\bigl[ l_k (\tau_N) p(\tau_N) -  l_k (\tau_1) p(\tau_1)\bigr] \,.
    \end{multline}
    From \eqref{eq:lagrange-kronecker-delta}, it is apparent that the endpoint terms vanish if $k \notin \{ 1,\, N \}$, and that these terms cancel-out if $k \in \{ 1, \, N \}$.
    Furthermore, from Lemma \ref{lem:lagrange-inner-prod}, the integral term can be simplified, yielding
    \begin{equation}
            \sum_{i \in \setC} \dualD_{ki} p(\tau_i) = \frac{1}{w_k} w_k \dot{p}(\tau_k) = \dot{p}(\tau_k) \,.
    \end{equation}
\end{proof}

% \begin{rem}
%     Due to Theorem \ref{thm:dual-matrix-is-diff-matrix}, disregarding the points at the boundaries, the condition \eqref{eq:kkt-x} is the direct algebraic transcription of the costate equation expressed in \eqref{eq:condition-state}.
% \end{rem}

\subsection{Covector mapping}
It has been shown that there is a one-to-one correspondence between the discrete dual variables of the optimal control problem (with respect to which the KKT conditions above have been written) and the Lagrange multipliers that are obtained from the solver in use, see, \emph{e.g.}, \cite{Ross2001,Ross2003,Fahroo2008,Garg2011,Benson2006,GarridoMScThesis}. For the augmented cost in \eqref{eq:augmented-cost} and the KKT Hamiltonian definition in \eqref{eq:kkt-hamiltonian}, this mapping is given as:
\begin{align}
    \bm{\lambda}_k &= \frac{2}{w_k(t_f - t_0)} \tilde{\bm{\lambda}}_k \,, \quad k \in \setC \\
    \bm{\mu}_k &= \frac{2}{w_k(t_f - t_0)} \tilde{\bm{\mu}}_k \,, \quad k \in \setC \\
    \bm{\nu}_0 &= \tilde{\bm{\nu}}_0 \\
    \bm{\nu}_f &= \tilde{\bm{\nu}}_f
\end{align}
where $\tilde{\bm{\lambda}}_k$, $\tilde{\bm{\mu}}_k$, $\tilde{\bm{\nu}}_0$, and $\tilde{\bm{\nu}}_f$ denote the Lagrange multipliers which are respectively associated with the differential constraints, path constraints and initial and terminal endpoint constraints.

\section{Numerical results}
In this section we implement the new method for two classical optimal control problems.
These are the low-thrust orbit raising problem \cite{Fahroo2008,Garg2011,Bryson1975} and the scalar nonlinear initial value problem \cite{Garg2011}, the later of which has a known analytic solution.
The complex step method for partial derivates is used with a step size of \num{1e-12}, see, \emph{e.g.}, \cite{Alonso2003}, and the NLP solver IPOPT \cite{Waechter2005} was employed.

Table \ref{tab:ipopt-options} shows the solver settings used for all numerical results that follow.
The tolerance values shown in Table~\ref{tab:ipopt-options} directly enforce the precision with which the KKT conditions are satisfied.
\begin{table}[bt!]
    \centering
    \noindent
    \caption{List of IPOPT options used for the benchmark problems.}
    \label{tab:ipopt-options}
    \vspace{1mm}
    \begin{tabular}{lc}
        \toprule
        Option & Value \\ \midrule
        \texttt{tol} & \num{1e-9} \\
        \texttt{dual\_inf\_tol} & \num{1e-9} \\
        \texttt{acceptable\_tol} & \num{1e-9} \\
        \texttt{max\_iter} & \num{1e3} \\
        \texttt{hessian\_approximation} & \texttt{'limited-memory'} \\
        \texttt{linear\_solver} & \texttt{'mumps'} \\
        \bottomrule
    \end{tabular}
\end{table}



\subsection{Time-invariant orbit raising problem}
The orbit raising problem \cite{Bryson1975,Fahroo2008,Garg2011} is concerned with the fixed-time transfer between two circular orbits such that the radius of the final orbit is maximized.
Here it has been adapted such that the mass of the system is included as a state, and therefore the dynamic equations are time-invariant.
The problem is stated as follows.
\begin{prob} \label{prob:orbit-raising}
    Determine $r(t)$, $\theta(t)$, $v_r(t)$, $v_\theta(t)$, $m(t)$ and $\beta(t)$, $\forall t \in \closedinterval{0}{t_f}$ such that
\begin{equation}
     -r(t_f)
\end{equation}
is minimized, subject to the equations of motion
\begin{align}
    \dot{r}(t) &= v_r(t) \,, \\ 
    \dot{\theta}(t) &= \tfrac{v_\theta(t)}{r(t)} \,, \\ 
    \dot{v}_r(t) &= \tfrac{v_\theta^2(t)}{r(t)} - \tfrac{\mu}{r^2(t)} + \tfrac{T}{m(t)} \sin \beta(t) \,,   \\
    \dot{v}_\theta(t) &= - \tfrac{v_r(t) v_\theta(t)}{r(t)} + \tfrac{T}{m(t)} \cos \beta(t) \,, \\ 
    \dot{m}(t) &=-\alpha \,, 
\end{align}
the initial conditions
\begin{align}
    r(0) &= 1 \,, \\
    \theta(0) &= 0 \,, \\
    v_r(0) &= 0 \,, \\
    v_\theta(0) &= 1 \,, \\
    m(0) &= 1 \,,
\end{align}
and the terminal conditions
\begin{align}
    v_r(t_f) &= 0 \,, \\
    v_\theta(t_f) - \sqrt{\tfrac{\mu}{r(t_f)}} &= 0 \,,
\end{align}
where $t_f = \num{3.32}$, $T = \num{0.1405}$, $\mu =\num{1}$, and $\alpha = \num{0.0749}$.
\end{prob}

Problem \ref{prob:orbit-raising} is solved with the new method for $N=\num{25}$. In this context, Fig. \ref{fig:orbit-raising-state} and Fig. \ref{fig:orbit-raising-costate} show the obtained trajectories of state and costate, respectively. The profiles are in agreement with literature, see, \emph{e.g.}, \cite{Garg2011,Fahroo2008}.
% Figure environment removed
% Figure environment removed



\subsection{Convergence behaviour for an analytic problem}

In order to test the convergence rate of the new method, we shall consider a problem for which the solution is known. In this context we shall take a scalar nonlinear initial value problem from \cite{Garg2011} and \cite{Garg2010}, which is stated as follows.
\begin{prob} \label{prob:convergence}
    Determine $x(t)$ and $u(t)$, $\forall t \in \closedinterval{0}{2}$ such that
    \begin{equation}
        -x(2)
    \end{equation}
    is minimized, subject to the differential constraint
    \begin{equation}
        \dot{x}(t) = \tfrac{5}{2}( x(t) u(t) - x(t) - u^2(t) ) \,,
    \end{equation}
    and the initial condition $x(0) = 1$.
\end{prob}

The solution to Problem \ref{prob:convergence} is as follows, see \cite{Garg2011,Garg2010}:
\begin{align}
    x^*(t) &= 4/( 1 + 3 \exp{\tfrac{5t}{2}} )\\
    u^*(t) &= x^*(t)/2 \\
    \lambda^*(t) &= - \frac{\exp{\bigl(2 \ln(1 + 3 \exp{(\tfrac{5t}{2})}) - \tfrac{5t}{2}\bigr)} }{6 + 9 \exp(5) + \exp(-5) } 
\end{align}

We now show the achieved precision of the new method with respect to the known solution for many degrees of polynomial interpolation $N$ and we compare the result with the alternative Gauss-Lagrange methods, namely, the Gauss method \cite{Benson2004,Rao2010}, the Radau method \cite{Garg2011,Patterson2014} and the standard Lobatto method \cite{Ross2003,Fahroo2008}.

Figures \ref{fig:convergence-state}, \ref{fig:convergence-control}, and \ref{fig:convergence-costate} show the maximum absolute value of the errors obtained for state, control and costate, respectively, in relation to the known solution \emph{at the points where differential constraints are enforced}. These quantities are computed as follows
\begin{align}
    E_x &= \max_{k \in \setC}{\lvert \hat{x}_k - x^*(t_k) \rvert} \\
    E_u &= \max_{k \in \setC}{\lvert \hat{u}_k - u^*(t_k) \rvert} \\
    E_\lambda &= \max_{k \in \setC}{\lvert \hat{\lambda}_k - \lambda^*(t_k) \rvert} \,,
\end{align}
where $\hat{x}_k$, $\hat{u}_k$, and $\hat{\lambda}_k$ are, respectively, the discrete solutions of state, control and costate obtained with each method.
% Figure environment removed
% Figure environment removed
% Figure environment removed

As shown in previous research, see \cite{Garg2011,Garg2010}, the standard Lobatto method, which employs a square differentiation matrix, is, in general, the most inaccurate.
In addition, note that the IPOPT solver failed to reach optimality when using the standard Lobatto method for $N=\{30,34,35\}$. The absence of the respective points is evident in Fig. \ref{fig:convergence-costate}.
In contrast, we note that the new method exhibits identical precision as the method of Radau with regards to the state and control solutions.

% Finally, with respect to the costate solution, the new method exhibits comparable precision with the Radau method.
% has virtually the same accuracy as the methods of Gauss and Radau with regards to the state and control solutions.




\section{Community Contributions}
\label{sec:community}

% Figure environment removed

% Figure environment removed
\label{fig:angular_resolution}

Developing a published reconstruction is often a long-term effort internal to an experiment, meaning methods are slow to implement and difficult to compare.
\texttt{Prometheus} remedies this by allowing users to readily generate the large data sets necessary to test the feasibility and relative performance of different reconstruction methods.
In this section we highlight two submissions at ICRC 2023 that utilize \texttt{Prometheus} in such a way.
The first of these works is a software-focused effort that looks to improve the speed of first reconstructions at detectors and the second is a hardware-focused effort concerning GPU alternatives for low power computing. 

Ref.~\cite{Yu:2023pos}, proposes sparse-submanifold convolutions (SSCNNs) as an alternative to the convolutional neural network (CNN) and traditional trigger-level event reconstructions currently used in neutrino telescopes. 
Their directional and energy algorithm is trained on a data set of 412892 events and tested on a further 50000 events, cut from a set of 3 million generated by \texttt{Prometheus}. 
As seen in Fig.~\ref{fig:event_rates}, the SSCNN is capable of running at speeds comparable to the neutrino telescope trigger rates.
In Fig. 5, it achieves a median angular resolution below $4^\circ$ for the highest-energy trigger-level events, which matches or outperforms currently employed reconstructions.
Their work could be applied to improve on-site reconstructions and filtering for notable events at any detector. 

Ref.~\cite{Jin:2023}, is another \texttt{Prometheus} user submission that explores hardware accelerators to improving event reconstruction. 
Specifically, they look at how a Tensor Processing Unit (TPU) compatible algorithm could lower energy consumption while running comparable operations to the current GPU-based ones. 
Their model reaches a median angular resolution around $5^\circ$ above $10^3$ GeV. 
Meanwhile, approximate peak total power consumption drops from 100W on the best performing GPU (Apple M1 Pro chip) down to 3W on the Google Edge TPU.
The portion of the power consumption directed to the ML accelerator component drops from 15W to 2W.
This submission illustrates \texttt{Prometheus}'s utility in testing more speculative ideas and providing proof-of-concepts to encourage further work, particularly in the direction of machine learning.

The findings of both works are generic to any ice or water neutrino telescope.
Their models are trained and tested on example detector geometries rather than the geometry of any existing or proposed detector.
Since \texttt{Prometheus} is fully open-source, we hope it will facilitate not only easier iteration on these techniques but also greater collaboration in sharing and adapting them.


\section{\add{Assembly-level Verification for Page Table Traversal and Mapping}}
\label{sec:experiment_appendix}
%To both validate and demonstrate the value of the modal approach to reasoning about virtual memory management, 
% we study several
% We validate our logic by studying
% distillations of key VMM functionality.
% real concerns of virtual memory managers.
% Recall from Section \ref{sec:logic} that virtual points-to assertions work just like regular points-to assertions, by design.

\replace{
In this section we verify several critical and challenging pieces of VMM code.
First, in several stages, we work up to mapping a new page in the current address space.
This requires a number of independently challenging substeps: dynamically traversing a page table to find
the appropriate L1 entry to update; inserting additional levels of the page table if necessary (updating
the VMM invariants along the way);
converting the physical addresses found in intermediate entries into the corresponding virtual addresses
that can be used for memory access;
installing the new mapping;
and collecting sufficient resources to form a virtual points-to assertion.
Of these, only the second-to-last step (installing the correct mapping into the
current address space) has previously been formally verified with respect to a machine model with address translation.
Second, we formally verify a switch into a new address space as part of a task switch,
the first such verification handling both old and new processes' assertions (in different address spaces) at the time of the switch.
}{
While our logic was developed and proven sound for x86-64 assembly,
Section \ref{sec:traversingC} described verification of software page table walking code (\lstinline|pte_get_next_table| and \lstinline|walkpgdir|)
as if at the level of C for improved readability.
This appendix describes the actual assembly-level verification carried out in Rocq.
Careful readers of both Section \ref{sec:traversingC} and this appendix will notice
strong similarities in the assertions and and reasoning, for good reason:
The C code in Section \ref{sec:traversingC} was the original kernel code that was compiled
(with no optimizations) to x86-64 assembly and verified with our logic, and the proof outlines
in that section largely back-port the assembly proofs back to C.
\looseness=-1
}

\add{
 This section describes the assembly proofs without reference to the C outlines given in Section \ref{sec:traversingC}.
 The main additional details of note at the assembly level are:
 \begin{itemize}
 \item Accurate treatment of register management (particularly the AMD64 System V calling convention) leading to more direct correspondence
       with our logic
 \item The assembly is naturally more verbose than the C, so the proof outlines are relatively more sparse, with assertions written
       only for key updates.
 \item Bitwise manipulations of page table entries are harder to follow than C's bitfield access support.
       Multiple manipulations which are each explicit in C become adjacent (or sometimes non-adjacent) bitwise operations.
       The critical ones are commented in the assembly figures.
 \item And compared to the C-based presentation earlier, there are differences in logical variable names. For example,
       the assembly proofs use \textsf{entry} as the name for the \emph{physical} address of the entry modified by
       \lstinline|pte_get_next_table| in Figure \ref{fig:calltopteinitialize}, whereas to make sense of the C code
       in Figure \ref{fig:calltopteinitializeC} we used \textsf{entry} consistently with the C variable name and introduced
       separate logical names for physical addresses. This propagates to figures presenting larger invariants separately,
       as they also refer to the logical names from the proofs.
 \end{itemize}
}

%\begin{comment}
%\todo[inline]{Identity mappings are difficult, and our current approach won't quite work. Consider trying to have a virtual pointsto for an actual page table entry (i.e., that one could use to update a page table mapping), while also having a virtual pointsto for an address that entry mapped. With the current (let's call it v1) solution, we can't actually have both of those simultaneously!  That's because the PTE pointsto will assert full ownership of the physical memory cell holding the PTE as its data value, while the virtual pointsto for the data mapped by that entry will \emph{also} assert (fractional) ownership of all entries a page table walk would traverse.
%}
%\todo[inline,color=violet]{This doesn't seem to cause issues with the mapping/unmapping examples, only with changing intermediate page table pointers. The mapping example requires a virtual pointsto for the blank PTE, and once filled in that ownership can be immediately split to create the 512 new virtual pointsto assertions for the newly mapped page. Conversely, for unmapping we'd assume ownership of all the relevant virtual pointsto assertions for the page we're unmapping, at which point we can (with a bit of work) show that they all correspond to the same L1 PTE, and extract the 512 fractional shares of that entry from the pointsto assertions.  But changing intermediate page tables, as one would do for coallescing or splitting a superpage while preserving the virtual-to-physical mappings, couldn't be done without some really complicated separating implication tricks.}
%\todo[inline,color=green]{One possible approach to resolving this, which we came up with in our Tuesday meeting, is to recognize that the current (v1) virtual points-to is too strong, because it really doesn't care about \emph{owning} those fractional resources, it only cares that \emph{something} ensures the correct page table walk exists. Iris has a ghost map resource where authoritative ownership of an individual key-value pair can be handled as a resource.  (Colin was using this in the filesystem cache.)
%We can use that mechanism to separate the virtual-to-physical translation from the physical memory involved (Kolanski and Klein may have done something similar for different reasons): (fractional) virtual points-to assertions can be defined in terms of (fractional) ownership of these authoritative ghost map entry assertions, plus sharing an invariant that the current installed page table respects all entries of the mapping. Unmapping collects the authoritative map kvpairs from collecting the assertions, and then can remove them from the ghost map and update the page tables. Critically, physical ownership of the page tables then lives in the invariant on the current page table, so some virtual pointsto assertions can refer to memory in those page tables.
%This still works with the modality, since that invariant is also semantically a predicate on a page table root.
%Let's call this v2.
%}
%\end{comment}
\subsection{Traversing Live Page Tables}
\label{sec:traversing}
We build up to the main task of mapping a new page after traversing page tables in software.
The mapping operation of Figure \ref{fig:mapping_code} assumes an operation \textsf{walkpgdir} which must traverse the page tables
in order to locate the address of the L1 entry to update --- 
% possibly allocating tables for levels 3, 2, and 1 in the process,
% installing them into levels 4, 3, and 2, along the way.
possibly allocating new L3, L2, and L1 tables as necessary.
Traversing the page tables is itself challenging functionality to verify: loading the current table root from \lstinline|cr3| is straightforward
(a \lstinline|mov| instruction), however this produces the physical address of \lstinline|cr3|, not the virtual address the kernel code would use to access that memory.
This problem repeats at each level of the page table: assuming the code has \emph{somehow} read the appropriate L4 (or L3, or L2) entry, those entries again
yield physical addresses, not virtual.

\subsubsection{Loading Page-Table Address Value}
We will discuss access to the level 4 table later (Section \ref{wlkpgdir}). But for subsequent levels, the base address of level $n$ must be
fetched from the appropriate entry in the level $n+1$ table.
This is the role of \lstinline|pte_get_next_table| (Figures \ref{fig:calltopteinitialize} and \ref{fig:p2v}):
it is passed the virtual address of the page table entry in level $n+1$, and should return the \emph{virtual} 
address of the \emph{base} of the level $n$ table
indicated by that entry.
If the entry is empty (i.e., this is a sparse part of the page table representation),
the code also allocates a page for the level $n$ table, installs it in the level $n+1$ entry, and establishes appropriate invariants.
Figure \ref{fig:calltopteinitialize} presents the initial part of the function, which performs the allocation if necessary.
Figure \ref{fig:p2v} (discussed in Section \ref{sec:p2v}) deals with the cases where no allocation is necessary \emph{or} the allocation has already
been performed by the code in this figure.
\looseness=-1

Note that the specification does \emph{not} assume a specific page table level --- logical parameter \textsf{v} represents the level
of the entry passed as an argument, and this code
is used for all three level transitions when traversing page tables (4 to 3, 3 to 2, 2 to 1).
This comes into play with a subtlety of the specification of \lstinline|pte_get_next_table| that we will
revisit several times: \lstinline|pte_get_next_table|'s specification
assumes it is given a virtual \emph{vpte-pointsto}
(a virtual points-to exposing the underlying physical address instead of existentially quantifying it;
 see Section \ref{sec:mapnew}) granting access to the specified entry,
but its postcondition does not yield new virtual points-to assertions!
Instead it merely computes the base virtual address of the next table, and returns adequate capabilities (discussed in Section \ref{subsec:identitymappings})
for the \emph{caller} to construct a vpte-pointsto for any entry of the next table level (if this is not an L1 entry ---
the caller knows which level of the table this is for).
\looseness=-1

Within \textsf{get\_next\_table}, after a standard function prologue, the code 
loads the entry pointed to by the argument (logical variable \textsf{entry} in the proof outline).
This is a page table entry: a 64-bit word divided into bit-fields for
the physical address of the next table, and control bits like the valid bit, as discussed in 
Section \ref{sec:backgroundonmachinemodel}.



\ifPLDI
Line \ref{line:mask_present} checks % In the condensed figure, it's all on one line
\else
Lines \ref{line:mask_present}--\ref{line:check_entry_present} check
\fi
if the entry's ``present'' bit is set.
If it is zero, a new page must be allocated for the next level of the table --- which is done by the fall-through
from Line \ref{line:check_entry_present_jump}'s conditional jump. Otherwise the code jumps ahead to
the case for the next level already existing, which is discussed in Section \ref{sec:p2v} and Figure \ref{fig:p2v}.
First, we must discuss another refinement of the address space invariant, establishing
enough structure on the page tables themselves to allow the traversal.
The code for allocating a new level of the page table must establish this extended invariant.

%wshiftll (wshiftll (natToWord 64 entry) (WordImpl.concat (WordImpl.zero 56) (WordImpl.from_nat 8 12 ^& WordImpl.concat (WordImpl.zero 2) WO~1~1~1~1~1~1)) ^& constf)
%(WordImpl.concat (WordImpl.zero 56) (natToWord 8 12 ^& WordImpl.concat (WordImpl.zero 2) WO~1~1~1~1~1~1))
%
%wshiftll
 %      (wshiftll
%          ((((natToWord 64 entry ^& WordImpl.concat (WordImpl.zero 32) consta ^| WordImpl.concat (WordImpl.zero 32) (natToWord 32 2))
%             ^& WordImpl.concat (WordImpl.zero 32) constb ^| WordImpl.concat (WordImpl.zero 32) (natToWord 32 4)) ^& constd
%            ^| wshiftll
%                 (wshiftll (nextpaddr ^+ ^~ (natToWord 64 KERNBASE))
%                    (WordImpl.concat (WordImpl.zero 56) (WordImpl.from_nat 8 12 ^& WordImpl.concat (WordImpl.zero 2) WO~1~1~1~1~1~1))
%                  ^& constf)
%                 (WordImpl.concat (WordImpl.zero 56) (WordImpl.from_nat 8 12 ^& WordImpl.concat (WordImpl.zero 2) WO~1~1~1~1~1~1)))
%           ^& WordImpl.concat (WordImpl.zero 32) conste ^| wone 64)
%          (WordImpl.concat (WordImpl.zero 56) (WordImpl.from_nat 8 12 ^& WordImpl.concat (WordImpl.zero 2) WO~1~1~1~1~1~1)) ^& constf)
%       (WordImpl.concat (WordImpl.zero 56) (natToWord 8 12 ^& WordImpl.concat (WordImpl.zero 2) WO~1~1~1~1~1~1)) 
% Figure environment removed

\subsubsection{Identity Mappings}
\label{subsec:identitymappings}
Kernels need to convert between physical and virtual addresses, in both directions.
Traversing the page tables in software is the simplest way to convert a virtual address to a physical address; this is the context we are working up to.
However, implementing this virtual-to-physical (V2P) translation in this way ironically requires physical-to-virtual (P2V) translation,
because the addresses stored in page table entries are physical, but memory accesses issued by the OS code use virtual addresses.
% There is no universal way to convert physical addresses to virtual --- doing so relies on the kernel maintaining careful invariants or
% additional data structures to enable P2V translation.
\looseness=-1

Because VMM operations are performance-critical for many workloads, most kernels 
maintain invariants that enable very fast P2V conversions (rather than adding another data structure).
Most kernels maintain an invariant on their page tables that the virtual address of any page used for a page table 
% lives at a virtual address whose value 
is \emph{a constant offset from the physical address} --- a practice sometimes referred to as \emph{identity mapping} 
(even though the physical-to-virtual translation
is typically not literally the identity function, but adding a non-zero constant offset).\footnote{Some kernels do this for all physical memory on the machine, simplifying interaction
with DMA devices.
On newer platforms like RISC-V, this sometimes truly is an identity mapping ---
x86-64 machines are forced into offsets by backwards compatibility with bootloaders that cannot access the full memory space of the
machine.
}

For this reason we extend the per-address-space invariant as in Figure \ref{fig:peraspaceinvariant_with_p2v_extension}, to also track which
addresses we can perform a P2V conversion on by a adding a constant offset.
$\Xi$ is another ghost map, from physical addresses to the level of the page table they represent (1--4).
\emph{Only} physical addresses in $\Xi$ can undergo P2V conversion. 
Section \ref{sec:p2v} describes the actual conversion,
but we describe the invariant here 
because adding new level 3/2/1 tables must maintain the invariant.

% Figure environment removed

For each $\paddr\mapsto \textsf{v} \in\Xi$, the invariant tracks a virtual points-to justifying that virtual address $\paddr+\textsf{KERNBASE}$ maps to physical address $\paddr$
(the ``Ghost translation'' in Figure \ref{fig:peraspaceinvariant_with_p2v_extension});
fractional ownership of the physical memory for that page table entry;
and for valid entries (with the present bit set) above L1, ghost map tokens for every entry in the table pointed to by the entry, which can be used
to repeat the process one level down. 
% (L1 entries point to data pages, whose physical memory ownership resides in some virtual points-to).
The assertion on Line \ref{line:conditional_children} of Figure \ref{fig:calltopteinitialize} comes from the invariant one level up; 
if the valid bit is set,
the code can return those child tokens without the conditional guard.
\looseness=-1

The fractional ownership of the entry's physical memory is subtle. Recall that $\textsf{L}_{4}\_\textsf{L}_{1}\_\textsf{PointsTo}$ retains some physical
ownership of each page table entry that is traversed (proportional to how many virtual addresses share the entry).
So in general the invariant cannot keep full permission to the memory in this part of the invariant, or it would overlap the page table walk for virtual points-to
assertions. But in the case where the entry is invalid, we may need to write to it (e.g., to install a reference to a next-level table, as we do in Figure \ref{fig:calltopteinitialize}),
which requires full permission. Fortunately, the entry can only be in use if its valid bit is set; if the valid bit is not set we know
that no virtual points-to entry in $\delta$/$\theta$ holds any partial ownership.
Thus we use the invariant portion annotated as ``Entry validity'' in Figure \ref{fig:peraspaceinvariant_with_p2v_extension} to capture this:
if the entry is invalid the invariant holds full ownership of the entry, so it can be updated; while if the entry is valid,
the invariant owns only a constant non-zero fragment sufficient to read the entry, but not modify it (which would invalidate some virtual points-to assertions):
\begin{equation*}
 \ulcorner \textsf{qfrac} = 1 \leftrightarrow \; \lnot\textsf{entry\_present }(\vale) \urcorner \tag{*}
\end{equation*}
Thus the fractional ownership of the physical location is enough for Line \ref{line:read_entry_contents} in Figure \ref{fig:calltopteinitialize} to access the entry, though in \lstinline|get_next_table|
the caller has pulled that piece of information out of the invariant and passed it for the entry at hand.
This removal appears explicitly in assertions,
as the argument to the invariant is $\Xi\setminus\{\mathsf{entry}\}$ (indexing by the set $\Xi$ allows us to borrow the physical resources
for a specific page table entry out of the invariant, and later put them back).
Line \ref{line:check_entry_present_jump}'s conditional then determines in the fall-through case that the bit is not set, which 
together with other facts entails $\textsf{qfrac} = 1$ at Line \ref{line:after_concluding_qfrac1},
and permits storing a new entry (in ellided code around Line \ref{line:install_new_entry}).
\looseness=-1

This seemingly-simple piece of code has a highly non-trivial correctness argument, which depends critically on detailed invariants on how access to page table
entries is shared between parts of the kernel. No prior work has engaged with this problem.

% Concretely speaking, going back to Line 15 in Figure \ref{fig:calltopteinitialize}, to read the value referenced by physical address \textsf{entry} while preserving the soundness of memory mappings, our extended invariant introduces the side condition (*)
% \begin{equation*}
%  \ulcorner \textsf{qfrac} = 1 \leftrightarrow \; \lnot\textsf{entry\_present }(\vale) \urcorner \tag{*}
% \end{equation*}
% assuring that looking the identity mapping for \textsf{entry} is safe under the subtle justification which equates the full ownership to the non/presence of the entry which can only be known when investigated in Line 21 in Figure \ref{fig:calltopteinitialize}.
\begin{comment}
 % Figure environment removed
\end{comment}


 \subsubsection{Installing a New Table}
 After obtaining the identity mapping for \textsf{entry}, we are able to load the \textsf{entry\_val} into \textsf{rdi}, and check the presence bit through
\ifPLDI
Line \ref{line:mask_present} % in condensed version, all on same line
\else
Lines \ref{line:mask_present}--\ref{line:check_entry_present} 
\fi
in Figure \ref{fig:calltopteinitialize}.
Accessing the presence bit and checking the value allows us to exploit the condition (*) that was just discussed when verifying the allocation
path (i.e., when the entry is invalid  and Lines \ref{line:alloc_path_start}--\ref{line:alloc_path_end} in Figure \ref{fig:calltopteinitialize}
must allocate the next level of tables).
This operation is subtle. To reiterate: the operation requires that the relevant table entry is readable, but the exact portion of ownership 
returned must be determined by inspecting the valid bit of the value in memory --- so full ownership is returned only for unused entries.
When the bit is not set, that entails full ownership of the entry's memory ($\textsf{qfrac} = 1$) and justifies writing to that memory.
Otherwise, the code jumps past the end of this listing, to the following code at the top of Figure \ref{fig:p2v} (which is also the
continuation of this code).

% Figure environment removed

If the entry is not set, \textsf{pte\_initialize} (Line \ref{line:call_to_pte_initialize} in Figure \ref{fig:calltopteinitialize}) 
allocates a physical page (internally utilizing the only unverified (trusted) code in our case studies, the page-allocator's \textsf{kalloc},\footnote{
  This is an allocator for regions of pre-zeroed physical memory that is mapped, but not accessed by the allocator itself,
  as is typical for slab allocators~\cite{bonwick1994slab}.
  Its verification would be similar to verifying a usermode \textsf{malloc} verifications~\cite{Chlipala2013Bedrock,wickerson2010explicit},
  just with additional invariants on the memory pool.
} 
on Line \ref{line:call_to_kalloc} in Figure \ref{pteinitializespec}). 
Since we are using \textsf{pte\_initialize} for page-table address allocation, we must relate this newly
allocated physical address to the identity mapping map $\Xi$ --- 
see Line \ref{line:page_of_caps} in Figure \ref{fig:calltopteinitialize}, where
\texttt{kalloc}'s specification guarantees it has returned memory from a designated memory
pool that is already mapped
\ifPLDI
\else
\footnote{A reasonable reader might wonder where this pool
initially comes from, and how it might grow when needed. Typically an initial mapping subject to this identity mapping
constraint is set up prior to transition to 64-bit kernel code (notably,
a page table must exist \emph{before} virtual memory is enabled during boot, as part of enabling it is setting
a page table root).
Growing this pool later requires cooperation of physical memory range allocation and virtual memory range allocation,
typically by starting general virtual address allocation at the highest physical memory address plus the identity mapping offset.
This reserves the virtual addresses corresponding to all physical addresses plus the offset for later use in this pool,
as needed.
} 
\fi
and satisfies the offset invariants.
% \todo[inline,color=blue]{colin frontier.
% Stuck with line 31 onwards in Figure 7. rax holds nextpaddr, but I think that should be entrypfn, and 
% the explicit entrypfn id token assertion should go away, as its covered by the forall assertion.
% then the postcondition for pte-initialize should have a specific level now for the entries,
% like 0, which can be updated in the view shift on line 42.
% }
% Focusing on the specification of \textsf{pte\_initialize} separately in Figure \ref{fig:pteinitializespec}, 
% we right immediately realize that instead of seeing see a physical pointsto for the fresly page-table address 
% (e.g. $\mathsf{nextpaddr} \mapsto_{\mathsf{p}} \mathsf{w64\_0}$) deliberately in the post-conditoin in Lines 15-16,
%  we observe a full-ownership token representing the knowledge that a frame and all the entries indexed from this 
% frame are freshly allocated with full-ownership to be a part of the identity map, $\Xi$. 
The soundness argument of this specification relies on the fact that these freshly allocated resources are part 
of an entry construction that has not been completed yet: the presence bit is set 
(Line \ref{line:install_new_entry} in Figure \ref{fig:calltopteinitialize}) after these freshly allocated resources are incorporated to the 
entry construction via the page-frame portion of the PTE. In other words, the side condition, (*),
 formalizes that any access to the entry with these resources is \textit{invalid} (in the sense of not necessarily
having accompanying resources) until the entry is marked present (and thus the memory returned from \textsf{kalloc}
moves into the page table invariant.

\add{Note that the C presentation in Figure \ref{fig:calltopteinitializeC}
omitted the precondition on the implication of Figure \ref{fig:calltopteinitialize}'s Line \ref{line:page_of_caps},
which is logically equivalent to \textsf{True} since \textsf{entry\_present} checks if the present bit is set in an entry,
and \textsf{pte\_initialize} sets that bit. The actual invariant has this form here, and in the postcondition
of \lstinline|pte_initialize| (Figure \ref{pteinitializespec}), to match the conditional form from earlier in
\lstinline|pte_get_next_table| (which is also provably true when the check of the present bit
determines that the entry was already valid/present).
Our proof discharges the conditional at the join point, rather than eagerly in each branch.
}

\subsubsection{Physical-to-Virtual Conversion with \textsf{P2V}}
\label{sec:p2v}
Once we know the entry refers to a physical address in the identity mapping range ($\Xi$)
(via the branch at Line \ref{line:check_entry_present_jump}, or  by allocating and installing a new entry
as just discussed for Lines \ref{line:check_entry_present_jump}--\ref{line:end_of_allocation_path}), 
we can convert this frame address to a corresponding virtual address via the identity mappings
discussed in Section \ref{subsec:identitymappings} and Figure \ref{fig:peraspaceinvariant_with_p2v_extension}.
in the last lines of \lstinline|pte_get_next_table| shown in Figure \ref{fig:p2v} (the continuation of Figure \ref{fig:calltopteinitialize}).
This is a critical piece of the full page table walk verification.
In our small kernel (Line \ref{line:p2v} in Figure \ref{fig:p2v}), as in larger kernels, the C macro \texttt{P2V} common to many kernels
is actually just addition by the constant offset mentioned in Section \ref{subsec:identitymappings}.
But the correctness of this simple instruction is quite subtle.
%  and cannot be proven 
% without the extended invariant (Figure \ref{fig:peraspaceinvariant_with_p2v_extension})
% worked out Section \ref{subsec:identitymappings}.

% Figure environment removed
Figure \ref{fig:p2v} shows the verification of the end of \lstinline|pte_get_next_table| specialized to the case where 
where no allocation was necessary (i.e., the conditional on Line \ref{line:check_entry_present} of Figure \ref{fig:calltopteinitialize} was taken).
In this case, the true present bit allows access to the child tokens from Line \ref{line:conditional_children} of Figure \ref{fig:calltopteinitialize},
which is then refined to the assertion on Line \ref{line:children} of Figure \ref{fig:p2v}.
The code loads \lstinline|rcx| with the offset value \textsf{KERNBASE}, which gives us the value of the virtual address ($\textsf{entry}_{\textsf{pfn}}$ \textsf{+KERNBASE})
of the \emph{base} of the next level of the page table.
% \todo[inline]{the next sentence depends on having figure 10 updated to reflect the page-worth of tokens}
While we could now convert this address to a virtual points-to, this is not necessarily the correct thing to do.
The caller \lstinline|walkpgdir| (discussed next) uses \lstinline|pte_get_next_table| to retrieve just the base address,
because only the caller knows which entry in the subsequent table will be accessed (it depends on the corresponding bits from the virtual
address being translated). So instead we pass back the per-address-space invariant with the identity mapping resources for \lstinline|entry|
pulled out. The caller determines which entry in that table must actually
be accessed --- by selecting the appropriate index into the 512 ghost map tokens returned in the postcondition,
and using the ghost translation and physical location portions of the invariant to assemble a vpte-pointsto
that justifies the caller's subsequent access to a particular entry of the returned table.
% in the identity map ($\Xi\setminus\{entry\}$) of the kernel invariant.
% the logical update in Specification  Lines 5-10 to 10-14 for obtaining virtual-pointsto resource for the frame 
% ($\textsf{entry}_{\textsf{pfn}}$) by removing it from the ghost map ($\Xi\setminus\{entry\}\cup \{\textsf{entry}_{\textsf{pfn}}) \}$) 
% in Line 5 and compute the identity mapping for this physical frame address in Line 13 in Figure \ref{fig:p2v}).

\subsubsection{Walking Page-Table Tree: Calling \textsf{pte\_get\_next\_table} for Each Level}
\label{wlkpgdir}
% Figure environment removed

% Figure environment removed
Implementing a software page-table walk amounts to calling \textsf{pte\_get\_next\_table} for each level as shown in Figure \ref{walkpgdir}. 
The key part of the specification and proof for a page table walk is accumulation of memory mappings for the page-table entries 
visited and frame addresses for page-tables. 
For example, Lines \ref{line:ex_l4_vpte} and \ref{line:ex_l3_vpte} in Figure \ref{walkpgdir} show the virtual pte-pointsto assertions for L4 and L3 entries.
In the final post-condition, we expect the accumulation of these resources from each level -- $\textsf{R}_{\textsf{walk}}$ -- 
which allows us to construct and return the path to the L1 entry in the tree to insert a new page.  

This is the code which performs most actual physical-to-virtual conversions using the identity mapping portion of the per-address-space invariant.
\lstinline|walkpgdir| accepts a \emph{virtual} pointer to the base of the L4 table, and the address to translate.
The precondition provides knowledge that the virtual base of the L4 is at the appropriate offset from the current \lstinline|cr3| value,
but does not provide a virtual points-to assertion --- because the function must calculate (Lines \ref{line:start_pml4_calc}--\ref{line:end_pml4_calc})
which entry it needs access to.
Instead the precondition has 512 identity map tokens, guaranteeing that every entry on the page is subject to the identity mapping invariant.
Line \ref{line:end_pml4_calc} calculates the virtual address of the relevant entry, and the subsequent view shift
pulls that entry out of the identity mapping ($\Xi$) and fetches its corresponding resources as
described by Figure \ref{fig:peraspaceinvariant_with_p2v_extension} and Section \ref{subsec:identitymappings}.
The ghost translation and physical location are used to form the virtual pte-pointsto for the L4 entry
(Line \ref{line:first_pte_pointsto}), with the entry validity and next-level indexing
satisfying the rest of the precondition for \lstinline|pte_get_next_table|.
\lstinline|pte_get_next_table| then, as described earlier, checks the valid bit in the indicated
entry and either returns the (unconditional) tokens for the L3 entry physical addresses (if valid), or
allocates into the entry and returns new (also unconditional) tokens for the L3 entry physical addresses.
\lstinline|pte_get_next_table|'s first call (Line \ref{line:first_getnext_call}) returns
the virtual address of the base of the L3 table (a \emph{page directory pointer}, so PDP, in official
x86-64 terminology). Then the situation to move from that pointer to the base of the L2
is just like the process just followed: the proof calculates the address of the relevant
L3 entry, uses the appropriate L3 identity mapping token to construct a virtual pte-pointsto to that entry,
and passes that along with additional resources pulled out of the invariant to another call to
\lstinline|pte_get_next_table|. That call then returns the base of an L2 table, and the process
repeats until the function returns the virtual address of the relevant L1 entry.
That will then be used in the next section by the caller of \lstinline|walkpgdir|
to install a new mapping.


% \textsf{walkpgdir}, as a client, holds the knowledge that there exists an identity mapping for the physical entry address (\textsf{entry})
%  in the root page table ($\textsf{L}_{4}$):  $\mathsf{entry} \mapsto_{\textsf{id}} \textsf{\_}$ in Specification Line 3 is a partially owned
%  token for accessing and looking up the resources in the identity map, $\Xi$, to construct the \textit{virtual-to-physical} pointsto relation 
% $\textsf{entry+KERNBASE} \mapsto_{\textsf{vpte,qfrac}} \textsf{entry \entry\_val}$ with the virtual address (\textsf{entry+KERNBASE}) obtained 
% by offsetting the physical address (\textsf{entry}). With this knowledge on the root-page-table-entry, we can start traversing the page-table 
% tree which requires locating the address of the next table -- a call to \textsf{pte\_get\_next\_table} shown in Figure \ref{fig:calltopteinitialize}. 
% Beyond a frame, the precondition before Line 15 requires the current address space invariant, and knowledge that \textsf{entry} is mapped to a 
% random entry value, subtly, 
% the operation also, at least, requires that the relevant table entry is readable, but the exact portion of ownership 
% returned must be determined by inspecting the valid bit
% of the value in memory --- so full ownership is returned only for unused entries.
% This is a simple piece of code whose functionality is critical and whose correctness is highly non-trivial. No prior work engages with this problem.



%% Figure environment removed



%\caption{Traversing page-tables, and allocating entries as needed while mapping-a-page in Figure \ref{fig:mappingcode}.}
% \citet{kolanski08vstte,kolanski09tphols} verified a single code block with their logic which was roughly Figure \ref{fig:mapping_code} for a 2-level ARM
% page table, but several critical complexities our work deals with were not addressed.
% First, beyond the limitations discussed in Section \ref{sec:overly-restrictive}, Kolanski and Klein assumed that virtual addresses
% for page tables at each level were given as parameters rather than verifying any conversion from physical addresses to virtual addresses (or even axiomatizing their lookup).
% In contrast, our verification articulates the address space invariant from which the physical-to-virtual translation can be implemented.
% Second, our proof deals with the construction of a valid virtual points-to \emph{to the PTE to update} in mapping, which Kolanski and Klein also
% assumed was given.
% \todo{some of this is really an argument for our verification being more thorough, rather than being about our logic}

% Reasoning about the page table walk in their logic would have required 
% could reason about the walk, but would need to explicitly prove that all other invariants
% of the kernel, the current address space, and all other address spaces of interest were preserved by each update, because their model
% only supports separation within a single address space. In our model, this follows for free from making
% our separation logic directly aware of address translation and internalizing assumptions about other address spaces as further separable assertions.
% Kolanski and Klein did address part of the walk information for a 2-level page table (a possible ARM configuration), but 

% \textsc{seL4} currently still trusts address translations; it models page tables as a data structure in regular memory, thus not capturing the possibility that even
% temporarily destroying the mappings and restoring them can actually crash the OS. \textsc{CertiKOS} papers share little in the way of precise details about
% their virtual memory management, but because their core technology is based on a fork of \textsc{CompCert}, whose model of memory is
% a set of unordered block allocations, we can infer their proofs must also trust these translations.


\subsection{Mapping a New Page}
\label{sec:mapnew}
One of the key tasks of a page fault handler in a general-purpose OS kernel is
to map new pages into an address space by writing into an existing page table via a call\\
\centerline{\textsf{vaspace\_mappage(pte\_t *pml4, void *va,uintptr\_t fpaddr)}}\\
in Figure \ref{fig:mapping_code}.
To do so, with a given allocated a fresh page (\textsf{fpaddr}), then calculate the appropriate
known-valid page table walks (via \textsf{walkpgdir} Line \ref{line:call_walkpgdir} in Figure \ref{fig:mapping_code})  and update 
the appropriate L1 page table entry (Line 35 in Figure \ref{fig:mapping_code});
unmapping is the reverse of the logic we discuss here.
\looseness=-1
%\lstset{
%  columns=fullflexible,
%  numbers=left,
%  basicstyle=\ttfamily,
%  keywordstyle=\color{blue}\bfseries,
%  morekeywords={mov,add,call},
%  emph={rsp,rdx,rax,rbx,rbp,rsi,rdi,rcx,r8,r9,r10,r11,r12,r13,r14,r15},
%  emphstyle=\color{green},
%  emph={[2]cr3},
%  emphstyle={[2]\color{violet}},
%  morecomment=[l]{;;},
%  mathescape
%}
% Figure environment removed

In Figure \ref{fig:mapping_code}, we see an address ($\vaddr$) currently not
mapped to a page ($\theta \; !!\; \vaddr = \texttt{None}$). Mapping a fresh
physical page to back the desired virtual page first requires ensuring
the existence of a memory location for an appropriate L1 table entry.
The code uses a helper function \lstinline{walkpgdir} (discussed again in Section \ref{sec:traversing}).
\textsf{walkpgdir}'s postcondition contains virtual \emph{PTE} pointsto assertions ($\mapsto_{\textsf{vpte}}$)
both for ensuring partial page table walk reaching the
L1 entry (l1e) by asserting that higher levels of the page table exist (R$_{\textsf{walk}}$ in Figure \ref{fig:rwalk}), 
and for allowing access to the memory of the L1 entry via virtual address (R$_{\textsf{l1e}}$ in Figure \ref{fig:rwalk}).

% After obtaining a virtual address \textsf{pte\_addr} in \textsf{rax} backed 
% by the physical memory for the L1 entry that will be used to translate the virtual addresses
% we are mapping, we save it to \textsf{r14} to be updated later in Line 9.

%In the precondition, we see Line 12 allocates a fresh page-aligned, zero-initialized page  (at \textsf{fpaddr}),
%returning a pre-filled PTE entry in \textsf{rax} ($+3$ sets the lower 2 bits).

% , to hold the freshly
% allocated physical page address (\textsf{fpaddr}) in Line X.

We already discussed for the upper level page-tables how the entry-present checks are handled.
However, for L1 entries this check is left to the caller of the 
traversal function \textsf{walkpgdir}. In other words, unlike what we see in R$_{\textsf{walk}}$ for the upper levels where all entry-present
checks have already been performed, the specification in R$_{\textsf{l1e}}$ ensures that page table entry for L1 needs to be checked at the caller site. 
By doing so, as we see in Figure \ref{fig:mapping_code}, the page reference \textsf{fpaddr} is linked to back the virtual address \textsf{va} 
only if it is not already referring to a physical resource (Lines \ref{line:mappage_pte_present_start}--\ref{line:mappage_pte_present_end} in Figure \ref{fig:mapping_code}). 

The crucial step in addition to traversing the page table in Figure \ref{walkpgdir} is actually updating the L1 entry (Line \ref{line:updatepfn} in Figure \ref{fig:mapping_code}),
via the virtual address (\textsf{pt\_entry+KERNBASE}) known to translate to the appropriate physical address, in our example the L1
table entry address ($\textsf{PTE\_ADDR\_TO\_PFN(fpaddr)}$).

Unlike the only prior work verifying analogous code for mapping a new page~\cite{kolanski08vstte,kolanski09tphols}, our proof above
does \emph{not} need to reason directly over the operational semantics,
making this the first verification we know of for mapping a virtual memory page that 
stays entirely at the program logic level.
\looseness=-1
% By incorporating verification of the
% \lstinline|ensure_L1| function (see Section \ref{sec:traversing}), our verification also directly handles several subtle aspects which
% were axiomatized in prior work.
\ifPLDI
\else
\subsection{Unmapping a Page}
\todo[inline]{update (esp. line refs) for new mapping code}
The reverse operation, unmapping a designated page that is currently mapped,
would essentially be the reverse of
the reasoning around line 22 above: given the virtual points-to assertions for all 512
machine words of memory that the L1 entry would map,
and information about the physical location, 
full permission on the L1 entry could be obtained, allowing the construction of a
full virtual PTE pointer for it, setting to 0, and reclaiming the now-unmapped physical memory.
\fi


% % Figure environment removed

% \subsection{Change of Address Space}
% A critical piece of \emph{trusted} code in current verified OS kernels is the assembly code to change the current address space; current verified OS kernels currently 
% lack effective ways to specify and reason about this low-level operation, for reasons outlined in Section \ref{sec:relwork}.

% Figure \ref{fig:swtch} gives simplified code for a basic task switch, the heart of an OS scheduler implementation. This is code that saves the context (registers and stack)
% of the running thread (here in a structure pointed to by \lstinline|rdi|'s value shown in Lines \ref{line:start_save}--\ref{line:end_save} of Figure \ref{fig:swtch}) and restores the context of 
% an existing thread (from \lstinline|rsi| shown in abbreviated Lines \ref{line:start_restore}--\ref{line:end_restore}), including the corresponding change of address space for a target thread in another process.
% This code assumes the System V AMD64 ABI calling convention, where the normal registers not mentioned are caller-save, and therefore saved on the stack of the thread
% that calls this code, as well as on the new stack of the thread that is restored, thus only the callee-save registers and \texttt{cr3} must be 
% restored.\footnote{We are simplifying by ignoring non-integer registers (e.g., floating point, vector registers),
% and the caller-save registers should be initialized to 0 to avoid leaking information across processes, but this captures the key challenges.}
% With the addition of a return instruction, this code would satisfy the C function signature\footnote{This is the function in UNIX 6th Edition 
% with the infamous ``You are not expected to understand this'' comment~\cite{lions1996lions}.}
% \centerline{\lstinline[language=C]|void swtch(context_t* save, context_t* restore);|}\\
% A call to this code begins executing one thread (until just before Line \ref{line:end_save}) in one address space ($\rtv$), whose information will be saved in a structure at address $old$,
% and finishes execution executing a different thread in a different address space (Line \ref{line:end_restore} on) whose information is initially in $new$.

% Because this code does not directly update the instruction pointer, it is worth explaining \emph{how} this switches threads: by switching address spaces and stacks. 
% This is meant to be called with a return address for the current thread stored on the current stack when called. 
% The precondition of the return address on the initial stack requires the callee-save register values at the time of the call: those stored in the first 
% half of the code.
% Likewise, part of the invariant of the stack of the second thread, the one being restored, is that the return address on \emph{that} stack requires the saved 
% callee-save registers stored in that context to be in registers as its precondition.

% The wrinkle, and the importance of the modal treatment of assertions, is that the target thread's precondition is \emph{relative to its address space}, 
% not the address space of the calling thread, which is reflected by
% the other-space modality 
% $[\rtv']( I\texttt{ASpace}(\theta,\Xi,m) \ast \texttt{Pother})$
% in the specfication. 
% The precondition of this code,
% in context, would include that the initial stack pointer (before \lstinline|rsp| is updated)
% has a return address expecting the then-current callee-save register values and 
% suitably updated (i.e., post-return) stack in the \emph{current} (initial) address space;
% this would be part of \textsf{P} in the precondition.
% The specification also requires that
% the stack pointer saved in the context to restore expects the same of the saved registers and stack 
% \emph{in the other address space}. 
% The other-space modality plays a critical role here; \textsf{Pother} would contain these assumptions in the other
% address space.
% \looseness=-1

% % Lines 10--16 save the current context into memory (in the current address space).
% % Line 22 saves the initial page table root.
% % Lines 33--38 begin restoring the target context, including the stack pointer (line 33),
% % which may not be mapped in the address space at that time: it is the stack for the context being
% % loaded into the CPU.
% % The actual address switch occurs on line 45, which is verified with our modal rule for updating \lstinline|cr3|,
% % and thus shifts resources in and out of other-space modalities as appropriate.

% The postcondition is analagous to the precondition, but interpreted \emph{in the new address space}: the then-current (updated) stack would have a return address expecting the new (restored) register values (again, in \textsf{Pother}),
% and the saved context's invariant captures the precondition for restoring its execution \emph{in the previous address space} (as part of \textsf{P}). 

% Immediately after the page table switch, assertions about the saved and restored contexts are
% guarded by a modality for the retiring
% address space \rtv{} (Line \ref{line:modality_switch}), per
% \textsc{WriteToRegCtlFromRegModal} (Figure \ref{fig:wpdamd}),
% because
% there is no guarantee that the data structures of the previous address space are mapped in the new address space.
% The ability to transfer that points-to information out of that modality is specific to a given kernel's design. 
% Kernels that map kernel memory into all address spaces would need invariants
% that justified moving those assertions out of the other-space modality.
% % Following Spectre and Meltdown, this kernel design became less prevalent because speculative execution of accesses to kernel addresses could leak information even if the access did eventually cause a fault (the user/kernel mode permission check was done after fetching data from memory). Thus many modern kernels have reverted to the older kernel design where the kernel inhabits its own unique address space, and user processes have only enough extra material mapped in their address spaces to switch into the kernel (CPUs do not speculate past updates to \texttt{cr3}).
% \looseness=-1

% While prior work has verified context switches within a single address space~\cite{ni2007contexts}, and context switches
% without any code before or after~\cite{syeda2020formal} (i.e., not reasoning about the impact of address space change
% on what data was accessible), this is the first verification handling both.
% \looseness=-1

% \begin{comment}
% \[  
% $\specline{\exists (\entryf ,\;\entrytr,\; \entrytw,\; \entryo,\;\textsf{pte\_addr },\paddr) \; \ldotp\textsf{P} \ast  I\texttt{ASpace}(\theta,m) \ast  \texttt{r14}\mapsto_{\textsf{r}} \_ \ast \texttt{rdi}\mapsto_{r} \vaddr \ast \texttt{rax}\mapsto_{\textsf{r}} \textsf{ pte\_addr} \; \ast }_{\rtv}$
% $\specline{ \ulcorner  \texttt{addr\_L1 }(\vaddr, \entryo) = \paddr \urcorner \ast \ulcorner \texttt{entry\_present } \entryf \land \texttt{entry\_present } \entrytr \land  \texttt{entry\_present } \entrytw \urcorner \; \ast}_{\rtv}$
% $\specline{\nfpointsto{\mask\vaddr\maskfour\rtv}{\mask\vaddr\maskfouroff\rtv}\entryf\qone\naddr \; \ast \nfpointsto{\mask\vaddr\maskthree\entryf}{\mask\vaddr\maskthreeoff\entryf}\entrytr\qtwo\naddr \ast}_{\rtv}$ 
% $\specline{  \nfpointsto{\mask\vaddr\masktwo\entrytr}{\mask\vaddr\masktwooff\entrytr}\paddr\qthree\entryo \;\ast \texttt{pte\_addr} \mapsto_{\texttt{vpte}} \paddr \;(\texttt{wzero 64}) \ast \texttt{rax}\mapsto_{\textsf{r}} \texttt{pte\_addr}  }_{\rtv}$
% mov r14, rax ;; Save that before another call
% $\specline{\textsf{P} \ast  I\texttt{ASpace}(\theta,m) \ast  \texttt{r14}\mapsto_{\textsf{r}} \texttt{pte\_addr} \ast \texttt{rdi}\mapsto_{\textsf{r}} \vaddr \ast \texttt{rax}\mapsto_{\textsf{r}} \textsf{ pte\_addr} \; \ast }_{\rtv}$
% $\specline{ \nfpointsto{\mask\vaddr\maskfour\rtv}{\mask\vaddr\maskfouroff\rtv}\entryf\qone\naddr \ast \ulcorner \texttt{entry\_present } \entryf \land \texttt{entry\_present } \entrytr \land  \texttt{entry\_present } \entrytw \urcorner \ast}_{\rtv}$ 
% $\specline{  \nfpointsto{\mask\vaddr\maskthree\entryf}{\mask\vaddr\maskthreeoff\entryf}\entrytr\qtwo\naddr \ast \nfpointsto{\mask\vaddr\masktwo\entrytr}{\mask\vaddr\masktwooff\entrytr}\paddr\qthree\entryo \;\ast}_{\rtv}$
% $\specline{\texttt{pte\_addr} \mapsto_{\texttt{vpte}} \paddr \;(\texttt{wzero 64}) \ast \texttt{rax}\mapsto_{\textsf{r}} \texttt{pte\_addr}  }_{\rtv}$
% call alloc_phys_page_or_panic
% $\specline{\textsf{P} \ast  I\texttt{ASpace}(\theta,m) \ast  \texttt{r14}\mapsto_{\textsf{r}} \texttt{pte\_addr} \ast \texttt{rdi}\mapsto_{\textsf{r}} \vaddr \;\ast \nfpointsto{\mask\vaddr\maskfour\rtv}{\mask\vaddr\maskfouroff\rtv}\entryf\qone\naddr \ast}_{\rtv}$ 
% $\specline{  \nfpointsto{\mask\vaddr\maskthree\entryf}{\mask\vaddr\maskthreeoff\entryf}\entrytr\qtwo\naddr \ast \nfpointsto{\mask\vaddr\masktwo\entrytr}{\mask\vaddr\masktwooff\entrytr}\paddr\qthree\naddr \;\ast}_{\rtv}$
% $\specline{\texttt{pte\_addr} \mapsto_{\texttt{vpte}} \paddr\; (\texttt{wzero 64})  \ast \ulcorner \texttt{entry\_present } \entryf \land \texttt{entry\_present } \entrytr \land  \texttt{entry\_present } \entrytw \urcorner}_{\rtv}$
% $\specline{\exists \texttt{ fpaddr} \ldotp \ulcorner \texttt{aligned fpaddr} \urcorner \ast \texttt{rax}\mapsto_{\textsf{r}} \texttt{fpaddr+3} \ast \texttt{fpaddr} \mapsto_{\textsf{p}} (\texttt{wzero 64}) \ast \ulcorner \texttt{entry\_present (fpaddr+3)}\urcorner}_{\rtv}$
% ;; Calculate new L1 entry
% mov [r14], rax ;; store the page table entry, mapping the page
% $\specline{\textsf{P} \ast  I\texttt{ASpace}(\theta,m) \ast  \texttt{r14}\mapsto_{\textsf{r}} \texttt{pte\_addr} \ast \texttt{rdi}\mapsto_{\textsf{r}} \vaddr \;\ast \nfpointsto{\mask\vaddr\maskfour\rtv}{\mask\vaddr\maskfouroff\rtv}\entryf\qone\naddr \ast}_{\rtv}$ 
% $\specline{  \nfpointsto{\mask\vaddr\maskthree\entryf}{\mask\vaddr\maskthreeoff\entryf}\entrytr\qtwo\naddr \ast \nfpointsto{\mask\vaddr\masktwo\entrytr}{\mask\vaddr\masktwooff\entrytr}\paddr\qthree\entryo \;\ast}_{\rtv}$
% $\specline{\texttt{pte\_addr} \mapsto_{\texttt{vpte}} \paddr \;(\texttt{fpaddr+3}) \; \ast \ulcorner \texttt{entry\_present } \entryf \land \texttt{entry\_present } \entrytr \land  \texttt{entry\_present } \entrytw \urcorner }_{\rtv}$
% $\specline{\ulcorner \texttt{aligned fpaddr} \urcorner \ast \texttt{rax}\mapsto_{\textsf{r}} \texttt{fpaddr+3} \ast \texttt{fpaddr} \mapsto_{\textsf{p}} (\texttt{wzero 64}) \ast \ulcorner \texttt{entry\_present fpaddr+3}\urcorner}_{\rtv}$
% $\;\;\;\;\;\;\;\;\;\;\;\;\;\;\;\;\;\;\;\;\;\;\;\;\;\;\;\;\;\;\;\;\;\;\;\;\;\;\;\;\;\;\;\; \sqsubseteq $
% $\specline{\textsf{P} \ast  I\texttt{ASpace}(\theta,m) \ast  \texttt{r14}\mapsto_{\textsf{r}} \texttt{pte\_addr} \ast \texttt{rdi}\mapsto_{\textsf{r}} \vaddr \ast }_{\rtv}$
% $\specline{\textsf{L}_{4}\_\textsf{L}_{1}\_\textsf{PointsTo}(\vaddr,\entryf,\entrytr,\entrytw,\fpaddr+3) \ast \ulcorner \theta \;!!\;\vaddr = \texttt{None}\urcorner \; \ast}_{\rtv}$
% $\specline{\ulcorner \texttt{aligned fpaddr} \urcorner \ast \texttt{rax}\mapsto_{\textsf{r}} \texttt{fpaddr+3} \ast \texttt{fpaddr} \mapsto_{\textsf{p}} (\texttt{wzero 64}) }_{\rtv}$
% $\;\;\;\;\;\;\;\;\;\;\;\;\;\;\;\;\;\;\;\;\;\;\;\;\;\;\;\;\;\;\;\;\;\;\;\;\;\;\;\;\;\;\;\; \sqsubseteq $
% $\specline{\textsf{P} \ast  I\texttt{ASpace} (<[\vaddr:=\texttt{fpaddr}]> \theta,m) \ast}_{\rtv}$
% $\specline{\ulcorner \texttt{aligned fpaddr} \urcorner \ast \texttt{fpaddr} \mapsto_{\textsf{p}} \textsf{ wzero 64} \ast \ghostmaptoken{\delta{}s}{\rtv}{\delta}  \ast\sumwalkabs\vaddr\qfrac\fpaddr}_{\rtv}$
% $\;\;\;\;\;\;\;\;\;\;\;\;\;\;\;\;\;\;\;\;\;\;\;\;\;\;\;\;\;\;\;\;\;\;\;\;\;\;\;\;\;\;\;\; \sqsubseteq $
%   $\specline{\textsf{P} \ast  I\texttt{ASpace} (<[\vaddr:=\texttt{fpaddr}]> \theta,m) \ast \vaddr \mapsto_{\textsf{vpte}}\; \{\qfrac\} \;\fpaddr \textsf{ wzero 64}}_{\rtv}$
% $\;\;\;\;\;\;\;\;\;\;\;\;\;\;\;\;\;\;\;\;\;\;\;\;\;\;\;\;\;\;\;\;\;\;\;\;\;\;\;\;\;\;\;\; \sqsubseteq $
% $\specline{\textsf{P} \ast  I\texttt{ASpace} (<[\vaddr:=\texttt{fpaddr}]> \theta,m) \ast \vaddr \mapsto_{\textsf{v}}\; \{\qfrac\} \textsf{wzero 64}}_{\rtv}$
% \end{comment}

\section{Conclusion and Future Work}
In this work, I design corruption-robust algorithms for the Lipschitz contextual search problem. I present the \emph{agnostic checking} technique and demonstrate its effectiveness in designing corruption-robust algorithms. There are several open problems for future research. First, in the algorithm I propose for pricing loss, the schedule for agnostic checks is fixed upfront. Can the learner design an adaptive checking schedule for the pricing loss? Second, this work assumes the learner has knowledge of the Lipschitz constant $L$. Can the learner design efficient no-regret algorithms without knowledge of $L$? 
\subsubsection*{Acknowledgments}
This publication was made possible by an ETH AI Center doctoral fellowship to Manish Prajapat. We would like to thank Mohammad Reza Karimi, Pragnya Alatur and Riccardo De Santi for the insightful discussions. We thank Bhavya Sukhija and Alizée Pace for reviewing the manuscript.

The project has received funding from the European Research Council (ERC) under the European Union’s Horizon 2020 research and innovation program grant agreement No 815943 and the Swiss National Science Foundation under NCCR Automation grant agreement 51NF40 180545.


\bibliographystyle{ICRC}
\bibliography{main}

% \subsubsection*{Acknowledgments}
This publication was made possible by an ETH AI Center doctoral fellowship to Manish Prajapat. We would like to thank Mohammad Reza Karimi, Pragnya Alatur and Riccardo De Santi for the insightful discussions. We thank Bhavya Sukhija and Alizée Pace for reviewing the manuscript.

The project has received funding from the European Research Council (ERC) under the European Union’s Horizon 2020 research and innovation program grant agreement No 815943 and the Swiss National Science Foundation under NCCR Automation grant agreement 51NF40 180545.


% \begin{thebibliography}{99}
% \bibitem{...}
% ....
% \end{thebibliography}

%% Full authors list (ONLY FOR COLLABORATIONS)
%\clearpage
%\section*{Full Authors List: \Coll\ Collaboration}
%
%\noindent \textbf{Note comment afterwards:} Collaborations have the possibility to provide an authors list in xml format which will be used while generating the DOI entries making the full authors list searchable in databases like Inspire HEP. For instructions please go to icrc2021.desy.de/proceedings or contact us under icrc2021proc@desy.de.\\
%
%\scriptsize
%\noindent
%first.author$^1$, 
%second.author$^2$, 
%third.author$^3$ % .... more names
%and 
%last.author$^{n}$ \\
%
%\noindent
%$^1$first.affiliation.
%$^2$second.affiliation. % .... more affiliation
%$^{m}$last.affiliation.

\end{document}
