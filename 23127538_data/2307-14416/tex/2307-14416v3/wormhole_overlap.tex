
\documentclass[11pt]{article}
\pdfoutput=1

%\usepackage{color}

%%% Formatting
\usepackage{formatting}

%%% Custom Commands
\usepackage{shortcuts}

\usepackage{braket}

\newcommand{\sa}[1]{\textcolor{red}{(SA) #1}}

\begin{document} 
\section{Bag of gold paradox for the CFT observer}

From the microscopic quantum mechanical description of the CFT observer, there seems to be too many states on the cosmology, while the linear resources to describe it are limited by the finite microcanonical entropy of the CFT. A similar `bag-of-gold' paradox arises for high temperature PETS describing microstates of two-sided neutral black holes \cite{Balasubramanian:2022gmo}, and at zero temperature for extremal supersymmetric black holes \cite{Lin:2022rzw}. In those cases, the number of states in the black hole interior seems to overcount the Bekenstein-Hawking entropy, $S_{BH} = A/4G$.    

The resolution is that these states overlap generically in the Hilbert space of the CFT. For the black hole, the states typically overlap by an amount $\rho(E)^{-1/2}\sim e^{-S_{BH}(E)/2}$ on each microcanonical band. The magnitude of this overlap can be computed in Euclidean gravity, from connected wormhole contributions to the moments of the inner product \cite{Balasubramanian:2022gmo}.  As we will show, for the cosmology, the states overlap by a much larger amount, $\rho(E)^{-1/2}\sim e^{-S_{gas}(E)/2}$. The reason behind is that different states are distinguished by small details of the wavefunction, while the dominant part of the wavefunction is always 

To see this, consider a familly of cosmological microstates $\Psi_a[\phi]$, with $a=1,...,\Omega$, which are orthogonal in the naive effective field theory 
\be 
\int \mathcal{D}\phi \mathcal{D}\phi' \,\rho_{\Co}(\phi,\phi')\Psi_a[\phi]\Psi_b[\phi]^* = \delta_{ab}\;.
\ee 

The question is whether the corresponding CFT states $\ket{\Psi_a} $ overlap. For states which are dominated by 
\be 
a
\ee 

Consider two states $\ket{\Psi_+} $ and $\ket{\Psi_-}$, both prepared by an Euclidean path integral, which consist of the original state \eqref{eq:PETS}, with the addition of a single particle living on the cosmology $\mathscr{C}$. Let us assume that the gravitational effective theory is Einstein gravity in AdS coupled to the thin shell and to the bulk matter, with the addition of the probe particle carrying the internal degree of freedom 
\be\label{eq:bulkEFT}
I[X]=-\frac{1}{16\pi G}\int_X ( R -2\Lambda ) +\dfrac{1}{8\pi G} \int_{\partial X} K + \int_{\mathcal{W}} \sigma +  I_{\text{matter}}  + I_{\text{ct}} \;. 
\ee 
Let us assume that the extra probe particle carries an internal degree of freedom, like the spin $\sigma_z$, which is conserved in the effective field theory \eqref{eq:bulkEFT}. The two states $\ket{\Psi_{\pm}}$ correspond to the choice $\ket{\pm z}$ respectively. 

One can easily prepare the states $\ket{\Psi_{\pm}}$ with similar Euclidean path integrals, with the addition of operator insertions that create the qubit (see Fig. \ref{fig:naiveoverlap}). Assuming that effective bulk dynamics commutes with the internal $\sigma_z$ of the qubit, then the gravity calculation of the overlap gives zero 
\be 
\overline{\bra{\Psi_+}\ket{\Psi_-}} = 0\;,
\ee 
where the bar denotes that the computation has been done in the gravitational effective field theory.

% Figure environment removed

% Figure environment removed


However, when computing the square of the overlap, there exists a gravitational connected contribution to this quantity
\be\label{eq:overlapsq}
\overline{|\bra{\Psi_+}\ket{\Psi_-}|^2} = \dfrac{Z_2}{Z^+_1Z_1^-}\;,
\ee 
coming from a Euclidean wormhole solution. In this formula $Z_i = e^{-I[X_i]} Z_{\text{bulk}}[X_i]$ is the Euclidean gravitational path integral evaluated on the manifold $X_i$, where $I[X_i]$ is the gravitational action and $Z_{\text{bulk}}[X_i]$ is the 1-loop determinant of the bulk quantum fields around the saddle.

\subsubsection*{Gravitational actions}


Let's start from the classical contribution to the normalization, $I[X_1]$. We can divide the contribution into three terms (see Fig. \ref{fig:norm})
\be\label{eq:gravacnorm}
I[X_1] = I[X_L] + I[X_R] + I[X_{\mathscr{C}}] \;.
\ee 

% Figure environment removed

Each of the terms can be easily computed, with the careful addition of counterterms $I_{\text{ct}}$ which renormalize the action. The value of each contribution is 
\begin{gather}
 I[X_L] = I[X_R]  = \tbeta E_0\;,\\
 I[X_{\mathscr{C}}] =  \dfrac{d}{8\pi G}\,\text{Vol}[X_{\mathscr{C}}]\,+\,m\dfrac{d-2}{d-1}  L[\gamma_\mathcal{W}]  \label{eq:actioncosmo}\;,
\end{gather}
where $E_0 = - \frac{V_\Omega \ell^{d-2}}{8\pi G} c_d$ is the Casimir energy of the CFT, where $c_d$ is a non-negative constant which vanishes for odd values of $d$. The two quantities in \eqref{eq:actioncosmo} are given by
\begin{gather}
	L[\gamma_{\mathcal{W}}] = 2\int_{R_*}^{r_\infty} \dfrac{\text{d}R}{\sqrt{-V_{\text{eff}}(R)}}\,,\label{eq:length}\\[.4cm]
	\text{Vol}[X_{\mathscr{C}}] = \dfrac{4V_\Omega}{d} \int_{R_*}^{r_\infty}\dfrac{\text{d}R\; R^d}{f_+(R)}\, \sqrt{\dfrac{f_+(R) + V_{\text{eff}}(R)}{- V_{\text{eff}}(R)}}\;.\label{eq:vol}
\end{gather}

Similarly, we can compute the gravitational action of the wormhole $X_2$ (see Fig. \ref{fig:overlapsq})
\be\label{eq:gravacwh}
I[X_2] = 4\tbeta E_0 + 2I[X_{\mathscr{C}}] \;.
\ee 

\subsubsection*{One-loop determinants}

Note that \eqref{eq:gravacwh} is identically twice of  \eqref{eq:gravacnorm}. This means that the ratio between gravitational actions in \eqref{eq:overlapsq} is identically one, and we are left with one-loop determinants
\be 
\overline{|\bra{\Psi_+}\ket{\Psi_-}|^2} = \dfrac{Z_2}{Z^+_1Z_1^-} = \dfrac{Z_{\text{bulk}}[X_2]}{Z_{\text{bulk}}[X_1]^2}\;.
\ee 

Each one-loop determinant can be computed from a Gaussian integral, in zeta function regularization. That is, assume that the matter (and gravitons) have a quadratic action around $X_i$ of the form \footnote{One can extend this to the case of fermions. In the case of gravitons or gauge fields, one needs to account for the fact that the operator $A_i$ has zero modes (pure gauge configurations), so one needs to include Fadeev-Popov ghosts to gauge fix.}
\be 
I_{\text{bulk}}[X_i] = \dfrac{1}{2}\int_{X_i} \phi A_i \phi\;,
\ee 
where $A_i$ is some second order differential operator that depends on the background metric.In the case of an Euclidean manifold $X_i$, $A_i$ will be real, elliptic and self-adjoint.  The one-loop determinant is then
\be
Z_{\text{bulk}}[X_i] = (\det(4L^2\pi^{-1}A_i))^{-1/2}\;,
\ee 
where $L$ is some normalization constant with dimension of length. This formal expression suffers from UV divergences from the unbound spectrum of $A_i$.  In zeta function regularization, one defines the generalized zeta function in terms of the eigenvalues of $A_i$, denoted by $\lambda_{i,n}$, as
\be 
\zeta_i(s) = \sum_{n=1}^\infty \, \lambda_{i,n}^{-s}\;.
\ee 
This function is analytic for $\text{Re}(s) > \frac{d+1}{2}$ in $d+1$ dimensions, but can be analytically continued to $s=0$. It thus provides a regularization of the one-loop determinant in curved spacetimes
\be\label{eq:oneloopzetafn}
\log Z_{\text{bulk}}[X_i] = \dfrac{1}{2}\zeta_i'(0) + \dfrac{1}{2}\log (\frac{\pi}{4 L^2}) \zeta_i(0)\;.
\ee 

Obviously, the explicit computation of the one-loop determinant is very difficult for a general $X_i$ which lacks of symmetries. We will however use the trick that when the mass of our shell is very large, the spaces $X_i$ reduce to multiple copies of Euclidean AdS, which are effectively disconnected. This will give us an idea of the order of magnitude of the wormhole contribution also for smaller masses, where the computation of the eigenfunctions is much more involved, but where we don't expect for the order of magnitude of the result to change.


\subsection*{Large mass and universality}

In the large mass limit $m\ell \rightarrow \infty$, $\Delta \tau_\pm \rightarrow 0$, $\tbeta = \beta$, and we have that $X_2$ effectively becomes two disconnected copies of AdS at temperature $2\beta$. Similarly, $X_1$ becomes two copies of AdS space at temperature $\beta$. In this case the wormhole contribution is simply given by the square of the thermal partition function of a gas of particles in AdS
\be 
\overline{|\bra{\Psi_+}\ket{\Psi_-}|^2} = \dfrac{Z(2\beta)^2}{Z(\beta)^2} = e^{-2S^{(2)}_{\text{bulk}}}\;,
\ee 
i.e. $S^{(2)}_{\text{bulk}}$ is the second R\'{e}nyi entropy of the bulk quantum fields in AdS. 

In the case of a conformally coupled scalar, the second R\'{e}nyi is (see e.g. \cite{HawkingPage})
\be 
S^{(2)}_{\text{bulk}} \approx \dfrac{f_2(d)\ell^d}{\beta^d}\;,
\ee 
where $f(d)$ is a $O(1)$ number that depends on the dimensions, and can be computed using \eqref{eq:oneloopzetafn}.




\subsection*{Counting microcanonical Hilbert space dimension}

Assume that instead of a qubit, we have a particle with $k$ independent internal states in the gravitational effective field theory (similar to the west coast model). It is easy to see that the argument that we have presented generalizes further to higher moments of the overlap, where now we consider the family of states $\mathscr{F}_q = \lbrace \ket{\Psi_1}, .... \ket{\Psi_q} \rbrace$ representing different internal states of the particle on the cosmology.

The higher moments of the overlaps can be likewise computed from a connected wormhole configuration. In the universal limit they give
\be 
\left.\overline{\bra{\Psi_{i_1}}\ket{\Psi_{i_2}}\bra{\Psi_{i_2}}\ket{\Psi_{i_3}}...\bra{\Psi_{i_n}}\ket{\Psi_{i_1}}}\right|_{\text{conn.}} = \dfrac{Z(n\beta)^2}{Z(\beta)^{2n}} = e^{-2(n-1)S^{(n)}_{\text{bulk}}}\;,
\ee 
where $S^{(n)}_{\text{bulk}}$ is the $n$-th R\'{e}nyi entropyi of the radiation.

By running the resolvent argument, one can indeed see that the microcanonical Hilbert space dimension that these states span $\mathscr{F}^{E}_q = \lbrace \Pi_E\ket{\Psi_1}, ..., \Pi_E\ket{\Psi_n}\rbrace$ (for $E$ associated to $\beta$) is precisely given by the microcanonical entropy of the radiation
\be 
\text{dim}\left(\text{span}[ \mathscr{F}^{E}_q]\right) = \begin{cases}
    q \hspace{.4cm}\text{if } q \leq e^{\mathbf{S}}\,,\\
    e^{\mathbf{S}} \hspace{.4cm}\text{if } q > e^{\mathbf{S}}\,,
\end{cases}
\ee 
where the microcanonical entropy is
\be 
\mathbf{S}(E) = g(d) (E\ell)^{\frac{d}{d+1}}\;.
\ee 
Here $g(d)$ is some $O(1)$ constant that depends on the field content and the dimension. 


\MS{See if there is a regime of $\beta$ for which the AdS wormhole computing the overlap dominates but the state is still a black hole. This would mean that there overlap is much larger than naively expected.}

\MS{Provide microscopic interpretation of the counting, in terms of tails of the wavefunction.}

\end{document}