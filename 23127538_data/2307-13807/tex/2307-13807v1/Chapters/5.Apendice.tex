\section*{Appendix}
The apendix provide a list and description of the three databases utilized in this study for the estimation of probabilities and obtaining odds from various bookmakers. The variables employed in the prediction of match outcomes, with a dagger ($\dagger$) indicating their usage, and the target variable marked by an asterisk ($*$), are also specified.

\begin{multicols}{2}

% football data uk
\subsection*{Football Data U.K.}

The "Football Data U.K." database contains records representing individual matches played in the English Premier League from the 2013/2014 season through the 2020/2021 season. The database includes a range of variables that provide valuable insights into match characteristics and team performance. However, it should be noted that a subset of variables was excluded from the analysis for matches prior to 2019, as these variables were not available in the reported files during that time period.
\subsubsection*{Fetched Variables}

\begin{itemize}[leftmargin=*]
    \setlength{\itemindent}{0em}
    
    \item date\nospacecolon \  Date in day/month/year on which the match took place.
    \item hometeam\nospacecolon \  Home team name.
    \item awayteam\nospacecolon \  Name of the team playing as visitor.
    \item fthg\nospacecolon \  Goals scored at the end of the home team's game.
    \item ftag\nospacecolon \  Goals scored at the end of the game by the away team.
    \item ftr\nospacecolon \  Final score. The possible values are\nospacecolon \  \textit{H, D, A}. Which represent that the home team won, drew or that the visiting team won, respectively.
    \item referee\nospacecolon \  Name of the main referee who directed the match.
    \item \_h, \_d, \_a\nospacecolon \  The bookmakers' odds for the possible outcomes of the match. The information is collected on Tuesdays and Fridays. The bookmakers are:
        \begin{itemize}
            \item b365\nospacecolon \  Bet365.
            \item bw\nospacecolon \  Bwin.
            \item iw\nospacecolon \  Interwetten.
            \item ps\nospacecolon \  Pinnacle Sports.
            \item vc\nospacecolon \  Victor Chandler.
            \item wh\nospacecolon \  William Hill.
        \end{itemize}
    \item$\dagger$psch, pscd, psca\nospacecolon \  Pinnacle Sports' final odds, that is, an instant before the match starts, for the result that the home team wins, draws or that the away team wins.

\end{itemize}

\subsubsection*{Generated Variables}

\begin{itemize}[leftmargin=*]
    \setlength{\itemindent}{0em}
    
    \item$\dagger$matchweek\nospacecolon \  The day on which the respective matches are played.
    \item$*$result\nospacecolon \  The variable \textit{ftr} renamed.
    \item season\nospacecolon \  Season in which they are playing. If the season is 2013/2014, 13 is captured. 
    \item maxo\_\nospacecolon \  The maximum odds for the three possible outcomes (H, T, V) of the six bookmakers mentioned at the beginning. \textit{Note}\nospacecolon \  Pinnacle Sports final odds are not considered. \textit{Note 2}\nospacecolon \  These are the odds used for betting.
    \item market\_tracktake\nospacecolon \  The market commission. That is, the sum of the inverse of the maximum odds found for each outcome.
    \item diff\_\nospacecolon \  The relative difference between the odds -without the respective commission of the collected and the final Pinnacle Sports odds.
\end{itemize}


% understats
\subsection*{Understats}

The Understats database provides match-level data for each team in the English Premier League. Each record in the table represents a match that a team has played, rather than the match itself. This means that if there are 380 matches in the Premier League in a season, there will be 760 records in this table for that season.

It is important to note that observations for teams on the first matchweek of each season were removed from the dataset. This is because the variance between the last game of the previous season and the first game of the current season tends to be very large due to changes in the player market and initial team performances, as mentioned on the present work. Removing these observations helps to avoid potential biases in the data caused by these factors.

\subsubsection*{Fetched Variables}

\begin{itemize}[leftmargin=*]
    \item h\_a\nospacecolon \  Character that represents whether the team plays as home or away team, whose values are \textit{a}, \textit{h}, respectively.
    \item xG\nospacecolon \  Number of goals expected in the match by the team. 
    \item xGA\nospacecolon \  Number of goals expected in the match by the opposing team.
    \item npxG\nospacecolon \  Number of expected goals in the match by the team without taking penalties into account.
    \item npxGA\nospacecolon \  Number of goals expected in the match by the opposing team without taking penalties into account.
    \item npxGD\nospacecolon \  Difference between npxG and npxGA.
    \item deep\footnote{These statistics present great inconsistencies with respect to the official Understats page. Likewise, since they were obtained through an R and Python package, the methodology with which these variables were obtained is unknown.}\nospacecolon \  Number of passes completed by the team in the last quarter of the court -on the opposing team's side.
    \item deep\_allowed\footnotemark[1]\nospacecolon \  Number of passes completed by the opposing team in the last quarter of the court -on the team's side-.
    \item scored\nospacecolon \  Number of goals scored by the team in the match.
    \item missed\nospacecolon \  Number of goals conceded by the opposing team in the match.
    \item xpts\nospacecolon \  Number of expected points. It is the expected result for the team.
    \item result\nospacecolon \  Result of the match for the team. Possible values are \textit{w, d, l} representing that the team won, drew or lost the match, respectively.
    \item date\nospacecolon \  Date in year-month-day when the match took place.
    \item wins, draws, loses\nospacecolon \  Dummy variables representing whether the team won, drew or lost the match, respectively.
    \item pts\nospacecolon \  Points obtained by the result of the match for the team. Winning, drawing or losing awards 3, 1 and 0 points, respectively.
    \item ppda.att\footnotemark[1]\nospacecolon \  Total passes made by the team when attacking divided by the number of defensive actions of the opposing team (interceptions + tackles + fouls). Metric suggested by Colin Trainor.
    \item ppda.def\footnotemark[1]\nospacecolon \  Total passes made by the team when defending divided by the number of defensive actions by the opposing team.
    \item ppda.att\footnotemark[1]\nospacecolon \  Total passes made by the opposing team when attacking divided by the number of the team's defensive actions (interceptions + tackles + fouls). Metric suggested by Colin Trainor.
    \item ppda\_allowed.def\footnotemark[1]\nospacecolon \  Total passes completed by the opposing team while defending divided by the number of defensive team actions.
    \item team\_id\nospacecolon \  Id with which Understats identifies the team.
    \item team\_name\nospacecolon \  Name with which Understats identifies the team.
    \item league\_name\nospacecolon \  Name by which Understats identifies the league.
    \item year\nospacecolon \  Season number of the match. If the season is 2013/2014, 2013 is captured.
    \item matchweek\nospacecolon \  Day of the season of the current match. There are 38 matchweeks in total.
\end{itemize}

\subsubsection*{Generated Variables}
\begin{itemize}[leftmargin=*]
    \item$\dagger$position\_table:The position in tables for the current day before the games.
    \item$\dagger$total\_points\nospacecolon \  The total points of the team for the current day before the games.
    \item $\dagger$promoted\_team\nospacecolon \  Dummy variable indicating whether the team was promoted to the EPL in the current season.
    \item$\dagger$big\_six\nospacecolon \  Dummy variable that indicates whether the team is a Big Six. That is, if the team is Arsenal, Chelsea, Liverpool, Manchester City, Manchester United or Tottenham.
    \item season\nospacecolon \  The last two digits of the variable \texttt{year}.
    \item$\dagger$npxGD\_ma\nospacecolon \  It is the weighted average, with $\xi = 0.1$, of the \textit{npxGD} from the first record of the team until the current matchday prior to the games.
  \item$\dagger$npxGD\_var\nospacecolon \  Is the weighted variance, with $\xi = 0.1$, of the \textit{npxGD} from the team's first record to the present day prior to the games.    
    
\end{itemize}


\subsection*{SoFIFA}

The author utilized web scraping techniques to download tables from the SoFIFA page for the English Premier League. The data obtained represents the statistics of teams for each week of each season, with a one-matchday delay to reflect the team's status in the corresponding week prior to the matches to be played.

In the dataset, each record represents a team in a specific week. However, there are cases where more than one team's statistics were reported in a single week. To address this, only the last record reported on the page for each week was considered. In instances where there is no record available for a particular week, data from the previous week was utilized instead.

\subsubsection*{Fetched Variables}
\begin{itemize}[leftmargin=*]
    \item name\_team\nospacecolon \  Name under which SoFIFA identifies the team.
    \item id\nospacecolon \  Id with which SoFIFA identifies the team.
    \item$\dagger$ova\nospacecolon \  Rating from 1 to 100 of the team's overall performance up to that week.
   \item$\dagger$att\nospacecolon \  Rating from 1 to 100 of the team's attack up to that week.
    \item$\dagger$mid\nospacecolon \   Rating from 1 to 100 of the team's average up to that week.
    \item$\dagger$def\nospacecolon \  Rating from 1 to 100 of the team's defense up to that week.
    \item$\dagger$transfer\_budget\nospacecolon \  Budget for the transfer market, in millions of euros, of the team for that season.
    \item speed\footnote{Despite being variables with excellent information, for the 2019 seasons onwards, SoFIFA stopped updating these values and they became constant for all teams. So these variables were no longer, in their entirety, informative.}\nospacecolon \  Type of speed the team plays with.
    \item dribbling\nospacecolon \  Type of the number of dribbles with which the team plays\footnotemark[2]..
    \item passing\nospacecolon \  Type of passes with which the team plays. They can be very risky, normal or safe passes.\footnotemark[2].
    \item positioning\nospacecolon \  Formation with which the team plays.
    \item crossing\footnotemark [2]\nospacecolon \  Type of band changes in the passes with which the team plays.
    \item aggression\footnotemark[2]\nospacecolon \  Aggressiveness with which the team defends.
    \item pressure\footnotemark[2]\nospacecolon \  Pressure with which the team defends.
    \item team\_width\nospacecolon \  Width of the formation with which the team plays.
    \item defender\_line\footnotemark[2]\nospacecolon \  Type of mark with which the team defends.
    \item dp\nospacecolon \  Number of the team's domestic prestige. It is rated from 1 to 20.
   \item$\dagger$ip\nospacecolon \  Number of the international prestige of the team. It is graded from 1 to 20.
    \item players\nospacecolon \  Number of players registered by the team to play in the current EPL season.
    \item$\dagger$saa\nospacecolon \  Average age of the starting roster for that season as of that date.
    \item taa\nospacecolon \  Average age of the team for that season at that date.
    \item date\nospacecolon \  Date in year-month-day when SoFIFA published the EA Sports data of the teams.
    \item fifa\nospacecolon \  Name and number of the EA Sports FIFA video game.
    \item year\_week\nospacecolon \  Date in year-week when SoFIFA published the EA Sports data of the teams. The date is in ISO 8601 format.

\end{itemize}

\subsubsection*{Generated Variables}
No new variables were transformed or generated.

\end{multicols}

\subsection*{Main Database}
The main database used in the machine learning models consists of various variables, each undergoing a specific transformation before being used to train the neural networks. It should be noted that the standardization or normalization transformations applied to the variables have a time window of one matchweek, meaning that the calculations are based on data from the same week. These transformations are important for ensuring that the variables are appropriately scaled and prepared for input into the neural networks.
\begin{multicols}{3}
\begin{enumerate}[leftmargin=*]
    \item matchweek\footnote{Normalized. That is for say $s = (x - x_{(1)}) / (x_{(n)} - x_{(1)})$.}

    \item position\_table\_home\footnote{Inversely normalized. In other words $s = (x_{(n)} - x) / (x_{(n)} - x_{(1)})$.}
    \item total\_pts\_\footnote{Standarized. i.e. $t = (x - \bar{x}) / \hat{s}$.}
    \item npxGD\_ma\_home
    \item npxGD\_var\_home
    \item big\_six\_home
    \item promoted\_team\_home

    \item position\_table\_away\footnotemark[15]
    \item total\_pts\_away\footnotemark[16]
    \item npxGD\_ma\_away
    \item npxGD\_var\_away
    \item big\_six\_away
    \item promoted\_team\_away

    \item ova\_home\footnotemark[16]
    \item att\_home\footnotemark[16]
    \item mid\_home\footnotemark[16]
    \item def\_home\footnotemark[16]
    \item transfer\_budget\_home\footnote{Normalized, with the maximum observation cliffed at 100.}
    \item ip\_home\footnotemark[16]
    \item saa\_home\footnotemark[14]

    \item ova\_away\footnotemark[16]
    \item att\_away\footnotemark[16]
    \item mid\_away\footnotemark[16]
    \item def\_away\footnotemark[16]
    \item transfer\_budget\_away\footnotemark[17]
    \item ip\_away\footnotemark[16]
    \item saa\_away\footnotemark[14]

    \item proba\_h
    \item proba\_d
    \item proba\_a
\end{enumerate}
\end{multicols}