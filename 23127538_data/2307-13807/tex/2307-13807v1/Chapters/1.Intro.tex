% ==================================================
%           Capitulo 1
% ==================================================
\section{Introduction}

This study aims to address the question of which bets a rational gambler, considering their risk appetite, should select in order to maximize expected utility. The concept of optimality is defined based on the Von Neumann-Morgenstern Classical Utility Theory. We employ deep learning techniques to estimate event odds, and subsequently employ the Sharpe Ratio and Kelly's financial criteria to identify the optimal set of bets.

The primary objective of this study is to assess the performance of the Sharpe Ratio and the Kelly Criterion within the context of real-world sports betting, specifically during the second half of the 2020-2021 season of the English Premier League (EPL). Additionally, compare the impact of underlying assumptions on the aforementioned criteria, shedding light on their significance.


%%%%%%%%%%%%%%%%%%
% apuestas       %
%%%%%%%%%%%%%%%%%%
\subsection{Sports Betting}
The analysis of a bet can be approached as a decision problem defined by the tuple $(D, E, C, (\succeq))$. Within our mathematical framework, this decision tuple $(D, E, C, (\succeq))$ encompasses fundamental elements for making rational decisions within the realm of sports betting. The decision set $D$ represents the available choices or bets that a bettor can select. The events set $E$ defines the potential outcomes or events associated with these bets. The consequence set $C$ encompasses the various outcomes or consequences that arise from each event, denoted as $c = c(d, e)$. The preference relation $(\succeq)$ captures the bettor's subjective preferences and enables comparisons between different bets based on personal criteria \cite{de2017theory}. By defining and analyzing these elements, our approach aims to assist rational gamblers in choosing optimal bets that maximize their expected utility and align with their preferences \cite{mendoza2011estadistica}. Specifically, we seek the optimal bet $d_{*} \in D$ based on the bettor's preferences $(\succeq)$. These preferences are contingent upon the consequences $c \in C$, which in this case include the odds and probability of the event $e \in E$ \cite{zimmermann2013predicting}. 

We denote $\ell$ as the percentage of the wealth wagered on an event $e$, hence $\underline{\ell}$ represents the wager vector. Consequently, the decision $d_{\underline{\ell}}$ signifies the act of placing a bet of $\underline{\ell}$ on the events $e \in E$.

\subsubsection{Odds}
Within the domain of sports betting, let $W_{0}$ represent the initial wealth and $W_{1}$ denote the wealth obtained for placing a bet on event $e$ but not taking in account the initial quantity $W_{0}$. In the event's occurrence, the outcome manifests as a gross profit of $W_{1} + W_{0}$, while in its non-occurrence, the bet is forfeited, resulting in a loss. The probabilities associated with these outcomes are denoted by $p$ and $1-p$ respectively. It is assumed that all wagers result in a positive return, that is, $W_{1} > 0$.

On the other hand, the odds $\ou$ are defined as the ratio between the gross profit and the initial wealth, i.e. $\ou = (W_{1} + W_{0})/W_{0} = 1 + W_{1}/W_{0}$. It is \textit{assumed that the odds and probabilities of the events are fixed over time}.

\subsubsection{Returns}
Formally, the net return $\varrho$ from betting \$1 on event $e$ is the random variable $\varrho = \ou - 1$ with probability $p$ and $\varrho = - 1$ with probability $1-p$. In the case where there are $m$ events, it is defined as the random vector $\underline{\varrho}$, where the $i$-th entry is the random variable $\varrho_{i}$. 
To facilitate notation, we introduce the odds vector $\underline{\ou}:= (\ou_{1}, \ou_{2}, \dots, \ou_{m})'$, representing the individual odds per event. Similarly, we define the probability vector $\underline{p}$ in the same manner. Additionally, we denote the diagonal matrix of odds as $D_{\ou} := \text{diag}(\ou_{1}, \ou_{2}, \dots, \ou_{m})$.
\begin{equation} \label{eq:1.ganancia_neta}
    \text{i.e.} \quad \underline{\varrho} = D_{\ou} \mathscr{m} - \mathbf{1}, \quad \mathscr{m} \sim \text{multinomial}(1; \underline{p}).
\end{equation}

Likewise, the total return $R(\underline{\ell})$ of one hundred percent of the initial wealth for betting $\ell_{i}$ percent of the wealth to the event $e_{i}$ is equal to the initial wealth plus the random gains or consequences \cite{whitrow2007algorithms}. In mathematical notation,
\begin{equation} \label{eq:1.ganancia_total}
    R(\underline{\ell}) = 1 + \sum_{i=1}^{m}\ell_{i} \varrho_{i} = 1 + \underline{\ell}' \underline{\varrho}.
\end{equation}

The analysis assumes certain conditions regarding the nature of the betting system. Specifically, it assumes the absence of \textit{short selling} and borrowing, and assumes that money is infinitely divisible with no minimum bet requirement. These assumptions can be summarized as follows: First, it is implied that the wager amount $\ell_{i}$ for each event $i$ is non-negative ($\ell_{i} \geq 0$) for all $i$. Second, the assumption is made that the total wager across all events satisfies the constraint $\sum_{i}^{m} \ell_{i} \leq 1$. Lastly, the last two assumptions state that the wager amount $\ell_{i}$ for each event $i$ is bounded within the range $[0,1]$.

\subsubsection{Betting Market} \label{sct:1.1MercadoApuestas}
By definition, for a bet to be fair one would expect the net return to be zero, i.e. $\mathbb{E}[\varrho] = 0$. The above happens if and only if $\ou = 1/p$. But, the odds of a bookmaker are always of the form $\sum_{i}^{m}1/\ou_{i} = 1 + tt$ \cite{sumpter2016soccermatics}, where the $tt$ is known as the commission (commonly known as track take) that a casino charges. Therefore $tt > 0$. However, since the betting market is a non-efficient market \cite{jakobsson2007testing}, one can find odds $\underline{\ou}$ of different casinos such that $tt \leq 0$. When the market commission is negative, arbitrage is obtained \footnote{By arbitrage, it is mean that there is a price advantage between bookmakers such that, regardless of the outcome, a fixed strategy \textit{always} makes money.} . The strategy that exploits this phenomenon is of the form $\ell_{i}^{(A)} := \ou_{i}^{-1}/\sum_{j}\ou_{j}^{-1}$, since fixing the event $e_{i}$ the total return $R(\underline{\ell}_{A}) = 1 + \ell_{i}^{(A)} \ou_{i} - \sum_{j}^{m} \ell_{j}^{(A)} = 1/(1+tt), \quad \forall e_{i} \in E$. It will be shown below, that, this strategy matches the vector of strategies found by the Sharpe Ratio Criterion for non-efficient markets.

%%%%%%%%%%%%%%%%%%
% incertidumbre %
%%%%%%%%%%%%%%%%%%

\subsection{Uncertainty}
This section provides a compilation of key findings from Frequentist Statistical Inference and Information Theory, which serve as justifications for the methods employed in this research.
\subsubsection{Information Theory}
In information theory, the concept of entropy is introduced to quantify the uncertainty associated with a phenomenon represented by the random variable $X$. Entropy, denoted as $H(X)$, is defined as $H(X) := - \mathbb{E}_{F}[\log(X)] = - \int{\mathcal{X}}\log d F(x)$, where the random variable $X$ follows a distribution characterized by $F$ \cite{shannon2001mathematical}. In other words, we express $X$ as $X \sim F$. It is worth noting that the density function $f$ is related to the distribution function $F$ through the expression $f(x) = \frac{d}{dx}F(x)$. The entropy tends to approach $0^+$ when uncertainty is low and approaches infinity in the presence of high uncertainty. Cross-entropy, denoted as $H(F, G) := - \mathbb{E}_{F}[\log(g(X))]$, quantifies the difference between distributions $F$ and $G$ for the same phenomenon. Furthermore, the Kullback-Leibler Divergence (KL-Divergence) \cite{kullback1997information} provides a means to estimate the disparity between distributions $F$ and $G$, which is given by 
\begin{equation}\label{eq:1kld}
    \dkl{F}{G} := \mathbb{E}_{F}\left[ \log \left(f(X)/g(X) \right) \right].
\end{equation}
Two important properties of the KL-Divergence are its direct relationship to cross-entropy and its equivalence to the minimization of KL-Divergence when maximizing the likelihood of a random sample. This equivalence is demonstrated through $\max\{L(\underline{\theta}; \underline{x})\} = 
\max \left \{\frac{1}{n}\sum_{j}^{n}\log(f(\underline{x}_{j}; \underline{\theta}))  \right\} = \max \left \{ \mathbb{E}_{\hat{F}_{\text{emp}}}[\log(f(\underline{x}_{j}; \underline{\theta}))] \right\} = \min \{H(\hat{F}_{\text{emp}}, F)\} = \min \{\dkl{\hat{F}_{\text{emp}}}{F}\}$ \cite{nielsen2015neural}.

\subsubsection{Deep Learning}
In the realm of event prediction, deep learning methods have gained prominence due to their effectiveness in nonlinear scenarios. They excel in identifying relationships between covariates and exhibit flexibility in optimizing objective functions to address diverse requirements for the same phenomenon \cite{nielsen2015neural}. For instance, optimizing the KL-Divergence for observed and predicted data is a valuable tool employed to estimate the odds of EPL matches in the study.

With this approach, it becomes possible to make more informed wagers with bookmakers. However, to complete the process, it is essential to determine the specific events to bet on and the corresponding wager allocation based not only on the probabilities estimated but also on the market odds.


%%%%%%%%%%%%%%%%%%
% utilidad       %
%%%%%%%%%%%%%%%%%%

\subsection{Utility Theory}\label{sct1:utility}
In order to avoid contradictions and paradoxes  associated with the moral value given to money, this study adopts the Von Neumann-Morgenstern Utility Axioms \cite{von2007theory} as the basis for its methodology. Leveraging this theory offers several notable advantages, including the recognition that the nominal value of money differs from its moral value, as well as the establishment of a clear correspondence between qualitative preferences and quantitative utilities in the context of gambling.

Consider a probability space $(\Omega, \mathscr{F}, \mathbb{P})$, where $\Omega$ represents the sample space, $\mathscr{F}$ denotes the associated sigma algebra, and $\mathbb{P}$ is the probability measure. Within this context, let $X$ be a random variable that follows a distribution characterized by $F$, and takes on values $x$ belonging to the set $\mathcal{X}$. The probability distribution $F$ is referred to as a "lottery" over the set $\mathcal{X}$. The objective is to establish preference relations $(\succeq)$ over the set of lotteries, denoted by $\mathscr{L} = \{F | F \text{ is a probability distribution over } \mathcal{X}\}$ \cite{von2007theory}. In essence, making a decision $d \in D$ to choose a lottery $F \in \mathscr{L}$ is tantamount to studying $D$ itself, given the fixed probabilities. Relaxing the assumption of fixed probabilities would transform the gambling problem into a Bayesian framework \cite{mendoza2011estadistica}.

%\subsubsection{Utilidades}
As previously mentioned, the moral value of money differs from its nominal value \cite{kahneman2003perspective} \cite{bernoulli1954exposition}. To capture this distinction, the moral value associated with a monetary outcome $x$ is modeled using a Bernoulli Utility Function $u\nospacecolon\mathcal{X} \rightarrow \mathbb{R}$. Similarly, to quantify the utility of a lottery $F$, a Von Neumann-Morgenstern Utility Functional $U\nospacecolon\mathcal{X} \rightarrow \mathbb{R}$ is employed. According to the Expected Utility Theorem \cite{von2007theory}, the utility functional can be expressed as $U(F) = \int_{\mathcal{X}} u dF = \mathbb{E}_{F}[u(X)]$. Additionally, it follows that $U(F_X) \geq U(F_Y) \iff F_X \succeq F_Y$ \cite{von2007theory}. Consequently, a gambler's preferences can be quantified through the utility function $u$, which is determined by the individual's risk profile.

\subsubsection{Utility Functions}
In economic practice, the utility function $u$ is commonly assumed to be an increasing function that exhibits decreasing marginal rates of substitution. This implies that $u'(x) > 0$ and $u''(x) < 0$ for $x \in \mathcal{X}$. Such assumptions capture the notion that individuals with higher wealth tend to exhibit lower levels of risk aversion \cite{kahneman2011thinking}.

Modern Portfolio Theory, pioneered by Harry Markowitz, suppose a quadratic utility function, where $u(x)$ is a polynomial of degree two, given by $u(x) = \beta_0 + \beta_1 x + \beta_2 x^2$ \cite{markowitz1968portfolio}. Consequently, the utility of a lottery can be expressed as \nospacecolon
\begin{equation} \label{eq:1.quadratic_utility}
U(F) = \mathbb{E}_F[X] = \beta_0 + \beta_1 \mathbb{E}[X] + \beta_2 (\text{Var}(X) + \mathbb{E}[X]^2) = W(\mu, \sigma).
\end{equation}
Here, $W$ denotes a utility function that depends on the mean $\mu$ and variance $\sigma^2$ of the random variable $X$. Thus, the utility of a lottery can be fully characterized by its mean and variance. In fact, $F_1 \succeq F_2 \iff W(\mu_1, \sigma_1) \geq W(\mu_2, \sigma_2)$. To align with the principle that greater wealth is always preferred, the function $W$ must be monotonically increasing in $\mu$, while also being monotonically decreasing in $\sigma$. In other words, for the same expected return, individuals with risk-averse preferences prefer lotteries with lower variance.

On another note, Daniel Bernoulli argued in his work "Exposition of a New Theory of the Measurement of Risk" that the change in utility experienced by an individual is inversely proportional to their wealth \cite{bernoulli1954exposition}. Consequently, Bernoulli suggested that utility functions follow a logarithmic form\nospacecolon
\begin{equation}\label{eq:1.log_utility}
u'(x) = \frac{1}{x} \implies u(x) = \log(x) + C.
\end{equation}
Once the utility functions have been characterized and the probabilities of the events have been estimated, the objective is to identify the lottery that maximizes expected utility in order to determine the optimal bet.