% ==================================================
%           Capitulo 2
% ==================================================
\section{Betting Strategies}

In this section, we address the problem of determining the best strategy for a set of $r$ random rewards within the betting system described previously.

%%%%%%%%%%%%%%%%%%
% sharpe ratio   %
%%%%%%%%%%%%%%%%%%
\subsection{Sharpe Ratio}
Under the assumption (\ref{eq:1.quadratic_utility}) that a rational gambler's utility follows a quadratic form, the objective is to identify the best strategy in the universe $\Psi = \{\psi = (\mu, \sigma)| \sum_{k = 1}^{r} \ell_{k} = 1\}$ \cite{markowitz1968portfolio}. Here, $\mu = \sum_{k}^{r} \ell_{k} \mu_k$ represents the return of the portfolio, and $\sigma^{2} = \underline{\ell}' \Sigma \underline{\ell}$ denotes the portfolio's variance. The covariance matrix $\Sigma$ captures the covariances between the $r$ random rewards, while $\mu_{k} = \mathbb{E}[\varrho_{k}]$ represents the expected value of bet $k$.

A rational strategy aims to minimize the portfolio variance while maintaining a specified expected return level $\mu_{*}$. Such a strategy $\underline{\ell}_{*}$ can be obtained by solving the following optimization problem\nospacecolon
\begin{equation} \label{eq:3.min_fixedreturn}
    \argmin_{\substack{\underline{\ell} \geq \mathbf{0}}} \{ \underline{\ell}'  \Sigma \underline{\ell}\} \quad \text{subject to} \quad 
        \underline{\ell}' \underline{\mu} = \mu_{*}, \quad
        \sum_{k}^{r} \ell_k = 1.
\end{equation}

In the context of sports gambling, where simultaneous bets are placed, the portfolio return is given by $\underline{\mu} = D_{\underline{\ou}}\underline{p} - \mathbf{1}$, and the covariance matrix is $\Sigma = D_{\ou}\left(\text{diag}(\underline{p}) - \underline{p} \underline{p}'\right) D_{\ou}$, as shown in equation (\ref{eq:1.ganancia_neta}). In the case of simultaneous bets, $\Sigma$ becomes a block diagonal matrix, denoted as $\Sigma = \text{diag}\left(\Sigma_{1}, \Sigma_{2}, \dots, \Sigma_{r}\right)$.

The set of optimal portfolios with minimum variance, for all possible return levels, is referred to as the Efficient Frontier \cite{elbannan2015capital}. To identify the best portfolio within this optimal set, the Sharpe Ratio was used, which is defined as the ratio between the difference of the portfolio return and the risk-free rate $R_{f}$, and the standard deviation of the portfolio \cite{sharpe1998sharpe}. The Sharpe Ratio is given by\nospacecolon
$
S(\underline{\ell}) := (\underline{\ell}' \underline{\mu} - R_{f})/\sqrt{\underline{\ell}' \Sigma \underline{\ell}}.
$
To facilitate the optimization process, it is beneficial to transform the Sharpe Ratio problem into a convex optimization problem by introducing an additional dimension. This transformation helps avoid issues related to non-convexity and local optima \cite{cornuejols2006optimization}. It is introduced a change of variable $\underline{y} = \kappa \underline{\ell}$, assuming a feasible solution exists such that $\underline{\ell}' \underline{\mu} > R_{f}$, and fix a scalar $\kappa > 0$ such that $(\underline{\mu} - R_{f}\mathbf{1})'\underline{y} = 1$. The resulting convex optimization problem in $r+1$ dimensions is as follows \cite{cornuejols2006optimization}\nospacecolon

\begin{equation} \label{eq:2.min_sharperatio_convex}
\argmin_{\substack{\underline{y} \geq \mathbf{0}}} \left \{ \underline{y}' \Sigma \underline{y} \right \} \quad \text{subject to} \quad 
    \begin{matrix}
        \left(\underline{\mu} - R_{f}\mathbf{1}\right)'\underline{y} = 1 \\
        \quad \sum_{k}^{r}y_{k} = \kappa \\
        \kappa > 0
    \end{matrix} .
\end{equation}

As mentioned in Section \ref{sct:1.1MercadoApuestas}, in a sports betting market where the commission $tt$ is negative and under the strategy $\underline{\ell}_{A}$, the gross return $R(\underline{\ell}_{A}) > 1$. The Sharpe Criterion helps identify such market inconsistencies by converging on the optimal strategy $\ell_{A}$. This occurs because $\mathbb{E}[R(\underline{\ell}_{A})] = \mathbb{E}[1 + \underline{\ell}_{A}' \underline{\varrho}] = 1 + \underline{\ell}_{A}'(D{\ou}\underline{p} - \mathbf{1}) = 1/(1 + tt) > 1$. Furthermore, $\text{Var}(R(\underline{\ell}_{A})) = \underline{\ell}_{A}' \text{Var}(D_{\ou} \mathscr{m} - \mathbf{1}) \underline{\ell}_{A} = \underline{\ell}_{A}' (D_{\ou}(\text{diag}(\underline{p}) - \underline{p} \underline{p}') D_{\ou}) \underline{\ell}_{A} = 0$. Thus, the optimal strategy under quadratic utilities and arbitrage is $\underline{\ell}_{A}$. This is due to the fact that the Sharpe Ratio is positively infinite, as the numerator is positive and the denominator is zero, assuming the risk-free asset is zero, which is reasonable for 90-minutes bets.

In summary, assuming quadratic utilities, the mean and variance of the returns $R_k$ alone provide both necessary and sufficient information to determine the optimal market portfolio that maximizes the Sharpe Ratio Criterion, which is a crucial aspect discussed in this section. It is important to highlight that this criterion exploits arbitrage opportunities, enabling strategies with no risk at all. Furthermore, the allocation between the portfolio and the risk-free asset $R_f$ can range from 0\% to 100\% of the total wealth, depending on the individual's risk tolerance. However, it is important to note that, when investing the entire budget in such a portfolio, the probability of ruin becomes positive which is a downside of this model. 

%%%%%%%%%%%%%%%%%%
% kelly criterion%
%%%%%%%%%%%%%%%%%%
% NO GPT
\subsection{Kelly Criterion}
The Kelly Criterion is a strategy aimed at maximizing long-term wealth by effectively balancing the potential for large returns with the risk of losses \cite{thorp1966beat}. The formula for determining the optimal strategy, denoted as $\ell_{*}$, is derived as follows.

\subsubsection{Classical Bivariate Kelly Criterion}
Consider a sequence of random rewards $\left\{R_{j}\right\}_{j}^{n}$, and let $W_{n}$ denote the final wealth of an individual who reinvests their returns according to a fixed strategy $\ell$. At time $n$, the individual's wealth is given by $W_n = W_{0} \prod_{j}^{n}R_{j}(\ell)$, where $W_{0}$ represents the initial wealth. By defining $G_{n} := W_{n}/W_{0}$ and taking logarithms, obtain the random walk expression  $G_{n} = \sum_{j}^{n}\log{(R_{j}(\ell))}$, which exhibits a drift term equal to the expected value $\mathbb{E}[\log{R_{j}(\ell)}]$. In the other side, if $S_n$ is the number of victories at time $n$ then $S_{n} \sim \text{binomial}(n, p)$. Hence, the following relationship\nospacecolon
\begin{equation*}
    \begin{split}
        W_n &=\underbrace{(1 + (\ou - 1) \ell)^{S_{n}}}_{\text{Winnings}}\underbrace{(1 - \ell)^{n - S_{n}}}_{\text{Losses}} W_0,  \\
        \iff G_{n} &= S_{n} \log{(1 + (\ou - 1) \ell)} + (n - S_{n})\log{(1 - \ell)}.
    \end{split}
\end{equation*}
Since $G_{n}$ is a sum of independent and identically distributed (i.i.d.) random variables $\log(R_{j}(\ell))$,  according to Borel's Law of Large Numbers \cite{slln_borel},
\begin{equation} \label{eq2:log_growth}
     \lim_{n \to \infty} \frac{1}{n} G_{n}(\ell) = p \log{(1 + (\ou - 1) \ell)} + (1 - p) \log{(1 - \ell)}, \quad {\text{with probability 1.}}
\end{equation}
The expression (\ref{eq2:log_growth}) denoted as to $G(\ell)$ is defined as the \textit{wealth log-growth rate} by John Kelly \cite{kelly2011new}. Since $G$ is a function of the strategy $\ell$, taking the derivative of $G$ with respect to $\ell$ eads us to the optimal solution. Thus,
\begin{equation} \label{eq2:kelly_criterion}
    G'(\ell_{*}) = 0 \iff \ell_{*} = \frac{\ou p - 1}{\ou - 1}.
\end{equation}
Recalling that in fair gambling (\ref{sct:1.1MercadoApuestas}), the odds of the event $e \in E$ are the reciprocal of the probabilities of this events. However, if $1/\ou = \tilde{p} \neq p$, hence $\ell_{*} = (p-\tilde{p})/(1-\tilde{p})$. Rearranging terms, it is obtained,
\begin{equation} \label{eq2:kelly_dkl}
    G(\ell_{*}) = p \log \left (1 + (\ou - 1)\frac{p - \Tilde{p}}{1 - \Tilde{p}} \right) + (1-p) \log \left(1 - \frac{p - \Tilde{p}}{1 - \Tilde{p}} \right) 
    = p \log \left(\frac{p}{\Tilde{p}} \right) + (1-p) \log \left(\frac{1 - p}{1 - \Tilde{p}} \right) 
    = \dkl{p}{\Tilde{p}}.
\end{equation}
Thus, the maximum log-growth is equal to the KL-Divergence (\ref{eq:1kld}). Therefore, the greater the disparity between the odds and the actual probability observed by the bettor, the greater the competitive advantage.

\subsubsection{Properties} \label{2ss:properties}
The Kelly Criterion possesses several noteworthy properties that contribute to its significance and effectiveness in optimizing long-term wealth accumulation.

Firstly, the log-growth rate $G(\ell)$ associated with the Kelly Criterion exhibits a unique optimal strategy, as established by Eduard Thorp in his paper "Optimal gambling systems for favorable games" \cite{thorp1969optimal}.There exists a critical threshold $\ell_{c} > \ell_{*}$, where $\ell_{*}$ represents the Kelly Criterion strategy, such that $G(\ell)$ transitions from positive to negative, reaching a value of zero. This property remarks emphasizes the distinct nature of different strategies in relation to their alignment with this critical threshold $\ell_{c}$.

Thorp's research \cite{thorp1969optimal} also underscores the significant impact of the chosen strategy on capital growth. When $G(\ell) > 0$, the wealth $W_{n}$ grows infinitely with probability 1, highlighting the potential for substantial wealth accumulation. Conversely, for $G(\ell) < 0$, $W_{n}$ converges to zero over time. In the case where $G(\ell) = 0$, the wealth exhibits interesting behavior, with the upper limit $\lim\sup W_{n}$ tending to infinity and the lower limit $\lim\inf W_{n}$ approaching zero as the investment horizon extends indefinitely. These findings demonstrate the sensitivity of capital growth to the chosen strategy and its profound influence on long-term financial outcomes.

Additionally, the superiority of the Kelly Criterion strategy $\ell_{*}$ over alternative strategies $\ell$ is established by Breiman \cite{breiman1961optimal}. Irrespective of the specific alternative strategy employed, portfolios adhering to the Kelly Criterion consistently outperform other strategies with probability one in terms of wealth accumulation. As the investment horizon extends indefinitely, the wealth $W_{n}(\ell_{})$ of a portfolio following the Kelly Criterion experiences infinite growth relative to the wealth $W_{n}(\ell)$ of portfolios employing alternative strategies. That is to say $\lim W_{n}(\ell_{*})/W_{n}(\ell) = \infty$ when $n \to \infty$, with probability 1. This highlights the remarkable advantage of the Kelly Criterion in maximizing long-term wealth accumulation.

The properties of the Kelly Criterion underscore its significance and effectiveness in optimizing long-term wealth accumulation. The distinct nature of strategies in relation to the critical threshold, the sensitivity of capital growth to the chosen strategy, and the superior performance of the Kelly Criterion strategy over alternatives all contribute to its importance. By aligning with the Kelly Criterion, investors can enhance their wealth accumulation potential, leading to more favorable financial outcomes in the long run.

\subsubsection{Multivariate Kelly Criterion}
Expanding Kelly's Criterion to encompass multivariate bets, where there are $m$ possible events associated with the same phenomenon, when event $e_{i}$ occurs the raw return is denoted as $R(e_{i}; \underline{\ell}) = 1 + \ou_{i}\ell_{i} - \sum_{j}^{m}\ell_{j}$, where $\underline{\ell}$ represents the vector of strategies corresponding to each event. 


Considering $S_i$as the number of occurrences of the $i$-th outcome where $\sum_{j}^{m}S_j = n$, then the wealth at trail $n$ is given by $W_n(\underline{\ell}) = \prod_{i}^{m}\left (1 + \ou_{i}\ell_{i} - \sum_{j}^{m}\ell_{j} \right)^{S_{i}}W_0$. By taking logarithms and considering the limit as $n$ approaches infinity, the log-growth rate is obtained as followed \cite{smoczynski2010explicit}\nospacecolon
\begin{equation*}
    G(\underline{\ell}) = \sum_{i = 1}^{m} p_{i} \log \left(1 + \ou_{i}\ell_{i} - \sum_{j=1}^{m}\ell_{j} \right), \quad \text{with probability 1.}
\end{equation*}

The formulation of the Multivariate Kelly Criterion in a matrix-based representation constitutes an original contribution. By introducing the probability vector $\underline{p} \in [0,1]^{m}$ as the vector of probabilities associated with each event and the consequences matrix $W = \left[\underline{w}_{1}|\underline{w}_{2}|\dotsc|\underline{w}_{m}\right]$, where $\underline{w}_{j} = \ou_{j}\hat{\mathbf{e}}_{j}$ and $\hat{\mathbf{e}}_{j}$ is the $j$-th canonical vector, the problem of determining the Multivariate Kelly Criterion can be cast. The objective is to maximize the expression 

\begin{equation} \label{eq2:multiple_kelly}
	\maxi \left \{\underline{p}' \log\left(\mathbf{1} + W'\underline{\ell} - \sum_{i=1}^{m}\ell_{i} \mathbf{1} \right)\right \} \quad \text{subject to} \quad \sum_{i=1}^{m}\ell_{i} \leq 1, \quad \underline{\ell} \geq \mathbf{0}. 
\end{equation}

Importantly, the optimization problem associated with the Multivariate Kelly Criterion exhibits a concave structure, enabling its solution through convex optimization algorithms like Successive Quadratic Programming (SQP) \cite{busseti2016risk}. While Smoczynski has proposed an algorithm for determining $\underline{\ell}_{*}$ in his work \cite{smoczynski2010explicit}, a closed-form solution is not available. The formulation of this criterion in matrix form significantly contributes to a deeper comprehension of its characteristics. By focusing on the functional form of the log growth and obtaining analytical gradients, we can effectively maximize the criterion function. Furthermore, this matrix-based approach facilitates practical implementation by eliminating the need for iterative loops and relying solely on linear algebraic operations. As a result, this formulation not only enhances theoretical understanding but also provides valuable insights for the efficient application of the Multivariate Kelly Criterion.

\subsubsection{Multivariate and Simultaneous Kelly Criterion}
In the most general case, the multivariate criterion has been extended to incorporate simultaneous random rewards. This extension represents a novel development in the field. In this scenario, a set of $r$ independent random rewards occurs simultaneously, each with $m_{k}$ possible events. The probability space is defined as $\Omega = \bigtimes_{k}^{r}\Omega_{k}$, resulting in a total of $M = \sum_{k}^{r} m_{k}$ events and $N = \prod_{k}^{r} m_{k}$ possibilities. The strategy vector is defined as the concatenation of the betting vectors of the $r$ random rewards, denoted as $\underline{\ell} = (\underline{\ell}_{1}, \underline{\ell}_{2}, \dots, \underline{\ell}_{r})' \in \R^{M}$. Similarly, the decimal odds vector is represented as $\underline{\ou}$ and the net returns vector as $\underline{\varrho}$. The overall raw return of the strategy, denoted as $R(\underline{\ell})$, can be expressed as $R(\underline{\ell}) = 1 + \underline{\ell}'\underline{\varrho}$. The matrix of consequences, denoted as $W \in \R^{M \times N}$, represents the profit possibilities, with each column corresponding to a specific profit outcome $\underline{\omega}_{j_{k}} = \underline{\omega}_{j}$ for each one of the sample spaces $\Omega_{k} \subseteq \Omega$.
\begin{equation*}
    \text{i.e.} \quad \underline{w}_{j} = D_{\ou} (\hat{\mathbf{e}}_{j_{1}}, \hat{\mathbf{e}}_{j_{2}}, \dots, \hat{\mathbf{e}}_{j_{r}})', \quad j_{k} \in \underline{\omega}_{j}, \forall k = 1, 2, \dots, M.
\end{equation*}
Since the outcomes of each bet are assumed to be independent, the probability vector is given by 
\begin{equation} \label{eq2:probas_simult}
    \underline{p}_{i} = \mathbb{P}\left[\underline{\varrho}(\underline{\omega}_{i}) \right] = \mathbb{P}\left[ \bigcap_{k}^{r}\underline{\varrho}_{k}(\omega_{i,k}) \right] = \prod_{k}^{r} \mathbb{P}\left [\underline{\varrho}_{k}(\omega_{i,k}) \right].
\end{equation}
Consequently, the Multivariate and Simultaneous Kelly Criterion is formulated as the optimization problem\nospacecolon
\begin{equation} \label{eq2:multiple_simult_kelly}
	\maxi \left \{\underline{p}' \log\left(\mathbf{1} + W'\underline{\ell} - \sum_{i=1}^{M}\ell_{i} \mathbf{1} \right)\right \} \quad \text{subject to} \quad \sum_{i=1}^{M}\ell_{i} \leq 1, \quad \underline{\ell} \geq \mathbf{0}. 
\end{equation}

The extension of the Multivariate Kelly Criterion to incorporate simultaneous and multiple random rewards presents significant advantages and opportunities for optimizing decision-making in intricate betting scenarios. As previously discussed in the context of the Multiple Kelly Criterion, this expanded framework offers a distinct advantage by enabling analytical and programmatically efficient implementation. By leveraging the functional form of the criterion, decision-makers can employ a numerical approach that is both robust and stable within the realm of matrix algebra. 

The aforementioned properties of the Kelly Criterion highlight its resilience and competitive edge in the pursuit of long-term wealth accumulation. The existence of a distinctive optimal strategy, coupled with the profound influence of strategy selection on capital growth, solidifies the Criterion's efficacy and prominence within the realm of financial decision-making. Building upon these foundational principles, we can now delve into a real-world application that pits the Kelly Criterion against the Sharpe Criterion: betting in the English Premier League. This practical demonstration will further illuminate the practical implications and comparative performance of these two criteria in a tangible and relevant context.

























































