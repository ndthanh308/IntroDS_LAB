% ==================================================
%           Capitulo 3
% ==================================================
\section{Conclusions}

This study aimed to address several key aspects\nospacecolon \ first, identifying the optimal betting strategy for a rational gambler seeking to maximize expected utility, taking into account logarithmic or quadratic utilities. Second, analyzing the characteristics and performance of complete and restricted betting strategies to gain insights into their respective qualities. Third, exploring the qualitative and quantitative differences between portfolios constructed based on the Kelly Criterion and the Sharpe Ratio Criterion in a real-life betting scenario. Fourth, developing statistical learning models to forecast outcomes in the English Premier League. Notably, this study made significant contributions by introducing a matrix-based formulation (\ref{eq2:multiple_kelly}) for the multivariate Kelly Criterion and formulating the optimization problem for the multiple and simultaneous Kelly Criterion (\ref{eq2:multiple_simult_kelly}), enhancing the understanding and applicability of these approaches in practical settings.

The successful demonstration of the first objective of this study involved the development of a systematic method to identify optimal bets within the framework of $(D, E, C, (\succeq))$. The method followed a step-by-step approach\nospacecolon \ first, by defining the set $E$ to determine the elements of $C$, and subsequently constructing the set $D$ from these two sets. Second, the probabilities of the joint events were estimated. Third, the optimal decision $d_{\ell_{*}}$ was determined by maximizing the expected utility of returns, denoted as $\EUT{R(\underline{\ell})}$, while ensuring that $d_{\ell_{*}}$ belongs to the set $D$. Finally, appropriate metrics were defined to assess the performance of the strategies in terms of returns.

Regarding the second aspect investigated, it was observed that although the restricted strategies exhibited better convergence and higher returns compared to the complete strategies, they also displayed higher variance and lower diversification. Notably, in extreme cases, the Sharpe Ratio criterion led to the possibility of the player's ruin, as evidenced by the complete loss of funds two weeks prior to the conclusion of the study. Moreover, it was theoretically established that the advantages of exploiting arbitrage opportunities diminish when stakes are limited. Furthermore, the narrowing of the set of possible bets in restricted strategies resulted in a compromise of the fundamental properties of the criteria. Both the Kelly Criterion and the Sharpe Ratio Criterion exhibit similarities and differences in their portfolio outcomes. Firstly, unlike portfolios based on the Sharpe Ratio, Kelly-based portfolios offer protection against the risk of player ruin. For instance, on average, the full investment strategy and the Kelly-constrained strategy wagered 98\% and 83\% of the total wealth, respectively. Secondly, it is noteworthy that the logarithmic growth achieved through the Kelly Criterion is not invariant to fractional bets, which sets it apart from the Sharpe Ratio approach. Consequently, fractionalizing the Kelly strategy prior to optimization results in a distinct strategy compared to the fractional strategy $f\underline{\ell}$ derived post-optimization.

Thirdly, it is noteworthy that portfolios utilizing logarithmic utilities tend to exhibit relatively lower levels of diversification compared to portfolios employing quadratic utilities, as a proportion of the total stake invested in the portfolio. This observation was addressed by examining the maximum stake per matchweek relative to the total amount wagered in the portfolio, revealing that the Kelly Criterion maximum bet averaged 29.7\% compared to the maximum Sharpe Ratio's bet average of 25.9\%. Consequently, logarithmic portfolios tend to display higher volatility, resulting in more pronounced gains or losses. However, when considering the total budget of \$1, the maximum Kelly Criterion stakes average 1.9\% less than the Sharpe Ratio approach. This discrepancy arises due to the logarithmic nature of the growth function $G(\underline{\ell})$, which tends towards negative infinity as $\sum_{i}\ell_{i}$ approaches 1, as demonstrated by Thorp \cite{thorp1969optimal}.

Fourth, while the constituent elements of the portfolios are similar for both the Kelly Criterion and the Sharpe Ratio Criterion (see Table \ref{tab:table_mtchw_kelly_mktz}), the strategies themselves are not proportionally aligned between the two methods. This divergence is evident even in extreme cases such as arbitrage, where notable differences are observed in the resulting portfolios (see Table \ref{tab:table_arbitrage}).

In regards to the deep learning model, several considerations need to be addressed. Despite the advantages in terms of predictive power, flexibility, and relaxed assumptions offered by neural networks in modeling sporting events, it is important to acknowledge that the present study faced limitations due to the low volume of available data. Nevertheless, employing a deep learning approach still represents an improvement over traditional multinomial regression for sports modeling. Furthermore, the selection of variables used in the model remains an area of ongoing investigation. It is evident that there is a deficiency in the number of variables available to accurately determine match outcomes, particularly at the individual player level. Despite this limitation, Zimmerman suggests that there exists an empirical benchmark of 75\% in terms of predictive accuracy for sporting events \cite{zimmermann2013predicting}. Moreover, Hub{\'a}{\v{c}}ek emphasizes that the selection of variables holds greater significance than the specific statistical model employed used \cite{hubavcek2019exploiting}. These perspectives underscore the importance of further refining the variable selection process to enhance the predictive capabilities of the model.

% Figure environment removed

Finally, it is imperative to validate the underlying assumptions of the models. The cornerstone of the developed method relies on the premise that the actual probabilities of events are known. However, upon examining the predictions' variability in Table \ref{tab:table_models_metrics} and the classification errors illustrated in Table \ref{tab:mat_conf}, doubts arise regarding the veracity of this assumption. Notably, the model's a priori cross entropy was anticipated to be 1.0318, whereas the empirical entropy based on frequencies amounts to 1.0631. Furthermore, the maximum entropy for a three-event phenomenon is calculated to be 1.0986. Consequently, the certainty surrounding this assumption is called into question.

Furthermore, it is crucial to evaluate whether the model's performance of 35.8\% can be attributed to "luck" or "skill." To address this, the model is subjected to a test against 500 Dirichlet simulations with unit parameters, conducted within the same test period but without restrictions and employing strategies bounded to the same fraction $f$ as the model in question. The results reveal that, on average, the model outperforms 78 out of 100 simulations. This serves as compelling evidence that the model's performance extends beyond mere chance. Nevertheless, despite investigating the relationship between log-growth and performance for the simulations, no linear evidence supporting such a connection is found (p-value of 0.595).

% % Figure environment removed

\subsection{Future Research}
The findings and hypotheses explored in this study open up various avenues for future research. These avenues span both financial methodologies and predictive modeling in the context of sports betting.

From a financial perspective, it would be of great theoretical and empirical interest to examine the long-term behavior of returns when encountering arbitrage opportunities. Exploring regularization techniques, particularly in the context of $L_{2}$ regularization for betting portfolios, could also yield valuable insights. Additionally, investigating the divergences in portfolio composition and returns between restricted and full strategies, considering both quadratic and logarithmic utilities, through simulation studies would provide further understanding.

In the realm of predictive modeling, there are multiple aspects worth exploring. One avenue is adopting a Bayesian perspective to analyze the prediction problem, complementing the frequentist approach employed in this research. Furthermore, incorporating player-level data in addition to aggregate team data could enhance the predictive model's accuracy and granularity. The inclusion of data from other soccer leagues could also contribute to a more comprehensive and robust modeling approach.

It is important to acknowledge that the work presented in this study is grounded in the Von Neumann and Morgenstein's Axioms of Preference. However, it is well-known that these axioms may not always hold in reality. Exploring the disparities between theoretical preferences and empirical observations, given the same available information, would shed light on the limitations and challenges associated with relying solely on axiomatic models. Furthermore, investigating optimal policies under Reinforcement Learning models could provide valuable insights into the dynamic decision-making processes within the realm of sports betting.

Overall, the future research directions outlined above have the potential to further advance the understanding of financial strategies, predictive modeling approaches, and the complex dynamics of decision-making in sports betting beyond this paper.