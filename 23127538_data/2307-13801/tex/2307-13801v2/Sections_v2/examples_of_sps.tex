In this section, we consider two classes of examples of practical relevance in quantum information processing for which \Cref{assum:finite-degree} (or \ref{assum:finite-degree-time-dep},\ref{assum:multimode-finite-degree},\ref{assum:multimode-finite-degree-time-dep}) trivially holds and derive \Cref{assum:sobolev-stability} (or \ref{assum:sobolev-stability-time-dep}, \ref{assum:multimode-sobolev-preserving}, \ref{assum:multimode-sobolev-stability-time-dep}). Particular care will be given to finding time-independent upper bounds on the $W^{k,1}\to W^{k,1}$ norm of the semigroup. For this, the overall strategy is as follows: given the generator $(\cL,\cT_f)$, we prove that there are coefficients $\mu_{k_r}\ge 0, c_{k_r}>0$ for a divergent sequence $\{k_r\}_{r \in \N}$ such that for all state $\rho\in\cT_f$
\begin{equation}\label{eq:examples-assum2-step1}
    \begin{aligned}
        \tr[\cL(\rho) (\Nind+\1)^{k_r/2}] &\leq - c_{k_r}\tr[\rho (\Nind+\1)^{k_r/2}] + \mu_{k_r} \\
        & \leq (\mu_{k_r}- c_{k_r})\tr[\rho (\Nind+\1)^{k_r/2}]\,,
    \end{aligned}
\end{equation}
where we have used $\tr[\rho (\Nind+\1)^{k_r/2}]\geq \tr[\rho] = 1$ in the second inequality.
Then, \Cref{thm:generation-theorem} can be applied, which shows that for all $k \in \R_+$, the closure of $(\cL,\cT_f)$ generates a positivity preserving $C_0$-semigroup $(\cP_t)_{t\ge 0}$ on $W^{k,1}$. In the case $k \in \{k_r\}_{r \in \N}$:
\begin{equation*}
    \|\cP_t(x)\|_{W^{k,1}}\leq e^{|\mu_k- c_k|\,t}\|x\|_{W^{k,1}}\,.
\end{equation*}
for all $x\in W^{k,1}$. The bounds for the intermediate values of $k$ can be obtained using \Cref{lem:interpolation-lemma}. One can strengthen the above bounds using \Cref{eq:examples-assum2-step1} as follows:

\begin{prop}\label{prop-ex:uniformly-bounded-semigroup}
    Let $(\cL,\cT_f)$ be an operator satisfying \Cref{assum:finite-degree} and \Cref{eq:examples-assum2-step1}. Then, for all $k\in\N$, the closure of $(\cL,\cT_f)$ generates a positivity preserving $C_0$-semigroup $(\cP_t)_{t\ge 0}$ on $W^{k,1}$. For all $r \in \N$ and all states $\rho\in W^{k,1}$,
    \begin{equation*}
        \|\cP_t(\rho)\|_{W^{k_r,1}}\leq \max\left\{\|\rho\|_{W^{k_r,1}},\,\frac{\mu_{k_r}}{c_{k_r}}\right\} \, .
    \end{equation*}
    For a general $k \in \R_+$ and $x \in W^{k, 1}$ one obtains
    \begin{equation}\label{eq:improved-semigroup-bound}
         \|\cP_t(x)\|_{W^{k_r,1}}\leq \gamma_k \|x\|_{W^{k_r,1}} \, ,
    \end{equation}
    where $\gamma_k = \max\{1, \frac{\mu_k}{c_k}\}$ for $k \in \{k_r\}_{r \in \N}$ and an interpolated time-independent constant in all other cases. Note that for $k > 0$ and $\rho \in W^{k, 1}$ there exists a sequence $\{t_n\}_{n \in \N}$, such that
    \begin{equation*}
        \lim_{t_n\rightarrow\infty}\cP_{t_n}(\rho)=\overline{\rho}
    \end{equation*}
    for $\overline{\rho}\in W^{k,1}$. Similar conclusions hold in multi-mode as well as time-dependent settings.
\end{prop}
\begin{proof}
    By assumption, \Cref{thm:generation-theorem} shows that the closure of $(\cL,\cT_f)$ defines a positivity preserving, quasi-contractive semigroup $(\cP_t)_{t\ge 0}$. Moreover, for $k \in \{k_r\}_{r \in \N}$, $\rho(t)\coloneqq \cP_t(\rho)$
    \begin{equation*}
        \begin{aligned}
            \frac{d}{dt}\|\rho(t)\|_{W^{k,1}}&=\tr[\cL(\rho(t))(N+\1)^{k/2}]\\
            &\leq-c_{k}\tr[\rho(t) (\Nind+\1)^{k/2}] + \mu_{k}\\
            &=-c_{k}\|\rho\|_{W^{k,1}} + \mu_{k}\,.
        \end{aligned}
    \end{equation*}
    Thus, for $\|\rho(t)\|_{W^{k,1}}\geq\frac{\mu_k}{c_k}$, we have $\frac{d}{dt}\|\rho(t)\|_{W^{k,1}}\leq0$, which concludes the bound. Using the positivity preserving property of the semigroup and that $\norm{\cdot}_1 \le \norm{\cdot}_{W^{k, 1}}$ one can lift the bound to \Cref{eq:improved-semigroup-bound} for general $x \in W^{k, 1}$ and \Cref{thm:stein-weiss} allows us to conclude 
    \begin{equation*}
         \|\cP_t(x)\|_{W^{k_r,1}}\leq \gamma_k \|x\|_{W^{k_r,1}}
    \end{equation*}
    extend to all $k \in \R_+$. Finally, for every $k > 0$, every sequence $n\rightarrow \cP_{t_n}(\rho)$ is uniformly bounded in $W^{k,1}$ so that the compact embedding shows that there exists a converging subsequence in $W^{k-\varepsilon,1}$ for $\varepsilon$ suitably chosen, which is also converging in $W^{k,1}$. This finishes the proof.
\end{proof}
To achieve the inequality stated in \Cref{eq:examples-assum2-step1}, we will make heavy use of the following simple commutation relations: given a real-valued function $f:\mathbb{N}\to\mathbb{R}$,  
\begin{equation}\label{eq:symmetry-function}
    \begin{aligned}
        af(\Nind+j\1)=f(\Nind + (j+1)\1)a,&\quad\quad a^\dagger\,1_{>j}f(\Nind-j\1)=f(\Nind - (j+1)\1)a^\dagger\,1_{>j}\,,\\
        f(\Nind-j\1)a\,1_{>j}=af(\Nind - (j+1)\1)1_{>j},&\quad\quad f(\Nind+j\1)a^\dagger=a^\dagger f(\Nind + (j+1)\1)\,,
    \end{aligned}
\end{equation}
where the operators above are defined e.g.~on $\cH_f$. We also use the canonical commutation relation to write $(\ad)^la^l$ as a function of $N$ (see \Cref{lem:l-ccr}): 
    \begin{align*}
            &(\ad)^la^l=(N-(l-1)\1)(N-(l-2)\1)\cdots(N-\1)N\\
            & a^l(\ad)^l=(N+\1)(N+2\1)\cdots(N+(l-1)\1)(N+l\1)\,.
    \end{align*}
In the following, we adopt the notations:
\begin{align*}
    \cL[L]\coloneqq L(\cdot)L^\dagger -\frac{1}{2}\,\{L^\dagger L,\,\cdot\}\,\qquad \text{ and }\qquad \cH[H]\coloneqq -i[H,\cdot]\,.
\end{align*}
Although this notation collides with the one for the Hilbert space, the meaning can always be deduced from context.

\subsection{Quantum Ornstein Uhlenbeck semigroup}

We start with the generator of the quantum Ornstein Uhlenbeck semigroup \cite{Cipriani.2000,Carbone.2007} defined by
\begin{equation}
    \cL_{\operatorname{qOU}} = \lambda^2 \cL[a] + \mu^2 \cL[\ad]
\end{equation}
 for $\mu, \lambda \ge0$. Given an suitably domain $\cD(\cL_{\operatorname{qOU}})$, the operator $(\overline{\cL}_{\operatorname{qOU}},\cD(\cL_{\operatorname{qOU}}))$
is known to generate a quantum dynamical semigroup $(\cP_t^{\operatorname{qOU}})_{t\ge 0}$. Here, we further show that the quantum Ornstein Uhlenbeck semigroup defines a semigroup on all $W^{k,1}$. This is the topic of the following lemma: 

\begin{lem}\label{lem-ex:qOU-differential-stability}
    Let $(\cL_{\operatorname{qOU}},\cT_f)$ be the generator of the quantum Ornstein Uhlenbeck semigroup and $k \in \N$. Then, there exist constants $\mu_k$ explicated in \eqref{eq-ex:qou-lambda>mu} such that, for all states $\rho\in\cT_f$,
    \begin{equation*}
        \begin{aligned}
            \tr[\cL_{\operatorname{qOU}}(\rho)(N+\1)^{\frac{k}{2}}]\le\begin{cases}
                \frac{k}{4}(\mu^2-\lambda^2)\tr\big[\rho(N+\1)^{k/2}\big]+\mu_k&\lambda>\mu\\
                \frac{k}{2}(2\mu^2+k)\tr\big[\rho(N+\1)^{k/2}\big]&\lambda\leq\mu
            \end{cases}
        \end{aligned}\,.
    \end{equation*}
   Therefore, the semigroup $e^{t\mathcal{L}_{\operatorname{qOU}}}$ is a Sobolev and positivity preserving quantum Markov semigroup satisfying for all states $\rho\in W^{k,1}$
    \begin{equation*}
        \begin{aligned}
            \|e^{t\mathcal{L}_{\operatorname{qOU}}}(\rho)\|_{W^{k,1}}\leq\begin{cases}
                \max\left\{\|\rho\|_{W^{k,1}},\,\frac{4\mu_{k}}{k(\mu^2-\lambda^2)}\right\}&\lambda>\mu\\
                e^{t\frac{k}{2}(2\mu^2+k)}\|\rho\|_{W^{k,1}}&\lambda\leq\mu
            \end{cases}\,.
        \end{aligned}
    \end{equation*}
\end{lem}
\begin{proof}
    We consider $\cL_{\operatorname{qOU}}^\dagger(f(N))$ where $f(x)=(x+1)^{k/2} 1_{x\ge -1}$. By \Cref{eq:symmetry-function},
    \begin{equation*}
        \begin{aligned}
            \cL_{\operatorname{qOU}}^\dagger(f(N))&=\lambda^2N(f(N-\1)-f(N))+\mu^2(N+\1)(f(N+\1)-f(N))\,.
        \end{aligned}
    \end{equation*}
    Note that the case $k=0$ follows from the GKLS form and $k=2$ is by definition of $f$ trivially given by $(\mu^2-\lambda^2)N+\1$. 
    Next, we define an auxiliary function which will also prove useful in the following proofs: 
    \begin{equation}\label{eq:f-g-l-function}
        g_l(x) = \begin{cases}
            f(x) - f(x - l) & x \ge l;\\
            f(x) & l > x \ge 0;\\
            0 & 0 > x\,.
        \end{cases}
    \end{equation}
    It allows us to redefine $\cL_{\operatorname{qOU}}^\dagger(f(N))$ by 
    \begin{equation*}
        \begin{aligned}
            \cL_{\operatorname{qOU}}^\dagger(f(N))&=-\lambda^2Ng_1(N)+\mu^2(N+\1)g_1(N+\1)\,.
        \end{aligned}
    \end{equation*}
    Then, applying \Cref{lem:upper-lower-bound-gl} to the spectral decomposition of the polynomial in the number operator above, we get 
    \begin{equation*}
        \begin{aligned}
            &\cL_{\operatorname{qOU}}^\dagger(f(N))\\
            &\quad\leq \frac{k}{2}(\mu^2-\lambda^2)(N+\1)^{k/2}+\lambda^2\frac{k}{2}(N+\1)^{k/2-1}+1_{k\geq3}(N+\1)^{k/2-2}\frac{k^2}{8}+\mu^2\frac{2-k}{2}\ketbra{0}{0}\\
            &\quad\leq \frac{k}{2}(\mu^2-\lambda^2)(N+\1)^{k/2}+\frac{k}{2}\left(\lambda^2+\mu^2+{k}\right)(N+\1)^{k/2-1}
        \end{aligned}
    \end{equation*}
    where we separated the vacuum state from the rest of the decomposition. Note that this bound can also be used when $k=1$ since $(N+\1)^{-1/2}$ is then bounded by $1$. Therefore, we assume $k\geq3$ in the following and start with the case $\lambda>\mu$ so that the leading order is negative. Then, we use half of the latter to bound the other terms by a constant. This is done by the following classical optimization
    \begin{equation*}
        \sup_{x\geq0}\left(-x^\nu+cx^{\nu-1}\right)=c^\nu\left(\frac{(\nu-1)^{\nu-1}}{\nu^\nu}\right)
    \end{equation*}
    for $\nu\geq1$ and $c\geq0$ defined as 
    \begin{equation*}
        c=2\frac{\lambda^2+\mu^2+k}{\lambda^2-\mu^2}\qquad\text{and}\qquad\nu=\frac{k}{2}\,.
    \end{equation*}
    Then,
    \begin{equation}\label{eq-ex:qou-lambda>mu}
        \begin{aligned}
            \cL_{\operatorname{qOU}}^\dagger(f(N))&\leq \frac{k}{4}(\mu^2-\lambda^2)(N+\1)^{k/2}+c^\nu\left(\frac{(\nu-1)^{\nu-1}}{\nu^\nu}\right)=:\frac{k}{4}(\mu^2-\lambda^2)(N+\1)^{k/2}+\mu_k^{\lambda>\mu}
        \end{aligned}
    \end{equation}
    The second case is $\lambda\leq\mu$, which can be easily upper bounded by
    \begin{equation*}
        \begin{aligned}
            &\cL_{\operatorname{qOU}}^\dagger(f(N))\leq \frac{k}{2}(2\mu^2+k)(N+\1)^{k/2}
        \end{aligned}\,.
    \end{equation*}
    This completes the proof of the statement by \Cref{thm:generation-theorem} and \Cref{prop-ex:uniformly-bounded-semigroup}.
\end{proof}

\subsection{Photon-dissipation and CAT qubits}\label{sec:cat-qubits}

Next, we consider a family of Lindbladians that has been recently studied in the setting of error correction with continuous variable quantum systems. For an introduction to the field, we refer the interested reader to the following lecture notes \cites{Preskill.2021}{Guillaud.2023}. The abstract idea here is that the code-space is continuously protected by a dissipative evolution, i.e.~an evolution which is exponentially converging for $t\rightarrow\infty$ to an invariant subspace --- the code-space. This behavior is achieved through the so-called $l$-photon dissipation generated for $\kappa>0$ and $\alpha\in\C$ by
\begin{equation}\label{eq:l-photon-dissipation}
    \kappa\cL[a^l-\alpha^l]\,,
\end{equation}
where we sometimes omit the identity so that $\alpha^l\coloneqq \alpha^l\1$ in what follows. The invariant subspace (code-space) to which the evolution is exponentially converging \cite{Azouit.2016} is defined by 
\begin{equation*}
    \cC_l\coloneqq\spa\left\{\ketbra{\alpha_1}{\alpha_2}\,:\,\alpha_1,\alpha_2\in\left\{\alpha e^{\frac{i2\pi j}{l}}\,|\,j\in\{0,...,l-1\}\right\}\right\}\,,
\end{equation*}
where $\ket{\alpha}$ denotes the coherent state
\begin{equation*}
    \ket{\alpha}=e^{-\frac{|\alpha|^2}{2}}\sum_{n=0}^{\infty}\frac{\alpha^n}{\sqrt{n!}}\ket{n}\,.
\end{equation*}
and satisfies $a\ket{\alpha}=\alpha\ket{\alpha}$ by definition.

Besides the $l$-photon dissipation, we consider the CAT qubit error correction protocol introduced in \cite{Guillaud.2019} associated to the $2$-photon dissipation and code-space $\cC_2$ and with corresponding universal gate-set generated by the following: for some parameters $T,\kappa,\varepsilon>0$,

\medskip
\medskip 

\textit{Identity-gate:}
\begin{equation}\label{eq:cat-identity}
    \kappa\cL[a^2-\alpha^2]
\end{equation}

\textit{$Z(\theta)$-gate:}
\begin{equation}\label{eq:cat-z}
    \kappa\cL[a^2-\alpha^2]+\varepsilon\cH[a+\ad]
\end{equation}

\textit{$X$-gate:}
\begin{equation}\label{eq:cat-X}
    \kappa\cL[a^2-e^{2i\pi t/T}\alpha^2]
\end{equation}

\textit{$\operatorname{CNOT}$-gate:}
\begin{equation}\label{eq:cat-cnot}
    \kappa\cL[a^2-\alpha^2]+\kappa\cL[b^2-\alpha^2-\frac{\alpha}{2}(1-e^{2i\pi t/T})(a-\alpha)]
\end{equation}

\textit{Toffoli-gate:}
\begin{equation}\label{eq:cat-toffoli}
    \kappa\cL[a^2-\alpha^2]+\kappa\cL[b^2-\alpha^2]+\kappa\cL[c^2-\alpha^2+\frac{1}{4}(1-e^{2i\pi t/T})(ab-\alpha(a+b)+\alpha^2)]\,.
\end{equation}

Note that the CNOT gate acts on two modes and the Toffoli on three modes, where the annihilation and creation operators on the second mode are denoted by $b$, resp.~$b^\dagger$ and on the third by $c$ and $c^\dagger$. In the following, we prove that the above operators generate Sobolev-preserving quantum dynamical semigroups, with the exception of the Toffoli gate. Due to its more complicated structure, we leave the analysis of the latter to future work.

We start by proving that the $l$-photon dissipation satisfies \Cref{eq:examples-assum2-step1}, and therefore that it generates a Sobolev preserving semigroup by \Cref{prop-ex:uniformly-bounded-semigroup}.

\begin{lem}[$l$-photon dissipation]\label{lem:l-diss}
    For any $k\ge 1$, $l\ge 2$, $\alpha \in\mathbb{C}$ and any state $\rho\in\cT_f$, 
    \begin{align*}
        \tr\big[\cL[a^l-\alpha^l ](\rho)(N+\1)^{k/2}\big]&\le -\frac{l}{2} \tr\big[\rho\,(N+\1)^{\nu}\big]+\frac{l}{2}\mu_k^{(l)}\le -\frac{l}{2} \tr\big[\rho\,(N+\1)^{k/2}\big]+\frac{l}{2}\mu_k^{(l)}\,,
    \end{align*}
    where $\mu_k^{(l)}=\Delta_l^\nu\left(\frac{(\nu-1)^{\nu-1}}{\nu^\nu}\right)$ with $\nu=l+\frac{k}{2}-1$ and $\Delta_l=(l+1)l+2|\alpha|^lkl^{k/2 - 1}\sqrt{l!}$. Therefore, $\cL_l\coloneqq \cL[a^l-\alpha^l]$ generates a Sobolev and positivity preserving quantum Markov semigroup satisfying for all states $\rho\in W^{k,1}$
    \begin{equation*}
        \|e^{t\cL_l}(\rho)\|_{W^{k,1}}\leq\max\Big\{\|\rho\|_{W^{k,1}}, \mu_k^{(l)}\Big\}\,.
    \end{equation*}
\end{lem}
\begin{proof}
    By \Cref{eq:symmetry-function}, we have for $f(x)=(x+1)^{k/2} 1_{x\ge -1}$:
    \begin{equation*}
        \begin{aligned}
            \cL[a_1^l-\alpha^l]^\dagger(f(N))&=(\ad)^lf(N)a^l-\frac{1}{2}\Big((\ad)^la^lf(N)+f(N)(\ad)^la^l\Big)\\
            &\quad+\frac{1}{2}(\overline{\alpha}^la^lf(N)-\overline{\alpha}^lf(N)a^l+\alpha^l f(N)(\ad)^l-\alpha^l(\ad)^lf(N))\\
            &=(\ad)^la^l\Big(f(N-l\1 )-f(N)\Big)\\
            &\quad+\frac{1}{2}\left[\overline{\alpha}^la^l\Big(f(N)-f(N-l\1)\Big)+\alpha^l\Big(f(N)-f(N-l\1)\Big)(\ad)^l\right]\,.
        \end{aligned}
    \end{equation*}
    In what follows, we use the function defined in \Cref{eq-appx:f-g-l-function}
    \begin{equation*}
        g_l(x) = \begin{cases}
            f(x) - f(x - l) & x \ge l;\\
            f(x) & l > x \ge 0;\\
            0 & 0 > x\,.
        \end{cases}
    \end{equation*}  
    Using the canonical commutation relation to write $(a^\dagger)^la^l$ as a function of $N$ (cf.~\Cref{lem:l-ccr}) and with help of the notation
    \begin{equation}\label{eq:notation-product}
        N_k[r:j]\coloneqq (N_k+r\1)\cdots (N_k+j\1)
    \end{equation}
    with the convention $N_k[r:j]=\1$ whenever $r>j$, we thus have that
    \begin{align*}
        \tr\big[\rho\,\cL[a^l-\alpha^l]^\dagger(f(N))\big]&=-\tr\big[\rho\,N[-l+1:0]g_l(N)\big]+\frac{1}{2}\tr\big[\rho\,(\overline{\alpha}^la^lg_l(N)+\alpha^lg_l(N)(\ad)^l)\big]\,,
    \end{align*}
    Since $g_l$ is positive and increasing, the last term above can be upper bounded by \Cref{lem:two-point-hamiltonian-bound},
    \begin{equation*}
        \begin{aligned}
            \frac{1}{2}\tr\big[\rho(\overline{\alpha}^la^lg_l(N)+{\alpha}^lg_l(N)(\ad)^l)\big]&{\leq}\,{|\alpha|^l}\tr[\rho\, g_l(N+l\1)\sqrt{N[1:l]}] \\
            &\overset{(1)}{\leq}|\alpha|^l \frac{kl^{k/2}}{2}\,\tr[\rho\, (N+\1)^{k/2-1}\sqrt{N[1:l]}]\\
            &\leq|\alpha|^lkl^{k/2}\sqrt{l!}\,\tr[\rho\, (N+\1 )^{k/2-1+\frac{l}{2}}]\,.
        \end{aligned}
    \end{equation*}
    In $(1)$ above, we used \Cref{lem:upper-lower-bound-gl} for the bound
    \begin{align*}
        g_l(N+l\1) \le  \frac{kl}{2}\,(N+l\1)^{k/2-1}\leq\frac{kl^{k/2}}{2}\,(N+\1)^{k/2-1}\,.
    \end{align*}
    Therefore, we have proven that 
    \begin{equation*}
        \begin{aligned}
            \tr\big[\rho\cL[a^l-\alpha^l]^\dagger(f(N))\big]&\le -\tr\big[\rho\,N[-l+1:0]\,g_l(N)\big]\\
            &\qquad+|\alpha|^lkl^{k/2}\sqrt{l!}\,\tr[\rho\, (N+\1 )^{k/2-1+\frac{l}{2}}]\,.
        \end{aligned}
    \end{equation*}
    %\begin{equation*}
    %    \begin{aligned}
    %        \tr\big[\rho\cL[a^l-\alpha^l]^\dagger(f(N))\big]&\le -\tr\big[\rho\,N[-l+1:0]\,g_l(N)\big]\\
    %        &\qquad+|\alpha|^lkl^{k/2+1}\sqrt{l!}\,\tr[\rho\, (N+\1 )^{k/2-1+\frac{l}{2}}]\,.
    %    \end{aligned}
    %\end{equation*}
    Next, we upper bound the first term above 
    \begin{equation*}
        \begin{aligned}
            \tr\big[\rho\,N[-l+1:0]\,g_l(N)\big]&\overset{(3)}{\geq}l\tr\big[\rho\,N[-l+1:0](N+\1)^{k/2-1}\big]\\
            &\overset{(4)}{\geq}l\,\tr\big[\rho\,(N+\1)^{l+k/2-1}\,\big]-\frac{(l+1)l^2}{2}\tr\big[\rho\,(N+\1 )^{l+k/2-2}\big]\,.
        \end{aligned}
    \end{equation*}
    In $(3)$, we used  \Cref{lem:upper-lower-bound-gl} below with the fact that $N[-l+1:0]$ is supported on the Fock states $|n\rangle$ with $n\ge l-1$;
    in $(4)$ we used that 
    \begin{align*}
        N[-l+1:0]&=\sum_{n\ge 0}\,(n-l+1)\dots n \,\ketbra{n}{n}\\
        &=\sum_{n\ge l}\,(n-l+1)\dots n\,\ketbra{n}{n}\\
        &\overset{(5)}{\ge} \sum_{n\ge l}\left((n+1)^l-\frac{(l+1)l}{2}(n+1)^{l-1}\right)\ketbra{n}{n}\\
        &\ge (N+\1)^l-\frac{(l+1)l}{2}\,(N+\1)^{l-1}\,,
    \end{align*}
    where $(5)$ comes from \Cref{lem:bounds-ccr-l-product} below, whereas the last inequality follows from the fact that $l\ge 2$. To sum up, we showed that
    \begin{equation}\label{eq-ex:l-dissipation-upper-bound}
        \begin{aligned}
            \tr\big[\cL[a^l-\alpha^l](\rho)(f(N))\big]&\le -l\tr\big[\rho\,(N+\1)^{l+k/2-1}\,\big]+\frac{(l+1)l^2}{2}\tr\big[\rho\,(N+\1 )^{l+k/2-2}\big]\\
           &\qquad+|\alpha|^lkl^{k/2}\sqrt{l!}\,\tr[\rho\, (N+\1 )^{k/2-1+\frac{l}{2}}]\\
           &\le -l\tr\big[\rho\,(N+\1)^{l+k/2-1}\,\big]\\
           &\qquad+\frac{l}{2}\biggl(\underbrace{{(l+1)l}+2|\alpha|^lkl^{k/2 - 1}\sqrt{l!}}_{\eqqcolon \Delta_l}\biggr)\,\tr\big[\rho\,(N+\1 )^{l+k/2-2}\big]
        \end{aligned}
    \end{equation}
    %\begin{equation}\label{eq-ex:l-dissipation-upper-bound}
    %    \begin{aligned}
    %        \tr\big[\cL[a^l-\alpha^l](\rho)(f(N))\big]&\le -l\tr\big[\rho\,(N+\1)^{l+k/2-1}\,\big]+\frac{(l+1)l^2}{2}\tr\big[\rho\,(N+\1 )^{l+k/2-2}\big]\\
    %       &\qquad+|\alpha|^lkl^{k/2+1}\sqrt{l!}\,\tr[\rho\, (N+\1 )^{k/2-1+\frac{l}{2}}]\\
    %       &\le -l\tr\big[\rho\,(N+\1)^{l+k/2-1}\,\big]\\
    %       &\qquad+\frac{l}{2}\biggl(\underbrace{{(l+1)l}+2|\alpha|^lkl^{k/2}\sqrt{l!}}_{\eqqcolon \Delta_l}\biggr)\,\tr\big[\rho\,(N+\1 )^{l+k/2-2}\big]
    %    \end{aligned}
    %\end{equation}
    where we used again that $l\ge 2$ in the last inequality. Half of the leading order term can be used to control the second term by a constant. For that, we use the spectral decomposition of the operator $N$ so that the above problem can be reduced to the following simple optimization: 
    \begin{equation}\label{eq:optimization}
        \sup_{x\geq0}\left(-x^\nu+\Delta_lx^{\nu-1}\right)=\Delta_l^\nu\left(\frac{(\nu-1)^{\nu-1}}{\nu^\nu}\right)
    \end{equation}
    for $\nu\geq1$ defined as 
    \begin{equation*}
        \nu=l+\frac{k}{2}-1\,.
    \end{equation*}
    The result follows after invoking \Cref{prop-ex:uniformly-bounded-semigroup}. 
\end{proof}
\begin{rmk}\label{rmk:l-diss-multimode}
    The single-mode bound proved above can be generalized to the multimode setting, with generated given for some $\alpha_j\in\mathbb{C}$, $j\in[m]$, by 
    \begin{equation*}
        \cL_l^{(m)}\coloneqq\sum_{j=1}^m\cL[a_j^l-\alpha_j^l]\,.
    \end{equation*}
    Since all the bounds used in the proof of \Cref{lem:l-diss} were derived at the operator level, we directly get for $\k \in \N^m$
    \begin{align*}
        \tr\big[\cL_l^{(m)}(\rho)(N+\1)^{\k/2}\big]&\le \sum_{i=1}^m-\frac{l}{2}\tr\big[\rho\,(N_i+\1)^{l-1}(N + \1)^{\k/2}\big]+\mu^{(l)}_{k_i}\tr\big[\rho\,\prod_{j\neq i}(N_j+\1)^{k_j/2}\big]
    \end{align*}
\end{rmk}
For later references, we single out the case $l=2$.
\begin{cor}[$2$-photon dissipation]\label{lem:cat-identity}
    For any integers $k\ge 1$, $\alpha\in\mathbb{C}$ and any state $\rho\in\cT_f$, 
    \begin{align*}
        	\tr\big[\cL[a^2-\alpha^2](\rho)(N+\1)^{k/2}\big]&\le -\tr\big[\rho\,(N+\1)^{k/2}\big]+\mu^{(2)}_k
    \end{align*}
    where $\mu^{(2)}_k=(\Delta^{(2)}_k)^\nu\left(\frac{(\nu-1)^{\nu-1}}{\nu^\nu}\right)$ with $\nu=\frac{k}{2}+1$ and $\Delta_2=6+2\sqrt{2!}|\alpha|^2k2^{k/2 - 1}$. Therefore, $\cL_2\coloneqq \cL[a^2-\alpha^2 ]$ generates a Sobolev and positivity preserving quantum Markov semigroup which satisfies for all states $\rho$ in $W^{k,1}$
    \begin{equation*}
        \|e^{t\cL_2}(\rho)\|_{W^{k,1}}\leq\max\left\{\|\rho\|_{W^{k,1}},\mu^{(2)}_k\right\}\,.
    \end{equation*}
\end{cor}
From the above bounds, we directly get the property of Sobolev preservation for the $X$-gate:
\begin{cor}[$X$-gate]
    For any $T>0$, $\alpha\in\mathbb{C}$, $k\in\N$ and all states $\rho\in\cT_f$,
    \begin{equation*}
        \tr[\cL[a^2-e^{2i\pi t/T}\alpha^2](\rho)(N+\1)^{k/2}]\leq-\tr[\rho(N+\1)^{k/2}]+\mu_k^{(2)}\,,
    \end{equation*}
    where $\mu_k^{(2)}$ is defined in \Cref{lem:cat-identity}. Therefore, $\cL[a^2-e^{2i\pi t/T}\alpha^2]$ generates a Sobolev and positivity preserving quantum evolution system $\cP_{t,t_0}$ which satisfies for all states $\rho\in W^{k,1}$
    \begin{equation*}
        \|\cP_{t,t_0}(\rho)\|_{W^{k,1}}\leq\max\left\{\|\rho\|_{W^{k,1}},\mu_k^{(2)}\right\}\,.
    \end{equation*}
\end{cor}
\begin{proof}
  The statement directly follows from \Cref{lem:cat-identity} and $|e^{2i\pi t/T}\alpha^2|=|\alpha^2|$
\end{proof}

Next, we consider a Hamiltonian of degree $d_H = 2(l-1)$ and show that together with the $l$-photon dissipation the sum $\cL[a_l-\alpha^l]+\cH[H]$ satisfies \Cref{assum:sobolev-stability}. We assume that $H$ has the following polynomial representation: for $\lambda_{i,j}\in\mathbb{C}$ with $\max_{i,j}|\lambda_{i,j}|=\Lambda$,
\begin{equation}\label{Hpolyrep}
    H=p(a,\ad)=\sum_{\substack{i \le j\\i+j \le d_H}}\lambda_{i,j}a^i(\ad)^j + \overline{\lambda_{i,j}} a^j (\ad)^i \, . 
\end{equation}
 Note that any monomial in $a,\ad$ of degree at most $d_H$ can be achieved from the representation above thanks to the CCR.
 
\begin{lem}\label{lem:l-diss-hamiltonian}
    Let $\cL_l\coloneqq \cL[a^l-\alpha^l]$, $\alpha\in\mathbb{C}$, be the $l$-photon dissipation and $H$ as in \eqref{Hpolyrep}. Then, for all states $\rho\in\cT_f$
    \begin{equation}
        \begin{aligned}
            \tr[(\cL_l+\cH[H])(\rho)(N+\1)^{k/2}] &\leq-\frac{l}{2}\,\tr[\rho(N+\1)^{k/2}]+\frac{l}{2}\mu_k\,.\label{eqdiffLH}
        \end{aligned}
    \end{equation}
    for $\mu_k,\nu\geq1$ defined by 
    \begin{equation*}
        \mu_k=c^\nu\left(\frac{(\nu-1)^{\nu-1}}{\nu^\nu}\right)\quad\text{with}\quad c={(l+1)l}+2|\alpha|^lkl^{k/2-1}\sqrt{l!}+\Lambda(2l)^{k/2}\sqrt{(2l)!}\,,\quad\nu=l+\frac{k}{2}-1\,.
    \end{equation*}
    Therefore, $\cL_l+\cH[H]$ generates a Sobolev and positivity preserving quantum Markov semigroup which satisfies for all states $\rho\in W^{k,1}$
    \begin{equation}\label{eqintegratedLH}
        \|e^{t(\cL_l+\cH[H])}(\rho)\|_{W^{k,1}}\leq\max\Big\{\|\rho\|_{W^{k,1}},\mu_k\Big\}\,.
    \end{equation}
\end{lem}
\begin{proof}
    We reuse the bound given in \Cref{eq-ex:l-dissipation-upper-bound}:
    \begin{equation*}
        \begin{aligned}
            \tr[\rho\cL_l^\dagger(f(N))]\leq-l\tr\big[\rho\,(N+\1)^{l+k/2-1}\,\big]+\frac{l}{2}\Delta_l\,\tr\big[\rho\,(N+\1 )^{l+k/2-2}g_l(N)\big]\,,
        \end{aligned}
    \end{equation*}
    where $f(x)=(x+1)^{k/2} 1_{x\ge -1}$ and $\Delta_l={(l+1)l}+2|\alpha|^lkl^{k/2 - 1}\sqrt{l!}$\,. To upper bound
    \begin{equation*}
        \begin{aligned}
            \tr[\cH[H](\rho)(N+\1)^{k/2}]&=i\tr[{\rho[(N+\1)^{k/2},H]}]\,,
        \end{aligned}
    \end{equation*}
    we define $g_u$ similarly to \Cref{eq:f-g-l-function} by 
    \begin{equation*}
        g_u(x) = \begin{cases}
            f(x) - f(x - u) & x \ge u-1;\\
            f(x) & u-1 > x \ge 0;\\
            0 & 0 > x\,.
        \end{cases}
    \end{equation*}
    For $d_H=0$ the bound is trivial, so we assume $d_H\geq1$. Then, we compute
    \begin{equation*}
        \begin{aligned}
            i[&f(N),H]\\
            &=i\sum_{\substack{0\leq j< i\\0<i+j\leq d_H}}f(N)(\lambda_{i,j}(\ad)^ia^j+\overline{\lambda_{i,j}}(\ad)^ja^i)-(\lambda_{i,j}(\ad)^ia^j+\overline{\lambda_{i,j}}(\ad)^ja^i)f(N)\\
            &=i\sum_{\substack{0\leq j< i\\0<i+j\leq d_H}}\lambda_{i,j}f(N)N[-i+1:-i+j](\ad)^{i-j}+\overline{\lambda_{i, j}}a^{i-j}f(N-i+j)N[-i+1:-i+j]\\
            &\qquad\qquad-\lambda_{i,j}N[-i+1:-i+j]f(N-i+j)(\ad)^{i-j}-\overline{\lambda_{i, j}}a^{i-j}N[-i+1:-i+j]f(N)\\
            &=i\sum_{\substack{0< r\leq i\\0<2i-r\leq d_H}}-\overline{\lambda_{i,i-r}}a^{r}N[-i+1:-r]g_r(N)+\lambda_{i,i-r}g_r(N)N[-i+1:-r](\ad)^{r}\\
            &\overset{(1)}{\leq} \sum_{\substack{0< r\leq i\\0<2i-r\leq d_H}}2|\lambda_{i,i-r}|\sqrt{(N+\1)\cdots(N+r\1)}g_{r}(N+r\1)N[r-i+1:0]\\
            &\overset{(2)}{\leq} \sum_{\substack{0< r\leq i\\0<2i-r\leq d_H}}2\sqrt{r!}|\lambda_{i,i-r}|g_{r}(N+r\1)(N+\1)^{i-r/2}\\
        \end{aligned}
    \end{equation*}
    %\begin{equation*}
    %    \begin{aligned}
    %        i[&f(N),H]\\
    %        &=i\sum_{\substack{0\leq j< i\\0<i+j\leq d_H}}f(N)(\lambda_{i,j}(\ad)^ia^j+\overline{\lambda_{i,j}}(\ad)^ja^i)-(\lambda_{i,j}(\ad)^ia^j+\overline{\lambda_{i,j}}(\ad)^ja^i)f(N)\\
    %        &=i\sum_{\substack{0\leq j< i\\0<i+j\leq d_H}}\lambda_{i,j}f(N)N[-i+1:-i+j](\ad)^{i-j}+\overline{\lambda_{i, j}}a^{i-j}f(N-i+j)N[-i+1:-i+j]\\
    %        &\qquad\qquad-\lambda_{i,j}N[-i+1:-i+j]f(N-i+j)(\ad)^{i-j}-\overline{\lambda_{i, j}}a^{i-j}N[-i+1:-i+j]f(N)\\
    %        &=i\sum_{\substack{0< r\leq i\\0<2i-r\leq d_H}}-\overline{\lambda_{i,i-r}}a^{r}N[-i+1:-r]g_r(N)+\lambda_{i,i-r}g_r(N)N[-i+1:-r](\ad)^{r}\\
    %        &\overset{(1)}{\leq} \sum_{\substack{0< r\leq i\\0<2i-r\leq d_H}}(r+1)|\lambda_{i,i-r}|\sqrt{(N+\1)\cdots(N+r\1)}g_{r}(N+r\1)N[r-i+1:0]\\
    %        &\overset{(2)}{\leq} \sum_{\substack{0< r\leq i\\0<2i-r\leq d_H}}(r+1)\sqrt{r!}|\lambda_{i,i-r}|g_{r}(N+r\1)(N+\1)^{i-r/2}\\
    %    \end{aligned}
    %\end{equation*}
    where we have used \Cref{lem:two-point-hamiltonian-bound} in $(1)$ and \ref{lem:bounds-ccr-l-product} in $(2)$. Next, we use the boundedness assumption $|\lambda_{i,j}|\leq\Lambda$ for all $i,j\in\{1,...,d_H\}$:
    \begin{equation*}
        \begin{aligned}
            i\tr[[H,\rho](N+\1)^{k/2}]&\leq 2\Lambda\sum_{i=1}^{d_H}\sum_{r=1}^i\sqrt{r!}\tr[\rho g_{r}(N+r)(N+\1)^{i-r/2}]\\
            &\overset{(3)}{\leq}2\Lambda\sum_{i=1}^{d_H}\sum_{r=1}^i\sqrt{r!}r^{k/2-1}\tr[\rho (N+\1)^{k/2+i-r/2-1}]\\
            &\leq \Lambda (d_H + 1)d_H\sqrt{d_H!}d_H^{k/2-1}\tr[\rho (N+\1)^{k/2+d_H/2-1}]\,,
        \end{aligned}
    \end{equation*}
    %\begin{equation*}
    %    \begin{aligned}
    %        i\tr[[H,\rho](N+\1)^{k/2}]&\leq \Lambda\sum_{i=1}^{d_H}\sum_{r=1}^i (r+1)\sqrt{r!}\tr[\rho g_{r}(N+r)(N+\1)^{i-r/2}]\\
    %        &\overset{(3)}{\leq}\Lambda\sum_{i=1}^{d_H}\sum_{r=1}^i (r+1)\sqrt{r!}r^{k/2-1}\tr[\rho (N+\1)^{k/2+i-r/2-1}]\\
    %        &\leq \Lambda(d_H+1)d_H^2\sqrt{d_H!}d_H^{k/2-1}\tr[\rho (N+\1)^{k/2+d_H/2-1}]\,,
    %    \end{aligned}
    %\end{equation*}
    where we used \Cref{lem:upper-lower-bound-gl} in $(3)$. As the above function is monotone in $d_H$ we can w.l.o.g assume $d_H=2(l-1)$ and conclude
    \begin{equation*}
        \begin{aligned}
            \tr[(\cL_l+\cH[H])(f(N))]&\leq-l\tr\big[\rho\,(N+\1)^{l+k/2-1}\,\big]\\
            &\qquad\qquad+\frac{l}{2}\left(\Delta_l+\Lambda(2l)^{k/2}\sqrt{(2l)!}\right)\tr[\rho (N)(N+\1)^{l+k/2-2}]\,.
        \end{aligned}
    \end{equation*}
    % \begin{equation*}
    %     \begin{aligned}
    %         \tr[(\cL_l+\cH[H])(f(N))]&\leq-l\tr\big[\rho\,(N+\1)^{l+k/2-1}\,\big]\\
    %         &\qquad\qquad+\frac{l}{2}\left(\Delta_l+\Lambda(2l)^{k/2+1}\sqrt{(2l)!}\right)\tr[\rho (N)(N+\1)^{l+k/2-2}]\,.
    %     \end{aligned}
    % \end{equation*}
   The same optimization as in \Cref{eq:optimization} provides   inequality \eqref{eqdiffLH}. Inequality \eqref{eqintegratedLH} follows after invoking \Cref{prop-ex:uniformly-bounded-semigroup}.
\end{proof}

\begin{lem}[$Z(\theta)$-gate]\label{lem:z-theta-energetic-stab}
    For any state $\rho\in\cT_f$, $\alpha\in\mathbb{C}$, $\varepsilon>0$ and $k\in\N$ 
    \begin{equation*}
        \tr[(\varepsilon\cH[a+\ad]+\cL[a^2+\alpha^2])(\rho)(N+\1)^{k/2}]\leq -\,\tr\big[\rho\, (N+\1)^{k/2}\big]+\mu_k\,.
    \end{equation*}
    where $\mu_k\geq0$ is defined by 
    \begin{equation*}
        \mu_k=(\Delta_2+\varepsilon4k)^\nu\left(\frac{(\nu-1)^{\nu-1}}{\nu^\nu}\right)\qquad\text{with}\qquad\nu=\frac{k}{2}+1\,.
    \end{equation*}
    Therefore, $\varepsilon\cH[a+\ad]+\cL[a^2+\alpha^2]$ generates a Sobolev and positivity preserving quantum Markov semigroup which satisfies for all states $\rho\in W^{k,1}$
    \begin{equation}\label{etlboundfff}
        \|e^{t(\varepsilon\cH[a+\ad]+\cL[a^2+\alpha^2])}(\rho)\|_{W^{k,1}}\leq\max\Big\{\|\rho\|_{W^{k,1}},\mu_k\Big\}\,.
    \end{equation}
\end{lem}
\begin{proof}
    By \Cref{eq-ex:l-dissipation-upper-bound} in \Cref{lem:l-diss},
    \begin{equation*}
        \begin{aligned}
            \tr\big[\cL[a^2-\alpha^2](\rho)(f(N))\big]&\le -2\tr\big[\rho\,(N+\1)^{k/2+1}\,\big]\\
           &\qquad+\biggl(\underbrace{6+2|\alpha|^2k2^{k/2 - 1}\sqrt{2}}_{\eqqcolon \Delta_2}\biggr)\,\tr\big[\rho\,(N+\1 )^{k/2}\big]
        \end{aligned}
    \end{equation*}
    % \begin{equation*}
    %     \begin{aligned}
    %         \tr\big[\cL[a^2-\alpha^2](\rho)(f(N))\big]&\le -2\tr\big[\rho\,(N+\1)^{k/2+1}\,\big]\\
    %        &\qquad+\biggl(\underbrace{6+2|\alpha|^2k2^{k/2}\sqrt{2}}_{\eqqcolon \Delta_2}\biggr)\,\tr\big[\rho\,(N+\1 )^{k/2}\big]
    %     \end{aligned}
    % \end{equation*}
    where $f(x)=(x+1)^{k/2} 1_{x\ge -1}$. Next, by \Cref{eq:symmetry-function}, \Cref{lem:two-point-hamiltonian-bound} and \Cref{lem:upper-lower-bound-gl}, we have that
    \begin{equation*}
        \begin{aligned}
            \tr[\cH[a+\ad](\rho)f(N)]&=i\tr[\rho\left(f(N)(a+\ad)-(a+\ad)f(N)\right)]\\
            &=\tr[\rho\left(-iag_1(N)+ig_1(N)\ad \right)]\\
            &\leq 2\,\tr[\rho g_1(N+\1 )\sqrt{N+\1}]\\
            &\leq 2k\tr[\rho(N+\1)^{k/2-\frac{1}{2}}]\,,
        \end{aligned}
    \end{equation*}
    %\blue{
    %\begin{equation*}
    %    \begin{aligned}
    %        \tr[\cH[a+\ad](\rho)f(N)]&=i\tr[\rho\left(f(N)(a+\ad)-(a+\ad)f(N)\right)]\\
    %        &=\tr[\rho\left(-iag_1(N)+ig_1(N)\ad \right)]\\
    %        &\leq 2\,\tr[\rho g_1(N+\1 )\sqrt{N+\1}]\\
    %        &\leq 4k\tr[\rho(N+\1)^{k/2-\frac{1}{2}}]\,,
    %    \end{aligned}
    %\end{equation*}
    %}
    %\begin{equation*}
    %    \begin{aligned}
    %        \tr[\cH[a+\ad](\rho)f(N)]&=i\tr[\rho\left(f(N)(a+\ad)-(a+\ad)f(N)\right)]\\
    %        &=\tr[\rho\left(-iag_1(N)+ig_1(N)\ad \right)]\\
    %        &\leq 4\,\tr[\rho g_1(N+\1 )\sqrt{N+\1}]\\
    %        &\leq 8k\tr[\rho(N+\1)^{k/2-\frac{1}{2}}]\,,
    %    \end{aligned}
    %\end{equation*}
    where we recall that
    \begin{equation}\label{eq-appx:f-g-l-functionlequal1}
        g_1(x) = \begin{cases}
            f(x) - f(x - 1) & x \ge 0;\\
            0 & 0 > x\,.
        \end{cases}
    \end{equation}
    Thus,
    \begin{equation*}
        \begin{aligned}
            \tr\big[(\epsilon\cH[a+a^\dagger]+\cL[a^2-\alpha^2])(\rho)(f(N))\big]&\le -2\tr\big[\rho\,(N+\1)^{k/2+1}\,\big]+\left(\Delta_2+\varepsilon2k\right)\,\tr\big[\rho\,(N+\1 )^{k/2}\big]
        \end{aligned}
    \end{equation*}
    % \begin{equation*}
    %     \begin{aligned}
    %         \tr\big[(\epsilon\cH[a+a^\dagger]+\cL[a^2-\alpha^2])(\rho)(f(N))\big]&\le -2\tr\big[\rho\,(N+\1)^{k/2+1}\,\big]+\left(\Delta_2+\varepsilon8k\right)\,\tr\big[\rho\,(N+\1 )^{k/2}\big]
    %     \end{aligned}
    % \end{equation*}
    Then the same optimization as in \Cref{eq:optimization}, i.e.
    \begin{equation*}
        \sup_{x\geq0}\left(-x^\nu+(\Delta_2+\varepsilon2k)x^{\nu-1}\right)=(\Delta_2+\varepsilon2k)^\nu\left(\frac{(\nu-1)^{\nu-1}}{\nu^\nu}\right)
    \end{equation*}
    % \begin{equation*}
    %     \sup_{x\geq0}\left(-x^\nu+(\Delta_2+\varepsilon8k)x^{\nu-1}\right)=(\Delta_2+\varepsilon8k)^\nu\left(\frac{(\nu-1)^{\nu-1}}{\nu^\nu}\right)
    % \end{equation*}
    for $\nu\geq1$ defined as 
    \begin{equation*}
        \nu=\frac{k}{2}+1
    \end{equation*}
    ends the proof of the differential upper bound, and \eqref{etlboundfff} follows from \Cref{prop-ex:uniformly-bounded-semigroup}.
\end{proof}

\begin{prop}[CNOT-gate]\label{lem:assum2-CNOT}
    For all $\k\coloneqq (k_1,k_2) \in \N^2$ such that
    \begin{equation*}
        32|\alpha|k_12^{k_1/2-1/2}\leq k_2\,,
    \end{equation*}
     there exists a constant $\mu_{\k}$ such that for all states $\rho \in \cT_f$
    \begin{equation*}
        \begin{aligned}
            &\tr[\left(\cL[a^2-\alpha^2]+\cL[b^2-\alpha^2-\frac{\alpha}{2}(1-e^{2i\pi t/T})(a-\alpha)]\right)(\rho)(N_1 + \1)^{k_1/2}(N_2 + \1)^{k_2/2}]\\
            &\qquad\qquad\qquad\qquad\qquad\qquad\qquad\qquad\qquad\leq-\frac{1+k_2}{8}\tr[\rho\Bigl((N_1+\1)^{k_1/2}(N_2+\1)^{k_2/2}\Bigr)]+\mu_{\k}\,.
        \end{aligned}
    \end{equation*}
    Therefore, the CNOT-gate generates a Sobolev and positivity preserving quantum Markov semigroup which satisfies for all states $\rho\in W^{\k,1}$
    \begin{equation}\label{lastclaimsobolevbound}
        \|\cP^{\operatorname{CNOT}}_{t,t_0}(\rho)\|_{W^{\k,1}}\leq\max\left\{\|\rho\|_{W^{\k,1}},\frac{8\mu_{\k}}{1+k_2}\right\}\,.
    \end{equation}
    For a general $\k \in \R_+^2$ and $x \in W^{\k, 1}$ one obtains
    \begin{equation*}
         \|\cP^{\operatorname{CNOT}}_{t,t_0}(x)\|_{W^{\k_r,1}}\leq \gamma_k \|x\|_{W^{\k_r,1}} \, ,
    \end{equation*}
    where $\gamma_{\k} = \max\{1,\frac{8\mu_{\k}}{1+k_2}\}$ for $\k \in \{\k_r\}_{r \in \N}$ and an interpolated constant in all other cases. Additionally, for $\k > 0$ and $\rho \in W^{\k, 1}$ there exists a sequence $\{t_n\}_{n \in \N_{\geq1}}$ and a $\overline{\rho}\in W^{k,1}$ so that
    \begin{equation*}
        \lim_{t_n\rightarrow\infty}\cP^{\operatorname{CNOT}}_{t_n,t_0}(\rho)=\overline{\rho}\,.
    \end{equation*}
\end{prop}
\begin{proof}
    We denote $f(x_1,x_2)=f_1(x_1)f(x_2)$ with $f_{1}(x_{1})=(x_{1}+1)^{k_{1}/2} 1_{x_{1}\ge -1}$, $f_{2}(x_{2})=(x_{2}+1)^{k_{2}/2} 1_{x_{2}\ge -1}$, and rewrite the CNOT-generator as
    \begin{equation*}
        \begin{aligned}
            \cL\coloneqq \cL[a^2-\alpha^2]&+\cL[b^2-\alpha^2-\frac{\alpha}{2}(1-e^{2i\pi t/T})(a-\alpha)]=\cL[a^2-\alpha^2]+\cL[b^2+za+w]
        \end{aligned}
    \end{equation*}
    where $z\coloneqq-\frac{\alpha}{2}(1-e^{2i\pi t/T})$ and $w\coloneqq -\alpha(z+\alpha)$. As in the previous proofs, we investigate the action of the adjoint on $f(N)\coloneqq f(N_1,N_2)$
    \begin{equation*}
        \tr[\cL(\rho)f(N)]=\tr[\rho\cL^\dagger(f(N))]\,.
    \end{equation*}
    We first focus on the second Lindbladian $\cL[b^2+za+w]$: we first consider, for $n\coloneqq (n_1,n_2)\in\N^2$, 
    \begin{equation*}
        \begin{aligned}
            &\cL[b^2+za+w]^\dagger(\ketbra{n}{n})\\
            &\quad =((\bd)^2+\overline{z}\ad+\overline{w})\ketbra{n}{n}(b^2+za+w)-\frac{1}{2}\left\{((\bd)^2+\overline{z}\ad+\overline{w})(b^2+za+w),\ketbra{n}{n}\right\}\\
            &\quad =F_1(n)\ketbra{n}{n}+F_2(n) \ketbra{n_1,n_2+2}{n_1,n_2+2}+F_3(n) \ketbra{n_1+1,n_2}{n_1+1,n_2}\\
            &\quad\qquad + \big(F_4(n) \ketbra{n_1,n_2+2}{n_1+1,n_2}+h.c.\big)+\big(F_5(n) \ketbra{n_1+1,n_2-2}{n} + h.c.\big) \\
            &\quad\qquad +\big(F_6(n)\ketbra{n_1,n_2-2}{n}+h.c.\big) +\big( F_7(n)\ketbra{n_1-1,n_2}{n} + h.c.\big)\\
            &\quad\qquad + \big(F_8(n)\ketbra{n_1,n_2+2}{n}+h.c. \big)+\big(F_9(n)\ketbra{n_1+1,n_2}{n}+h.c. \big)  \\
            &\quad\qquad + \big(F_{10}(n)\ketbra{n_1-1,n_2+2}{n}+h.c.\big)\,,
        \end{aligned}
    \end{equation*}
    where the notation $h.c.$ above stands for Hermitian conjugate, $|n\rangle=0$ whenever $n\notin \mathbb{N}^2$ by convention, and where
    \begin{align*}
        &F_1(n)\coloneqq - n_2(n_2-1)-|z|^2n_1\\
        &F_2(n)\coloneqq (n_2+1)(n_2+2) \\
        &F_3(n)\coloneqq |z|^2(n_1+1) \\
        &F_4(n)\coloneqq  z\sqrt{(n_1+1)(n_2+1)(n_2+2)}\\
        &F_5(n)\coloneqq -\frac{1}{2}\overline{z}\sqrt{(n_1+1)n_2(n_2-1)}\\
        &F_6(n)\coloneqq  -\frac{1}{2}\overline{w}\sqrt{n_2(n_2-1)}\\
        &F_7(n)\coloneqq -\frac{1}{2}\overline{w}z\sqrt{n_1}\\
        &F_8(n)\coloneqq \frac{1}{2} w\sqrt{(n_2+1)(n_2+2)}\\
        &F_9(n)\coloneqq \frac{1}{2}\overline{z}w\sqrt{n_1+1}\\
        &F_{10}(n)\coloneqq -\frac{1}{2}z\sqrt{n_1(n_2+1)(n_2+2)}\,.
    \end{align*}
    In the next step, we regroup the $17$ terms into terms differing only by a shift:
    \medskip
    \noindent\textit{Case 0:} Diagonal terms, involving $F_1$, $F_2$ and $F_3$,
    \begin{equation*}
        \begin{aligned}
            C_0(n)\coloneqq F_1(n) \ketbra{n}{n}+F_2(n)\ketbra{n_1,n_2+2}{n_1,n_2+2} + F_3(n)\ketbra{n_1+1,n_2}{n_1+1,n_2}\,.
        \end{aligned}
    \end{equation*}
    
    \noindent\textit{Case 1:} Terms of the form $\ketbra{n_1+1,n_2-2}{n_1,n_2}$, involving $\overline{F}_4$, $F_5$ and $\overline{F}_{10}$,
    \begin{equation*}
        \begin{aligned}
            C_1(n)\coloneqq \overline{F}_4(n) \ketbra{n_1+1,n_2}{n_1,n_2+2} + F_5(n) \ketbra{n_1+1,n_2-2}{n} + \overline{F}_{10}(n) \ketbra{n}{n_1-1,n_2+2} \,.
        \end{aligned}
    \end{equation*}
    
    \noindent\textit{Case 1':} Terms of the form $\ketbra{n_1,n_2}{n_1+1,n_2-2}$, involving $F4$, $\overline{F}_5$ and $F_{10}$,
    \begin{equation*}
        \begin{aligned}
            C_{1'}(n)&\coloneqq {F}_4(n) \ketbra{n_1,n_2+2}{n_1+1,n_2} + \overline{F}_5(n) \ketbra{n}{n_1+1,n_2-2} + {F}_{10}(n) \ketbra{n_1-1,n_2+2}{n} \,.
        \end{aligned}
    \end{equation*}

    \noindent\textit{Case 2:} Terms of the form $\ketbra{n_1,n_2-2}{n_1,n_2}$, involving $F_6$ and $\overline{F}_8$,
    \begin{equation*}
        \begin{aligned}
            C_{2}(n)\coloneqq {F}_6(n) \ketbra{n_1,n_2-2}{n}+\overline{F}_8(n)\ketbra{n}{n_1,n_2+2} \,.\\
        \end{aligned}
    \end{equation*}
    
    \noindent\textit{Case 2':} Terms of the form $\ketbra{n_1,n_2}{n_1,n_2-2}$, involving $\overline{F}_6$ and ${F}_8$,
    \begin{equation*}
        \begin{aligned}
            C_{2'}(n)\coloneqq \overline{F}_6(n) \ketbra{n}{n_1,n_2-2}+{F}_8(n) \ketbra{n_1,n_2+2}{n} \,.\\
        \end{aligned}
    \end{equation*}

    \noindent\textit{Case 3:} Terms of the form $\ketbra{n_1-1,n_2}{n_1,n_2}$, involving $F_7$ and $\overline{F}_9$,
    \begin{equation*}
        \begin{aligned}
            C_{3}(n)\coloneqq {F}_7(n) \ketbra{n_1-1,n_2}{n}+\overline{F}_9(n) \ketbra{n}{n_1+1,n_2} \,.\\
        \end{aligned}
    \end{equation*}

   \noindent\textit{Case 3':} Terms of the form $\ketbra{n_1,n_2}{n_1-1,n_2}$, involving $\overline{F}_7$ and ${F}_9$,
    \begin{equation*}
        \begin{aligned}
            C_{3'}(n)\coloneqq \overline{F}_7(n) \ketbra{n}{n_1-1,n_2}+{F}_9(n)\ketbra{n_1+1,n_2}{n} \,.\\
        \end{aligned}
    \end{equation*}
    To summarize, we have decomposed $\cL[b^2+za+w]^\dagger(\ketbra{n}{n})$ into the sum 
    \begin{align}\label{eqLnn}
        \cL[b^2+za+w]^\dagger(|n\rangle\langle n|)=C_0(n)+C_1(n)+C_{1'}(n)+C_2(n)+C_{2'}(n)+C_3(n)+C_{3'}(n)\,.
    \end{align}
    Next, we introduce the functions $g_{j,l}:\mathbb{N}\to \mathbb{R}$, $j\in\{1,2\}$, $l\in\mathbb{N}$, as 
    \begin{equation*}
        g_{j, l}(x) = \begin{cases}
            f_j(x) - f_j(x - l) & x \ge l;\\
            f_j(x) & l > x \ge 0;\\
            0 & 0 > x\,.
        \end{cases}
    \end{equation*}
    Multiplying \Cref{eqLnn} by $f(n)$ and summing over $n\in\mathbb{N}^2$, we find that 
    \begin{align*}
        \cL[b^2+za+w]^\dagger(f(N))=\hat{C}_0+\hat{C}_1+\hat{C}_{1'}+\hat{C}_2+\hat{C}_{2'} +\hat{C}_3+\hat{C}_{3'}\,,
    \end{align*}
    with 
    \begin{align*}
        &\hat{C}_0\coloneqq \sum_{n}\,C_0(n)=\sum_n -\,\Big(f_1(n_1)g_{2,2}(n_2)n_2(n_2-1)+f_2(n_2)g_{1,1}(n_1)|z|^2n_1\Big) \ketbra{n}{n}\\
        &\hat{C}_1\coloneqq \sum_{n}\,C_1(n)\\
        &\qquad =\sum_{n}\,-\frac{\overline{z}}{2}\sqrt{(n_1+1)(n_2-1)n_2}\Bigl(f_1(n_1)g_{2,2}(n_2)+g_{1,1}(n_1+1)f_2(n_2-2)\Bigr)\ketbra{n_1+1,n_2-2}{n}\\
        &\hat{C}_{1'}\coloneqq \hat{C}_1^\dagger \\
        &\qquad =\sum_n\,-\frac{{z}}{2}\sqrt{(n_1+1)(n_2-1)n_2}\Bigl(f_1(n_1)g_{2,2}(n_2)+g_{1,1}(n_1+1)f_2(n_2-2)\Bigr)\ketbra{n}{n_1+1,n_2-2}\\
        &\hat{C}_2\coloneqq \sum_n\,C_2(n) = \sum_n-\frac{\overline{w}}{2}\sqrt{(n_2-1)n_2}f_1(n_1)g_{2,2}(n_2)\ketbra{n_1,n_2-2}{n}\\
        &\hat{C}_{2'}\coloneqq \hat{C}_2^\dagger= \sum_n-\frac{{w}}{2}\sqrt{(n_2-1)n_2}f_1(n_1)g_{2,2}(n_2)\ketbra{n}{n_1,n_2-2}\\
        &\hat{C}_3\coloneqq \sum_n\,C_3(n) = \sum_n-\frac{\overline{w}z}{2}\sqrt{n_1}g_{1,1}(n_1)f_2(n_2)\ketbra{n_1-1,n_2}{n}\\
        &\hat{C}_{3'}\coloneqq \hat{C}_3^\dagger =\sum_n-\frac{{w}\overline{z}}{2}\sqrt{n_1}g_{1,1}(n_1)f_2(n_2)\ketbra{n}{n_1-1,n_2}\,.
    \end{align*}
    We will use an upper bound on $\hat{C}_0(n)$ in what follows:    
    \begin{equation*}
        \begin{aligned}
            &C_0(n) =- \Big(f_1(n_1)g_{2,2}(n_2)(n_2-1)n_2+|z|^2g_{1,1}(n_1)f_2(n_2)n_1\Big)\\
            &\le -\Big( f_1(n_1)g_{2,2}(n_2)1_{n_2\geq2}((n_2+1)^{2}-3(n_2+1))+|z|^2g_{1,1}(n_1)f_2(n_2)n_1\Big)\\
            &\overset{(1)}{\le} -\left\{ k_2f_1(n_1)(n_2+1)^{k_2/2}1_{n_2\geq2}\Big((n_2+1)-1_{k_2\geq3}\frac{k_2}{2}-6\Big) + |z|^2 1_{n_1\geq1}(n_1+1)^{k_1/2-1}f_2(n_2)n_1
            \right\}\\
            &\equiv C_{0'}(n)\,,
        \end{aligned}
    \end{equation*}
    where $(1)$ follows from \Cref{lem:upper-lower-bound-gl}. We denote this upper bound by $\hat{C}_{0'}=\sum_n C_{0'}(n)\ketbra{n}{n}$. Recall that, by \Cref{lem:l-diss} and \Cref{rmk:l-diss-multimode},
    \begin{equation}\label{eq-ex:CNOT-diss-rep}
        \begin{aligned}
            \cL[a^2-\alpha^2]^\dagger(f(N))&\leq - \,(N_1+\1)^{k_1/2+1}(N_2+\1)^{k_2/2}
            +\mu_{k_1}^{(2)}\, (N_2+\1)^{k_2/2} \\
            &=\sum_{n} \Big(-(n_1+1)f(n)+\mu_{k_1}^{(2)}\,f_2(n_2)\Big)\ketbra{n}{n}=:\hat{C}_{4}\,,
        \end{aligned}
    \end{equation}
    where $\mu_{k_1}^{(2)}=\Delta_2^\nu\left(\frac{(\nu-1)^{\nu-1}}{\nu^\nu}\right)$ with $\nu=\frac{k_1}{2}+1$ and $\Delta_2=6+2|\alpha|^2k_12^{k_1/2-1}\sqrt{2}$. Therefore, the diagonal contribution of $\cL^\dagger(f(N))$ can be controlled by 
    \begin{equation*}
        \begin{aligned}
            ( {C}_{0'}+{C}_4 )(n) &\coloneqq -\biggl(k_2f_1(n_1)(n_2+1)^{k_2/2+1}1_{n_2\geq2}+|z|^21_{n_1\geq1}f(n)+(n_1+1)^{k_1/2+1}f_2(n_2)\biggr)\\
            &\qquad+k_2f_1(n_1)(n_2+1)^{k_2/2}1_{n_2\geq2}\biggl(1_{k_2\geq3}\frac{k_2}{2}+6\biggr)\\
            &\qquad+|z|^21_{n_1\geq1}(n_1+1)^{k_2/2-1}f_2(n_2)\\
            &\qquad+\mu_{k_1}^{(2)}(n_2+1)^{k_2/2}\\
            &\le -\frac{1}{2}\biggl(k_2f_1(n_1)(n_2+1)^{k_2/2+1}+(n_1+1)^{k_1/2+1}f_2(n_2)\biggr)+\Delta_0\\
            &\eqqcolon -x(n)
        \end{aligned}
    \end{equation*}
    where the constant $\Delta_0$ is achieved by splitting off half of the negative leading order terms in order to control the lower order positive contributions (see for example the proof of \Cref{lem:l-diss}).
    Next, we consider operators of the form
    \begin{equation}\label{eq:2x2}
        -x_1\ketbra{e_1}{e_1}-x_2\ketbra{e_2}{e_2}+y\ketbra{e_2}{e_1}+\overline{y}\ketbra{e_1}{e_2}\,,
    \end{equation}
    where $\{e_1, e_2\}$ forms an orthonormal basis of a two-dimensional Hilbert space and $x_1, x_2 \in \R, y \in \C$. The operator in \Cref{eq:2x2} has the eigenvalues
    \begin{equation}
        \lambda_+ = \frac{-x_1 - x_2 + \sqrt{(x_1 - x_2)^2 + 4|y|^2}}{2}, \quad \lambda_- = \frac{-x_1 - x_2 - \sqrt{(x_1 - x_2)^2 + 4|y|^2}}{2}
    \end{equation}
    % Next, we consider operators of the form 
    % \begin{equation}\label{eq:2x2}
    %     -x_1\ketbra{e_{0}}{e_0}-x_2\ketbra{e_{m-1}}{e_{m-1}}+y\ketbra{e_{m-1}}{e_0}+\overline{y}\ketbra{e_0}{e_{m-1}}\,,
    % \end{equation}
    % \pg{Why are we considering this big of a basis? It would suffice to look at $e_0, e_1$ for everything that follows.}
    % where $\{e_j\}_{j=0}^{m-1}$ is an orthonormal basis of an $m$ dimensional Hilbert space and $x_1,x_2\in\R$, $y\in\C$.
    % The operator \ref{eq:2x2} has the following eigenvalues:
    % \begin{center}
    %     \begin{tabular}{ c c }
    %         Eigenvalue & Multiplicity \\[0.5ex]\hline
    %         $0$ & $m-2$ \\  
    %         $\frac{-x_1-x_2+\sqrt{(x_1-x_2)^2+4|y|^2}}{2}$ & $1$\\
    %         $\frac{-x_1-x_2-\sqrt{(x_1-x_2)^2+4|y|^2}}{2}$ & $1$
    %     \end{tabular}.
    % \end{center}
    Moreover, 
    \begin{equation*}
        \frac{-x_1-x_2+\sqrt{(x_1-x_2)^2+4|y|^2}}{2}\leq -\min\{x_1,x_2\}+|y|\,.
    \end{equation*}
    Using this bound, we control each of the off-diagonal operators $\hat{C}_i+\hat{C}_{i'}$, $i\in\{1,2,3\}$, in terms of $\frac{1}{4}X$, where $X=\sum_n x(n) \ketbra{n}{n}$. 
    \begin{equation*}
        \begin{aligned}
            -\frac{1}{4}X+\hat{C}_1+\hat{C}_{1'}&\coloneqq \sum_n-\frac{1}{4}x(n)\ketbra{n}{n}+y_1\ketbra{n}{n_1+1,n_2-2}+\overline{y}_1\ketbra{n_1+1,n_2-2}{n}\\
            &\le \sum_{n|n_2\ge 2}-\frac{1}{8}x(n)\ketbra{n}{n}-\frac{1}{8}x(n_1+1,n_2-2)\ketbra{n_1+1,n_2-2}{n_1+1,n_2-2}\\
            &\quad\qquad + y_1\ketbra{n}{n_1+1,n_2-2}+\overline{y}_1\ketbra{n_1+1,n_2-2}{n}
        \end{aligned}
    \end{equation*}
    with
    \begin{equation*}
        y_1 = y_1(n_1,n_2) = -\frac{{z}}{2}\sqrt{(n_1+1)(n_2-1)n_2}\Bigl(f_1(n_1)g_{2,2}(n_2)+g_{1,1}(n_1+1)f_2(n_2-2)\Bigr)
    \end{equation*}
    so that
    \begin{equation}\label{eqcaseC1C1'}
         -  \frac{1}{4}X+\hat{C}_1+\hat{C}_{1'}\leq \sum_{n|n_2\ge 2}\left(-\min\{x_1,x_2\}+|y_1|\right)(\ketbra{n}{n} + \ketbra{n_1 + 1, n_2 - 2}{n_1 + 1, n_2 - 2})
    \end{equation}
    where $x_1=\frac{1}{8}x(n_1,n_2)$ and $x_2=\frac{1}{8}x(n_1+1,n_2-2)$. Moreover, for $n_2\geq2$
    \begin{equation*}\label{eq-ex:cnot-matrix-bound}
        \begin{aligned}
            |y_1|&\overset{(1)}{\leq} \frac{|z|}{2}\sqrt{(n_1+1)(n_2-1)n_2}\Bigl(f_1(n_1)2k_2(n_2+1)^{k_2/2-1}+k_1(n_1+2)^{k_1/2-1}f_2(n_2-2)\Bigr)\\
            &\leq|z|k_2(n_1+1)^{k_1/2+1/2}(n_2+1)^{k_2/2}+\frac{|z|k_1}{2}\sqrt{(n_2-1)^2+n_2-1}\,(n_1+2)^{k_1/2-1/2}f_2(n_2-2)\\
            &\leq|z|k_2(n_1+1)^{k_1/2+1/2}(n_2+1)^{k_2/2}+|z|k_1\,(n_1+2)^{k_1/2-1/2}(n_2-1)^{k_2/2+1}\,,
        \end{aligned}
    \end{equation*}
    where $(1)$ follows from  \Cref{lem:upper-lower-bound-gl}. At this stage, we consider two cases: 
    
    \medskip
    
    \noindent Case (i): $x_2\ge x_1$. In that case, $-\min\{x_1,x_2\}+|y_1|=-x_1+|y_1|$, and therefore
    \begin{equation*}
        \begin{aligned}
           -\min\{x_1,x_2\}+|y_1|&\leq-\frac{1}{16}\biggl(k_2f_1(n_1)(n_2+1)^{k_2/2+1}+(n_1+1)^{k_1/2+1}f_2(n_2)\biggr)+\frac{1}{8}\Delta_0\\
           &+\underbrace{|z|k_2(n_1+1)^{k_1/2+1/2}(n_2+1)^{k_2/2}}_{=:A_1}+\underbrace{|z|k_1\,(n_1+2)^{k_1/2-1/2}(n_2-1)^{k_2/2+1}}_{=:A_2}\,.
        \end{aligned}
    \end{equation*}
    Note that the first positive non-constant term $A_1$ can be controlled with half the negative contribution in the first term by a constant using the same type of polynomial optimization as in the proof of \Cref{lem:l-diss}. For the last term, i.e. $A_2$, we use the assumption 
    \begin{equation}\label{assumptionequationok}
        |z|k_12^{k_1/2-1/2}\leq |\alpha|k_12^{k_1/2-1/2}\leq\frac{1}{32}k_2,
    \end{equation}
    which allows us to control $A_2$ with the other half of the first term, as we already did with $A_1$. Recall the definition $z=-\frac{\alpha}{2}(1-e^{2i\pi t/T})$. Summarising the above considerations we can conclude the existence of a constant $\tilde{\Delta}'_1$ such that 
    \begin{equation*}
        \begin{aligned}
           -\min\{x_1,x_2\}+|y_1|\leq\tilde{\Delta}'_1\,.
        \end{aligned}
    \end{equation*}
    
    \medskip
    
    \noindent Case (ii): $x_2\leq x_1$. In that case $-\min\{x_1,x_2\}+|y_1|=-x_2+|y_1|$, and therefore
    \begin{equation*}
        \begin{aligned}
           -\min\{x_1,x_2\}+|y_1|&=-\frac{1}{16}\biggl(k_2f_1(n_1+1)(n_2-1)^{k_2/2+1}+(n_1+2)^{k_1/2+1}f_2(n_2-2)\biggr)+\frac{1}{8}\Delta_0\\
           &+|z|k_2(n_1+1)^{k_1/2+1/2}(n_2+1)^{k_2/2}+|z|k_1\,(n_1+2)^{k_1/2-1/2}(n_2-1)^{k_2/2+1}\,.
        \end{aligned}
    \end{equation*}
    To upper bound the above, we use again the assumption \eqref{assumptionequationok}, which implies the existence of a constant $\tilde{\Delta}'_1$ such that
    \begin{equation*}
        \begin{aligned}
           -\min\{x_1,x_2\}+|y_1|&\leq\tilde{\Delta}'_1\,.
        \end{aligned}
    \end{equation*}
    Combining cases (i) and (ii) above, denoting $\Delta_1\coloneqq \max\{\Tilde{\Delta}_1,\Tilde{\Delta}_1'\}$ and plugging the bounds into \eqref{eqcaseC1C1'}, we arrive at
    \begin{align}\label{lastC1}
         -\frac{1}{4}X+\hat{C}_1+\hat{C}_{1'}\le \Delta_1 \sum\limits_{n|n_2 \ge 2}(\ketbra{n}{n} + \ketbra{n_1 + 1, n_2 - 2}{n_1 + 1, n_2 - 2})
    \end{align}
   % \begin{align}\label{lastC1}
   %     -\frac{1}{4}X+\hat{C}_1+\hat{C}_{1'}\le \Delta_1\sum_{n|n_2\ge 2}\sum_{m_1=n_1}^{n_1+1}\sum_{m_2=n_2-2}^{n_2} \ketbra{m_1,m_2}{m_1,m_2}\,.
   %  \end{align}
    Next, we control $-\frac{1}{4}X+\hat{C}_2+\hat{C}_{2'}$. Here, we have
    \begin{equation*}
        y_2=y_2(n_1,n_2)=-\frac{\overline{w}}{2}\sqrt{(n_2-1)n_2}f_1(n_1)g_{2,2}(n_2)\,,
    \end{equation*}
    $x_1=\frac{1}{8}x(n)$\,, and $x_2=\frac{1}{8}x(n_1,n_2-2)$. By \Cref{lem:upper-lower-bound-gl}, we have that
    \begin{align*}
        |y_2|&\le {|{w}|}\sqrt{(n_2-1)n_2}f_1(n_1)k_2(n_2+1)^{k_2/2-1}\,.
    \end{align*}
    Therefore, the negative contribution from $\min\{x_1,x_2\}$ has leading order in both variables $n_1$ and $n_2$, which implies the existence of a constant $\Delta_2$ such that
    \begin{equation*}
        -\min\{x_1,x_2\}+|y_2|\leq\Delta_2\,.
    \end{equation*}
    Hence,
    \begin{align}\label{lastC2}
         -\frac{1}{4}X+\hat{C}_2+\hat{C}_{2'}\le \Delta_2 \sum\limits_{n|n_2 \ge 2}(\ketbra{n}{n} + \ketbra{n_1, n_2 - 2}{n_1, n_2 - 2})
    \end{align}
   % \begin{align}\label{lastC2}
   %      -\frac{1}{4}X+\hat{C}_2+\hat{C}_{2'}\le \Delta_2\,\sum_{n|n_2\ge 2}\sum_{m_2=n_2-2}^{n_2} \ketbra{n_1,m_2}{n_1,m_2}\,.
   % \end{align}
    Finally, we consider $-\frac{1}{4}X+\hat{C}_3+\hat{C}_{3'}$. In this case, 
    \begin{equation*}
        y_3=y_3(n_1,n_2)=-\frac{\overline{w}z}{2}\sqrt{n_1}g_{1,1}(n_1)f_2(n_2)\,,
    \end{equation*}
    $x_1=\frac{1}{8}x(n)$\,, and $x_2=\frac{1}{8}x(n_1-1,n_2)$. Similarly to the above, we can argue the existence of a constant $\Delta_3$ such that
    \begin{equation*}
        -\max\{x_1,x_2\}+|y_3|\leq\Delta_3\,.
    \end{equation*}
    Hence,
    \begin{align}\label{lastC3}
         -\frac{1}{4}X+\hat{C}_3+\hat{C}_{3'}\le \Delta_3 \sum\limits_{n|n_1 \ge 1} (\ketbra{n}{n} + \ketbra{n_1 - 1, n_2}{n_1 - 1, n_2}
    \end{align}
    % \begin{align}\label{lastC3}
    %     -\frac{1}{4}X+\hat{C}_3+\hat{C}_{3'}\le \Delta_3\,\sum_{n|n_1\ge 1}\sum_{m_1=n_1-1}^{n_1}\ketbra{m_1,n_2}{m_1,n_2}\,.
    % \end{align}
    Combining \eqref{lastC1}, \eqref{lastC2} and \eqref{lastC3}, we have shown that
    \begin{align*}
        \cL^\dagger(f(N))&\leq -\frac{X}{4}+\Delta_1\sum_{n|n_2\ge 2}(\ketbra{n}{n} + \ketbra{n_1 + 1, n_2 - 2}{n_1 + 1, n_2 - 2})\\
        &+\Delta_2\sum_{n|n_2\ge 2}(\ketbra{n}{n} + \ketbra{n_1, n_2 - 2}{n_1, n_2 - 2}) \\
        &+ \Delta_3 \sum\limits_{n |n_1 \ge 1} (\ketbra{n}{n} + \ketbra{n_1 - 1, n_2}{n_1 - 1, n_2})\\
        &\le -\frac{X}{4}+ 2(\Delta_1+\Delta_2+\Delta_3)\1\\
        &= -\frac{1}{8}\Big(k_2f_1(N_1)(N_2+1)^{k_2/2+1}+(N_1+1)^{k_1/2+1}f_2(N_2)\Big) +\mu_{\k}\,\1\\
        &\le -\frac{1+k_2}{8}f(N)+\mu_{\k}\1\,,
    \end{align*}
    %\begin{align*}
    %     \cL^\dagger(f(N))&\leq -\frac{X}{4}+\Delta_1\sum_{n|n_2\ge 2}\sum_{m_1=n_1}^{n_1+1}\sum_{m_2=n_2-2}^{n_2}\ketbra{m_1,m_2}{m_1,m_2}\\
    %     &+\Delta_2\sum_{n|n_2\ge 2}\sum_{m_2=n_2-2}^{n_2} \ketbra{n_1,m_2}{n_1,m_2}+\Delta_3 \sum_{n|n_1\ge 1}\sum_{m_1=n_1-1}^{n_1} \ketbra{m_1,n_2}{m_1,n_2}\\
    %     &\le -\frac{X}{4}+ (6\Delta_1+3\Delta_2+2\Delta_3)\1\\
    %     &= -\frac{1}{8}\Big(k_2f_1(N_1)(N_2+1)^{k_2/2+1}+(N_1+1)^{k_1/2+1}f_2(N_2)\Big) +\mu_{\k}\,\1\\
    %     &\le -\frac{1+k_2}{8}f(N)+\mu_{\k}\1\,,
    %\end{align*}
    with
    \begin{equation*}
        \mu_{\k}\coloneqq\frac{\Delta_0}{4}+2(\Delta_1+\Delta_2+\Delta_3)\,.
    \end{equation*}
    % \begin{equation*}
    %     \mu_{\k}\coloneqq\frac{\Delta_0}{4}+6\Delta_1+3\Delta_2+2\Delta_3\,.
    % \end{equation*}
    The claim \eqref{lastclaimsobolevbound} finally follows from \Cref{prop-ex:uniformly-bounded-semigroup}. 
\end{proof}