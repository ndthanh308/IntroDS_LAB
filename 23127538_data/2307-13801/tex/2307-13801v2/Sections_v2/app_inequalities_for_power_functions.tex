Many bounds in Sections \ref{sec:examples-sobolev-preserving-semigroup} and \ref{sec:example-perturbation-bounds} can be deduced from bounds on real-valued functions acting on the spectrum of the number operator $N$. Especially, the following functions, first introduced in \Cref{eq:f-g-l-function}, will require special attention: Let $l,k\in\N$, $f(x)=(x+1)^{k/2} 1_{x\ge -1}$, and 
\begin{equation}\label{eq-appx:f-g-l-function}
    g_l(x) = \begin{cases}
        f(x) - f(x - l) & x \ge l-1;\\
        f(x) & l-1 > x \ge 0;\\
        0 & 0 > x\,.
    \end{cases}
\end{equation}
\begin{lem}\label{appx-lem:monotonicity-g}
    Let $g_l$ be defined in \Cref{eq-appx:f-g-l-function} for $l,k\in\N$. Then, for all $k\geq2$ and $x\in\R$
    \begin{align}
        g_l(x)&\leq g_{l+1}(x)\label{appx-eq:monotonicity-k}\,,\\
        g_l(x-l)&\leq g_l(x)\label{appx-eq:monotonicity-x}\,.
    \end{align}
\end{lem}
\begin{proof}
    By the monotonicity and non-negativity of $f(x)=(x+1)^{k/2} 1_{x\ge -1}$,
    \begin{equation*}
        \begin{rcases}
            x\ge l-1 & f(x)-f(x-l)\\
            l-1>x\geq 0 & f(x)\\
            0>x& 0
        \end{rcases}
        =g_l(x)\leq g_{l+1}(x)=
        \begin{cases}
            f(x)-f(x-(l+1)) & x\ge l \\
            f(x) & l>x\geq 0 \\
            0 & 0>x
        \end{cases}\,,
    \end{equation*}
    which proves Inequality \ref{appx-eq:monotonicity-k}. For Inequality \ref{appx-eq:monotonicity-x}, we consider the following cases:
    \begin{equation*}
        \begin{rcases}
                f(x-l)-f(x-2l)\\
                f(x-l)\\
                0 \\
                0
        \end{rcases}
        =g_l(x-l)\leq g_l(x)=
        \begin{cases}
            f(x)-f(x-l) & \qquad x\geq 2l-1\\
            f(x)-f(x-l) & \qquad 2l-1>x\geq l-1 \\
            f(x) & \qquad l-1>x\geq 0 \\
            0 & \qquad 0>x
        \end{cases}\,.
    \end{equation*}
   For $x<l-1$, the inequalities are clear by the non-negativity of $f$. The case $l-1\leq x<2l-1$ follows by
    \begin{equation*}
        2\frac{f(x-l)}{f(x)}=2\left(1-\frac{l}{x+1}\right)^{k/2}\leq2\left(1-\frac{1}{2}\right)^{k/2}=1
    \end{equation*}
    and the last case $x\geq 2l-1$ follows by monotonicity of $g_l$:
    \begin{equation*}
        \frac{2}{k}g_l'(x-l)=(x-l)^{k/2-1}-(x-2k)^{k/2-1}\geq0\,.\vspace{-2ex}
    \end{equation*}
\end{proof}

Next, we prove upper and lower bounds for $g_l$:
\begin{lem}\label{lem:upper-lower-bound-gl}
     Let $g_l:\R\rightarrow\R_{\geq0}$ be defined in \Cref{eq-appx:f-g-l-function} for $l\in\N$. Then, for all $x\in\R$ and $k\in\N$,
    \begin{equation*}
        \begin{rcases}
            x\geq l-1 & (x+1)^{k/2-1}\frac{kl}{2}-1_{k\geq3}(x+1)^{k/2-2}\,\,\frac{(kl)^2}{8} \\
            x\geq l-1 & (x+1)^{k/2-1}l\\
            l-1>x\geq0&(x+1)^{k/2}\\
            0>x&0
        \end{rcases}
        \leq g_l(x)
    \end{equation*}
    and 
    \begin{equation*}
        g_l(x)\leq 
        \begin{cases}
            \frac{kl}{2}\left(1+1_{k=1}\right)(x+1)^{k/2-1} & x\geq 0\\
            (x+1)^{k/2}&x\geq0 \\
            0 & 0>x
        \end{cases}\,.
    \end{equation*}
\end{lem}
\begin{proof}
    The case $k=0$ is trivial. We start with the upper bounds. By monotonicity of $g_l$, it is enough to prove the first upper bound just for $x\geq l-1$. For $k=1$,
    \begin{equation*}
        g_l(x)=(x+1)^{-1/2}\frac{l}{2}\int_0^1\left(1-s\frac{l}{x+1}\right)^{-1/2}ds\leq(x+1)^{-1/2}\frac{l}{2}\int_0^1\left(1-s\right)^{-1/2}ds=(x+1)^{-1/2}l.
    \end{equation*}
For $k\geq2$,
    \begin{equation*}
        g_l(x)=\frac{k}{2}\int_0^l(x+1-s)^{k/2-1}ds\leq\frac{kl}{2}(x+1)^{k/2-1}\,,
    \end{equation*}
    which finishes the proof of the first upper bound. The other two bounds are obvious by definition. Next, we consider the lower bounds. The case $x<l-1$ is trivial so we are left with proving
    \begin{equation*}
        (x+1)^{k/2-1}\frac{kl}{2}-\delta_{k\geq3}(x+1)^{k/2-2}\frac{(kl)^2}{8}\leq g_l(x)
    \end{equation*}
    for $x\geq l-1$. For $k=1$, the integral representation can be lower bounded as
    \begin{equation*}
        \begin{aligned}
            g_l(x)&=(x+1)^{-1/2}\frac{l}{2}\int_{0}^1\left(1-s\frac{l}{x+1}\right)^{-1/2}ds\geq(x+1)^{-1/2}\frac{l}{2}\,.
        \end{aligned}
    \end{equation*}
    For $k=2$, it is again easy to calculate the quantity $g_l(x)=l$, and for $k=3$
    \begin{equation*}
        \begin{aligned}
            g_l(x)&=(x+1)^{1/2}\frac{3l}{2}\int_{0}^1\left(1-s_1\frac{l}{x+1}\right)^{1/2}ds_1\\
            &=(x+1)^{1/2}\frac{3l}{2}-(x+1)^{-1/2}l^2\frac{3}{4}\iint_{0}^1s_1\left(1-s_1s_2\frac{l}{x+1}\right)^{-1/2}ds_2ds_1\\
            &\geq (x+1)^{1/2}\frac{3l}{2}-(x+1)^{-1/2}l^2\frac{3}{4}\iint_{0}^1s_1\left(1-s_1s_2\right)^{-1/2}ds_2ds_1\\
            &=(x+1)^{1/2}\frac{3l}{2}-(x+1)^{-1/2}\frac{l^2}{2}.
        \end{aligned}
    \end{equation*}
    Finally, the case $k\geq4$ is given by
    \begin{equation*}
        \begin{aligned}
            g_l(x)&=(x+1)^{k/2-1}\frac{kl}{2}\int_{0}^1\left(1-s_1\frac{l}{x+1}\right)^{k/2-1}ds_1\\
            &=(x+1)^{k/2-1}\frac{kl}{2}-(x+1)^{k/2-2}l^2\frac{k(k-2)}{4}\iint_{0}^1s_1\left(1-s_1s_2\frac{l}{x+1}\right)^{k/2-2}ds_2ds_1\\
            &\geq(x+1)^{k/2-1}\frac{kl}{2}-(x+1)^{k/2-2}l^2\frac{k(k-2)}{4}\int_{0}^1s_1ds_1\\
            &\geq(x+1)^{k/2-1}\frac{kl}{2}-(x+1)^{k/2-2}\frac{(kl)^2}{8}\,
        \end{aligned}
    \end{equation*}
    which proves the first non-trivial lower bound for $x\geq l-1$. Next, we consider  
    \begin{equation*}
        g_l(x)\geq(x+1)^{k/2-1}l\,.
    \end{equation*}
    The inequality is obvious for $k<2$ by the same idea as before and for $k\geq2$
    \begin{equation*}
        \begin{aligned}
            g_l(x)&=(x+1)^{k/2-1}\frac{kl}{2}\int_{0}^1\left(1-s_1\frac{l}{x+1}\right)^{k/2-1}ds_1\\
            &\geq(x+1)^{k/2-1}\frac{kl}{2}\int_{0}^1\left(1-s_1\right)^{k/2-1}ds_1\\
            &\geq(x+1)^{k/2-1}l
        \end{aligned}
    \end{equation*}
which ends the proof. 
\end{proof}

\begin{lem}\label{lem:bounds-ccr-l-product}
    Let $l\in\N$ and $x\geq l$, then
    \begin{equation*}
        \begin{aligned}
            (x+1)^l-\frac{(l+1)l}{2}(x+1)^{l-1}&\leq&((x+1)-l)\cdots ((x+1)-1)&\leq&(x+1)^l\\
            (x+1)^l&\leq&(x+1)\cdots (x+1+(l-1)&\leq&l!(x+1)^l
        \end{aligned}
    \end{equation*}
\end{lem}
\begin{proof}
    To prove \Cref{lem:bounds-ccr-l-product}, we redefine $y=x+1$ and rewrite the first product as
    \begin{equation*}
        p_l(y)\coloneqq(y-l)\cdots (y-1)\eqqcolon y^l-\frac{(l+1)l}{2}y^{l-1}+r_{l-2}(y)
    \end{equation*}
    where $r_{l-2}$ is a polynomial of degree $l-2$. The proof idea is to show that $r_{l-2}(y)$ is non-negative for all $y\geq l+1$, which  proves the inequality. The non-negativity of the polynomial $r_{l-2}$ can be proven by induction over $l$: The statement is directly clear for $l=1$ and $l=2$. Next, we assume that $r_{l-2}$ is non-negative for all $x\geq l+1$ and show that $r_{l-1}$ is for all $x\geq l+2$. 
    \begin{equation*}
        \begin{aligned}
            p_{l+1}(y)&=(y-(l+1))p_l(y)\\
            &=(y-(l+1))\left(y^l-\frac{(l+1)l}{2}y^{l-1}+r_{l-2}(y)\right)\\
            &=y^{l+1}-\frac{(l+1)(l+2)}{2}y^{l}+\frac{(l+1)^2l}{2}y^{l-1}+(y-(l+1))r_{l-2}(y)\\
            &=y^{l+1}-\frac{(l+1)(l+2)}{2}y^{l}+r_{l-1}(y).
        \end{aligned}
    \end{equation*}
    For the second product $(x+1)\cdots ((x+1)+l-1)$ the lower bound is clear and the upper bound follows by 
    \begin{equation*}
        (x+1)(x+2)\cdots ((x+1)+l-1)=(l-1)!\left(\frac{x}{1}+1\right)\cdots \left(\frac{x}{l}+1\right)\leq l!(x+1)^l.
    \end{equation*}
\end{proof}