In this section, we establish a perturbative analysis at any time scale for the semigroups considered in \Cref{sec:examples-sobolev-preserving-semigroup}. In finite dimensions, \cite[Theorem 6]{Szehr_2013} gives a quantitative bound which controls the perturbation of a quantum dynamical semigroup under the condition that the latter converges exponentially fast to a unique invariant state $\tau$: for two generators $\cL$ and $\cL+\cK$, if $\cL$ satisfies $\|e^{t\cL} - \tr(.)\,\tau\|_{1\rightarrow 1} \le c e^{- \omega t}$ for all $t\geq0$ and some $c, \omega > 0$, then
\begin{equation*}
   \forall\rho,\sigma \text{ states},\quad  \norm{e^{t\cL}(\rho) - e^{t(\cL+\cK)}(\sigma)}_1 \le 
    \begin{cases}
        \norm{\rho - \sigma}_1 + t \norm{\cK}_{1\rightarrow 1}\,, &t < \hat{t}\\
        c e^{- \omega t} \norm{\rho - \sigma}_1 + \frac{\log(c) + 1 - c e^{-\omega t}}{\omega} \norm{ \cK}_{1\rightarrow 1}\,, &  t \ge \hat{t}
    \end{cases}
\end{equation*}
where $\hat{t} \coloneqq \frac{\log(c)}{\omega}$. The result can be easily extended to the case of bounded generators in infinite dimensions, although proving the exponential decay for the semigroup generated by $\cL$ is not easy. The situation becomes even trickier in the case of unbounded generators since the use of a Duhamel integral as in the proof in finite dimensions requires a proper justification. It is precisely these issues that we are interested in and want to address here.

\subsection{Gaussian perturbations of the quantum Ornstein Uhlenbeck semigroup}

The quantum Ornstein Uhlenbeck semigroup is well-known to correspond to a so-called beam-splitter channel of exponentially decreasing transmissivity $e^{-(\lambda^2-\mu^2)t}$ with unique Gaussian invariant state (see \cite{DePalma.2018}):
\begin{equation*}
    \sigma\coloneqq \frac{\lambda^2-\mu^2}{\mu^2}\sum_{k=0}^\infty \left(\frac{\mu^2}{\lambda^2}\right)^k\,\ketbra{k}{k}\,.
\end{equation*}
While quantitative statements about the convergence of this semigroup towards $\sigma$ are known \cite{Cipriani.2000,Carbone.2007,Carlen.2017,DePalma.2018}, 
they do not necessarily imply convergence in trace distance in contrast to their finite-dimensional analogues. In contrast, the semigroup is known to contract a certain kind of quantum Wasserstein distance, which we introduce now. First, given a bounded, self-adjoint operator $X\in\cB(\cH)$, we call $X$ a Lipschitz observable if $aX$, $a^\dagger X$ are bounded, and if $Xa$ and $Xa^\dagger$ are closable operators with bounded closures $\overline{Xa}$ and $\overline{Xa^\dagger}$. In this case, we denote by $\partial_a(X)\coloneqq aX-\overline{Xa}$ and $\partial_{a^\dagger}(X)=a^\dagger X-\overline{Xa^\dagger}$. The Lipschitz constant of $X$ is then defined as
\begin{align*}
    \|X\|_{\operatorname{Lip}}\coloneqq \max\big\{\|\partial_a(X)\|_\infty,\,\|\partial_{a^\dagger}(X)\|_\infty\big\}\,.
\end{align*}
We denote the set of Lipschitz observables by $\operatorname{Lip}$. Next,  any $T\in\cT_{1,\operatorname{sa}}$, we denote
\begin{align*}
    \|T\|_{W_1}\coloneqq \sup\,\left\{ \tr\big[X\,T\big]:\,X\in\operatorname{Lip},\,\|X\|_{\operatorname{Lip}}\le 1\right\}\,.
\end{align*}
In \cite[Proposition 6.4]{Gao.2021}, the authors showed that, for any $T\in\cT_{1,\operatorname{sa}}$ and $t>0$,
\begin{align}\label{regularization}
    \|e^{t\cL_{\operatorname{qOU}}}(T)\|_1\le \sqrt{\frac{e^{-(\lambda^2-\mu^2)t}}{1-e^{-(\lambda^2-\mu^2)t}}}\,\Big(\|a\sigma-\sigma a\|_1+\|a^\dagger \sigma-\sigma a^\dagger \|_1\Big)\,\|T\|_{W_1} \,.
\end{align}
Moreover, using the canonical commutation relations, one can also prove the following identities (see e.g.~\cite{Carlen.2017}, or \cite[Proposition 6.2]{Gao.2021}): for any two states $\rho_1,\rho_2\in \cT_{1, \operatorname{sa}}$,  
\begin{equation}\label{eq:qou-exponential-dampening}
    \|e^{t\cL_{\operatorname{qOU}}}(\rho_1-\rho_2)\|_{W_1}\le e^{-\frac{(\lambda^2-\mu^2)t}{2}}\,\|\rho_1-\rho_2\|_{W_1}\,.
\end{equation}
In the next proposition, we use these conditions to find a perturbation bound for any Gaussian perturbation of the quantum Ornstein Uhlenbeck semigroup. 

\begin{prop}\label{propqOUperturb}
    Let $(\cL_{\operatorname{qOU}},\cT_f)$ be the generator of the quantum Ornstein Uhlenbeck semigroup with $\lambda>\mu\geq0$ and $(\varepsilon\cL_G,\cT_f)\coloneqq (\varepsilon\cL[{\gamma a+\eta\ad}],\cT_f)$ a Gaussian perturbation with $\gamma,\eta\in\mathbb{R}$, $\varepsilon>0$. Then, assuming $\lambda^2-\mu^2+|\gamma|^2-|\eta|^2> 0$, $\cL_{\operatorname{qOU}}+\varepsilon\cL_G$ generates a positivity and Sobolev preserving semigroup on $W^{k,1}$ for $k\geq1$, and there exist uniformly bounded functions $C(\varepsilon),D(\varepsilon)$ depending on $\lambda,\mu,|\eta|,|\gamma|$ such that, for all $t\ge 0$ and states $\rho\in W^{2,1}$
    \begin{equation}\label{eq-ex:perturbation-bound-qOU}
        \Big\|\left(e^{t \cL_{\operatorname{qOU}}}-e^{t(\cL_{\operatorname{qOU}}+\varepsilon\cL_G)}\right)(\rho)\Big\|_{1}\leq \varepsilon\, C(\varepsilon)\, \max\Big\{\norm{\rho}_{W^{2, 1}} ,D(\varepsilon)\Big\}\,.
    \end{equation}
\end{prop}
\begin{proof}
    The generation of a Sobolev preserving semigroup was already stated in Lemma \ref{lem-ex:qOU-differential-stability} for $\cL_{\operatorname{qOU}}$ and its proof can easily be extended to $\cL_{\operatorname{qOU}}+\varepsilon \cL_G$. For instance, given a state $\rho\in\cT_f$, one can show that
     \begin{equation*}
         \tr[\cL_{G}(\rho)(N+\1)]\leq -(|\gamma|^2-|\eta|^2)\,\tr[\rho N]+|\eta|^2\,,
     \end{equation*}    
    We have also seen in the proof of \Cref{lem-ex:qOU-differential-stability} that $\tr[\rho \cL_{\operatorname{qOU}}]\le -(\lambda^2-\mu^2)\tr[\rho N]+\mu^2$, so that
     \begin{align*}
        \tr[(\cL_{\operatorname{qOU}}+\varepsilon \cL_G)(\rho)(N+\1)]\le -(\lambda^2-\mu^2+\varepsilon|\gamma|^2-\varepsilon|\eta|^2)\tr[\rho (N+\1)]+\lambda^2+\varepsilon|\gamma|^2\,.
    \end{align*} 
    Therefore, by \Cref{prop-ex:uniformly-bounded-semigroup} we have that, as long as $\lambda^2-\mu^2+\varepsilon|\gamma|^2-\varepsilon|\eta|^2> 0$, for all states $\rho\in W^{2,1}$, $\rho \ge 0$ and all $t\ge 0$,
    \begin{align*}
        \|e^{t(\cL_{\operatorname{qOU}}+\varepsilon \cL_G)}(\rho)\|_{W^{2,1}}\le \max\left\{\norm{\rho}_{W^{2, 1}}, \frac{\lambda^2+\varepsilon|\gamma|^2}{\lambda^2-\mu^2+\varepsilon|\gamma|^2-\varepsilon|\eta|^2}\right\} \,.
    \end{align*}
    Next, for $\rho\in\cT_f$ and $0<u<t $, and denoting $\cL\equiv \cL_{\operatorname{qOU}}$ and $\widetilde{\cL}\equiv \cL_{\operatorname{qOU}}+\varepsilon \cL_G$,
    \begin{align*}
        \Big\|\Big(e^{t\cL}-e^{t\widetilde{\cL}}\Big)(\rho)\Big\|_1\le    \Big\|e^{u\cL}\left(e^{(t-u)\cL}-e^{(t-u)\widetilde{\cL}}\right)(\rho)\Big\|_1 +\Big\| \Big(e^{u\cL}-e^{u\widetilde{\cL}}\Big)e^{(t-u)\widetilde{\cL}}(\rho)\Big\|_1\equiv A+B\,.
    \end{align*}
    We use \Cref{regularization}, so that
    \begin{align*}
    	A&\le c_u \Big\|\Big(e^{(t-u)\cL}-e^{(t-u)\widetilde{\cL}}\Big)(\rho)\Big\|_{W_1}\\
    	&\le c_u\,\varepsilon\, \int_0^{t-u}\,\Big\|e^{s\cL}\cL_G\,e^{(t-u-s)\widetilde{\cL}}(\rho)\Big\|_{W_1}\,ds\\
    	&=  c_u\,\varepsilon\, \int_0^{t-u}\,e^{-\frac{(\lambda^2-\mu^2)s}{2}}\Big\|\cL_G\,e^{(t-u-s)\widetilde{\cL}}(\rho)\Big\|_{W_1}\,ds\,,
    \end{align*}
    where $c_u\coloneqq \sqrt{\frac{e^{-(\lambda^2-\mu^2)u}}{1-e^{-(\lambda^2-\mu^2)u}}}\,\,\Big(\|\partial_a(\sigma)\|_1+\|\partial_{a^\dagger}(\sigma)\|_1\Big)$. Moreover, denoting $\widetilde{\rho}_v\coloneqq e^{v\widetilde{\cL}}(\rho)$ and $b=\gamma a+\eta a^\dagger$, since $\widetilde{\rho}_v\in W^{2,1}$ for all $v\ge 0$,
    \begin{align*}
        \Big\|\cL_G\,\widetilde{\rho}_v\Big\|_{W_1}\,&=\sup_{\|X\|_{\operatorname{Lip}}\le 1}\,\tr[X \cL_G \widetilde{\rho}_v]\\
        &=\frac{1}{2}\,\sup_{\|X\|_{\operatorname{Lip}}\le 1}\, \tr[\partial_{b^\dagger}(X)b \widetilde{\rho}_v-\partial_b(X)\,\widetilde{\rho}_vb^\dagger]\\
        &\le (|\eta|+|\gamma|)\, \Big(\|b\widetilde{\rho}_v\|_1+\|\widetilde{\rho}_vb^\dagger\|_1\Big)\\
        &\le (|\eta|+|\gamma|)\, \Big(\|b\,(N+\1)^{-\frac{1}{2}}\|+\|(N+\1)^{-\frac{1}{2}}\,b^\dagger\|\Big)\,\|\widetilde{\rho}_v\|_{W^{2,1}}\\
        &\le (|\eta|+|\gamma|) \Big(\|b\,(N+\1)^{-\frac{1}{2}}\|+\|(N+\1)^{-\frac{1}{2}}\,b^\dagger\|\Big)\\
        &\qquad \qquad \qquad \cdot \max\left\{\norm{\rho}_{W^{2, 1}}, 
         \frac{\lambda^2+\varepsilon|\gamma|^2}{\lambda^2-\mu^2+\varepsilon|\gamma|^2-\varepsilon|\eta|^2} \right\}  \, .
    \end{align*}
    On the other hand, 
    \begin{align*}
        B&\le \varepsilon\,\int_0^u\,\|e^{s\cL} \cL_Ge^{(t-s)\widetilde{\cL}}(\rho)\|_1\,ds\\
        &\le\,u\varepsilon\,\max_{s\in[0,u]}\, \|\cL_Ge^{(t-s)\widetilde{\cL}}(\rho)\|_1 \\
        &\le u\varepsilon \|\cL_G \circ \mathcal{W}^{-2}\|_{\cT_1\to \cT_1}\,\max_{s\in[0,u]}\,\|e^{(t-s)\widetilde{\cL}}(\rho)\|_{W^{2,1}}\\
        &\le u\varepsilon \|\cL_G \circ \mathcal{W}^{-2}\|_{\cT_1\to \cT_1}\, \max\left\{ \|\rho\|_{W^{2,1}}, 
        \frac{\lambda^2+\varepsilon|\gamma|^2}{\lambda^2-\mu^2+\varepsilon|\gamma|^2-\varepsilon|\eta|^2} \right\}\,.
    \end{align*}
    The result follows from a simple bound on $\|\cL_G\circ \cW^{-2}\|_{\cT_1\to \cT_1}$ with the help of H\"{o}lder's inequality and a standard density argument, and by choosing $u$ appropriately.
\end{proof}

\subsection{Photon-dissipation and CAT qubits}\label{subsec:cat-perturbation}

As mentioned before, one crucial property of the underlying evolution in continuous error correction is that it is exponentially converging to the code-space. In the spirit of \cite{Szehr_2013}, we prove a large time perturbation result for the $l$-photon dissipation perturbed by a Hamiltonian evolution. It is clear that this can be generalized to dissipative perturbations. First, we recall the exponential convergence of the $l$-photon dissipation (\cite[Theorem 2]{Azouit.2016}):
\begin{equation}\label{eq:exponential-convergence}
    \tr[Le^{t\cL_l}(\rho) L^\dagger]\leq e^{-l!t}\tr[L\rho L^\dagger]\,.
\end{equation}
Additionally, it is shown that there is a unique limit $\overline{\rho}$ of $e^{t\cL_l}(\rho)$ for $t\rightarrow\infty$. We show large-time perturbation bounds by combining this bound with our established generation theory for Sobolev and positivity-preserving Markov semigroups. We start with the $l$-photon dissipation perturbed by the Hamiltonian introduced in \Cref{lem:l-diss}, i.e.~$H=p_H(a,\ad)$ with $d_H\leq2(l-1)$.

\begin{thm}\label{thm:l-diss-hamiltonian-perturbation}
    Let $\cL_l$ be the generator of the $l$-photon dissipation and $p_H\in\C[X,Y]$ with $\deg(p_H)=d_H\leq2(l-1)$ such that $H=p_H(a,\ad)$ is a symmetric operator. Then, there exist explicit constants $c,\gamma>0$ such that for $\varepsilon \ge 0$ and all states $\rho\in W^{2(l + d_H + 2),1}$
    \begin{equation*}
        \begin{aligned}
            \Big|\tr[L\left(e^{t\cL_l}(\rho)-e^{t(\cL_l+\varepsilon\cH[H])}(\rho)\right)L^\dagger]\Big|\leq\varepsilon c\left(1-e^{-l!t}\right)\max\{\gamma,\|\rho\|_{W^{2(l + d_H + 2),1}}\}\,.
        \end{aligned}
    \end{equation*}
\end{thm}
\begin{proof}
    The proof consists in applying \Cref{lem:l-diss-hamiltonian} in combination with \Cref{eq:exponential-convergence}. Let $\rho \in \cT_f$ and $\cW^k = (N + \1)^{k/4} \cdot (N + \1)^{k/4}$, then
    \begin{equation*}
        \begin{aligned}
            &\tr[L\biggl(e^{t\cL_l}(\rho)-e^{t(\cL_l+\varepsilon\cH[H])}(\rho)\biggr)L^\dagger]\\
            &\qquad= \varepsilon \int_{0}^t \tr[L e^{s\cL_l}\cW^{-2(l + 2)}\cW^{2(l + 2)}\cH{[H]}e^{(t-s)(\cL_l+\varepsilon\cH[H])}(\rho)L^\dagger] ds\\
            &\qquad\overset{(1)}{\leq}\varepsilon \int_{0}^t\tr[L e^{s\cL_l}\cW^{-2(l + 2)}(\1)L^\dagger] ds \, 2\Lambda d_H^{l + 2} \sqrt{d_H!}\max\left\{\gamma_\varepsilon,\,\|\rho\|_{W^{2(l + d_H + 2),1}}\right\}\\
            &\qquad \overset{(2)}{\leq}\varepsilon \frac{\pi^2}{3}\Lambda d_H^2\sqrt{d_H!}(1+|\alpha|^l\,(l+1)\,\sqrt{l!}+|\alpha|^{2l})\frac{1}{l!}(1-e^{-l!t})\max\left\{\gamma_\varepsilon,\,\|\rho\|_{W^{2(l + d_H + 2),1}}\right\}\\
            &\qquad\eqqcolon\varepsilon c(1-e^{-l!t})\max\left\{\gamma_\varepsilon,\,\|\rho\|_{W^{2(l + d_H + 2),1}}\right\}
        \end{aligned}
    \end{equation*}
    where $\Lambda$ denotes the larges coefficient of $\cH[H]$ in absolute value. Note that the Bochner integral in the calculation is well-defined by the of the boundedness of the integrand w.r.t.~$s$ and the same argumentation as \Cref{thm:semigroup-perturbation}. Besides the boundedness above, the Sobolev preserving property of the involved semigroups imply by construction that the integral is Sobolev preserving. Therefore, the integral commutes with the the map $x\mapsto\tr[LxL^\dagger]$ for $x\in W^{2(l+d_H+2),1}$. In $(1)$ we used that $L e^{s\cL_l}(\cdot) L^\dagger$ preserves positivity and
    %where issues with continuity and permutation of integrals and unbounded operators do not play a role since the semigroups preserve the Sobolev spaces and $\rho \in \cT_f$ (q.f. \Cref{thm:semigroup-perturbation}). In $(1)$ we used that $L e^{s\cL_l}(\cdot) L^\dagger$ preserves positivity and
    \begin{align*}
        \cW^{2(l + 1)} \cH[H] e^{t - s(\cL_l + \varepsilon \cH[H]}(\rho) &\le \norm{\cW^{2(l + 2)} \cH[H] e^{(t - s)(\cL_l + \varepsilon \cH[H]}(\rho)}_\infty \1\\
        &\le \norm{\cW^{2(l + 2)} \cH[H] e^{(t - s)(\cL_l + \varepsilon \cH[H]}(\rho)}_1 \1\\
        &\le 2 \Lambda d_H^{l + 2} \sqrt{d_H!} \norm{e^{(t - s)(\cL_l + \varepsilon \cH[H]}(\rho)}_{W^{2(l + d_H + 2)}} \1\\
        &\le 2 \Lambda d_H^{l + 2} \sqrt{d_H!} \max\{\gamma_\varepsilon, \norm{\rho}_{W^{2(l + d_H + 2)}}\} \1 \, . 
    \end{align*}
    In the above estimation, we used $(N + \1)^{-d_H}$ to control $\cH[H]$. We further used the Sobolev preserving property of the semigroup from \Cref{lem:l-diss-hamiltonian}. Finally, applying the decay (q.v.~\Cref{eq:exponential-convergence}) to $\cW^{-2(l + 1)}(\1) = (N + \1)^{-(l + 2)}$ we estimated in $(2)$:
    \begin{align*}
        \int_{0}^t\tr[L e^{s\cL_l}\cW^{-2(l + 2)}(\1)L^\dagger] ds &\le \int\limits_{0}^t e^{-l! t} \tr[L(N + \1)^{-(l + 2)}L^\dagger]\\
        &\le (1 - e^{-l!t}) (1+|\alpha|^l\,(l+1)\,\sqrt{l!}+|\alpha|^{2l}) \frac{\pi^2}{6}
    \end{align*}
    with the bound 
    \begin{equation*}
        \|L(N+1)^{-l - 2}L^\dagger\|_1\leq (1+|\alpha|^l\,(l+1)\,\sqrt{l!}+|\alpha|^{2l}) \frac{\pi^2}{6}
    \end{equation*}
    that follows from
    \begin{equation}
        \begin{aligned}
            L^\dagger L&=(\ad)^l a^l-\overline{\alpha}^la^l-\alpha^l(\ad)^l+|\alpha|^{2l}\\
            &\leq(N-(l-1)\1)\cdots(N-\1)N+|\alpha|^l\,(l+1)\,\sqrt{(N+l\1)\cdots (N+\1)}+|\alpha|^{2l}\\
            &\leq (N+\1)^l+|\alpha|^l\,(l+1)\,\sqrt{l!}\,(N+\1)^{l/2}+|\alpha|^{2l}\,.
        \end{aligned}
    \end{equation}
    and the $\norm{(N + \1)^{-2}}_1 = \frac{\pi^2}{6}$. This concludes that claim.
    % For that, we show that
    % \begin{equation*}
    %     \|\cW^{\blue{2(l + 1)}}\cH[H]\cW^{-6l+4}\|_{1\rightarrow1}<\infty\,,
    % \end{equation*}
    % where we recall that $\cW^k(\cdot) \coloneqq (N+\1)^{k/4}(\cdot)(N+\1)^{k/4}$. For a state $\rho\in\cT_f$ and $Y\coloneqq\cW^{-6l+4}(\rho)$, we have by Hölder's inequality
    % \begin{equation*}
    %     \begin{aligned}
    %         \|\cW^{2l}\cH[H](Y)\|_{1}&\leq\|(N+\1)^{l/2}H(N+\1)^{-3/2l+1}\rho(N+\1)^{-l+1}\|_1\\
    %         &\quad+\|(N+\1)^{-l+1}\rho(N+\1)^{-3/2l+1}H(N+\1)^{l/2}\|_1\\
    %         &\leq \|\rho\|_1\left(\|(N+\1)^{l/2}H(N+\1)^{-3/2l+1}\|_\infty+\|(N+\1)^{-3/2l+1}H(N+\1)^{l/2}\|_\infty\right)\\
    %         &\leq 2\|\rho\|_1\Lambda d_H^2\sqrt{d_H!}\,,
    %     \end{aligned}
    % \end{equation*}
    % where $\Lambda$ bounds the maximal absolute coefficient of the polynomial. Note that the polynomial is of the form in \Cref{eq:ccr-polynomial-representation}. Similarly, one can show 
    % \begin{equation*}
    %     \begin{aligned}
    %         \|\cH[H](Y))\|_{1}\leq 2\|\rho\|_1\Lambda d_H^2\sqrt{d_H!}
    %     \end{aligned}
    % \end{equation*}
    % where $Y\coloneqq\cW^{-4(l-1)}(\rho)$. Moreover, we established in \Cref{lem:l-diss-hamiltonian} the existence of a constant $\gamma_\varepsilon\geq0$ such that
    % \begin{equation*}
    %     \|e^{t(\cL_l+\varepsilon\cH[H])}(\rho)\|_{W^{k,1}}\leq \max\left\{\gamma_\varepsilon,\,\|\rho\|_{W^{k,1}}\right\}\,,
    % \end{equation*}
    % where
    % \begin{equation*}
    %     \gamma_\varepsilon=\tilde{c}^\nu\left(\frac{(\nu-1)^{\nu-1}}{\nu^\nu}\right)\quad\text{with}\quad \tilde{c}={(l+1)l}+2|\alpha|^lkl^{k/2}\sqrt{l!}+\varepsilon\Lambda(2l)^{\blue{k/2}}\sqrt{(2l)!}\,,\quad\nu=l+\frac{k}{2}-1\,.
    % \end{equation*}
    % Using these two results, we can show that there exists a constant $c_1>0$ such that
    % \begin{equation*}
    %     \begin{aligned}
    %         \|e^{s\cL_l}\cH[H]e^{(t-s)(\cL_l+\varepsilon\cH[H])}(\rho)\|_1&\leq\|\cH[H]\cW^{-4(l-1)}\cW^{4(l-1)}e^{(t-s)(\cL_l+\varepsilon\cH[H])}(\rho)\|_1\\
    %         &\leq c_1\|\cW^{4(l-1)}e^{(t-s)(\cL+\varepsilon\cH[H])}(\rho)\|_1\\
    %         &\leq c_1\max\{\gamma_\varepsilon,\|\rho\|_{W^{4(l-1),1}}\}\,.
    %     \end{aligned}
    % \end{equation*}
    % Next, we bound 
    % \blue{
    % \begin{equation}\label{eq:L-bound}
    %     \begin{aligned}
    %         \|L(N+1)^{-l - 2}L^\dagger\|_1\leq (1+|\alpha|^l\,(l+1)\,\sqrt{l!}+|\alpha|^{2l}) \frac{\pi^2}{6}
    %     \end{aligned}
    % \end{equation}
    % }
    % which follows from
    % \begin{equation*}
    %     \begin{aligned}
    %         L^\dagger L&=(\ad)^l a^l-\overline{\alpha}^la^l-\alpha^l(\ad)^l+|\alpha|^{2l}\\
    %         &\leq(N-(l-1)\1)\cdots(N-\1)N+|\alpha|^l\,(l+1)\,\sqrt{(N+l\1)\cdots (N+\1)}+|\alpha|^{2l}\\
    %         &\leq (N+\1)^l+|\alpha|^l\,(l+1)\,\sqrt{l!}\,(N+\1)^{l/2}+|\alpha|^{2l}\,.
    %     \end{aligned}
    % \end{equation*}
    % \blue{
    % and the fact that $\norm{(N + \1)^{-2}}_1 = \frac{\pi^2}{6}$.
    % }
    % This shows that the bounded vector-valued functions in $s$ considered below are continuous so that the following Bochner integral is well-defined.
    % \begin{equation*}
    %     \begin{aligned}
    %         &\tr[L\biggl(e^{t\cL_l}(\rho)-e^{t(\cL_l+\varepsilon\cH[H])}(\rho)\biggr)L^\dagger]\\
    %         &\qquad=\varepsilon\tr[L\int_{0}^te^{s\cL_l}\cW^{-2(\blue + 1)}\cW^{2(l + 1)}\cH{[H]}e^{(t-s)(\cL_l+\varepsilon\cH[H])}(\rho)dsL^\dagger]\\
    %         &\qquad\overset{(1)}{\leq}\varepsilon\tr[L\int_{0}^te^{s\cL_l}\cW^{\blue{-2(l + 1)}}(\1)dsL^\dagger]2\Lambda d_H^2\sqrt{d_H!}\max\left\{\gamma_\varepsilon,\,\|\rho\|_{W^{\blue{6l},1}}\right\}\\
    %         &\qquad\overset{(2)}{\leq}\varepsilon2\Lambda d_H^2\sqrt{d_H!}(1+|\alpha|^l\,(l+1)\,\sqrt{l!}+|\alpha|^{2l})\frac{1}{l!}(1-e^{-l!t})\max\left\{\gamma_\varepsilon,\,\|\rho\|_{W^{6l-4,1}}\right\}\\
    %         &\qquad\eqqcolon\varepsilon c(1-e^{-l!t})\max\left\{\gamma_\varepsilon,\,\|\rho\|_{W^{\blue{6l},1}}\right\}
    %     \end{aligned}
    % \end{equation*}
    % where in $(1)$ we have used that $e^{s\cL_l}$ and $L\cdot L^\dagger$ are positivity preserving as well as the following bound 
    % \begin{equation*}
    %     \begin{aligned}
    %         \cW^{2l}\cH[H]e^{(t-s)(\cL_l+\varepsilon\cH[H])}(\rho)&\leq\|\cW^{2l}\cH[H]e^{(t-s)(\cL_l+\varepsilon\cH[H])}(\rho)\|_\infty\1\\
    %         &\leq\|\cW^{2l}\cH[H]e^{(t-s)(\cL_l+\varepsilon\cH[H])}(\rho)\|_1\1\\
    %         &\leq 2\Lambda d_H^2\sqrt{d_H!}\max\left\{\gamma_\varepsilon,\,\|\rho\|_{W^{6l-4,1}}\right\}\1
    %     \end{aligned}
    % \end{equation*}
    % holds. In $(2)$, we pull the closed operator $L\cdot L^\dagger$ inside the Bochner integral, used the exponential convergence in \Cref{eq:exponential-convergence}, and finally bounded the remaining term by \Cref{eq:L-bound}. Following the same steps as above with $\cH[H]$ replaced by $-\cH[H]$ leads to the result.
\end{proof}

Special cases of the above result include the $X$ or $Z(\theta)$ gate. 
% The explicit structure and small number $l\in\N$ allow us to improve the perturbation bound in the following way:

\begin{cor}[$Z(\theta)$-gate]
    Let $\cL_2$ be the $2$-photon dissipation. Then, for all $\varepsilon\in[0,1]$ and all states $\rho\in W^{10,1}$
    \begin{equation*}
        \begin{aligned}
            \Big|\tr[L\left(e^{t\cL_2}(\rho)-e^{t(\cL_2+\varepsilon\cH[a+\ad])}(\rho)\right)L^\dagger]\Big|&\leq \varepsilon 2(1+6|\alpha|^2+|\alpha|^{4})(1-e^{-2t})\max\left\{\gamma_\varepsilon,\|\rho\|_{W^{10,1}}\right\}
        \end{aligned}
    \end{equation*}
    where $\gamma_\varepsilon=\frac{1}{25}\left(6+\sqrt{2}\,2^6\,5\,|\alpha|^2+\varepsilon4^5\sqrt{24}\right)^6$\,.
\end{cor}
% \begin{proof}
%     We follow a similar strategy as with the proof in \Cref{thm:l-diss-hamiltonian-perturbation}.
%     As in the proof of \Cref{thm:l-diss-hamiltonian-perturbation}, we first prove the bound 
%     \begin{equation*}
%         \|\cW^4\cH[a+\ad]\cW^{-6}\|_{1\rightarrow1}\leq 16<\infty\,.
%     \end{equation*}
%     Indeed, by the triangle and Hölder's inequalities,
%     \begin{equation*}
%         \begin{aligned}
%             \|\cW^4&\cH[a+\ad]\cW^{-6}(\rho)\|_{1}\\
%             &\leq\|(N+\1)(a+\ad)(N+\1)^{-3/2}\rho(N+\1)^{-1/2}\|_1\\
%             &\quad+\|(N+\1)^{-1/2}\rho(N+\1)^{-3/2}(a+\ad)(N+\1)\|_1\\
%             &\leq 2\|\rho\|_1\left(\|(N+\1)(a+\ad)(N+\1)^{-3/2}\|_\infty+\|(N+\1)^{-3/2}(a+\ad)(N+\1)\|_\infty\right)\\
%             &\leq 16\|\rho\|_1\,,
%         \end{aligned}
%     \end{equation*}
%     which can be seen by applying a vector in the Fock basis representation to calculate the infinity norms. Moreover, applying \Cref{lem:z-theta-energetic-stab} for $k=6$,
%     \begin{equation*}
%         \|e^{t(\cL_2+\varepsilon\cH[a+\ad])}(\rho)\|_{W^{6,1}}\leq\max\left\{\frac{3^7}{4^2}\left(1+\sqrt{2}2^4\,|\alpha|^2+\varepsilon\blue{24}\right)^4,\|\rho\|_{W^{6,1}}\right\}\,.
%     \end{equation*}
%     By following the proof strategy of \Cref{thm:l-diss-hamiltonian-perturbation}, the following Bochner integral is well-defined and
%     \begin{equation*}
%         \begin{aligned}
%             &\tr[L\biggl(e^{t\cL_2}(\rho)-e^{t(\cL_2+\varepsilon\cH[a+\ad])}(\rho)\biggr)L^\dagger]\\
%             &\qquad = \varepsilon\tr[L\int_0^te^{s\cL_2}\cW^{-4}\cW^{4}\cH{[a+\ad]}\cW^{-6} \cW^{6}e^{t(\cL_2+\varepsilon\cH[a+\ad])}(\rho)ds L^\dagger]\\
%             &\red{\qquad\leq\varepsilon\tr[\int_0^tLe^{s\cL_2}\cW^{-4}(\1)L^\dagger ds]16\max\left\{\frac{3^7}{4^2}\left(1+\sqrt{2}2^4\,|\alpha|^2+\varepsilon\blue{24}\right)^4,\|\rho\|_{W^{6,1}}\right\}}\\
%             &\qquad\leq\varepsilon8(1+6|\alpha|^2+|\alpha|^{4})(1-e^{-2t})\max\left\{\frac{3^7}{4^2}\left(1+\sqrt{2}2^4\,|\alpha|^2+\varepsilon\blue{24}\right)^4,\|\rho\|_{W^{6,1}}\right\}
%         \end{aligned}
%     \end{equation*}
%     finishes the proof after repeating the argument for $-\cH[H]$ in order to get a bound on the absolute value of the trace above. 
% \end{proof}

\subsection{Application: entropic and capacity continuity bounds}

Here, we provide one basic application to the perturbation bounds found in this section. We recall the definition of the energy-constrained diamond norm:

\begin{defi}[see \cite{Shirokov.2018,winter2017energy}]
    Given $E\ge 0$ and any two completely positive, trace-preserving maps $\cN,\cM:\cT_1(\cH_m)\to\cT_1(\cH_m)$, their energy constrained diamond norm distance is defined as
    \begin{align}\label{ECnorm}
        \|\cN-\cM\|_\diamond^E\coloneqq \sup_{\cH_r}\,\sup_{\substack{\rho\in \cD(\cH_m\otimes \cH_r)\\\tr\big[\rho_{m}N_m\big]\le E}}\,\big\|(\cN-\cM)\otimes\id_R(\rho)\big\|_1\,,
    \end{align}
    where $N_m\coloneqq \sum_{i=1}^m\,a_i^\dagger a_i$ denotes the total photon number operator, and where the supremum is over all bipartite states $\rho_{mr}$ on $\cH_\otimes \cH_r$ with reduced state $\rho_m$ on $\cH_m$ of average total photon number at most $E$, for some arbitrary separable Hilbert space $\cH_r$. 
\end{defi}

Not that we denote the set of quantum states over a separable Hilbert space as $\cD(\cH) := \{\rho \in \cT_{1, \operatorname{sa}}(\cH) \;:\; \rho \ge 0, \tr[\rho] = 1\}$ in this section.
In words, the energy-constrained diamond norm is a measure of distinguishability between quantum channels with entanglement assistance, and where the input states used for this task are restricted to a physically relevant set of energy-limited states. By the same reasoning as for the usual diamond norm, the supremum in \eqref{ECnorm} can be restricted to $\cH_r\cong \cH_m$, and the optimization can be restricted to pure states on $\cH_m\otimes \cH_r$. Moreover, it turns out that the above definition is equivalent to one where the input state is energy limited on both $m$ and $r$, as introduced in \cite{pirandola2017fundamental}:
\begin{align*}
    \vertiii{\cN-\cM}_\diamond^E\coloneqq \sup_{\substack{\rho\in \cD(\cH_m\otimes \cH_{m'})\\ \tr\big[\rho(N_m+N_{m'})\big]\le E}}\,\big\|(\cN-\cM)\otimes\id_{m'}(\rho)\big\|_1\,,
\end{align*}
for $\cH_{m'}\cong \cH_m$, and where the difference with \eqref{ECnorm} lies in the input states in the above optimization are energy constrained in both their inputs. Clearly, from the definitions we have that 
\begin{align}\label{tripledoubleECequiv}
    \vertiii{\cN-\cM}_\diamond^E\le \|\cN-\cM\|_\diamond^E\le \vertiii{\cN-\cM}_\diamond^{2E}\,.
\end{align}
The second bound above simply results from taking a state $\rho_{mm'}$ for which $\tr\big[\rho_mN_m\big]\le E$ and unitarily rotating the second register so that $\rho_m'=\rho_m$, and therefore $\tr\big[\rho_{m'}N_{m'}\big]\le E$, too.

Similarly, in the next lemma, we establish a straightforward connection between the energy-constrained diamond norm distance between two channels and certain norms between Sobolev spaces:
\begin{lem}\label{lemmaequivECW}
    For any two completely positive, trace preserving maps $\cN,\cM:\cT_1(\cH_1)\to \cT_1(\cH_1)$ and $E\ge 0$,
    \begin{align*}
        \|\cN-\cM\|_{\diamond}^E= (1+E)\,\sup_{\rho\in\cD}\,\frac{\|(\cN-\cM)\otimes \operatorname{id}_{M'}(\rho)\|_1}{\|\rho\|_{W^{(2,0),1}}}\,,
    \end{align*}
    where $\cD\coloneqq \cD(\cH_m\otimes \cH_{m'})$ above. Similar identities hold in multi-mode settings.
\end{lem}
\begin{proof}
    This is direct from
    \begin{align*}
        \|{\cN-\cM}\|_\diamond^{E}&=\sup_{\substack{\rho \in \cD\\\tr[\rho N]\le E}}\,\|(\cN-\cM)\otimes\id(\rho)\|_1\\
        &= \sup_{\substack{\rho \in \cD\\\|\rho\|_{W^{(2,0),1}}\le 1+E}}\,\|(\cN-\cM)\otimes\id(\rho)\|_1\,.
    \end{align*}
\end{proof}

\Cref{lemmaequivECW} can be used in combination with perturbation bounds and entropic continuity bounds like those derived e.g.~in \cite{winter2016tight,winter2017energy,shirokov2017tight,Shirokov.2018} in order to control the deviation of energy-constrained channel capacities in presence of a Lindbladian perturbation. Since these considerations go beyond the scope of the present paper, we do not pursue them further. Instead, we provide a basic illustration of the method we propose in the case of Gaussian perturbations of Gaussian semigroups by combining \Cref{propqOUperturb} with \Cref{lemmaequivECW}.

\begin{cor}\label{corECDN}
    Let $(\cL_{\operatorname{qOU}},\cT_f)$ be the generator of the quantum Ornstein Uhlenbeck semigroup with $\lambda>\mu\geq0$ and $(\varepsilon\cL_G,\cT_f)\coloneqq (\varepsilon\cL[{\gamma a+\eta\ad}],\cT_f)$ a Gaussian perturbation with $\gamma,\eta\in\mathbb{R}$, $\varepsilon>0$. Then, assuming $\lambda^2-\mu^2+|\gamma|^2-|\eta|^2> 0$ as in \Cref{propqOUperturb}, there exist uniformly bounded functions $C(\varepsilon)$, $D(\varepsilon)$ such that, for all $t\ge 0$:
    \begin{equation}
        \Big\|e^{t \cL_{\operatorname{qOU}}}-e^{t(\cL_{\operatorname{qOU}}+\varepsilon\cL_G)}\Big\|_{\diamond}^E\leq\,(1+E) \varepsilon\, C(\varepsilon)\, \max\Big\{1,D(\varepsilon)\Big\}\,.
    \end{equation}
\end{cor}