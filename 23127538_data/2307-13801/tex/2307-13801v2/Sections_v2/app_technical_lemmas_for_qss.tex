\begin{lem}[Continuity of $G(z)$]\label{lem:continuity-G}
    Let $k_0 < k_1 \in \R_+$ and $T: W^{k_j, 1} \to W^{k_j, 1}$, be a linear map with $\norm{T}_{W^{k_j, 1} \to W^{k_j, 1}} \le M_j$, bounded by $M_j \ge 0$ for $j = 1, 2$ respectively. Further let $\theta \in [0, 1]$ and $k_\theta = (1-\theta) k_0 +  \theta k_1$ and $x \in \cT_f$, then the map
    \begin{align*}
        G: S &\coloneqq \{ z \in \C \::\; 0 \le \Re(z) \le 1\} \to \cT_{1, \operatorname{sa}} \\
        z & \mapsto G(z) = (N + \1)^{\frac{k(z)}{4}} T\Big((N + \1)^{\frac{k_\theta - k(z)}{4}} x (N + \1)^{\frac{k_\theta - k(z)}{4}}\Big) (N + \1)^{\frac{k(z)}{4}}
    \end{align*}
    with $k(z) = (1-z) k_0 + z k_1$, is well-defined, uniformly bounded and continuous.
\end{lem}
\begin{proof}
   In order to prove the claim, we decompose $G$ using the following auxiliary functions:
    \begin{equation}\label{eq:G_1}
        \begin{aligned}
             G_1: S \times W^{k_1, 1} &\to \cT_{1, \operatorname{sa}}\\
            (z, y) &\mapsto (N + \1)^{\frac{k(z)}{4}} y (N + \1)^{\frac{k(z)}{4}}
        \end{aligned}
    \end{equation}
    and 
    \begin{equation}\label{eq:G_2}
        \begin{aligned}
               G_2: S &\to \cT_f \subset W^{k_1, 1}\\
               z &\mapsto (N + \1)^{\frac{k_\theta - k(z)}{4}} x (N + \1)^{\frac{k_\theta - k(z)}{4}} \, . 
        \end{aligned}
    \end{equation}
    We clearly have that $G_1(z, \cdot):W^{k_1, 1} \to \cT_{1, \operatorname{sa}}$ is a bounded linear map for all $z \in S$, since
    \begin{equation}\label{eq:bound-G_1}
        \norm{G_1(z, y)}_1 = \norm{(N + \1)^{\frac{\Re(k(z))}{4}} y (N + \1)^{\frac{\Re(k(z))}{4}}}_1 \le \norm{(N + \1)^{\frac{k_1}{4}} y (N + \1)^{\frac{k_1}{4}}}_1 = \norm{y}_{W^{k_1, 1}}
    \end{equation}
    where we used that $k_0 \le \Re(k(z)) \le k_1$ and $(N + \1)^{i\frac{\Im(k(z))}{4}}$ is a unitary that can be absorbed into the norm. Next, we will show that $G_1(\cdot, y): S \to \cT_{1, \operatorname{sa}}$ is continuous for all $y \in W^{k_1, 1}$. For that first note that, for $y \in \cT_f$, the claim follows directly from the continuity of $z \mapsto (n + 1)^{\frac{k(z)}{4}}$ with $n \in \N$ as a map from $S$ to $\C$. This is because all the involved operators can be considered finite dimensional using a cut-off of the Fock-basis. For a general $y \in W^{k_1, 1}$, we find $(y_n)_{n \in \N} \subset \cT_f$, s.t. $y_n \to y$ in $W^{k_1, 1}$, hence for all $n \in \N$
    \begin{align*}
        \lim\limits_{z \to z_0} &\norm{G_1(z, y) - G_1(z_0, y)}_1 \\
        &\le \lim\limits_{z \to z_0} \norm{G_1(z, y - y_n)}_1 + \norm{G_1(z, y_n) - G_1(z_0, y_n)}_1 + \norm{G_1(z_0, y_n - y)}_1\\
        &\le \lim\limits_{z \to z_0} \norm{G_1(z, y_n) - G_1(z_0, y_n)}_1 + 2 \norm{y - y_n}_{W^{k_1, 1}}\\
        &\le 2 \norm{y - y_n}_{W^{k_1, 1}} \, , 
    \end{align*}
    where we used \Cref{eq:bound-G_1}. Taking the limit $n \to \infty$ concludes the claim that $G_1(\cdot, y): S \to \cT_{1, \operatorname{sa}}$ is continuous for all $y \in W^{k_1, 1}$. We further have that $G_2$ as a map from $S$ to $W^{k_1, 1}$ is continuous, since $x \in \cT_f$ and the maps $z \mapsto (n + 1)^{\frac{k(z)}{4}}$ for $n \in \N$ are continuous as maps $S \to \C$. This suffices since $x \in \cT_f$, hence all involved operators can be made finite dimensional via a cut-off in the Fock-basis again.\par 
    We can now write 
    \begin{equation*}
        G(z) = G_1(z, T(G_2(z)))
    \end{equation*}
    where $T(G_2(z)) \in W^{k_1, 1}$ as $T:W^{k_1, 1} \to W^{k_1, 1}$ and $G_2(z) \in \cT_f \subset W^{k_1, 1}$ for all $z \in S$. This not only gives us that $G$ is well-defined but also allows us to get
    \begin{align*}
        \norm{G(z)}_1 &= \norm{G_1\left(z, T(G_2(z))\right)}_1 \\
        &\le \norm{T(G_2(z))}_{W^{k_1, 1}} \\
        &\le \norm{T}_{W^{k_1, 1} \to W^{k_1, 1}} \norm{G_2(z)}_{W^{k_1, 1}}\\
        &\le \norm{T}_{W^{k_1, 1} \to W^{k_1, 1}} \norm{(N + \1)^{\frac{k_\theta - k_0}{4}} x (N + \1)^{\frac{k_\theta - k_0}{4}}}_{W^{k_1, 1}}
    \end{align*}
    where we again used \Cref{eq:bound-G_1}, giving us a bound independent of $z$. Further, using again \Cref{eq:bound-G_1} we can conclude continuity, since 
    \begin{align*}
        \lim\limits_{z \to z_0} \norm{G(z) - G(z_0)}_1 &\le \lim\limits_{z \to z_0} \norm{G_1\left(z, T\left\{G_2(z) - G_2(z_0)\right\}\right)}_1 \\
        &\hspace{2cm} + \lim\limits_{z \to z_0}\norm{G_1\left(z, T\left(G_2(z_0)\right)\right) - G_1\left(z_0, T\left(G_2(z_0)\right)\right)}_1\\
        &\le \lim\limits_{z \to z_0}\norm{T}_{W^{k_1, 1} \to W^{k_1, 1}} \norm{G_2(z) - G_2(z_0)}_{W^{k_1, 1}} \\
        & \hspace{2cm} + \lim\limits_{z \to z_0} \norm{G_1\left(z, T\left(G_2(z_0)\right)\right) - G_1\left(z_0, T\left(G_2(z_0)\right)\right)}_1\\
        &= 0
    \end{align*}
    where in addition we used the continuity of $G_1(\cdot, y): S \to \cT_{1, \operatorname{sa}}$ for $y \in W^{k_1, 1}$ and $G_2:S \mapsto W^{k_1, 1}$.
\end{proof}

\begin{lem}[Differentiability of G(z)]\label{lem:differentiability-G}
    Let $$k_0 < k_1 \in \R_+$$, $T: W^{k_j, 1} \to W^{k_j, 1}$, be a linear map with $\norm{T}_{W^{k_j, 1} \to W^{k_j, 1}} \le M_j$, bounded by $M_j \ge 0$ for $j = 1, 2$ respectively. Further let $\theta \in [0, 1]$ and $k_\theta = (1-\theta) k_0 +  \theta k_1$ and $x \in \cT_f$, then the map
    \begin{align*}
        G: S &\coloneqq \{ z \in \C \::\; 0 \le \Re(z) \le 1\} \to \cT_{1, \operatorname{sa}} \\
        z & \mapsto G(z) = (N + \1)^{\frac{k(z)}{4}} T\left((N + \1)^{\frac{k_\theta - k(z)}{4}} x (N + \1)^{\frac{k_\theta - k(z)}{4}}\right) (N + \1)^{\frac{k(z)}{4}}
    \end{align*}
    with $k(z) = (1-z) k_0 + z k_1$, is holomorphic on $\mathring{S} \coloneqq \{z \in \C \; : \; 0 < \Re(z) < 1\}$.
\end{lem}
\begin{proof}
    To prove the claim, we follow a similar strategy as with \Cref{lem:continuity-G}. We will again use the auxiliary functions \Cref{eq:G_1} and \Cref{eq:G_2}. We begin by showing that for a fixed $y \in W^{k, 1}$, $G_1(\cdot, y):S \to \cT_{1, \operatorname{sa}}$ is holomorphic on $\mathring{S}$ and initially even simplify to the case $y \in \cT_f$. In this setting, all operators involved can be assumed to be linear maps on a finite subspace, by just taking a cut-off in the Fock-basis as we did before. This allows us to Taylor expand around $z_0 \in \mathring{S}$
    \begin{align*}
        (N + \1)^{\frac{k(z)}{4}} y (N + \1)^{\frac{k(z)}{4}} = G_1(z_0, y) +  G_1'(z_0, y) (z - z_0) + \int\limits_{[z_0, z]} G_1''(\omega, y)(\omega - z_0) \,  d\omega
    \end{align*}
    where the integral is a path integral along the line segment $[z_0, z]$ and
    \begin{align*}
         G_1'(z_0, y) &= \frac{k_0 - k_1}{4}\, \left(\log(N + \1) (N + \1)^{\frac{k(z_0)}{4}} y (N + \1)^{\frac{k(z_0)}{4}}\right. \\
        &\hspace{3cm} \left. + (N + \1)^{\frac{k(z_0)}{4}} y (N + \1)^{\frac{k(z_0)}{4}} \log(N + \1)\right)
    \end{align*}
    and 
    \begin{align*}
        G_1''(\omega, y) &= \left(\frac{k_0 - k_1}{4}\right)^2\left(\log^2(N + \1)(N + \1)^{\frac{k(\omega)}{4}} y (N + \1)^{\frac{k(\omega)}{4}}\right.\\
        &\hspace{3cm} + 2 \log(N + \1)(N + \1)^{\frac{k(\omega)}{4}} y (N + \1)^{\frac{k(\omega)}{4}} \log(N + \1) \\
        &\hspace{6cm} +\left. (N + \1)^{\frac{k(\omega)}{4}} y (N + \1)^{\frac{k(\omega)}{4}} \log^2(N + \1)\right)
    \end{align*}
    are linear in $y$. From this representation, we can immediately deduce holomorphy of $G_1(\cdot, y):S \to \cT_{1, \operatorname{sa}}$ at $z_0 \in \mathring{S}$ and hence on all of $\mathring{S}$. To lift holomorphy from $y \in \cT_f$ to $y \in W^{k_1, 1}$, we note that for $z_0 \in \mathring{S}$ there exists $C_{z_0} \ge 0$ such that for $y \in \cT_f$
    \begin{equation}\label{eq:bound-G'1}
        \norm{G'(z_0, y)}_1 \le C_{z_0} \norm{y}_{W^{k_1, 1}}
    \end{equation}
    and further for $\omega \in B_\varepsilon(z_0) \coloneqq \{z \in \C \;:\; |z - z_0| < \varepsilon\} \subset \mathring{S}$ there exists $C_{\varepsilon, z_0} \ge 0$ such that
    \begin{equation}\label{eq:bound-G''1}
        \norm{G''(\omega, y)}_1 \le C_{\varepsilon, z_0} \norm{y}_{W^{k_1, 1}} \, . 
    \end{equation}
    We will only show that given $\omega$ as above, 
    \begin{equation}\label{eq:boundedness-subterms}
        \norm{\log^2(N + \1) (N + \1)^{\frac{k(\omega)}{4}} y (N + \1)^{\frac{k(\omega)}{4}} y (N + \1)^{\frac{k(\omega)}{4}}}_1 \le \tilde{C}_{\varepsilon, z_0} \norm{y}_{W^{k_1, 1}} \, . 
    \end{equation}
    Using the same reasoning for the other terms of \Cref{eq:bound-G'1} and \Cref{eq:bound-G''1} in combination with triangle inequality immediately gives the claims. Note first that we can reduce $k(\omega)$ to its real part since the imaginary part only produces a unitary $(N + \1)^{i \Im(k(\omega))}$ that can be absorbed into the norm. We call the real part $r(\omega)$ for now. Since $\omega \in B_\varepsilon(z_0) \subset \mathring{S}$ we find a $\delta_\varepsilon > 0$ independent of $\omega$, such that $|r(\omega) - k_1| < \delta_\varepsilon$ or more precisely $r(\omega) - k_1 \le - \delta_\varepsilon$. Hence using Hölder's inequality, we can deduce
    \begin{align*}
         &\norm{\log^2(N + \1) (N + \1)^{\frac{k(\omega)}{4}} y (N + \1)^{\frac{k(\omega)}{4}} y (N + \1)^{\frac{k(\omega)}{4}}}_1\\
         &\le \norm{\log^2(N + \1) (N + \1)^{\frac{r(\omega) - k_1}{4}}}_\infty \norm{(N + \1)^{\frac{r(\omega) - k_1}{4}}}_\infty \norm{y}_{W^{k_1, 1}}\\
         &\le \norm{\log^2(N + \1) (N + \1)^{-\frac{\delta_\varepsilon}{4}}}_\infty \norm{(N + \1)^{-\frac{\delta_\varepsilon}{4}}}_\infty \norm{y}_{W^{k_1, 1}}\\
         &\le \norm{\log^2(N + \1) (N + \1)^{-\frac{\delta_\varepsilon}{4}}}_\infty \norm{y}_{W^{k_1, 1}}
    \end{align*}
    where we used that $x \mapsto e^{k x}$ for $k \ge 0$ is monotone and further that $(N + \1)^{-\frac{\delta_\varepsilon}{4}}$ is a contraction. Lastly, we have that $x \mapsto \frac{\log^2(x + 1)}{(x + 1)^{\frac{\delta_\varepsilon}{4}}}$ is a bounded function for $x \ge 0$ with a bound we call $\tilde{C}_{\delta_\varepsilon}$. This allows us to estimate $\norm{\log^2(N + \1) (N + \1)^{-\frac{\delta_\varepsilon}{4}}}_\infty \le \tilde C_{\delta_\varepsilon}$, which concludes \Cref{eq:boundedness-subterms} and therefore also \Cref{eq:bound-G'1} and \Cref{eq:bound-G''1}.\par 
    For a general $y \in W^{k_1, 1}$ and $z_0 \in \mathring{S}$ \Cref{eq:bound-G'1} allows us to conclude that $G'_1(z_0, y) \in \cT_{1, \operatorname{sa}}$ is well defined. Further, for $z \in B_\varepsilon(z_0)$ and $(y_n)_{n \in \N} \subset \cT_f$ with $y_n \to y$ in $W^{k_1, 1}$, we have for all $n \in \N$
    \begin{equation}
        \begin{aligned}
            \norm{\frac{G_1(z, y_n) - G_1(z_0, y_n)}{z - z_0} - G_1'(z_0, y_n)}_1 &\le \frac{1}{|z - z_0|} \int\limits_{[z_0, z]} \norm{G''_1(\omega, y_n)}_1 |(\omega - z_0) d\omega| \\
            &\le C_{\varepsilon, z_0} |z - z_0| \norm{y_n}_{W^{k_1, 1}}
        \end{aligned}
    \end{equation}
    where we used the expansion and \Cref{eq:bound-G''1}. Now we can take the limit $n \to \infty$ on both sides, as all objects involved are stable w.r.t. that limit (using \Cref{lem:continuity-G} and \Cref{eq:bound-G'1}). We get
    \begin{equation}
        \norm{\frac{G_1(z, y) - G_1(z_0, y)}{z - z_0} - G_1'(z_0, y)}_1 \le  C_{\varepsilon, z_0} |z - z_0| \norm{y}_{W^{k_1, 1}}
    \end{equation}
    which immediately lets us deduce holomorphy of $G_1(\cdot, y):S \to \cT_{1, \operatorname{sa}}$ on $\mathring{S}$ for $y \in W^{k_1, 1}$.\par
    For $G_2:S \to W^{k_1, 1}$ the holomorphy immediately follows from the fact that $x \in \cT_f$, which again allows reducing the analysis to a finite-dimensional subspace by taking a cut-off in the Fock basis again. Lastly, we have that $T(G_2(z)) \in W^{k_1, 1}$ for all $z \in S$, which finally gives us that for $z_0 \in \mathring{S}$ and for $z \in B_\varepsilon(z_0) \subset \mathring{S}$
    \begin{align*}
        &\norm{\frac{G(z) - G(z_0)}{z - z_0} - (G_1'(z_0, T(G_2(z_0))) + G_1(z_0, T\{G_2'(z_0)\})} \\
        &\le \norm{\frac{G_1(z, T(G_2(z_0))) - G_1(z_0, T(G_2(z_0)))}{z - z_0} - G_1'(z_0, T(G_2(z_0)))}_1 \\
        &\hspace{1cm} + \norm{T}_{W^{k_1, 1} \to W^{k_1, 1}}\norm{\frac{G_2(z) - G_2(z_0)}{z - z_0} - G'_2(z_0)}_{W^{k_1, 1}}
    \end{align*}
    where we used linearity of $G_1(z, \cdot)$, $G_1'(z, \cdot)$ and $T$. In addition, we used the bound on $G_1(z, \cdot)$ from \Cref{eq:bound-G_1} and $G'_2$ to denote the derivative of $G_2$. Now the differentiability of $G_1(\cdot, y)$ and $G_2$ at $z_0$ immediately gives the differentiability of $G$ at $z_0$, which concludes the proof as $z_0 \in \mathring{S}$ was arbitrary.
\end{proof}