A quantum evolution in bosonic systems is described by a master equation 
\begin{equation}\label{eq:mastereq}
    \frac{d}{dt} x(t) = \cL(x(t)) \quad x(0) \in \cD(\cL) \quad\text{and}\quad t \ge 0 \, . 
\end{equation}
where $\cL$ is potentially unbounded. In the following, we state two sufficient assumptions for the existence and uniqueness of an operator-valued solution to \eqref{eq:mastereq} in terms of a semigroup. In other words, we prove a generation theorem for bosonic quantum Markov semigroups. This is generalized in \Cref{sec:timedependentgeneration} to the case of time-dependent generators. 

\subsection{Strongly continuous bosonic semigroups}\label{subsec:time-indep-generation}

We start with the time-independent setting, for which we need two working assumptions. The first assumption is motivated by the so-called GKSL \cite{Lindblad.1976,Gorini1976} form that generators of quantum dynamical semigroups over finite-dimensional quantum systems take, as well as our natural choice to consider jump and Hamiltonian operators described by polynomials in the annihilation and creation operators:

\begin{assum}\label{assum:finite-degree}
    The operator $(\cL,\cT_f)$ has $\operatorname{GKSL}$ form, i.e.~for $x\in\cT_f$ 
    \begin{equation}\label{eq:lindblad}
        \begin{aligned}
            \cL:\cT_f \to \cT_f \quad x \mapsto\cL(x) &= - i [H, x] + \sum\limits_{j = 1}^K L_j x L_j^\dagger  - \frac{1}{2}\{L_j^\dagger L_j, x\} \\
            &\coloneqq Gx + x G^\dagger + \sum\limits_{j = 1}^K L_j x L_j^\dagger\, , 
        \end{aligned}
    \end{equation}
    for some $K\in\mathbb{N}$ and with $G=-iH-\frac{1}{2}\sum_{j=1}^K L_j^\dagger L_j$, where $\{A,B\}=AB+BA$ denotes the anticommutator of two operators $A,B$ on a suitable domain. For the above equation to make sense, the operators $H$ and $L_j$ are assumed to be polynomials of the creation and annihilation operators, i.e.~$H\coloneqq p_H(a,\ad)$ and $L_j\coloneqq p_j(a,\ad)$, and $H$ is assumed to be symmetric. This ensures that $\cT_f$ is invariant under $\cL$.
   We denote the degree of $p_H$ by $d_H\coloneqq\deg(p_H)$, those of $p_j$ by $d_j\coloneqq\deg(p_j)$, and $d\coloneqq\max\{d_1,...,d_K,d_H\}$.
\end{assum}

The second assumption will lead to the semigroup being Sobolev preserving, which allows us not only to prove the existence and uniqueness of the evolution generated by \eqref{eq:mastereq} but further to conduct a perturbation analysis as well as to extend our results to the case of a time-dependent Master equation:

\begin{assum}\label{assum:sobolev-stability}
    There exists a non-negative sequence $\{k_r\}_{r \in \N}\rightarrow\infty$ s.t.~for all $r \in \N$ there exist $\omega_{k_r} \ge 0$ such that for all positive semi-definite $x \in \cT_f$
    \begin{equation}\label{eq:Assumptionsobolevstability}
        \tr[\cL(x) (N + \1)^{k_r/2}] \le \omega_{k_r} \tr[x (N + \1)^{k_r/2}] \, .
    \end{equation}
\end{assum}

We are now ready to state and prove the main theorem of the section: 
\begin{thm}[Generation of bosonic semigroups]\label{thm:generation-theorem}
    Let $(\cL, \cD(\cL))$ be an operator defined on $\cT_{1,\operatorname{sa}}$. If $(\cL, \cD(\cL))$ satisfies \Cref{assum:finite-degree} and \Cref{assum:sobolev-stability}, then the closure $\overline{\cL}$ generates a strongly continuous, positivity preserving semigroup $(\cP_t)_{t\ge 0}$ on $W^{k, 1}$ for all $k\geq 0$ with 
    \begin{equation}
        \norm{\cP_t}_{W^{k, 1} \to W^{k, 1}} \le e^{\omega_k t} \,\quad \forall t\ge 0\, . 
    \end{equation}
   where $\omega_k = \frac{k_{r_1} - k}{k_{r_1} - k_{r_0}}\omega_{k_{r_0}} + \frac{k - k_{r_0}}{k_{r_1} - k_{r_0}}\omega_{k_{r_1}}$ for an $r$ such that $k_{r_0}\leq k <k_{r_1}$. Finally, for $k = 0$, the semigroup is contractive and trace-preserving.
\end{thm}

Before proving \Cref{thm:generation-theorem}, we provide two examples for which \Cref{assum:sobolev-stability} is not satisfied.

\begin{ex}[Pure birth process \texorpdfstring{\cite[Ex.~3.3.]{Davies.1977}}{[9,Ex.~3.3.]}]\label{eq:no-sobolev-stability}
    Let $L=(\ad)^2$ and $G=-\frac{1}{2}a^2(\ad)^2$, i.e.~
    \begin{equation*}
        \cL(x)=Gx+x G^\dagger+Lx L^\dagger\,.
    \end{equation*}
    By construction, this generator satisfies \Cref{assum:finite-degree}. However, one can show that it is not trace-preserving, and therefore it cannot satisfy \Cref{assum:sobolev-stability}.
\end{ex}

\paragraph{\emph{Proof strategy}:} Our proof is partly inspired by \cite{Davies.1977}, however, \Cref{assum:sobolev-stability} allows us to go beyond minimal semigroups and to thereby generate a trace-preserving semigroup. An important intermediate step is that the considered generators are also generators on $W^{k,1}$, which will allow us to provide simple perturbation analysis on specific examples in \Cref{sec:example-perturbation-bounds}. Our proof starts with \Cref{lem:semigroup-of-G-for-positivity}, where we show that $G_\varepsilon\coloneqq G-\varepsilon (N+\1)^{4d}$ is a generator on the Fock space and the implemented semigroup $t\mapsto e^{tG_\varepsilon}\cdot e^{tG_\varepsilon^\dagger}$ admits $\cT_f$ as a core. Then \Cref{lem:semigroup-of-G-for-sobolev-stability} extends the result to semigroups on $W^{k, 1}$ for all $k \in \R_+$. By the compact embedding lemma \ref{thm:compact-embedding-weighted-spaces} for $W^{k, 1}$ in $\cT_{1,\operatorname{sa}}$, we can transfer these properties to the unperturbed evolution. Next, we more closely follow the method introduced in \cite{Davies.1977}. In particular, we prove that a perturbed version of \Cref{eq:lindblad} generates a Sobolev and positivity preserving $C_0$-semigroup.

\begin{lem}\label{lem:semigroup-of-G-for-positivity}
    For $\varepsilon > 0$, the closure of the operator
    \begin{equation*}
        \cG_\varepsilon: \cT_f \to \cT_f, \qquad x \mapsto G x + x G^\dagger - \varepsilon \{(N + \1)^{4d}, x\}\,,
    \end{equation*}
    where $G$ is defined in \Cref{eq:lindblad}, generates a strongly continuous, contractive, positivity preserving semigroup on $\cT_{1,\operatorname{sa}}$.
\end{lem}
\begin{proof}
    The proof is structured in the following two steps: 
    \begin{itemize}
        \item[1)] The closure of $G_\varepsilon:\cH_f \to \cH$, $\ket{\psi} \mapsto G_\varepsilon\ket{\psi} \coloneqq (-\varepsilon(N + \1)^{4d} + G)\ket{\psi}$ generates a strongly continuous contractive semigroup on $\cH$, which we denote by $(P_t^\varepsilon)_{t \ge 0}$.
        \item[2)] The family of maps $(\cP_t^\varepsilon \coloneqq P_t^\varepsilon \cdot (P_t^\varepsilon)^\dagger:\cT_{1,\operatorname{sa}} \to \cT_{1,\operatorname{sa}})_{t \ge 0}$, with $(P_t^\varepsilon)_{t \ge 0}$ from step 1, defines a strongly continuous, contractive, positivity preserving semigroup on $\cT_{1,\operatorname{sa}}$ generated by the closure of $\cG_\varepsilon$.
    \end{itemize}
    
    \textit{Step 1)} By \Cref{assum:finite-degree} there exists $p_\varepsilon \in \C[X,Y]$ such that $G_\varepsilon = p_\varepsilon(a, a^\dagger)$ which shows by \Cref{lem:formal-polynomial-ccr-adjoint-core} that $G_\varepsilon$ is closed with domain $\cD(N^{4d})$. We will now show dissipativity for $G_\varepsilon$ and $G_\varepsilon^\dagger$ to conclude the claim using \Cref{cor:lumer-phillips}. It suffices also to consider $G^\dagger_\varepsilon$ on $\cH_f$, as it is a core by \Cref{lem:formal-polynomial-ccr-adjoint-core} ($\deg(G)=2d$) and therefore dissipativity of $G_\varepsilon^\dagger$ on $\cH_f$ directly implies dissipativity of $G^\dagger_\varepsilon$ on all of its domain. We only show the dissipativity of $G_\varepsilon$, since the proof for $G_\varepsilon^\dagger$ is completely analogous. Let $\ket{\psi}\in \cH_f$, then for any $\lambda>0$
    \begin{align*}
        \norm{(\lambda - G_{\varepsilon}) \ket{\psi}}^2 &= \lambda^2 \braket{\psi\,,\psi} + \braket{G_\varepsilon\psi, G_\varepsilon\psi} - \lambda(\braket{G_\varepsilon\psi,\psi} + \braket{\psi, G_\varepsilon\psi})\\
        &\ge\lambda^2 \braket{\psi\,,\psi} - \lambda(\braket{G_\varepsilon\psi,\psi} + \braket{\psi, G_\varepsilon\psi})\\
        &\ge \lambda^2 \braket{\psi\,,\psi} + \varepsilon \lambda (\braket{(N + \1)^{4d}\psi,\psi} + \braket{(N + \1)^{4d}\psi,\psi}) - \lambda(\braket{G\psi,\psi} + \braket{\psi, G\psi})\\
        &\ge \lambda^2 \braket{\psi\,,\psi} - \lambda(\braket{G\psi,\psi} + \braket{\psi, G\psi}) \, . 
    \end{align*}
    By the requirement of \Cref{assum:finite-degree}, it is clear that $\tr[\cL(x)] = 0$ for $x \in \cT_f$, using the cyclicity of the trace. For the explicit case of a pure state $x = \ketbra{\psi}{\psi} \in \cT_f$ where the last inclusion holds due to $\ket{\psi} \in \cH_f$, we get
    \begin{equation*}
        \lambda\braket{\psi, -(G^\dagger + G) {\psi}} = \lambda \sum\limits_{j = 1}^K \braket{L_j\psi, L_j {\psi}} \ge 0 \, .
    \end{equation*}
    Hence, we conclude 
    \begin{equation*}
        \norm{(\lambda - G_\varepsilon)\ket{\psi}}^2 \ge \lambda^2 \braket{\psi,\psi} = \lambda^2 \norm{\ket{\psi}}^2 \, . 
    \end{equation*}
    Taking the square root on both sides proves the claim. Note that for $G_\varepsilon^\dagger$ all steps are similar due to the simple observation that $\braket{G\psi,\psi} + \braket{\psi, G\psi} = \braket{G^\dagger\psi,\psi} + \braket{\psi, G^\dagger\psi}$ for $\ket{\psi} \in\cH_f$.
    
    \textit{Step 2)} That the implemented semigroup\footnote{A discussion on implemented semigroups can be found in \cites{Alber.2001}.} $(\cP_t)_{t \ge 0}$ is a strongly continuous, positivity-preserving contractive semigroup that can be easily checked. We further get from \cite[Prop.~2.1]{Davies.1977} that it is generated by the closure of the operator
    \begin{equation*}
        \widetilde{\cG}_\varepsilon: \cD(\widetilde \cG_\varepsilon) = \{R(1,\overline{G}_\varepsilon) x R(1,\overline{G}_\varepsilon)^\dagger \; : \; x \in \cT_{1,\operatorname{sa}}\}  \to \cT_{1,\operatorname{sa}}, \quad x \mapsto \overline{G}_\varepsilon x + x \overline{G}_\varepsilon^\dagger
    \end{equation*}
    where $\overline{G}_\varepsilon$ is the closure of $G_\varepsilon$, $\overline{G}_\varepsilon^\dagger$ its adjoint, and $R( 1,\overline{G}_\varepsilon)$ its resolvent on $\cH$, respectively. Since $\cH_f$ is a core for the generator $\overline{G}_\varepsilon$, the set $O \coloneqq (\1 - \overline{G}_\varepsilon)\cH_f = (\1 - G_\varepsilon)\cH_f$ is dense in $\cH$, which in turn means $\cO = \text{span}\{\ketbra{\psi}{\varphi} \;:\; \ket{\psi}, \ket{\varphi} \in O\}$ is dense in $\cT_{1,\operatorname{sa}}$. A simple calculation further shows that $R(1,\overline{G}_\varepsilon)\cO R(1,\overline{G}_\varepsilon)^\dagger = \cT_f$. Hence for $y \in \cD(\widetilde \cG_\varepsilon)$, we find $x \in \cT_{1,\operatorname{sa}}$ and a sequence $\{x_n\}_{n \in \N} \subseteq \cO$ with $x_n \to x$ for $n \to \infty$ such that
    \begin{align*}
        \widetilde \cG_\varepsilon(y) &= \widetilde \cG_\varepsilon(R(1,\overline{G}_\varepsilon) x R(1,\overline{G}_\varepsilon)) = \widetilde \cG_\varepsilon(R(1,\overline{G}_\varepsilon) \lim\limits_{n \to \infty}x_n R(1,\overline{G}_\varepsilon)^\dagger)\\
        &= \lim\limits_{n \to \infty} \widetilde\cG_\varepsilon(R(1,\overline{G}_\varepsilon)x_n R(1,\overline{G}_\varepsilon)^\dagger)\\
        &= \lim\limits_{n \to \infty} \cG_\varepsilon(y_n)
    \end{align*}
    where we used the continuity of the map $\widetilde \cG_\varepsilon(R(1,\overline{G}_\varepsilon) \cdot R(1,\overline{G}_\varepsilon)^\dagger)$ and set $y_n = R(1,\overline{G}_\varepsilon) x_n R(1,\overline{G}_\varepsilon)^\dagger$. Note that $\{y_n\}_{n \in \N}$ is by construction a convergent sequence on $\cT_f$. In the last equality, we used that $\widetilde \cG_\varepsilon$ and $\cG_\varepsilon$ agree on $\cT_f$. This shows not only that $\cG_\varepsilon$ is closable but further that its closure is the closure of $\widetilde \cG_\varepsilon$. Hence the closure of $\cG_\varepsilon$ is the generator of $(\cP_t^\varepsilon)_{t \ge 0}$.
\end{proof}
Using the above lemma, we are now able to prove the following.

\begin{lem}\label{lem:semigroup-of-G-for-sobolev-stability}
    For all $k \ge 0$ and $\varepsilon > 0$, the closure of the operator $\cG_\varepsilon$ from \Cref{lem:semigroup-of-G-for-positivity} generates a strongly continuous, positivity preserving semigroup $(\cP^\varepsilon_t)_{t \ge 0}$ on $W^{k, 1}$ such that, for all $t\ge 0$, 
    \begin{equation*}
        \norm{\cP^\varepsilon_t}_{W^{k, 1} \to W^{k, 1}} \le e^{\omega_k t} 
    \end{equation*}
    where $\omega_k = \frac{k_{r_1} - k}{k_{r_1} - k_{r_0}}\omega_{k_{r_0}} + \frac{k - k_{r_0}}{k_{r_1} - k_{r_0}}\omega_{k_{r_1}}$ for an $r$ such that $k_{r_0}\leq k <k_{r_1}$. Finally, for $k = 0$, the semigroup is contractive.
\end{lem}
\begin{proof}
   Without loss of generality, we can restrict to $k \in \{k_r\}_{r \in \N}$ as for $k$ inbetween, we can interpolate between the $\{k_r\}_{r \in \N}$ (shown below) and $k = 0$ (shown in \Cref{lem:semigroup-of-G-for-positivity}) using \Cref{lem:interpolation-lemma}. Let $\varepsilon > 0$. In the following proof, the closure, domain and boundedness of an operator are always with respect to the Banach space $W^{k, 1}$ if not stated otherwise. We will show the claim, by first arguing that $\cG_\varepsilon$ is closable, that all $\lambda > \omega_k$ are in the resolvent set of the closure $\overline{\cG}_\varepsilon$ and further that $\norm{R(\lambda,\overline{\cG}_\varepsilon)}_{W^{k, 1} \to W^{k, 1}} \le \tfrac{1}{\lambda - \omega_k}$. By \Cref{thm:hille-yosida}, the above immediately gives the existence of the semigroup on $W^{k, 1}$ and provides us with the claimed bound. The property of positivity preservation traces back to the representation of the semigroup via the Euler approximation and therefore the positivity of the resolvent: for any $x\in \cT_{1,\operatorname{sa}}$:
    \begin{equation}
        \cP_{t}^{\varepsilon}(x)=\lim_{n\to\infty}\,\left(\frac{n}{t}\right)^n\,R(n/t,\overline{\cG}_\varepsilon)^{n}(x)\,.
    \end{equation}
    
    The claims are proven in three steps.
    \begin{itemize}
        \item[\normalfont Step 1.]\label{step:step-1} Show that $\cG_\varepsilon:\cT_f \to \cT_f$ is closable and there exists a $\lambda > \omega_k$ such that $\lambda - \overline{\cG}_\varepsilon:\cD(\overline{\cG}_\varepsilon) \to W^{k, 1}$ is bijective.
        \item[\normalfont Step 2.] Using \Cref{assum:sobolev-stability} and \Cref{lem:semigroup-of-G-for-positivity} we prove that if $\lambda > \omega_k$ is in the resolvent set of $\overline{\cG}_{\varepsilon}$, we not only have that the resolvent is positivity preserving but further $$\norm{R(\lambda,\overline{\cG}_\varepsilon)}_{W^{k, 1} \to W^{k, 1}} \le \frac{1}{\lambda - \omega_k}\,.$$
        \item[\normalfont Step 3.] The surjectivity of $\lambda - \overline{\cG}_\varepsilon$ for a specific $\lambda > \omega_k$ from step 1.~and the bound on the resolvent from step 2.~allow us to successively use the series expansion of the resolvent as it is done in \cite[Prop.~IV.1.3]{Engel.2000} to get that $(\omega_k, \infty)$ is in the resolvent set of $\overline{\cG}_\varepsilon$, and therefore conclude the proof.
    \end{itemize}
    \medskip 
    
   \noindent  Proof of step 1.~We introduce the map 
    \begin{equation*}
        \cI_{d, \varepsilon}: \cT_f \to \cT_f, \qquad x \mapsto \cI_{d, \varepsilon}(x) \coloneqq - \varepsilon\{(N + \1)^{4d}, x\} \, . 
    \end{equation*}
    For $\lambda \ge 0$, $x \in \cT_f$, we can use \Cref{lem:(n+1)-(n+1)-properties} to write
    \begin{equation*}
        (\lambda - \cG_\varepsilon)(x) = (\1 - \cG_0 \circ (\lambda - \cI_{d, \varepsilon})^{-1}) \circ (\lambda - \cI_{d, \varepsilon})(x)
    \end{equation*}
    where $\circ$ is the function composition and $\lambda - \cI_{d, \varepsilon}:\cT_f \to \cT_f$ is a bijection, with bounded inverse (see \Cref{lem:(n+1)-(n+1)-properties}) between dense subspaces of $W^{k, 1}$. This means in particular that it is closable and that its closure has a bounded inverse. We will hence focus on the map $\1 - \cG_0 \circ (\lambda - \cI_{d, \varepsilon})^{-1}:\cT_f \to \cT_f$ and show that it is bounded (on the dense subset $\cT_f$ of $W^{k, 1}$) and hence uniquely extendable to all of $W^{k, 1}$. This then immediately gives us that $\lambda - \cG_0:\cT_f \to \cT_f$ is closable as it is the composition of a map with a dense range succeeded by a bounded map. Note first that we can apply \Cref{lem:infinitesimal-boundedness-W-k-1} to $\cG_0$ to get that there exists $C_k \ge 0$ such that for all $\kappa > 0$ and $x \in \cT_f$
    \begin{equation*}
        \norm{\cG_0(x)}_{W^{k, 1}} \le \kappa \norm{\cI_{d, \varepsilon}(x)}_{W^{k, 1}} + \frac{C_k}{\kappa\varepsilon} \norm{x}_{W^{k, 1}}\,.
    \end{equation*}
    Using the bijectivity of $\lambda - \cI_{d, \varepsilon}:\cT_f \to \cT_f$ we get for $x \in \cT_f$
    \begin{align*}
         \norm{\cG_0 \circ (\lambda - \cI_{d, \varepsilon})^{-1}x}_{W^{k, 1}} \le \kappa\norm{\cI_{d, \varepsilon} \circ (\lambda - \cI_{d, \varepsilon})^{-1}(x)}_{W^{k, 1}} + \frac{C_k}{\varepsilon \kappa} \norm{(\lambda - \cI_{d, \varepsilon})^{-1}(x)}_{W^{k, 1}}\\
         \le ({2}\kappa + \frac{C_k}{\kappa \varepsilon} \frac{1}{\lambda + 2\varepsilon}) \norm{x}_{W^{k, 1}} =: f_k(\lambda, \kappa) \norm{x}_{W^{k, 1}}
    \end{align*}
    where we used properties of $\cI_{d, \varepsilon}$ derived in \Cref{lem:(n+1)-(n+1)-properties}. This gives us not only that $\cG_0 \circ (\lambda - \cI_{d, \varepsilon})^{-1}:\cT_f \to \cT_f$ is bounded, hence uniquely extendable to a bounded map on $W^{k, 1}$ but for a fixed $\kappa < {\frac{1}{2}}$ and $\lambda > \lambda_\kappa$ where $\lambda_\kappa$ is chosen s.t. $f_k(\lambda_\kappa, \kappa) < 1$, we get that its closure is a strict contraction on $W^{k, 1}$. As a direct consequence, we find that again for $\lambda > \lambda_\kappa$ the closure of $\1 - \cG_0 \circ (\lambda - \cI_{d, \varepsilon}): \cT_f \to \cT_f$ is invertible with bounded inverse, and that its inverse function is just given by the geometric series of the closure of $\cG_0 \circ (\lambda - \cI_{d, \varepsilon})^{-1}$. To conclude, we can set $\lambda = 0$ in the above result and get that $-\cG_\varepsilon$ and hence $\cG_\varepsilon$ is closable and further that for $\kappa < \frac{1}{2}$ all $\lambda$ with $\lambda > \lambda_\kappa$ are in the resolvent set of $\overline{\cG}_{\varepsilon}$.
    \medskip
    
    \noindent Proof of step 2. Let $\lambda > \omega_k$ be in the resolvent set of $\overline{\cG}_\varepsilon$. From the compact embedding of $W^{k, 1}$ in $\cT_{1,\operatorname{sa}}$, we immediately get that $R(\lambda,\overline{\cG}_\varepsilon):W^{k, 1} \to W^{k, 1}$ agrees with the respective restricted resolvent of the closure $\widehat{\cG}_\varepsilon$ of $\cG_\varepsilon$ on $\cT_{1,\operatorname{sa}}$ that we obtained in \Cref{lem:semigroup-of-G-for-positivity}. We know that the latter resolvent is positivity preserving, as the semigroup is. This is due to the following integral representation for strongly continuous semigroups \cite[Thm.~II.1.10 (i)]{Engel.2000}: for all $x\in x(\widehat{\cG}_\varepsilon)$,
    \begin{equation}
        R(\lambda, \widehat{\cG}_\varepsilon)(x)=\int_0^\infty\,e^{-\lambda s}\,e^{s\widehat{\cG}_\varepsilon}(x)\,ds\,.
    \end{equation}
     Hence $R(\lambda,\overline{\cG}_\varepsilon)$ is positivity preserving as well. Using \Cref{assum:sobolev-stability}, we have that for $x \in \cT_f$, $x$ positive semi-definite,
    \begin{equation*}
        \tr[\cL(x) (N + \1)^{k/2}] \le \omega_k \tr[x (N + \1)^{k/2}] \, .
    \end{equation*}
    Adding non-negative terms, using the cyclicity of the trace and splitting up $\cL$ gives us
    \begin{align*}
        &\sum\limits_{j = 1}^K\tr[(N+\1)^{k/4}L_j x L_j^\dagger(N+\1)^{k/4}] + (\lambda - \omega_k)\tr[(N + \1)^{k/4} x (N + \1)^{k/4}] \\
        &\qquad\qquad\qquad\qquad\qquad\qquad\qquad\qquad\qquad\qquad\qquad\qquad \le \tr[(N + \1)^{k/4}(\lambda - \cG_\varepsilon)(x)(N + \1)^{k/4}]\,,
    \end{align*}
    and therefore
    \begin{align*}
        (\lambda - \omega_k) \norm{x}_{W^{k, 1}} \le \norm{(\lambda - \cG_\varepsilon)(x)}_{W^{k, 1}} \, 
    \end{align*}
    where we have just dropped non-negative terms and used $\tr[\cdot] \le \norm{\cdot}_1$ with equality if the argument is positive semi-definite. Since $\overline{\cG}_\varepsilon$ is the closure of $\cG_\varepsilon$, the above inequality extends to $x \in \cD( \overline{\cG}_\varepsilon)$, $x$ positive semi-definite and $\overline{\cG}_\varepsilon$ instead of $\cG_\varepsilon$. Together with the positivity preserving property of the resolvent, this gives us that for all $x \in W^{k, 1}$, $x$ positive semi-definite
    \begin{equation}\label{eq:resolvent-bound-G-positive-semidefinite}
        \norm{R(\lambda,\overline{\cG}_\varepsilon)x}_{W^{k, 1}} \le \frac{1}{\lambda - \omega_k} \norm{x}_{W^{k, 1}} \, . 
    \end{equation}
    For a general $x \in W^{k, 1}$, we set $x_\pm = \frac{1}{(N + \1)^{k/4}}\,[(N + 1)^{k/4} x (N + \1)^{k/4}]_\pm \,\frac{1}{(N + \1)^{k/4}} \in W^{k, 1}$, where $[\cdot]_\pm$ denotes the positive, resp. the negative part of a self-adjoint trace-class operator. We clearly have that $x = x_+ - x_-$ and further that $x_+, x_-$ are positive semi-definite by construction. Hence
    \begin{align*}
        \norm{R(\lambda,\overline{\cG}_\varepsilon)x}_{W^{k, 1}} &\le \norm{R(\lambda,\overline{\cG}_\varepsilon )x_+}_{W^{k, 1}} + \norm{R(\lambda,\overline{\cG}_\varepsilon)x_-}_{W^{k, 1}}\\
        &\le \frac{1}{\lambda - \omega_k}(\norm{x_+}_{W^{k, 1}} + \norm{x_-}_{W^{k, 1}}) = \frac{1}{\lambda - \omega_k} \norm{x_+ - x_-}_{W^{k, 1}}\\
        &= \frac{1}{\lambda - \omega_k}\, \norm{x}_{W^{k, 1}}
    \end{align*}
    where we used \Cref{eq:resolvent-bound-G-positive-semidefinite} and the construction of $x_+$ and $x_-$, which concludes step 2.
    \medskip
    
    \noindent Proof of step 3. From step 1.~we get that there exists a $\lambda > \omega_k$ in the resolvent set of $\overline{\cG}_\varepsilon$ whereas step 2.~tells us that, for this $\lambda$, the resolvent is bounded by $\frac{1}{\lambda - \omega_k}$. We can use the same proof strategy as in \cite[Prop.~II.3.14 (ii)]{Engel.2000} where the authors employ the series expansion of the resolvent and its explicit bound to make conclusions about the resolvent set. Following their steps we first get that $(\omega_k, 2\lambda - \omega_k)$ is part of the resolvent set, and then using step 2.~again, we obtain the positivity preservation property as well as the explicit bound for all of those resolvents. This allows us to successively use these arguments and conclude that the resolvent set contains $(\omega_k, \infty)$.
\end{proof}

Putting together the results from \Cref{lem:semigroup-of-G-for-positivity} and \Cref{lem:semigroup-of-G-for-sobolev-stability}, we are now able to get rid of the perturbation $\cI_{d, \varepsilon}$.

\begin{lem}\label{lem:eliminating-perturbation-G}
    The closure of 
    \begin{equation*}
        \cG:\cT_f \to \cT_f, \quad x \mapsto \cG(x) = Gx + x G^\dagger\,,
    \end{equation*}
    where $G$ is defined in \Cref{eq:lindblad}, generates a strongly continuous, positivity preserving semigroup $(\cP_t)_{t\ge 0}$ on $W^{k, 1}$ for all $k \in \N$ with 
    \begin{equation*}
        \norm{\cP_t}_{W^{k, 1} \to W^{k, 1}} \le e^{\omega_k t} \, . 
    \end{equation*}
    where $\omega_k = \frac{k_{r_1} - k}{k_{r_1} - k_{r_0}}\omega_{k_{r_0}} + \frac{k - k_{r_0}}{k_{r_1} - k_{r_0}}\omega_{k_{r_1}}$ for an $r$ such that $k_{r_0}\leq k <k_{r_1}$. Finally, for $k = 0$, the semigroup is contractive.
\end{lem}
\begin{proof}
    The proof is a direct application of \Cref{lem:approximation-lemma} to the semigroups we obtained in \Cref{lem:semigroup-of-G-for-sobolev-stability} taking $\varepsilon \to 0$. Since the semigroups in \Cref{lem:semigroup-of-G-for-sobolev-stability} were positivity preserving, so is the obtained semigroup in the limit $\varepsilon\to 0$ (c.f.~\Cref{lem:approximation-lemma}).
\end{proof}

We are now ready to prove the main Theorem of the section.

\begin{proof}[Proof of \Cref{thm:generation-theorem}]
    The proof strategy is inspired by \cite[Thm.~2.5]{Davies.1977}. It however makes use of \Cref{lem:approximation-lemma} to avoid the issues discussed in \cite[§3]{Davies.1977}. Let $k \in \{k_r\}_{r \in \N}$ or $k = 0$ for the moment. Note that from \Cref{assum:finite-degree}, we can conclude $\omega_k = 0$ for $k = 0$ in \Cref{eq:Assumptionsobolevstability}. We first define for $\delta \in (0, 1)$ the map
    \begin{align*}
        \cL_\delta:\cT_f \to \cT_f, \quad x \mapsto \cL_\delta(x) &= G x + x G^\dagger + \delta\sum\limits_{j = 1}^K L_j x L_j^\dagger =: \cG(x) + \delta\Sigma(x) , 
    \end{align*}
    and show that its closure defines a strongly continuous, positivity preserving semigroup $(\cP_t^\delta)_{t \ge 0}$ on $W^{k, 1}$ which further satisfies
    \begin{equation*}
        \norm{\cP_t^\delta}_{W^{k, 1} \to W^{k, 1}} \le e^{\omega_k t} \, . 
    \end{equation*}
    We first note that for $\widetilde \lambda > 0$, a rearrangement of \Cref{eq:Assumptionsobolevstability} using cyclicity of the trace and that $\tr[\cdot] \le \norm{\cdot}_1$ with equality if the argument is positive semi-definite gives
    \begin{equation*}
        \norm{\Sigma(x)}_{W^{k, 1}} \le \norm{(\widetilde \lambda + \omega_k - \cG)(x)}_{W^{k, 1}} 
    \end{equation*}
    for $x \in \cT_f$ and $x\ge 0$. Now using that $\cG$ is closable (\Cref{lem:eliminating-perturbation-G}) and its resolvent positivity preserving we can conclude  for $\lambda \coloneqq \widetilde \lambda + \omega_k > \omega_k$, $x \in (\lambda - \cG)\cT_f$ and $x \ge 0$, 
    \begin{equation*}
        \norm{\Sigma \circ R(\lambda,\overline{\cG}) (x)}_{W^{k, 1}} \le \norm{x}_{W^{k, 1}}\,.
    \end{equation*}
    Applying similar methods as in step 2. of the proof of \Cref{lem:semigroup-of-G-for-sobolev-stability}, we can extend the above inequality to general $x \in (\lambda - \cG) \cT_f$. Hence, $\Sigma \circ R(\lambda,\overline{\cG})$ is contractive on the dense set $(\lambda - \cG)\cT_f$ and positivity preserving, since both $\Sigma$ and $R(\lambda,\overline{\cG})$ are. It can therefore be uniquely extended to a positivity preserving contractive map on all of $W^{k, 1}$ which we will call $\cA_\lambda$ in the following. As a consequence $(\cL_\delta, \cD(\cL_\delta))$ is closable and $\lambda > \omega_k$ in the resolvent set of the closure. Both facts follow from the representation
    \begin{equation*}
        (\lambda - \cL_\delta) = (\1 - \delta\Sigma \circ R(\lambda,\overline{\cG})) \circ (\lambda - \cG)
    \end{equation*}
    which decomposes $\lambda - \cL_\delta$ into a composition of a closable map with a dense range and a map that is bounded on that range. We further get for the resolvent of the closure
    \begin{equation*}
        R(\lambda,\overline{\cL}_\delta) = R(\lambda,\overline{\cG}) \sum\limits_{n = 0}^\infty \delta^n \cA_\lambda^n \,,  
    \end{equation*}
    which immediately lets us conclude that the resolvent is positivity preserving as $\cA_\lambda$ and $R(\lambda,\overline{\cG})$ are. Lastly, we will show that for $\lambda > \omega_k$
    \begin{equation}\label{eq:bound-resolvent-L_r}
        \norm{R(\lambda,\overline{\cL}_\delta)}_{W^{k, 1} \to W^{k, 1}} \le \frac{1}{\lambda - \omega_k} \, . 
    \end{equation}
    To obtain this inequality we again rearrange \Cref{assum:sobolev-stability}, add non-negative terms, use cyclicity of the trace and that $\tr[\cdot] \le \norm{\cdot}_1$ with equality if the argument is positive semi-definite, to conclude that for $x \in (\lambda - \cL_r)\cT_f$, $x$ positive semi-definite,
    \begin{equation*}
        \norm{R(\lambda,\overline{\cL}_\delta) x}_{W^{k, 1}} \le \frac{1}{\lambda - \omega_k}\norm{x}_{W^{k, 1}}\,.
    \end{equation*}
    We again extend the above bound to all $x \in (\lambda - \cL_\delta)\cT_f$ analogously to step 2 in the proof of \Cref{lem:eliminating-perturbation-G}. Using that $(\lambda - \cL_\delta)\cT_f$ is dense then gives \Cref{eq:bound-resolvent-L_r}. Employing \Cref{thm:lumer-phillips}, we get that indeed for all $\delta \in (0, 1)$ the closure of $(\cL_\delta, \cD(\cL_\delta))$ generates a strongly continuous semigroup which is positivity preserving since the resolvent is and satisfies the claimed bound. To now fill the gap between $0$ and the $\{k_r\}_{r \in \N}$ respectively, we interpolate between the semigroups (q.v.~\Cref{lem:interpolation-lemma}), obtaining $e^{t\omega_k}$ where $\omega_k = \frac{k_{r_1} - k}{k_{r_1} - k_{r_0}}\omega_{k_{r_0}} + \frac{k - k_{r_0}}{k_{r_1} - k_{r_0}}\omega_{k_{r_1}}$ for an $r$ such that $k_{r_0}\leq k <k_{r_1}$, as the bound of the interpolated semigroups. Now that we have the result for all $k \ge 0$ we can employ \Cref{lem:approximation-lemma} and take the limit $\delta \to 1$ to obtain the assertion. The contractivity and trace-preserving property of the semigroup in the case $k = 0$  just follows from the GKSL form of $(\cL, \cD(\cL))$, i.e.~$\tr[\cL(x)] = 0$ for $x \in \cT_f$, or put differently \Cref{assum:finite-degree}.
\end{proof}

\subsection{Bosonic evolution systems}\label{sec:timedependentgeneration}

Next, we consider time-dependent generators in GKSL form. For this, we modify Assumptions \ref{assum:finite-degree} and \ref{assum:sobolev-stability} in the following way: 

\begin{assum}\label{assum:finite-degree-time-dep}
    The operator $(\cL_s,\cT_f)$ has $\operatorname{GKSL}$ form, i.e.~for $x\in\cT_f$ and $s\in[0,\infty)$ 
    \begin{equation}\label{eq:lindblad-time-dep}
        \begin{aligned}
            \cL_s:\cT_f \to \cT_f \quad x \mapsto\cL_s(x) &= - i [H(s), x] + \sum\limits_{j = 1}^K L_j(s) x L_j^\dagger(s)  - \frac{1}{2}\{L_j^\dagger(s) L_j(s), x\} \\
            &\coloneqq G(s)x + x G^\dagger(s) + \sum\limits_{j = 1}^K L_j(s) x L_j^\dagger(s)\, , 
        \end{aligned}
    \end{equation}
    where $K\in\mathbb{N}$, $G(s)=-iH(s)-\frac{1}{2}\sum_{j=1}^K L_j^\dagger(s) L_j(s)$, and $H(s)\coloneqq p_{H(s)}(a,a^\dagger), L_j(s)\coloneqq p_{j,s}(a,a^\dagger)$ are polynomials of the creation and annihilation operators with time-dependent, differentiable coefficients. Again, $d_H\coloneqq\sup_{s\ge 0}\,\deg(p_{H(s)})<\infty$, $d_j\coloneqq\sup_{s\ge 0}\deg(p_{j,s})<\infty$, and $d\coloneqq\max\{d_1,...,d_K,d_H\}$.
\end{assum}

The next assumption will lead to the evolution system being Sobolev preserving, which allows us not only to prove the existence and uniqueness of the evolution generated by \eqref{eq:mastereq} but further to conduct a perturbation analysis as well as to extend our results to the case of a time-dependent Master equation:

\begin{assum}\label{assum:sobolev-stability-time-dep}
    There exists a non-negative sequence $\{k_r\}_{r \in \N}\rightarrow\infty$ s.t.~for all $r \in \N$ there exist $\omega_{k_r} \ge 0$ such that for all $s \in \R_+$ and $x \in \cT_f$ positive semi-definite,
    \begin{equation}
        \tr[\cL_s(x) (N + \1)^{k/2}] \le \omega_{k_r} \tr[x (N + \1)^{k/2}] \, .
    \end{equation}
    Note that the coefficients $\omega_{k_r}$ are independent of $s$.
\end{assum}

Under the above assumptions we can state the generation theorem for evolution systems as follows:

\begin{thm}[Generation of bosonic evolution systems]\label{thm:timedep-generation-theorem}
    Let $(\cL_s, \cD(\cL_s))_{s\in[ 0,\infty)}$ be a family of operators that fulfill \Cref{assum:finite-degree-time-dep} and \Cref{assum:sobolev-stability-time-dep}. Then $(\overline{\cL}_s, \cD(\overline{\cL}_s))_{s \in \R_+}$ gives rise to a unique evolution system $(\cP_{t,s})_{0\le s\le t}$ on $W^{k, 1}$ for all $k\ge 0$ with the following properties
    \begin{enumerate}
        \item $\cP_{t,s} (W^{k + 4d, 1}) \subseteq W^{k + 4d, 1}$ for all $0\le s\le t$
        \item For any $x \in W^{k + 4d, 1}$, the family $( \cP_{t,s}(x))_{0\le s\le t}$ is the unique solution to the initial value problem
        \begin{equation}\label{eq:time-dep-init-value-problem}
            \frac{d}{dt} x(t) = \overline{\cL}_t(x(t)) \qquad t \in [s, \infty), \; x(s) = x\,.
        \end{equation}
    \end{enumerate}
    For $k=0$, the evolution system is contractive and trace-preserving.
\end{thm}
\begin{proof}
    We assume w.l.o.g.~that $s \in [0,1]$ is fixed since the same argument works for all compact intervals. \Cref{thm:generation-theorem} shows that $(\overline{\cL}_s,\cD(\overline{\cL}_s))$ generates an $\omega_k$-quasi-contractive semigroup $(\cP_t^s)_{t \ge 0}$ on $W^{k, 1}$. Next, we realize that $W^{k + 4d, 1}$ are $\cL_s$-admissible subspaces, where we recall that $d$ denotes the degree of $\cL_s$. This already proves assumptions (1) and (2) in \Cref{thm:time-dependent-semigroups}. Since the coefficients of the polynomials $p_H$ and $p_j$ are continuous and operators of the form
    \begin{equation*}
        (N+\1)^{k/4} a^j(\ad)^l (N+\1)^{-(k/4+d)}\,,
    \end{equation*}
    for $j+l\leq d$, are bounded (see \Cref{lem:boundedness-polynomials}) w.r.t.~the operator norm, we have by Hölder inequality that
    \begin{equation*}
        s \mapsto (N + \1)^{k/4} \cL_s((N + \1)^{-k/4 + d} (\cdot) (N + \1)^{-k/4 + d})(N + \1)^{k/4} =: \cA(s)
    \end{equation*}
    is a bounded and uniformly continuous family of operators. Therefore,
    \begin{equation*}
        s\mapsto\cL_s\in\cB(W^{k,1},W^{k+4d,1})
    \end{equation*}
    is uniformly continuous, which proves condition (3) in \Cref{thm:time-dependent-semigroups}.
    Hence \Cref{thm:time-dependent-semigroups} provides the existence and uniqueness of an evolution system on $W^{k, 1}$. By repeating the above arguments on $\cY\coloneqq W^{k+4d}$, i.e.~by choosing our $\cL_s$-admissible subspace as $W^{k + 8d, 1}$, \Cref{thm:time-dependent-semigroups} provides existence and uniqueness of a solution on $\cY = W^{k + 4d, 1}$ which agrees with the former one on $W^{k, 1}$ by the compact embedding of $W^{k + 4d, 1}$ into $W^{k, 1}$. Therefore, conditions (4) and (5) are satisfied for the evolution system on $W^{k, 1}$ and the admissible subspace $\cY=W^{k + 4d, 1}$, which through \Cref{thm:time-dependent-semigroups} proves the claim. Moreover, the evolution system is positivity preserving because it can be constructed by a concatenation of time-independent positivity preserving semigroups (see \Cref{thm:generation-theorem} and \cites[Eq.~5.3.5]{Pazy.1983}). Contractivity and the property of trace preservation are a consequence of the fact that $\omega_0$ can be chosen to be $0$.
\end{proof}