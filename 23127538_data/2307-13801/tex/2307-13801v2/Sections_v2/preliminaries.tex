We begin with a short review of valuable tools from Banach space theory in \Cref{subsec:banach-space} and build upon them to prove a compact embedding theorem for a class of weighted spaces in \Cref{subsec:weighted-norms-compact-embedding}. We then recall standard results from the theory of strongly continuous semigroups as well as evolution systems in \Cref{subsec:c0-semigroups}. These will play an essential role in \Cref{sec:polynomial-generators}. Finally, we introduce continuous variable quantum systems, provide some valuable properties of polynomials of annihilation and creation operators, and introduce the notion of a \textit{Sobolev preserving semigroup}, which are the main objects of study in the remainder of the paper.

\subsection{Basic Banach space theory}\label{subsec:banach-space}

We start with a brief recap on notions from the theory of Banach spaces that will be needed in this paper, and refer to \cites[Chap.~III]{Kato.1995}[Chap.~IV]{Conway.1994}[Chap.~2-3]{Simon2015}[Chap.~2-3]{Hille.1957} for more details. Let $(\cX,\|\cdot\|_{\cX})$ be a {Banach space}. We denote the space of bounded operators between two Banach spaces $\cX$ and $\cY$ by $\cB(\cX,\cY)$, with $\cB(\cX,\cX)=\cB(\cX)$. The identity map in $\cB(\cX)$ is denoted by $\1_{\cX}$, or simply $\1$ when the underlying space is clear from the context. The operator norm is denoted by
\begin{equation}\label{eq:operator-norm}
    \|A\|_{\cX\rightarrow\cY}=\|A:\cX\to \cY\|\coloneqq\sup_{\|x\|_{\cX}=1}\|A(x)\|_{\cY}.
\end{equation}
We recall that the linear space $\cB(\cX,\cY)$ equipped with the operator norm is a Banach space since $(\cY,\|\cdot\|_{\cY})$ is a Banach space. An operator $A:\cX\rightarrow\cY$ is compact if the image sequence $\{Ax_n\}_{n\in\N}\subset\cY$ of any bounded sequence $\{x_n\}_{n\in\N}\subset\cX$ has a converging subsequence. In particular, every operator which can be approximated by a sequence of finite rank operators is compact \cites[Thm.~3.1.9]{Simon2015}.

More generally, an unbounded operator $A$ is a linear map $A:\cD(A)\subset\cX\rightarrow\cY$ defined on its domain $\cD(A)\subset\cX$. If the domain is dense in $\cX$, the operator is said to be densely defined. In this paper, all unbounded operators are densely defined. Note that the addition and concatenation of two unbounded operators $(A,\cD(A))$ and $(B,\cD(B))$ is defined on $\cD(A+B)=\cD(A)\cap\cD(B)$ and $\cD(AB)=B^{-1}(\cD(A))$ (cf.~\cite[Sec.~III§5.1]{Kato.1995}).
An operator $(A,\cD(A))$ is closed iff its graph $\{(x,A(x)):x\in\cD(A)\}$ is a closed set in the product space $\cX\times\cY$. A bounded operator is closed iff its domain is closed. By convention, we extend all densely defined and bounded operators by the {bounded linear extension theorem} to bounded operators on $\cX$ \cite[Thm.~2.7-11]{Kreyszig.1989}. An operator is called {closable} if there exist a closed {extension}, where $\overline{A}$ is an extension of $A$ if $\cD(A)\subset\cD(\overline{A})$ and $Ax=\overline{A}x$ for all $x\in\cD(A)$. The closure of $A$ is denoted by $\overline{A}$ \cite[Sec.~7.1]{Simon2015}. We also recall that for an unbounded operator $(A,\cD(A))$  on $\cX$, a {core} for $A$ is a subset $\cD_0\subset\cD(A)$ which is dense in $\cD(A)$ w.r.t.~the {graph norm} $\|\cdot\|_A\coloneqq\|A\cdot\|_{\cX}+\|\cdot\|_{\cX}$ of $A$ (cf.~\cite[Def. 1.6]{Simon2015}). Given two linear operators $(\cL,\cD(\cL))$ and $(\cE,\cD(\cE))$ on $\cX$, the operator $(\cE,\cD(\cE))$ is relatively $\cL$-bounded if $\cD(\cE)\subseteq \cD(\cL)$ and there are $a,b \geq 0$ for all $x\in\cD(\cL)$ such that
\begin{equation}\label{eq:relative-bounded}
    \|\cE(x)\|_{\cX}\leq a\|\cL(x)\|_{\cX}+b\|x\|_{\cX}.
\end{equation}
\smallskip
For a closed linear operator $(\cL, \cD(\cL))$ on a Banach space $\cX$ we call 
\begin{equation*}
    \rho(\cL) \coloneqq \{\lambda \in \C: \lambda - \cL:\cD(\cL) \to \cX \text{ is bijective}\}
\end{equation*}
the resolvent set of $(\cL, \cD(\cL))$. For $\lambda \in \rho(\cL)$ we call the inverse $$R(\lambda, \cL) \coloneqq (\lambda - \cL)^{-1}$$ the resolvent, which is, by the closed graph theorem, a bounded operator on $\cX$.

Besides the convergence w.r.t.~the operator norm (i.e.~uniform convergence), a sequence of operators $\{A_k\}_{k\in\N}$ defined on a common domain $\cD(A)$ converges strongly if $\lim_{k\rightarrow\infty}\|A_kx-Ax\|_{\cY}=0$ for all $x\in\cD(A)$. On the basis of the underlying topologies associated with these two convergences, one can define the Bochner integral of vector and operator-valued maps on a compact interval equipped with the Lebesgue measure, e.g.~$f:[a,b]\rightarrow\cX$ and $F:[a,b]\rightarrow \cB(\cX)$ with $a<b$. Under the assumption that the function $f$ or $F$ can be approximated by a step function and that the real-valued integral
\begin{equation*}
    \int_{a}^b\|f(s)\|_{\cX}\,ds\qquad\text{or}\qquad\int_{a}^b\|F(s)\|_{\cX\rightarrow\cX}\,ds
\end{equation*}
is bounded, the Bochner integrals are defined by standard approximation with step functions. Since all the vector-valued maps considered in this work are continuous, the Bochner integral is always well-defined and coincides with the Riemann and Pettis integrals (more details can be found in \cite{Gordon.1991}). Similar to the real-valued case, the integral satisfies the triangle inequality w.r.t.~the norm, is invariant under closed linear transformations, and satisfies the fundamental theorem of calculus if the map is continuously differentiable \cites[Sec.~3.7-8]{Hille.1957}.

Two special cases of Banach spaces that we will consider are Hilbert spaces and bounded operators defined on Hilbert spaces. We denote the latter by $\cB(\cH)$ and use $\norm{\cdot}_\infty$ for their norm. Given a separable Hilbert space $\cH$ and $A\in\cB(\cH)$, its {adjoint} $A^\dagger$ is uniquely defined by
\begin{equation}\label{eq:adjoint}
    \braket{A\phi\,,\varphi}=\braket{\phi\,,A^\dagger\varphi}
\end{equation}
for all $\phi,\varphi\in\cH$. The space of all bounded, {self-adjoint} operators, i.e.~$A=A^\dagger$, is denoted by $\cB_{\operatorname{sa}}(\cH)$. A special case of self-adjoint operators is those with finite support with respect to a fixed orthonormal basis $\{|n\rangle\}_n$, whose set we denote by $\cT_f\equiv \cT_f(\cH) \coloneqq \{A\in\cB_{\operatorname{sa}}(\cH):\exists M\in\N\,:\,A = \sum_{n,m}^M a_{nm} \ketbra{n}{m}\}$. 
More generally, by a slight abuse of notations, we denote the formal adjoint of an unbounded operator $(A,\cD(A))$ on $\cH$ as $A^\dagger:\cD(A^\dagger)\rightarrow\cH$, where the latter satisfies \Cref{eq:adjoint} for all $\phi\in\cD(A)$ and for all $\varphi$ in the maximally defined domain 
\begin{equation*}
    \cD(A^\dagger)\coloneqq\{\ket{\varphi}\in\cH:\ket{\phi}\mapsto\braket{A\phi\,,\varphi}\text{ is bounded}\}.
\end{equation*}
The operator $A$ is called {symmetric} if for all $\ket{\phi},\ket{\varphi}\in\cD(A)$, $\braket{A\phi\,,\varphi}= \braket{\phi\,,A\varphi}$. $A$ is called self-adjoint if $\cD(A)=\cD(A^\dagger)$ and $A^\dagger=A$. An operator $(A,\cD(A))$ is positive if $\langle A\phi,\phi\rangle\ge 0$ for all $\phi\in\cD(A)$. In this case, we write $A\ge 0$. More generally we write $A\ge B$ if $A-B\ge 0$ with $\cD(A-B)=\cD(A)\cap\cD(B)$. 
The trace of a positive operator $A$ is defined by $\tr[A]=\sum_{n\in\N}\bra{n}A\ket{n}$. When $A\in\cB(\cH)$, its {trace-norm} is defined by $\|A\|_1\coloneqq\tr\big[|A|\big]$. Bounded operators with finite trace-norm are called trace-class and their class we denote by $\cT_1\equiv \cT_1(\cH)$. We will most often consider the Banach space of self-adjoint trace-class operators denoted by $\cT_{1,\, \operatorname{sa}} \equiv \cT_{1,\, \operatorname{sa}}(\cH) \coloneqq \{A \in \cB_{\operatorname{sa}}(\cH):\norm{A}_1<\infty\}$. We define the set of density operators by $\cS\equiv \cS(\cH)\coloneqq\{\rho:\rho\geq0\text{ and }\tr[\rho]=1\}$.

\subsection{Weighted norms and compact embeddings}\label{subsec:weighted-norms-compact-embedding}

For a Banach space $(\cX,\|\cdot\|_{\cX})$ and an invertible operator $(\cW,\cD(\cW))$ on $\cX$, a natural way of defining a new norm out of $\|\cdot\|_{\cX}$ is via the following procedure:
\begin{defi}[Weighted normed space]
    Let $(\cW,\cD(\cW))$ be an invertible linear operator. Then, $\|X\|_{\cW}\coloneqq\|\cW(X)\|_{\cX}$ for $X\in\cD(\cW)$ defines a norm on $\cD(\cW)$. In the following, we denote by $\|P\|_{\cW\to\cW}\coloneqq \sup_{\|X\|_\cW\le 1}\|P(X)\|_\cW$.
\end{defi}

In the next lemma, we prove that completeness of the space $(\cX,\|\cdot\|_{\cX})$ is preserved by the closedness of $(\cW,\cD(\cW))$:
\begin{lem}\label{thm:weighted-banach-space}
    Let $(\cW,\cD(\cW))$ be an invertible linear operator on the Banach space $\cX$. Then, the weighted normed space $(\cD(\cW),\|\cdot\|_{\cW})$ is a Banach space and the norm $\|\cdot\|_{\cW}$ is equivalent to the {graph norm} of $\cW$.
\end{lem}
\begin{proof}
    First, it is clear that $\|\cdot\|_{\cW}$ defines a norm because the linearity of $\cW$ directly implies homogeneity and the triangle inequality, while the injectivity of $\cW$ gives  positive definiteness. By the closed graph theorem, we can conclude that $\cW^{-1}$ is bounded, and therefore $\cW$ is closed (see \cite[Sec.~III.2]{Kato.1995} and \cite[Thm.~2.11.5]{Hille.1957}). Moreover,
    \begin{equation*}
        \|X\|_{\cW}=\|\cW(X)\|_{\cX}\leq\|X\|_{\cX}+\|\cW(X)\|_{\cX}\leq(\|\cW^{-1}\|_{\cX\to\cX}+1)\|X\|_{\cW}
    \end{equation*}
    shows that the graph norm of $\cW$ is equivalent to $\|\cdot\|_{\cW}$. By definition, the graph is a closed operator on the Banach space $\cX\times\cX$ such that the vector space $\cD(\cX)$ equipped with the graph norm is a Banach space. The statement hence follows.
\end{proof}
Next, we introduce a compact embedding \cite[Sec.~5.7]{Evans.2010} for weighted normed spaces, i.e.~we want to reduce the compact embedding to a relation between the defining operators:
\begin{defi}\label{def:compact-embedding}
    Let $(\cX_1,\|\cdot\|_{\cX_1})$ and $(\cX_2,\|\cdot\|_{\cX_2})$ be Banach spaces such that $\cX_1\subset\cX_2$. We say that $\cX_1$ is compactly embedded in $\cX_2$, and denote this condition as $\cX_1\Subset\cX_2$ iff
    \begin{itemize}
        \item[$-$] $\exists c\geq0$ such that $\|\cdot\|_{\cX_2}\leq c\|\cdot\|_{\cX_1}$ and
        \item[$-$] any bounded sequence in $\cX_1$ has a converging subsequence in $\cX_2$ (i.e.~is precompact).
    \end{itemize}
\end{defi}

\begin{lem}[Compact embedding]\label{thm:compact-embedding-weighted-spaces}
    Let $(\cW_1,\cD(\cW_1))$ and $(\cW_2,\cD(\cW_2))$ be invertible linear operators on $\cX$ with bounded inverses and $\cD(\cW_1)\subset\cD(\cW_2)$. Then $(\cD(\cW_1),\|\cdot\|_{\cW_1})$ is compactly embedded in $(\cD(\cW_2),\|\cdot\|_{\cW_2})$ iff the extension of $\cW_2\cW_1^{-1}$ is a compact operator on $\cX$. 
\end{lem}
\begin{proof}
    First, we prove that the first condition in \Cref{def:compact-embedding} is equivalent to the boundedness of $\cW_2\cW_1^{-1}$, which is defined on $\cX$ by the {bounded linear extension theorem} \cite[Thm.~2.7-11]{Kreyszig.1989}. By assuming that $\cW_2\cW_1^{-1}$ is a bounded operator, which is implied by compactness, 
    \begin{equation*}
        \|X\|_{\cW_2}=\|\cW_2\cW_1^{-1}\cW_1(X)\|_{\cX}\leq\|\cW_2\cW_1^{-1}\|_{\cX\to\cX}\|X\|_{\cW_1}.
    \end{equation*}
    Now, if there is a $c\geq0$ such that $\|X\|_{\cW_2}\leq c\|X\|_{\cW_1}$ then boundedness is given by
    \begin{equation*}
        \|\cW_2\cW_1^{-1}(X)\|_{\cX}\leq c\|\cW_1\cW_1^{-1}(X)\|_{\cX}=c\|X\|_{\cX}.
    \end{equation*}
    Let $\{X_k\}_{k\in\N}\subset\cX$ be a bounded sequence and assume the second condition in \Cref{def:compact-embedding}. Then, $\cW_1^{-1}(X_k)$ is a bounded sequence in $(\cD(\cW_1),\|\cdot\|_{\cW_1})$ which admits a converging subsequence in $(\cD(\cW_2),\|\cdot\|_{\cW_2})$ by assumption. Therefore, $\cW_2\cW^{-1}_1(X_k)$ has a converging subsequence in $(\cX,\|\cdot\|_\cX)$. Conversely, assume $\cW_2\cW_1^{-1}$ is compact and $X_k$ a bounded sequence in $(\cD(\cW_1),\|\cdot\|_{\cW_1})$. By definition $\cW_1(X_k)$ is a bounded sequence in $(\cX,\|\cdot\|_\cX)$ so that $\cW_2\cW_1^{-1}(\cW_1X_k)=\cW_2(X_k)$ has a converging subsequence.
\end{proof}

\begin{rmk}
    A simple example of compact embedding is provided by classical Sobolev spaces $W^{k,p}(\mathbb{R})$, $k\in\mathbb{N}$ and $1\le p\le \infty$, with the norm $\|f\|_{k,p}\coloneqq \left(\sum_{i=0}^k\|f^{(i)}\|_p^p\right)^{{1}/{p}}$, where $\|f\|_p$ denotes the $L^p$ norm of $f$ with respect to the Lebesgue measure. Quantum extensions of these spaces recently appeared in \cite{Lafleche.2023,Becker.2021}. Here we will use the latter extension which we recall in Section \ref{subsec:bosonic-systems}.  
\end{rmk}

\subsection{Strongly continuous semigroups and evolution systems}\label{subsec:c0-semigroups}

The evolution of a quantum system is often described by a formal differential equation called the master equation. In order to rigorously study solutions of a {master equation}, the theory of $C_0$-semigroups constitutes an essential toolbox. While detailed expositions to this theory can be found e.g.~in the books \cite[Chap.~II]{Engel.2000}\cite[Chap.~9]{Kato.1995} or \cite[Chap.~X-XIII]{Hille.1957}, here we provide a short overview and introduce concepts that are relevant for the present paper. A family of operators $(P_t)_{t\geq0}\subset\cB(\cX)$ is called a {$C_0$-semigroup} if it satisfies the following properties: 
\begin{itemize}
    \item[$-$] $P_tP_s=P_{t+s}$ for all $t,s\geq0$;
    \item[$-$] $P_0=\1$, the identity map on $\cX$; and
    \item[$-$] $t\mapsto P_t$ is strongly continuous at $0$.
\end{itemize}    
To every $C_0$-semigroup one can associate a linear operator that is in the most general case unbounded, densely defined and closed. This operator determines the semigroup uniquely and is called its generator \cite[Thm.~II.1.4]{Engel.2000}. We will typically denote it by $(\cL, \cD(\cL))$, where
\begin{equation}
    \cD(\cL) = \{x \in \cX : t \mapsto P_t(x) \text{ differentiable on } \R_+\}
\end{equation}
and for $x \in\cD(\cL)$
\begin{equation}
    \cL(x) = \lim\limits_{t \to 0^+} \frac{1}{t}(P_t(x) - x) \, , 
\end{equation}
where the limit is with respect to the topology induced by $\cX$. The semigroup leaves the domain of its generator invariant and further commutes with it on its domain allowing us to justify the well-posedness of the following differential equation on $\cX$
\begin{equation}\label{eq:time-indep-master-equation}
    \frac{d}{dt} x(t) = \cL(x(t)) \quad x(0) \in \cD(\cL) \quad\text{and}\quad t \ge 0 \, . 
\end{equation}
From the above considerations, this equation  has a strongly continuous solution given by the semigroup, i.e.~$P_t(x(0)) = x(t)$. Indeed the semigroup is the unique solution (asking for a continuously differentiable map $t \mapsto x(t)$) to this so-called master equation \cite[Prop.~II.6.2]{Engel.2000}. The formulation as a master equation or initial value problem also reveals the origin of the term ``generator'', since for bounded linear operators the solution to these problems is just given by the semigroup involving the exponential of the generator (i.e.~$(e^{t\cL})_{t \ge 0}$). When the operator $\mathcal{L}$ is bounded, the conditions of existence and uniqueness are immediately satisfied using the series expansion of the exponential. For unbounded operators, the answer is no longer straightforward and requires a different representation of the exponential. One possible choice involves the resolvent of the generator.

The well-known generation theorems by Lumer and Phillips, Hille and Yosida and Feller, Miyadera and Phillips all rely on the resolvent satisfying specific bounds, either directly, or indirectly e.g.~by dissipativity of the underlying operator $\cL$. Below we recall the theorems by Hille and Yosida and Lumer and Phillips, as they are going to be used frequently throughout this paper. It is noteworthy that the first two theorems are generalized by the third with the last allowing for the generation of semigroups that satisfy the bound
\begin{equation*}
    \norm{e^{t \cL}}_{\cX \to \cX} \le c \,e^{\omega t}
\end{equation*}
with  $\omega \in \R$ and $c \ge 0$. If $c = 1$, we call the semigroup $\omega$-quasi contractive, and if further $\omega \le 0$ we call it contractive. We start with the generation theorem by Hille and Yosida that gives necessary and sufficient conditions on an operator to be the generator of a contractive $C_0$-semigroups by imposing constraints on its resolvent.

\begin{thm}[Hille-Yosida]\label{thm:hille-yosida}
    A linear operator $(\cL,\cD(\cL))$ on $\cX$ generates a strongly continuous $\omega$-quasi contraction semigroup iff $(\cL,\cD(\cL))$ is closed, densely defined, the resolvent set contains $(\omega, \infty)$ and for all $\lambda \in (\omega, \infty)$ one has 
    \begin{equation*}
        \| R(\lambda,\cL)\|_{\cX\rightarrow\cX}\leq\frac{1}{\lambda - \omega}\,.
    \end{equation*}
\end{thm}

The other seminal result, which we use in the present paper is a modified formulation of Theorem \ref{thm:hille-yosida} due to Lumer and Phillips \cite[Thm.~II.3.15]{Engel.2000} which, instead of asking for a certain bound on the resolvent, requires certain properties for $\cL$ among which $\omega$-{dissipativity}: 

\begin{defi}\label{def:dissipativity}
    For $\omega \geq 0$, an operator $(\cL,\cD(\cL))$ on $\cX$ is $\omega$-{quasi dissipative} if for all $x \in \cD(\cL)$ and $\lambda > 0$
    \begin{equation*}
        \norm{(\lambda-(\cL-\omega))x}_{\cX} \geq \lambda \norm{x}_{\cX}.
    \end{equation*}
    If $\omega = 0$, we call the operator dissipative. 
\end{defi}
In what follows, the notation $\rg$ stands for the range of an operator $(\cL, \cD(\cL))$, defined as $\rg(\cL) = \{\cL(x):x \in \cD(\cL)\}$. Then we can state the theorem in the following way:

\begin{thm}[Lumer-Phillips]\label{thm:lumer-phillips}
    Let $(\cL,\cD(\cL))$ be a densely defined, $\omega$-dissipative operator on $\cX$. Then, the closure $\overline{\cL}$ generates a $\omega$-quasi contraction semigroup iff there exists a $\lambda>0$ such that $\rg(\lambda-(\cL-\omega))$ is dense in $\cX$.
\end{thm}

\begin{rmk}
    When $\cL$ is dissipative and there is a $\lambda>0$ such that $\rg(\lambda-\cL)$ is dense in $\cX$, then this holds for all $\lambda > 0$ \cite[Prop.~II.3.14]{Engel.2000}.
\end{rmk}

In the case of a $C_0$-semigroup on a Hilbert space $\cH$, the following result proves useful:
\begin{prop}\label{cor:lumer-phillips}
    Let $(G,\cD(G))$ be a densely defined linear operator on $\cH$ and assume that $G$ and $G^\dagger$ are $\omega$-quasi dissipative. Then, $\overline{G}$ generates a $\omega$-quasi contraction $C_0$-semigroup on $\cH$. Moreover, iff $(G,\cD(G))$ generates a $\omega$-quasi contraction semigroup, $G$ and $G^\dagger$ are $\omega$-quasi dissipative.
\end{prop}

In this paper, we are also concerned with extensions of the above theory to the setting of time-dependent $C_0$-semigroups. In this case, we refer to the family as a \textit{$C_0$-evolution system}: A two-parameter family of bounded operators $( P_{t,s})_{0\le s\le t}$ is called an \textit{evolution system} if
\begin{itemize}
    \item[$-$]$P_{t,t}=\1$,
    \item[$-$]$P_{t,r}P_{r,s}=P_{t,s}$ for all $0\le s\le r\le t$, and
    \item[$-$]$(t,s)\mapsto P_{t,s}$ is strongly continuous.
\end{itemize}

A well-known class of evolution systems is given by $C_0$-semigroup after imposing $P_{t,s}=P_{t-s}$. A subtle difference between semigroups and evolution systems is that the latter are not necessarily differentiable for any $x\neq 0$ \cites[p.~478]{Engel.2000}. Here, we recall sufficient conditions under which the following master equation admits a unique \textit{solution operator}:
\begin{equation}\label{eq:time-dep-master-equation}
    \frac{\partial}{\partial t}x(t)=\cL_t(x(t))\quad\text{and}\quad x(s)=x_s\quad\text{ for }0\leq s\leq t\,.
\end{equation}
We start with a set of assumptions often referred to as being of \textit{hyperbolic type} \cites[Chap.~5]{Pazy.1983}[pp.~127~ff.]{Giuseppe.1976} and which allow for the generation of an evolution system starting from a $C_0$-semigroup. Here, the existence of a so-called admissible subspace plays an important role:
\begin{defi}[Admissible subspaces]\label{defi:admissible-spaces}
  For a $C_0$-semigroup $(P_t)_{t\geq0}\subset\cB(\cX)$ with generator $(\cL,\cD(\cL))$, a subspace $(\cY\subset\cX,\|\cdot\|_{\cY})$ is called admissible for $(P_t)_{t\ge 0}$, or simply $\cL$-admissible if $\cY$ is an invariant closed subspace of the semigroup, i.e.~$e^{t\cL}\cY\subset\cY$, and $e^{t\cL}|_{\cY}$ defines a $C_0$-semigroup on $(\cY,\|\cdot\|_\cY)$. Similarly, $(\cY,\|\cdot\|_{\cY})$ is an admissible subspace of an evolution system $P_{0\leq s\leq t}$ if it is an invariant closed subspace of the evolution system and $(P_{0\leq s\leq t}|_{\cY})_{0\leq s \leq t}$ defines an evolution system on $(\cY,\|\cdot\|_\cY)$.  
\end{defi}
Our first basic assumption is that for every fixed $s$ the operator $(\cL_s, \cD(\cL_s))$ generates a $C_0$-semigroup. Moreover,
\begin{itemize}
    \item[$(1)$\hspace{1ex}] $(\cL_s)_{ s\ge 0}$ is a \textit{stable} family, i.e.~there is $c\ge 0$ and $\omega\in\R$ such that $\|e^{t\cL_s}\|_{\cX\rightarrow\cX}\leq c \,e^{\omega t}$ for all $ s\ge 0$;
    \item[$(2)$\hspace{1ex}] There exists a subspace $\cY\subset \cX$ and a norm $\|\cdot\|_\cY$ on $\cY$ endowing $\cY$ with a Banach space structure, such that for all $ s\ge 0$, $(\cY,\|\cdot\|_\cY)$ is \textit{$\cL_s$-admissible} and $(\cL_s)_{s\ge 0}$ is stable on $(\cY,\|\cdot\|_\cY)$\,;
    \item[$(3)$\hspace{1ex}] Finally, the map $s\mapsto\cL_s\in\cB((\cY,\|\cdot\|_\cY),(\cX,\|.\|_\cX))$ is uniformly continuous.
\end{itemize}
Under these assumptions, \cite[Thm.~3.1]{Pazy.1983} shows the existence of a unique evolution system $(P_{t,s})_{0\le s\le t}$. If one further requests this evolution system to have the following properties
\begin{itemize}
    \item[$(4)$] $P_{t,s} \cY \subseteq \cY$ for $0\le s\le t$; 
    \item[$(5)$] $(s, t) \mapsto P_{s,t}$ is strongly continuous on $(\cY, \norm{\cdot}_{\cY})$;
\end{itemize}
one obtains the following theorem:
\begin{thm}[Time-dependent semigroups \texorpdfstring{\cite[Thm.~3.1,~4.3]{Pazy.1983}}{}]\label{thm:time-dependent-semigroups}
    Let $\{(\cL_s,\cD(\cL_s)\}_{s\ge 0}$ be a family of generators of $C_0$-semigroups, which satisfy assumption $(1-3)$. Then, there exists a unique evolution system which satisfies
    \begin{itemize}
        \item[$-$] $\|P_{t,s}\|_{\cX\rightarrow\cX}\leq c\,e^{(t-s)\omega}$ for all $0\le s\le t$;
        \item[$-$] $\lim_{t\downarrow s}\frac{1}{t-s}(P_{t,s}x-x)=\cL_sx$ for all $x\in \cY$; and
        \item[$-$]$\frac{\partial}{\partial s}P_{t,s}x=-P_{t,s}\cL_sx$ for all $x\in\cY$ and $0\le s\le t$.
    \end{itemize}
    The two limits above are both with respect to the topology induced by $\|.\|_\cX$. If further (4) and (5) hold then for every $v\in \cY$, $P_{t,s}v$ is a unique solution for the initial value problem in \Cref{eq:time-dep-master-equation} in $(\cY,\|.\|_\cX)$.
\end{thm}

\begin{rmk*}[Kato's \texorpdfstring{$C^1$}{C1}-condition]
    For a time-independent domain $\cD$, the above conditions directly follow if $\{(\cL_s,\cD)\}_{ s\ge 0}$ is a stable family of generators and if $s\mapsto\cL_s$ is strongly continuously differentiable w.r.t.~$\|.\|_\cX$ \cite[Chap.~5, Thm.~4.8]{Pazy.1983}.
\end{rmk*}

Finally, we discuss the general idea of how to control perturbed semigroups by means of certain admissible subsets associated with the domain of an invertible operator. More explicit variants are postponed to \Cref{sec:example-perturbation-bounds}. 
\begin{thm}\label{thm:semigroup-perturbation}
    Let $(\cL,\cD(\cL))$ and $(\cL+ \cK,\cD(\cL+\cK))$ be two generators of $C_0$-semigroups on $\cX$, for an operator $(\cK,\cD(\cK))$. Moreover, let $(\cW,\cD(\cW))$ be an invertible operator on $\cX$ with bounded inverse, such that $\cD(\cW)$ is an $\cL+\cK$-admissible subspace {(see \Cref{defi:admissible-spaces})} and such that $\cK\cW^{-1}$ is bounded. Then, for all $t\ge 0$,
    \begin{equation*}
        \|e^{t\cL}-e^{t(\cL+\cK)}:\cW\to\cX\|\leq {t} \,\|\cK\cW^{-1}\|_{\cX\to \cX}\; \int_{0}^1\| e^{(1-s)t\cL}\|_{\cX\to \cX}\;\|e^{st(\cL+\cK)}\|_{\cW\to \cW}\;ds\,.
    \end{equation*}
    In particular, for all $t\ge 0$  and $x\in \cD(\cW)$, the following equation holds in the Bochner sense:
    \begin{equation*}
        (e^{t\cL}-e^{t(\cL+\cK)})x= {t} \int_{0}^1 e^{(1-s)t\cL} \cK e^{st(\cL+\cK)}x\;ds.
    \end{equation*}
\end{thm}
\begin{proof}
    See \Cref{thm-appx:semigroup-perturbation}.
\end{proof}
\begin{rmk}
    In words, Theorem \ref{thm:semigroup-perturbation} shows that the integral equation for semigroups is well-defined by generalizing the standard method that requires the following relative boundedness condition (see e.g.~\cite[Chapter 2]{Kato.1995}): for all $x\in \cD(\cL+\cK)$, $x\in\cD(\cK)$ and
    \begin{equation*}
        \|\cK x\|_{\cX}\leq \|(\cL+\cK) x\|_{\cX}.
    \end{equation*}
    Indeed, the choice $\cW\coloneqq \1-(\cL+\cK)$ with the resolvent $R(1,\cL+\cK)$ as its bounded inverse shows that  our scheme is a generalization of the above. Clearly, $\cW$ generates an admissible subspace and $\|\cK \cW^{-1}x\|_{\cX}\leq \|x\|_{\cX}$ shows the implication. Note also that the above bound can be extended to evolution systems,
\end{rmk}

\subsection{Continuous variable quantum systems}\label{subsec:bosonic-systems}

An important feature in quantum physics is that of the indistinguishability of particles which results in the introduction of Bosonic and Fermionic particles \cite[Chap.~5.2]{Bratteli.1981}. In the second quantization formalism, a Bosonic or continuous variable quantum system can be described by the Fock space $\cH=L^2(\mathbb{R})$ endowed with an orthonormal (Fock) basis $\{\ket{n}\}_{n=0}^\infty$, where $n$ labels the number of photons present in a given mode. The space of vectors with finite support is denoted by $\cH_f = \{\ket{\psi}\in\cH: \exists M\in\N\,:\,\ket{\psi}= \sum_{n=0}^M \braket{n\,,\psi}\ket{n}\}$, where $\langle \varphi,\psi\rangle$ denotes the standard inner product on $L^2(\mathbb{R})$. Next, we define the \textit{annihilation} and \textit{creation} operators through the following relations
\begin{equation*}
    a\ket{n}=\sqrt{n}\ket{n-1},\quad a\ket{0}=0,\quad \ad\ket{n}=\sqrt{n+1}\ket{n+1}\,.
\end{equation*}
The operators $a$ and $a^\dagger$ satisfy the \textit{canonical commutation relation} (CCR), i.e.~$[a,\ad]=\1$ on $\cH_f$. We can construct the number operator of the latter two as,
\begin{equation}\label{eq:number-operator}
    \Nind = \ad a = \sum_{n=0}^\infty n\ketbra{n}{n} \,.
\end{equation}
It counts the number of photons in a mode. All of $a$, $\ad$, and $N$, although linear, are unbounded operators, hence are only defined on a (dense) subset of $\cH$, namely
\begin{equation}
    \cD(a^\dagger)=\cD(a)=\{\ket{\phi}\in\cH:\|a\ket{\phi}\|<\infty\}=\{\ket{\phi}=\sum_{n=0}^\infty\lambda_n\ket{n}:\sum_{n=0}^\infty n|\lambda_n|^2<\infty\}=\cD(N^{\frac{1}{2}})\,.
\end{equation}
Note that the above domains are {maximal}, i.e.~$\cD(a)=\{\ket{\phi}\in\cH:a\ket{\phi}\in \cH\}$.
In most parts of the paper, we consider operators constructed by polynomials $p\in\C[X,Y]$ of $a$, $\ad$ where the variables $X$ and $Y$ are considered non-commuting, i.e. $XY$ is a different polynomial then $YX$. Using the CCR, we can always assume without loss of generality that the polynomial has the following normal form: there exist complex coefficients $\lambda_{ij}$ and $\mu_{kl}$ such that
\begin{equation}\label{eq:ccr-polynomial-representation}
    p(a\,,\ad)=\sum_{i + 2j \le \deg(p)}\lambda_{ij}(\ad)^iN^j+\,\sum_{k + 2l\le \deg(p)}\mu_{kl}N^la^k\,.
\end{equation}
One possible domain of these operators can be described by the degree $d$ of $p$ (see \Cref{sec-appx:annihilation-creation}):
\begin{equation}\label{eq:domain-ccr-polynomial}
    \cD(p(a,\ad))=\cD(N^{d/2}).
\end{equation}
Next, we add the number operator to the power of twice the leading order to the polynomial, i.e.
\begin{equation}\label{eq:polynomial+number-op}
    \Tilde{p}(a,\ad)\coloneqq (N+\1)^{2d}+p(a,\ad)
\end{equation}
which allows us to show that the domain is maximal in the sense that the operator is closed. Note that he choice of $(N+\1)^{2d}$ is adapted to the proof of \Cref{lem:semigroup-of-G-for-positivity}. Moreover, we prove that $\cH_f$ is a core for the considered polynomial (cf.~\cite[Sec.~7.1]{Simon2015}).

\begin{lem}[Adjoint and core of polynomials of $a, \ad$]\label{lem:formal-polynomial-ccr-adjoint-core}
    Let $p\in\C[X,Y]$ be a polynomial on $\C$ and $(p(a,\ad),\cD(N^{d/2}))$ the unbounded operator in normal form \eqref{eq:domain-ccr-polynomial}. Then, $p(a,\ad)$ is closable and there is a $c\geq0$ such that for all $\phi\in\cD(N^{d/2})$
    \begin{equation*}
        \|p(a,\ad)\ket{\phi}\|\leq c\|(\1+N)^{d/2}\ket{\phi}\|\,.
    \end{equation*}
    The modification $\tilde{p}(a,\ad)=(N+\1)^{2d}+p(a,\ad)$ is closed with domain $\cD(\Tilde{p}(a,\ad))=\cD(\Tilde{p}(a,\ad)^\dagger)=\cD(N^{2d})$ and core $\cH_f$.
\end{lem}
\begin{proof}
   See \Cref{lem-appx:formal-polynomial-ccr-adjoint-core}.
\end{proof}

\begin{rmk*}
    The above lemma will allow us to reduce the analysis of the unbounded operator $p(a,\ad)$ in the strong topology to that on finite sums.  
\end{rmk*}

We end this preliminary section by introducing a family of weighted Banach spaces which we coin as \textit{Bosonic Sobolev spaces} in analogy with classical harmonic analysis. The Bosonic Sobolev space of order $k\in {\mathbb{R}_+}$ is defined on
\begin{equation*}
    \cD(\cW^k)=\{(\cW^k)^{-1}(x) \in\cT_{1,\,\operatorname{sa}} \;:\; x \in \cT_{1,\,\operatorname{sa}}\}
\end{equation*}
via
\begin{equation}\label{eq:bosonic-symmetric-weight}
    \cW^k(x)\coloneqq(\1+N)^{k/4} x (\1+N)^{k/4}\,.
\end{equation}

Since the inverse $(\cW^k)^{-1}(x)=(\1+N)^{-k/4} x (\1+N)^{-k/4}$ is a bounded operator, $(\cD(\cW^k), \|\cdot\|_{\cW^k})$ is a Banach space by \Cref{thm:weighted-banach-space}. For the sake of notation, we define
\begin{equation}\label{eq:sobolev-space}
    W^{k,1}\coloneqq\cD(\cW^k)\quad\text{and}\quad\|\cdot\|_{W^{k,1}}\coloneqq\|\cdot\|_{\cW^k}.
\end{equation}
For $k=0$, $\cW^k=\1$ and $(\cD(\cW^0),\|\cdot\|_{W^{0,1}})=(\cT_{1,\,\operatorname{sa}},\|\cdot\|_1)$. 

\begin{lem}\label{lem:sobolev-embedding}
    Let ${k < k' \in \mathbb{R}_+ := [0, \infty)}$. Then, 
    \begin{equation*}
        W^{k',1}\Subset W^{k,1}.
    \end{equation*}
\end{lem}
\begin{proof}
    The proof relies on the abstract \Cref{thm:compact-embedding-weighted-spaces}, which is applied for different values $k\in{\R_+}$ of 
    \begin{equation*}
        \cW^k(x)\coloneqq(\1+N)^{k/4} x (\1+N)^{k/4}.
    \end{equation*}
    For $k'>k$, the operator $\cW^{k}\cW^{-k'}=\cW^{k-k'}$ is bounded. Next we show compactness by proving that $\cW^{-l}$ with $-l=k-k'$ is approximated by a sequence of finite rank operators:
    \begin{equation*}
        \cW^{-l}_{f,M}(x)\coloneqq\sum_n^M(1+n)^{-l/4}\ketbra{n}{n} x \sum_m^M(1+m)^{-l/4}\ketbra{m}{m},
    \end{equation*}
    which can be deduced through Hölder's inequality
    \begin{align*}
        \|\cW^{-l}(x)-\cW^{-l}_{f,M}(x)\|_1&=\|\sum_{m,n>M}(1+n)^{-l/4}\ketbra{n}{n} x (1+m)^{-l/4}\ketbra{m}{m}\|_1\leq M^{-l/2}\|x\|_1.
    \end{align*}
    Since finite rank operators are compact by the Bolzano-Weierstrass theorem and the limit is a compact operator again \cite[Thm.~2.13.4]{Hille.1957}, the operator $\cW^{-l}$ is a compact operator on $\cT_{1,\,\operatorname{sa}}$. Applying \Cref{thm:compact-embedding-weighted-spaces} shows that 
    \begin{equation*}
        W^{k',1}\Subset W^{k,1}\,.\vspace{-2ex}
    \end{equation*}
\end{proof}

The following theorem will become helpful later and has an analogue in the theory of commutative $L_p$ spaces, which inspired its name. Although it can be proved by interpolation theory, we provide a more rudimentary approach using only Hadamard's three-line theorem.

\begin{thm}[Stein-Weiss theorem for Bosonic Sobolev spaces]\label{thm:stein-weiss}
    Let $k_0 < k_1 \in \R_+$ and $T: W^{k_j, 1} \to W^{k_j, 1}$, be a linear map with $\norm{T}_{W^{k_j, 1} \to W^{k_j, 1}} \le M_j$ for some $M_j \ge 0$, $j = 1, 2$. Then for $\theta \in [0, 1]$, $T: W^{k_\theta, 1} \to W^{k_\theta, 1}$ with $k_\theta = (1-\theta) k_0 + \theta  k_1$ obtained by restriction of the input $T:W^{k_0, 1} \to W^{k_0, 1}$ to $W^{k_\theta, 1} \cap W^{k_0, 1}$ is a well defined bounded linear map with
    \begin{equation}
        \norm{T}_{W^{k_\theta, 1} \to W^{k_\theta, 1}} \le M_0^{1-\theta} M_1^{ \theta} \, . 
    \end{equation}
\end{thm}
\begin{proof}
    We have that $k_0 < k_1$ and hence $W^{k_1, 1} \Subset W^{k_0, 1}$. Let $k_\theta = (1-\theta) k_0 + \theta k_1$ with $\theta \in (0, 1)$ and $x \in \cT_f \subseteq W^{k_1, 1} \Subset W^{k_\theta, 1} \Subset W^{k_0, 1}$. We will show
    \begin{equation}
        \norm{T(x)}_{W^{k_\theta, 1}} \le M_0^{1-\theta} M_1^{\theta} \norm{x}_{W^{k_\theta, 1}}
    \end{equation}
    which proves that $T$ can be uniquely extended to a bounded linear map on $W^{k_\theta, 1}$ that agrees with the restriction of $T:W^{k_0, 1} \to W^{k_0, 1}$, to $W^{k_\theta, 1} \cap W^{k_0, 1}$. This agreement on intersections is due to the compact embeddings of the Bosonic Sobolev spaces into one another. For $x \in \cT_f$ and $z \in S \coloneqq \{z \in \C \;:\; 0 \le \Re(z) \le 1\}$, we define $k(z) = (1-z) k_0 + z k_1$ so that for a fixed $Z \in \cB(\cH)$ with $\norm{Z}_\infty \le 1$,
    \begin{equation}
        \begin{aligned}
             g: S &\to \C,\,g(z) = \tr[(N + \1)^{\frac{k(z)}{4}} T\Big((N + \1)^{\frac{k_\theta - k(z)}{4}} x (N + \1)^{\frac{k_\theta - k(z)}{4}}\Big) (N + \1)^{\frac{k(z)}{4}} Z]
        \end{aligned}
    \end{equation}
    is well-defined, uniformly bounded, and continuous on $S$ (q.v.~\Cref{lem:continuity-G}) and further holomorphic on $\mathring{S} \coloneqq \{z \in \C \;:\; 0 < \Re(z) < 1\}$ (q.v.~\Cref{lem:differentiability-G}). Note that for $\theta\in(0,1)$
    \begin{equation}
        |g(\theta)| = \left|\tr[(N + \1)^{\frac{k_\theta}{4}} T(x) (N + \1)^{\frac{k_\theta}{4}} Z]\right| \, ,
    \end{equation}
    for $t\in\R$
    \begin{equation}\label{eq:g(it)}
        \begin{aligned}
            |g(it)| &= \left|\tr\left[(N + \1)^{\frac{k_0 + it(k_1 - k_0)}{4}}T\Big((N + \1)^{\frac{k_\theta - k(it)}{4}} x (N + \1)^{\frac{k_\theta - k(it)}{4}}\Big) (N + \1)^{\frac{k_0 + it(k_1 - k_0)}{4}}Z\right]\right|\\
            &\le \norm{T\Big((N + \1)^{\frac{k_\theta - k_0 + it(k_0 - k_1)}{4}} x (N + \1)^{\frac{k_\theta - k_0 + it(k_0 - k_1)}{4}}\Big)}_{W^{k_0, 1}} \norm{Z}_\infty\\
            &\le \norm{T}_{W^{{k_0}, 1} \to W^{{k_0}, 1}} \norm{(N + \1)^{\frac{k_\theta - k_0}{4}} x (N + \1)^{\frac{k_\theta - k_0}{4}}}_{W^{k_0, 1}}\\
            &= M_0 \norm{x}_{W^{k_\theta, 1}}\,,
        \end{aligned}
    \end{equation}
    and similarly
    \begin{equation}\label{eq:g(it + 1)}
        \begin{aligned}
            |g(1 + it)| & \le \norm{T\Big((N + \1)^{\frac{k_\theta - k_1 + it(k_0 - k_1)}{4}} x (N + \1)^{\frac{k_\theta - k_1 + it(k_0 - k_1)}{4}}\Big)}_{W^{k_1, 1}}\norm{Z}_\infty\\
            &\le \norm{T}_{W^{k_1, 1} \to W^{k_1, 1}} \norm{(N + \1)^{\frac{k_\theta - k_1 + it(k_0 - k_1)}{4}} x (N + \1)^{\frac{k_\theta - k_1 + it(k_0 - k_1)}{4}}}_{W^{k_1, 1}}\\
            &=M_1 \norm{x}_{W^{k_\theta, 1}} \, . 
        \end{aligned}
    \end{equation}
    An application of Hadamard's three-lines theorem now gives us that for $G_0 \coloneqq \sup\limits_{t \in \R} |g(it)|$ and $G_1 \coloneqq \sup\limits_{t \in \R} |g(it + 1)|$
    \begin{equation}
        |g(\theta)| \le G_0^{1-\theta} G_1^{ \theta} \le M_0^{1-\theta} M_1^{ \theta} \norm{x}_{W^{k_\theta, 1}} \, ,
    \end{equation}
where the last inequality follows from the bounds in \Cref{eq:g(it)} and \Cref{eq:g(it + 1)}. Since $Z$ was arbitrary, we can deduce that 
    \begin{equation}
        \begin{aligned}
            \norm{T(x)}_{W^{k_\theta, 1}} &= \sup\left\{ \left|\tr[(N + \1)^{\frac{k_\theta}{4}} T(x) (N + \1)^{\frac{k_\theta}{4}} Z ]\right| \;:\; \norm{Z}_\infty \le 1 \right\} \\
            &\le M_0^{1-\theta} M_1^{ \theta} \norm{x}_{W^{k_\theta, 1}}
        \end{aligned}
    \end{equation}
    where we used the dual characterisation of $\norm{\cdot}_1$. This concludes the claim.
\end{proof}

With the Sobolev embedding at hand, we introduce the notion of a \textit{Sobolev preserving semigroup} as a semigroup defined on a sequence of Bosonic Sobolev spaces.

\begin{defi}[Sobolev preserving evolution system]\label{defi:sobolev-preserving-semigroups}
    Let $(\cP_t)_{t\ge 0}$ be a $C_0$-semigroup on $\cT_{1,\operatorname{sa}}$. We then call $(\cP_t)_{t\ge 0}$ \textit{Sobolev preserving} if there exists a divergent sequence $\{k_r\}_{r \in \N} \to \infty$, such that for all $r \in \N$, $W^{k_r, 1}$ is an admissible subspace for $(\cP_t)_{t\ge 0}$. Similarly for $(\cP_{t,s})_{0\leq s\leq t}$ an evolution system on $\cT_{1,\operatorname{sa}}$, we call $(\cP_{t,s})_{0\leq s\leq t}$ \textit{Sobolev preserving} if for all $r \in \N$, $W^{k_r, 1}$ is admissible for $(\cP_{t,s})_{0\leq s\leq t}$ to $W^{k_r, 1}$.
\end{defi}
Note that with the Stein-Weiss theorem for Bosonic Sobolev spaces, \Cref{thm:stein-weiss}, one can immediately interpolate a semigroup and evolution system defined on $W^{k_0, 1}$ and $W^{k_1, 1}$ to $W^{k_\theta, 1}$ with $k_\theta = (1 - \theta)k_0  + \theta k_1$, $\theta \in [0, 1]$ (q.v.~\Cref{lem:interpolation-lemma}). This means the above definition is equivalent to the definition, requiring that for all $k \in \R_+$, $W^{k, 1}$ are admissible subspaces of the semigroup or evolution system respectively.

The following example shows that not every semigroup is Sobolev preserving:
\begin{ex}
    An example of a $C_0$-semigroup that is not Sobolev preserving is the depolarizing semigroup, i.e.~$\cP_t(\rho) = e^{-t} \rho + (1 - e^{-t})\tr[\rho] \sigma$ where $\sigma$ is a quantum state with $\tr[N^{\frac{1}{2}}\sigma N^{\frac{1}{2}}] = \infty$. Then for a quantum state $\rho \in W^{2, 1}$ we find that 
    \begin{equation*}
        \norm{\cP_t(\rho)}_{W^{2, 1}} = \infty \quad \qquad \forall t > 0 \, .
    \end{equation*}
\end{ex}