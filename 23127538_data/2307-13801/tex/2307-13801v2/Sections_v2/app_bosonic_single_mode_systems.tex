In this section, we prove some basic properties of polynomials of annihilation and creation operators. We shortly repeat the normal form of our polynomials in $a$ and $\ad$ given in \Cref{eq:ccr-polynomial-representation}: 
\begin{equation*}
    p(a\,,\ad)=\sum_{i + 2j \le \deg(p)}\lambda_{ij}(\ad)^iN^j+\,\sum_{k + 2l\le \deg(p)}\mu_{kl}N^la^k
\end{equation*}
with coefficients $\lambda_{ij}, \mu_{kl}\in\C$. The modification considered in our proof (see \Cref{lem:semigroup-of-G-for-positivity}) is given in \Cref{eq:polynomial+number-op} by
\begin{equation*}
    \Tilde{p}(a,\ad)\coloneqq N^{2d}+p(a,\ad)\,.
\end{equation*}
Then, we start proving a simple representation of a domain of $p(a,\ad)$ and $\Tilde{p}(a,\ad)$ which extends the domain $\cH_f$: For $n\in\N$
\begin{equation}\label{eq:domain-poly}
    \begin{aligned}
        p(a\,,\ad)\ket{n}&=\sum_{i+2j\leq d}\lambda_{ij}n^{j}\sqrt{i!\binom{n+i}{i}}\ket{n+i}+\sum_{k+2l\leq d}\mu_{kl}(n-k)^l\sqrt{k!\binom{n}{k}}\ket{n-k}
    \end{aligned}
\end{equation}
where $d\coloneqq\deg(p)$. This directly implies for $|\phi\rangle=\sum_n\phi_n|n\rangle\in\cD(N^{d/2})$
\begin{equation*}
    \begin{aligned}
        p(a\,,\ad)\ket{\phi}&=\sum_{n=0}^\infty\sum_{i+2j\leq d}\phi_n\lambda_{ij}n^{j}\sqrt{i!\binom{n+i}{i}}\ket{n+i}+\sum_{n=0}^\infty\sum_{k+2l\leq d}\phi_n\mu_{kl}(n-k)^l\sqrt{k!\binom{n}{k}}\ket{n-k}\\
        &=\sum_{n=0}^\infty\sum_{i+2j\leq d}\left(\phi_{n-i}\lambda_{ij}(n-i)^{j}\sqrt{i!\binom{n}{i}}+\phi_{n+i}\mu_{ij}n^j\sqrt{i!\binom{n+i}{i}}\right)\ket{n}\,.\\
    \end{aligned}
\end{equation*}
Then, the leading order of the summands in $n$ is maximal of order $d/2$ so that $\cD(N^{d/2})$ is a domain of $p(a,\ad)$. For the modified polynomial $\Tilde{p}$, there is sequence of functions $R_n:\cH \rightarrow\R$ with asymptotics strictly smaller than $n^{4d}$ such that
\begin{equation}\label{eq:leading-order-poly}
    \|\tilde{p}(a,\ad)\ket{\phi}\|^{2}=\sum_{n=0}^\infty |\phi_n|^2n^{4d}+R(\phi)\,.
\end{equation}
Having the domain above in mind, we are able to prove that $p$ is closable and $\tilde{p}$ is a closed operator with core $\cH_f$:

\begin{lem}[Adjoint and core of polynomials of $a, \ad$]\label{lem-appx:formal-polynomial-ccr-adjoint-core}
    Let $p\in\C[X,Y]$ be a polynomial on $\C$ and $(p(a,\ad),\cD(N^{d/2}))$ the unbounded operator in normal form \eqref{eq:domain-ccr-polynomial}. Then, $p(a,\ad)$ is closable and there is a $c\geq0$ such that for all $\ket{\phi}\in\cD(N^{d/2})$
    \begin{equation*}
        \|p(a,\ad)\ket{\phi}\|\leq c\|(\1+N)^{d/2}\ket{\phi}\|\,.
    \end{equation*}
    The modification $\tilde{p}(a,\ad)=N^{2d}+p(a,\ad)$ is closed with $\cD(\Tilde{p}(a,\ad))=\cD(\Tilde{p}(a,\ad)^\dagger)=\cD(N^{2d})$ and core $\cH_f$.
\end{lem}
\begin{proof}
    First, note that the relative boundedness w.r.t.~the number operator is a direct consequence of \Cref{eq:domain-poly}.
    To prove that $p(a,\ad)$ is closable, we show that $\cD(N^{d/2})\subset\cD(p(a,\ad)^\dagger)$: we recall that the adjoint is defined via boundedness of the functional 
    \begin{equation*}
        \cD(p(a,\ad))\ni\ket{\phi}\mapsto\braket{p(a,\ad)\phi\,,\varphi}
    \end{equation*}
    for $\phi\in\cD(p(a,\ad)^\dagger)$. Since for all $n,m\in\N$
    \begin{equation*}
        \braket{a\,n\,,m}=\delta_{n,0}\delta_{n-1,m}\sqrt{n}=\delta_{n,m+1}\sqrt{m+1}=\braket{n\,,\ad\,m}
    \end{equation*}
    and by the definition of the domains, we know that for all $\ket{\phi},\ket{\psi}\in\cD(N^{k+\frac{l}{2}})$ 
    \begin{equation*}
        \braket{N^k(\ad)^l \phi\,,\psi}=\braket{\phi\,,a^lN^k\psi}\quad\text{and}\quad\braket{a^lN^k\phi\,,\psi}=\braket{\phi\,,N^k(\ad)^l\psi}\,.
    \end{equation*}
    By sesquilinearity of the scalar product and \Cref{eq:domain-ccr-polynomial}, the above equations hold for all polynomials $p\in\C[X,Y]$ which proves $\cD(N^{d/2})\subset\cD(p(a,\ad)^\dagger)$. Since $\cD(N^{d/2})$ is a dense subspace of $\cH$, Theorem 7.1.1 in \cite{Simon2015} shows that $p(a,\ad)$ is closable. Next, we show that the modified polynomial 
    \begin{equation*}
        (N+\1)^{2d}+p(a,\ad)
    \end{equation*}
    is already closed. Actually, we show $\cD(N^{2d})=\cD(\Tilde{p}(a,\ad)^\dagger)$, i.e.~the domain is maximal, by contradiction. Assume that there exists a $\varphi\in\cD(\Tilde{p}(a,\ad)^\dagger)\backslash\cD(\Tilde{p}(a,\ad))$ and define $P_M$ to be the projection on the first $M$ Fock basis elements. Then, we use the representation in \Cref{eq:ccr-polynomial-representation} so that, denoting by $\Tilde{p}^\dagger$ the polynomial where we took the complex conjugate of the coefficients and swapped the coordinates,
    \begin{equation*}
        P_M\Tilde{p}^\dagger(a,\ad)=P_M(N+\1)^{2d}P_M+P_{M}\left(\sum_{i + 2j \le d}\overline{\lambda}_{ij}N^ja^iP_{M-i}+\,\overline{\mu}_{ij}(\ad)^iN^jP_{M+i}\right)P_{M+d}\,.
    \end{equation*}
    We then can define the state
    \begin{equation*}
        \ket{\phi_M}\coloneqq\frac{P_M\Tilde{p}^\dagger(a,\ad)\ket{\varphi}}{\|P_M\Tilde{p}^\dagger(a,\ad)\ket{\varphi}\|}\in\cH_f
    \end{equation*}
    Next, we use $\phi_M$ to get a lower bound on the operator norm of 
    \begin{equation*}
        \cD(p(a,\ad))\ni\ket{\phi}\mapsto\braket{\Tilde{p}(a,\ad)\phi\,,\varphi}
    \end{equation*}
    by
    \begin{equation*}
        \begin{aligned}
            \sup_{\|\phi\|=1}|\braket{\Tilde{p}(a,\ad)\phi\,,\varphi}|&\geq|\braket{\Tilde{p}(a,\ad)\phi_M\,,\varphi}|=|\braket{\phi_M\,,P_M\Tilde{p}^\dagger(a,\ad)\varphi}|=\|P_M\Tilde{p}^\dagger(a,\ad)\varphi\|.
        \end{aligned}
    \end{equation*}
    Now, by definition of $\Tilde{p}$ and denoting by $\varphi_n$ the coefficients of $|\varphi\rangle$ in the Fock basis,
    \begin{equation*}
        \|P_M\Tilde{p}^\dagger(a,\ad)\varphi\|^2 =\Big\|\sum_{n=0}^{M+d}\varphi_nP_M\left((N+\1)^{2d}+p^\dagger(a,\ad)\right)\ket{n}\Big\|^2\,,
    \end{equation*}
    where $p^\dagger$ is defined similarly to $\tilde{p}^\dagger$. By assumption $\phi\notin\cD(N^{2d})$, so that the above sequence is diverging for $M\rightarrow\infty$ to infinity (see \Cref{eq:leading-order-poly}). This contradicts the assumption and shows $\cD(\tilde{p}(a,\ad)^\dagger)=\cD(N^{2d})$ as well as $\tilde{p}(a,\ad)^\dagger=\tilde{p}^\dagger(a,\ad)$. Moreover, $p(a,\ad)$ is by Theorem 7.1.1 in \cite{Simon2015} a closed operator. Since $\{\ket{n}\}_{n \in \N}$ is an orthonormal basis and $N$ a multiplication operator on that basis, we can immediately conclude that $\cH_f$ is a core for $N$ and further for all $(\1 + N)^k$, $k \ge 0$. Since $\Tilde{p}(a,\ad)$ is closed w.r.t.~$\cD(N^{p/2+1})$, $\cH_f$ is also a core of $\Tilde{p}(a,\ad)$.
\end{proof}

Having in mind that polynomials of the annihilation and creation are closed operators on certain domains, we use the canonical commutation relation $[a,\ad]=\1$ in the following lemma:
\begin{lem}\label{lem:l-ccr}
    Let $l\in\N$, then the following hold on $\cH_f$ and can be extended to maximal domains by taking limits
    \begin{equation}
        \begin{aligned}
            (\ad)^la^l&=(N-(l-1)\1)(N-(l-2)\1)\cdots(N-\1)N\,,\\
            a^l(\ad)^l&=(N+\1)(N+2\1)\cdots(N+(l-1)\1)(N+l\1)\,.
        \end{aligned}
    \end{equation}
\end{lem}
\begin{proof}
    The above equalities can be proven by induction over $l\in\N$. The cases $l\in\{0,1\}$ are trivial by definition. Next assume that the equation holds for $l\in\N$, then by \Cref{eq:symmetry-function}
    \begin{equation*}
        \begin{aligned}
            (\ad)^{l+1}a^{l+1}&=(\ad)^{l}Na^{l}=(N-l)(\ad)^{l}a^{l}=(N-l)\cdots N
        \end{aligned}
    \end{equation*}
    which finishes the proof by induction. The second expression can be proven by induction as well, and the induction start is again clear by the CCR. Next, we assume the equation for $l\in\N$. Then, \Cref{eq:symmetry-function} shows
    \begin{equation*}
        a^{l+1}(\ad)^{l+1}=a^{l}(N+\1)(\ad)^{l}=a^{l}(\ad)^{l}(N+(l+1)\1)=(N+\1)\cdots(N+(l+1)\1)
    \end{equation*}
    which completes the induction.
\end{proof}

\begin{lem}\label{lem:two-point-hamiltonian-bound}
    Let $\ell_1,\ell_2,k_1,k_2\in\N$ with $\min\{\ell_1,k_1\}=\min\{\ell_2,k_2\}=0$, $z\in\C$, and $h:\mathbb{N}^2\rightarrow\mathbb{R}$ a positive function that is increasing in each of its variables. Then,
    \begin{equation*}
        \begin{aligned}
            &za_1^{\ell_1}a_2^{\ell_2}h(N_1,N_2)(\ad_1)^{k_1}(\ad_2)^{k_2}+\overline{z}a_1^{k_1}a_2^{k_2}h(N_1,N_2)(\ad_1)^{\ell_1}(\ad_2)^{\ell_2}\\
            &\qquad\qquad\qquad\qquad \leq 2|z|\widetilde{h}_{m_1,m_2}(N_1+m_1I,N_2+m_2I)\,,
        \end{aligned}
    \end{equation*}
    where $m_1\coloneqq\max\{\ell_{1},k_{1}\}$, $m_2\coloneqq \max\{\ell_2,k_2\}$ and $$\widetilde{h}_{m_1,m_2}(n_1,n_2)={\prod_{j=n_1-m_1+1}^{n_1} \sqrt{j}\prod_{i=n_2-m_2+1}^{n_2}\sqrt{i}} \,\,\,h(n_1,n_2)\,1_{n_1\ge m_1}1_{n_2\ge m_2}\,,$$
    where we introduced the notation $1_{x\ge m}$ for the indicator function on the set $\{x:\,x\ge m\}$, and where by convention we take $\prod_{i=a}^b=1$ when $a>b$.
\end{lem}
\begin{proof}
    We define $K\coloneqq za_1^{\ell_1}a_2^{\ell_2}h(N_1,N_2)(\ad_1)^{k_1}(\ad_2)^{k_2}+\overline{z}a_1^{k_1}a_2^{k_2}h(N_1,N_2)(\ad_1)^{\ell_1}(\ad_2)^{\ell_2}$ and represent it in the $2$-mode Fock basis:
    \begin{equation*}
        \begin{aligned}
            K&=\sum_{n_1,n_2}\,h(n_1,n_2)\,  \big( z a_1^{\ell_1}a_2^{\ell_2}\,\ketbra{n_1,n_2}{n_1,n_2}(a_1^\dagger)^{k_1}(a_2^\dagger)^{k_2}+\overline{z} a_1^{k_1}a_2^{k_2}\ketbra{n_1,n_2}{n_1,n_2}(a_1^\dagger)^{\ell_1}(a_2^\dagger)^{\ell_2} \big)\\
            &=\sum_{\substack{n_1\geq m_1\\n_2\geq m_2}} g_{\substack{\ell_1,\ell_2\\k_1,k_2}}(n_1,n_2)\\
            &\qquad \qquad \left(z\ketbra{n_1-\ell_1,n_2-\ell_2}{n_1-k_1,n_2-k_2}+\overline{z}\ketbra{n_1-k_1,n_2-k_2}{n_1-\ell_1,n_2-\ell_2}\right)\,, 
        \end{aligned}
    \end{equation*}
    where 
    \begin{equation*}
        \begin{aligned}
            &g_{\substack{\ell_1,\ell_2\\k_1,k_2}}(n_1,n_2)\\
            &\qquad \coloneqq h(n_1,n_2) \sqrt{n_1\dots(n_1-\ell_1+1)n_1\dots (n_1-k_1+1)n_2\dots (n_2-\ell_2+1)n_2\dots (n_2-k_2+1)}\,. 
        \end{aligned}
    \end{equation*}
    By assumption, since $\min\{\ell_1,k_1\}=\min\{\ell_2,k_2\}=0$, we have that $g_{\substack{\ell_1,\ell_2\\k_1,k_2}}(n_1,n_2)=\widetilde{h}_{m_1,m_2}(n_1,n_2)$ for $n_1\ge m_1$, $n_2\ge m_2$, and 

    \begin{align*}
        &K=\sum_{n_1,n_2\in \mathbb{N}}\widetilde{h}_{m_1,m_2}(n_1+m_1,n_2+m_2)\\
        &\qquad \qquad\left(z\ketbra{n_1+k_1,n_2+k_2}{n_1+\ell_1,n_2+\ell_2}+\overline{z}\ketbra{n_1+\ell_1,n_2+\ell_2}{n_1+k_1,n_2+k_2}\right)\,.
    \end{align*}
    Next, we consider the constituents of the above sum individually. Note that the operator
    \begin{equation}\label{eq:block-matrix}
        \begin{aligned}
            z\ketbra{n_1+k_1,n_2+k_2}{n_1+\ell_1,n_2+\ell_2}+\overline{z}\ketbra{n_1+\ell_1,n_2+\ell_2}{n_1+k_1,n_2+k_2}
        \end{aligned}
    \end{equation}
    \begin{equation*}
        \left(\begin{array}{*{6}{c}}
        \tikzmark{left}0 &0 &\ast &0 &0 &0 \\               
        0 &0 &0 &\ast &0 &0 \\      
        \ast &0 &0\tikzmark{right} &0 &\ast &0 \\\DrawBox[thick]
        0 &\ast &0 &\tikzmark{left}0 &0 &\ast \\
        0 &0 &\ast &0 &0 &0 \\
        0 &0 &0 &\ast &0 &0\tikzmark{right} \\\DrawBox[thick]
        \end{array}\right)\,.
    \end{equation*}
    can be embedded into an operator on an two dimensional space of the form 
    \begin{equation*}
        z\ketbra{e_1}{e_2} + \overline{z} \ketbra{e_2}{e_1} \, , 
    \end{equation*}
    where $\ket{e_1}$ and $\ket{e_2}$ are orthonormal vectors. For $\ket{e_1}=\ket{e_2}$, $z+\overline{z}\leq2|z|$ shows 
    \begin{equation*}
         z\ketbra{e_1}{e_2} + \overline{z} \ketbra{e_2}{e_1} \le |z|(\ketbra{e_1}{e_1} + \ketbra{e_2}{e_2})\, . 
    \end{equation*}
    In the case $\ket{e_1}\neq\ket{e_2}$, we have
    \begin{center}
        \begin{tabular}{ c c }
            Eigenvalue & Eigenvectors \\[0.5ex]\hline
            $|z|$ & $\ket{\psi}=\frac{1}{\sqrt{2}|z|}(|z|\ket{e_1}+z\ket{e_{2}})$\\
            $-|z|$ & $\ket{\varphi}=\frac{1}{\sqrt{2}|z|}(|z|\ket{e_1}-z\ket{e_{2}})$
        \end{tabular}.
    \end{center}
    so that 
    \begin{equation*}
        z\ketbra{e_{2}}{e_1}+\overline{z}\ketbra{e_1}{e_{2}}=|z| \ketbra{\psi}{\psi} - |z| \ketbra{\varphi}{\varphi}\leq|z|\ketbra{\psi}{\psi}\leq|z| (\ketbra{e_1}{e_1} + \ketbra{e_2}{e_2})\,.
    \end{equation*}
    This allows us to estimate
    \begin{align*}
        K &\le \sum\limits_{n_1, n_2 \in \N} \widetilde h_{m_1, m_2}(n_1 + m_1, n_2 + m_2) |z| (\ketbra{n_1 + k_1, n_2 + k_2}{n_1 + k_1, n_2 + k_2}\\
        &\hspace{7cm} + \ketbra{n_1 + l_1, n_2 + l_2}{n_1 + l_1, n_2 + l_2})\\
        &\le 2 |z| \widetilde h_{m_1, m_2}(N_1 + m_1 I, N_2 + m2 I)
    \end{align*}
    employing the monotonicity of $\widetilde h_{m_1, m_2}$ in both arguments in the last step.
    %\tm{Next, we split the above sum into sums of matrix blocks as depicted in the following matrix in the Fock basis:
    %\begin{equation*}
    %    \left(\begin{array}{*{6}{c}}
    %    \tikzmark{left}0 &0 &\ast &0 &0 &0 \\               
    %    0 &0 &0 &\ast &0 &0 \\      
    %    \ast &0 &0\tikzmark{right} &0 &\ast &0 \\\DrawBox[thick]
    %    0 &\ast &0 &\tikzmark{left}0 &0 &\ast \\
    %    0 &0 &\ast &0 &0 &0 \\
    %    0 &0 &0 &\ast &0 &0\tikzmark{right} \\\DrawBox[thick]
    %    \end{array}\right)\,.
    %\end{equation*}
    %For fixed $j_{1}, j_2$ and $n_{1},n_2\in\N$, the operator 
    %\begin{equation}\label{eq:block-matrix}
     %   \begin{aligned}
     %       z\ketbra{n_1+k_1,n_2+k_2}{n_1+\ell_1,n_2+\ell_2}+\overline{z}\ketbra{n_1+\ell_1,n_2+\ell_2}{n_1+k_1,n_2+k_2}
     %   \end{aligned}
    %\end{equation}
    %is of the form 
    %\pg{Why aren't we considering just $e_0, e_1$ instead of a $m$-fold basis? The strategy work completely analogous.}
    %\begin{equation}\label{eq:rank2}
    %    z\ketbra{e_{m-1}}{e_0}+\overline{z}\ketbra{e_0}{e_{m-1}}\,,
    %\end{equation}
    %where $\{e_j\}_{j=0}^{m-1}$ is an orthonormal basis of an $m$ dimensional Hilbert space. Then, the above operator has the following eigenvalues and eigenvectors:
    %\begin{center}
    %    \begin{tabular}{ c c c }
    %        Eigenvalue & Eigenvectors & Multiplicity \\[0.5ex]\hline
    %        $0$ & $\{e_1,...,e_{m-2}\}$ & $m-2$ \\  
    %        $|z|$ & $|z|\ket{e_0}+z\ket{e_{m-1}}$ & $1$\\
    %        $-|z|$ & $|z|\ket{e_0}-z\ket{e_{m-1}}$ & $1$
    %    \end{tabular}.
    %\end{center}
    %Defining $m=(m_1+1)(m_2+1)$, the matrix given in \Cref{eq:block-matrix} has the  eigenvalues $\{-|z|,0,|z|\}$, and can therefore be upper bounded as
    %\begin{align*}
    %    z\ketbra{e_{m-1}}{e_0}+\overline{z}\ketbra{e_0}{e_{m-1}}\le |z|\, \sum_{i=0}^{m-1}\ketbra{e_i}{e_i}\,.
    %\end{align*}
    %Since each of the projections upper bounding any operator of the form \eqref{eq:rank2} appears $M$ times in the sum, we find
    %\begin{equation*}
     %   \begin{aligned}
     %       K&\le \sum_{n_1,n_2\in \mathbb{N}}\widetilde{h}_{m_1,m_2}(n_1+m_1,n_2+m_2)|z|\sum_{i=0}^{m_1}\sum_{j=0}^{m_2}\ketbra{n_1+i,n_2+j}{ n_1+i,n_2+j}\\
     %       &\leq (m_1+1)(m_2+1)\,|z|\,\widetilde{h}_{m_1,m_2}(N_1+m_1I,N_2+m_2I)\,,
     %   \end{aligned}
    %\end{equation*}
    %where we also used that $\widetilde{h}$ is increasing in each of its variables.}
\end{proof}