Quantum processes are often described in physics by their infinitesimal action during short-time intervals. When the process at a given time $t$ can be assumed to be independent of previous times, an assumption often referred to as the memorylessness condition, the resulting dynamics can be formally described via a so-called master equation. The latter is an initial value problem of the form
\begin{equation*}
    \frac{\partial}{\partial t} \rho(t)=\cL(\rho(t))\,,\qquad \rho(0)=\rho_0\,.
\end{equation*}
In the case of uniformly continuous dual dynamics over the algebra $\mathcal{B}(\cH)$ of bounded operators over a Hilbert space $\cH$ (and in fact more generally on an arbitrary von Neumann algebra), the seminal results by \citeauthor{Lindblad.1976} and \citeauthor{Gorini.1976} classify the generators of a quantum dynamical semigroup in terms of the following so-called GKSL form: $\cL$ is a bounded operator on the space $\cT_1(\cH)$ of trace-class operators that satisfies
\begin{equation}\label{eqGKSL}
    \cL(\rho) = - i[H, \rho] + \sum\limits_{j = 1}^K L_j \rho L_j^\dagger - \frac{1}{2}\{L_j^\dagger L_j, \rho\}\, 
\end{equation}
for a Hamiltonian $H=H^\dagger\in\cB(\cH)$ and so-called Lindblad operators $L_j\in\cB(\cH)$, where $\{A,B\}\coloneqq AB+BA$ denotes the anticommutator of two bounded operators $A,B\in\cB(\cH)$. In other words, the state at time $t\ge 0$ is described as $e^{t\cL}(\rho)$, where the exponential can be defined e.g.~in terms of its converging Taylor series. In that case, the set $(e^{t\cL})_{t\ge 0}$ defines a quantum Markov semigroup (QMS), which is a time-continuous family of completely positive, trace-preserving maps. 

Since their introduction, QMS have become a standard tool and were extensively studied in various areas of mathematical physics and quantum information processing. Unfortunately, an extension of the GKSL form \eqref{eqGKSL} is known to fail for strongly continuous evolutions in general and thus requires additional assumptions. Conversely, unbounded operators satisfying an equation like \eqref{eqGKSL} on a suitable domain can fail at generating quantum Markovian dynamics. Simple counterexamples can be constructed in the context of continuous variable (CV) quantum systems over $\cH=L^2(\cH)$ as follows: denoting the creation and annihilation operators associated with a harmonic oscillator by $a^\dagger$ and $a$, respectively, the $2$-photon pure birth process formally defined as in \eqref{eqGKSL} with $K=1$, $H=0$ and $L_1=(a^\dagger)^2$ leads to a semigroup satisfying the master equation, but failing at preserving the trace \cite[Example 3.3.]{Davies.1977}. Similar problems were encountered later on by Fagnola et al.~\cite{Chebotarev.1998}, who considered the problem in the Heisenberg instead of the Schrödinger picture. They solved the appearing issues by imposing additional technical conditions on the generators in question. A thorough analysis of semigroups that have GKSL form but fail to be trace-preserving, can be found in \cite{Siemon.2017}, where the authors also discuss the possibility of generators deviating from the GKSL form. 

In contrast, recent years have seen remarkable progress in the use of CV quantum systems. These systems encompass a wide range of applications in various areas of quantum information, including quantum communication \cite{braunstein2005quantum,holevo2001evaluating,wolf2007quantum,takeoka2014fundamental,pirandola2017fundamental,wilde2017converse,rosati2018narrow,lami2022exact}, sensing \cite{aasi2013enhanced,zhang2018noon,meyer2001experimental,mccormick2019quantum} or simulation \cite{flurin2017observing}, enabled by advancements in non-classical radiation sources \cite{ourjoumtsev2006generating,kurochkin2014distillation,huang2015optical,reimer2016generation,eichler2011observation,zhong2013squeezing}.
Given the technological and experimental relevance of CV systems, there is a pressing need for a rigorous mathematical theory of quantum dynamical semigroups over such systems.

One specific area where CV systems governed by a Lindblad master equation like \eqref{eqGKSL} have recently gained significant attention on theoretical as well as experimental grounds is in the field of bosonic quantum error correcting codes
\cite{gottesman2001encoding,Mirrahimi.2014,Guillaud.2019,Joshi.2021,Chamberland.2022,ofek2016extending,michael2016new,Leghtas.2015,Rosenblum.2018,Campagne.2020,Berdou.2023}. In particular, a certain class of CV codes known as CAT qubit codes has focused the attention of the community for their property of dynamically preserving quantum information through the action of a class of suitably engineered QMS which, loosely referred to as CAT dissipation in the present introduction. However, a mathematically rigorous analysis of these codes has only gotten little attention to the best of our knowledge, with the notable exception of \cite{Azouit.2016}.

Much theoretical work has focused on the more tractable generators of Gaussian dynamics semigroups, where the generator $\cL$ is expressed as a quadratic form in the creation and annihilation operators \cite{Hudson.1984, Fagnola.1994, Cipriani.2000, Agredo.2021,gaussgene}. For those generators, the Feller property as well as properties of the spectrum and convergence results are known \cite{Cipriani.2000,Fagnola.2003,Carbone.2007,Carlen.2017,DePalma.2018}. Since generators of CAT dissipation typically involve higher order monomials in $a$ and $a^\dagger$, the establishment of a more general theory of CV quantum Markov semigroups including them is timely.

% Hide subsection in the toc
\addtocontents{toc}{\protect\setcounter{tocdepth}{1}}
\subsection{Framework}
\addtocontents{toc}{\protect\setcounter{tocdepth}{2}}

In this paper, we consider an operator $\cL$ on the space $\cT_f$ of finite linear combinations of rank-one operators of the form $\ketbra{k}{l}$, where given $k\in\mathbb{N}$, $\ket{k}\in L^2(\mathbb{R})$ denotes the $k$-photon Fock state. We further assume that $\cL$ satisfies the following two conditions: (i) $\cL$ has a GKSL structure \eqref{eqGKSL}, where the Hamiltonian $H$ as well as the jump operators $L_j$ are polynomials of the annihilation and creation operators; (ii) the following condition is satisfied: for a divergent sequence $\{k_r\}_{r \in \N}$ in $\R_+$, there exist real coefficients $\{w_{k_r}\}_{r \in \N}$ such that for all states $\rho\in \cT_f$:
\begin{equation*}
    \tr[\cL(\rho)(N+\1)^{k_r/2}]\leq w_{k_r}\tr[\rho(N+\1)^{k_r/2}]\,.
\end{equation*}
Above, $N\coloneqq a^\dagger a=\sum_{n\in \mathbb{N}}n\ketbra{n}{n}$ denotes the photon number operator.
This assumption implies not only that $\cL$ defines a quantum dynamical semigroup, but also a quasi-contractive semigroup on the weighted Banach spaces $(\cD(\cW^k),\|\cW^k(\cdot)\|_{1})$ defined through the operator
\begin{equation*}
    \cW(\cdot)\coloneqq (N+\1)^{1/4}\,(\cdot)\,(N+\1)^{1/4}\,.
\end{equation*}
In the latter, we refer to these spaces as Sobolev spaces and denote them by $W^{k,1}$ in analogy with their classical analogues (see also \cite{Becker.2021}). Next, we call operators $\cL$ that satisfy both conditions (i) and (ii) generators of Sobolev preserving quantum dynamical semigroups. Indeed, in our first main result, we show that such operators generate QMSs with the extra property that the latter preserve Sobolev spaces. More precisely:

\begin{thm*}[Generation of bosonic semigroups, see \Cref{thm:generation-theorem}]
    Let $(\cL, \cD(\cL))$ be an operator defined on the Banach space $\cT_{1,\operatorname{sa}}$ of self-adjoint, trace-class operators. If $(\cL, \cD(\cL))$ satisfies conditions (i) and (ii) above, then its closure $\overline{\cL}$ generates a strongly continuous, positivity preserving semigroup $(\cP_t)_{t\ge 0}$ on $W^{k, 1}$ for all $k \in \R_+$ with 
    \begin{equation*}
        \norm{\cP_t}_{W^{k, 1} \to W^{k, 1}} \le e^{\omega_k t} \,\quad \forall t\ge 0\, . 
    \end{equation*}
    where $\omega_k = \frac{k_{r_1} - k}{k_{r_1} - k_{r_0}}\omega_{k_{r_0}} + \frac{k - k_{r_0}}{k_{r_1} - k_{r_0}}\omega_{k_{r_1}}$ for an $r$ such that $k_{r_0}\leq k <k_{r_1}$.
    Finally, for $k = 0$, the semigroup is contractive and trace-preserving.
\end{thm*}

Additionally, our setup is directly suited to the establishment of a perturbation analysis akin to 
the result reported in \cite{Szehr_2013} in the finite dimensional setting. Moreover, in some cases, our analysis allows us to conclude the existence of adherence points for the dynamics in the large time limit. We manage to prove the requirements (i)-(ii) of the generation theorems as well as rigorous perturbation analysis for several examples including CAT dissipations as well as Gaussian and quantum Ornstein Uhlenbeck generators. For the latter, we show for instance the following perturbation bound for all $t\ge 0$ (see \Cref{propqOUperturb} and \Cref{corECDN}):

\begin{prop*}
    Let $(\cL_{\operatorname{qOU}},\cT_f)$ be the generator of the quantum Ornstein Uhlenbeck semigroup with jump operators ${\lambda}a$ and $\mu a^\dagger$, $\lambda>\mu\geq0$ and $(\cL_G,\cT_f)$ a Gaussian perturbation with unique jump $\gamma a+\eta a^\dagger$ with $\gamma,\eta\in\mathbb{R}$, and $\varepsilon>0$. Then, assuming $\lambda^2-\mu^2+|\gamma|^2-|\eta|^2> 0$, $\cL_{\operatorname{qOU}}+\varepsilon\cL_G$ generates a positivity and Sobolev preserving semigroup on $W^{k,1}$ for $k\geq1$, and there exist uniformly bounded functions $C(\varepsilon),D(\varepsilon)$ depending on $\lambda,\mu,|\eta|,|\gamma|$ such that, for all $t\ge 0$ and all state $\rho\in W^{k,1}$,
    \begin{equation}
        \Big\|\left(e^{t \cL_{\operatorname{qOU}}}-e^{t(\cL_{\operatorname{qOU}}+\varepsilon\cL_G)}\right)(\rho)\Big\|_{1}\leq \varepsilon\, C(\varepsilon)\, \max\Big\{\|\rho\|_{W^{2,1}},D(\varepsilon)\Big\}\,.
    \end{equation}
    In particular, for all $t\ge 0$
    \begin{equation}
        \Big\|e^{t \cL_{\operatorname{qOU}}}-e^{t(\cL_{\operatorname{qOU}}+\varepsilon\cL_G)}\Big\|_{\diamond}^E\leq\,(1+E) \varepsilon\, C(\varepsilon)\, \max\Big\{1,D(\varepsilon)\Big\}\,,
    \end{equation}
    where $\|.\|_\diamond^E$ denotes the energy-constrained diamond norm defined in \Cref{ECnorm}.
\end{prop*}

We also note that our theory extends to the case of a time-dependent generator as well as to the multimode setting $\cH=L^2(\mathbb{R}^m)$, $m\ge 1$, see \Cref{sec:timedependentgeneration,sec:multi-mode-extension}. To prove our generation theorems, the compactly embedded Sobolev spaces play a crucial role and provide an interesting proof strategy, which follows the original method of Davies \cite{Davies.1977} by an explicit reduction to the seminal theorems by Hille, Yosida \cite{Hille.1957} and Feller, Myadera, Lumer and Phillips \cite[Thm.~II.3.8]{Engel.2000}.

\addtocontents{toc}{\protect\setcounter{tocdepth}{1}}
\subsection{CAT dissipations}
\addtocontents{toc}{\protect\setcounter{tocdepth}{2}}

As mentioned before, the interest in continuous variable QMS has been reignited by the modelling capabilities of a certain class of error-corrected universal quantum computing architectures. In \cite{Azouit.2015} and later in \cite{Azouit.2016}, Azouit, Sarlette, and Rouchon prove the well-posedness of the dynamics that stabilises an $l$ dimensional code-space, with a generator given for a fixed $\alpha\in\mathbb{R}$ by 

\begin{align}\label{deflphotondissip}
    \cL_l(\rho) = L_l\rho L_l^\dagger-\frac{1}{2}\big\{L_l^\dagger L_l,\rho\big\}\qquad\text{with}\qquad L_l\coloneqq a^l-\alpha^l\1\,.
\end{align}

In addition, they identified invariant operators of the dynamic and further showed that the semigroup exponentially drives states towards the code-space spanned by these invariants. By constructing a Banach space from composites of the generator, i.e. $L_l=a^l-\alpha^l\1$, compactly embedded in the self-adjoint trace class operators, they judiciously circumvented the problems previously encountered when trying to take limits of the minimal semigroups. This procedure was very much tailored towards the simple structure of the generator and also relied on a favourable commutation relation of $a^l-\alpha^l\1$ and $(a^l-\alpha^l\1)^\dagger$, which one cannot hope for in general. In contrast, here we do not use parts of the generator to create our compactly embedded spaces, but instead use the most natural candidate at hand, namely the number operator $N$. Generalising the idea of Azouit et al.~we take limits of sequences of semigroups for which our CV Sobolev spaces are admissible subspaces in order to prove our generation theorems. Combining their exponential dynamical convergence, stated as
\begin{equation*}
    \tr[L_l\left(e^{t\cL_l}(\rho)-\overline{\rho}\right) L_l^\dagger]\leq e^{-l!t}\tr[L_l|\rho-\overline{\rho}| L_l^\dagger]\,,
\end{equation*}
where $\overline{\rho}$ is a $\rho$-dependent state in the code-space, with our generation and perturbation theory, we can for example show that any $l$-photon dissipation perturbed by a Hamiltonian admits the following large-time perturbation bounds (see \Cref{thm:l-diss-hamiltonian-perturbation}):
\begin{thm*}
    Let $\cL_l$ be the $l$-photon dissipation defined in \Cref{deflphotondissip}    
    and $p_H\in\C[X,Y]$ with $\deg(p_H)=d_H\leq2(l-1)$ such that $H=p_H(a,\ad)$ is a symmetric operator. Then, there exist constants $c,\gamma>0$ depending on $\alpha$ and $l$  such that for $\varepsilon\geq0$ and all states $\rho\in W^{6l-4,1}$
    \begin{equation*}
        \begin{aligned}
            \Big|\tr[L\left(e^{t\cL_l}(\rho)-e^{t(\cL_l+\varepsilon\cH[H])}(\rho)\right)L^\dagger]\Big|\leq\varepsilon c\left(1-e^{-l!t}\right)\max\{\gamma,\|\rho\|_{W^{6l-4,1}}\}
        \end{aligned}
    \end{equation*}
    where $\cH[H](\rho)\coloneqq -i[H,\rho]$.
\end{thm*}

The same idea can be extended to more general setups and thereby extends the result by \citeauthor{Szehr_2013} from finite dimensions to the case of strongly continuous semigroups over infinite dimensional systems. 

% Hide the subsection in the toc
\addtocontents{toc}{\protect\setcounter{tocdepth}{1}}
\subsection{Outline of the paper}
\addtocontents{toc}{\protect\setcounter{tocdepth}{2}}

In \Cref{sec:preliminaries}, we begin with an introduction to basic Banach space and operator theory, followed by a short overview of this theory in the context of Hilbert spaces and their associated bounded, compact and trace-class operator spaces. Building upon that we then introduce in \Cref{subsec:weighted-norms-compact-embedding} the notion of compact embeddings and weighted Banach spaces, followed by basic semigroup theory in \Cref{subsec:c0-semigroups}. More specific to our application, we then briefly recapitulate Bosonic Hilbert spaces, and relevant operators thereon, and introduce our Bosonic Sobolev spaces. We prove that they are compactly embedded into one another and provide an interpolation theorem in the spirit of the Stein-Weiss theorem for weighted $L_p$ spaces. In \Cref{sec:polynomial-generators}, we begin by showing the generation theorem in the time-independent case and then employ this theorem in \Cref{sec:timedependentgeneration} to prove a generation theorem for generators composed of polynomials in $a$ and $a^\dagger$ with coefficients that are differentiable functions of time. We extend our analysis to the multimode setting in \Cref{sec:multi-mode-extension} where we lift the generation theorems from the chapter before.\par
\Cref{sec:examples-sobolev-preserving-semigroup} begins with a short proposition making better use of tighter input-output moments of the generator and showing the existence of adherence points in the asymptotic time regime for semigroups that admit such bounds. We then proceed to prove the generation theorem for the quantum Ornstein Uhlenbeck generator as well as for a family of CAT dissipations in \Cref{sec:cat-qubits}. This section is then followed by large time perturbation bounds for both the quantum Ornstein Uhlenbeck semigroup as well as the CAT dissipations in \Cref{sec:example-perturbation-bounds}.