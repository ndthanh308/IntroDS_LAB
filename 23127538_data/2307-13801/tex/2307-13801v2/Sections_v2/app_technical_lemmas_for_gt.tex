\begin{lem}\label{lem:(n+1)-(n+1)-properties}
    For $d \ge 0$ and $\varepsilon > 0$, define the operator 
    \begin{equation*}
        \cI_{d, \varepsilon}: \cT_f \to \cT_f, \qquad x \mapsto \cI_{d, \varepsilon}(x) \coloneqq - \varepsilon\{(N + \1)^{4d}, x\} \, . 
    \end{equation*}
    For $\lambda \ge 0$, we have that $\lambda - \cI_{d, \varepsilon}:\cT_f \to \cT_f$ is bijective. While for all $k \in \R_+$ and $x \in \cT_f$ one further has
    \begin{align}
        & \norm{(\lambda - \cI_{d, \varepsilon})^{-1}(x)}_{W^{k, 1} } \le \frac{1}{\lambda + 2 \varepsilon} \norm{x}_{W^{k, 1}}\,;\tag{1}\label{item:(1)}\\
       &  \norm{\cI_{d, \varepsilon} \circ (\lambda - \cI_{d, \varepsilon})^{-1}(x)}_{W^{k, 1}} \le {2}\norm{x}_{W^{k, 1}} \tag{2}\label{eq:inequality-resolvent-(n+1)4d}\, . 
    \end{align}
\end{lem}
\begin{proof}
    For $\lambda \ge 0$ define the following linear operator
    \begin{align*}
        (\lambda - \cI_{d, \varepsilon})^{-1}:\cT_f &\to \cT_f,\\
        x = \sum\limits_{\text{finite}} x_{nm} \ketbra{n}{m} &\mapsto (\lambda - \cI_{d, \varepsilon})^{-1}(x) \coloneqq \sum\limits_{\text{finite}} x_{nm} \frac{1}{\varepsilon(n + 1)^{4d} + \varepsilon(m + 1)^{4d} + \lambda} \ketbra{n}{m} \, .
    \end{align*}
    or alternatively 
    \begin{align*}
        (\lambda - \cI_{d, \varepsilon})^{-1}(x) &= \int\limits_{0}^\infty e^{-(\varepsilon(N + \1)^{4d} + \lambda/2) s} x e^{-(\varepsilon(N + \1)^{4d} + \lambda/2) s} ds\\
        &=  \int\limits_{0}^\infty  \sum\limits_{\text{finite}} e^{-(\varepsilon(n + \1)^{4d} + \lambda/2)s} x_{nm} e^{-(\varepsilon(m + \1)^{4d} + \lambda/2)s} \ketbra{n}{m} ds \, .
    \end{align*}
    The integral representation allows us to deduce that $(\lambda - \cI_{d, \varepsilon})^{-1}$ preserves positivity. Using the first expression it is straightforward to show that this map is indeed the inverse to $\lambda - \cI_{d, \varepsilon}:\cT_f \to \cT_f$. The bound $\norm{(\lambda - \cI_{d, \varepsilon})^{-1} x}_{W^{k, 1} } \le \frac{1}{\lambda + 2 \varepsilon} \norm{x}_{W^{k, 1}}$ can be shown, using the integral representation and Hölder inequality:
    \begin{align*}
        \norm{(\lambda - \cI_{d, \varepsilon})^{-1}(x)}_{W^{k, 1}} &\le \int\limits_{0}^\infty \norm{e^{-\varepsilon(N + \1)^{4d} + \lambda/2)s} (N + \1)^{k/4} x (N + \1)^{k/4} e^{-\varepsilon(N + \1)^{4d} + \lambda/2)s}}_1\\
        &\le \int\limits_{0}^\infty \norm{e^{-\varepsilon(N + \1)^{4d} + \lambda/2)s}}_\infty^2 ds \norm{x}_{W^{k, 1}}\\
        &= \int\limits_{0}^\infty e^{-(2\varepsilon + \lambda)s} \norm{x}_{W^{k, 1}} = \frac{1}{\lambda + 2\varepsilon} \norm{x}_{W^{k, 1}} \, .
    \end{align*}
    Issues arising from the unbounded nature of $N$ can be ignored in the above estimations, as we can take a finite cut-off in the Fock basis due to $x \in \cT_f$. For \eqref{eq:inequality-resolvent-(n+1)4d}, we have that on $\cT_f$, $-\cI_{d, \varepsilon}\circ (\lambda - \cI_{d, \varepsilon})^{-1} = \1 - \lambda (\lambda - \cI_{d, \varepsilon})^{-1}$, i.e. for $x \in \cT_f$
    \begin{equation}
        \norm{(\1 - \lambda (\lambda - \cI_{d, \varepsilon})^{-1})x}_{W^{k, 1}} = \norm{-\cI_{d, \varepsilon}\circ (\lambda - \cI_{d, \varepsilon})^{-1} (x)}_{W^{k, 1}}
    \end{equation}
    where the LHS can be upper bounded by $(1 + \frac{\lambda}{\lambda + 2\varepsilon})\norm{x}_{W^{k, 1}} \le 2 \norm{x}_{W^{k, 1}}$ using \eqref{item:(1)}. This proves the last claim.
\end{proof}

\begin{lem}\label{lem:boundedness-polynomials}
    For $p \in \C[X,Y]$ a polynomial of degree $d$ and 
    \begin{equation*}
        A: \cH_f \to \cH, \qquad \ket\psi = \sum\limits_{\text{finite}}\psi_n \ket{n} \mapsto p(a, a^\dagger)\ket{\psi} = \sum\limits_{\text{finite}}\psi_n p(a, a^\dagger)\ket{n} \, ,
    \end{equation*}
    we get that for all $k \in \R_+$ and $d' \ge d$
    \begin{align*}
        B_1: \cH_f \to \cH, \qquad \ket\psi \mapsto (N + \1)^{k} A (N + \1)^{-k - d'} \ket\psi\\
        B_2: \cH_f \to \cH, \qquad \ket\psi \mapsto (N + \1)^{-k - d'} A (N + \1)^{k} \ket\psi
    \end{align*}
    are bounded and therefore can be uniquely extended to a bounded map on $\cH$, with the same bound.
\end{lem}
\begin{proof}
    Since the proof for $B_1$ and $B_2$ are almost completely analogous, we will only show it here for $B_1$. The canonical commutation relation allows us to rewrite $A$ as a finite linear combination of monomials of the form $(a^\dagger)^i N^j$ and $a^i N^j$ with $i + j/2 \le d$. Now by triangle inequality for the norm on $\cH$ and since the sum of these monomials comprising $A$ are finite, for the claim to be true it suffices to show that 
    \begin{equation*}
        (N + \1)^k(a^\dagger)^i N^j (N + \1)^{-k - d}:\cH_f \to \cH, \qquad (N + \1)^ka^i N^j (N + \1)^{-k - d}:\cH_f \to \cH
    \end{equation*}
    are bounded, and hence can be uniquely extended to a bounded map on $\cH$. We only give the argument for the first map, since it is almost completely analogous to the second one. Let $\ket\psi = \sum\limits_{n = 0}^M\psi_n \ket{n} \in \cH_f$, then
    \begin{align*}
         \ket{\varphi} \coloneqq (N + \1)^k(a^\dagger)^i N^j (N + \1)^{-k - d'} \ket{\psi} &= \sum\limits_{n = 0}^M\psi_n (N + \1)^k(a^\dagger)^i N^j (N + \1)^{-k - d'}\ket{n}\\
         &= \sum\limits_{n = 0}^M\psi_n \frac{(n + 1 + j)^k}{(n + 1)^k} \frac{n^j \prod_{l = 1}^i \sqrt{n + l}}{(n + 1)^{d'}} \ket{n + i} \, . 
    \end{align*}
    Hence
    \begin{align*}
        \norm{\varphi}^2 &= \sum\limits_{n = 0}^M \frac{(n + 1 + j)^{2k}}{(n + 1)^{2k}} \frac{n^{2j} \prod_{l = 1}^i (n + l)}{(n + 1)^{2d'}} |\psi_n|^2\\
        &\le \sum\limits_{n = 0}^M \frac{(n + 1 + d)^{2k}}{(n + 1)^{2k}} \frac{(n + d)^{2j}(n + d)^i}{(n + 1)^{2d}} |\psi_n|^2\\
        &\le \sum\limits_{n = 0}^M d^{2k} d^{2d} |\psi_n|^2\\
        &= d^{2(k + d)} \norm{\psi}^2 \, ,
    \end{align*}
    where we used that $i + j/2 \le d$ and $d \le d'$. Hence $(N + \1)^k (a^\dagger)^i N^j(N + \1)^{-k - d'}:\cH_f \to \cH$ is bounded by $d^{k + d}$ and can be uniquely extended to a bounded linear map on $\cH$. This concludes the claim.
\end{proof}

\begin{lem}\label{lem:infinitesimal-boundedness-W-k-1}
    Let $K \in \N$. For $i = 1, \hdots, K$ let $p_{i, 1}, p_{i, 2} \in \C[X,Y]$ polynomials of degree $d_{i, 1}$, $d_{i, 2}$ such that 
    \begin{equation*}
        \cA: \cT_f \to \cT_f, \qquad x \mapsto \cA(x) = \sum\limits_{i = 1}^KA_{i, 1}x A_{i, 2} = \sum\limits_{i = 1}^K p_{i, 1}(a, a^\dagger) x \, p_{i, 2}(a, a^\dagger)
    \end{equation*}
    where the action of $p_{1/2}(a, a^\dagger)$ on $x$ is defined via the action of $a$ and $a^\dagger$ on $\ketbra{n}{m}$. We then have that for all $k \ge 0$, $d \ge \max\limits_{i = 1, \hdots, K}\max\{d_{i, 1}, d_{i, 2}\}$ there exists $C_k \ge 0$, s.t. for all $\varepsilon \ge 0$ and $\forall x \in \cT_f$
    \begin{equation*}
        \norm{\cA(x)}_{W^{k, 1}} \le \varepsilon\norm{\{(N + \1)^{4d}, x\}}_{W^{k, 1}} + \frac{C_k}{\varepsilon} \norm{x}_{W^{k, 1}} \, . 
    \end{equation*}
\end{lem}
\begin{proof}
    Let $k \in \R_+$. The first step is to show that there exists $c_k \ge 0$, s.t. for all $x \in \cT_f$
    \begin{equation}\label{eq:proof-first-stage-inequality}
        \norm{\cA(x)}_{W^{k, 1}} \le c_k \norm{(N + \1)^d x (N + \1)^d}_{W^{k, 1}} \, . 
    \end{equation}
    The argument reduces to showing that for $i = 1, \hdots, K$ and $x \in \cT_f$
    \begin{equation*}\label{eq:proof-first-stage-inequality-substage}
        \norm{(N + \1)^{k/4} A_{i, 1} x A_{i, 2} (N + \1)^{k/4}}_1 \le c_{i, k} \norm{(N + \1)^d x (N + \1)^d}_{W^{k, 1}}
    \end{equation*}
    since the sum comprising $\cA$ is finite. Note that the trace norm on the LHS is the one on the trace-class operators since the argument might not necessarily be self-adjoint. For $x \in \cT_f$, we have
    \begin{align*}
        &\norm{(N + \1)^{k/4} A_{i, 1} x A_{i, 2} (N + \1)^{k/4}}_1\\
        &\hspace{0.5cm} = \norm{(N + \1)^{k/4} A_{i, 1} (N + \1)^{-k/4 - d} (N + \1)^{k/4 + d}x  (N + \1)^{k/4 + d}  (N + \1)^{-k/4 - d} A_{i, 2} (N + \1)^{k/4}}_1\\
        &\hspace{0.5cm}\le \norm{(N + \1)^{k/4} A_{i, 1} (N + \1)^{-k/4 - d}}_\infty \norm{x}_{W^{k+4d, 1}} \norm{(N + \1)^{-k/4 - d} A_{i, 2} (N + \1)^{k/4}}_\infty\\
        &\hspace{0.5cm}\le c_{i, k} \norm{(N + \1)^d x (N + \1)^d}_{W^{k, 1}} \, 
    \end{align*}
    where we used \Cref{lem:boundedness-polynomials} to argue that the operators involved are bounded and we can employ Hölder's inequality to split them off. Subsequently, we replaced the operator norms with the constant $c_{i, k}$. Now in the second step we show that for all $\varepsilon > 0$ and $x \in \cT_f$
    \begin{equation*}
        \norm{(N + \1)^d x (N + \1)^d}_{W^{k, 1}} \le \varepsilon \norm{\{(N + \1)^{4d}, x\}}_{W^{k, 1}} + \frac{1}{4\varepsilon} \norm{x}_{W^{k, 1}} \, . 
    \end{equation*}
    Combining this with \Cref{eq:proof-first-stage-inequality} then immediately provides the claim.
    Therefore, let $\lambda > 0$ and $x \in \cT_f$ with $x \ge 0$. We find
    \begin{align*}
        &\norm{(N + 1)^d(\lambda - \cI_{d, 1})^{-1}(x) (N + \1)^d}_{W^{k, 1}}\\
        &\hspace{2cm} = \tr[(\lambda - \cI_{d, 1})^{-1}\{(N + \1)^{k/4 + d} x (N + \1)^{k/4 + d}\}]\\
        &\hspace{2cm} = \int\limits_{0}^\infty \tr[e^{-(2(N + \1)^{4d} + \lambda) s}(N + \1)^{2d} (N + \1)^{k/4} x (N + \1)^{k/4}] ds\\
        &\hspace{2cm} = \tr\Big[\int\limits_{0}^\infty e^{-(2(N + \1)^{4d} + \lambda) s}(N + \1)^{2d} ds\,  (N + \1)^{k/4} x (N + \1)^{k/4}\Big]\\
        &\hspace{2cm} = \tr[\frac{(N + \1)^{2d}}{(N + \1)^{4d} + \lambda}(N + \1)^{k/4} x (N + \1)^{k/4}] \, ,
    \end{align*}
    where we used the map $\cI_{d, 1}$ from \Cref{lem:(n+1)-(n+1)-properties}, the integral representation of its resolvent $(\lambda - \cI_{d, 1})^{-1}$ and that this resolvent preserves positivity. Further, we applied the cyclicity of the trace and conveniently suppressed issues that might arise from the unbounded nature of $N$ by taking a cut-off in the Fock basis. This is justified by $x \in \cT_f$. Now we can use that $(N + \1)^{k/4} x (N + \1)^{k/4} \ge 0$ to bound the RHS of the above chain of inequalities to get
    \begin{align*}
        \norm{(N + 1)^d(\lambda - \cI_{d, 1})^{-1}(x) (N + \1)^d}_{W^{k, 1}} &\le \sup\limits_{s \ge 1} \frac{s}{s^2 + \lambda} \tr[(N + \1)^{k/4} x (N + \1)^{k/4}]\\
        &= \sup\limits_{s \ge 1} \frac{s}{s^2 + \lambda} \norm{x}_{W^{k, 1}} \, .
    \end{align*}
    For a general $x \in \cT_f$, we can set $y = (N + \1)^{k/4} x (N + \1)^{k/4}$ decompose into $y = y_+ - y_-$ the positive and negative part of $y$ respectively and then set $x_\pm = (N + \1)^{-k/4} y_\pm (N + \1)^{-k/4}$. We clearly have that $x_\pm \in \cT_f$, $x = x_+ - x_-$ and $x_\pm \ge 0$ as $(N + \1)^{-k/4} \cdot (N + \1)^{-k/4}$ preserves positivity. This allows us to apply what we have shown above to obtain
    \begin{align*}
        \norm{(N + 1)^d(\lambda - \cI_{d, 1})^{-1}(x) (N + \1)^d}_{W^{k, 1}} &\le \norm{(N + 1)^d(\lambda - \cI_{d, 1})^{-1}(x_+) (N + \1)^d}_{W^{k, 1}}\\
        &\hspace{2cm} + \norm{(N + 1)^d(\lambda - \cI_{d, 1})^{-1}(x_-) (N + \1)^d}_{W^{k, 1}}\\
        &\le \sup\limits_{s \ge 1} \frac{s}{s^2 +\lambda}(\norm{x_+}_{W^{k, 1}} + \norm{x_-}_{W^{k, 1}})\\
        &= \sup\limits_{s \ge 1} \frac{s}{s^2 +\lambda} (\norm{y_+}_1 + \norm{y_-}_1)\\
        &= \sup\limits_{s \ge 1} \frac{s}{s^2 +\lambda} \norm{y}_1\\
        &= \sup\limits_{s \ge 1} \frac{s}{s^2 +\lambda} \norm{x}_{W^{k, 1}}
    \end{align*}
    Lastly we can use the bijectivity of $(\lambda - \cI_{d, 1})$ on $\cT_f$ (q.v.~\Cref{lem:(n+1)-(n+1)-properties}) and triangle inequality to conclude 
    \begin{equation}
        \norm{(N + \1)^d x (N + \1)^d}_{W^{k, 1}} \le \sup\limits_{s \ge 0} \frac{s}{s^2 + \lambda} \norm{\cI_{d, 1}(x)}_{W^{k, 1}} + \lambda \sup\limits_{s \ge 0} \frac{s}{s^2 + \lambda} \norm{x}_{W^{k, 1}}\,.
    \end{equation}
    Choosing $\lambda = \frac{1}{4\varepsilon^2}$, we find that $\sup\limits_{s \ge 1} \frac{s}{s^2 + \lambda} < \varepsilon$, hence
    \begin{equation}
        \norm{(N + \1)^d x (N + \1)^d}_{W^{k, 1}} \le \varepsilon \norm{\{(N +\1)^{4d}, x\}}_{W^{k, 1}} + \frac{1}{4\varepsilon} \norm{x}_{W^{k, 1}}
    \end{equation}
\end{proof}

\begin{lem}[Interpolation Lemma]\label{lem:interpolation-lemma}
    Let $k_0 < k_1 \in \R_+$ and $(\cL, \cD(\cL))$ an operator on $W^{k_j, 1}$, $j = 0, 1$ respectively. Further, assume that the closure of $(\cL, \cD(\cL))$ defines a strongly continuous semigroup $(\cP_t^j)_{t \ge 0}$ with 
    \begin{equation}
        \norm{\cP_t^{j}}_{W^{k_j, 1} \to W^{k_j, 1}} \le M_j e^{\omega_j t} \quad \forall t \ge 0 \, 
    \end{equation}
    in both spaces, respectively. Then for $\theta \in [0, 1]$ and $k_\theta = \theta k_1 + (1 - \theta) k_0$ the following are true
    \begin{enumerate}
        \item\label{item:int-pol-1} The closure of $(\cL, \cD(\cL))$ defines a strongly continuous semigroup on $W^{k_\theta, 1}$ with 
        \begin{equation}\label{eq:interpolation-semigroup-bound}
            \norm{\cP^\theta_t}_{W^{k_\theta, 1} \to W^{k_\theta, 1}} \le M_0^{1 - \theta} M_1^\theta e^{(\omega_{k_0}(1 - \theta) + \omega_{k_1} \theta) t} \quad \forall t \ge 0\, . 
        \end{equation}
        \item\label{item:int-pol-2} $(\cP^\theta_t)_{t \ge 0}$ agrees with $(\cP_t^j)_{t \ge 0}$ on $W^{k_j, 1} \cap W^{k_\theta, 1}$ for $j = 1, 2$.
    \end{enumerate}
\end{lem}
\begin{proof}
    We begin with the second claim and only cover $k_j = k_0$ as the other case only requires minor changes that are left to the reader. Let $\theta \in (0, 1)$ and the closure of $(\cL, \cD(\cL))$ the generator of $(\cP_t^0)_{t \ge 0}$ and $(\cP_t^\theta)_{t \ge 0}$ on the respective spaces. Since $k_0 < k_\theta$ and hence $W^{k_\theta, 1} \Subset W^{k_0, 1}$, we have that the closure of $(\cL, \cD(\cL))$ in $W^{k_\theta, 1}$ agrees with the restriction of the closure in $W^{k_0, 1}$. As the semigroup is completely determined by its generator, we find that the semigroups agree on $W^{k_\theta, 1}$.\newline
    For the first claim, note that the semigroups $(\cP_t^j)_{t \ge 0}$, $j = 1, 2$ agree on $W^{k_1, 1}$ by \Cref{item:int-pol-2}, which allows us to employ the Stein-Weiss theorem for Bosonic Sobolev spaces (\Cref{thm:stein-weiss}) to conclude \Cref{eq:interpolation-semigroup-bound}. It remains to check that the families of bounded maps $(\cP_t^\theta)_{t \ge 0}$ are strongly continuous semigroups generated by the closure of $(\cL, \cD(\cL))$ on $W^{k_\theta, 1}$. We have that $\cP_0^\theta = \1$  and $\cP_t^\theta \cP_s^\theta = \cP_{t + 1}^\theta$ $\forall t, s \ge 0$ as a consequence of $\cP_t^0 |_{W^{k_\theta, 1}} = \cP_t^\theta$ $\forall t$ (this equality holds by \Cref{thm:stein-weiss}). The strong continuity follows, due to $\cP^1_t = \cP_t^\theta|_{W^{k_1, 1}}$, as for $x \in W^{k_1, 1}$
    \begin{equation*}
        \lim\limits_{t \to 0} \norm{\cP_t^\theta(x)  - x}_{W^{k_\theta, 1}} = \lim\limits_{t \to 0} \norm{\cP^1_t(x) - x}_{W^{k_\theta, 1}} \le \lim\limits_{t \to 0} \norm{\cP^1_t(x) - x}_{W^{k_1, 1}} = 0
    \end{equation*}
    where we used $W^{k_1, 1} \Subset W^{k_\theta, 1}$ and that $\cP^1_t$ is a strongly continous semigroup on $W^{k_1, 1}$. For general $x \in W^{k_\theta, 1}$, we find $(x_n)_{n \in \N} \subset W^{k_\theta, 1}$ converging to $x$ in $W^{k_\theta, 1}$ and for all $n \in \N$
    \begin{align*}
         \lim\limits_{t \to 0} \norm{\cP_t^\theta(x)  - x}_{W^{k_\theta, 1}} &\le \lim\limits_{t \to 0}\left[ (1 + M_0^{1 - \theta} M_1^\theta e^{(\omega_{k_0}(1 - \theta) + \omega_{k_1} \theta) t}) \norm{x - x_n}_{W^{k_\theta, 1}} + \norm{\cP_t^\theta(x_n)  - x_n}\right]\\
         &\le (1 + M_0^{1 - \theta} M_1^\theta)\norm{x - x_n}_{W^{k_\theta, 1}}
    \end{align*}
    which concludes the strong continuity. It remains to argue that the closure of $(\cL, \cD(\cL))$ on $W^{k_\theta, 1}$ is indeed the generator of $(\cP_t^\theta)_{t \ge 0}$. By \cite[Sec. II.2.3]{Engel.2000}, we find that the restriction of the generator $(\hat{\cL}, \cD(\hat{\cL}))$ of $(\cP_t^\theta)_{t \ge 0}$ to $W^{k_1, 1}$ is the generator of $(\cP_t^1)_{t \ge 0}$ with $(\cL, \cD(\cL))$ being a core for this restricted generator on $W^{k_1, 1}$ by assumption. This in particular means that for $\lambda > \omega_k$, $\lambda - \cL:\cD(\cL) \to W^{k_1, 1}$ has a dense range in $W^{k_1, 1}$, which further allows us to conclude that $\lambda - \cL:\cD(\cL) \to W^{k_\theta, 1}$ has a dense range in $W^{k_\theta, 1}$. Now as it is $\omega_{k_\theta}$-quasi dissipative (being the restriction of the generator of the semigroup $(\cP_t^\theta)_{t \ge 0}$) we can conclude that indeed $(\cL, \cD(\cL))$ is closable in $W^{k_\theta, 1}$ with the closure $(\hat{\cL}, \cD(\hat{\cL}))$ (c.f. \cite[Proposition 3.14]{Engel.2000}). 
\end{proof}

\begin{lem}[Approximation Lemma]\label{lem:approximation-lemma}
    Let $K \in \N$. For $i = 1, \hdots, K$ let $p_{i, 1}, p_{i, 2} \in \C[X,Y]$ polynomials of degree $d_{i, 1}$, $d_{i, 2}$ and $\{a_{i, n}\}_{n \in \N} \subset \C$ convergent sequences with limits $a_i \in \C$ such that $\{(\cA_n, \cD(\cA_n) = \cT_f)\}_{n \in \N}$ is an operator sequence with
    \begin{equation}
       \cA_n :\cT_f \to \cT_f, \quad x \mapsto \cA_n(x) \coloneqq \sum\limits_{i = 1}^K a_{i, n} A_{i, 1} \,x\, A_{i, 2} \coloneqq \sum\limits_{i = 1}^K a_{i, n} p_{i, 1}(a, a^\dagger) \,x\, p_{i, 2}(a, a^\dagger) \, . 
    \end{equation}
    If for all $k \in \R_+$ there exists $M_k, \omega_k$ such that for all $n \in \N$ the closure of $(\cA_n, \cD(\cA_n))$ generates a strongly continuous semigroup $(\cP_t^n)_{t \ge 0}$ on $W^{k, 1}$ with
    \begin{equation}\label{eq:uniform-bound-semigroups}
        \norm{\cP_t^n}_{W^{k, 1} \to W^{k, 1}} \le M_k e^{\omega_k t} \quad \forall t \in \R \, ,
    \end{equation}
    then the closure of $(\cA, \cD(\cA) = \cT_f)$, the pointwise limit of $(\cA_n, \cD(\cA_n))$, defines a strongly continuous semigroup on $W^{k, 1}$ for $k \ge 0$ as well. We further get that the semigroups generated by the closure of $(\cA_n, \cD(\cA_n))$ converge uniformly (in time) on compact intervals to the semigroup generated by the closure of $(\cA, \cD(\cA))$ and that \Cref{eq:uniform-bound-semigroups} also holds for the limiting semigroup.
\end{lem}
\begin{proof}
    Let $k \in \R_+$. To prove the lemma, we first note that $(\cA, \cD(\cA))$ is densely defined and the pointwise limit of $\{(\cA_n, \cD(\cA_n))\}_{n \in \N}$. To employ the second Trotter-Kato approximation theorem, which implies the claim (see the version in \cite[Thm.~III.4.9]{Engel.2000}), we need to show that there exists $\lambda > 0$ such that $(\lambda - \cA, \cD(\cA))$ has dense range in $W^{k, 1}$. We will do so by showing that the closure of the range contains $\cT_f$ which is a dense subset of $W^{k, 1}$. Therefore let $\lambda > \max\{\omega_k, \omega_{k + 4d}\}$ (with $\omega_\cdot$ from \Cref{eq:uniform-bound-semigroups} and $d$ the maximal degree of the polynomials but at least one, i.e.~$d = \max\limits_{i = 1, \hdots, K} \max\{d_{i, 1}, d_{i, 2}, 1\}$). Let $\xi \in \cT_f$ arbitrary, we then have that for all $n \in \N$ the operator $(\lambda - \cA_n, \cD(\cA_n))$ has dense range in $W^{k + 4d, 1}$, meaning in particular we find a sequence $\{x_{n, m}\}_{m \in \N}$ which is convergent in $W^{k + 4d, 1}$ and further
    \begin{equation}
        \lim\limits_{m \to \infty} \norm{(\lambda - \cA_n)(x_{n, m}) - \xi}_{W^{k + 4d, 1}}  = 0 \, .
    \end{equation}
    In addition, we can choose the sequence such that for all $m \in \N$, $\norm{(\lambda - \cA_n)(x_{n, m})}_{W^{k + 4d, 1}} \le \norm{\xi}_{W^{k + 4d,1}} + 1$. Due to the $\omega_{k + 4d}$-quasi dissipativity of $\cA_n$ (it is a generator of a strongly continuous semigroup with a bound given in \Cref{eq:uniform-bound-semigroups}) this immediately implies $\norm{x_{n, m}}_{W^{k + 4d, 1}} \le M_{k + 4d}\frac{\norm{\xi}_{W^{k + 4d, 1}} + 1}{\lambda - \omega_{k + 4d}} =: c_{\xi}$, i.e.~the set $\{x_{n, m}\}_{n, m \in \N}$ is bounded in $W^{k + 4d, 1}$. We now have that for $n, m \in \N$
    \begin{equation}\label{eq:dense-range-argument-inequality}
        \begin{aligned}
            \norm{(\lambda - \cA)(x_{n, m}) - \xi}_{W^{k, 1}} &\le \norm{(\lambda - \cA_n)(x_{n, m}) - \xi}_{W^{k, 1}} + \norm{(\cA - \cA_n)(x_{n, m})}_{W^{k, 1}}\\
            &\le \norm{(\lambda - \cA_n)(x_{n, m}) - \xi}_{W^{k, 1}} + \sum\limits_{i = 1}^K c_{i, k} |a_i - a_{i, n}|  \norm{x_{n, m}}_{W^{k + 4d, 1}}\\
            &\le \norm{(\lambda - \cA_n)(x_{n, m}) - \xi}_{W^{k, 1}} + \sum\limits_{i = 1}^K |a_i - a_{i, n}| c_{i, k} c_\xi\\
            &\le  \norm{(\lambda - \cA_n)(x_{n, m}) - \xi}_{W^{k, 1}} + C\sum\limits_{i = 1}^K |a_i - a_{i, n}| \, .
        \end{aligned}
    \end{equation}
    In the first line we used triangle inequality, in the second one the explicit form of $\cA$ and $\cA_n$ and then that there exists $c_{k, i} \ge 0$ such that 
    \begin{align*}
        \norm{(N + \1)^{k/4}A_{i, 1} x_{n, m} A_{i, 2} (N + \1)^{k/4}}_1\le c_{k, i} \norm{(N + \1)^d x_{n, m} (N + \1)^d}_{W^{k, 1}}
    \end{align*}
    as in the proof of \Cref{lem:infinitesimal-boundedness-W-k-1}. Lastly we used the uniform bound $c_\xi$ and set $C = \max_{i = 1, \hdots, K} c_{i,k} c_\xi$. By a proper choice of a subsequence of $\{x_{n, m}\}_{n, m \in \N}$, we get that the RHS of \Cref{eq:dense-range-argument-inequality} vanishes. Since $\{x_{n, m}\}_{n, m \in \N}$ is bounded in $W^{k + 4d, 1}$ it is in particular precompact in $W^{k, 1}$ (as of the compact embedding of the Sobolev spaces), meaning we can further choose the aforementioned sequence to be convergent in $W^{k, 1}$. Let us call it $\{y_n\}_{n \in \N} \subset \cT_f$. To summarise, for the chosen $\lambda$ and $\xi \in \cT_f$ arbitrary we have constructed a sequence $\{y_n\}_{n \in \N} \subset \cD(\cA)$ which is convergent in $W^{k, 1}$ and further $\{(\lambda - \cA)(y_n)\}_{n \in \N}$ converges to $\xi$ in $W^{k, 1}$. Hence the closure of the range of $(\lambda - \cA, \cD(\cA))$ contains $\cT_f$ a dense subset of $W^{k, 1}$, which concludes the proof. 
\end{proof}

\begin{rmk*}
    In the above lemma, it suffices to assume that the semigroups are Sobolev preserving, as one can interpolate between the sequence elements to obtain semigroups for $k \in \R_+$.
\end{rmk*}