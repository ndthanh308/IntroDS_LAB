\documentclass[11pt,reqno]{amsart}

\usepackage[utf8]{inputenc}
\usepackage{hyperref}
\hypersetup{citecolor=blue}
\usepackage{cleveref}
\usepackage[style=numeric,maxbibnames=99]{biblatex}
\DefineBibliographyExtras{UKenglish}{\def\finalandcomma{\addcomma}}
\addbibresource{references.bib}
\usepackage{import}
\usepackage{packages}
\usepackage{definitions}

\newcommand{\pg}[1]{{\color{purple}PG:~#1}}
\newcommand{\tm}[1]{{\color{mygreen}TM:~#1}}
\newcommand{\ca}[1]{{\color{blue}CR:~#1}}

\raggedbottom

\title{Energy preserving evolutions over Bosonic systems}
\author[Gondolf, M\"obus, Rouz\'e]{Paul Gondolf$^{1}$, Tim M\"obus$^{2,3}$, Cambyse Rouz\'e$^{2,3}$}
\email{paul.gondolf@uni-tuebingen.de, tim.moebus@tum.de, rouzecambyse@gmail.com}
\address{$^1$ Department of Mathematics, Eberhard Karls University T\"{u}bingen, Germany}
\address{$^2$ Department of Mathematics, Technical University of Munich,  M\"unchen, Germany}
\address{$^3$ Munich Center for Quantum Science and Technology (MCQST),  M\"unchen, Germany}

\begin{document}
\begin{abstract}
    The exponential convergence to invariant subspaces of quantum Markov semigroups plays a crucial role in quantum information theory. One such example is in bosonic error correction schemes, where dissipation is used to drive states back to the code-space --- an invariant subspace protected against certain types of errors. In this paper, we investigate perturbations of quantum dynamical semigroups that operate on continuous variable (CV) systems and admit an invariant subspace. First, we prove a generation theorem for quantum Markov semigroups on CV systems under the physical assumptions that (i) the generator has GKSL form with corresponding jump operators defined as polynomials of annihilation and creation operators; and (ii) the (possibly unbounded) generator increases all moments in a controlled manner. Additionally, we show that the level sets of operators with bounded first moments are admissible subspaces of the evolution, providing the foundations for a perturbative analysis. Our results also extend to time-dependent semigroups. We apply our general framework to two settings of interest in continuous variables quantum information processing. First, we provide a new scheme for deriving continuity bounds on the energy-constrained capacities of Markovian perturbations of Quantum dynamical semigroups. Second, we provide a quantitative analysis of the dampening of continuous-time evolutions generating a universal gate set for CAT-qubits outside their code-space.
\end{abstract}
\maketitle
\vspace{-0.5cm}
\tableofcontents

\newpage
\section{Introduction}\label{sec:introduction}
\section{Introduction}
Current quantum hardware is unable to carry out universal quantum computations due to the buildup of errors that occur during the computation. 
The magnitude of the individual error is currently above the value that the Threshold Theorem requires in order to kick-start quantum error correction and fault-tolerant quantum computation~\cite[Section 10.6]{nielsen_chuang_2010}. 
Although the experimentally achieved fidelity rates are promising and the error bounds are inching closer to the required threshold, we will have to work for the foreseeable future with quantum hardware with errors that build-up during the computation.  This implies that we can only do a limited number of steps before the output of the computation has become completely uncorrelated with the intended one.

For fault-tolerant quantum computing, we repeat four steps: 
1) We apply a number of single and two-qubit quantum gates, in parallel whenever possible; 
2) We perform a syndrome measurement on a subset of the qubits; 
3) We perform fast classical computations to determine which errors have occurred and how to correct them; 
and, 4) We apply correction terms based on the classical computations.
We then repeat these four steps with a next sequence of gates. 
These four steps are essential to fault-tolerant quantum computing. 


The starting point of this work is to use the four steps outlined above, not to carry out error correction and fault-tolerant computation, but to enhance short, constant-depth, {\em uncorrected} quantum circuits that perform single qubit gates and {\em nearest-neighbor} two qubit gates. 
Since in the long run we will have to implement error-correction and fault-tolerant computation anyhow, and this is done by such a four-step process, why not make other use of this architecture? Moreover, on some of the quantum hardware platforms, these operations are already in place.
Embracing this idea we naturally arrive at the question: what is the computational power of \textit{low-depth} quantum-classical circuits organized as in the four steps outlined above? 
We thus investigate circuits that execute a small, ideally constant, number of stages, where at each stage we may apply, in parallel, single qubit gates and {\em nearest-neighbor} two qubit gates, followed by measurements, followed by low-depth classical computations of which the outcome can control quantum gates in later stages. 
It is not clear, at first, whether such circuits, especially with constant depth, can do anything remotely useful. 
But we will see that this is indeed the case: many quantum computations can be done by such circuits in constant depth. 
By parallelizing quantum computations in this way, we improve the overall computational capabilities of these circuits, as we do not incur errors on qubits that are idle, simply because qubits are not idle for a very long time. 
Furthermore, reducing the depth of quantum circuits, at the cost of increasing width, allows the circuit to be run faster even if errors occur.

The first usage of such a four-step layout, not to do error correction, but to perform computations, can be found in the paradigm of measurement-based quantum computing~\cite{gottesman1999demonstrating,raussendorf2001one,jozsa2006introduction,clark2007generalised}: 
A universal form of quantum computing where a quantum state is prepared and operations are performed by measuring qubits in different bases, depending on previous measurements and intermediate measurements.

\citeauthor{PhamSvore2013} were the first to formalize the four-step protocol for performing computations~\cite{PhamSvore2013}. They included specific hardware topologies by considering two-dimensional graphs for imposing constraints on qubit interactions. In their model, they develop circuits for particularly useful multi-qubit gates, including specifying costs in the width, number of qubits, depth, number of concurrent time steps, size, and total number of non-Identity operations.
As a result, they find an algorithm that factors integers in polylogarithmic depth.
\citeauthor{Browne:2011} showed that the main tool in the work by \citeauthor{PhamSvore2013}, the fan-out gate, can also be replaced by additional log-depth classical computations in the measurement-based quantum computing setting~\cite{Browne:2011}.

More recently, \citeauthor{Cirac:2021} introduced a scheme to implement unitary operations involving quantum circuits combined with Local Operations and Classical Communication ($\mathsf{LOCC}$) channels: $\mathsf{LOCC}$-assisted quantum circuits~\cite{Cirac:2021}. Similarly to the four-step scheme we just described, they allow for a short depth circuit to be run on the qubits, followed by one round of $\mathsf{LOCC}$, in which ancilla qubits are measured and local unitaries are applied based on the measurement outcomes. They show that in this model any 1D transitionally invariant matrix-product state (MPS) with fixed bond dimension is in the same phase of matter as the trivial state. Similar ideas can be found in~\cite{TVV_NonAbelianTopologicalOrder_2022, tantivasadakarn2021long}.

In this work, we introduce a new model, called \textit{Local Alternating Quantum-Classical Computations} ($\LAQCC$). In this model we alternate between running quantum circuits (constrained by locality), ending in the measurement of a subset of qubits, and fast classical computations based on the measurement results. The outcome of the classical computations are then used to control future quantum circuits. We allow for flexibility in this model, by giving different constraints to the power of both the quantum circuits and the classical circuits as well as the number of alternations between them. 
Most attention will be given to $\LAQCC$ containing quantum circuits of constant depth, classical circuits of logarithmic depth and at most a constant number of alternations between them. 
Any circuit constructed in this model is considered to be of constant depth. 
We restrict ourselves to logarithmic depth classical computations, as this is the first natural and non-trivial extension beyond constant-depth classical computations. 
Constant-depth classical computations do however also have an equivalent constant-depth quantum implementation.

The definition of $\LAQCC$ sharpens the original definition of \citeauthor{PhamSvore2013} by adding constraints to the intermediate classical computations. This allows us to bound the power of $\LAQCC$ from above. 

The main result of \citeauthor{Cirac:2021}, that 1D translational invariant MPS with fixed bond dimension can be prepared by $\mathsf{LOCC}$-assisted circuits, relies on local symmetries of the MPS. These symmetries allow them to prepare local states (on a constant number of qubits) and glue them together by doing one round of the appropriate entangling measurement and corrections, after which they run a round of local unitaries to get the desired result. This general scheme for preparing states that exhibit an MPS description with the appropriate local symmetries requires only geometrically local unitaries and one round of measurement and corrections an therefore is accessible in $\LAQCC$. Studying different local symmetries, known as Symmetry Protected Topological (SPT) phases of matter, to find measurement-based constant depth circuits for states is a broad ongoing field of research~\cite{TVV_NonAbelianTopologicalOrder_2022, tantivasadakarn2021long, smith2023deterministic}. 
All these schemes have a $\LAQCC$ implementation.

%$\LAQCC$-circuits also exist for general schemes of preparing local states, based on the local tensors, and gluing them together using one round of entangled measurement and corrections, based on the local symmetry. 
%The main result of \citeauthor{Cirac:2021}, that 1D translational invariant MPS with fixed bond dimension can be prepared by $\mathsf{LOCC}$-assisted circuits, relies heavily on local symmetries of the MPS and as a result also has an equivalent $\LAQCC$ implementation. 
%The corrections applied after the measurement round are local unitaries depending on the local symmetries of the MPS. 

 

%This general scheme of preparing local states, based on the local tensors, and gluing it together by doing one round of entangled measurement and corrections, based on the local symmetry, is accessible in $\LAQCC$.
Note however that \citeauthor{Cirac:2021} also suggest a circuit for the $W$-state.
This circuit uses sequentially and dependent measurement-based corrections of the ancilla qubits. 
These dependent measurements translate to sequential alternations between the quantum and classical circuits and therefore increase the total depth to linear depth, exceeding the constant-depth constraints imposed by $\LAQCC$-circuits. 

We study the power of the $\LAQCC$ model with respect to state preparation, showing that even with only constant quantum-depth and logarithmic classical depth it remains possible to prepare states with long-range entanglement.
Another surprising result is that it is unlikely that $\LAQCC$ circuits are classically simulatable. We show that any instantaneous quantum polynomial-time (IQP) circuit~\cite{Bremner2010,Shepherd2009} has an $\LAQCC$ implementation.
Classical simulation of IQP circuits implies the collapse of the polynomial hierarchy to the third level, which is not believed to be true~\cite{Bremner2017}. Therefore, we expect that $\LAQCC$ circuits are unlikely to be classically simulatable. We bound the power of $\LAQCC$ by showing that it is contained in $\QNC^1$, the class of polynomial-size, log-depth circuits.

Next, we also study the power that intermediate classical calculations can add to quantum computations, by considering a new model that alternates between polynomially many polynomial-depth quantum circuits and unbounded classical computations
We study this model by doing a complexity theoretical analysis, where we draw inspiration from the notions of complexity given by \citeauthor{RosenthalYuen:2022}, \citeauthor{MetgerYuen:2023}, and \citeauthor{Aaronson:2004}.
All three complexity notions are based on the notion of state preparation, instead of more traditional definition of complexity such as the decidability of a computational problem. 
The first two consider classes based on sequences of quantum states preparable by a polynomial-sized quantum circuit, where the circuits are uniformly generated by a computational class, for instance, the class $\mathsf{PSPACE}$, which results in the complexity class $\mathsf{StatePSPACE}$~\cite{RosenthalYuen:2022,MetgerYuen:2023}.
The third notion considers a relative complexity, where the complexity is measured between two given states, and is measured by the number of gates, from a given gate-set, required to transform one state in another state~\cite{Aaronson:2004}. 
For our definition of state preparation complexity, we drop the uniformity constraint from~\cite{RosenthalYuen:2022,MetgerYuen:2023} and define a class as $\mathsf{StateX}$, which refers to states preparable by circuits of type $\mathsf{X}$. 
As an example, if $\mathsf{X} = \QNC^0$, this results in the class $\mathsf{StateQNC^0}$, which is the set of states preparable from the $\ket{0}^n$ state by poly-size constant-depth circuits. 
This notion is similar to the relative complexity from~\cite{Aaronson:2004}, where one state is the  $\ket{0}^n$ state and instead of counting the number of gates we consider the set of states preparable by a fixed number of gates. Using this notion of complexity we show that any state preparable by an $\LAQCC^*$ circuit is also preparable by a $\mathsf{PostQPoly}$ circuit, the class of circuits of polynomial depth with an additional post-selection gate. 

All Clifford circuits have a constant-depth $\LAQCC$ implementation, implying that any stabilizer state can be implemented by a constant-depth $\LAQCC$ circuit, see Section~\ref{sec:clifford_circuits} for a proof of this statement. 
Efficient circuits for stabilizer states have been known already through measurement-based quantum computing. Therefore this paper focuses on the preparation of non-stabilizer states, and as a surprising result we find novel constant-depth protocols for four very natural classes of non-stabilizer states.
Despite the extensive research into these four classes of non-stabilizer states and the many applications of them, no efficient constant- or low-depth state preparation protocols are known yet. We specifically consider these four classes as they are all often used as initial states in other algorithms.

The first state is a uniform superposition over an arbitrary number of states. 
This state finds applications in many quantum algorithms, as they often start with a uniform superposition over multiple states. 
This superposition is often achieved by applying Hadamard gates to every qubit due to its simplicity to prepare. 
Yet, the analysis of many algorithms, such as Shor's algorithm~\cite{Shor:1997}, would benefit from a different initial superposition. 
The circuit to prepare the uniform superposition over an arbitrary number of states uses an exact version of Grover search as a subroutine, that turns a probabilistic circuit, with a known constant probability of success, into a deterministic circuit. 
We use the circuit for preparing a uniform superposition over an arbitrary number of states as a subroutine in the next two quantum state preparation protocols. 

The second state is the $W$-state, the uniform superposition over all computational basis states of Hamming-weight~$1$, a natural long-ranged entangled state that displays a fundamentally nonequivalent type of entanglement from the Greenberger–Horne–Zeilinger state~\cite{WState:2000}, for which $\LAQCC$-type constant-depth circuits were previously known~\cite{PhamSvore2013, Cirac:2021}. 
The $W$-state is often used as benchmark for new quantum hardware~\cite{Haffner2005,Neeley2010,GarciaPerez:2021}. 
A novel way to prepare the $W$-state therefore gives a new way to benchmark different quantum devices with each other. 
A circuit for preparing the $W$-state was given in~\cite{Cirac:2021}, but this implementation requires sequentially alternating measurements followed by local unitaries, which in the $\LAQCC$ model is not considered to be of constant depth. 
We improve this protocol by giving an $\LAQCC$ implementation of the $W$-state, based on a compress-uncompress method that links the one-hot and binary encoding of integers.

The third state considered is the Dicke state, a generalization of the $W$-state, a superposition over all computational basis states with Hamming-weight $k$~\cite{Dicke:1954}. 
Dicke states have relevance in various practical settings.
For instance, for quantum game theory~\cite{zdemir2007}, quantum storage~\cite{Bacon_Compress:2006,Plesch:2010}, quantum error correction~\cite{ouyang2014permutation}, quantum metrology~\cite{toth2012multipartite}, and quantum networking~\cite{prevedel2009experimental}. 
Dicke states have been used as a starting state for variational optimization algorithms, most notably Quantum Alternating Operator Ansatz (QAOA)~\cite{Hadfield2019}, to find solutions to problems such as Maximum k-vertex Cover~\cite{Brandhofer2022,cook2020quantum}.
The ground states of physical Hamiltonians describing one-dimensional chains tend to show a resemblance to Dicke states such as states resulting from the Bethe ansatz, making them an ideal starting state when investigating the ground state behavior of these Hamiltonians~\cite{TDL_BetheAnsatzDerivation:2010,B_ExcitedStateQuantumPhaseTransitions:2013,DickeTransitions:2021}. 
For instance, the algorithm by \citeauthor{van2021preparing}, who give an algorithm to prepare the Bethe ansatz eigenstates of the spin-1/2 XXZ spin chain, starts by first preparing a Dicke state~\cite{van2021preparing}. 
A Dicke-state preparation protocol based on the compress-uncompress methodology used in the $W$-state furthermore finds applications in entanglement distillation, where the entanglement of a large state is concentrated on only a few qubits. 
Efficient deterministic circuits for preparing Dicke states have been proposed by \citeauthor{bartschi2019deterministic}~\cite{bartschi2019deterministic, bartschi2022deterministic_short_depth}. 
They provide a quantum circuit of depth $\mathO(k \log(\frac{n}{k}))$, allowing arbitrary connectivity, to prepare a Dicke state, which they conjecture to be optimal when $k$ is constant. 
In this work, we provide a constant-depth $\LAQCC$ circuit below their conjectured bound already for constant $k$. 
However, this does not directly disprove their conjecture, as we allow for intermediate measurements and classical computations. 
More significantly, we even construct constant-depth $\LAQCC$ circuits for $k = \mathO(\sqrt{n})$ greatly improving their bound.
This construction extends the compress-uncompress method for the $W$-state combined with additional subroutines. 

We continue with a log-depth state preparation protocol for the Dicke-state for arbitrary $k$. 
This protocol implements an efficient transformation between the factoradic number representation and the combinatorial number representation of a positive integer. 
The combinatorial number representation relates directly to the Dicke state. 
The provided efficient transformation between number representation systems might be of independent interest. 

We conclude by modifying our protocol for preparing a Dicke-state to a protocol that prepares quantum many-body scar states in constant-depth. 
These states have low entanglement and longer coherence times than states with similar energy density.
These characteristics make many-body scar states interesting to analyze and relevant within physics.
Many-body scar states appear for instance in the AKLT model~\cite{AKLT:1987,MRBAR:2018,MRB:2018} and different spin models~\cite{SI:2019,MOBFR:2020}.
Known methods for preparing these states have polynomial-depth~\cite{Gustafson:2023}, whereas our circuit has constant depth. 

% We conclude by studying the power that intermediate classical calculations can add to quantum computations. 
% In this study, we define a new model that relaxes constant-depth quantum circuits to polynomial depth quantum circuits, log-depth classical calculations to unbounded classical computations and a constant number of alternations to a polynomial number of alternations. 
% We call this model $\LAQCC^*$. 
% We study this model by doing a complexity theoretical analysis, where we draw inspiration from the notions of complexity given by \citeauthor{RosenthalYuen:2022}, \citeauthor{MetgerYuen:2023}, and \citeauthor{Aaronson:2004}.
% All three complexity notions are based on the notion of state preparation, instead of more traditional definition of complexity such as the decidability of a computational problem. 
% The first two consider classes based on sequences of quantum states preparable by a polynomial-sized quantum circuit, where the circuits are uniformly generated by a computational class, for instance, the class $\mathsf{PSPACE}$, which results in the complexity class $\mathsf{StatePSPACE}$~\cite{RosenthalYuen:2022,MetgerYuen:2023}.
% The third notion considers a relative complexity, where the complexity is measured between two given states, and is measured by the number of gates, from a given gate-set, required to transform one state in another state~\cite{Aaronson:2004}. 
% For our definition of state preparation complexity, we drop the uniformity constraint from~\cite{RosenthalYuen:2022,MetgerYuen:2023} and define a class as $\mathsf{StateX}$, which refers to states preparable by circuits of type $\mathsf{X}$. 
% As an example, if $\mathsf{X} = \QNC^0$, this results in the class $\mathsf{StateQNC^0}$, which is the set of states preparable from the $\ket{0}^n$ state by poly-size constant-depth circuits. 
% This notion is similar to the relative complexity from~\cite{Aaronson:2004}, where one state is the  $\ket{0}^n$ state and instead of counting the number of gates we consider the set of states preparable by a fixed number of gates. Using this notion of complexity we show that any state preparable by an $\LAQCC^*$ circuit is also preparable by a $\mathsf{PostQPoly}$ circuit, the class of circuits of polynomial depth with an additional post-selection gate. 

\paragraph{Summary of results}
\begin{itemize}
    \item We give a new definition of a computational model that captures the power of the four step process: applying a constant number of layers of one- and two-qubit gates; performing a syndrome measurement; perform a fast classical computation determining corrections; apply corrections. We call this model \emph{Local Alternating Quantum Classical Computations}, or $\LAQCC$ for short. In this model we bound the allowed quantum operations, intermediate classical calculations, and number of rounds separately. In Section~\ref{sec:LAQCC_model} we define this model and give a list of operations based on results from literature contained in this computational model. In some of these operations we explicitly use that we allow for multiple, but at most constant, rounds  of corrections.
    \item  We show show that there exist $\LAQCC$ circuits that can not be weakly simulated in Section~\ref{sec:IQP_in_LAQCC}. We further show that for every $\LAQCC$ circuit there exists a $\QNC^1$ circuit simulating it perfectly, in Section~\ref{sec:LAQCC_in_QNC1}.
    \item We introduce a new type computational complexity for preparing states and show that the extension of $\LAQCC$ where we allow a polynomial number of rounds and unbounded classical computation, is contained in $\mathsf{PostQPoly}$, the class of polynomial circuits with post-selection, in Section~\ref{sec:Complexity results}.
    \item We show a protocol to prepare the uniform superposition state of size $q$ in $\LAQCC$ using $\mathO(\ceil{\log_2(q)}^2)$ qubits in Section~\ref{sec:superposition_modulo_q}. 
    \item We show a protocol to prepare the $W_n$ state in $\LAQCC$ using $\mathO(n\log(n))$ qubits in Section~\ref{sec:W_state_in_LAQCC}.
    \item We show two ways of preparing the Dicke-$(n,k)$ state. The first method is in $\LAQCC$, works up to $k = \mathO(\sqrt{n})$, uses $\mathO(n^2\log(n))$ qubits, and is found in Section~\ref{sec:dicke:small_k}. The second method is in $\LAQCC\text{-}\mathsf{LOG}$ (an extension of $\LAQCC$ allowing for logarithmic number of alterations instead of constant), works for any $k$, uses $\mathO(\text{poly}(n))$ qubits, and is found in Section~\ref{sec:Dicke_in_LAQCC_LOG}. 
    \item We extend on our $\LAQCC$ method of generating Dicke-$(n,k)$ states for $k = \mathO(\sqrt{n})$ and show a protocol to generate many-body scar states for a particular Hamiltonian in $\LAQCC$ (Section~\ref{sec:many_body_scar}). 
\end{itemize}
Summarized in a table, we provide the following state generation protocols:
\begin{table}[htb]
\centering
\begin{tabular}{l|l|l|l}
\textbf{State description} & \textbf{Width} & \textbf{Depth} & \textbf{Implementation}\\
\hline 
Uniform superposition mod $q$: $\frac{1}{\sqrt{q}} \sum_{i = 0}^{q-1}\ket{i}$ & $\mathO(\ceil{\log^2 q})$ & $\mathO(1)$ & Section~\ref{sec:superposition_modulo_q}\\

$W$-state: $\frac{1}{\sqrt{n}}\sum_{i = 0}^{n-1}\ket{e_i}$ & $\mathO(n \log n)$ & $\mathO(1)$ & Section~\ref{sec:W_state_in_LAQCC}\\

Dicke-$(n,k)$, $k = \mathO(\sqrt{n})$: $\binom{n}{k}^{-1/2}\sum_{x \in \{0,1\}^n: |x| = k} \ket{x}$ &  $\mathO(n^2\log n)$ & $\mathO(1)$ 
&Section~\ref{sec:dicke:small_k}\\

Dicke-$(n,k)$: $\binom{n}{k}^{-1/2}\sum_{x \in \{0,1\}^n: |x| = k} \ket{x}$ & $\mathO(\text{poly}(n))$ & $\mathO(\log n)$ &Section~\ref{sec:Dicke_in_LAQCC_LOG}\\

QMBS: $\ket{S_k} = \frac{1}{k! \sqrt{\mathcal N(n,k)}}(Q^\dagger)^k \ket{\Omega}$ &  $\mathO(n^2\log n)$ & $\mathO(1)$  &  Section~\ref{sec:many_body_scar}
\end{tabular}
\caption{Summary of state preparation protocols given in this paper.}
\label{tab:sate_prep}
\end{table}
In the entry for the quantum many-body scar state $Q$ denotes the raising operator and $\mathcal N(n,k)=\binom{n-k-1}{k}$. 
Section~\ref{sec:many_body_scar} will provide more details on the variables and the implementation. 

\paragraph{Organization of the paper}
\noindent We first introduce relevant preliminaries in Section~\ref{sec:preliminaries}. 
In Section~\ref{sec:LAQCC_model} we formally define the class of Local Alternating Quantum-Classical Computations ($\LAQCC$). We also show that any Clifford circuit can be implemented in constant depth $\LAQCC$ (a result based on a result from measurement-based quantum computing~\cite{jozsa2006introduction}). 
This result allows us to give many useful multi-qubit gates and routines in Section~\ref{sec:gates_created_in_LAQCC}. 
Beyond that we show that constant depth $\LAQCC$ circuits are contained in $\QNC^1$ and that any $\mathsf{IQP}$ circuit has an $\LAQCC$ implementation.
We conclude this section with an analysis of a more powerful instantiation of $\LAQCC$ and show an inclusion with respect to the class $\mathsf{PostQPoly}$, which is the class of circuits of polynomial depth with one additional post-selection gate. 
In Section~\ref{sec:state_prep_in_LAQCC} we give $\LAQCC$ circuit implementations for preparing the uniform superposition over an arbitrary number of states, the $W$-state and the Dicke state up to $k = \mathO(\sqrt{n})$. We furthermore give a log-depth circuit implementation for preparing the Dicke state for any $k$. We conclude by showing a $\LAQCC$ circuit for generating many body scar states of a particular type of Hamiltonian.



\section{Preliminaries}\label{sec:preliminaries}
\section{Preliminaries}
In this section, we describe the necessary background for automated planning and the significance of the International Planning Competition. 

% \subsection{Ontology}
% A formal ontology is typically represented as a set of concepts, relations, and axioms. A concept represents a set of objects or entities that share common properties, while a relation represents a connection or association between two or more concepts. Axioms are statements that define the relationships between concepts and relations. It is a formal representation of knowledge that is designed to facilitate automated reasoning and information processing. It acts as a structured vocabulary that describes a domain and promotes interoperability, data integration, and communication between humans and machines. Formally, an ontology $O$ can be represented as a tuple $(C, R, A)$, where $C$ is the set of concepts, $R$ is the set of relations, and $A$ is the set of axioms. Each concept \textit{c} $\in$ $C$ can be represented as a set of attributes, denoted as $Att(c)$. Similarly, each relation \textit{r} $\in$ $R$ can be represented as a set of attributes, denoted as $Att(r)$.

% Ontology is a branch of philosophy that deals with the nature of existence and being. In the field of computer science, however, ontology refers to a formal representation of knowledge that is designed to facilitate automated reasoning and information processing. It is a structured vocabulary that describes a domain and promotes interoperability, data integration, and communication between humans and machines. Various tools and methodologies, including Protege and ontology editors, are available for ontology creation. Ontologies are increasingly important in artificial intelligence, knowledge engineering, and the semantic web, and researchers are exploring their potential in diverse domains and applications.

% Figure environment removed

\subsection{Automated Planning}

Automated planning, also known as AI planning, is the process of finding a sequence of actions that will transform an initial state of the world into a desired goal state \cite{ghallab2004automated}. It involves constructing a plan or a sequence of actions that will achieve a specified objective while respecting any constraints or limitations that may be present. Formally, automated planning can be defined as a tuple $(S, A, T, I, G)$, where:
\begin{itemize}
    \item $S$ is the set of possible states of the world
    \item $A$ is the set of possible actions that can be taken
    \item $T$ is the transition function that describes the effects of taking an action on the current state of the world
    \item $I$ is the initial state of the world
    \item $G$ is the desired goal state
\end{itemize}
Using this notation, the problem of automated planning can be framed as finding a sequence of actions $\prec a_1, a_2, ..., a_k\succ$ that will transform the initial state $I$ into the goal state $G$, while respecting any constraints or limitations on the actions. 
 % In automated planning, 
 A problem is defined in terms of a domain and a problem instance. The domain defines the possible actions that can be taken and the effects of each action, while the problem instance specifies the initial state of the world and the desired goal state. 
Various techniques can be used to solve the planning problem, such as search algorithms, constraint-based reasoning, and optimization methods. These techniques involve exploring the space of possible plans and selecting the one that satisfies the objective and any constraints. Figure \ref{fig:planning_bw} illustrates an automated planning scenario for the blocksworld domain, where an initial state can be transformed into a goal state by executing a sequence of actions.

% \noindent \textbf{Attributes modeled about a domain.}
%   %\noindent \textbf{Attributes modeled in a domain file}
%  \begin{enumerate}
%      \item \textbf{Requirements:} A list of requirements that the planner must satisfy in order to solve the domain. Requirements include durative actions, conditional effects, or negative preconditions. For example, in blocksworld domain with types involved, one of the requirements is \emph{typing}.
%     \item \textbf{Predicates:} Predicates are fundamental elements in the planning domain that define the properties of the world. They are used to describe the initial and goal states, as well as the preconditions and effects of actions. Predicates are usually defined as logical expressions over a set of variables, where each variable can take on a finite number of values. In the context of planning, predicates are typically used to represent facts about the world that can be true or false, such as the location of an object or the status of a machine. For example, in blocksworld domain, the predicate \verb|(on b1 b2)| could indicate that block 'b2' is on top of block 'b1'.
%      \item \textbf{Actions:} Actions are the basic units of change in the planning domain. They represent atomic operations that can be performed to transform the world from one state to another. Each action has a name, a set of parameters, preconditions that must be satisfied before the action can be executed, and effects that describe the changes that the action makes to the world. Actions can be used to model a wide variety of operations, ranging from simple movements or transformations to complex processes such as planning or decision-making. For example, in blocksworld domain, the action \verb|unstack b2 b1| can be used to unstack block 'b2' from block 'b1'. 
     
%      \item \textbf{Preconditions:} Preconditions are the conditions that must be true before an action can be executed. They are usually defined using predicates and can involve multiple variables. Preconditions can also be negative, which means that a certain condition must not be true for an action to be executed. In planning, preconditions ensure that actions are only executed when the necessary conditions have been met, such as ensuring that a machine is turned off before it is serviced. For example, in blocksworld domain, the action \verb|unstack b2 b1| has a precondition of \verb|(on b1 b2)|, meaning that for the action to be valid, the block 'b2' should be on top of block 'b1'.
     
%      \item \textbf{Effects:} Effects describe the changes that an action makes to the world. They are usually defined using predicates and can involve multiple variables. Effects can be positive, which means that a certain condition becomes true after the action is executed, or negative, which means that a certain condition becomes false after the action is executed. In the context of planning, effects are used to model the changes that result from executing an action, such as moving an object from one location to another or turning a machine on. For example, in blocksworld domain, when the action \verb|unstack b2 b1| is executed, one of its effect is \verb|(not (on b1 b2))|, indicating that block 'b2' is no longer on top of block 'b1'.
     
%      \item \textbf{Constants:} Constants are values that are fixed and do not change during the execution of the planning problem. They are used to represent objects or entities in the world that have a fixed value, such as the speed limit on a road. Constants can be used to simplify the planning problem by reducing the number of variables that need to be considered and by providing a fixed set of values that can be used in predicates and actions. For example, in blocksworld domain, the constant \emph{table} could represent the surface on which the blocks are initially placed.
     
%      \item \textbf{Types:} Types are used to classify objects or entities in the world based on their attributes or properties. They are used to define the domain of values that a variable can take on and can be used to constrain the values that are assigned to variables. In the context of planning, types are typically used to group related objects or entities together, such as cars or bicycles, and to specify the properties that are common to all members of a type, such as their color or size. For example, in blocksworld domain with types involved, one can represent the predicate as \verb|(on ?x - block ?y - block)| stating that the parameters in the predicate are of type \emph{block}.

%  \end{enumerate}


% ######### Shorter version for AI Planning preliminaries
% \subsection{Automated Planning}

% Automated planning, also known as AI planning, finds actions transforming an initial world state into a goal state \cite{ghallab2004automated}. It involves creating a plan, respecting constraints, defined as $(S, A, T, I, G)$ where $S$ is the world states set, $A$ is the actions set, $T$ is the state transition function, $I$ is the initial state, and $G$ is the goal state. The challenge is to find actions $\prec a_1, a_2, ..., a_k\succ$ converting $I$ to $G$ under constraints. 

% A problem has a domain (defining actions and effects) and an instance (specifying initial and goal states). Various techniques can be used to solve the planning problem, such as search algorithms, constraint-based reasoning, and optimization methods. These techniques involve exploring the space of possible plans and selecting the one that satisfies the objective and any constraints. Figure \ref{fig:planning_bw} illustrates an automated planning scenario for the blocksworld domain, where an initial state can be transformed into a goal state by executing a sequence of actions.

\noindent \textbf{Attributes modeled about a domain.}
 \begin{enumerate}
     \item \textbf{Requirements:} A list of requirements that the planner must satisfy to solve the given domain, e.g., \emph{typing} in blocksworld with types.
     \item \textbf{Predicates:} Define world properties, e.g., \verb|(on b1 b2)| in blocksworld.
     \item \textbf{Actions:} Units of change with preconditions and effects, e.g., \verb|unstack b2 b1| in blocksworld.
     \item \textbf{Preconditions:} Conditions for action execution, e.g., \verb|(on b1 b2)| for \\ \verb|unstack b2 b1|.
     \item \textbf{Effects:} Post-action world changes, e.g., \verb|(not (on b1 b2))| after \\ \verb|unstack b2 b1|.
     \item \textbf{Constants:} Fixed values, e.g., \emph{table} in blocksworld.
     \item \textbf{Types:} Classifications based on attributes, e.g., \\ \verb|(on ?x - block ?y - block)| in typed blocksworld.
 \end{enumerate}

\noindent \textbf{Attributes modeled about a problem instance from a domain.}
\begin{enumerate}
    \item \textbf{Name:} The name of the planning problem.
    \item \textbf{Domain:} The name of the planning domain that the problem belongs to.
    \item \textbf{Objects:} A list of objects that are present in the planning problem. Objects are typically defined in terms of their type and name. In the example shown in Figure \ref{fig:planning_bw}, objects are b1, b2, and b3.
    \item \textbf{Initial State:} A description of the initial state of the world, including the values of all relevant predicates. Figure \ref{fig:planning_bw} represents an example initial state.
    \item \textbf{Goal State:} A description of the desired goal state of the world, including the values of all relevant predicates. Figure \ref{fig:planning_bw} represents an example goal state.
\end{enumerate}

% \vspace{2cm}
\subsection{International Planning Competition (IPC)}

% IPC serves as a significant means of assessing and comparing various planning systems. By presenting new planners and benchmark problems each year, the competitions aim to stimulate the advancement of new planning methodologies and reflect current trends and challenges in the field. The competition comprises multiple tracks, each covering various planning problems such as classical, temporal, and probabilistic planning. These tracks include benchmark problems that evaluate the performance of planners concerning parameters such as plan quality, plan length, and run time. The results of these competitions provide insights into the current state-of-the-art in planning and help identify the strengths and weaknesses of different planning systems. IPC can serve as an excellent starting point for building a planning-related ontology as the benchmark problems used in these competitions can provide a comprehensive overview of the domain and the types of problems that planners need to solve. 

IPC is pivotal for evaluating and contrasting planning systems. Introducing new planners and benchmarks, it promotes innovative planning methodologies and reflects the field's evolving challenges. The competition has multiple tracks, such as classical and probabilistic planning, with benchmarks assessing plan quality, length, and run time. IPC results offer a glimpse into the latest in planning, highlighting system pros and cons. The benchmarks from IPC are ideal for crafting a planning-related ontology, encapsulating the domain's breadth and planners' challenges.


\section{Sobolev preserving quantum Markov semigroups}\label{sec:polynomial-generators}
A quantum evolution in bosonic systems is described by a master equation 
\begin{equation}\label{eq:mastereq}
    \frac{d}{dt} x(t) = \cL(x(t)) \quad x(0) \in \cD(\cL) \quad\text{and}\quad t \ge 0 \, . 
\end{equation}
where $\cL$ is potentially unbounded. In the following, we state two sufficient assumptions for the existence and uniqueness of an operator-valued solution to \eqref{eq:mastereq} in terms of a semigroup. In other words, we prove a generation theorem for bosonic quantum Markov semigroups. This is generalized in \Cref{sec:timedependentgeneration} to the case of time-dependent generators. 

\subsection{Strongly continuous bosonic semigroups}\label{subsec:time-indep-generation}

We start with the time-independent setting, for which we need two working assumptions. The first assumption is motivated by the so-called GKSL \cite{Lindblad.1976,Gorini1976} form that generators of quantum dynamical semigroups over finite-dimensional quantum systems take, as well as our natural choice to consider jump and Hamiltonian operators described by polynomials in the annihilation and creation operators:

\begin{assum}\label{assum:finite-degree}
    The operator $(\cL,\cT_f)$ has $\operatorname{GKSL}$ form, i.e.~for $x\in\cT_f$ 
    \begin{equation}\label{eq:lindblad}
        \begin{aligned}
            \cL:\cT_f \to \cT_f \quad x \mapsto\cL(x) &= - i [H, x] + \sum\limits_{j = 1}^K L_j x L_j^\dagger  - \frac{1}{2}\{L_j^\dagger L_j, x\} \\
            &\coloneqq Gx + x G^\dagger + \sum\limits_{j = 1}^K L_j x L_j^\dagger\, , 
        \end{aligned}
    \end{equation}
    for some $K\in\mathbb{N}$ and with $G=-iH-\frac{1}{2}\sum_{j=1}^K L_j^\dagger L_j$, where $\{A,B\}=AB+BA$ denotes the anticommutator of two operators $A,B$ on a suitable domain. For the above equation to make sense, the operators $H$ and $L_j$ are assumed to be polynomials of the creation and annihilation operators, i.e.~$H\coloneqq p_H(a,\ad)$ and $L_j\coloneqq p_j(a,\ad)$, and $H$ is assumed to be symmetric. This ensures that $\cT_f$ is invariant under $\cL$.
   We denote the degree of $p_H$ by $d_H\coloneqq\deg(p_H)$, those of $p_j$ by $d_j\coloneqq\deg(p_j)$, and $d\coloneqq\max\{d_1,...,d_K,d_H\}$.
\end{assum}

The second assumption will lead to the semigroup being Sobolev preserving, which allows us not only to prove the existence and uniqueness of the evolution generated by \eqref{eq:mastereq} but further to conduct a perturbation analysis as well as to extend our results to the case of a time-dependent Master equation:

\begin{assum}\label{assum:sobolev-stability}
    There exists a non-negative sequence $\{k_r\}_{r \in \N}\rightarrow\infty$ s.t.~for all $r \in \N$ there exist $\omega_{k_r} \ge 0$ such that for all positive semi-definite $x \in \cT_f$
    \begin{equation}\label{eq:Assumptionsobolevstability}
        \tr[\cL(x) (N + \1)^{k_r/2}] \le \omega_{k_r} \tr[x (N + \1)^{k_r/2}] \, .
    \end{equation}
\end{assum}

We are now ready to state and prove the main theorem of the section: 
\begin{thm}[Generation of bosonic semigroups]\label{thm:generation-theorem}
    Let $(\cL, \cD(\cL))$ be an operator defined on $\cT_{1,\operatorname{sa}}$. If $(\cL, \cD(\cL))$ satisfies \Cref{assum:finite-degree} and \Cref{assum:sobolev-stability}, then the closure $\overline{\cL}$ generates a strongly continuous, positivity preserving semigroup $(\cP_t)_{t\ge 0}$ on $W^{k, 1}$ for all $k\geq 0$ with 
    \begin{equation}
        \norm{\cP_t}_{W^{k, 1} \to W^{k, 1}} \le e^{\omega_k t} \,\quad \forall t\ge 0\, . 
    \end{equation}
   where $\omega_k = \frac{k_{r_1} - k}{k_{r_1} - k_{r_0}}\omega_{k_{r_0}} + \frac{k - k_{r_0}}{k_{r_1} - k_{r_0}}\omega_{k_{r_1}}$ for an $r$ such that $k_{r_0}\leq k <k_{r_1}$. Finally, for $k = 0$, the semigroup is contractive and trace-preserving.
\end{thm}

Before proving \Cref{thm:generation-theorem}, we provide two examples for which \Cref{assum:sobolev-stability} is not satisfied.

\begin{ex}[Pure birth process \texorpdfstring{\cite[Ex.~3.3.]{Davies.1977}}{[9,Ex.~3.3.]}]\label{eq:no-sobolev-stability}
    Let $L=(\ad)^2$ and $G=-\frac{1}{2}a^2(\ad)^2$, i.e.~
    \begin{equation*}
        \cL(x)=Gx+x G^\dagger+Lx L^\dagger\,.
    \end{equation*}
    By construction, this generator satisfies \Cref{assum:finite-degree}. However, one can show that it is not trace-preserving, and therefore it cannot satisfy \Cref{assum:sobolev-stability}.
\end{ex}

\paragraph{\emph{Proof strategy}:} Our proof is partly inspired by \cite{Davies.1977}, however, \Cref{assum:sobolev-stability} allows us to go beyond minimal semigroups and to thereby generate a trace-preserving semigroup. An important intermediate step is that the considered generators are also generators on $W^{k,1}$, which will allow us to provide simple perturbation analysis on specific examples in \Cref{sec:example-perturbation-bounds}. Our proof starts with \Cref{lem:semigroup-of-G-for-positivity}, where we show that $G_\varepsilon\coloneqq G-\varepsilon (N+\1)^{4d}$ is a generator on the Fock space and the implemented semigroup $t\mapsto e^{tG_\varepsilon}\cdot e^{tG_\varepsilon^\dagger}$ admits $\cT_f$ as a core. Then \Cref{lem:semigroup-of-G-for-sobolev-stability} extends the result to semigroups on $W^{k, 1}$ for all $k \in \R_+$. By the compact embedding lemma \ref{thm:compact-embedding-weighted-spaces} for $W^{k, 1}$ in $\cT_{1,\operatorname{sa}}$, we can transfer these properties to the unperturbed evolution. Next, we more closely follow the method introduced in \cite{Davies.1977}. In particular, we prove that a perturbed version of \Cref{eq:lindblad} generates a Sobolev and positivity preserving $C_0$-semigroup.

\begin{lem}\label{lem:semigroup-of-G-for-positivity}
    For $\varepsilon > 0$, the closure of the operator
    \begin{equation*}
        \cG_\varepsilon: \cT_f \to \cT_f, \qquad x \mapsto G x + x G^\dagger - \varepsilon \{(N + \1)^{4d}, x\}\,,
    \end{equation*}
    where $G$ is defined in \Cref{eq:lindblad}, generates a strongly continuous, contractive, positivity preserving semigroup on $\cT_{1,\operatorname{sa}}$.
\end{lem}
\begin{proof}
    The proof is structured in the following two steps: 
    \begin{itemize}
        \item[1)] The closure of $G_\varepsilon:\cH_f \to \cH$, $\ket{\psi} \mapsto G_\varepsilon\ket{\psi} \coloneqq (-\varepsilon(N + \1)^{4d} + G)\ket{\psi}$ generates a strongly continuous contractive semigroup on $\cH$, which we denote by $(P_t^\varepsilon)_{t \ge 0}$.
        \item[2)] The family of maps $(\cP_t^\varepsilon \coloneqq P_t^\varepsilon \cdot (P_t^\varepsilon)^\dagger:\cT_{1,\operatorname{sa}} \to \cT_{1,\operatorname{sa}})_{t \ge 0}$, with $(P_t^\varepsilon)_{t \ge 0}$ from step 1, defines a strongly continuous, contractive, positivity preserving semigroup on $\cT_{1,\operatorname{sa}}$ generated by the closure of $\cG_\varepsilon$.
    \end{itemize}
    
    \textit{Step 1)} By \Cref{assum:finite-degree} there exists $p_\varepsilon \in \C[X,Y]$ such that $G_\varepsilon = p_\varepsilon(a, a^\dagger)$ which shows by \Cref{lem:formal-polynomial-ccr-adjoint-core} that $G_\varepsilon$ is closed with domain $\cD(N^{4d})$. We will now show dissipativity for $G_\varepsilon$ and $G_\varepsilon^\dagger$ to conclude the claim using \Cref{cor:lumer-phillips}. It suffices also to consider $G^\dagger_\varepsilon$ on $\cH_f$, as it is a core by \Cref{lem:formal-polynomial-ccr-adjoint-core} ($\deg(G)=2d$) and therefore dissipativity of $G_\varepsilon^\dagger$ on $\cH_f$ directly implies dissipativity of $G^\dagger_\varepsilon$ on all of its domain. We only show the dissipativity of $G_\varepsilon$, since the proof for $G_\varepsilon^\dagger$ is completely analogous. Let $\ket{\psi}\in \cH_f$, then for any $\lambda>0$
    \begin{align*}
        \norm{(\lambda - G_{\varepsilon}) \ket{\psi}}^2 &= \lambda^2 \braket{\psi\,,\psi} + \braket{G_\varepsilon\psi, G_\varepsilon\psi} - \lambda(\braket{G_\varepsilon\psi,\psi} + \braket{\psi, G_\varepsilon\psi})\\
        &\ge\lambda^2 \braket{\psi\,,\psi} - \lambda(\braket{G_\varepsilon\psi,\psi} + \braket{\psi, G_\varepsilon\psi})\\
        &\ge \lambda^2 \braket{\psi\,,\psi} + \varepsilon \lambda (\braket{(N + \1)^{4d}\psi,\psi} + \braket{(N + \1)^{4d}\psi,\psi}) - \lambda(\braket{G\psi,\psi} + \braket{\psi, G\psi})\\
        &\ge \lambda^2 \braket{\psi\,,\psi} - \lambda(\braket{G\psi,\psi} + \braket{\psi, G\psi}) \, . 
    \end{align*}
    By the requirement of \Cref{assum:finite-degree}, it is clear that $\tr[\cL(x)] = 0$ for $x \in \cT_f$, using the cyclicity of the trace. For the explicit case of a pure state $x = \ketbra{\psi}{\psi} \in \cT_f$ where the last inclusion holds due to $\ket{\psi} \in \cH_f$, we get
    \begin{equation*}
        \lambda\braket{\psi, -(G^\dagger + G) {\psi}} = \lambda \sum\limits_{j = 1}^K \braket{L_j\psi, L_j {\psi}} \ge 0 \, .
    \end{equation*}
    Hence, we conclude 
    \begin{equation*}
        \norm{(\lambda - G_\varepsilon)\ket{\psi}}^2 \ge \lambda^2 \braket{\psi,\psi} = \lambda^2 \norm{\ket{\psi}}^2 \, . 
    \end{equation*}
    Taking the square root on both sides proves the claim. Note that for $G_\varepsilon^\dagger$ all steps are similar due to the simple observation that $\braket{G\psi,\psi} + \braket{\psi, G\psi} = \braket{G^\dagger\psi,\psi} + \braket{\psi, G^\dagger\psi}$ for $\ket{\psi} \in\cH_f$.
    
    \textit{Step 2)} That the implemented semigroup\footnote{A discussion on implemented semigroups can be found in \cites{Alber.2001}.} $(\cP_t)_{t \ge 0}$ is a strongly continuous, positivity-preserving contractive semigroup that can be easily checked. We further get from \cite[Prop.~2.1]{Davies.1977} that it is generated by the closure of the operator
    \begin{equation*}
        \widetilde{\cG}_\varepsilon: \cD(\widetilde \cG_\varepsilon) = \{R(1,\overline{G}_\varepsilon) x R(1,\overline{G}_\varepsilon)^\dagger \; : \; x \in \cT_{1,\operatorname{sa}}\}  \to \cT_{1,\operatorname{sa}}, \quad x \mapsto \overline{G}_\varepsilon x + x \overline{G}_\varepsilon^\dagger
    \end{equation*}
    where $\overline{G}_\varepsilon$ is the closure of $G_\varepsilon$, $\overline{G}_\varepsilon^\dagger$ its adjoint, and $R( 1,\overline{G}_\varepsilon)$ its resolvent on $\cH$, respectively. Since $\cH_f$ is a core for the generator $\overline{G}_\varepsilon$, the set $O \coloneqq (\1 - \overline{G}_\varepsilon)\cH_f = (\1 - G_\varepsilon)\cH_f$ is dense in $\cH$, which in turn means $\cO = \text{span}\{\ketbra{\psi}{\varphi} \;:\; \ket{\psi}, \ket{\varphi} \in O\}$ is dense in $\cT_{1,\operatorname{sa}}$. A simple calculation further shows that $R(1,\overline{G}_\varepsilon)\cO R(1,\overline{G}_\varepsilon)^\dagger = \cT_f$. Hence for $y \in \cD(\widetilde \cG_\varepsilon)$, we find $x \in \cT_{1,\operatorname{sa}}$ and a sequence $\{x_n\}_{n \in \N} \subseteq \cO$ with $x_n \to x$ for $n \to \infty$ such that
    \begin{align*}
        \widetilde \cG_\varepsilon(y) &= \widetilde \cG_\varepsilon(R(1,\overline{G}_\varepsilon) x R(1,\overline{G}_\varepsilon)) = \widetilde \cG_\varepsilon(R(1,\overline{G}_\varepsilon) \lim\limits_{n \to \infty}x_n R(1,\overline{G}_\varepsilon)^\dagger)\\
        &= \lim\limits_{n \to \infty} \widetilde\cG_\varepsilon(R(1,\overline{G}_\varepsilon)x_n R(1,\overline{G}_\varepsilon)^\dagger)\\
        &= \lim\limits_{n \to \infty} \cG_\varepsilon(y_n)
    \end{align*}
    where we used the continuity of the map $\widetilde \cG_\varepsilon(R(1,\overline{G}_\varepsilon) \cdot R(1,\overline{G}_\varepsilon)^\dagger)$ and set $y_n = R(1,\overline{G}_\varepsilon) x_n R(1,\overline{G}_\varepsilon)^\dagger$. Note that $\{y_n\}_{n \in \N}$ is by construction a convergent sequence on $\cT_f$. In the last equality, we used that $\widetilde \cG_\varepsilon$ and $\cG_\varepsilon$ agree on $\cT_f$. This shows not only that $\cG_\varepsilon$ is closable but further that its closure is the closure of $\widetilde \cG_\varepsilon$. Hence the closure of $\cG_\varepsilon$ is the generator of $(\cP_t^\varepsilon)_{t \ge 0}$.
\end{proof}
Using the above lemma, we are now able to prove the following.

\begin{lem}\label{lem:semigroup-of-G-for-sobolev-stability}
    For all $k \ge 0$ and $\varepsilon > 0$, the closure of the operator $\cG_\varepsilon$ from \Cref{lem:semigroup-of-G-for-positivity} generates a strongly continuous, positivity preserving semigroup $(\cP^\varepsilon_t)_{t \ge 0}$ on $W^{k, 1}$ such that, for all $t\ge 0$, 
    \begin{equation*}
        \norm{\cP^\varepsilon_t}_{W^{k, 1} \to W^{k, 1}} \le e^{\omega_k t} 
    \end{equation*}
    where $\omega_k = \frac{k_{r_1} - k}{k_{r_1} - k_{r_0}}\omega_{k_{r_0}} + \frac{k - k_{r_0}}{k_{r_1} - k_{r_0}}\omega_{k_{r_1}}$ for an $r$ such that $k_{r_0}\leq k <k_{r_1}$. Finally, for $k = 0$, the semigroup is contractive.
\end{lem}
\begin{proof}
   Without loss of generality, we can restrict to $k \in \{k_r\}_{r \in \N}$ as for $k$ inbetween, we can interpolate between the $\{k_r\}_{r \in \N}$ (shown below) and $k = 0$ (shown in \Cref{lem:semigroup-of-G-for-positivity}) using \Cref{lem:interpolation-lemma}. Let $\varepsilon > 0$. In the following proof, the closure, domain and boundedness of an operator are always with respect to the Banach space $W^{k, 1}$ if not stated otherwise. We will show the claim, by first arguing that $\cG_\varepsilon$ is closable, that all $\lambda > \omega_k$ are in the resolvent set of the closure $\overline{\cG}_\varepsilon$ and further that $\norm{R(\lambda,\overline{\cG}_\varepsilon)}_{W^{k, 1} \to W^{k, 1}} \le \tfrac{1}{\lambda - \omega_k}$. By \Cref{thm:hille-yosida}, the above immediately gives the existence of the semigroup on $W^{k, 1}$ and provides us with the claimed bound. The property of positivity preservation traces back to the representation of the semigroup via the Euler approximation and therefore the positivity of the resolvent: for any $x\in \cT_{1,\operatorname{sa}}$:
    \begin{equation}
        \cP_{t}^{\varepsilon}(x)=\lim_{n\to\infty}\,\left(\frac{n}{t}\right)^n\,R(n/t,\overline{\cG}_\varepsilon)^{n}(x)\,.
    \end{equation}
    
    The claims are proven in three steps.
    \begin{itemize}
        \item[\normalfont Step 1.]\label{step:step-1} Show that $\cG_\varepsilon:\cT_f \to \cT_f$ is closable and there exists a $\lambda > \omega_k$ such that $\lambda - \overline{\cG}_\varepsilon:\cD(\overline{\cG}_\varepsilon) \to W^{k, 1}$ is bijective.
        \item[\normalfont Step 2.] Using \Cref{assum:sobolev-stability} and \Cref{lem:semigroup-of-G-for-positivity} we prove that if $\lambda > \omega_k$ is in the resolvent set of $\overline{\cG}_{\varepsilon}$, we not only have that the resolvent is positivity preserving but further $$\norm{R(\lambda,\overline{\cG}_\varepsilon)}_{W^{k, 1} \to W^{k, 1}} \le \frac{1}{\lambda - \omega_k}\,.$$
        \item[\normalfont Step 3.] The surjectivity of $\lambda - \overline{\cG}_\varepsilon$ for a specific $\lambda > \omega_k$ from step 1.~and the bound on the resolvent from step 2.~allow us to successively use the series expansion of the resolvent as it is done in \cite[Prop.~IV.1.3]{Engel.2000} to get that $(\omega_k, \infty)$ is in the resolvent set of $\overline{\cG}_\varepsilon$, and therefore conclude the proof.
    \end{itemize}
    \medskip 
    
   \noindent  Proof of step 1.~We introduce the map 
    \begin{equation*}
        \cI_{d, \varepsilon}: \cT_f \to \cT_f, \qquad x \mapsto \cI_{d, \varepsilon}(x) \coloneqq - \varepsilon\{(N + \1)^{4d}, x\} \, . 
    \end{equation*}
    For $\lambda \ge 0$, $x \in \cT_f$, we can use \Cref{lem:(n+1)-(n+1)-properties} to write
    \begin{equation*}
        (\lambda - \cG_\varepsilon)(x) = (\1 - \cG_0 \circ (\lambda - \cI_{d, \varepsilon})^{-1}) \circ (\lambda - \cI_{d, \varepsilon})(x)
    \end{equation*}
    where $\circ$ is the function composition and $\lambda - \cI_{d, \varepsilon}:\cT_f \to \cT_f$ is a bijection, with bounded inverse (see \Cref{lem:(n+1)-(n+1)-properties}) between dense subspaces of $W^{k, 1}$. This means in particular that it is closable and that its closure has a bounded inverse. We will hence focus on the map $\1 - \cG_0 \circ (\lambda - \cI_{d, \varepsilon})^{-1}:\cT_f \to \cT_f$ and show that it is bounded (on the dense subset $\cT_f$ of $W^{k, 1}$) and hence uniquely extendable to all of $W^{k, 1}$. This then immediately gives us that $\lambda - \cG_0:\cT_f \to \cT_f$ is closable as it is the composition of a map with a dense range succeeded by a bounded map. Note first that we can apply \Cref{lem:infinitesimal-boundedness-W-k-1} to $\cG_0$ to get that there exists $C_k \ge 0$ such that for all $\kappa > 0$ and $x \in \cT_f$
    \begin{equation*}
        \norm{\cG_0(x)}_{W^{k, 1}} \le \kappa \norm{\cI_{d, \varepsilon}(x)}_{W^{k, 1}} + \frac{C_k}{\kappa\varepsilon} \norm{x}_{W^{k, 1}}\,.
    \end{equation*}
    Using the bijectivity of $\lambda - \cI_{d, \varepsilon}:\cT_f \to \cT_f$ we get for $x \in \cT_f$
    \begin{align*}
         \norm{\cG_0 \circ (\lambda - \cI_{d, \varepsilon})^{-1}x}_{W^{k, 1}} \le \kappa\norm{\cI_{d, \varepsilon} \circ (\lambda - \cI_{d, \varepsilon})^{-1}(x)}_{W^{k, 1}} + \frac{C_k}{\varepsilon \kappa} \norm{(\lambda - \cI_{d, \varepsilon})^{-1}(x)}_{W^{k, 1}}\\
         \le ({2}\kappa + \frac{C_k}{\kappa \varepsilon} \frac{1}{\lambda + 2\varepsilon}) \norm{x}_{W^{k, 1}} =: f_k(\lambda, \kappa) \norm{x}_{W^{k, 1}}
    \end{align*}
    where we used properties of $\cI_{d, \varepsilon}$ derived in \Cref{lem:(n+1)-(n+1)-properties}. This gives us not only that $\cG_0 \circ (\lambda - \cI_{d, \varepsilon})^{-1}:\cT_f \to \cT_f$ is bounded, hence uniquely extendable to a bounded map on $W^{k, 1}$ but for a fixed $\kappa < {\frac{1}{2}}$ and $\lambda > \lambda_\kappa$ where $\lambda_\kappa$ is chosen s.t. $f_k(\lambda_\kappa, \kappa) < 1$, we get that its closure is a strict contraction on $W^{k, 1}$. As a direct consequence, we find that again for $\lambda > \lambda_\kappa$ the closure of $\1 - \cG_0 \circ (\lambda - \cI_{d, \varepsilon}): \cT_f \to \cT_f$ is invertible with bounded inverse, and that its inverse function is just given by the geometric series of the closure of $\cG_0 \circ (\lambda - \cI_{d, \varepsilon})^{-1}$. To conclude, we can set $\lambda = 0$ in the above result and get that $-\cG_\varepsilon$ and hence $\cG_\varepsilon$ is closable and further that for $\kappa < \frac{1}{2}$ all $\lambda$ with $\lambda > \lambda_\kappa$ are in the resolvent set of $\overline{\cG}_{\varepsilon}$.
    \medskip
    
    \noindent Proof of step 2. Let $\lambda > \omega_k$ be in the resolvent set of $\overline{\cG}_\varepsilon$. From the compact embedding of $W^{k, 1}$ in $\cT_{1,\operatorname{sa}}$, we immediately get that $R(\lambda,\overline{\cG}_\varepsilon):W^{k, 1} \to W^{k, 1}$ agrees with the respective restricted resolvent of the closure $\widehat{\cG}_\varepsilon$ of $\cG_\varepsilon$ on $\cT_{1,\operatorname{sa}}$ that we obtained in \Cref{lem:semigroup-of-G-for-positivity}. We know that the latter resolvent is positivity preserving, as the semigroup is. This is due to the following integral representation for strongly continuous semigroups \cite[Thm.~II.1.10 (i)]{Engel.2000}: for all $x\in x(\widehat{\cG}_\varepsilon)$,
    \begin{equation}
        R(\lambda, \widehat{\cG}_\varepsilon)(x)=\int_0^\infty\,e^{-\lambda s}\,e^{s\widehat{\cG}_\varepsilon}(x)\,ds\,.
    \end{equation}
     Hence $R(\lambda,\overline{\cG}_\varepsilon)$ is positivity preserving as well. Using \Cref{assum:sobolev-stability}, we have that for $x \in \cT_f$, $x$ positive semi-definite,
    \begin{equation*}
        \tr[\cL(x) (N + \1)^{k/2}] \le \omega_k \tr[x (N + \1)^{k/2}] \, .
    \end{equation*}
    Adding non-negative terms, using the cyclicity of the trace and splitting up $\cL$ gives us
    \begin{align*}
        &\sum\limits_{j = 1}^K\tr[(N+\1)^{k/4}L_j x L_j^\dagger(N+\1)^{k/4}] + (\lambda - \omega_k)\tr[(N + \1)^{k/4} x (N + \1)^{k/4}] \\
        &\qquad\qquad\qquad\qquad\qquad\qquad\qquad\qquad\qquad\qquad\qquad\qquad \le \tr[(N + \1)^{k/4}(\lambda - \cG_\varepsilon)(x)(N + \1)^{k/4}]\,,
    \end{align*}
    and therefore
    \begin{align*}
        (\lambda - \omega_k) \norm{x}_{W^{k, 1}} \le \norm{(\lambda - \cG_\varepsilon)(x)}_{W^{k, 1}} \, 
    \end{align*}
    where we have just dropped non-negative terms and used $\tr[\cdot] \le \norm{\cdot}_1$ with equality if the argument is positive semi-definite. Since $\overline{\cG}_\varepsilon$ is the closure of $\cG_\varepsilon$, the above inequality extends to $x \in \cD( \overline{\cG}_\varepsilon)$, $x$ positive semi-definite and $\overline{\cG}_\varepsilon$ instead of $\cG_\varepsilon$. Together with the positivity preserving property of the resolvent, this gives us that for all $x \in W^{k, 1}$, $x$ positive semi-definite
    \begin{equation}\label{eq:resolvent-bound-G-positive-semidefinite}
        \norm{R(\lambda,\overline{\cG}_\varepsilon)x}_{W^{k, 1}} \le \frac{1}{\lambda - \omega_k} \norm{x}_{W^{k, 1}} \, . 
    \end{equation}
    For a general $x \in W^{k, 1}$, we set $x_\pm = \frac{1}{(N + \1)^{k/4}}\,[(N + 1)^{k/4} x (N + \1)^{k/4}]_\pm \,\frac{1}{(N + \1)^{k/4}} \in W^{k, 1}$, where $[\cdot]_\pm$ denotes the positive, resp. the negative part of a self-adjoint trace-class operator. We clearly have that $x = x_+ - x_-$ and further that $x_+, x_-$ are positive semi-definite by construction. Hence
    \begin{align*}
        \norm{R(\lambda,\overline{\cG}_\varepsilon)x}_{W^{k, 1}} &\le \norm{R(\lambda,\overline{\cG}_\varepsilon )x_+}_{W^{k, 1}} + \norm{R(\lambda,\overline{\cG}_\varepsilon)x_-}_{W^{k, 1}}\\
        &\le \frac{1}{\lambda - \omega_k}(\norm{x_+}_{W^{k, 1}} + \norm{x_-}_{W^{k, 1}}) = \frac{1}{\lambda - \omega_k} \norm{x_+ - x_-}_{W^{k, 1}}\\
        &= \frac{1}{\lambda - \omega_k}\, \norm{x}_{W^{k, 1}}
    \end{align*}
    where we used \Cref{eq:resolvent-bound-G-positive-semidefinite} and the construction of $x_+$ and $x_-$, which concludes step 2.
    \medskip
    
    \noindent Proof of step 3. From step 1.~we get that there exists a $\lambda > \omega_k$ in the resolvent set of $\overline{\cG}_\varepsilon$ whereas step 2.~tells us that, for this $\lambda$, the resolvent is bounded by $\frac{1}{\lambda - \omega_k}$. We can use the same proof strategy as in \cite[Prop.~II.3.14 (ii)]{Engel.2000} where the authors employ the series expansion of the resolvent and its explicit bound to make conclusions about the resolvent set. Following their steps we first get that $(\omega_k, 2\lambda - \omega_k)$ is part of the resolvent set, and then using step 2.~again, we obtain the positivity preservation property as well as the explicit bound for all of those resolvents. This allows us to successively use these arguments and conclude that the resolvent set contains $(\omega_k, \infty)$.
\end{proof}

Putting together the results from \Cref{lem:semigroup-of-G-for-positivity} and \Cref{lem:semigroup-of-G-for-sobolev-stability}, we are now able to get rid of the perturbation $\cI_{d, \varepsilon}$.

\begin{lem}\label{lem:eliminating-perturbation-G}
    The closure of 
    \begin{equation*}
        \cG:\cT_f \to \cT_f, \quad x \mapsto \cG(x) = Gx + x G^\dagger\,,
    \end{equation*}
    where $G$ is defined in \Cref{eq:lindblad}, generates a strongly continuous, positivity preserving semigroup $(\cP_t)_{t\ge 0}$ on $W^{k, 1}$ for all $k \in \N$ with 
    \begin{equation*}
        \norm{\cP_t}_{W^{k, 1} \to W^{k, 1}} \le e^{\omega_k t} \, . 
    \end{equation*}
    where $\omega_k = \frac{k_{r_1} - k}{k_{r_1} - k_{r_0}}\omega_{k_{r_0}} + \frac{k - k_{r_0}}{k_{r_1} - k_{r_0}}\omega_{k_{r_1}}$ for an $r$ such that $k_{r_0}\leq k <k_{r_1}$. Finally, for $k = 0$, the semigroup is contractive.
\end{lem}
\begin{proof}
    The proof is a direct application of \Cref{lem:approximation-lemma} to the semigroups we obtained in \Cref{lem:semigroup-of-G-for-sobolev-stability} taking $\varepsilon \to 0$. Since the semigroups in \Cref{lem:semigroup-of-G-for-sobolev-stability} were positivity preserving, so is the obtained semigroup in the limit $\varepsilon\to 0$ (c.f.~\Cref{lem:approximation-lemma}).
\end{proof}

We are now ready to prove the main Theorem of the section.

\begin{proof}[Proof of \Cref{thm:generation-theorem}]
    The proof strategy is inspired by \cite[Thm.~2.5]{Davies.1977}. It however makes use of \Cref{lem:approximation-lemma} to avoid the issues discussed in \cite[§3]{Davies.1977}. Let $k \in \{k_r\}_{r \in \N}$ or $k = 0$ for the moment. Note that from \Cref{assum:finite-degree}, we can conclude $\omega_k = 0$ for $k = 0$ in \Cref{eq:Assumptionsobolevstability}. We first define for $\delta \in (0, 1)$ the map
    \begin{align*}
        \cL_\delta:\cT_f \to \cT_f, \quad x \mapsto \cL_\delta(x) &= G x + x G^\dagger + \delta\sum\limits_{j = 1}^K L_j x L_j^\dagger =: \cG(x) + \delta\Sigma(x) , 
    \end{align*}
    and show that its closure defines a strongly continuous, positivity preserving semigroup $(\cP_t^\delta)_{t \ge 0}$ on $W^{k, 1}$ which further satisfies
    \begin{equation*}
        \norm{\cP_t^\delta}_{W^{k, 1} \to W^{k, 1}} \le e^{\omega_k t} \, . 
    \end{equation*}
    We first note that for $\widetilde \lambda > 0$, a rearrangement of \Cref{eq:Assumptionsobolevstability} using cyclicity of the trace and that $\tr[\cdot] \le \norm{\cdot}_1$ with equality if the argument is positive semi-definite gives
    \begin{equation*}
        \norm{\Sigma(x)}_{W^{k, 1}} \le \norm{(\widetilde \lambda + \omega_k - \cG)(x)}_{W^{k, 1}} 
    \end{equation*}
    for $x \in \cT_f$ and $x\ge 0$. Now using that $\cG$ is closable (\Cref{lem:eliminating-perturbation-G}) and its resolvent positivity preserving we can conclude  for $\lambda \coloneqq \widetilde \lambda + \omega_k > \omega_k$, $x \in (\lambda - \cG)\cT_f$ and $x \ge 0$, 
    \begin{equation*}
        \norm{\Sigma \circ R(\lambda,\overline{\cG}) (x)}_{W^{k, 1}} \le \norm{x}_{W^{k, 1}}\,.
    \end{equation*}
    Applying similar methods as in step 2. of the proof of \Cref{lem:semigroup-of-G-for-sobolev-stability}, we can extend the above inequality to general $x \in (\lambda - \cG) \cT_f$. Hence, $\Sigma \circ R(\lambda,\overline{\cG})$ is contractive on the dense set $(\lambda - \cG)\cT_f$ and positivity preserving, since both $\Sigma$ and $R(\lambda,\overline{\cG})$ are. It can therefore be uniquely extended to a positivity preserving contractive map on all of $W^{k, 1}$ which we will call $\cA_\lambda$ in the following. As a consequence $(\cL_\delta, \cD(\cL_\delta))$ is closable and $\lambda > \omega_k$ in the resolvent set of the closure. Both facts follow from the representation
    \begin{equation*}
        (\lambda - \cL_\delta) = (\1 - \delta\Sigma \circ R(\lambda,\overline{\cG})) \circ (\lambda - \cG)
    \end{equation*}
    which decomposes $\lambda - \cL_\delta$ into a composition of a closable map with a dense range and a map that is bounded on that range. We further get for the resolvent of the closure
    \begin{equation*}
        R(\lambda,\overline{\cL}_\delta) = R(\lambda,\overline{\cG}) \sum\limits_{n = 0}^\infty \delta^n \cA_\lambda^n \,,  
    \end{equation*}
    which immediately lets us conclude that the resolvent is positivity preserving as $\cA_\lambda$ and $R(\lambda,\overline{\cG})$ are. Lastly, we will show that for $\lambda > \omega_k$
    \begin{equation}\label{eq:bound-resolvent-L_r}
        \norm{R(\lambda,\overline{\cL}_\delta)}_{W^{k, 1} \to W^{k, 1}} \le \frac{1}{\lambda - \omega_k} \, . 
    \end{equation}
    To obtain this inequality we again rearrange \Cref{assum:sobolev-stability}, add non-negative terms, use cyclicity of the trace and that $\tr[\cdot] \le \norm{\cdot}_1$ with equality if the argument is positive semi-definite, to conclude that for $x \in (\lambda - \cL_r)\cT_f$, $x$ positive semi-definite,
    \begin{equation*}
        \norm{R(\lambda,\overline{\cL}_\delta) x}_{W^{k, 1}} \le \frac{1}{\lambda - \omega_k}\norm{x}_{W^{k, 1}}\,.
    \end{equation*}
    We again extend the above bound to all $x \in (\lambda - \cL_\delta)\cT_f$ analogously to step 2 in the proof of \Cref{lem:eliminating-perturbation-G}. Using that $(\lambda - \cL_\delta)\cT_f$ is dense then gives \Cref{eq:bound-resolvent-L_r}. Employing \Cref{thm:lumer-phillips}, we get that indeed for all $\delta \in (0, 1)$ the closure of $(\cL_\delta, \cD(\cL_\delta))$ generates a strongly continuous semigroup which is positivity preserving since the resolvent is and satisfies the claimed bound. To now fill the gap between $0$ and the $\{k_r\}_{r \in \N}$ respectively, we interpolate between the semigroups (q.v.~\Cref{lem:interpolation-lemma}), obtaining $e^{t\omega_k}$ where $\omega_k = \frac{k_{r_1} - k}{k_{r_1} - k_{r_0}}\omega_{k_{r_0}} + \frac{k - k_{r_0}}{k_{r_1} - k_{r_0}}\omega_{k_{r_1}}$ for an $r$ such that $k_{r_0}\leq k <k_{r_1}$, as the bound of the interpolated semigroups. Now that we have the result for all $k \ge 0$ we can employ \Cref{lem:approximation-lemma} and take the limit $\delta \to 1$ to obtain the assertion. The contractivity and trace-preserving property of the semigroup in the case $k = 0$  just follows from the GKSL form of $(\cL, \cD(\cL))$, i.e.~$\tr[\cL(x)] = 0$ for $x \in \cT_f$, or put differently \Cref{assum:finite-degree}.
\end{proof}

\subsection{Bosonic evolution systems}\label{sec:timedependentgeneration}

Next, we consider time-dependent generators in GKSL form. For this, we modify Assumptions \ref{assum:finite-degree} and \ref{assum:sobolev-stability} in the following way: 

\begin{assum}\label{assum:finite-degree-time-dep}
    The operator $(\cL_s,\cT_f)$ has $\operatorname{GKSL}$ form, i.e.~for $x\in\cT_f$ and $s\in[0,\infty)$ 
    \begin{equation}\label{eq:lindblad-time-dep}
        \begin{aligned}
            \cL_s:\cT_f \to \cT_f \quad x \mapsto\cL_s(x) &= - i [H(s), x] + \sum\limits_{j = 1}^K L_j(s) x L_j^\dagger(s)  - \frac{1}{2}\{L_j^\dagger(s) L_j(s), x\} \\
            &\coloneqq G(s)x + x G^\dagger(s) + \sum\limits_{j = 1}^K L_j(s) x L_j^\dagger(s)\, , 
        \end{aligned}
    \end{equation}
    where $K\in\mathbb{N}$, $G(s)=-iH(s)-\frac{1}{2}\sum_{j=1}^K L_j^\dagger(s) L_j(s)$, and $H(s)\coloneqq p_{H(s)}(a,a^\dagger), L_j(s)\coloneqq p_{j,s}(a,a^\dagger)$ are polynomials of the creation and annihilation operators with time-dependent, differentiable coefficients. Again, $d_H\coloneqq\sup_{s\ge 0}\,\deg(p_{H(s)})<\infty$, $d_j\coloneqq\sup_{s\ge 0}\deg(p_{j,s})<\infty$, and $d\coloneqq\max\{d_1,...,d_K,d_H\}$.
\end{assum}

The next assumption will lead to the evolution system being Sobolev preserving, which allows us not only to prove the existence and uniqueness of the evolution generated by \eqref{eq:mastereq} but further to conduct a perturbation analysis as well as to extend our results to the case of a time-dependent Master equation:

\begin{assum}\label{assum:sobolev-stability-time-dep}
    There exists a non-negative sequence $\{k_r\}_{r \in \N}\rightarrow\infty$ s.t.~for all $r \in \N$ there exist $\omega_{k_r} \ge 0$ such that for all $s \in \R_+$ and $x \in \cT_f$ positive semi-definite,
    \begin{equation}
        \tr[\cL_s(x) (N + \1)^{k/2}] \le \omega_{k_r} \tr[x (N + \1)^{k/2}] \, .
    \end{equation}
    Note that the coefficients $\omega_{k_r}$ are independent of $s$.
\end{assum}

Under the above assumptions we can state the generation theorem for evolution systems as follows:

\begin{thm}[Generation of bosonic evolution systems]\label{thm:timedep-generation-theorem}
    Let $(\cL_s, \cD(\cL_s))_{s\in[ 0,\infty)}$ be a family of operators that fulfill \Cref{assum:finite-degree-time-dep} and \Cref{assum:sobolev-stability-time-dep}. Then $(\overline{\cL}_s, \cD(\overline{\cL}_s))_{s \in \R_+}$ gives rise to a unique evolution system $(\cP_{t,s})_{0\le s\le t}$ on $W^{k, 1}$ for all $k\ge 0$ with the following properties
    \begin{enumerate}
        \item $\cP_{t,s} (W^{k + 4d, 1}) \subseteq W^{k + 4d, 1}$ for all $0\le s\le t$
        \item For any $x \in W^{k + 4d, 1}$, the family $( \cP_{t,s}(x))_{0\le s\le t}$ is the unique solution to the initial value problem
        \begin{equation}\label{eq:time-dep-init-value-problem}
            \frac{d}{dt} x(t) = \overline{\cL}_t(x(t)) \qquad t \in [s, \infty), \; x(s) = x\,.
        \end{equation}
    \end{enumerate}
    For $k=0$, the evolution system is contractive and trace-preserving.
\end{thm}
\begin{proof}
    We assume w.l.o.g.~that $s \in [0,1]$ is fixed since the same argument works for all compact intervals. \Cref{thm:generation-theorem} shows that $(\overline{\cL}_s,\cD(\overline{\cL}_s))$ generates an $\omega_k$-quasi-contractive semigroup $(\cP_t^s)_{t \ge 0}$ on $W^{k, 1}$. Next, we realize that $W^{k + 4d, 1}$ are $\cL_s$-admissible subspaces, where we recall that $d$ denotes the degree of $\cL_s$. This already proves assumptions (1) and (2) in \Cref{thm:time-dependent-semigroups}. Since the coefficients of the polynomials $p_H$ and $p_j$ are continuous and operators of the form
    \begin{equation*}
        (N+\1)^{k/4} a^j(\ad)^l (N+\1)^{-(k/4+d)}\,,
    \end{equation*}
    for $j+l\leq d$, are bounded (see \Cref{lem:boundedness-polynomials}) w.r.t.~the operator norm, we have by Hölder inequality that
    \begin{equation*}
        s \mapsto (N + \1)^{k/4} \cL_s((N + \1)^{-k/4 + d} (\cdot) (N + \1)^{-k/4 + d})(N + \1)^{k/4} =: \cA(s)
    \end{equation*}
    is a bounded and uniformly continuous family of operators. Therefore,
    \begin{equation*}
        s\mapsto\cL_s\in\cB(W^{k,1},W^{k+4d,1})
    \end{equation*}
    is uniformly continuous, which proves condition (3) in \Cref{thm:time-dependent-semigroups}.
    Hence \Cref{thm:time-dependent-semigroups} provides the existence and uniqueness of an evolution system on $W^{k, 1}$. By repeating the above arguments on $\cY\coloneqq W^{k+4d}$, i.e.~by choosing our $\cL_s$-admissible subspace as $W^{k + 8d, 1}$, \Cref{thm:time-dependent-semigroups} provides existence and uniqueness of a solution on $\cY = W^{k + 4d, 1}$ which agrees with the former one on $W^{k, 1}$ by the compact embedding of $W^{k + 4d, 1}$ into $W^{k, 1}$. Therefore, conditions (4) and (5) are satisfied for the evolution system on $W^{k, 1}$ and the admissible subspace $\cY=W^{k + 4d, 1}$, which through \Cref{thm:time-dependent-semigroups} proves the claim. Moreover, the evolution system is positivity preserving because it can be constructed by a concatenation of time-independent positivity preserving semigroups (see \Cref{thm:generation-theorem} and \cites[Eq.~5.3.5]{Pazy.1983}). Contractivity and the property of trace preservation are a consequence of the fact that $\omega_0$ can be chosen to be $0$.
\end{proof}

\section{Multi-mode extension}\label{sec:multi-mode-extension}
This section discusses the extension of \Cref{sec:polynomial-generators} to the multi-mode setting. Since the details are almost completely analogous to the single-mode situation, we choose to elaborate only at places where some ambiguities might remain. Let us first fix the notations for this setting. We consider the Hilbert space of an $m$-mode system, $m\in\N$, whose Hilbert space we conveniently denote by $\cH_m=L^2(\mathbb{R}^m)$. We further use $\cB(\cH_m)$ for the bounded, $\cT_1$ for the trace class, and $\cT_{1, \operatorname{sa}}$ for the self-adjoint trace class operators. Now we define $\cT_f$ to be
\begin{equation*}
    \cT_f \coloneqq \{x = \sum\limits_{\text{finite}} f_{\n, \p}  \ketbra{n_{1}}{p_{1}} \otimes \hdots \otimes \ketbra{n_m}{p_m} \;:\; f_{\n, \p} \in \C, \; x = x^\dagger\} \, ,
\end{equation*}
where $\n = (n_1, \hdots, n_m) \in \N^m$ and $\p$ analogously function as an index in $f_{\n, \p}$. For $\k = (k_1, \hdots, k_m) \in \R_+^m$, we define $\k \prec \k'$ if $k_j < k_j'$ for all $j = 1, \hdots, m$. Analogously, we define $\preceq$. We set for $\k \in \R_+^m$
\begin{equation*}
    (N + \1)^\k \coloneqq (N_1 + \1)^{k_1} \otimes \hdots \otimes (N_m + \1)^{k_m},
\end{equation*} 
and with this define $W^{\k, 1}$, $\norm{\,\cdot\,}_{W^{\k, 1}}$. Remark that the latter spaces are Banach spaces and in correspondence to \Cref{lem:sobolev-embedding} we find:
\begin{lem}\label{lem:multimode-sobolev-embedding}
    Let $\k, \k' \in \N^m$ with $\k \prec \k'$, then 
    \begin{equation}
        W^{\k', 1} \Subset W^{\k, 1} \, . 
    \end{equation}
\end{lem}
The strategy to prove the above claims is analogous to the single-mode case. Next, we slightly generalize the single-mode results \Cref{thm:stein-weiss} and \Cref{defi:sobolev-preserving-semigroups} to the multimode setting. 

\begin{thm}[Stein-Weiss theorem for multimode Bosonic Sobolev spaces]
    Let $\k_0, \k_1 \in \R_+^m$, $\k_0 \prec \k_1$ and $T: W^{\k_j, 1} \to W^{\k_j, 1}$ a linear map with $\norm{T}_{W^{\k_j, 1} \to W^{\k_j, 1}} \le M_j$, bounded by $M_j \ge 0$ for $j = 0, 1$ respectively. Then for $\theta \in [0, 1]$, $T: W^{\k_\theta, 1} \to W^{\k_\theta, 1}$ with $\k_\theta = (1 - \theta) \k_0 + \theta \k_1$ obtained by restriction of the input of  $T:W^{\k_0, 1} \to W^{\k_0, 1}$ to $W^{\k_\theta, 1} \cap W^{\k_0, 1}$, is a well defined bounded linear map with
    \begin{equation}
        \norm{T}_{W^{\k_\theta, 1} \to W^{\k_\theta, 1}} \le M_0^{(1 - \theta)} M_1^\theta \, . 
    \end{equation}
\end{thm}

In the multimode setting, we cannot interpolate between the elements of the divergent sequence to obtain $W^{\k, 1}$ as admissible subspace for all $0 \prec \k$ but only for elements in the convex hull of the divergent sequence. The property of being Sobolev preserving is again defined for a sequence $\{\k_r\}_{r \in \N}$ such that $\lim\limits_{r \to \infty} \min\limits_{j = 1, \hdots, m} k_{j, r} = \infty$.

\begin{defi}[Sobolev preserving semigroup/evolution system in multimode Sobolev spaces]
    Let $(\cP_t)_{t\ge 0}$ be a $C_0$-semigroup on $\cT_{1,\operatorname{sa}}$. We then call $(\cP_t)_{t \ge 0}$ \textit{Sobolev preserving} if there exists a divergent sequence $\{\k_r\}_{r \in \N} \subset \R_+^m$, in the sense that $\lim\limits_{r \to \infty} \min\limits_{j = 1, \hdots, m} k_{j, r} = \infty$, s.t. for all $r \in \N$, $W^{\k_r, 1}$ is an admissible subspace for $(\cP_t)_{t\ge 0}$. Similarly for an evolution system $(\cP_{t,s})_{0\leq s\leq t}$ on $\cT_{1, \operatorname{sa}}$, we call it $(\cP_{t,s})_{0\leq s\leq t}$ \textit{Sobolev preserving} if for all $r \in \N$, $W^{\k_r, 1}$ is admissible for $(\cP_{t,s})_{0\leq s\leq t}$ to $W^{\k_r, 1}$.
\end{defi}

With these preliminaries in place we can now lift \Cref{assum:finite-degree}, \Cref{assum:sobolev-stability}, \Cref{assum:finite-degree-time-dep}, and \Cref{assum:sobolev-stability-time-dep}.

\begin{assum}\label{assum:multimode-finite-degree}
    The operator $(\cL, \cT_f)$ has $\operatorname{GKSL}$ form, i.e. for $x \in \cT_f$,
    \begin{equation}
        \begin{aligned}
            \cL:\cT_f \to \cT_f \quad x \mapsto\cL(x) &= - i [H, x] + \sum\limits_{j = 1}^K L_j x L_j^\dagger  - \frac{1}{2}\{L_j^\dagger L_j, x\} \\
            &\coloneqq Gx + x G^\dagger + \sum\limits_{j = 1}^K L_j x L_j^\dagger\, , 
        \end{aligned}
    \end{equation}
    for some $K \in \N$ and with $G = - iH - \frac{1}{2}\sum\limits_{j = 1}^K L_j^\dagger L_j$, where $\{A, B\} = AB + BA$ denotes the anti-commutator of two operators $A, B$ on a suitable domain. Further $H$ and $L_j$ are assumed to be polynomials of the creation and annihilation operators, i.e. $H \coloneqq p_H(a_1, a_1^\dagger, \hdots, a_m, a_m^\dagger)$, $L_j \coloneqq p_j(a_1, a_1^\dagger, \hdots, a_m, a_m^\dagger)$ and $H$ symmetric. This ensures that $\cT_f$ is invariant under $\cL$. We denote the degree of $p_H$ by $d_H \coloneqq \deg p_H$, those of $p_j$ by $d_j \coloneqq \deg p_j$, and $d \coloneqq \max\{d_1, \hdots, d_K, d_H\}$.
\end{assum}

The second assumption becomes:

\begin{assum}\label{assum:multimode-sobolev-preserving}
     There exists a non-negative sequence $\{\k_r\}_{r \in \N} \subset \R_+^m$, in the sense that \\ $\lim\limits_{r \to \infty} \min\limits_{j = 1, \hdots, m} k_{j, r} = \infty$, s.t.~for every $r \in \N$, there exist $\omega_{\k_r} \ge 0$ such that for all positive semi-definite $x \in \cT_f$
    \begin{equation}\label{eq:Multimodeassumptionsobolevstability}
        \tr[\cL(x) (N + \1)^{\k_r/2}] \le \omega_{\k_r} \tr[x (N + \1)^{\k_r/2}] \, .
    \end{equation}
\end{assum}

Then employing the single-mode strategy, we obtain the following theorem.

\begin{thm}[Generation of multimode bosonic semigroups]\label{thm:multimode-generation-theorem}
    Let $(\cL, \cD(\cL))$ be an operator defined on $\cT_{1,\operatorname{sa}}$. If $(\cL, \cD(\cL))$ satisfies \Cref{assum:multimode-finite-degree} and \Cref{assum:multimode-sobolev-preserving}, then the closure $\overline{\cL}$ generates a strongly continuous, positivity preserving semigroup $(\cP_t)_{t\ge 0}$ on $W^{\k_r, 1}$ for all $\{\k_r\}_{r \in \N}$ from \Cref{assum:multimode-sobolev-preserving}. We further find that the semigroup satisfies the bound
    \begin{equation}
        \norm{\cP_t}_{W^{\k_r, 1} \to W^{\k_r, 1}} \le e^{\omega_{\k_r} t} \,\quad \forall t\ge 0\, .
    \end{equation}
   In the special case $\k = 0$, the semigroup is contractive and trace-preserving.
\end{thm}

Note that we can extend the above semigroups to the convex hull of $\{\k_r\}_{r \in \N} \cup \{0\}$ using a generalisation of the interpolation lemma for single mode semigroups (q.v. \cref{lem:interpolation-lemma}). 

Lastly, we can also generalize the generation theorem for evolution systems modifying the assumptions accordingly.

\begin{assum}\label{assum:multimode-finite-degree-time-dep}
    The operator $(\cL_s,\cT_f)$ has $\operatorname{GKSL}$ form, i.e.~for $x\in\cT_f$ and $s\in[0,\infty)$ 
    \begin{equation}\label{eq:multimode-lindblad-time-dep}
        \begin{aligned}
            \cL_s:\cT_f \to \cT_f \quad x \mapsto\cL_s(x) &= - i [H(s), x] + \sum\limits_{j = 1}^K L_j(s) x L_j^\dagger(s)  - \frac{1}{2}\{L_j^\dagger(s) L_j(s), x\} \\
            &\coloneqq G(s)x + x G^\dagger(s) + \sum\limits_{j = 1}^K L_j(s) x L_j^\dagger(s)\, , 
        \end{aligned}
    \end{equation}
    where $K\in\mathbb{N}$, $G(s)=-iH(s)-\frac{1}{2}\sum_{j=1}^K L_j^\dagger(s) L_j(s)$, and $H(s)\coloneqq p_{H(s)}(a_1,a^\dagger_1, \hdots, a_m, a^\dagger_m), L_j(s)\coloneqq p_{j,s}(a_1,a^\dagger_1, \hdots, a_m, a^\dagger_m)$ are polynomials of the creation and annihilation operators with time-dependent, differentiable coefficients. Again, $d_H\coloneqq\sup_{s\ge 0}\,\deg(p_{H(s)}) < \infty$, $d_j\coloneqq\sup_{s\ge 0}\deg(p_{j,s})<\infty$, and $d\coloneqq\max\{d_1,...,d_K,d_H\}$.
\end{assum}

The second assumption in the time-dependent case generalizes to the following:

\begin{assum}\label{assum:multimode-sobolev-stability-time-dep}
  There exists a divergent sequence $\{\k_r\}_{r \in \N} \subset \R_+^m$, meaning $\lim\limits_{r \to \infty} \min\limits_{j = 1, \hdots, m} k_{j, r} = \infty$, s.t. for every $r \in \N$, there exist $\omega_{\k_r} \ge 0$ such that for all $s \in \R_+$ and $x \in \cT_f$ positive semi-definite,
    \begin{equation}
        \tr[\cL_s(x) (N + \1)^{\k_r/2}] \le \omega_{\k_r} \tr[x (N + \1)^{\k_r/2}] \, .
    \end{equation}
    Note that the coefficients $\omega_{\k_r}$ are independent of $s$.
\end{assum}

Under the above assumptions we can state the generation theorem for multimode evolution systems as follows:

\begin{thm}[Generation of multimode bosonic evolution systems]\label{thm:multimode-timedep-generation-theorem}
    Let $(\cL_s, \cD(\cL_s))_{s\in[ 0,\infty)}$ be a family of operators that fulfills \Cref{assum:multimode-finite-degree-time-dep} and \Cref{assum:multimode-sobolev-stability-time-dep}. Then 
    $(\overline{\cL}_s, \cD(\overline{\cL}_s))_{s \in \R_+}$ gives rise to a unique evolution system $(\cP_{t,s})_{0\le s\le t}$ on $W^{\k_r, 1}$ for all $r \in \N$ with the following properties: for $\k_{r'}$ with $\min\limits_{j = 1, \hdots, m}|k_{j, r} - k_{j, r'}| \ge d$
    \begin{enumerate}
        \item $\cP_{t,s} (W^{\k_{r'}, 1}) \subseteq W^{\k_{r'}, 1}$ for all $0\le s\le t$;
        \item For any $x \in W^{\k_{r'}, 1}$, the family $(\cP_{t,s}(x))_{0\le s\le t}$ is the unique solution to the initial value problem
        \begin{equation}\label{eq:multimode-time-dep-init-value-problem}
            \frac{d}{dt} x(t) = \overline{\cL}_t(x(t)) \qquad t \in [s, \infty), \; x(s) = x\,.
        \end{equation}
    \end{enumerate}
    For $\k = 0$ as a special case, we get that the evolution system is contractive and trace-preserving.
\end{thm}

\section{Examples of Sobolev preserving semigroups}\label{sec:examples-sobolev-preserving-semigroup}
In this section, we consider two classes of examples of practical relevance in quantum information processing for which \Cref{assum:finite-degree} (or \ref{assum:finite-degree-time-dep},\ref{assum:multimode-finite-degree},\ref{assum:multimode-finite-degree-time-dep}) trivially holds and derive \Cref{assum:sobolev-stability} (or \ref{assum:sobolev-stability-time-dep}, \ref{assum:multimode-sobolev-preserving}, \ref{assum:multimode-sobolev-stability-time-dep}). Particular care will be given to finding time-independent upper bounds on the $W^{k,1}\to W^{k,1}$ norm of the semigroup. For this, the overall strategy is as follows: given the generator $(\cL,\cT_f)$, we prove that there are coefficients $\mu_{k_r}\ge 0, c_{k_r}>0$ for a divergent sequence $\{k_r\}_{r \in \N}$ such that for all state $\rho\in\cT_f$
\begin{equation}\label{eq:examples-assum2-step1}
    \begin{aligned}
        \tr[\cL(\rho) (\Nind+\1)^{k_r/2}] &\leq - c_{k_r}\tr[\rho (\Nind+\1)^{k_r/2}] + \mu_{k_r} \\
        & \leq (\mu_{k_r}- c_{k_r})\tr[\rho (\Nind+\1)^{k_r/2}]\,,
    \end{aligned}
\end{equation}
where we have used $\tr[\rho (\Nind+\1)^{k_r/2}]\geq \tr[\rho] = 1$ in the second inequality.
Then, \Cref{thm:generation-theorem} can be applied, which shows that for all $k \in \R_+$, the closure of $(\cL,\cT_f)$ generates a positivity preserving $C_0$-semigroup $(\cP_t)_{t\ge 0}$ on $W^{k,1}$. In the case $k \in \{k_r\}_{r \in \N}$:
\begin{equation*}
    \|\cP_t(x)\|_{W^{k,1}}\leq e^{|\mu_k- c_k|\,t}\|x\|_{W^{k,1}}\,.
\end{equation*}
for all $x\in W^{k,1}$. The bounds for the intermediate values of $k$ can be obtained using \Cref{lem:interpolation-lemma}. One can strengthen the above bounds using \Cref{eq:examples-assum2-step1} as follows:

\begin{prop}\label{prop-ex:uniformly-bounded-semigroup}
    Let $(\cL,\cT_f)$ be an operator satisfying \Cref{assum:finite-degree} and \Cref{eq:examples-assum2-step1}. Then, for all $k\in\N$, the closure of $(\cL,\cT_f)$ generates a positivity preserving $C_0$-semigroup $(\cP_t)_{t\ge 0}$ on $W^{k,1}$. For all $r \in \N$ and all states $\rho\in W^{k,1}$,
    \begin{equation*}
        \|\cP_t(\rho)\|_{W^{k_r,1}}\leq \max\left\{\|\rho\|_{W^{k_r,1}},\,\frac{\mu_{k_r}}{c_{k_r}}\right\} \, .
    \end{equation*}
    For a general $k \in \R_+$ and $x \in W^{k, 1}$ one obtains
    \begin{equation}\label{eq:improved-semigroup-bound}
         \|\cP_t(x)\|_{W^{k_r,1}}\leq \gamma_k \|x\|_{W^{k_r,1}} \, ,
    \end{equation}
    where $\gamma_k = \max\{1, \frac{\mu_k}{c_k}\}$ for $k \in \{k_r\}_{r \in \N}$ and an interpolated time-independent constant in all other cases. Note that for $k > 0$ and $\rho \in W^{k, 1}$ there exists a sequence $\{t_n\}_{n \in \N}$, such that
    \begin{equation*}
        \lim_{t_n\rightarrow\infty}\cP_{t_n}(\rho)=\overline{\rho}
    \end{equation*}
    for $\overline{\rho}\in W^{k,1}$. Similar conclusions hold in multi-mode as well as time-dependent settings.
\end{prop}
\begin{proof}
    By assumption, \Cref{thm:generation-theorem} shows that the closure of $(\cL,\cT_f)$ defines a positivity preserving, quasi-contractive semigroup $(\cP_t)_{t\ge 0}$. Moreover, for $k \in \{k_r\}_{r \in \N}$, $\rho(t)\coloneqq \cP_t(\rho)$
    \begin{equation*}
        \begin{aligned}
            \frac{d}{dt}\|\rho(t)\|_{W^{k,1}}&=\tr[\cL(\rho(t))(N+\1)^{k/2}]\\
            &\leq-c_{k}\tr[\rho(t) (\Nind+\1)^{k/2}] + \mu_{k}\\
            &=-c_{k}\|\rho\|_{W^{k,1}} + \mu_{k}\,.
        \end{aligned}
    \end{equation*}
    Thus, for $\|\rho(t)\|_{W^{k,1}}\geq\frac{\mu_k}{c_k}$, we have $\frac{d}{dt}\|\rho(t)\|_{W^{k,1}}\leq0$, which concludes the bound. Using the positivity preserving property of the semigroup and that $\norm{\cdot}_1 \le \norm{\cdot}_{W^{k, 1}}$ one can lift the bound to \Cref{eq:improved-semigroup-bound} for general $x \in W^{k, 1}$ and \Cref{thm:stein-weiss} allows us to conclude 
    \begin{equation*}
         \|\cP_t(x)\|_{W^{k_r,1}}\leq \gamma_k \|x\|_{W^{k_r,1}}
    \end{equation*}
    extend to all $k \in \R_+$. Finally, for every $k > 0$, every sequence $n\rightarrow \cP_{t_n}(\rho)$ is uniformly bounded in $W^{k,1}$ so that the compact embedding shows that there exists a converging subsequence in $W^{k-\varepsilon,1}$ for $\varepsilon$ suitably chosen, which is also converging in $W^{k,1}$. This finishes the proof.
\end{proof}
To achieve the inequality stated in \Cref{eq:examples-assum2-step1}, we will make heavy use of the following simple commutation relations: given a real-valued function $f:\mathbb{N}\to\mathbb{R}$,  
\begin{equation}\label{eq:symmetry-function}
    \begin{aligned}
        af(\Nind+j\1)=f(\Nind + (j+1)\1)a,&\quad\quad a^\dagger\,1_{>j}f(\Nind-j\1)=f(\Nind - (j+1)\1)a^\dagger\,1_{>j}\,,\\
        f(\Nind-j\1)a\,1_{>j}=af(\Nind - (j+1)\1)1_{>j},&\quad\quad f(\Nind+j\1)a^\dagger=a^\dagger f(\Nind + (j+1)\1)\,,
    \end{aligned}
\end{equation}
where the operators above are defined e.g.~on $\cH_f$. We also use the canonical commutation relation to write $(\ad)^la^l$ as a function of $N$ (see \Cref{lem:l-ccr}): 
    \begin{align*}
            &(\ad)^la^l=(N-(l-1)\1)(N-(l-2)\1)\cdots(N-\1)N\\
            & a^l(\ad)^l=(N+\1)(N+2\1)\cdots(N+(l-1)\1)(N+l\1)\,.
    \end{align*}
In the following, we adopt the notations:
\begin{align*}
    \cL[L]\coloneqq L(\cdot)L^\dagger -\frac{1}{2}\,\{L^\dagger L,\,\cdot\}\,\qquad \text{ and }\qquad \cH[H]\coloneqq -i[H,\cdot]\,.
\end{align*}
Although this notation collides with the one for the Hilbert space, the meaning can always be deduced from context.

\subsection{Quantum Ornstein Uhlenbeck semigroup}

We start with the generator of the quantum Ornstein Uhlenbeck semigroup \cite{Cipriani.2000,Carbone.2007} defined by
\begin{equation}
    \cL_{\operatorname{qOU}} = \lambda^2 \cL[a] + \mu^2 \cL[\ad]
\end{equation}
 for $\mu, \lambda \ge0$. Given an suitably domain $\cD(\cL_{\operatorname{qOU}})$, the operator $(\overline{\cL}_{\operatorname{qOU}},\cD(\cL_{\operatorname{qOU}}))$
is known to generate a quantum dynamical semigroup $(\cP_t^{\operatorname{qOU}})_{t\ge 0}$. Here, we further show that the quantum Ornstein Uhlenbeck semigroup defines a semigroup on all $W^{k,1}$. This is the topic of the following lemma: 

\begin{lem}\label{lem-ex:qOU-differential-stability}
    Let $(\cL_{\operatorname{qOU}},\cT_f)$ be the generator of the quantum Ornstein Uhlenbeck semigroup and $k \in \N$. Then, there exist constants $\mu_k$ explicated in \eqref{eq-ex:qou-lambda>mu} such that, for all states $\rho\in\cT_f$,
    \begin{equation*}
        \begin{aligned}
            \tr[\cL_{\operatorname{qOU}}(\rho)(N+\1)^{\frac{k}{2}}]\le\begin{cases}
                \frac{k}{4}(\mu^2-\lambda^2)\tr\big[\rho(N+\1)^{k/2}\big]+\mu_k&\lambda>\mu\\
                \frac{k}{2}(2\mu^2+k)\tr\big[\rho(N+\1)^{k/2}\big]&\lambda\leq\mu
            \end{cases}
        \end{aligned}\,.
    \end{equation*}
   Therefore, the semigroup $e^{t\mathcal{L}_{\operatorname{qOU}}}$ is a Sobolev and positivity preserving quantum Markov semigroup satisfying for all states $\rho\in W^{k,1}$
    \begin{equation*}
        \begin{aligned}
            \|e^{t\mathcal{L}_{\operatorname{qOU}}}(\rho)\|_{W^{k,1}}\leq\begin{cases}
                \max\left\{\|\rho\|_{W^{k,1}},\,\frac{4\mu_{k}}{k(\mu^2-\lambda^2)}\right\}&\lambda>\mu\\
                e^{t\frac{k}{2}(2\mu^2+k)}\|\rho\|_{W^{k,1}}&\lambda\leq\mu
            \end{cases}\,.
        \end{aligned}
    \end{equation*}
\end{lem}
\begin{proof}
    We consider $\cL_{\operatorname{qOU}}^\dagger(f(N))$ where $f(x)=(x+1)^{k/2} 1_{x\ge -1}$. By \Cref{eq:symmetry-function},
    \begin{equation*}
        \begin{aligned}
            \cL_{\operatorname{qOU}}^\dagger(f(N))&=\lambda^2N(f(N-\1)-f(N))+\mu^2(N+\1)(f(N+\1)-f(N))\,.
        \end{aligned}
    \end{equation*}
    Note that the case $k=0$ follows from the GKLS form and $k=2$ is by definition of $f$ trivially given by $(\mu^2-\lambda^2)N+\1$. 
    Next, we define an auxiliary function which will also prove useful in the following proofs: 
    \begin{equation}\label{eq:f-g-l-function}
        g_l(x) = \begin{cases}
            f(x) - f(x - l) & x \ge l;\\
            f(x) & l > x \ge 0;\\
            0 & 0 > x\,.
        \end{cases}
    \end{equation}
    It allows us to redefine $\cL_{\operatorname{qOU}}^\dagger(f(N))$ by 
    \begin{equation*}
        \begin{aligned}
            \cL_{\operatorname{qOU}}^\dagger(f(N))&=-\lambda^2Ng_1(N)+\mu^2(N+\1)g_1(N+\1)\,.
        \end{aligned}
    \end{equation*}
    Then, applying \Cref{lem:upper-lower-bound-gl} to the spectral decomposition of the polynomial in the number operator above, we get 
    \begin{equation*}
        \begin{aligned}
            &\cL_{\operatorname{qOU}}^\dagger(f(N))\\
            &\quad\leq \frac{k}{2}(\mu^2-\lambda^2)(N+\1)^{k/2}+\lambda^2\frac{k}{2}(N+\1)^{k/2-1}+1_{k\geq3}(N+\1)^{k/2-2}\frac{k^2}{8}+\mu^2\frac{2-k}{2}\ketbra{0}{0}\\
            &\quad\leq \frac{k}{2}(\mu^2-\lambda^2)(N+\1)^{k/2}+\frac{k}{2}\left(\lambda^2+\mu^2+{k}\right)(N+\1)^{k/2-1}
        \end{aligned}
    \end{equation*}
    where we separated the vacuum state from the rest of the decomposition. Note that this bound can also be used when $k=1$ since $(N+\1)^{-1/2}$ is then bounded by $1$. Therefore, we assume $k\geq3$ in the following and start with the case $\lambda>\mu$ so that the leading order is negative. Then, we use half of the latter to bound the other terms by a constant. This is done by the following classical optimization
    \begin{equation*}
        \sup_{x\geq0}\left(-x^\nu+cx^{\nu-1}\right)=c^\nu\left(\frac{(\nu-1)^{\nu-1}}{\nu^\nu}\right)
    \end{equation*}
    for $\nu\geq1$ and $c\geq0$ defined as 
    \begin{equation*}
        c=2\frac{\lambda^2+\mu^2+k}{\lambda^2-\mu^2}\qquad\text{and}\qquad\nu=\frac{k}{2}\,.
    \end{equation*}
    Then,
    \begin{equation}\label{eq-ex:qou-lambda>mu}
        \begin{aligned}
            \cL_{\operatorname{qOU}}^\dagger(f(N))&\leq \frac{k}{4}(\mu^2-\lambda^2)(N+\1)^{k/2}+c^\nu\left(\frac{(\nu-1)^{\nu-1}}{\nu^\nu}\right)=:\frac{k}{4}(\mu^2-\lambda^2)(N+\1)^{k/2}+\mu_k^{\lambda>\mu}
        \end{aligned}
    \end{equation}
    The second case is $\lambda\leq\mu$, which can be easily upper bounded by
    \begin{equation*}
        \begin{aligned}
            &\cL_{\operatorname{qOU}}^\dagger(f(N))\leq \frac{k}{2}(2\mu^2+k)(N+\1)^{k/2}
        \end{aligned}\,.
    \end{equation*}
    This completes the proof of the statement by \Cref{thm:generation-theorem} and \Cref{prop-ex:uniformly-bounded-semigroup}.
\end{proof}

\subsection{Photon-dissipation and CAT qubits}\label{sec:cat-qubits}

Next, we consider a family of Lindbladians that has been recently studied in the setting of error correction with continuous variable quantum systems. For an introduction to the field, we refer the interested reader to the following lecture notes \cites{Preskill.2021}{Guillaud.2023}. The abstract idea here is that the code-space is continuously protected by a dissipative evolution, i.e.~an evolution which is exponentially converging for $t\rightarrow\infty$ to an invariant subspace --- the code-space. This behavior is achieved through the so-called $l$-photon dissipation generated for $\kappa>0$ and $\alpha\in\C$ by
\begin{equation}\label{eq:l-photon-dissipation}
    \kappa\cL[a^l-\alpha^l]\,,
\end{equation}
where we sometimes omit the identity so that $\alpha^l\coloneqq \alpha^l\1$ in what follows. The invariant subspace (code-space) to which the evolution is exponentially converging \cite{Azouit.2016} is defined by 
\begin{equation*}
    \cC_l\coloneqq\spa\left\{\ketbra{\alpha_1}{\alpha_2}\,:\,\alpha_1,\alpha_2\in\left\{\alpha e^{\frac{i2\pi j}{l}}\,|\,j\in\{0,...,l-1\}\right\}\right\}\,,
\end{equation*}
where $\ket{\alpha}$ denotes the coherent state
\begin{equation*}
    \ket{\alpha}=e^{-\frac{|\alpha|^2}{2}}\sum_{n=0}^{\infty}\frac{\alpha^n}{\sqrt{n!}}\ket{n}\,.
\end{equation*}
and satisfies $a\ket{\alpha}=\alpha\ket{\alpha}$ by definition.

Besides the $l$-photon dissipation, we consider the CAT qubit error correction protocol introduced in \cite{Guillaud.2019} associated to the $2$-photon dissipation and code-space $\cC_2$ and with corresponding universal gate-set generated by the following: for some parameters $T,\kappa,\varepsilon>0$,

\medskip
\medskip 

\textit{Identity-gate:}
\begin{equation}\label{eq:cat-identity}
    \kappa\cL[a^2-\alpha^2]
\end{equation}

\textit{$Z(\theta)$-gate:}
\begin{equation}\label{eq:cat-z}
    \kappa\cL[a^2-\alpha^2]+\varepsilon\cH[a+\ad]
\end{equation}

\textit{$X$-gate:}
\begin{equation}\label{eq:cat-X}
    \kappa\cL[a^2-e^{2i\pi t/T}\alpha^2]
\end{equation}

\textit{$\operatorname{CNOT}$-gate:}
\begin{equation}\label{eq:cat-cnot}
    \kappa\cL[a^2-\alpha^2]+\kappa\cL[b^2-\alpha^2-\frac{\alpha}{2}(1-e^{2i\pi t/T})(a-\alpha)]
\end{equation}

\textit{Toffoli-gate:}
\begin{equation}\label{eq:cat-toffoli}
    \kappa\cL[a^2-\alpha^2]+\kappa\cL[b^2-\alpha^2]+\kappa\cL[c^2-\alpha^2+\frac{1}{4}(1-e^{2i\pi t/T})(ab-\alpha(a+b)+\alpha^2)]\,.
\end{equation}

Note that the CNOT gate acts on two modes and the Toffoli on three modes, where the annihilation and creation operators on the second mode are denoted by $b$, resp.~$b^\dagger$ and on the third by $c$ and $c^\dagger$. In the following, we prove that the above operators generate Sobolev-preserving quantum dynamical semigroups, with the exception of the Toffoli gate. Due to its more complicated structure, we leave the analysis of the latter to future work.

We start by proving that the $l$-photon dissipation satisfies \Cref{eq:examples-assum2-step1}, and therefore that it generates a Sobolev preserving semigroup by \Cref{prop-ex:uniformly-bounded-semigroup}.

\begin{lem}[$l$-photon dissipation]\label{lem:l-diss}
    For any $k\ge 1$, $l\ge 2$, $\alpha \in\mathbb{C}$ and any state $\rho\in\cT_f$, 
    \begin{align*}
        \tr\big[\cL[a^l-\alpha^l ](\rho)(N+\1)^{k/2}\big]&\le -\frac{l}{2} \tr\big[\rho\,(N+\1)^{\nu}\big]+\frac{l}{2}\mu_k^{(l)}\le -\frac{l}{2} \tr\big[\rho\,(N+\1)^{k/2}\big]+\frac{l}{2}\mu_k^{(l)}\,,
    \end{align*}
    where $\mu_k^{(l)}=\Delta_l^\nu\left(\frac{(\nu-1)^{\nu-1}}{\nu^\nu}\right)$ with $\nu=l+\frac{k}{2}-1$ and $\Delta_l=(l+1)l+2|\alpha|^lkl^{k/2 - 1}\sqrt{l!}$. Therefore, $\cL_l\coloneqq \cL[a^l-\alpha^l]$ generates a Sobolev and positivity preserving quantum Markov semigroup satisfying for all states $\rho\in W^{k,1}$
    \begin{equation*}
        \|e^{t\cL_l}(\rho)\|_{W^{k,1}}\leq\max\Big\{\|\rho\|_{W^{k,1}}, \mu_k^{(l)}\Big\}\,.
    \end{equation*}
\end{lem}
\begin{proof}
    By \Cref{eq:symmetry-function}, we have for $f(x)=(x+1)^{k/2} 1_{x\ge -1}$:
    \begin{equation*}
        \begin{aligned}
            \cL[a_1^l-\alpha^l]^\dagger(f(N))&=(\ad)^lf(N)a^l-\frac{1}{2}\Big((\ad)^la^lf(N)+f(N)(\ad)^la^l\Big)\\
            &\quad+\frac{1}{2}(\overline{\alpha}^la^lf(N)-\overline{\alpha}^lf(N)a^l+\alpha^l f(N)(\ad)^l-\alpha^l(\ad)^lf(N))\\
            &=(\ad)^la^l\Big(f(N-l\1 )-f(N)\Big)\\
            &\quad+\frac{1}{2}\left[\overline{\alpha}^la^l\Big(f(N)-f(N-l\1)\Big)+\alpha^l\Big(f(N)-f(N-l\1)\Big)(\ad)^l\right]\,.
        \end{aligned}
    \end{equation*}
    In what follows, we use the function defined in \Cref{eq-appx:f-g-l-function}
    \begin{equation*}
        g_l(x) = \begin{cases}
            f(x) - f(x - l) & x \ge l;\\
            f(x) & l > x \ge 0;\\
            0 & 0 > x\,.
        \end{cases}
    \end{equation*}  
    Using the canonical commutation relation to write $(a^\dagger)^la^l$ as a function of $N$ (cf.~\Cref{lem:l-ccr}) and with help of the notation
    \begin{equation}\label{eq:notation-product}
        N_k[r:j]\coloneqq (N_k+r\1)\cdots (N_k+j\1)
    \end{equation}
    with the convention $N_k[r:j]=\1$ whenever $r>j$, we thus have that
    \begin{align*}
        \tr\big[\rho\,\cL[a^l-\alpha^l]^\dagger(f(N))\big]&=-\tr\big[\rho\,N[-l+1:0]g_l(N)\big]+\frac{1}{2}\tr\big[\rho\,(\overline{\alpha}^la^lg_l(N)+\alpha^lg_l(N)(\ad)^l)\big]\,,
    \end{align*}
    Since $g_l$ is positive and increasing, the last term above can be upper bounded by \Cref{lem:two-point-hamiltonian-bound},
    \begin{equation*}
        \begin{aligned}
            \frac{1}{2}\tr\big[\rho(\overline{\alpha}^la^lg_l(N)+{\alpha}^lg_l(N)(\ad)^l)\big]&{\leq}\,{|\alpha|^l}\tr[\rho\, g_l(N+l\1)\sqrt{N[1:l]}] \\
            &\overset{(1)}{\leq}|\alpha|^l \frac{kl^{k/2}}{2}\,\tr[\rho\, (N+\1)^{k/2-1}\sqrt{N[1:l]}]\\
            &\leq|\alpha|^lkl^{k/2}\sqrt{l!}\,\tr[\rho\, (N+\1 )^{k/2-1+\frac{l}{2}}]\,.
        \end{aligned}
    \end{equation*}
    In $(1)$ above, we used \Cref{lem:upper-lower-bound-gl} for the bound
    \begin{align*}
        g_l(N+l\1) \le  \frac{kl}{2}\,(N+l\1)^{k/2-1}\leq\frac{kl^{k/2}}{2}\,(N+\1)^{k/2-1}\,.
    \end{align*}
    Therefore, we have proven that 
    \begin{equation*}
        \begin{aligned}
            \tr\big[\rho\cL[a^l-\alpha^l]^\dagger(f(N))\big]&\le -\tr\big[\rho\,N[-l+1:0]\,g_l(N)\big]\\
            &\qquad+|\alpha|^lkl^{k/2}\sqrt{l!}\,\tr[\rho\, (N+\1 )^{k/2-1+\frac{l}{2}}]\,.
        \end{aligned}
    \end{equation*}
    %\begin{equation*}
    %    \begin{aligned}
    %        \tr\big[\rho\cL[a^l-\alpha^l]^\dagger(f(N))\big]&\le -\tr\big[\rho\,N[-l+1:0]\,g_l(N)\big]\\
    %        &\qquad+|\alpha|^lkl^{k/2+1}\sqrt{l!}\,\tr[\rho\, (N+\1 )^{k/2-1+\frac{l}{2}}]\,.
    %    \end{aligned}
    %\end{equation*}
    Next, we upper bound the first term above 
    \begin{equation*}
        \begin{aligned}
            \tr\big[\rho\,N[-l+1:0]\,g_l(N)\big]&\overset{(3)}{\geq}l\tr\big[\rho\,N[-l+1:0](N+\1)^{k/2-1}\big]\\
            &\overset{(4)}{\geq}l\,\tr\big[\rho\,(N+\1)^{l+k/2-1}\,\big]-\frac{(l+1)l^2}{2}\tr\big[\rho\,(N+\1 )^{l+k/2-2}\big]\,.
        \end{aligned}
    \end{equation*}
    In $(3)$, we used  \Cref{lem:upper-lower-bound-gl} below with the fact that $N[-l+1:0]$ is supported on the Fock states $|n\rangle$ with $n\ge l-1$;
    in $(4)$ we used that 
    \begin{align*}
        N[-l+1:0]&=\sum_{n\ge 0}\,(n-l+1)\dots n \,\ketbra{n}{n}\\
        &=\sum_{n\ge l}\,(n-l+1)\dots n\,\ketbra{n}{n}\\
        &\overset{(5)}{\ge} \sum_{n\ge l}\left((n+1)^l-\frac{(l+1)l}{2}(n+1)^{l-1}\right)\ketbra{n}{n}\\
        &\ge (N+\1)^l-\frac{(l+1)l}{2}\,(N+\1)^{l-1}\,,
    \end{align*}
    where $(5)$ comes from \Cref{lem:bounds-ccr-l-product} below, whereas the last inequality follows from the fact that $l\ge 2$. To sum up, we showed that
    \begin{equation}\label{eq-ex:l-dissipation-upper-bound}
        \begin{aligned}
            \tr\big[\cL[a^l-\alpha^l](\rho)(f(N))\big]&\le -l\tr\big[\rho\,(N+\1)^{l+k/2-1}\,\big]+\frac{(l+1)l^2}{2}\tr\big[\rho\,(N+\1 )^{l+k/2-2}\big]\\
           &\qquad+|\alpha|^lkl^{k/2}\sqrt{l!}\,\tr[\rho\, (N+\1 )^{k/2-1+\frac{l}{2}}]\\
           &\le -l\tr\big[\rho\,(N+\1)^{l+k/2-1}\,\big]\\
           &\qquad+\frac{l}{2}\biggl(\underbrace{{(l+1)l}+2|\alpha|^lkl^{k/2 - 1}\sqrt{l!}}_{\eqqcolon \Delta_l}\biggr)\,\tr\big[\rho\,(N+\1 )^{l+k/2-2}\big]
        \end{aligned}
    \end{equation}
    %\begin{equation}\label{eq-ex:l-dissipation-upper-bound}
    %    \begin{aligned}
    %        \tr\big[\cL[a^l-\alpha^l](\rho)(f(N))\big]&\le -l\tr\big[\rho\,(N+\1)^{l+k/2-1}\,\big]+\frac{(l+1)l^2}{2}\tr\big[\rho\,(N+\1 )^{l+k/2-2}\big]\\
    %       &\qquad+|\alpha|^lkl^{k/2+1}\sqrt{l!}\,\tr[\rho\, (N+\1 )^{k/2-1+\frac{l}{2}}]\\
    %       &\le -l\tr\big[\rho\,(N+\1)^{l+k/2-1}\,\big]\\
    %       &\qquad+\frac{l}{2}\biggl(\underbrace{{(l+1)l}+2|\alpha|^lkl^{k/2}\sqrt{l!}}_{\eqqcolon \Delta_l}\biggr)\,\tr\big[\rho\,(N+\1 )^{l+k/2-2}\big]
    %    \end{aligned}
    %\end{equation}
    where we used again that $l\ge 2$ in the last inequality. Half of the leading order term can be used to control the second term by a constant. For that, we use the spectral decomposition of the operator $N$ so that the above problem can be reduced to the following simple optimization: 
    \begin{equation}\label{eq:optimization}
        \sup_{x\geq0}\left(-x^\nu+\Delta_lx^{\nu-1}\right)=\Delta_l^\nu\left(\frac{(\nu-1)^{\nu-1}}{\nu^\nu}\right)
    \end{equation}
    for $\nu\geq1$ defined as 
    \begin{equation*}
        \nu=l+\frac{k}{2}-1\,.
    \end{equation*}
    The result follows after invoking \Cref{prop-ex:uniformly-bounded-semigroup}. 
\end{proof}
\begin{rmk}\label{rmk:l-diss-multimode}
    The single-mode bound proved above can be generalized to the multimode setting, with generated given for some $\alpha_j\in\mathbb{C}$, $j\in[m]$, by 
    \begin{equation*}
        \cL_l^{(m)}\coloneqq\sum_{j=1}^m\cL[a_j^l-\alpha_j^l]\,.
    \end{equation*}
    Since all the bounds used in the proof of \Cref{lem:l-diss} were derived at the operator level, we directly get for $\k \in \N^m$
    \begin{align*}
        \tr\big[\cL_l^{(m)}(\rho)(N+\1)^{\k/2}\big]&\le \sum_{i=1}^m-\frac{l}{2}\tr\big[\rho\,(N_i+\1)^{l-1}(N + \1)^{\k/2}\big]+\mu^{(l)}_{k_i}\tr\big[\rho\,\prod_{j\neq i}(N_j+\1)^{k_j/2}\big]
    \end{align*}
\end{rmk}
For later references, we single out the case $l=2$.
\begin{cor}[$2$-photon dissipation]\label{lem:cat-identity}
    For any integers $k\ge 1$, $\alpha\in\mathbb{C}$ and any state $\rho\in\cT_f$, 
    \begin{align*}
        	\tr\big[\cL[a^2-\alpha^2](\rho)(N+\1)^{k/2}\big]&\le -\tr\big[\rho\,(N+\1)^{k/2}\big]+\mu^{(2)}_k
    \end{align*}
    where $\mu^{(2)}_k=(\Delta^{(2)}_k)^\nu\left(\frac{(\nu-1)^{\nu-1}}{\nu^\nu}\right)$ with $\nu=\frac{k}{2}+1$ and $\Delta_2=6+2\sqrt{2!}|\alpha|^2k2^{k/2 - 1}$. Therefore, $\cL_2\coloneqq \cL[a^2-\alpha^2 ]$ generates a Sobolev and positivity preserving quantum Markov semigroup which satisfies for all states $\rho$ in $W^{k,1}$
    \begin{equation*}
        \|e^{t\cL_2}(\rho)\|_{W^{k,1}}\leq\max\left\{\|\rho\|_{W^{k,1}},\mu^{(2)}_k\right\}\,.
    \end{equation*}
\end{cor}
From the above bounds, we directly get the property of Sobolev preservation for the $X$-gate:
\begin{cor}[$X$-gate]
    For any $T>0$, $\alpha\in\mathbb{C}$, $k\in\N$ and all states $\rho\in\cT_f$,
    \begin{equation*}
        \tr[\cL[a^2-e^{2i\pi t/T}\alpha^2](\rho)(N+\1)^{k/2}]\leq-\tr[\rho(N+\1)^{k/2}]+\mu_k^{(2)}\,,
    \end{equation*}
    where $\mu_k^{(2)}$ is defined in \Cref{lem:cat-identity}. Therefore, $\cL[a^2-e^{2i\pi t/T}\alpha^2]$ generates a Sobolev and positivity preserving quantum evolution system $\cP_{t,t_0}$ which satisfies for all states $\rho\in W^{k,1}$
    \begin{equation*}
        \|\cP_{t,t_0}(\rho)\|_{W^{k,1}}\leq\max\left\{\|\rho\|_{W^{k,1}},\mu_k^{(2)}\right\}\,.
    \end{equation*}
\end{cor}
\begin{proof}
  The statement directly follows from \Cref{lem:cat-identity} and $|e^{2i\pi t/T}\alpha^2|=|\alpha^2|$
\end{proof}

Next, we consider a Hamiltonian of degree $d_H = 2(l-1)$ and show that together with the $l$-photon dissipation the sum $\cL[a_l-\alpha^l]+\cH[H]$ satisfies \Cref{assum:sobolev-stability}. We assume that $H$ has the following polynomial representation: for $\lambda_{i,j}\in\mathbb{C}$ with $\max_{i,j}|\lambda_{i,j}|=\Lambda$,
\begin{equation}\label{Hpolyrep}
    H=p(a,\ad)=\sum_{\substack{i \le j\\i+j \le d_H}}\lambda_{i,j}a^i(\ad)^j + \overline{\lambda_{i,j}} a^j (\ad)^i \, . 
\end{equation}
 Note that any monomial in $a,\ad$ of degree at most $d_H$ can be achieved from the representation above thanks to the CCR.
 
\begin{lem}\label{lem:l-diss-hamiltonian}
    Let $\cL_l\coloneqq \cL[a^l-\alpha^l]$, $\alpha\in\mathbb{C}$, be the $l$-photon dissipation and $H$ as in \eqref{Hpolyrep}. Then, for all states $\rho\in\cT_f$
    \begin{equation}
        \begin{aligned}
            \tr[(\cL_l+\cH[H])(\rho)(N+\1)^{k/2}] &\leq-\frac{l}{2}\,\tr[\rho(N+\1)^{k/2}]+\frac{l}{2}\mu_k\,.\label{eqdiffLH}
        \end{aligned}
    \end{equation}
    for $\mu_k,\nu\geq1$ defined by 
    \begin{equation*}
        \mu_k=c^\nu\left(\frac{(\nu-1)^{\nu-1}}{\nu^\nu}\right)\quad\text{with}\quad c={(l+1)l}+2|\alpha|^lkl^{k/2-1}\sqrt{l!}+\Lambda(2l)^{k/2}\sqrt{(2l)!}\,,\quad\nu=l+\frac{k}{2}-1\,.
    \end{equation*}
    Therefore, $\cL_l+\cH[H]$ generates a Sobolev and positivity preserving quantum Markov semigroup which satisfies for all states $\rho\in W^{k,1}$
    \begin{equation}\label{eqintegratedLH}
        \|e^{t(\cL_l+\cH[H])}(\rho)\|_{W^{k,1}}\leq\max\Big\{\|\rho\|_{W^{k,1}},\mu_k\Big\}\,.
    \end{equation}
\end{lem}
\begin{proof}
    We reuse the bound given in \Cref{eq-ex:l-dissipation-upper-bound}:
    \begin{equation*}
        \begin{aligned}
            \tr[\rho\cL_l^\dagger(f(N))]\leq-l\tr\big[\rho\,(N+\1)^{l+k/2-1}\,\big]+\frac{l}{2}\Delta_l\,\tr\big[\rho\,(N+\1 )^{l+k/2-2}g_l(N)\big]\,,
        \end{aligned}
    \end{equation*}
    where $f(x)=(x+1)^{k/2} 1_{x\ge -1}$ and $\Delta_l={(l+1)l}+2|\alpha|^lkl^{k/2 - 1}\sqrt{l!}$\,. To upper bound
    \begin{equation*}
        \begin{aligned}
            \tr[\cH[H](\rho)(N+\1)^{k/2}]&=i\tr[{\rho[(N+\1)^{k/2},H]}]\,,
        \end{aligned}
    \end{equation*}
    we define $g_u$ similarly to \Cref{eq:f-g-l-function} by 
    \begin{equation*}
        g_u(x) = \begin{cases}
            f(x) - f(x - u) & x \ge u-1;\\
            f(x) & u-1 > x \ge 0;\\
            0 & 0 > x\,.
        \end{cases}
    \end{equation*}
    For $d_H=0$ the bound is trivial, so we assume $d_H\geq1$. Then, we compute
    \begin{equation*}
        \begin{aligned}
            i[&f(N),H]\\
            &=i\sum_{\substack{0\leq j< i\\0<i+j\leq d_H}}f(N)(\lambda_{i,j}(\ad)^ia^j+\overline{\lambda_{i,j}}(\ad)^ja^i)-(\lambda_{i,j}(\ad)^ia^j+\overline{\lambda_{i,j}}(\ad)^ja^i)f(N)\\
            &=i\sum_{\substack{0\leq j< i\\0<i+j\leq d_H}}\lambda_{i,j}f(N)N[-i+1:-i+j](\ad)^{i-j}+\overline{\lambda_{i, j}}a^{i-j}f(N-i+j)N[-i+1:-i+j]\\
            &\qquad\qquad-\lambda_{i,j}N[-i+1:-i+j]f(N-i+j)(\ad)^{i-j}-\overline{\lambda_{i, j}}a^{i-j}N[-i+1:-i+j]f(N)\\
            &=i\sum_{\substack{0< r\leq i\\0<2i-r\leq d_H}}-\overline{\lambda_{i,i-r}}a^{r}N[-i+1:-r]g_r(N)+\lambda_{i,i-r}g_r(N)N[-i+1:-r](\ad)^{r}\\
            &\overset{(1)}{\leq} \sum_{\substack{0< r\leq i\\0<2i-r\leq d_H}}2|\lambda_{i,i-r}|\sqrt{(N+\1)\cdots(N+r\1)}g_{r}(N+r\1)N[r-i+1:0]\\
            &\overset{(2)}{\leq} \sum_{\substack{0< r\leq i\\0<2i-r\leq d_H}}2\sqrt{r!}|\lambda_{i,i-r}|g_{r}(N+r\1)(N+\1)^{i-r/2}\\
        \end{aligned}
    \end{equation*}
    %\begin{equation*}
    %    \begin{aligned}
    %        i[&f(N),H]\\
    %        &=i\sum_{\substack{0\leq j< i\\0<i+j\leq d_H}}f(N)(\lambda_{i,j}(\ad)^ia^j+\overline{\lambda_{i,j}}(\ad)^ja^i)-(\lambda_{i,j}(\ad)^ia^j+\overline{\lambda_{i,j}}(\ad)^ja^i)f(N)\\
    %        &=i\sum_{\substack{0\leq j< i\\0<i+j\leq d_H}}\lambda_{i,j}f(N)N[-i+1:-i+j](\ad)^{i-j}+\overline{\lambda_{i, j}}a^{i-j}f(N-i+j)N[-i+1:-i+j]\\
    %        &\qquad\qquad-\lambda_{i,j}N[-i+1:-i+j]f(N-i+j)(\ad)^{i-j}-\overline{\lambda_{i, j}}a^{i-j}N[-i+1:-i+j]f(N)\\
    %        &=i\sum_{\substack{0< r\leq i\\0<2i-r\leq d_H}}-\overline{\lambda_{i,i-r}}a^{r}N[-i+1:-r]g_r(N)+\lambda_{i,i-r}g_r(N)N[-i+1:-r](\ad)^{r}\\
    %        &\overset{(1)}{\leq} \sum_{\substack{0< r\leq i\\0<2i-r\leq d_H}}(r+1)|\lambda_{i,i-r}|\sqrt{(N+\1)\cdots(N+r\1)}g_{r}(N+r\1)N[r-i+1:0]\\
    %        &\overset{(2)}{\leq} \sum_{\substack{0< r\leq i\\0<2i-r\leq d_H}}(r+1)\sqrt{r!}|\lambda_{i,i-r}|g_{r}(N+r\1)(N+\1)^{i-r/2}\\
    %    \end{aligned}
    %\end{equation*}
    where we have used \Cref{lem:two-point-hamiltonian-bound} in $(1)$ and \ref{lem:bounds-ccr-l-product} in $(2)$. Next, we use the boundedness assumption $|\lambda_{i,j}|\leq\Lambda$ for all $i,j\in\{1,...,d_H\}$:
    \begin{equation*}
        \begin{aligned}
            i\tr[[H,\rho](N+\1)^{k/2}]&\leq 2\Lambda\sum_{i=1}^{d_H}\sum_{r=1}^i\sqrt{r!}\tr[\rho g_{r}(N+r)(N+\1)^{i-r/2}]\\
            &\overset{(3)}{\leq}2\Lambda\sum_{i=1}^{d_H}\sum_{r=1}^i\sqrt{r!}r^{k/2-1}\tr[\rho (N+\1)^{k/2+i-r/2-1}]\\
            &\leq \Lambda (d_H + 1)d_H\sqrt{d_H!}d_H^{k/2-1}\tr[\rho (N+\1)^{k/2+d_H/2-1}]\,,
        \end{aligned}
    \end{equation*}
    %\begin{equation*}
    %    \begin{aligned}
    %        i\tr[[H,\rho](N+\1)^{k/2}]&\leq \Lambda\sum_{i=1}^{d_H}\sum_{r=1}^i (r+1)\sqrt{r!}\tr[\rho g_{r}(N+r)(N+\1)^{i-r/2}]\\
    %        &\overset{(3)}{\leq}\Lambda\sum_{i=1}^{d_H}\sum_{r=1}^i (r+1)\sqrt{r!}r^{k/2-1}\tr[\rho (N+\1)^{k/2+i-r/2-1}]\\
    %        &\leq \Lambda(d_H+1)d_H^2\sqrt{d_H!}d_H^{k/2-1}\tr[\rho (N+\1)^{k/2+d_H/2-1}]\,,
    %    \end{aligned}
    %\end{equation*}
    where we used \Cref{lem:upper-lower-bound-gl} in $(3)$. As the above function is monotone in $d_H$ we can w.l.o.g assume $d_H=2(l-1)$ and conclude
    \begin{equation*}
        \begin{aligned}
            \tr[(\cL_l+\cH[H])(f(N))]&\leq-l\tr\big[\rho\,(N+\1)^{l+k/2-1}\,\big]\\
            &\qquad\qquad+\frac{l}{2}\left(\Delta_l+\Lambda(2l)^{k/2}\sqrt{(2l)!}\right)\tr[\rho (N)(N+\1)^{l+k/2-2}]\,.
        \end{aligned}
    \end{equation*}
    % \begin{equation*}
    %     \begin{aligned}
    %         \tr[(\cL_l+\cH[H])(f(N))]&\leq-l\tr\big[\rho\,(N+\1)^{l+k/2-1}\,\big]\\
    %         &\qquad\qquad+\frac{l}{2}\left(\Delta_l+\Lambda(2l)^{k/2+1}\sqrt{(2l)!}\right)\tr[\rho (N)(N+\1)^{l+k/2-2}]\,.
    %     \end{aligned}
    % \end{equation*}
   The same optimization as in \Cref{eq:optimization} provides   inequality \eqref{eqdiffLH}. Inequality \eqref{eqintegratedLH} follows after invoking \Cref{prop-ex:uniformly-bounded-semigroup}.
\end{proof}

\begin{lem}[$Z(\theta)$-gate]\label{lem:z-theta-energetic-stab}
    For any state $\rho\in\cT_f$, $\alpha\in\mathbb{C}$, $\varepsilon>0$ and $k\in\N$ 
    \begin{equation*}
        \tr[(\varepsilon\cH[a+\ad]+\cL[a^2+\alpha^2])(\rho)(N+\1)^{k/2}]\leq -\,\tr\big[\rho\, (N+\1)^{k/2}\big]+\mu_k\,.
    \end{equation*}
    where $\mu_k\geq0$ is defined by 
    \begin{equation*}
        \mu_k=(\Delta_2+\varepsilon4k)^\nu\left(\frac{(\nu-1)^{\nu-1}}{\nu^\nu}\right)\qquad\text{with}\qquad\nu=\frac{k}{2}+1\,.
    \end{equation*}
    Therefore, $\varepsilon\cH[a+\ad]+\cL[a^2+\alpha^2]$ generates a Sobolev and positivity preserving quantum Markov semigroup which satisfies for all states $\rho\in W^{k,1}$
    \begin{equation}\label{etlboundfff}
        \|e^{t(\varepsilon\cH[a+\ad]+\cL[a^2+\alpha^2])}(\rho)\|_{W^{k,1}}\leq\max\Big\{\|\rho\|_{W^{k,1}},\mu_k\Big\}\,.
    \end{equation}
\end{lem}
\begin{proof}
    By \Cref{eq-ex:l-dissipation-upper-bound} in \Cref{lem:l-diss},
    \begin{equation*}
        \begin{aligned}
            \tr\big[\cL[a^2-\alpha^2](\rho)(f(N))\big]&\le -2\tr\big[\rho\,(N+\1)^{k/2+1}\,\big]\\
           &\qquad+\biggl(\underbrace{6+2|\alpha|^2k2^{k/2 - 1}\sqrt{2}}_{\eqqcolon \Delta_2}\biggr)\,\tr\big[\rho\,(N+\1 )^{k/2}\big]
        \end{aligned}
    \end{equation*}
    % \begin{equation*}
    %     \begin{aligned}
    %         \tr\big[\cL[a^2-\alpha^2](\rho)(f(N))\big]&\le -2\tr\big[\rho\,(N+\1)^{k/2+1}\,\big]\\
    %        &\qquad+\biggl(\underbrace{6+2|\alpha|^2k2^{k/2}\sqrt{2}}_{\eqqcolon \Delta_2}\biggr)\,\tr\big[\rho\,(N+\1 )^{k/2}\big]
    %     \end{aligned}
    % \end{equation*}
    where $f(x)=(x+1)^{k/2} 1_{x\ge -1}$. Next, by \Cref{eq:symmetry-function}, \Cref{lem:two-point-hamiltonian-bound} and \Cref{lem:upper-lower-bound-gl}, we have that
    \begin{equation*}
        \begin{aligned}
            \tr[\cH[a+\ad](\rho)f(N)]&=i\tr[\rho\left(f(N)(a+\ad)-(a+\ad)f(N)\right)]\\
            &=\tr[\rho\left(-iag_1(N)+ig_1(N)\ad \right)]\\
            &\leq 2\,\tr[\rho g_1(N+\1 )\sqrt{N+\1}]\\
            &\leq 2k\tr[\rho(N+\1)^{k/2-\frac{1}{2}}]\,,
        \end{aligned}
    \end{equation*}
    %\blue{
    %\begin{equation*}
    %    \begin{aligned}
    %        \tr[\cH[a+\ad](\rho)f(N)]&=i\tr[\rho\left(f(N)(a+\ad)-(a+\ad)f(N)\right)]\\
    %        &=\tr[\rho\left(-iag_1(N)+ig_1(N)\ad \right)]\\
    %        &\leq 2\,\tr[\rho g_1(N+\1 )\sqrt{N+\1}]\\
    %        &\leq 4k\tr[\rho(N+\1)^{k/2-\frac{1}{2}}]\,,
    %    \end{aligned}
    %\end{equation*}
    %}
    %\begin{equation*}
    %    \begin{aligned}
    %        \tr[\cH[a+\ad](\rho)f(N)]&=i\tr[\rho\left(f(N)(a+\ad)-(a+\ad)f(N)\right)]\\
    %        &=\tr[\rho\left(-iag_1(N)+ig_1(N)\ad \right)]\\
    %        &\leq 4\,\tr[\rho g_1(N+\1 )\sqrt{N+\1}]\\
    %        &\leq 8k\tr[\rho(N+\1)^{k/2-\frac{1}{2}}]\,,
    %    \end{aligned}
    %\end{equation*}
    where we recall that
    \begin{equation}\label{eq-appx:f-g-l-functionlequal1}
        g_1(x) = \begin{cases}
            f(x) - f(x - 1) & x \ge 0;\\
            0 & 0 > x\,.
        \end{cases}
    \end{equation}
    Thus,
    \begin{equation*}
        \begin{aligned}
            \tr\big[(\epsilon\cH[a+a^\dagger]+\cL[a^2-\alpha^2])(\rho)(f(N))\big]&\le -2\tr\big[\rho\,(N+\1)^{k/2+1}\,\big]+\left(\Delta_2+\varepsilon2k\right)\,\tr\big[\rho\,(N+\1 )^{k/2}\big]
        \end{aligned}
    \end{equation*}
    % \begin{equation*}
    %     \begin{aligned}
    %         \tr\big[(\epsilon\cH[a+a^\dagger]+\cL[a^2-\alpha^2])(\rho)(f(N))\big]&\le -2\tr\big[\rho\,(N+\1)^{k/2+1}\,\big]+\left(\Delta_2+\varepsilon8k\right)\,\tr\big[\rho\,(N+\1 )^{k/2}\big]
    %     \end{aligned}
    % \end{equation*}
    Then the same optimization as in \Cref{eq:optimization}, i.e.
    \begin{equation*}
        \sup_{x\geq0}\left(-x^\nu+(\Delta_2+\varepsilon2k)x^{\nu-1}\right)=(\Delta_2+\varepsilon2k)^\nu\left(\frac{(\nu-1)^{\nu-1}}{\nu^\nu}\right)
    \end{equation*}
    % \begin{equation*}
    %     \sup_{x\geq0}\left(-x^\nu+(\Delta_2+\varepsilon8k)x^{\nu-1}\right)=(\Delta_2+\varepsilon8k)^\nu\left(\frac{(\nu-1)^{\nu-1}}{\nu^\nu}\right)
    % \end{equation*}
    for $\nu\geq1$ defined as 
    \begin{equation*}
        \nu=\frac{k}{2}+1
    \end{equation*}
    ends the proof of the differential upper bound, and \eqref{etlboundfff} follows from \Cref{prop-ex:uniformly-bounded-semigroup}.
\end{proof}

\begin{prop}[CNOT-gate]\label{lem:assum2-CNOT}
    For all $\k\coloneqq (k_1,k_2) \in \N^2$ such that
    \begin{equation*}
        32|\alpha|k_12^{k_1/2-1/2}\leq k_2\,,
    \end{equation*}
     there exists a constant $\mu_{\k}$ such that for all states $\rho \in \cT_f$
    \begin{equation*}
        \begin{aligned}
            &\tr[\left(\cL[a^2-\alpha^2]+\cL[b^2-\alpha^2-\frac{\alpha}{2}(1-e^{2i\pi t/T})(a-\alpha)]\right)(\rho)(N_1 + \1)^{k_1/2}(N_2 + \1)^{k_2/2}]\\
            &\qquad\qquad\qquad\qquad\qquad\qquad\qquad\qquad\qquad\leq-\frac{1+k_2}{8}\tr[\rho\Bigl((N_1+\1)^{k_1/2}(N_2+\1)^{k_2/2}\Bigr)]+\mu_{\k}\,.
        \end{aligned}
    \end{equation*}
    Therefore, the CNOT-gate generates a Sobolev and positivity preserving quantum Markov semigroup which satisfies for all states $\rho\in W^{\k,1}$
    \begin{equation}\label{lastclaimsobolevbound}
        \|\cP^{\operatorname{CNOT}}_{t,t_0}(\rho)\|_{W^{\k,1}}\leq\max\left\{\|\rho\|_{W^{\k,1}},\frac{8\mu_{\k}}{1+k_2}\right\}\,.
    \end{equation}
    For a general $\k \in \R_+^2$ and $x \in W^{\k, 1}$ one obtains
    \begin{equation*}
         \|\cP^{\operatorname{CNOT}}_{t,t_0}(x)\|_{W^{\k_r,1}}\leq \gamma_k \|x\|_{W^{\k_r,1}} \, ,
    \end{equation*}
    where $\gamma_{\k} = \max\{1,\frac{8\mu_{\k}}{1+k_2}\}$ for $\k \in \{\k_r\}_{r \in \N}$ and an interpolated constant in all other cases. Additionally, for $\k > 0$ and $\rho \in W^{\k, 1}$ there exists a sequence $\{t_n\}_{n \in \N_{\geq1}}$ and a $\overline{\rho}\in W^{k,1}$ so that
    \begin{equation*}
        \lim_{t_n\rightarrow\infty}\cP^{\operatorname{CNOT}}_{t_n,t_0}(\rho)=\overline{\rho}\,.
    \end{equation*}
\end{prop}
\begin{proof}
    We denote $f(x_1,x_2)=f_1(x_1)f(x_2)$ with $f_{1}(x_{1})=(x_{1}+1)^{k_{1}/2} 1_{x_{1}\ge -1}$, $f_{2}(x_{2})=(x_{2}+1)^{k_{2}/2} 1_{x_{2}\ge -1}$, and rewrite the CNOT-generator as
    \begin{equation*}
        \begin{aligned}
            \cL\coloneqq \cL[a^2-\alpha^2]&+\cL[b^2-\alpha^2-\frac{\alpha}{2}(1-e^{2i\pi t/T})(a-\alpha)]=\cL[a^2-\alpha^2]+\cL[b^2+za+w]
        \end{aligned}
    \end{equation*}
    where $z\coloneqq-\frac{\alpha}{2}(1-e^{2i\pi t/T})$ and $w\coloneqq -\alpha(z+\alpha)$. As in the previous proofs, we investigate the action of the adjoint on $f(N)\coloneqq f(N_1,N_2)$
    \begin{equation*}
        \tr[\cL(\rho)f(N)]=\tr[\rho\cL^\dagger(f(N))]\,.
    \end{equation*}
    We first focus on the second Lindbladian $\cL[b^2+za+w]$: we first consider, for $n\coloneqq (n_1,n_2)\in\N^2$, 
    \begin{equation*}
        \begin{aligned}
            &\cL[b^2+za+w]^\dagger(\ketbra{n}{n})\\
            &\quad =((\bd)^2+\overline{z}\ad+\overline{w})\ketbra{n}{n}(b^2+za+w)-\frac{1}{2}\left\{((\bd)^2+\overline{z}\ad+\overline{w})(b^2+za+w),\ketbra{n}{n}\right\}\\
            &\quad =F_1(n)\ketbra{n}{n}+F_2(n) \ketbra{n_1,n_2+2}{n_1,n_2+2}+F_3(n) \ketbra{n_1+1,n_2}{n_1+1,n_2}\\
            &\quad\qquad + \big(F_4(n) \ketbra{n_1,n_2+2}{n_1+1,n_2}+h.c.\big)+\big(F_5(n) \ketbra{n_1+1,n_2-2}{n} + h.c.\big) \\
            &\quad\qquad +\big(F_6(n)\ketbra{n_1,n_2-2}{n}+h.c.\big) +\big( F_7(n)\ketbra{n_1-1,n_2}{n} + h.c.\big)\\
            &\quad\qquad + \big(F_8(n)\ketbra{n_1,n_2+2}{n}+h.c. \big)+\big(F_9(n)\ketbra{n_1+1,n_2}{n}+h.c. \big)  \\
            &\quad\qquad + \big(F_{10}(n)\ketbra{n_1-1,n_2+2}{n}+h.c.\big)\,,
        \end{aligned}
    \end{equation*}
    where the notation $h.c.$ above stands for Hermitian conjugate, $|n\rangle=0$ whenever $n\notin \mathbb{N}^2$ by convention, and where
    \begin{align*}
        &F_1(n)\coloneqq - n_2(n_2-1)-|z|^2n_1\\
        &F_2(n)\coloneqq (n_2+1)(n_2+2) \\
        &F_3(n)\coloneqq |z|^2(n_1+1) \\
        &F_4(n)\coloneqq  z\sqrt{(n_1+1)(n_2+1)(n_2+2)}\\
        &F_5(n)\coloneqq -\frac{1}{2}\overline{z}\sqrt{(n_1+1)n_2(n_2-1)}\\
        &F_6(n)\coloneqq  -\frac{1}{2}\overline{w}\sqrt{n_2(n_2-1)}\\
        &F_7(n)\coloneqq -\frac{1}{2}\overline{w}z\sqrt{n_1}\\
        &F_8(n)\coloneqq \frac{1}{2} w\sqrt{(n_2+1)(n_2+2)}\\
        &F_9(n)\coloneqq \frac{1}{2}\overline{z}w\sqrt{n_1+1}\\
        &F_{10}(n)\coloneqq -\frac{1}{2}z\sqrt{n_1(n_2+1)(n_2+2)}\,.
    \end{align*}
    In the next step, we regroup the $17$ terms into terms differing only by a shift:
    \medskip
    \noindent\textit{Case 0:} Diagonal terms, involving $F_1$, $F_2$ and $F_3$,
    \begin{equation*}
        \begin{aligned}
            C_0(n)\coloneqq F_1(n) \ketbra{n}{n}+F_2(n)\ketbra{n_1,n_2+2}{n_1,n_2+2} + F_3(n)\ketbra{n_1+1,n_2}{n_1+1,n_2}\,.
        \end{aligned}
    \end{equation*}
    
    \noindent\textit{Case 1:} Terms of the form $\ketbra{n_1+1,n_2-2}{n_1,n_2}$, involving $\overline{F}_4$, $F_5$ and $\overline{F}_{10}$,
    \begin{equation*}
        \begin{aligned}
            C_1(n)\coloneqq \overline{F}_4(n) \ketbra{n_1+1,n_2}{n_1,n_2+2} + F_5(n) \ketbra{n_1+1,n_2-2}{n} + \overline{F}_{10}(n) \ketbra{n}{n_1-1,n_2+2} \,.
        \end{aligned}
    \end{equation*}
    
    \noindent\textit{Case 1':} Terms of the form $\ketbra{n_1,n_2}{n_1+1,n_2-2}$, involving $F4$, $\overline{F}_5$ and $F_{10}$,
    \begin{equation*}
        \begin{aligned}
            C_{1'}(n)&\coloneqq {F}_4(n) \ketbra{n_1,n_2+2}{n_1+1,n_2} + \overline{F}_5(n) \ketbra{n}{n_1+1,n_2-2} + {F}_{10}(n) \ketbra{n_1-1,n_2+2}{n} \,.
        \end{aligned}
    \end{equation*}

    \noindent\textit{Case 2:} Terms of the form $\ketbra{n_1,n_2-2}{n_1,n_2}$, involving $F_6$ and $\overline{F}_8$,
    \begin{equation*}
        \begin{aligned}
            C_{2}(n)\coloneqq {F}_6(n) \ketbra{n_1,n_2-2}{n}+\overline{F}_8(n)\ketbra{n}{n_1,n_2+2} \,.\\
        \end{aligned}
    \end{equation*}
    
    \noindent\textit{Case 2':} Terms of the form $\ketbra{n_1,n_2}{n_1,n_2-2}$, involving $\overline{F}_6$ and ${F}_8$,
    \begin{equation*}
        \begin{aligned}
            C_{2'}(n)\coloneqq \overline{F}_6(n) \ketbra{n}{n_1,n_2-2}+{F}_8(n) \ketbra{n_1,n_2+2}{n} \,.\\
        \end{aligned}
    \end{equation*}

    \noindent\textit{Case 3:} Terms of the form $\ketbra{n_1-1,n_2}{n_1,n_2}$, involving $F_7$ and $\overline{F}_9$,
    \begin{equation*}
        \begin{aligned}
            C_{3}(n)\coloneqq {F}_7(n) \ketbra{n_1-1,n_2}{n}+\overline{F}_9(n) \ketbra{n}{n_1+1,n_2} \,.\\
        \end{aligned}
    \end{equation*}

   \noindent\textit{Case 3':} Terms of the form $\ketbra{n_1,n_2}{n_1-1,n_2}$, involving $\overline{F}_7$ and ${F}_9$,
    \begin{equation*}
        \begin{aligned}
            C_{3'}(n)\coloneqq \overline{F}_7(n) \ketbra{n}{n_1-1,n_2}+{F}_9(n)\ketbra{n_1+1,n_2}{n} \,.\\
        \end{aligned}
    \end{equation*}
    To summarize, we have decomposed $\cL[b^2+za+w]^\dagger(\ketbra{n}{n})$ into the sum 
    \begin{align}\label{eqLnn}
        \cL[b^2+za+w]^\dagger(|n\rangle\langle n|)=C_0(n)+C_1(n)+C_{1'}(n)+C_2(n)+C_{2'}(n)+C_3(n)+C_{3'}(n)\,.
    \end{align}
    Next, we introduce the functions $g_{j,l}:\mathbb{N}\to \mathbb{R}$, $j\in\{1,2\}$, $l\in\mathbb{N}$, as 
    \begin{equation*}
        g_{j, l}(x) = \begin{cases}
            f_j(x) - f_j(x - l) & x \ge l;\\
            f_j(x) & l > x \ge 0;\\
            0 & 0 > x\,.
        \end{cases}
    \end{equation*}
    Multiplying \Cref{eqLnn} by $f(n)$ and summing over $n\in\mathbb{N}^2$, we find that 
    \begin{align*}
        \cL[b^2+za+w]^\dagger(f(N))=\hat{C}_0+\hat{C}_1+\hat{C}_{1'}+\hat{C}_2+\hat{C}_{2'} +\hat{C}_3+\hat{C}_{3'}\,,
    \end{align*}
    with 
    \begin{align*}
        &\hat{C}_0\coloneqq \sum_{n}\,C_0(n)=\sum_n -\,\Big(f_1(n_1)g_{2,2}(n_2)n_2(n_2-1)+f_2(n_2)g_{1,1}(n_1)|z|^2n_1\Big) \ketbra{n}{n}\\
        &\hat{C}_1\coloneqq \sum_{n}\,C_1(n)\\
        &\qquad =\sum_{n}\,-\frac{\overline{z}}{2}\sqrt{(n_1+1)(n_2-1)n_2}\Bigl(f_1(n_1)g_{2,2}(n_2)+g_{1,1}(n_1+1)f_2(n_2-2)\Bigr)\ketbra{n_1+1,n_2-2}{n}\\
        &\hat{C}_{1'}\coloneqq \hat{C}_1^\dagger \\
        &\qquad =\sum_n\,-\frac{{z}}{2}\sqrt{(n_1+1)(n_2-1)n_2}\Bigl(f_1(n_1)g_{2,2}(n_2)+g_{1,1}(n_1+1)f_2(n_2-2)\Bigr)\ketbra{n}{n_1+1,n_2-2}\\
        &\hat{C}_2\coloneqq \sum_n\,C_2(n) = \sum_n-\frac{\overline{w}}{2}\sqrt{(n_2-1)n_2}f_1(n_1)g_{2,2}(n_2)\ketbra{n_1,n_2-2}{n}\\
        &\hat{C}_{2'}\coloneqq \hat{C}_2^\dagger= \sum_n-\frac{{w}}{2}\sqrt{(n_2-1)n_2}f_1(n_1)g_{2,2}(n_2)\ketbra{n}{n_1,n_2-2}\\
        &\hat{C}_3\coloneqq \sum_n\,C_3(n) = \sum_n-\frac{\overline{w}z}{2}\sqrt{n_1}g_{1,1}(n_1)f_2(n_2)\ketbra{n_1-1,n_2}{n}\\
        &\hat{C}_{3'}\coloneqq \hat{C}_3^\dagger =\sum_n-\frac{{w}\overline{z}}{2}\sqrt{n_1}g_{1,1}(n_1)f_2(n_2)\ketbra{n}{n_1-1,n_2}\,.
    \end{align*}
    We will use an upper bound on $\hat{C}_0(n)$ in what follows:    
    \begin{equation*}
        \begin{aligned}
            &C_0(n) =- \Big(f_1(n_1)g_{2,2}(n_2)(n_2-1)n_2+|z|^2g_{1,1}(n_1)f_2(n_2)n_1\Big)\\
            &\le -\Big( f_1(n_1)g_{2,2}(n_2)1_{n_2\geq2}((n_2+1)^{2}-3(n_2+1))+|z|^2g_{1,1}(n_1)f_2(n_2)n_1\Big)\\
            &\overset{(1)}{\le} -\left\{ k_2f_1(n_1)(n_2+1)^{k_2/2}1_{n_2\geq2}\Big((n_2+1)-1_{k_2\geq3}\frac{k_2}{2}-6\Big) + |z|^2 1_{n_1\geq1}(n_1+1)^{k_1/2-1}f_2(n_2)n_1
            \right\}\\
            &\equiv C_{0'}(n)\,,
        \end{aligned}
    \end{equation*}
    where $(1)$ follows from \Cref{lem:upper-lower-bound-gl}. We denote this upper bound by $\hat{C}_{0'}=\sum_n C_{0'}(n)\ketbra{n}{n}$. Recall that, by \Cref{lem:l-diss} and \Cref{rmk:l-diss-multimode},
    \begin{equation}\label{eq-ex:CNOT-diss-rep}
        \begin{aligned}
            \cL[a^2-\alpha^2]^\dagger(f(N))&\leq - \,(N_1+\1)^{k_1/2+1}(N_2+\1)^{k_2/2}
            +\mu_{k_1}^{(2)}\, (N_2+\1)^{k_2/2} \\
            &=\sum_{n} \Big(-(n_1+1)f(n)+\mu_{k_1}^{(2)}\,f_2(n_2)\Big)\ketbra{n}{n}=:\hat{C}_{4}\,,
        \end{aligned}
    \end{equation}
    where $\mu_{k_1}^{(2)}=\Delta_2^\nu\left(\frac{(\nu-1)^{\nu-1}}{\nu^\nu}\right)$ with $\nu=\frac{k_1}{2}+1$ and $\Delta_2=6+2|\alpha|^2k_12^{k_1/2-1}\sqrt{2}$. Therefore, the diagonal contribution of $\cL^\dagger(f(N))$ can be controlled by 
    \begin{equation*}
        \begin{aligned}
            ( {C}_{0'}+{C}_4 )(n) &\coloneqq -\biggl(k_2f_1(n_1)(n_2+1)^{k_2/2+1}1_{n_2\geq2}+|z|^21_{n_1\geq1}f(n)+(n_1+1)^{k_1/2+1}f_2(n_2)\biggr)\\
            &\qquad+k_2f_1(n_1)(n_2+1)^{k_2/2}1_{n_2\geq2}\biggl(1_{k_2\geq3}\frac{k_2}{2}+6\biggr)\\
            &\qquad+|z|^21_{n_1\geq1}(n_1+1)^{k_2/2-1}f_2(n_2)\\
            &\qquad+\mu_{k_1}^{(2)}(n_2+1)^{k_2/2}\\
            &\le -\frac{1}{2}\biggl(k_2f_1(n_1)(n_2+1)^{k_2/2+1}+(n_1+1)^{k_1/2+1}f_2(n_2)\biggr)+\Delta_0\\
            &\eqqcolon -x(n)
        \end{aligned}
    \end{equation*}
    where the constant $\Delta_0$ is achieved by splitting off half of the negative leading order terms in order to control the lower order positive contributions (see for example the proof of \Cref{lem:l-diss}).
    Next, we consider operators of the form
    \begin{equation}\label{eq:2x2}
        -x_1\ketbra{e_1}{e_1}-x_2\ketbra{e_2}{e_2}+y\ketbra{e_2}{e_1}+\overline{y}\ketbra{e_1}{e_2}\,,
    \end{equation}
    where $\{e_1, e_2\}$ forms an orthonormal basis of a two-dimensional Hilbert space and $x_1, x_2 \in \R, y \in \C$. The operator in \Cref{eq:2x2} has the eigenvalues
    \begin{equation}
        \lambda_+ = \frac{-x_1 - x_2 + \sqrt{(x_1 - x_2)^2 + 4|y|^2}}{2}, \quad \lambda_- = \frac{-x_1 - x_2 - \sqrt{(x_1 - x_2)^2 + 4|y|^2}}{2}
    \end{equation}
    % Next, we consider operators of the form 
    % \begin{equation}\label{eq:2x2}
    %     -x_1\ketbra{e_{0}}{e_0}-x_2\ketbra{e_{m-1}}{e_{m-1}}+y\ketbra{e_{m-1}}{e_0}+\overline{y}\ketbra{e_0}{e_{m-1}}\,,
    % \end{equation}
    % \pg{Why are we considering this big of a basis? It would suffice to look at $e_0, e_1$ for everything that follows.}
    % where $\{e_j\}_{j=0}^{m-1}$ is an orthonormal basis of an $m$ dimensional Hilbert space and $x_1,x_2\in\R$, $y\in\C$.
    % The operator \ref{eq:2x2} has the following eigenvalues:
    % \begin{center}
    %     \begin{tabular}{ c c }
    %         Eigenvalue & Multiplicity \\[0.5ex]\hline
    %         $0$ & $m-2$ \\  
    %         $\frac{-x_1-x_2+\sqrt{(x_1-x_2)^2+4|y|^2}}{2}$ & $1$\\
    %         $\frac{-x_1-x_2-\sqrt{(x_1-x_2)^2+4|y|^2}}{2}$ & $1$
    %     \end{tabular}.
    % \end{center}
    Moreover, 
    \begin{equation*}
        \frac{-x_1-x_2+\sqrt{(x_1-x_2)^2+4|y|^2}}{2}\leq -\min\{x_1,x_2\}+|y|\,.
    \end{equation*}
    Using this bound, we control each of the off-diagonal operators $\hat{C}_i+\hat{C}_{i'}$, $i\in\{1,2,3\}$, in terms of $\frac{1}{4}X$, where $X=\sum_n x(n) \ketbra{n}{n}$. 
    \begin{equation*}
        \begin{aligned}
            -\frac{1}{4}X+\hat{C}_1+\hat{C}_{1'}&\coloneqq \sum_n-\frac{1}{4}x(n)\ketbra{n}{n}+y_1\ketbra{n}{n_1+1,n_2-2}+\overline{y}_1\ketbra{n_1+1,n_2-2}{n}\\
            &\le \sum_{n|n_2\ge 2}-\frac{1}{8}x(n)\ketbra{n}{n}-\frac{1}{8}x(n_1+1,n_2-2)\ketbra{n_1+1,n_2-2}{n_1+1,n_2-2}\\
            &\quad\qquad + y_1\ketbra{n}{n_1+1,n_2-2}+\overline{y}_1\ketbra{n_1+1,n_2-2}{n}
        \end{aligned}
    \end{equation*}
    with
    \begin{equation*}
        y_1 = y_1(n_1,n_2) = -\frac{{z}}{2}\sqrt{(n_1+1)(n_2-1)n_2}\Bigl(f_1(n_1)g_{2,2}(n_2)+g_{1,1}(n_1+1)f_2(n_2-2)\Bigr)
    \end{equation*}
    so that
    \begin{equation}\label{eqcaseC1C1'}
         -  \frac{1}{4}X+\hat{C}_1+\hat{C}_{1'}\leq \sum_{n|n_2\ge 2}\left(-\min\{x_1,x_2\}+|y_1|\right)(\ketbra{n}{n} + \ketbra{n_1 + 1, n_2 - 2}{n_1 + 1, n_2 - 2})
    \end{equation}
    where $x_1=\frac{1}{8}x(n_1,n_2)$ and $x_2=\frac{1}{8}x(n_1+1,n_2-2)$. Moreover, for $n_2\geq2$
    \begin{equation*}\label{eq-ex:cnot-matrix-bound}
        \begin{aligned}
            |y_1|&\overset{(1)}{\leq} \frac{|z|}{2}\sqrt{(n_1+1)(n_2-1)n_2}\Bigl(f_1(n_1)2k_2(n_2+1)^{k_2/2-1}+k_1(n_1+2)^{k_1/2-1}f_2(n_2-2)\Bigr)\\
            &\leq|z|k_2(n_1+1)^{k_1/2+1/2}(n_2+1)^{k_2/2}+\frac{|z|k_1}{2}\sqrt{(n_2-1)^2+n_2-1}\,(n_1+2)^{k_1/2-1/2}f_2(n_2-2)\\
            &\leq|z|k_2(n_1+1)^{k_1/2+1/2}(n_2+1)^{k_2/2}+|z|k_1\,(n_1+2)^{k_1/2-1/2}(n_2-1)^{k_2/2+1}\,,
        \end{aligned}
    \end{equation*}
    where $(1)$ follows from  \Cref{lem:upper-lower-bound-gl}. At this stage, we consider two cases: 
    
    \medskip
    
    \noindent Case (i): $x_2\ge x_1$. In that case, $-\min\{x_1,x_2\}+|y_1|=-x_1+|y_1|$, and therefore
    \begin{equation*}
        \begin{aligned}
           -\min\{x_1,x_2\}+|y_1|&\leq-\frac{1}{16}\biggl(k_2f_1(n_1)(n_2+1)^{k_2/2+1}+(n_1+1)^{k_1/2+1}f_2(n_2)\biggr)+\frac{1}{8}\Delta_0\\
           &+\underbrace{|z|k_2(n_1+1)^{k_1/2+1/2}(n_2+1)^{k_2/2}}_{=:A_1}+\underbrace{|z|k_1\,(n_1+2)^{k_1/2-1/2}(n_2-1)^{k_2/2+1}}_{=:A_2}\,.
        \end{aligned}
    \end{equation*}
    Note that the first positive non-constant term $A_1$ can be controlled with half the negative contribution in the first term by a constant using the same type of polynomial optimization as in the proof of \Cref{lem:l-diss}. For the last term, i.e. $A_2$, we use the assumption 
    \begin{equation}\label{assumptionequationok}
        |z|k_12^{k_1/2-1/2}\leq |\alpha|k_12^{k_1/2-1/2}\leq\frac{1}{32}k_2,
    \end{equation}
    which allows us to control $A_2$ with the other half of the first term, as we already did with $A_1$. Recall the definition $z=-\frac{\alpha}{2}(1-e^{2i\pi t/T})$. Summarising the above considerations we can conclude the existence of a constant $\tilde{\Delta}'_1$ such that 
    \begin{equation*}
        \begin{aligned}
           -\min\{x_1,x_2\}+|y_1|\leq\tilde{\Delta}'_1\,.
        \end{aligned}
    \end{equation*}
    
    \medskip
    
    \noindent Case (ii): $x_2\leq x_1$. In that case $-\min\{x_1,x_2\}+|y_1|=-x_2+|y_1|$, and therefore
    \begin{equation*}
        \begin{aligned}
           -\min\{x_1,x_2\}+|y_1|&=-\frac{1}{16}\biggl(k_2f_1(n_1+1)(n_2-1)^{k_2/2+1}+(n_1+2)^{k_1/2+1}f_2(n_2-2)\biggr)+\frac{1}{8}\Delta_0\\
           &+|z|k_2(n_1+1)^{k_1/2+1/2}(n_2+1)^{k_2/2}+|z|k_1\,(n_1+2)^{k_1/2-1/2}(n_2-1)^{k_2/2+1}\,.
        \end{aligned}
    \end{equation*}
    To upper bound the above, we use again the assumption \eqref{assumptionequationok}, which implies the existence of a constant $\tilde{\Delta}'_1$ such that
    \begin{equation*}
        \begin{aligned}
           -\min\{x_1,x_2\}+|y_1|&\leq\tilde{\Delta}'_1\,.
        \end{aligned}
    \end{equation*}
    Combining cases (i) and (ii) above, denoting $\Delta_1\coloneqq \max\{\Tilde{\Delta}_1,\Tilde{\Delta}_1'\}$ and plugging the bounds into \eqref{eqcaseC1C1'}, we arrive at
    \begin{align}\label{lastC1}
         -\frac{1}{4}X+\hat{C}_1+\hat{C}_{1'}\le \Delta_1 \sum\limits_{n|n_2 \ge 2}(\ketbra{n}{n} + \ketbra{n_1 + 1, n_2 - 2}{n_1 + 1, n_2 - 2})
    \end{align}
   % \begin{align}\label{lastC1}
   %     -\frac{1}{4}X+\hat{C}_1+\hat{C}_{1'}\le \Delta_1\sum_{n|n_2\ge 2}\sum_{m_1=n_1}^{n_1+1}\sum_{m_2=n_2-2}^{n_2} \ketbra{m_1,m_2}{m_1,m_2}\,.
   %  \end{align}
    Next, we control $-\frac{1}{4}X+\hat{C}_2+\hat{C}_{2'}$. Here, we have
    \begin{equation*}
        y_2=y_2(n_1,n_2)=-\frac{\overline{w}}{2}\sqrt{(n_2-1)n_2}f_1(n_1)g_{2,2}(n_2)\,,
    \end{equation*}
    $x_1=\frac{1}{8}x(n)$\,, and $x_2=\frac{1}{8}x(n_1,n_2-2)$. By \Cref{lem:upper-lower-bound-gl}, we have that
    \begin{align*}
        |y_2|&\le {|{w}|}\sqrt{(n_2-1)n_2}f_1(n_1)k_2(n_2+1)^{k_2/2-1}\,.
    \end{align*}
    Therefore, the negative contribution from $\min\{x_1,x_2\}$ has leading order in both variables $n_1$ and $n_2$, which implies the existence of a constant $\Delta_2$ such that
    \begin{equation*}
        -\min\{x_1,x_2\}+|y_2|\leq\Delta_2\,.
    \end{equation*}
    Hence,
    \begin{align}\label{lastC2}
         -\frac{1}{4}X+\hat{C}_2+\hat{C}_{2'}\le \Delta_2 \sum\limits_{n|n_2 \ge 2}(\ketbra{n}{n} + \ketbra{n_1, n_2 - 2}{n_1, n_2 - 2})
    \end{align}
   % \begin{align}\label{lastC2}
   %      -\frac{1}{4}X+\hat{C}_2+\hat{C}_{2'}\le \Delta_2\,\sum_{n|n_2\ge 2}\sum_{m_2=n_2-2}^{n_2} \ketbra{n_1,m_2}{n_1,m_2}\,.
   % \end{align}
    Finally, we consider $-\frac{1}{4}X+\hat{C}_3+\hat{C}_{3'}$. In this case, 
    \begin{equation*}
        y_3=y_3(n_1,n_2)=-\frac{\overline{w}z}{2}\sqrt{n_1}g_{1,1}(n_1)f_2(n_2)\,,
    \end{equation*}
    $x_1=\frac{1}{8}x(n)$\,, and $x_2=\frac{1}{8}x(n_1-1,n_2)$. Similarly to the above, we can argue the existence of a constant $\Delta_3$ such that
    \begin{equation*}
        -\max\{x_1,x_2\}+|y_3|\leq\Delta_3\,.
    \end{equation*}
    Hence,
    \begin{align}\label{lastC3}
         -\frac{1}{4}X+\hat{C}_3+\hat{C}_{3'}\le \Delta_3 \sum\limits_{n|n_1 \ge 1} (\ketbra{n}{n} + \ketbra{n_1 - 1, n_2}{n_1 - 1, n_2}
    \end{align}
    % \begin{align}\label{lastC3}
    %     -\frac{1}{4}X+\hat{C}_3+\hat{C}_{3'}\le \Delta_3\,\sum_{n|n_1\ge 1}\sum_{m_1=n_1-1}^{n_1}\ketbra{m_1,n_2}{m_1,n_2}\,.
    % \end{align}
    Combining \eqref{lastC1}, \eqref{lastC2} and \eqref{lastC3}, we have shown that
    \begin{align*}
        \cL^\dagger(f(N))&\leq -\frac{X}{4}+\Delta_1\sum_{n|n_2\ge 2}(\ketbra{n}{n} + \ketbra{n_1 + 1, n_2 - 2}{n_1 + 1, n_2 - 2})\\
        &+\Delta_2\sum_{n|n_2\ge 2}(\ketbra{n}{n} + \ketbra{n_1, n_2 - 2}{n_1, n_2 - 2}) \\
        &+ \Delta_3 \sum\limits_{n |n_1 \ge 1} (\ketbra{n}{n} + \ketbra{n_1 - 1, n_2}{n_1 - 1, n_2})\\
        &\le -\frac{X}{4}+ 2(\Delta_1+\Delta_2+\Delta_3)\1\\
        &= -\frac{1}{8}\Big(k_2f_1(N_1)(N_2+1)^{k_2/2+1}+(N_1+1)^{k_1/2+1}f_2(N_2)\Big) +\mu_{\k}\,\1\\
        &\le -\frac{1+k_2}{8}f(N)+\mu_{\k}\1\,,
    \end{align*}
    %\begin{align*}
    %     \cL^\dagger(f(N))&\leq -\frac{X}{4}+\Delta_1\sum_{n|n_2\ge 2}\sum_{m_1=n_1}^{n_1+1}\sum_{m_2=n_2-2}^{n_2}\ketbra{m_1,m_2}{m_1,m_2}\\
    %     &+\Delta_2\sum_{n|n_2\ge 2}\sum_{m_2=n_2-2}^{n_2} \ketbra{n_1,m_2}{n_1,m_2}+\Delta_3 \sum_{n|n_1\ge 1}\sum_{m_1=n_1-1}^{n_1} \ketbra{m_1,n_2}{m_1,n_2}\\
    %     &\le -\frac{X}{4}+ (6\Delta_1+3\Delta_2+2\Delta_3)\1\\
    %     &= -\frac{1}{8}\Big(k_2f_1(N_1)(N_2+1)^{k_2/2+1}+(N_1+1)^{k_1/2+1}f_2(N_2)\Big) +\mu_{\k}\,\1\\
    %     &\le -\frac{1+k_2}{8}f(N)+\mu_{\k}\1\,,
    %\end{align*}
    with
    \begin{equation*}
        \mu_{\k}\coloneqq\frac{\Delta_0}{4}+2(\Delta_1+\Delta_2+\Delta_3)\,.
    \end{equation*}
    % \begin{equation*}
    %     \mu_{\k}\coloneqq\frac{\Delta_0}{4}+6\Delta_1+3\Delta_2+2\Delta_3\,.
    % \end{equation*}
    The claim \eqref{lastclaimsobolevbound} finally follows from \Cref{prop-ex:uniformly-bounded-semigroup}. 
\end{proof}

\section{Perturbation bounds}\label{sec:example-perturbation-bounds}
In this section, we establish a perturbative analysis at any time scale for the semigroups considered in \Cref{sec:examples-sobolev-preserving-semigroup}. In finite dimensions, \cite[Theorem 6]{Szehr_2013} gives a quantitative bound which controls the perturbation of a quantum dynamical semigroup under the condition that the latter converges exponentially fast to a unique invariant state $\tau$: for two generators $\cL$ and $\cL+\cK$, if $\cL$ satisfies $\|e^{t\cL} - \tr(.)\,\tau\|_{1\rightarrow 1} \le c e^{- \omega t}$ for all $t\geq0$ and some $c, \omega > 0$, then
\begin{equation*}
   \forall\rho,\sigma \text{ states},\quad  \norm{e^{t\cL}(\rho) - e^{t(\cL+\cK)}(\sigma)}_1 \le 
    \begin{cases}
        \norm{\rho - \sigma}_1 + t \norm{\cK}_{1\rightarrow 1}\,, &t < \hat{t}\\
        c e^{- \omega t} \norm{\rho - \sigma}_1 + \frac{\log(c) + 1 - c e^{-\omega t}}{\omega} \norm{ \cK}_{1\rightarrow 1}\,, &  t \ge \hat{t}
    \end{cases}
\end{equation*}
where $\hat{t} \coloneqq \frac{\log(c)}{\omega}$. The result can be easily extended to the case of bounded generators in infinite dimensions, although proving the exponential decay for the semigroup generated by $\cL$ is not easy. The situation becomes even trickier in the case of unbounded generators since the use of a Duhamel integral as in the proof in finite dimensions requires a proper justification. It is precisely these issues that we are interested in and want to address here.

\subsection{Gaussian perturbations of the quantum Ornstein Uhlenbeck semigroup}

The quantum Ornstein Uhlenbeck semigroup is well-known to correspond to a so-called beam-splitter channel of exponentially decreasing transmissivity $e^{-(\lambda^2-\mu^2)t}$ with unique Gaussian invariant state (see \cite{DePalma.2018}):
\begin{equation*}
    \sigma\coloneqq \frac{\lambda^2-\mu^2}{\mu^2}\sum_{k=0}^\infty \left(\frac{\mu^2}{\lambda^2}\right)^k\,\ketbra{k}{k}\,.
\end{equation*}
While quantitative statements about the convergence of this semigroup towards $\sigma$ are known \cite{Cipriani.2000,Carbone.2007,Carlen.2017,DePalma.2018}, 
they do not necessarily imply convergence in trace distance in contrast to their finite-dimensional analogues. In contrast, the semigroup is known to contract a certain kind of quantum Wasserstein distance, which we introduce now. First, given a bounded, self-adjoint operator $X\in\cB(\cH)$, we call $X$ a Lipschitz observable if $aX$, $a^\dagger X$ are bounded, and if $Xa$ and $Xa^\dagger$ are closable operators with bounded closures $\overline{Xa}$ and $\overline{Xa^\dagger}$. In this case, we denote by $\partial_a(X)\coloneqq aX-\overline{Xa}$ and $\partial_{a^\dagger}(X)=a^\dagger X-\overline{Xa^\dagger}$. The Lipschitz constant of $X$ is then defined as
\begin{align*}
    \|X\|_{\operatorname{Lip}}\coloneqq \max\big\{\|\partial_a(X)\|_\infty,\,\|\partial_{a^\dagger}(X)\|_\infty\big\}\,.
\end{align*}
We denote the set of Lipschitz observables by $\operatorname{Lip}$. Next,  any $T\in\cT_{1,\operatorname{sa}}$, we denote
\begin{align*}
    \|T\|_{W_1}\coloneqq \sup\,\left\{ \tr\big[X\,T\big]:\,X\in\operatorname{Lip},\,\|X\|_{\operatorname{Lip}}\le 1\right\}\,.
\end{align*}
In \cite[Proposition 6.4]{Gao.2021}, the authors showed that, for any $T\in\cT_{1,\operatorname{sa}}$ and $t>0$,
\begin{align}\label{regularization}
    \|e^{t\cL_{\operatorname{qOU}}}(T)\|_1\le \sqrt{\frac{e^{-(\lambda^2-\mu^2)t}}{1-e^{-(\lambda^2-\mu^2)t}}}\,\Big(\|a\sigma-\sigma a\|_1+\|a^\dagger \sigma-\sigma a^\dagger \|_1\Big)\,\|T\|_{W_1} \,.
\end{align}
Moreover, using the canonical commutation relations, one can also prove the following identities (see e.g.~\cite{Carlen.2017}, or \cite[Proposition 6.2]{Gao.2021}): for any two states $\rho_1,\rho_2\in \cT_{1, \operatorname{sa}}$,  
\begin{equation}\label{eq:qou-exponential-dampening}
    \|e^{t\cL_{\operatorname{qOU}}}(\rho_1-\rho_2)\|_{W_1}\le e^{-\frac{(\lambda^2-\mu^2)t}{2}}\,\|\rho_1-\rho_2\|_{W_1}\,.
\end{equation}
In the next proposition, we use these conditions to find a perturbation bound for any Gaussian perturbation of the quantum Ornstein Uhlenbeck semigroup. 

\begin{prop}\label{propqOUperturb}
    Let $(\cL_{\operatorname{qOU}},\cT_f)$ be the generator of the quantum Ornstein Uhlenbeck semigroup with $\lambda>\mu\geq0$ and $(\varepsilon\cL_G,\cT_f)\coloneqq (\varepsilon\cL[{\gamma a+\eta\ad}],\cT_f)$ a Gaussian perturbation with $\gamma,\eta\in\mathbb{R}$, $\varepsilon>0$. Then, assuming $\lambda^2-\mu^2+|\gamma|^2-|\eta|^2> 0$, $\cL_{\operatorname{qOU}}+\varepsilon\cL_G$ generates a positivity and Sobolev preserving semigroup on $W^{k,1}$ for $k\geq1$, and there exist uniformly bounded functions $C(\varepsilon),D(\varepsilon)$ depending on $\lambda,\mu,|\eta|,|\gamma|$ such that, for all $t\ge 0$ and states $\rho\in W^{2,1}$
    \begin{equation}\label{eq-ex:perturbation-bound-qOU}
        \Big\|\left(e^{t \cL_{\operatorname{qOU}}}-e^{t(\cL_{\operatorname{qOU}}+\varepsilon\cL_G)}\right)(\rho)\Big\|_{1}\leq \varepsilon\, C(\varepsilon)\, \max\Big\{\norm{\rho}_{W^{2, 1}} ,D(\varepsilon)\Big\}\,.
    \end{equation}
\end{prop}
\begin{proof}
    The generation of a Sobolev preserving semigroup was already stated in Lemma \ref{lem-ex:qOU-differential-stability} for $\cL_{\operatorname{qOU}}$ and its proof can easily be extended to $\cL_{\operatorname{qOU}}+\varepsilon \cL_G$. For instance, given a state $\rho\in\cT_f$, one can show that
     \begin{equation*}
         \tr[\cL_{G}(\rho)(N+\1)]\leq -(|\gamma|^2-|\eta|^2)\,\tr[\rho N]+|\eta|^2\,,
     \end{equation*}    
    We have also seen in the proof of \Cref{lem-ex:qOU-differential-stability} that $\tr[\rho \cL_{\operatorname{qOU}}]\le -(\lambda^2-\mu^2)\tr[\rho N]+\mu^2$, so that
     \begin{align*}
        \tr[(\cL_{\operatorname{qOU}}+\varepsilon \cL_G)(\rho)(N+\1)]\le -(\lambda^2-\mu^2+\varepsilon|\gamma|^2-\varepsilon|\eta|^2)\tr[\rho (N+\1)]+\lambda^2+\varepsilon|\gamma|^2\,.
    \end{align*} 
    Therefore, by \Cref{prop-ex:uniformly-bounded-semigroup} we have that, as long as $\lambda^2-\mu^2+\varepsilon|\gamma|^2-\varepsilon|\eta|^2> 0$, for all states $\rho\in W^{2,1}$, $\rho \ge 0$ and all $t\ge 0$,
    \begin{align*}
        \|e^{t(\cL_{\operatorname{qOU}}+\varepsilon \cL_G)}(\rho)\|_{W^{2,1}}\le \max\left\{\norm{\rho}_{W^{2, 1}}, \frac{\lambda^2+\varepsilon|\gamma|^2}{\lambda^2-\mu^2+\varepsilon|\gamma|^2-\varepsilon|\eta|^2}\right\} \,.
    \end{align*}
    Next, for $\rho\in\cT_f$ and $0<u<t $, and denoting $\cL\equiv \cL_{\operatorname{qOU}}$ and $\widetilde{\cL}\equiv \cL_{\operatorname{qOU}}+\varepsilon \cL_G$,
    \begin{align*}
        \Big\|\Big(e^{t\cL}-e^{t\widetilde{\cL}}\Big)(\rho)\Big\|_1\le    \Big\|e^{u\cL}\left(e^{(t-u)\cL}-e^{(t-u)\widetilde{\cL}}\right)(\rho)\Big\|_1 +\Big\| \Big(e^{u\cL}-e^{u\widetilde{\cL}}\Big)e^{(t-u)\widetilde{\cL}}(\rho)\Big\|_1\equiv A+B\,.
    \end{align*}
    We use \Cref{regularization}, so that
    \begin{align*}
    	A&\le c_u \Big\|\Big(e^{(t-u)\cL}-e^{(t-u)\widetilde{\cL}}\Big)(\rho)\Big\|_{W_1}\\
    	&\le c_u\,\varepsilon\, \int_0^{t-u}\,\Big\|e^{s\cL}\cL_G\,e^{(t-u-s)\widetilde{\cL}}(\rho)\Big\|_{W_1}\,ds\\
    	&=  c_u\,\varepsilon\, \int_0^{t-u}\,e^{-\frac{(\lambda^2-\mu^2)s}{2}}\Big\|\cL_G\,e^{(t-u-s)\widetilde{\cL}}(\rho)\Big\|_{W_1}\,ds\,,
    \end{align*}
    where $c_u\coloneqq \sqrt{\frac{e^{-(\lambda^2-\mu^2)u}}{1-e^{-(\lambda^2-\mu^2)u}}}\,\,\Big(\|\partial_a(\sigma)\|_1+\|\partial_{a^\dagger}(\sigma)\|_1\Big)$. Moreover, denoting $\widetilde{\rho}_v\coloneqq e^{v\widetilde{\cL}}(\rho)$ and $b=\gamma a+\eta a^\dagger$, since $\widetilde{\rho}_v\in W^{2,1}$ for all $v\ge 0$,
    \begin{align*}
        \Big\|\cL_G\,\widetilde{\rho}_v\Big\|_{W_1}\,&=\sup_{\|X\|_{\operatorname{Lip}}\le 1}\,\tr[X \cL_G \widetilde{\rho}_v]\\
        &=\frac{1}{2}\,\sup_{\|X\|_{\operatorname{Lip}}\le 1}\, \tr[\partial_{b^\dagger}(X)b \widetilde{\rho}_v-\partial_b(X)\,\widetilde{\rho}_vb^\dagger]\\
        &\le (|\eta|+|\gamma|)\, \Big(\|b\widetilde{\rho}_v\|_1+\|\widetilde{\rho}_vb^\dagger\|_1\Big)\\
        &\le (|\eta|+|\gamma|)\, \Big(\|b\,(N+\1)^{-\frac{1}{2}}\|+\|(N+\1)^{-\frac{1}{2}}\,b^\dagger\|\Big)\,\|\widetilde{\rho}_v\|_{W^{2,1}}\\
        &\le (|\eta|+|\gamma|) \Big(\|b\,(N+\1)^{-\frac{1}{2}}\|+\|(N+\1)^{-\frac{1}{2}}\,b^\dagger\|\Big)\\
        &\qquad \qquad \qquad \cdot \max\left\{\norm{\rho}_{W^{2, 1}}, 
         \frac{\lambda^2+\varepsilon|\gamma|^2}{\lambda^2-\mu^2+\varepsilon|\gamma|^2-\varepsilon|\eta|^2} \right\}  \, .
    \end{align*}
    On the other hand, 
    \begin{align*}
        B&\le \varepsilon\,\int_0^u\,\|e^{s\cL} \cL_Ge^{(t-s)\widetilde{\cL}}(\rho)\|_1\,ds\\
        &\le\,u\varepsilon\,\max_{s\in[0,u]}\, \|\cL_Ge^{(t-s)\widetilde{\cL}}(\rho)\|_1 \\
        &\le u\varepsilon \|\cL_G \circ \mathcal{W}^{-2}\|_{\cT_1\to \cT_1}\,\max_{s\in[0,u]}\,\|e^{(t-s)\widetilde{\cL}}(\rho)\|_{W^{2,1}}\\
        &\le u\varepsilon \|\cL_G \circ \mathcal{W}^{-2}\|_{\cT_1\to \cT_1}\, \max\left\{ \|\rho\|_{W^{2,1}}, 
        \frac{\lambda^2+\varepsilon|\gamma|^2}{\lambda^2-\mu^2+\varepsilon|\gamma|^2-\varepsilon|\eta|^2} \right\}\,.
    \end{align*}
    The result follows from a simple bound on $\|\cL_G\circ \cW^{-2}\|_{\cT_1\to \cT_1}$ with the help of H\"{o}lder's inequality and a standard density argument, and by choosing $u$ appropriately.
\end{proof}

\subsection{Photon-dissipation and CAT qubits}\label{subsec:cat-perturbation}

As mentioned before, one crucial property of the underlying evolution in continuous error correction is that it is exponentially converging to the code-space. In the spirit of \cite{Szehr_2013}, we prove a large time perturbation result for the $l$-photon dissipation perturbed by a Hamiltonian evolution. It is clear that this can be generalized to dissipative perturbations. First, we recall the exponential convergence of the $l$-photon dissipation (\cite[Theorem 2]{Azouit.2016}):
\begin{equation}\label{eq:exponential-convergence}
    \tr[Le^{t\cL_l}(\rho) L^\dagger]\leq e^{-l!t}\tr[L\rho L^\dagger]\,.
\end{equation}
Additionally, it is shown that there is a unique limit $\overline{\rho}$ of $e^{t\cL_l}(\rho)$ for $t\rightarrow\infty$. We show large-time perturbation bounds by combining this bound with our established generation theory for Sobolev and positivity-preserving Markov semigroups. We start with the $l$-photon dissipation perturbed by the Hamiltonian introduced in \Cref{lem:l-diss}, i.e.~$H=p_H(a,\ad)$ with $d_H\leq2(l-1)$.

\begin{thm}\label{thm:l-diss-hamiltonian-perturbation}
    Let $\cL_l$ be the generator of the $l$-photon dissipation and $p_H\in\C[X,Y]$ with $\deg(p_H)=d_H\leq2(l-1)$ such that $H=p_H(a,\ad)$ is a symmetric operator. Then, there exist explicit constants $c,\gamma>0$ such that for $\varepsilon \ge 0$ and all states $\rho\in W^{2(l + d_H + 2),1}$
    \begin{equation*}
        \begin{aligned}
            \Big|\tr[L\left(e^{t\cL_l}(\rho)-e^{t(\cL_l+\varepsilon\cH[H])}(\rho)\right)L^\dagger]\Big|\leq\varepsilon c\left(1-e^{-l!t}\right)\max\{\gamma,\|\rho\|_{W^{2(l + d_H + 2),1}}\}\,.
        \end{aligned}
    \end{equation*}
\end{thm}
\begin{proof}
    The proof consists in applying \Cref{lem:l-diss-hamiltonian} in combination with \Cref{eq:exponential-convergence}. Let $\rho \in \cT_f$ and $\cW^k = (N + \1)^{k/4} \cdot (N + \1)^{k/4}$, then
    \begin{equation*}
        \begin{aligned}
            &\tr[L\biggl(e^{t\cL_l}(\rho)-e^{t(\cL_l+\varepsilon\cH[H])}(\rho)\biggr)L^\dagger]\\
            &\qquad= \varepsilon \int_{0}^t \tr[L e^{s\cL_l}\cW^{-2(l + 2)}\cW^{2(l + 2)}\cH{[H]}e^{(t-s)(\cL_l+\varepsilon\cH[H])}(\rho)L^\dagger] ds\\
            &\qquad\overset{(1)}{\leq}\varepsilon \int_{0}^t\tr[L e^{s\cL_l}\cW^{-2(l + 2)}(\1)L^\dagger] ds \, 2\Lambda d_H^{l + 2} \sqrt{d_H!}\max\left\{\gamma_\varepsilon,\,\|\rho\|_{W^{2(l + d_H + 2),1}}\right\}\\
            &\qquad \overset{(2)}{\leq}\varepsilon \frac{\pi^2}{3}\Lambda d_H^2\sqrt{d_H!}(1+|\alpha|^l\,(l+1)\,\sqrt{l!}+|\alpha|^{2l})\frac{1}{l!}(1-e^{-l!t})\max\left\{\gamma_\varepsilon,\,\|\rho\|_{W^{2(l + d_H + 2),1}}\right\}\\
            &\qquad\eqqcolon\varepsilon c(1-e^{-l!t})\max\left\{\gamma_\varepsilon,\,\|\rho\|_{W^{2(l + d_H + 2),1}}\right\}
        \end{aligned}
    \end{equation*}
    where $\Lambda$ denotes the larges coefficient of $\cH[H]$ in absolute value. Note that the Bochner integral in the calculation is well-defined by the of the boundedness of the integrand w.r.t.~$s$ and the same argumentation as \Cref{thm:semigroup-perturbation}. Besides the boundedness above, the Sobolev preserving property of the involved semigroups imply by construction that the integral is Sobolev preserving. Therefore, the integral commutes with the the map $x\mapsto\tr[LxL^\dagger]$ for $x\in W^{2(l+d_H+2),1}$. In $(1)$ we used that $L e^{s\cL_l}(\cdot) L^\dagger$ preserves positivity and
    %where issues with continuity and permutation of integrals and unbounded operators do not play a role since the semigroups preserve the Sobolev spaces and $\rho \in \cT_f$ (q.f. \Cref{thm:semigroup-perturbation}). In $(1)$ we used that $L e^{s\cL_l}(\cdot) L^\dagger$ preserves positivity and
    \begin{align*}
        \cW^{2(l + 1)} \cH[H] e^{t - s(\cL_l + \varepsilon \cH[H]}(\rho) &\le \norm{\cW^{2(l + 2)} \cH[H] e^{(t - s)(\cL_l + \varepsilon \cH[H]}(\rho)}_\infty \1\\
        &\le \norm{\cW^{2(l + 2)} \cH[H] e^{(t - s)(\cL_l + \varepsilon \cH[H]}(\rho)}_1 \1\\
        &\le 2 \Lambda d_H^{l + 2} \sqrt{d_H!} \norm{e^{(t - s)(\cL_l + \varepsilon \cH[H]}(\rho)}_{W^{2(l + d_H + 2)}} \1\\
        &\le 2 \Lambda d_H^{l + 2} \sqrt{d_H!} \max\{\gamma_\varepsilon, \norm{\rho}_{W^{2(l + d_H + 2)}}\} \1 \, . 
    \end{align*}
    In the above estimation, we used $(N + \1)^{-d_H}$ to control $\cH[H]$. We further used the Sobolev preserving property of the semigroup from \Cref{lem:l-diss-hamiltonian}. Finally, applying the decay (q.v.~\Cref{eq:exponential-convergence}) to $\cW^{-2(l + 1)}(\1) = (N + \1)^{-(l + 2)}$ we estimated in $(2)$:
    \begin{align*}
        \int_{0}^t\tr[L e^{s\cL_l}\cW^{-2(l + 2)}(\1)L^\dagger] ds &\le \int\limits_{0}^t e^{-l! t} \tr[L(N + \1)^{-(l + 2)}L^\dagger]\\
        &\le (1 - e^{-l!t}) (1+|\alpha|^l\,(l+1)\,\sqrt{l!}+|\alpha|^{2l}) \frac{\pi^2}{6}
    \end{align*}
    with the bound 
    \begin{equation*}
        \|L(N+1)^{-l - 2}L^\dagger\|_1\leq (1+|\alpha|^l\,(l+1)\,\sqrt{l!}+|\alpha|^{2l}) \frac{\pi^2}{6}
    \end{equation*}
    that follows from
    \begin{equation}
        \begin{aligned}
            L^\dagger L&=(\ad)^l a^l-\overline{\alpha}^la^l-\alpha^l(\ad)^l+|\alpha|^{2l}\\
            &\leq(N-(l-1)\1)\cdots(N-\1)N+|\alpha|^l\,(l+1)\,\sqrt{(N+l\1)\cdots (N+\1)}+|\alpha|^{2l}\\
            &\leq (N+\1)^l+|\alpha|^l\,(l+1)\,\sqrt{l!}\,(N+\1)^{l/2}+|\alpha|^{2l}\,.
        \end{aligned}
    \end{equation}
    and the $\norm{(N + \1)^{-2}}_1 = \frac{\pi^2}{6}$. This concludes that claim.
    % For that, we show that
    % \begin{equation*}
    %     \|\cW^{\blue{2(l + 1)}}\cH[H]\cW^{-6l+4}\|_{1\rightarrow1}<\infty\,,
    % \end{equation*}
    % where we recall that $\cW^k(\cdot) \coloneqq (N+\1)^{k/4}(\cdot)(N+\1)^{k/4}$. For a state $\rho\in\cT_f$ and $Y\coloneqq\cW^{-6l+4}(\rho)$, we have by Hölder's inequality
    % \begin{equation*}
    %     \begin{aligned}
    %         \|\cW^{2l}\cH[H](Y)\|_{1}&\leq\|(N+\1)^{l/2}H(N+\1)^{-3/2l+1}\rho(N+\1)^{-l+1}\|_1\\
    %         &\quad+\|(N+\1)^{-l+1}\rho(N+\1)^{-3/2l+1}H(N+\1)^{l/2}\|_1\\
    %         &\leq \|\rho\|_1\left(\|(N+\1)^{l/2}H(N+\1)^{-3/2l+1}\|_\infty+\|(N+\1)^{-3/2l+1}H(N+\1)^{l/2}\|_\infty\right)\\
    %         &\leq 2\|\rho\|_1\Lambda d_H^2\sqrt{d_H!}\,,
    %     \end{aligned}
    % \end{equation*}
    % where $\Lambda$ bounds the maximal absolute coefficient of the polynomial. Note that the polynomial is of the form in \Cref{eq:ccr-polynomial-representation}. Similarly, one can show 
    % \begin{equation*}
    %     \begin{aligned}
    %         \|\cH[H](Y))\|_{1}\leq 2\|\rho\|_1\Lambda d_H^2\sqrt{d_H!}
    %     \end{aligned}
    % \end{equation*}
    % where $Y\coloneqq\cW^{-4(l-1)}(\rho)$. Moreover, we established in \Cref{lem:l-diss-hamiltonian} the existence of a constant $\gamma_\varepsilon\geq0$ such that
    % \begin{equation*}
    %     \|e^{t(\cL_l+\varepsilon\cH[H])}(\rho)\|_{W^{k,1}}\leq \max\left\{\gamma_\varepsilon,\,\|\rho\|_{W^{k,1}}\right\}\,,
    % \end{equation*}
    % where
    % \begin{equation*}
    %     \gamma_\varepsilon=\tilde{c}^\nu\left(\frac{(\nu-1)^{\nu-1}}{\nu^\nu}\right)\quad\text{with}\quad \tilde{c}={(l+1)l}+2|\alpha|^lkl^{k/2}\sqrt{l!}+\varepsilon\Lambda(2l)^{\blue{k/2}}\sqrt{(2l)!}\,,\quad\nu=l+\frac{k}{2}-1\,.
    % \end{equation*}
    % Using these two results, we can show that there exists a constant $c_1>0$ such that
    % \begin{equation*}
    %     \begin{aligned}
    %         \|e^{s\cL_l}\cH[H]e^{(t-s)(\cL_l+\varepsilon\cH[H])}(\rho)\|_1&\leq\|\cH[H]\cW^{-4(l-1)}\cW^{4(l-1)}e^{(t-s)(\cL_l+\varepsilon\cH[H])}(\rho)\|_1\\
    %         &\leq c_1\|\cW^{4(l-1)}e^{(t-s)(\cL+\varepsilon\cH[H])}(\rho)\|_1\\
    %         &\leq c_1\max\{\gamma_\varepsilon,\|\rho\|_{W^{4(l-1),1}}\}\,.
    %     \end{aligned}
    % \end{equation*}
    % Next, we bound 
    % \blue{
    % \begin{equation}\label{eq:L-bound}
    %     \begin{aligned}
    %         \|L(N+1)^{-l - 2}L^\dagger\|_1\leq (1+|\alpha|^l\,(l+1)\,\sqrt{l!}+|\alpha|^{2l}) \frac{\pi^2}{6}
    %     \end{aligned}
    % \end{equation}
    % }
    % which follows from
    % \begin{equation*}
    %     \begin{aligned}
    %         L^\dagger L&=(\ad)^l a^l-\overline{\alpha}^la^l-\alpha^l(\ad)^l+|\alpha|^{2l}\\
    %         &\leq(N-(l-1)\1)\cdots(N-\1)N+|\alpha|^l\,(l+1)\,\sqrt{(N+l\1)\cdots (N+\1)}+|\alpha|^{2l}\\
    %         &\leq (N+\1)^l+|\alpha|^l\,(l+1)\,\sqrt{l!}\,(N+\1)^{l/2}+|\alpha|^{2l}\,.
    %     \end{aligned}
    % \end{equation*}
    % \blue{
    % and the fact that $\norm{(N + \1)^{-2}}_1 = \frac{\pi^2}{6}$.
    % }
    % This shows that the bounded vector-valued functions in $s$ considered below are continuous so that the following Bochner integral is well-defined.
    % \begin{equation*}
    %     \begin{aligned}
    %         &\tr[L\biggl(e^{t\cL_l}(\rho)-e^{t(\cL_l+\varepsilon\cH[H])}(\rho)\biggr)L^\dagger]\\
    %         &\qquad=\varepsilon\tr[L\int_{0}^te^{s\cL_l}\cW^{-2(\blue + 1)}\cW^{2(l + 1)}\cH{[H]}e^{(t-s)(\cL_l+\varepsilon\cH[H])}(\rho)dsL^\dagger]\\
    %         &\qquad\overset{(1)}{\leq}\varepsilon\tr[L\int_{0}^te^{s\cL_l}\cW^{\blue{-2(l + 1)}}(\1)dsL^\dagger]2\Lambda d_H^2\sqrt{d_H!}\max\left\{\gamma_\varepsilon,\,\|\rho\|_{W^{\blue{6l},1}}\right\}\\
    %         &\qquad\overset{(2)}{\leq}\varepsilon2\Lambda d_H^2\sqrt{d_H!}(1+|\alpha|^l\,(l+1)\,\sqrt{l!}+|\alpha|^{2l})\frac{1}{l!}(1-e^{-l!t})\max\left\{\gamma_\varepsilon,\,\|\rho\|_{W^{6l-4,1}}\right\}\\
    %         &\qquad\eqqcolon\varepsilon c(1-e^{-l!t})\max\left\{\gamma_\varepsilon,\,\|\rho\|_{W^{\blue{6l},1}}\right\}
    %     \end{aligned}
    % \end{equation*}
    % where in $(1)$ we have used that $e^{s\cL_l}$ and $L\cdot L^\dagger$ are positivity preserving as well as the following bound 
    % \begin{equation*}
    %     \begin{aligned}
    %         \cW^{2l}\cH[H]e^{(t-s)(\cL_l+\varepsilon\cH[H])}(\rho)&\leq\|\cW^{2l}\cH[H]e^{(t-s)(\cL_l+\varepsilon\cH[H])}(\rho)\|_\infty\1\\
    %         &\leq\|\cW^{2l}\cH[H]e^{(t-s)(\cL_l+\varepsilon\cH[H])}(\rho)\|_1\1\\
    %         &\leq 2\Lambda d_H^2\sqrt{d_H!}\max\left\{\gamma_\varepsilon,\,\|\rho\|_{W^{6l-4,1}}\right\}\1
    %     \end{aligned}
    % \end{equation*}
    % holds. In $(2)$, we pull the closed operator $L\cdot L^\dagger$ inside the Bochner integral, used the exponential convergence in \Cref{eq:exponential-convergence}, and finally bounded the remaining term by \Cref{eq:L-bound}. Following the same steps as above with $\cH[H]$ replaced by $-\cH[H]$ leads to the result.
\end{proof}

Special cases of the above result include the $X$ or $Z(\theta)$ gate. 
% The explicit structure and small number $l\in\N$ allow us to improve the perturbation bound in the following way:

\begin{cor}[$Z(\theta)$-gate]
    Let $\cL_2$ be the $2$-photon dissipation. Then, for all $\varepsilon\in[0,1]$ and all states $\rho\in W^{10,1}$
    \begin{equation*}
        \begin{aligned}
            \Big|\tr[L\left(e^{t\cL_2}(\rho)-e^{t(\cL_2+\varepsilon\cH[a+\ad])}(\rho)\right)L^\dagger]\Big|&\leq \varepsilon 2(1+6|\alpha|^2+|\alpha|^{4})(1-e^{-2t})\max\left\{\gamma_\varepsilon,\|\rho\|_{W^{10,1}}\right\}
        \end{aligned}
    \end{equation*}
    where $\gamma_\varepsilon=\frac{1}{25}\left(6+\sqrt{2}\,2^6\,5\,|\alpha|^2+\varepsilon4^5\sqrt{24}\right)^6$\,.
\end{cor}
% \begin{proof}
%     We follow a similar strategy as with the proof in \Cref{thm:l-diss-hamiltonian-perturbation}.
%     As in the proof of \Cref{thm:l-diss-hamiltonian-perturbation}, we first prove the bound 
%     \begin{equation*}
%         \|\cW^4\cH[a+\ad]\cW^{-6}\|_{1\rightarrow1}\leq 16<\infty\,.
%     \end{equation*}
%     Indeed, by the triangle and Hölder's inequalities,
%     \begin{equation*}
%         \begin{aligned}
%             \|\cW^4&\cH[a+\ad]\cW^{-6}(\rho)\|_{1}\\
%             &\leq\|(N+\1)(a+\ad)(N+\1)^{-3/2}\rho(N+\1)^{-1/2}\|_1\\
%             &\quad+\|(N+\1)^{-1/2}\rho(N+\1)^{-3/2}(a+\ad)(N+\1)\|_1\\
%             &\leq 2\|\rho\|_1\left(\|(N+\1)(a+\ad)(N+\1)^{-3/2}\|_\infty+\|(N+\1)^{-3/2}(a+\ad)(N+\1)\|_\infty\right)\\
%             &\leq 16\|\rho\|_1\,,
%         \end{aligned}
%     \end{equation*}
%     which can be seen by applying a vector in the Fock basis representation to calculate the infinity norms. Moreover, applying \Cref{lem:z-theta-energetic-stab} for $k=6$,
%     \begin{equation*}
%         \|e^{t(\cL_2+\varepsilon\cH[a+\ad])}(\rho)\|_{W^{6,1}}\leq\max\left\{\frac{3^7}{4^2}\left(1+\sqrt{2}2^4\,|\alpha|^2+\varepsilon\blue{24}\right)^4,\|\rho\|_{W^{6,1}}\right\}\,.
%     \end{equation*}
%     By following the proof strategy of \Cref{thm:l-diss-hamiltonian-perturbation}, the following Bochner integral is well-defined and
%     \begin{equation*}
%         \begin{aligned}
%             &\tr[L\biggl(e^{t\cL_2}(\rho)-e^{t(\cL_2+\varepsilon\cH[a+\ad])}(\rho)\biggr)L^\dagger]\\
%             &\qquad = \varepsilon\tr[L\int_0^te^{s\cL_2}\cW^{-4}\cW^{4}\cH{[a+\ad]}\cW^{-6} \cW^{6}e^{t(\cL_2+\varepsilon\cH[a+\ad])}(\rho)ds L^\dagger]\\
%             &\red{\qquad\leq\varepsilon\tr[\int_0^tLe^{s\cL_2}\cW^{-4}(\1)L^\dagger ds]16\max\left\{\frac{3^7}{4^2}\left(1+\sqrt{2}2^4\,|\alpha|^2+\varepsilon\blue{24}\right)^4,\|\rho\|_{W^{6,1}}\right\}}\\
%             &\qquad\leq\varepsilon8(1+6|\alpha|^2+|\alpha|^{4})(1-e^{-2t})\max\left\{\frac{3^7}{4^2}\left(1+\sqrt{2}2^4\,|\alpha|^2+\varepsilon\blue{24}\right)^4,\|\rho\|_{W^{6,1}}\right\}
%         \end{aligned}
%     \end{equation*}
%     finishes the proof after repeating the argument for $-\cH[H]$ in order to get a bound on the absolute value of the trace above. 
% \end{proof}

\subsection{Application: entropic and capacity continuity bounds}

Here, we provide one basic application to the perturbation bounds found in this section. We recall the definition of the energy-constrained diamond norm:

\begin{defi}[see \cite{Shirokov.2018,winter2017energy}]
    Given $E\ge 0$ and any two completely positive, trace-preserving maps $\cN,\cM:\cT_1(\cH_m)\to\cT_1(\cH_m)$, their energy constrained diamond norm distance is defined as
    \begin{align}\label{ECnorm}
        \|\cN-\cM\|_\diamond^E\coloneqq \sup_{\cH_r}\,\sup_{\substack{\rho\in \cD(\cH_m\otimes \cH_r)\\\tr\big[\rho_{m}N_m\big]\le E}}\,\big\|(\cN-\cM)\otimes\id_R(\rho)\big\|_1\,,
    \end{align}
    where $N_m\coloneqq \sum_{i=1}^m\,a_i^\dagger a_i$ denotes the total photon number operator, and where the supremum is over all bipartite states $\rho_{mr}$ on $\cH_\otimes \cH_r$ with reduced state $\rho_m$ on $\cH_m$ of average total photon number at most $E$, for some arbitrary separable Hilbert space $\cH_r$. 
\end{defi}

Not that we denote the set of quantum states over a separable Hilbert space as $\cD(\cH) := \{\rho \in \cT_{1, \operatorname{sa}}(\cH) \;:\; \rho \ge 0, \tr[\rho] = 1\}$ in this section.
In words, the energy-constrained diamond norm is a measure of distinguishability between quantum channels with entanglement assistance, and where the input states used for this task are restricted to a physically relevant set of energy-limited states. By the same reasoning as for the usual diamond norm, the supremum in \eqref{ECnorm} can be restricted to $\cH_r\cong \cH_m$, and the optimization can be restricted to pure states on $\cH_m\otimes \cH_r$. Moreover, it turns out that the above definition is equivalent to one where the input state is energy limited on both $m$ and $r$, as introduced in \cite{pirandola2017fundamental}:
\begin{align*}
    \vertiii{\cN-\cM}_\diamond^E\coloneqq \sup_{\substack{\rho\in \cD(\cH_m\otimes \cH_{m'})\\ \tr\big[\rho(N_m+N_{m'})\big]\le E}}\,\big\|(\cN-\cM)\otimes\id_{m'}(\rho)\big\|_1\,,
\end{align*}
for $\cH_{m'}\cong \cH_m$, and where the difference with \eqref{ECnorm} lies in the input states in the above optimization are energy constrained in both their inputs. Clearly, from the definitions we have that 
\begin{align}\label{tripledoubleECequiv}
    \vertiii{\cN-\cM}_\diamond^E\le \|\cN-\cM\|_\diamond^E\le \vertiii{\cN-\cM}_\diamond^{2E}\,.
\end{align}
The second bound above simply results from taking a state $\rho_{mm'}$ for which $\tr\big[\rho_mN_m\big]\le E$ and unitarily rotating the second register so that $\rho_m'=\rho_m$, and therefore $\tr\big[\rho_{m'}N_{m'}\big]\le E$, too.

Similarly, in the next lemma, we establish a straightforward connection between the energy-constrained diamond norm distance between two channels and certain norms between Sobolev spaces:
\begin{lem}\label{lemmaequivECW}
    For any two completely positive, trace preserving maps $\cN,\cM:\cT_1(\cH_1)\to \cT_1(\cH_1)$ and $E\ge 0$,
    \begin{align*}
        \|\cN-\cM\|_{\diamond}^E= (1+E)\,\sup_{\rho\in\cD}\,\frac{\|(\cN-\cM)\otimes \operatorname{id}_{M'}(\rho)\|_1}{\|\rho\|_{W^{(2,0),1}}}\,,
    \end{align*}
    where $\cD\coloneqq \cD(\cH_m\otimes \cH_{m'})$ above. Similar identities hold in multi-mode settings.
\end{lem}
\begin{proof}
    This is direct from
    \begin{align*}
        \|{\cN-\cM}\|_\diamond^{E}&=\sup_{\substack{\rho \in \cD\\\tr[\rho N]\le E}}\,\|(\cN-\cM)\otimes\id(\rho)\|_1\\
        &= \sup_{\substack{\rho \in \cD\\\|\rho\|_{W^{(2,0),1}}\le 1+E}}\,\|(\cN-\cM)\otimes\id(\rho)\|_1\,.
    \end{align*}
\end{proof}

\Cref{lemmaequivECW} can be used in combination with perturbation bounds and entropic continuity bounds like those derived e.g.~in \cite{winter2016tight,winter2017energy,shirokov2017tight,Shirokov.2018} in order to control the deviation of energy-constrained channel capacities in presence of a Lindbladian perturbation. Since these considerations go beyond the scope of the present paper, we do not pursue them further. Instead, we provide a basic illustration of the method we propose in the case of Gaussian perturbations of Gaussian semigroups by combining \Cref{propqOUperturb} with \Cref{lemmaequivECW}.

\begin{cor}\label{corECDN}
    Let $(\cL_{\operatorname{qOU}},\cT_f)$ be the generator of the quantum Ornstein Uhlenbeck semigroup with $\lambda>\mu\geq0$ and $(\varepsilon\cL_G,\cT_f)\coloneqq (\varepsilon\cL[{\gamma a+\eta\ad}],\cT_f)$ a Gaussian perturbation with $\gamma,\eta\in\mathbb{R}$, $\varepsilon>0$. Then, assuming $\lambda^2-\mu^2+|\gamma|^2-|\eta|^2> 0$ as in \Cref{propqOUperturb}, there exist uniformly bounded functions $C(\varepsilon)$, $D(\varepsilon)$ such that, for all $t\ge 0$:
    \begin{equation}
        \Big\|e^{t \cL_{\operatorname{qOU}}}-e^{t(\cL_{\operatorname{qOU}}+\varepsilon\cL_G)}\Big\|_{\diamond}^E\leq\,(1+E) \varepsilon\, C(\varepsilon)\, \max\Big\{1,D(\varepsilon)\Big\}\,.
    \end{equation}
\end{cor}

\section{Discussion and open questions}\label{sec:discussion}
\section{Discussion}
\label{sec: discussion}
\kmsdelete{In this work} We study \kmsreplace{Fairness-Aware PAC learning}{Fair-ERM} in the malicious noise model, and  in some cases allow 
the learner to maintain optimal overall accuracy despite the signal in Group $B$ being almost entirely washed out.
%when we allow learners to use the
%$\PQ$ randomized expansion of the hypothesis class $\mathcal{H}$
In particular we show that different fairness constraints have fundamentally different behavior in the presence of Malicious Noise, in terms of the amount of accuracy loss that a given level of Malicious Noise could cause a fairness-constrained learner to incur. 
The key to achieving our results, which are more optimistic than those in \cite{lampert}, is allowing for improper learners using the (P,Q)-randomized expansions of the given class $\mathcal{H}$.
%We \kmsreplace{present a picture of the}{prove upper and lower bounds on}
%accuracy loss for a range of fairness notions, given \kmsreplace{this simple randomization step.}{learning over $\PQ$.
%In general our results indicate Fair-ERM (given learning over $\PQ$) is more robust than claimed in \cite{lampert}.
The type of smoothness we create by using $\PQ$ seems to be a natural property that is likely shared by many natural hypothesis classes.

Fairness notions are motivated as a response to learned disparities when there is \kmsdelete{data corruption or} systemic error affecting \kmsdelete{the data for}
one group. 
Fairness notions are supposed to mitigate this by ruling out classifiers that have worse performance on a sub-group. 
This can peg both classifiers at a lower level of performance \kmsdelete{(e.g that the lower subgroup)} in order to \emph{motivate} \cite{hardt16} improving the data collection or labelling process to obtain more reliable performance. 
%So in \kmsreplace{some}{a} sense, sensitivity of the fairness notion to poor sub-group performance caused by malicious noise is the \textit{point} of fairness constraints! 
However, it also desirable that fairness constraints perform gracefully when subject to Malicious Noise because fairness constraints will be used in contexts where the data is unreliable and noisy and this might not be known to the learner.
This tension, exposed by our work, motivates 
%a revisiting of fairness notions from first principles approach and trying to axiomatize the 
%desired properties of a fairness intervention a la cryptography and privacy. \footnote{Work in multi-calibration \cite{multicalib} is a viable direction for this problem but it is unclear how 
%that and related notions behave with unreliable data. }
on going work studying the sensitivity level of fairness constraints. 
%If we we are to take a view, if a classifier is deployed 


\medskip

\emph{Acknowledgments:} We would like to thank Yu-Jie Liu for the fruitful discussions which initiated this project. Moreover, we would like to thank Marius Lemm, Angela Capel-Cuevas, Simone Warzel, Michael M. Wolf, Libor Caha, Vjosa Blakaj, Shin Ho Choe, Robert Salzmann, Simon Becker, Lauritz van Luijk, Niklas Galke for the valuable feedback and discussions on the topic. Moreover, we would like to thank Jochen Schmid for his help with questions on evolution systems. T.M.~and C.R.~acknowledge the support of the Munich Center for Quantum Sciences and Technology, C.R.~that of the Humboldt Foundation, and P.G., T.M., and C.R.~that of the Deutsche Forschungsgemeinschaft (DFG, German Research Foundation) – Project-ID 470903074 – TRR 352.

% References
\setlength{\bibitemsep}{0.5ex}
\printbibliography[heading=bibnumbered]
\vspace{2ex}
\addresseshere

\newpage

% Appendix
\appendix
\section{Semigroup perturbation theory}\label{appy:semigroup-perturbation-theory}
Here, we prove \Cref{thm:semigroup-perturbation} stated in the preliminary section. 
\begin{thm}\label{thm-appx:semigroup-perturbation}
    Let $(\cL,\cD(\cL))$ and $(\cL+ \cK,\cD(\cL+\cK))$ be two generators of $C_0$-semigroups on $\cX$, for an operator $(\cK,\cD(\cK))$. Moreover, let $(\cW,\cD(\cW))$ be an invertible operator on $\cX$ with bounded inverse, such that $\cD(\cW)$ is an $\cL+\cK$-admissible subspace {(see \Cref{defi:admissible-spaces})} and such that $\cK\cW^{-1}$ is bounded. Then, for all $x\in\cD(\cW)$,
    \begin{equation*}
        \|(e^{t\cL}-e^{t(\cL+\cK)})x\|_{\cX}\leq t \|\cK\cW^{-1}\|_{\cX\to \cX}\; \int_{0}^1\| e^{(1-s)t\cL}\|_{\cX\to \cX}\;\|e^{st(\cL-\cK)}\|_{\cW\to \cW}\;ds\|x\|_{\cW}.
    \end{equation*}
    and especially
    \begin{equation*}
        (e^{t\cL}-e^{t(\cL+\cK)})x= t \int_{0}^1 e^{(1-s)t\cL} \cK e^{st(\cL-\cK)}x\;ds.
    \end{equation*}
\end{thm}
\begin{proof}
    For simplicity, we assume that $t=1$. To be able to use the integral form, we start by considering the following vector-valued map: 
    \begin{equation*}
        [0,1]\ni s\mapsto e^{(1-s)\cL}\cK e^{s(\cL+\varepsilon\cK)}x.
    \end{equation*}
    It is clear that $s\mapsto  e^{(1-s)\cL}$ is a strongly continuous map of bounded operators. Moreover,
    \begin{align*}
        \|\cK e^{s(\cL+\cK)}x-\cK e^{(s+s')(\cL+\cK)}x\|_{\cX}&\leq\|\cK\cW^{-1}\|_{\cX}\, \|(e^{s(\cL+\cK)}x-e^{(s+s')(\cL+\cK)}x)\|_{\cW}
    \end{align*}
    shows that $s\mapsto\cK e^{s(\cL+\cK)}$ is strongly continuous because $s\mapsto e^{s(\cL+\cK)}$ defines a $C_0$-semigroup on $(\cD(\cW),\|\cdot\|_\cW)$. Then, for every converging sequence $(s_n)_{n\in\N}\rightarrow s$ for $n\rightarrow\infty$ the set
    \begin{equation*}
        \left\{\cK e^{s(\cL+\varepsilon\cK)}x\,|\,n\in\N\right\}
    \end{equation*}
    is relatively compact in $\cX$. Therefore, the strong continuity of $s\mapsto  e^{(1-s)\cL}$ is equivalent to uniform continuity by \cite[Prop.~A3]{Engel.2000}, the considered vector-valued map is continuous and we can use the fundamental theorem of calculus for the generalized Riemann integral so that
    \begin{equation*}
        \| (e^{\cL}-e^{\cL+\cK})x\|_{\cX}\leq\varepsilon\int_0^1\|  e^{(1-s)\cL}\cK e^{s(\cL+\varepsilon\cK)}x\|_{\cX}ds
    \end{equation*}
    proves the theorem.
\end{proof}

\section{Bosonic single mode system}\label{sec-appx:annihilation-creation}
In this section, we prove some basic properties of polynomials of annihilation and creation operators. We shortly repeat the normal form of our polynomials in $a$ and $\ad$ given in \Cref{eq:ccr-polynomial-representation}: 
\begin{equation*}
    p(a\,,\ad)=\sum_{i + 2j \le \deg(p)}\lambda_{ij}(\ad)^iN^j+\,\sum_{k + 2l\le \deg(p)}\mu_{kl}N^la^k
\end{equation*}
with coefficients $\lambda_{ij}, \mu_{kl}\in\C$. The modification considered in our proof (see \Cref{lem:semigroup-of-G-for-positivity}) is given in \Cref{eq:polynomial+number-op} by
\begin{equation*}
    \Tilde{p}(a,\ad)\coloneqq N^{2d}+p(a,\ad)\,.
\end{equation*}
Then, we start proving a simple representation of a domain of $p(a,\ad)$ and $\Tilde{p}(a,\ad)$ which extends the domain $\cH_f$: For $n\in\N$
\begin{equation}\label{eq:domain-poly}
    \begin{aligned}
        p(a\,,\ad)\ket{n}&=\sum_{i+2j\leq d}\lambda_{ij}n^{j}\sqrt{i!\binom{n+i}{i}}\ket{n+i}+\sum_{k+2l\leq d}\mu_{kl}(n-k)^l\sqrt{k!\binom{n}{k}}\ket{n-k}
    \end{aligned}
\end{equation}
where $d\coloneqq\deg(p)$. This directly implies for $|\phi\rangle=\sum_n\phi_n|n\rangle\in\cD(N^{d/2})$
\begin{equation*}
    \begin{aligned}
        p(a\,,\ad)\ket{\phi}&=\sum_{n=0}^\infty\sum_{i+2j\leq d}\phi_n\lambda_{ij}n^{j}\sqrt{i!\binom{n+i}{i}}\ket{n+i}+\sum_{n=0}^\infty\sum_{k+2l\leq d}\phi_n\mu_{kl}(n-k)^l\sqrt{k!\binom{n}{k}}\ket{n-k}\\
        &=\sum_{n=0}^\infty\sum_{i+2j\leq d}\left(\phi_{n-i}\lambda_{ij}(n-i)^{j}\sqrt{i!\binom{n}{i}}+\phi_{n+i}\mu_{ij}n^j\sqrt{i!\binom{n+i}{i}}\right)\ket{n}\,.\\
    \end{aligned}
\end{equation*}
Then, the leading order of the summands in $n$ is maximal of order $d/2$ so that $\cD(N^{d/2})$ is a domain of $p(a,\ad)$. For the modified polynomial $\Tilde{p}$, there is sequence of functions $R_n:\cH \rightarrow\R$ with asymptotics strictly smaller than $n^{4d}$ such that
\begin{equation}\label{eq:leading-order-poly}
    \|\tilde{p}(a,\ad)\ket{\phi}\|^{2}=\sum_{n=0}^\infty |\phi_n|^2n^{4d}+R(\phi)\,.
\end{equation}
Having the domain above in mind, we are able to prove that $p$ is closable and $\tilde{p}$ is a closed operator with core $\cH_f$:

\begin{lem}[Adjoint and core of polynomials of $a, \ad$]\label{lem-appx:formal-polynomial-ccr-adjoint-core}
    Let $p\in\C[X,Y]$ be a polynomial on $\C$ and $(p(a,\ad),\cD(N^{d/2}))$ the unbounded operator in normal form \eqref{eq:domain-ccr-polynomial}. Then, $p(a,\ad)$ is closable and there is a $c\geq0$ such that for all $\ket{\phi}\in\cD(N^{d/2})$
    \begin{equation*}
        \|p(a,\ad)\ket{\phi}\|\leq c\|(\1+N)^{d/2}\ket{\phi}\|\,.
    \end{equation*}
    The modification $\tilde{p}(a,\ad)=N^{2d}+p(a,\ad)$ is closed with $\cD(\Tilde{p}(a,\ad))=\cD(\Tilde{p}(a,\ad)^\dagger)=\cD(N^{2d})$ and core $\cH_f$.
\end{lem}
\begin{proof}
    First, note that the relative boundedness w.r.t.~the number operator is a direct consequence of \Cref{eq:domain-poly}.
    To prove that $p(a,\ad)$ is closable, we show that $\cD(N^{d/2})\subset\cD(p(a,\ad)^\dagger)$: we recall that the adjoint is defined via boundedness of the functional 
    \begin{equation*}
        \cD(p(a,\ad))\ni\ket{\phi}\mapsto\braket{p(a,\ad)\phi\,,\varphi}
    \end{equation*}
    for $\phi\in\cD(p(a,\ad)^\dagger)$. Since for all $n,m\in\N$
    \begin{equation*}
        \braket{a\,n\,,m}=\delta_{n,0}\delta_{n-1,m}\sqrt{n}=\delta_{n,m+1}\sqrt{m+1}=\braket{n\,,\ad\,m}
    \end{equation*}
    and by the definition of the domains, we know that for all $\ket{\phi},\ket{\psi}\in\cD(N^{k+\frac{l}{2}})$ 
    \begin{equation*}
        \braket{N^k(\ad)^l \phi\,,\psi}=\braket{\phi\,,a^lN^k\psi}\quad\text{and}\quad\braket{a^lN^k\phi\,,\psi}=\braket{\phi\,,N^k(\ad)^l\psi}\,.
    \end{equation*}
    By sesquilinearity of the scalar product and \Cref{eq:domain-ccr-polynomial}, the above equations hold for all polynomials $p\in\C[X,Y]$ which proves $\cD(N^{d/2})\subset\cD(p(a,\ad)^\dagger)$. Since $\cD(N^{d/2})$ is a dense subspace of $\cH$, Theorem 7.1.1 in \cite{Simon2015} shows that $p(a,\ad)$ is closable. Next, we show that the modified polynomial 
    \begin{equation*}
        (N+\1)^{2d}+p(a,\ad)
    \end{equation*}
    is already closed. Actually, we show $\cD(N^{2d})=\cD(\Tilde{p}(a,\ad)^\dagger)$, i.e.~the domain is maximal, by contradiction. Assume that there exists a $\varphi\in\cD(\Tilde{p}(a,\ad)^\dagger)\backslash\cD(\Tilde{p}(a,\ad))$ and define $P_M$ to be the projection on the first $M$ Fock basis elements. Then, we use the representation in \Cref{eq:ccr-polynomial-representation} so that, denoting by $\Tilde{p}^\dagger$ the polynomial where we took the complex conjugate of the coefficients and swapped the coordinates,
    \begin{equation*}
        P_M\Tilde{p}^\dagger(a,\ad)=P_M(N+\1)^{2d}P_M+P_{M}\left(\sum_{i + 2j \le d}\overline{\lambda}_{ij}N^ja^iP_{M-i}+\,\overline{\mu}_{ij}(\ad)^iN^jP_{M+i}\right)P_{M+d}\,.
    \end{equation*}
    We then can define the state
    \begin{equation*}
        \ket{\phi_M}\coloneqq\frac{P_M\Tilde{p}^\dagger(a,\ad)\ket{\varphi}}{\|P_M\Tilde{p}^\dagger(a,\ad)\ket{\varphi}\|}\in\cH_f
    \end{equation*}
    Next, we use $\phi_M$ to get a lower bound on the operator norm of 
    \begin{equation*}
        \cD(p(a,\ad))\ni\ket{\phi}\mapsto\braket{\Tilde{p}(a,\ad)\phi\,,\varphi}
    \end{equation*}
    by
    \begin{equation*}
        \begin{aligned}
            \sup_{\|\phi\|=1}|\braket{\Tilde{p}(a,\ad)\phi\,,\varphi}|&\geq|\braket{\Tilde{p}(a,\ad)\phi_M\,,\varphi}|=|\braket{\phi_M\,,P_M\Tilde{p}^\dagger(a,\ad)\varphi}|=\|P_M\Tilde{p}^\dagger(a,\ad)\varphi\|.
        \end{aligned}
    \end{equation*}
    Now, by definition of $\Tilde{p}$ and denoting by $\varphi_n$ the coefficients of $|\varphi\rangle$ in the Fock basis,
    \begin{equation*}
        \|P_M\Tilde{p}^\dagger(a,\ad)\varphi\|^2 =\Big\|\sum_{n=0}^{M+d}\varphi_nP_M\left((N+\1)^{2d}+p^\dagger(a,\ad)\right)\ket{n}\Big\|^2\,,
    \end{equation*}
    where $p^\dagger$ is defined similarly to $\tilde{p}^\dagger$. By assumption $\phi\notin\cD(N^{2d})$, so that the above sequence is diverging for $M\rightarrow\infty$ to infinity (see \Cref{eq:leading-order-poly}). This contradicts the assumption and shows $\cD(\tilde{p}(a,\ad)^\dagger)=\cD(N^{2d})$ as well as $\tilde{p}(a,\ad)^\dagger=\tilde{p}^\dagger(a,\ad)$. Moreover, $p(a,\ad)$ is by Theorem 7.1.1 in \cite{Simon2015} a closed operator. Since $\{\ket{n}\}_{n \in \N}$ is an orthonormal basis and $N$ a multiplication operator on that basis, we can immediately conclude that $\cH_f$ is a core for $N$ and further for all $(\1 + N)^k$, $k \ge 0$. Since $\Tilde{p}(a,\ad)$ is closed w.r.t.~$\cD(N^{p/2+1})$, $\cH_f$ is also a core of $\Tilde{p}(a,\ad)$.
\end{proof}

Having in mind that polynomials of the annihilation and creation are closed operators on certain domains, we use the canonical commutation relation $[a,\ad]=\1$ in the following lemma:
\begin{lem}\label{lem:l-ccr}
    Let $l\in\N$, then the following hold on $\cH_f$ and can be extended to maximal domains by taking limits
    \begin{equation}
        \begin{aligned}
            (\ad)^la^l&=(N-(l-1)\1)(N-(l-2)\1)\cdots(N-\1)N\,,\\
            a^l(\ad)^l&=(N+\1)(N+2\1)\cdots(N+(l-1)\1)(N+l\1)\,.
        \end{aligned}
    \end{equation}
\end{lem}
\begin{proof}
    The above equalities can be proven by induction over $l\in\N$. The cases $l\in\{0,1\}$ are trivial by definition. Next assume that the equation holds for $l\in\N$, then by \Cref{eq:symmetry-function}
    \begin{equation*}
        \begin{aligned}
            (\ad)^{l+1}a^{l+1}&=(\ad)^{l}Na^{l}=(N-l)(\ad)^{l}a^{l}=(N-l)\cdots N
        \end{aligned}
    \end{equation*}
    which finishes the proof by induction. The second expression can be proven by induction as well, and the induction start is again clear by the CCR. Next, we assume the equation for $l\in\N$. Then, \Cref{eq:symmetry-function} shows
    \begin{equation*}
        a^{l+1}(\ad)^{l+1}=a^{l}(N+\1)(\ad)^{l}=a^{l}(\ad)^{l}(N+(l+1)\1)=(N+\1)\cdots(N+(l+1)\1)
    \end{equation*}
    which completes the induction.
\end{proof}

\begin{lem}\label{lem:two-point-hamiltonian-bound}
    Let $\ell_1,\ell_2,k_1,k_2\in\N$ with $\min\{\ell_1,k_1\}=\min\{\ell_2,k_2\}=0$, $z\in\C$, and $h:\mathbb{N}^2\rightarrow\mathbb{R}$ a positive function that is increasing in each of its variables. Then,
    \begin{equation*}
        \begin{aligned}
            &za_1^{\ell_1}a_2^{\ell_2}h(N_1,N_2)(\ad_1)^{k_1}(\ad_2)^{k_2}+\overline{z}a_1^{k_1}a_2^{k_2}h(N_1,N_2)(\ad_1)^{\ell_1}(\ad_2)^{\ell_2}\\
            &\qquad\qquad\qquad\qquad \leq 2|z|\widetilde{h}_{m_1,m_2}(N_1+m_1I,N_2+m_2I)\,,
        \end{aligned}
    \end{equation*}
    where $m_1\coloneqq\max\{\ell_{1},k_{1}\}$, $m_2\coloneqq \max\{\ell_2,k_2\}$ and $$\widetilde{h}_{m_1,m_2}(n_1,n_2)={\prod_{j=n_1-m_1+1}^{n_1} \sqrt{j}\prod_{i=n_2-m_2+1}^{n_2}\sqrt{i}} \,\,\,h(n_1,n_2)\,1_{n_1\ge m_1}1_{n_2\ge m_2}\,,$$
    where we introduced the notation $1_{x\ge m}$ for the indicator function on the set $\{x:\,x\ge m\}$, and where by convention we take $\prod_{i=a}^b=1$ when $a>b$.
\end{lem}
\begin{proof}
    We define $K\coloneqq za_1^{\ell_1}a_2^{\ell_2}h(N_1,N_2)(\ad_1)^{k_1}(\ad_2)^{k_2}+\overline{z}a_1^{k_1}a_2^{k_2}h(N_1,N_2)(\ad_1)^{\ell_1}(\ad_2)^{\ell_2}$ and represent it in the $2$-mode Fock basis:
    \begin{equation*}
        \begin{aligned}
            K&=\sum_{n_1,n_2}\,h(n_1,n_2)\,  \big( z a_1^{\ell_1}a_2^{\ell_2}\,\ketbra{n_1,n_2}{n_1,n_2}(a_1^\dagger)^{k_1}(a_2^\dagger)^{k_2}+\overline{z} a_1^{k_1}a_2^{k_2}\ketbra{n_1,n_2}{n_1,n_2}(a_1^\dagger)^{\ell_1}(a_2^\dagger)^{\ell_2} \big)\\
            &=\sum_{\substack{n_1\geq m_1\\n_2\geq m_2}} g_{\substack{\ell_1,\ell_2\\k_1,k_2}}(n_1,n_2)\\
            &\qquad \qquad \left(z\ketbra{n_1-\ell_1,n_2-\ell_2}{n_1-k_1,n_2-k_2}+\overline{z}\ketbra{n_1-k_1,n_2-k_2}{n_1-\ell_1,n_2-\ell_2}\right)\,, 
        \end{aligned}
    \end{equation*}
    where 
    \begin{equation*}
        \begin{aligned}
            &g_{\substack{\ell_1,\ell_2\\k_1,k_2}}(n_1,n_2)\\
            &\qquad \coloneqq h(n_1,n_2) \sqrt{n_1\dots(n_1-\ell_1+1)n_1\dots (n_1-k_1+1)n_2\dots (n_2-\ell_2+1)n_2\dots (n_2-k_2+1)}\,. 
        \end{aligned}
    \end{equation*}
    By assumption, since $\min\{\ell_1,k_1\}=\min\{\ell_2,k_2\}=0$, we have that $g_{\substack{\ell_1,\ell_2\\k_1,k_2}}(n_1,n_2)=\widetilde{h}_{m_1,m_2}(n_1,n_2)$ for $n_1\ge m_1$, $n_2\ge m_2$, and 

    \begin{align*}
        &K=\sum_{n_1,n_2\in \mathbb{N}}\widetilde{h}_{m_1,m_2}(n_1+m_1,n_2+m_2)\\
        &\qquad \qquad\left(z\ketbra{n_1+k_1,n_2+k_2}{n_1+\ell_1,n_2+\ell_2}+\overline{z}\ketbra{n_1+\ell_1,n_2+\ell_2}{n_1+k_1,n_2+k_2}\right)\,.
    \end{align*}
    Next, we consider the constituents of the above sum individually. Note that the operator
    \begin{equation}\label{eq:block-matrix}
        \begin{aligned}
            z\ketbra{n_1+k_1,n_2+k_2}{n_1+\ell_1,n_2+\ell_2}+\overline{z}\ketbra{n_1+\ell_1,n_2+\ell_2}{n_1+k_1,n_2+k_2}
        \end{aligned}
    \end{equation}
    \begin{equation*}
        \left(\begin{array}{*{6}{c}}
        \tikzmark{left}0 &0 &\ast &0 &0 &0 \\               
        0 &0 &0 &\ast &0 &0 \\      
        \ast &0 &0\tikzmark{right} &0 &\ast &0 \\\DrawBox[thick]
        0 &\ast &0 &\tikzmark{left}0 &0 &\ast \\
        0 &0 &\ast &0 &0 &0 \\
        0 &0 &0 &\ast &0 &0\tikzmark{right} \\\DrawBox[thick]
        \end{array}\right)\,.
    \end{equation*}
    can be embedded into an operator on an two dimensional space of the form 
    \begin{equation*}
        z\ketbra{e_1}{e_2} + \overline{z} \ketbra{e_2}{e_1} \, , 
    \end{equation*}
    where $\ket{e_1}$ and $\ket{e_2}$ are orthonormal vectors. For $\ket{e_1}=\ket{e_2}$, $z+\overline{z}\leq2|z|$ shows 
    \begin{equation*}
         z\ketbra{e_1}{e_2} + \overline{z} \ketbra{e_2}{e_1} \le |z|(\ketbra{e_1}{e_1} + \ketbra{e_2}{e_2})\, . 
    \end{equation*}
    In the case $\ket{e_1}\neq\ket{e_2}$, we have
    \begin{center}
        \begin{tabular}{ c c }
            Eigenvalue & Eigenvectors \\[0.5ex]\hline
            $|z|$ & $\ket{\psi}=\frac{1}{\sqrt{2}|z|}(|z|\ket{e_1}+z\ket{e_{2}})$\\
            $-|z|$ & $\ket{\varphi}=\frac{1}{\sqrt{2}|z|}(|z|\ket{e_1}-z\ket{e_{2}})$
        \end{tabular}.
    \end{center}
    so that 
    \begin{equation*}
        z\ketbra{e_{2}}{e_1}+\overline{z}\ketbra{e_1}{e_{2}}=|z| \ketbra{\psi}{\psi} - |z| \ketbra{\varphi}{\varphi}\leq|z|\ketbra{\psi}{\psi}\leq|z| (\ketbra{e_1}{e_1} + \ketbra{e_2}{e_2})\,.
    \end{equation*}
    This allows us to estimate
    \begin{align*}
        K &\le \sum\limits_{n_1, n_2 \in \N} \widetilde h_{m_1, m_2}(n_1 + m_1, n_2 + m_2) |z| (\ketbra{n_1 + k_1, n_2 + k_2}{n_1 + k_1, n_2 + k_2}\\
        &\hspace{7cm} + \ketbra{n_1 + l_1, n_2 + l_2}{n_1 + l_1, n_2 + l_2})\\
        &\le 2 |z| \widetilde h_{m_1, m_2}(N_1 + m_1 I, N_2 + m2 I)
    \end{align*}
    employing the monotonicity of $\widetilde h_{m_1, m_2}$ in both arguments in the last step.
    %\tm{Next, we split the above sum into sums of matrix blocks as depicted in the following matrix in the Fock basis:
    %\begin{equation*}
    %    \left(\begin{array}{*{6}{c}}
    %    \tikzmark{left}0 &0 &\ast &0 &0 &0 \\               
    %    0 &0 &0 &\ast &0 &0 \\      
    %    \ast &0 &0\tikzmark{right} &0 &\ast &0 \\\DrawBox[thick]
    %    0 &\ast &0 &\tikzmark{left}0 &0 &\ast \\
    %    0 &0 &\ast &0 &0 &0 \\
    %    0 &0 &0 &\ast &0 &0\tikzmark{right} \\\DrawBox[thick]
    %    \end{array}\right)\,.
    %\end{equation*}
    %For fixed $j_{1}, j_2$ and $n_{1},n_2\in\N$, the operator 
    %\begin{equation}\label{eq:block-matrix}
     %   \begin{aligned}
     %       z\ketbra{n_1+k_1,n_2+k_2}{n_1+\ell_1,n_2+\ell_2}+\overline{z}\ketbra{n_1+\ell_1,n_2+\ell_2}{n_1+k_1,n_2+k_2}
     %   \end{aligned}
    %\end{equation}
    %is of the form 
    %\pg{Why aren't we considering just $e_0, e_1$ instead of a $m$-fold basis? The strategy work completely analogous.}
    %\begin{equation}\label{eq:rank2}
    %    z\ketbra{e_{m-1}}{e_0}+\overline{z}\ketbra{e_0}{e_{m-1}}\,,
    %\end{equation}
    %where $\{e_j\}_{j=0}^{m-1}$ is an orthonormal basis of an $m$ dimensional Hilbert space. Then, the above operator has the following eigenvalues and eigenvectors:
    %\begin{center}
    %    \begin{tabular}{ c c c }
    %        Eigenvalue & Eigenvectors & Multiplicity \\[0.5ex]\hline
    %        $0$ & $\{e_1,...,e_{m-2}\}$ & $m-2$ \\  
    %        $|z|$ & $|z|\ket{e_0}+z\ket{e_{m-1}}$ & $1$\\
    %        $-|z|$ & $|z|\ket{e_0}-z\ket{e_{m-1}}$ & $1$
    %    \end{tabular}.
    %\end{center}
    %Defining $m=(m_1+1)(m_2+1)$, the matrix given in \Cref{eq:block-matrix} has the  eigenvalues $\{-|z|,0,|z|\}$, and can therefore be upper bounded as
    %\begin{align*}
    %    z\ketbra{e_{m-1}}{e_0}+\overline{z}\ketbra{e_0}{e_{m-1}}\le |z|\, \sum_{i=0}^{m-1}\ketbra{e_i}{e_i}\,.
    %\end{align*}
    %Since each of the projections upper bounding any operator of the form \eqref{eq:rank2} appears $M$ times in the sum, we find
    %\begin{equation*}
     %   \begin{aligned}
     %       K&\le \sum_{n_1,n_2\in \mathbb{N}}\widetilde{h}_{m_1,m_2}(n_1+m_1,n_2+m_2)|z|\sum_{i=0}^{m_1}\sum_{j=0}^{m_2}\ketbra{n_1+i,n_2+j}{ n_1+i,n_2+j}\\
     %       &\leq (m_1+1)(m_2+1)\,|z|\,\widetilde{h}_{m_1,m_2}(N_1+m_1I,N_2+m_2I)\,,
     %   \end{aligned}
    %\end{equation*}
    %where we also used that $\widetilde{h}$ is increasing in each of its variables.}
\end{proof}

\section{Inequalities for power functions}
Many bounds in Sections \ref{sec:examples-sobolev-preserving-semigroup} and \ref{sec:example-perturbation-bounds} can be deduced from bounds on real-valued functions acting on the spectrum of the number operator $N$. Especially, the following functions, first introduced in \Cref{eq:f-g-l-function}, will require special attention: Let $l,k\in\N$, $f(x)=(x+1)^{k/2} 1_{x\ge -1}$, and 
\begin{equation}\label{eq-appx:f-g-l-function}
    g_l(x) = \begin{cases}
        f(x) - f(x - l) & x \ge l-1;\\
        f(x) & l-1 > x \ge 0;\\
        0 & 0 > x\,.
    \end{cases}
\end{equation}
\begin{lem}\label{appx-lem:monotonicity-g}
    Let $g_l$ be defined in \Cref{eq-appx:f-g-l-function} for $l,k\in\N$. Then, for all $k\geq2$ and $x\in\R$
    \begin{align}
        g_l(x)&\leq g_{l+1}(x)\label{appx-eq:monotonicity-k}\,,\\
        g_l(x-l)&\leq g_l(x)\label{appx-eq:monotonicity-x}\,.
    \end{align}
\end{lem}
\begin{proof}
    By the monotonicity and non-negativity of $f(x)=(x+1)^{k/2} 1_{x\ge -1}$,
    \begin{equation*}
        \begin{rcases}
            x\ge l-1 & f(x)-f(x-l)\\
            l-1>x\geq 0 & f(x)\\
            0>x& 0
        \end{rcases}
        =g_l(x)\leq g_{l+1}(x)=
        \begin{cases}
            f(x)-f(x-(l+1)) & x\ge l \\
            f(x) & l>x\geq 0 \\
            0 & 0>x
        \end{cases}\,,
    \end{equation*}
    which proves Inequality \ref{appx-eq:monotonicity-k}. For Inequality \ref{appx-eq:monotonicity-x}, we consider the following cases:
    \begin{equation*}
        \begin{rcases}
                f(x-l)-f(x-2l)\\
                f(x-l)\\
                0 \\
                0
        \end{rcases}
        =g_l(x-l)\leq g_l(x)=
        \begin{cases}
            f(x)-f(x-l) & \qquad x\geq 2l-1\\
            f(x)-f(x-l) & \qquad 2l-1>x\geq l-1 \\
            f(x) & \qquad l-1>x\geq 0 \\
            0 & \qquad 0>x
        \end{cases}\,.
    \end{equation*}
   For $x<l-1$, the inequalities are clear by the non-negativity of $f$. The case $l-1\leq x<2l-1$ follows by
    \begin{equation*}
        2\frac{f(x-l)}{f(x)}=2\left(1-\frac{l}{x+1}\right)^{k/2}\leq2\left(1-\frac{1}{2}\right)^{k/2}=1
    \end{equation*}
    and the last case $x\geq 2l-1$ follows by monotonicity of $g_l$:
    \begin{equation*}
        \frac{2}{k}g_l'(x-l)=(x-l)^{k/2-1}-(x-2k)^{k/2-1}\geq0\,.\vspace{-2ex}
    \end{equation*}
\end{proof}

Next, we prove upper and lower bounds for $g_l$:
\begin{lem}\label{lem:upper-lower-bound-gl}
     Let $g_l:\R\rightarrow\R_{\geq0}$ be defined in \Cref{eq-appx:f-g-l-function} for $l\in\N$. Then, for all $x\in\R$ and $k\in\N$,
    \begin{equation*}
        \begin{rcases}
            x\geq l-1 & (x+1)^{k/2-1}\frac{kl}{2}-1_{k\geq3}(x+1)^{k/2-2}\,\,\frac{(kl)^2}{8} \\
            x\geq l-1 & (x+1)^{k/2-1}l\\
            l-1>x\geq0&(x+1)^{k/2}\\
            0>x&0
        \end{rcases}
        \leq g_l(x)
    \end{equation*}
    and 
    \begin{equation*}
        g_l(x)\leq 
        \begin{cases}
            \frac{kl}{2}\left(1+1_{k=1}\right)(x+1)^{k/2-1} & x\geq 0\\
            (x+1)^{k/2}&x\geq0 \\
            0 & 0>x
        \end{cases}\,.
    \end{equation*}
\end{lem}
\begin{proof}
    The case $k=0$ is trivial. We start with the upper bounds. By monotonicity of $g_l$, it is enough to prove the first upper bound just for $x\geq l-1$. For $k=1$,
    \begin{equation*}
        g_l(x)=(x+1)^{-1/2}\frac{l}{2}\int_0^1\left(1-s\frac{l}{x+1}\right)^{-1/2}ds\leq(x+1)^{-1/2}\frac{l}{2}\int_0^1\left(1-s\right)^{-1/2}ds=(x+1)^{-1/2}l.
    \end{equation*}
For $k\geq2$,
    \begin{equation*}
        g_l(x)=\frac{k}{2}\int_0^l(x+1-s)^{k/2-1}ds\leq\frac{kl}{2}(x+1)^{k/2-1}\,,
    \end{equation*}
    which finishes the proof of the first upper bound. The other two bounds are obvious by definition. Next, we consider the lower bounds. The case $x<l-1$ is trivial so we are left with proving
    \begin{equation*}
        (x+1)^{k/2-1}\frac{kl}{2}-\delta_{k\geq3}(x+1)^{k/2-2}\frac{(kl)^2}{8}\leq g_l(x)
    \end{equation*}
    for $x\geq l-1$. For $k=1$, the integral representation can be lower bounded as
    \begin{equation*}
        \begin{aligned}
            g_l(x)&=(x+1)^{-1/2}\frac{l}{2}\int_{0}^1\left(1-s\frac{l}{x+1}\right)^{-1/2}ds\geq(x+1)^{-1/2}\frac{l}{2}\,.
        \end{aligned}
    \end{equation*}
    For $k=2$, it is again easy to calculate the quantity $g_l(x)=l$, and for $k=3$
    \begin{equation*}
        \begin{aligned}
            g_l(x)&=(x+1)^{1/2}\frac{3l}{2}\int_{0}^1\left(1-s_1\frac{l}{x+1}\right)^{1/2}ds_1\\
            &=(x+1)^{1/2}\frac{3l}{2}-(x+1)^{-1/2}l^2\frac{3}{4}\iint_{0}^1s_1\left(1-s_1s_2\frac{l}{x+1}\right)^{-1/2}ds_2ds_1\\
            &\geq (x+1)^{1/2}\frac{3l}{2}-(x+1)^{-1/2}l^2\frac{3}{4}\iint_{0}^1s_1\left(1-s_1s_2\right)^{-1/2}ds_2ds_1\\
            &=(x+1)^{1/2}\frac{3l}{2}-(x+1)^{-1/2}\frac{l^2}{2}.
        \end{aligned}
    \end{equation*}
    Finally, the case $k\geq4$ is given by
    \begin{equation*}
        \begin{aligned}
            g_l(x)&=(x+1)^{k/2-1}\frac{kl}{2}\int_{0}^1\left(1-s_1\frac{l}{x+1}\right)^{k/2-1}ds_1\\
            &=(x+1)^{k/2-1}\frac{kl}{2}-(x+1)^{k/2-2}l^2\frac{k(k-2)}{4}\iint_{0}^1s_1\left(1-s_1s_2\frac{l}{x+1}\right)^{k/2-2}ds_2ds_1\\
            &\geq(x+1)^{k/2-1}\frac{kl}{2}-(x+1)^{k/2-2}l^2\frac{k(k-2)}{4}\int_{0}^1s_1ds_1\\
            &\geq(x+1)^{k/2-1}\frac{kl}{2}-(x+1)^{k/2-2}\frac{(kl)^2}{8}\,
        \end{aligned}
    \end{equation*}
    which proves the first non-trivial lower bound for $x\geq l-1$. Next, we consider  
    \begin{equation*}
        g_l(x)\geq(x+1)^{k/2-1}l\,.
    \end{equation*}
    The inequality is obvious for $k<2$ by the same idea as before and for $k\geq2$
    \begin{equation*}
        \begin{aligned}
            g_l(x)&=(x+1)^{k/2-1}\frac{kl}{2}\int_{0}^1\left(1-s_1\frac{l}{x+1}\right)^{k/2-1}ds_1\\
            &\geq(x+1)^{k/2-1}\frac{kl}{2}\int_{0}^1\left(1-s_1\right)^{k/2-1}ds_1\\
            &\geq(x+1)^{k/2-1}l
        \end{aligned}
    \end{equation*}
which ends the proof. 
\end{proof}

\begin{lem}\label{lem:bounds-ccr-l-product}
    Let $l\in\N$ and $x\geq l$, then
    \begin{equation*}
        \begin{aligned}
            (x+1)^l-\frac{(l+1)l}{2}(x+1)^{l-1}&\leq&((x+1)-l)\cdots ((x+1)-1)&\leq&(x+1)^l\\
            (x+1)^l&\leq&(x+1)\cdots (x+1+(l-1)&\leq&l!(x+1)^l
        \end{aligned}
    \end{equation*}
\end{lem}
\begin{proof}
    To prove \Cref{lem:bounds-ccr-l-product}, we redefine $y=x+1$ and rewrite the first product as
    \begin{equation*}
        p_l(y)\coloneqq(y-l)\cdots (y-1)\eqqcolon y^l-\frac{(l+1)l}{2}y^{l-1}+r_{l-2}(y)
    \end{equation*}
    where $r_{l-2}$ is a polynomial of degree $l-2$. The proof idea is to show that $r_{l-2}(y)$ is non-negative for all $y\geq l+1$, which  proves the inequality. The non-negativity of the polynomial $r_{l-2}$ can be proven by induction over $l$: The statement is directly clear for $l=1$ and $l=2$. Next, we assume that $r_{l-2}$ is non-negative for all $x\geq l+1$ and show that $r_{l-1}$ is for all $x\geq l+2$. 
    \begin{equation*}
        \begin{aligned}
            p_{l+1}(y)&=(y-(l+1))p_l(y)\\
            &=(y-(l+1))\left(y^l-\frac{(l+1)l}{2}y^{l-1}+r_{l-2}(y)\right)\\
            &=y^{l+1}-\frac{(l+1)(l+2)}{2}y^{l}+\frac{(l+1)^2l}{2}y^{l-1}+(y-(l+1))r_{l-2}(y)\\
            &=y^{l+1}-\frac{(l+1)(l+2)}{2}y^{l}+r_{l-1}(y).
        \end{aligned}
    \end{equation*}
    For the second product $(x+1)\cdots ((x+1)+l-1)$ the lower bound is clear and the upper bound follows by 
    \begin{equation*}
        (x+1)(x+2)\cdots ((x+1)+l-1)=(l-1)!\left(\frac{x}{1}+1\right)\cdots \left(\frac{x}{l}+1\right)\leq l!(x+1)^l.
    \end{equation*}
\end{proof}

\section{Technical lemmas for the quantum Sobolev spaces}
\begin{lem}[Continuity of $G(z)$]\label{lem:continuity-G}
    Let $k_0 < k_1 \in \R_+$ and $T: W^{k_j, 1} \to W^{k_j, 1}$, be a linear map with $\norm{T}_{W^{k_j, 1} \to W^{k_j, 1}} \le M_j$, bounded by $M_j \ge 0$ for $j = 1, 2$ respectively. Further let $\theta \in [0, 1]$ and $k_\theta = (1-\theta) k_0 +  \theta k_1$ and $x \in \cT_f$, then the map
    \begin{align*}
        G: S &\coloneqq \{ z \in \C \::\; 0 \le \Re(z) \le 1\} \to \cT_{1, \operatorname{sa}} \\
        z & \mapsto G(z) = (N + \1)^{\frac{k(z)}{4}} T\Big((N + \1)^{\frac{k_\theta - k(z)}{4}} x (N + \1)^{\frac{k_\theta - k(z)}{4}}\Big) (N + \1)^{\frac{k(z)}{4}}
    \end{align*}
    with $k(z) = (1-z) k_0 + z k_1$, is well-defined, uniformly bounded and continuous.
\end{lem}
\begin{proof}
   In order to prove the claim, we decompose $G$ using the following auxiliary functions:
    \begin{equation}\label{eq:G_1}
        \begin{aligned}
             G_1: S \times W^{k_1, 1} &\to \cT_{1, \operatorname{sa}}\\
            (z, y) &\mapsto (N + \1)^{\frac{k(z)}{4}} y (N + \1)^{\frac{k(z)}{4}}
        \end{aligned}
    \end{equation}
    and 
    \begin{equation}\label{eq:G_2}
        \begin{aligned}
               G_2: S &\to \cT_f \subset W^{k_1, 1}\\
               z &\mapsto (N + \1)^{\frac{k_\theta - k(z)}{4}} x (N + \1)^{\frac{k_\theta - k(z)}{4}} \, . 
        \end{aligned}
    \end{equation}
    We clearly have that $G_1(z, \cdot):W^{k_1, 1} \to \cT_{1, \operatorname{sa}}$ is a bounded linear map for all $z \in S$, since
    \begin{equation}\label{eq:bound-G_1}
        \norm{G_1(z, y)}_1 = \norm{(N + \1)^{\frac{\Re(k(z))}{4}} y (N + \1)^{\frac{\Re(k(z))}{4}}}_1 \le \norm{(N + \1)^{\frac{k_1}{4}} y (N + \1)^{\frac{k_1}{4}}}_1 = \norm{y}_{W^{k_1, 1}}
    \end{equation}
    where we used that $k_0 \le \Re(k(z)) \le k_1$ and $(N + \1)^{i\frac{\Im(k(z))}{4}}$ is a unitary that can be absorbed into the norm. Next, we will show that $G_1(\cdot, y): S \to \cT_{1, \operatorname{sa}}$ is continuous for all $y \in W^{k_1, 1}$. For that first note that, for $y \in \cT_f$, the claim follows directly from the continuity of $z \mapsto (n + 1)^{\frac{k(z)}{4}}$ with $n \in \N$ as a map from $S$ to $\C$. This is because all the involved operators can be considered finite dimensional using a cut-off of the Fock-basis. For a general $y \in W^{k_1, 1}$, we find $(y_n)_{n \in \N} \subset \cT_f$, s.t. $y_n \to y$ in $W^{k_1, 1}$, hence for all $n \in \N$
    \begin{align*}
        \lim\limits_{z \to z_0} &\norm{G_1(z, y) - G_1(z_0, y)}_1 \\
        &\le \lim\limits_{z \to z_0} \norm{G_1(z, y - y_n)}_1 + \norm{G_1(z, y_n) - G_1(z_0, y_n)}_1 + \norm{G_1(z_0, y_n - y)}_1\\
        &\le \lim\limits_{z \to z_0} \norm{G_1(z, y_n) - G_1(z_0, y_n)}_1 + 2 \norm{y - y_n}_{W^{k_1, 1}}\\
        &\le 2 \norm{y - y_n}_{W^{k_1, 1}} \, , 
    \end{align*}
    where we used \Cref{eq:bound-G_1}. Taking the limit $n \to \infty$ concludes the claim that $G_1(\cdot, y): S \to \cT_{1, \operatorname{sa}}$ is continuous for all $y \in W^{k_1, 1}$. We further have that $G_2$ as a map from $S$ to $W^{k_1, 1}$ is continuous, since $x \in \cT_f$ and the maps $z \mapsto (n + 1)^{\frac{k(z)}{4}}$ for $n \in \N$ are continuous as maps $S \to \C$. This suffices since $x \in \cT_f$, hence all involved operators can be made finite dimensional via a cut-off in the Fock-basis again.\par 
    We can now write 
    \begin{equation*}
        G(z) = G_1(z, T(G_2(z)))
    \end{equation*}
    where $T(G_2(z)) \in W^{k_1, 1}$ as $T:W^{k_1, 1} \to W^{k_1, 1}$ and $G_2(z) \in \cT_f \subset W^{k_1, 1}$ for all $z \in S$. This not only gives us that $G$ is well-defined but also allows us to get
    \begin{align*}
        \norm{G(z)}_1 &= \norm{G_1\left(z, T(G_2(z))\right)}_1 \\
        &\le \norm{T(G_2(z))}_{W^{k_1, 1}} \\
        &\le \norm{T}_{W^{k_1, 1} \to W^{k_1, 1}} \norm{G_2(z)}_{W^{k_1, 1}}\\
        &\le \norm{T}_{W^{k_1, 1} \to W^{k_1, 1}} \norm{(N + \1)^{\frac{k_\theta - k_0}{4}} x (N + \1)^{\frac{k_\theta - k_0}{4}}}_{W^{k_1, 1}}
    \end{align*}
    where we again used \Cref{eq:bound-G_1}, giving us a bound independent of $z$. Further, using again \Cref{eq:bound-G_1} we can conclude continuity, since 
    \begin{align*}
        \lim\limits_{z \to z_0} \norm{G(z) - G(z_0)}_1 &\le \lim\limits_{z \to z_0} \norm{G_1\left(z, T\left\{G_2(z) - G_2(z_0)\right\}\right)}_1 \\
        &\hspace{2cm} + \lim\limits_{z \to z_0}\norm{G_1\left(z, T\left(G_2(z_0)\right)\right) - G_1\left(z_0, T\left(G_2(z_0)\right)\right)}_1\\
        &\le \lim\limits_{z \to z_0}\norm{T}_{W^{k_1, 1} \to W^{k_1, 1}} \norm{G_2(z) - G_2(z_0)}_{W^{k_1, 1}} \\
        & \hspace{2cm} + \lim\limits_{z \to z_0} \norm{G_1\left(z, T\left(G_2(z_0)\right)\right) - G_1\left(z_0, T\left(G_2(z_0)\right)\right)}_1\\
        &= 0
    \end{align*}
    where in addition we used the continuity of $G_1(\cdot, y): S \to \cT_{1, \operatorname{sa}}$ for $y \in W^{k_1, 1}$ and $G_2:S \mapsto W^{k_1, 1}$.
\end{proof}

\begin{lem}[Differentiability of G(z)]\label{lem:differentiability-G}
    Let $$k_0 < k_1 \in \R_+$$, $T: W^{k_j, 1} \to W^{k_j, 1}$, be a linear map with $\norm{T}_{W^{k_j, 1} \to W^{k_j, 1}} \le M_j$, bounded by $M_j \ge 0$ for $j = 1, 2$ respectively. Further let $\theta \in [0, 1]$ and $k_\theta = (1-\theta) k_0 +  \theta k_1$ and $x \in \cT_f$, then the map
    \begin{align*}
        G: S &\coloneqq \{ z \in \C \::\; 0 \le \Re(z) \le 1\} \to \cT_{1, \operatorname{sa}} \\
        z & \mapsto G(z) = (N + \1)^{\frac{k(z)}{4}} T\left((N + \1)^{\frac{k_\theta - k(z)}{4}} x (N + \1)^{\frac{k_\theta - k(z)}{4}}\right) (N + \1)^{\frac{k(z)}{4}}
    \end{align*}
    with $k(z) = (1-z) k_0 + z k_1$, is holomorphic on $\mathring{S} \coloneqq \{z \in \C \; : \; 0 < \Re(z) < 1\}$.
\end{lem}
\begin{proof}
    To prove the claim, we follow a similar strategy as with \Cref{lem:continuity-G}. We will again use the auxiliary functions \Cref{eq:G_1} and \Cref{eq:G_2}. We begin by showing that for a fixed $y \in W^{k, 1}$, $G_1(\cdot, y):S \to \cT_{1, \operatorname{sa}}$ is holomorphic on $\mathring{S}$ and initially even simplify to the case $y \in \cT_f$. In this setting, all operators involved can be assumed to be linear maps on a finite subspace, by just taking a cut-off in the Fock-basis as we did before. This allows us to Taylor expand around $z_0 \in \mathring{S}$
    \begin{align*}
        (N + \1)^{\frac{k(z)}{4}} y (N + \1)^{\frac{k(z)}{4}} = G_1(z_0, y) +  G_1'(z_0, y) (z - z_0) + \int\limits_{[z_0, z]} G_1''(\omega, y)(\omega - z_0) \,  d\omega
    \end{align*}
    where the integral is a path integral along the line segment $[z_0, z]$ and
    \begin{align*}
         G_1'(z_0, y) &= \frac{k_0 - k_1}{4}\, \left(\log(N + \1) (N + \1)^{\frac{k(z_0)}{4}} y (N + \1)^{\frac{k(z_0)}{4}}\right. \\
        &\hspace{3cm} \left. + (N + \1)^{\frac{k(z_0)}{4}} y (N + \1)^{\frac{k(z_0)}{4}} \log(N + \1)\right)
    \end{align*}
    and 
    \begin{align*}
        G_1''(\omega, y) &= \left(\frac{k_0 - k_1}{4}\right)^2\left(\log^2(N + \1)(N + \1)^{\frac{k(\omega)}{4}} y (N + \1)^{\frac{k(\omega)}{4}}\right.\\
        &\hspace{3cm} + 2 \log(N + \1)(N + \1)^{\frac{k(\omega)}{4}} y (N + \1)^{\frac{k(\omega)}{4}} \log(N + \1) \\
        &\hspace{6cm} +\left. (N + \1)^{\frac{k(\omega)}{4}} y (N + \1)^{\frac{k(\omega)}{4}} \log^2(N + \1)\right)
    \end{align*}
    are linear in $y$. From this representation, we can immediately deduce holomorphy of $G_1(\cdot, y):S \to \cT_{1, \operatorname{sa}}$ at $z_0 \in \mathring{S}$ and hence on all of $\mathring{S}$. To lift holomorphy from $y \in \cT_f$ to $y \in W^{k_1, 1}$, we note that for $z_0 \in \mathring{S}$ there exists $C_{z_0} \ge 0$ such that for $y \in \cT_f$
    \begin{equation}\label{eq:bound-G'1}
        \norm{G'(z_0, y)}_1 \le C_{z_0} \norm{y}_{W^{k_1, 1}}
    \end{equation}
    and further for $\omega \in B_\varepsilon(z_0) \coloneqq \{z \in \C \;:\; |z - z_0| < \varepsilon\} \subset \mathring{S}$ there exists $C_{\varepsilon, z_0} \ge 0$ such that
    \begin{equation}\label{eq:bound-G''1}
        \norm{G''(\omega, y)}_1 \le C_{\varepsilon, z_0} \norm{y}_{W^{k_1, 1}} \, . 
    \end{equation}
    We will only show that given $\omega$ as above, 
    \begin{equation}\label{eq:boundedness-subterms}
        \norm{\log^2(N + \1) (N + \1)^{\frac{k(\omega)}{4}} y (N + \1)^{\frac{k(\omega)}{4}} y (N + \1)^{\frac{k(\omega)}{4}}}_1 \le \tilde{C}_{\varepsilon, z_0} \norm{y}_{W^{k_1, 1}} \, . 
    \end{equation}
    Using the same reasoning for the other terms of \Cref{eq:bound-G'1} and \Cref{eq:bound-G''1} in combination with triangle inequality immediately gives the claims. Note first that we can reduce $k(\omega)$ to its real part since the imaginary part only produces a unitary $(N + \1)^{i \Im(k(\omega))}$ that can be absorbed into the norm. We call the real part $r(\omega)$ for now. Since $\omega \in B_\varepsilon(z_0) \subset \mathring{S}$ we find a $\delta_\varepsilon > 0$ independent of $\omega$, such that $|r(\omega) - k_1| < \delta_\varepsilon$ or more precisely $r(\omega) - k_1 \le - \delta_\varepsilon$. Hence using Hölder's inequality, we can deduce
    \begin{align*}
         &\norm{\log^2(N + \1) (N + \1)^{\frac{k(\omega)}{4}} y (N + \1)^{\frac{k(\omega)}{4}} y (N + \1)^{\frac{k(\omega)}{4}}}_1\\
         &\le \norm{\log^2(N + \1) (N + \1)^{\frac{r(\omega) - k_1}{4}}}_\infty \norm{(N + \1)^{\frac{r(\omega) - k_1}{4}}}_\infty \norm{y}_{W^{k_1, 1}}\\
         &\le \norm{\log^2(N + \1) (N + \1)^{-\frac{\delta_\varepsilon}{4}}}_\infty \norm{(N + \1)^{-\frac{\delta_\varepsilon}{4}}}_\infty \norm{y}_{W^{k_1, 1}}\\
         &\le \norm{\log^2(N + \1) (N + \1)^{-\frac{\delta_\varepsilon}{4}}}_\infty \norm{y}_{W^{k_1, 1}}
    \end{align*}
    where we used that $x \mapsto e^{k x}$ for $k \ge 0$ is monotone and further that $(N + \1)^{-\frac{\delta_\varepsilon}{4}}$ is a contraction. Lastly, we have that $x \mapsto \frac{\log^2(x + 1)}{(x + 1)^{\frac{\delta_\varepsilon}{4}}}$ is a bounded function for $x \ge 0$ with a bound we call $\tilde{C}_{\delta_\varepsilon}$. This allows us to estimate $\norm{\log^2(N + \1) (N + \1)^{-\frac{\delta_\varepsilon}{4}}}_\infty \le \tilde C_{\delta_\varepsilon}$, which concludes \Cref{eq:boundedness-subterms} and therefore also \Cref{eq:bound-G'1} and \Cref{eq:bound-G''1}.\par 
    For a general $y \in W^{k_1, 1}$ and $z_0 \in \mathring{S}$ \Cref{eq:bound-G'1} allows us to conclude that $G'_1(z_0, y) \in \cT_{1, \operatorname{sa}}$ is well defined. Further, for $z \in B_\varepsilon(z_0)$ and $(y_n)_{n \in \N} \subset \cT_f$ with $y_n \to y$ in $W^{k_1, 1}$, we have for all $n \in \N$
    \begin{equation}
        \begin{aligned}
            \norm{\frac{G_1(z, y_n) - G_1(z_0, y_n)}{z - z_0} - G_1'(z_0, y_n)}_1 &\le \frac{1}{|z - z_0|} \int\limits_{[z_0, z]} \norm{G''_1(\omega, y_n)}_1 |(\omega - z_0) d\omega| \\
            &\le C_{\varepsilon, z_0} |z - z_0| \norm{y_n}_{W^{k_1, 1}}
        \end{aligned}
    \end{equation}
    where we used the expansion and \Cref{eq:bound-G''1}. Now we can take the limit $n \to \infty$ on both sides, as all objects involved are stable w.r.t. that limit (using \Cref{lem:continuity-G} and \Cref{eq:bound-G'1}). We get
    \begin{equation}
        \norm{\frac{G_1(z, y) - G_1(z_0, y)}{z - z_0} - G_1'(z_0, y)}_1 \le  C_{\varepsilon, z_0} |z - z_0| \norm{y}_{W^{k_1, 1}}
    \end{equation}
    which immediately lets us deduce holomorphy of $G_1(\cdot, y):S \to \cT_{1, \operatorname{sa}}$ on $\mathring{S}$ for $y \in W^{k_1, 1}$.\par
    For $G_2:S \to W^{k_1, 1}$ the holomorphy immediately follows from the fact that $x \in \cT_f$, which again allows reducing the analysis to a finite-dimensional subspace by taking a cut-off in the Fock basis again. Lastly, we have that $T(G_2(z)) \in W^{k_1, 1}$ for all $z \in S$, which finally gives us that for $z_0 \in \mathring{S}$ and for $z \in B_\varepsilon(z_0) \subset \mathring{S}$
    \begin{align*}
        &\norm{\frac{G(z) - G(z_0)}{z - z_0} - (G_1'(z_0, T(G_2(z_0))) + G_1(z_0, T\{G_2'(z_0)\})} \\
        &\le \norm{\frac{G_1(z, T(G_2(z_0))) - G_1(z_0, T(G_2(z_0)))}{z - z_0} - G_1'(z_0, T(G_2(z_0)))}_1 \\
        &\hspace{1cm} + \norm{T}_{W^{k_1, 1} \to W^{k_1, 1}}\norm{\frac{G_2(z) - G_2(z_0)}{z - z_0} - G'_2(z_0)}_{W^{k_1, 1}}
    \end{align*}
    where we used linearity of $G_1(z, \cdot)$, $G_1'(z, \cdot)$ and $T$. In addition, we used the bound on $G_1(z, \cdot)$ from \Cref{eq:bound-G_1} and $G'_2$ to denote the derivative of $G_2$. Now the differentiability of $G_1(\cdot, y)$ and $G_2$ at $z_0$ immediately gives the differentiability of $G$ at $z_0$, which concludes the proof as $z_0 \in \mathring{S}$ was arbitrary.
\end{proof}

\section{Technical lemmas for the generation theorem}
\begin{lem}\label{lem:(n+1)-(n+1)-properties}
    For $d \ge 0$ and $\varepsilon > 0$, define the operator 
    \begin{equation*}
        \cI_{d, \varepsilon}: \cT_f \to \cT_f, \qquad x \mapsto \cI_{d, \varepsilon}(x) \coloneqq - \varepsilon\{(N + \1)^{4d}, x\} \, . 
    \end{equation*}
    For $\lambda \ge 0$, we have that $\lambda - \cI_{d, \varepsilon}:\cT_f \to \cT_f$ is bijective. While for all $k \in \R_+$ and $x \in \cT_f$ one further has
    \begin{align}
        & \norm{(\lambda - \cI_{d, \varepsilon})^{-1}(x)}_{W^{k, 1} } \le \frac{1}{\lambda + 2 \varepsilon} \norm{x}_{W^{k, 1}}\,;\tag{1}\label{item:(1)}\\
       &  \norm{\cI_{d, \varepsilon} \circ (\lambda - \cI_{d, \varepsilon})^{-1}(x)}_{W^{k, 1}} \le {2}\norm{x}_{W^{k, 1}} \tag{2}\label{eq:inequality-resolvent-(n+1)4d}\, . 
    \end{align}
\end{lem}
\begin{proof}
    For $\lambda \ge 0$ define the following linear operator
    \begin{align*}
        (\lambda - \cI_{d, \varepsilon})^{-1}:\cT_f &\to \cT_f,\\
        x = \sum\limits_{\text{finite}} x_{nm} \ketbra{n}{m} &\mapsto (\lambda - \cI_{d, \varepsilon})^{-1}(x) \coloneqq \sum\limits_{\text{finite}} x_{nm} \frac{1}{\varepsilon(n + 1)^{4d} + \varepsilon(m + 1)^{4d} + \lambda} \ketbra{n}{m} \, .
    \end{align*}
    or alternatively 
    \begin{align*}
        (\lambda - \cI_{d, \varepsilon})^{-1}(x) &= \int\limits_{0}^\infty e^{-(\varepsilon(N + \1)^{4d} + \lambda/2) s} x e^{-(\varepsilon(N + \1)^{4d} + \lambda/2) s} ds\\
        &=  \int\limits_{0}^\infty  \sum\limits_{\text{finite}} e^{-(\varepsilon(n + \1)^{4d} + \lambda/2)s} x_{nm} e^{-(\varepsilon(m + \1)^{4d} + \lambda/2)s} \ketbra{n}{m} ds \, .
    \end{align*}
    The integral representation allows us to deduce that $(\lambda - \cI_{d, \varepsilon})^{-1}$ preserves positivity. Using the first expression it is straightforward to show that this map is indeed the inverse to $\lambda - \cI_{d, \varepsilon}:\cT_f \to \cT_f$. The bound $\norm{(\lambda - \cI_{d, \varepsilon})^{-1} x}_{W^{k, 1} } \le \frac{1}{\lambda + 2 \varepsilon} \norm{x}_{W^{k, 1}}$ can be shown, using the integral representation and Hölder inequality:
    \begin{align*}
        \norm{(\lambda - \cI_{d, \varepsilon})^{-1}(x)}_{W^{k, 1}} &\le \int\limits_{0}^\infty \norm{e^{-\varepsilon(N + \1)^{4d} + \lambda/2)s} (N + \1)^{k/4} x (N + \1)^{k/4} e^{-\varepsilon(N + \1)^{4d} + \lambda/2)s}}_1\\
        &\le \int\limits_{0}^\infty \norm{e^{-\varepsilon(N + \1)^{4d} + \lambda/2)s}}_\infty^2 ds \norm{x}_{W^{k, 1}}\\
        &= \int\limits_{0}^\infty e^{-(2\varepsilon + \lambda)s} \norm{x}_{W^{k, 1}} = \frac{1}{\lambda + 2\varepsilon} \norm{x}_{W^{k, 1}} \, .
    \end{align*}
    Issues arising from the unbounded nature of $N$ can be ignored in the above estimations, as we can take a finite cut-off in the Fock basis due to $x \in \cT_f$. For \eqref{eq:inequality-resolvent-(n+1)4d}, we have that on $\cT_f$, $-\cI_{d, \varepsilon}\circ (\lambda - \cI_{d, \varepsilon})^{-1} = \1 - \lambda (\lambda - \cI_{d, \varepsilon})^{-1}$, i.e. for $x \in \cT_f$
    \begin{equation}
        \norm{(\1 - \lambda (\lambda - \cI_{d, \varepsilon})^{-1})x}_{W^{k, 1}} = \norm{-\cI_{d, \varepsilon}\circ (\lambda - \cI_{d, \varepsilon})^{-1} (x)}_{W^{k, 1}}
    \end{equation}
    where the LHS can be upper bounded by $(1 + \frac{\lambda}{\lambda + 2\varepsilon})\norm{x}_{W^{k, 1}} \le 2 \norm{x}_{W^{k, 1}}$ using \eqref{item:(1)}. This proves the last claim.
\end{proof}

\begin{lem}\label{lem:boundedness-polynomials}
    For $p \in \C[X,Y]$ a polynomial of degree $d$ and 
    \begin{equation*}
        A: \cH_f \to \cH, \qquad \ket\psi = \sum\limits_{\text{finite}}\psi_n \ket{n} \mapsto p(a, a^\dagger)\ket{\psi} = \sum\limits_{\text{finite}}\psi_n p(a, a^\dagger)\ket{n} \, ,
    \end{equation*}
    we get that for all $k \in \R_+$ and $d' \ge d$
    \begin{align*}
        B_1: \cH_f \to \cH, \qquad \ket\psi \mapsto (N + \1)^{k} A (N + \1)^{-k - d'} \ket\psi\\
        B_2: \cH_f \to \cH, \qquad \ket\psi \mapsto (N + \1)^{-k - d'} A (N + \1)^{k} \ket\psi
    \end{align*}
    are bounded and therefore can be uniquely extended to a bounded map on $\cH$, with the same bound.
\end{lem}
\begin{proof}
    Since the proof for $B_1$ and $B_2$ are almost completely analogous, we will only show it here for $B_1$. The canonical commutation relation allows us to rewrite $A$ as a finite linear combination of monomials of the form $(a^\dagger)^i N^j$ and $a^i N^j$ with $i + j/2 \le d$. Now by triangle inequality for the norm on $\cH$ and since the sum of these monomials comprising $A$ are finite, for the claim to be true it suffices to show that 
    \begin{equation*}
        (N + \1)^k(a^\dagger)^i N^j (N + \1)^{-k - d}:\cH_f \to \cH, \qquad (N + \1)^ka^i N^j (N + \1)^{-k - d}:\cH_f \to \cH
    \end{equation*}
    are bounded, and hence can be uniquely extended to a bounded map on $\cH$. We only give the argument for the first map, since it is almost completely analogous to the second one. Let $\ket\psi = \sum\limits_{n = 0}^M\psi_n \ket{n} \in \cH_f$, then
    \begin{align*}
         \ket{\varphi} \coloneqq (N + \1)^k(a^\dagger)^i N^j (N + \1)^{-k - d'} \ket{\psi} &= \sum\limits_{n = 0}^M\psi_n (N + \1)^k(a^\dagger)^i N^j (N + \1)^{-k - d'}\ket{n}\\
         &= \sum\limits_{n = 0}^M\psi_n \frac{(n + 1 + j)^k}{(n + 1)^k} \frac{n^j \prod_{l = 1}^i \sqrt{n + l}}{(n + 1)^{d'}} \ket{n + i} \, . 
    \end{align*}
    Hence
    \begin{align*}
        \norm{\varphi}^2 &= \sum\limits_{n = 0}^M \frac{(n + 1 + j)^{2k}}{(n + 1)^{2k}} \frac{n^{2j} \prod_{l = 1}^i (n + l)}{(n + 1)^{2d'}} |\psi_n|^2\\
        &\le \sum\limits_{n = 0}^M \frac{(n + 1 + d)^{2k}}{(n + 1)^{2k}} \frac{(n + d)^{2j}(n + d)^i}{(n + 1)^{2d}} |\psi_n|^2\\
        &\le \sum\limits_{n = 0}^M d^{2k} d^{2d} |\psi_n|^2\\
        &= d^{2(k + d)} \norm{\psi}^2 \, ,
    \end{align*}
    where we used that $i + j/2 \le d$ and $d \le d'$. Hence $(N + \1)^k (a^\dagger)^i N^j(N + \1)^{-k - d'}:\cH_f \to \cH$ is bounded by $d^{k + d}$ and can be uniquely extended to a bounded linear map on $\cH$. This concludes the claim.
\end{proof}

\begin{lem}\label{lem:infinitesimal-boundedness-W-k-1}
    Let $K \in \N$. For $i = 1, \hdots, K$ let $p_{i, 1}, p_{i, 2} \in \C[X,Y]$ polynomials of degree $d_{i, 1}$, $d_{i, 2}$ such that 
    \begin{equation*}
        \cA: \cT_f \to \cT_f, \qquad x \mapsto \cA(x) = \sum\limits_{i = 1}^KA_{i, 1}x A_{i, 2} = \sum\limits_{i = 1}^K p_{i, 1}(a, a^\dagger) x \, p_{i, 2}(a, a^\dagger)
    \end{equation*}
    where the action of $p_{1/2}(a, a^\dagger)$ on $x$ is defined via the action of $a$ and $a^\dagger$ on $\ketbra{n}{m}$. We then have that for all $k \ge 0$, $d \ge \max\limits_{i = 1, \hdots, K}\max\{d_{i, 1}, d_{i, 2}\}$ there exists $C_k \ge 0$, s.t. for all $\varepsilon \ge 0$ and $\forall x \in \cT_f$
    \begin{equation*}
        \norm{\cA(x)}_{W^{k, 1}} \le \varepsilon\norm{\{(N + \1)^{4d}, x\}}_{W^{k, 1}} + \frac{C_k}{\varepsilon} \norm{x}_{W^{k, 1}} \, . 
    \end{equation*}
\end{lem}
\begin{proof}
    Let $k \in \R_+$. The first step is to show that there exists $c_k \ge 0$, s.t. for all $x \in \cT_f$
    \begin{equation}\label{eq:proof-first-stage-inequality}
        \norm{\cA(x)}_{W^{k, 1}} \le c_k \norm{(N + \1)^d x (N + \1)^d}_{W^{k, 1}} \, . 
    \end{equation}
    The argument reduces to showing that for $i = 1, \hdots, K$ and $x \in \cT_f$
    \begin{equation*}\label{eq:proof-first-stage-inequality-substage}
        \norm{(N + \1)^{k/4} A_{i, 1} x A_{i, 2} (N + \1)^{k/4}}_1 \le c_{i, k} \norm{(N + \1)^d x (N + \1)^d}_{W^{k, 1}}
    \end{equation*}
    since the sum comprising $\cA$ is finite. Note that the trace norm on the LHS is the one on the trace-class operators since the argument might not necessarily be self-adjoint. For $x \in \cT_f$, we have
    \begin{align*}
        &\norm{(N + \1)^{k/4} A_{i, 1} x A_{i, 2} (N + \1)^{k/4}}_1\\
        &\hspace{0.5cm} = \norm{(N + \1)^{k/4} A_{i, 1} (N + \1)^{-k/4 - d} (N + \1)^{k/4 + d}x  (N + \1)^{k/4 + d}  (N + \1)^{-k/4 - d} A_{i, 2} (N + \1)^{k/4}}_1\\
        &\hspace{0.5cm}\le \norm{(N + \1)^{k/4} A_{i, 1} (N + \1)^{-k/4 - d}}_\infty \norm{x}_{W^{k+4d, 1}} \norm{(N + \1)^{-k/4 - d} A_{i, 2} (N + \1)^{k/4}}_\infty\\
        &\hspace{0.5cm}\le c_{i, k} \norm{(N + \1)^d x (N + \1)^d}_{W^{k, 1}} \, 
    \end{align*}
    where we used \Cref{lem:boundedness-polynomials} to argue that the operators involved are bounded and we can employ Hölder's inequality to split them off. Subsequently, we replaced the operator norms with the constant $c_{i, k}$. Now in the second step we show that for all $\varepsilon > 0$ and $x \in \cT_f$
    \begin{equation*}
        \norm{(N + \1)^d x (N + \1)^d}_{W^{k, 1}} \le \varepsilon \norm{\{(N + \1)^{4d}, x\}}_{W^{k, 1}} + \frac{1}{4\varepsilon} \norm{x}_{W^{k, 1}} \, . 
    \end{equation*}
    Combining this with \Cref{eq:proof-first-stage-inequality} then immediately provides the claim.
    Therefore, let $\lambda > 0$ and $x \in \cT_f$ with $x \ge 0$. We find
    \begin{align*}
        &\norm{(N + 1)^d(\lambda - \cI_{d, 1})^{-1}(x) (N + \1)^d}_{W^{k, 1}}\\
        &\hspace{2cm} = \tr[(\lambda - \cI_{d, 1})^{-1}\{(N + \1)^{k/4 + d} x (N + \1)^{k/4 + d}\}]\\
        &\hspace{2cm} = \int\limits_{0}^\infty \tr[e^{-(2(N + \1)^{4d} + \lambda) s}(N + \1)^{2d} (N + \1)^{k/4} x (N + \1)^{k/4}] ds\\
        &\hspace{2cm} = \tr\Big[\int\limits_{0}^\infty e^{-(2(N + \1)^{4d} + \lambda) s}(N + \1)^{2d} ds\,  (N + \1)^{k/4} x (N + \1)^{k/4}\Big]\\
        &\hspace{2cm} = \tr[\frac{(N + \1)^{2d}}{(N + \1)^{4d} + \lambda}(N + \1)^{k/4} x (N + \1)^{k/4}] \, ,
    \end{align*}
    where we used the map $\cI_{d, 1}$ from \Cref{lem:(n+1)-(n+1)-properties}, the integral representation of its resolvent $(\lambda - \cI_{d, 1})^{-1}$ and that this resolvent preserves positivity. Further, we applied the cyclicity of the trace and conveniently suppressed issues that might arise from the unbounded nature of $N$ by taking a cut-off in the Fock basis. This is justified by $x \in \cT_f$. Now we can use that $(N + \1)^{k/4} x (N + \1)^{k/4} \ge 0$ to bound the RHS of the above chain of inequalities to get
    \begin{align*}
        \norm{(N + 1)^d(\lambda - \cI_{d, 1})^{-1}(x) (N + \1)^d}_{W^{k, 1}} &\le \sup\limits_{s \ge 1} \frac{s}{s^2 + \lambda} \tr[(N + \1)^{k/4} x (N + \1)^{k/4}]\\
        &= \sup\limits_{s \ge 1} \frac{s}{s^2 + \lambda} \norm{x}_{W^{k, 1}} \, .
    \end{align*}
    For a general $x \in \cT_f$, we can set $y = (N + \1)^{k/4} x (N + \1)^{k/4}$ decompose into $y = y_+ - y_-$ the positive and negative part of $y$ respectively and then set $x_\pm = (N + \1)^{-k/4} y_\pm (N + \1)^{-k/4}$. We clearly have that $x_\pm \in \cT_f$, $x = x_+ - x_-$ and $x_\pm \ge 0$ as $(N + \1)^{-k/4} \cdot (N + \1)^{-k/4}$ preserves positivity. This allows us to apply what we have shown above to obtain
    \begin{align*}
        \norm{(N + 1)^d(\lambda - \cI_{d, 1})^{-1}(x) (N + \1)^d}_{W^{k, 1}} &\le \norm{(N + 1)^d(\lambda - \cI_{d, 1})^{-1}(x_+) (N + \1)^d}_{W^{k, 1}}\\
        &\hspace{2cm} + \norm{(N + 1)^d(\lambda - \cI_{d, 1})^{-1}(x_-) (N + \1)^d}_{W^{k, 1}}\\
        &\le \sup\limits_{s \ge 1} \frac{s}{s^2 +\lambda}(\norm{x_+}_{W^{k, 1}} + \norm{x_-}_{W^{k, 1}})\\
        &= \sup\limits_{s \ge 1} \frac{s}{s^2 +\lambda} (\norm{y_+}_1 + \norm{y_-}_1)\\
        &= \sup\limits_{s \ge 1} \frac{s}{s^2 +\lambda} \norm{y}_1\\
        &= \sup\limits_{s \ge 1} \frac{s}{s^2 +\lambda} \norm{x}_{W^{k, 1}}
    \end{align*}
    Lastly we can use the bijectivity of $(\lambda - \cI_{d, 1})$ on $\cT_f$ (q.v.~\Cref{lem:(n+1)-(n+1)-properties}) and triangle inequality to conclude 
    \begin{equation}
        \norm{(N + \1)^d x (N + \1)^d}_{W^{k, 1}} \le \sup\limits_{s \ge 0} \frac{s}{s^2 + \lambda} \norm{\cI_{d, 1}(x)}_{W^{k, 1}} + \lambda \sup\limits_{s \ge 0} \frac{s}{s^2 + \lambda} \norm{x}_{W^{k, 1}}\,.
    \end{equation}
    Choosing $\lambda = \frac{1}{4\varepsilon^2}$, we find that $\sup\limits_{s \ge 1} \frac{s}{s^2 + \lambda} < \varepsilon$, hence
    \begin{equation}
        \norm{(N + \1)^d x (N + \1)^d}_{W^{k, 1}} \le \varepsilon \norm{\{(N +\1)^{4d}, x\}}_{W^{k, 1}} + \frac{1}{4\varepsilon} \norm{x}_{W^{k, 1}}
    \end{equation}
\end{proof}

\begin{lem}[Interpolation Lemma]\label{lem:interpolation-lemma}
    Let $k_0 < k_1 \in \R_+$ and $(\cL, \cD(\cL))$ an operator on $W^{k_j, 1}$, $j = 0, 1$ respectively. Further, assume that the closure of $(\cL, \cD(\cL))$ defines a strongly continuous semigroup $(\cP_t^j)_{t \ge 0}$ with 
    \begin{equation}
        \norm{\cP_t^{j}}_{W^{k_j, 1} \to W^{k_j, 1}} \le M_j e^{\omega_j t} \quad \forall t \ge 0 \, 
    \end{equation}
    in both spaces, respectively. Then for $\theta \in [0, 1]$ and $k_\theta = \theta k_1 + (1 - \theta) k_0$ the following are true
    \begin{enumerate}
        \item\label{item:int-pol-1} The closure of $(\cL, \cD(\cL))$ defines a strongly continuous semigroup on $W^{k_\theta, 1}$ with 
        \begin{equation}\label{eq:interpolation-semigroup-bound}
            \norm{\cP^\theta_t}_{W^{k_\theta, 1} \to W^{k_\theta, 1}} \le M_0^{1 - \theta} M_1^\theta e^{(\omega_{k_0}(1 - \theta) + \omega_{k_1} \theta) t} \quad \forall t \ge 0\, . 
        \end{equation}
        \item\label{item:int-pol-2} $(\cP^\theta_t)_{t \ge 0}$ agrees with $(\cP_t^j)_{t \ge 0}$ on $W^{k_j, 1} \cap W^{k_\theta, 1}$ for $j = 1, 2$.
    \end{enumerate}
\end{lem}
\begin{proof}
    We begin with the second claim and only cover $k_j = k_0$ as the other case only requires minor changes that are left to the reader. Let $\theta \in (0, 1)$ and the closure of $(\cL, \cD(\cL))$ the generator of $(\cP_t^0)_{t \ge 0}$ and $(\cP_t^\theta)_{t \ge 0}$ on the respective spaces. Since $k_0 < k_\theta$ and hence $W^{k_\theta, 1} \Subset W^{k_0, 1}$, we have that the closure of $(\cL, \cD(\cL))$ in $W^{k_\theta, 1}$ agrees with the restriction of the closure in $W^{k_0, 1}$. As the semigroup is completely determined by its generator, we find that the semigroups agree on $W^{k_\theta, 1}$.\newline
    For the first claim, note that the semigroups $(\cP_t^j)_{t \ge 0}$, $j = 1, 2$ agree on $W^{k_1, 1}$ by \Cref{item:int-pol-2}, which allows us to employ the Stein-Weiss theorem for Bosonic Sobolev spaces (\Cref{thm:stein-weiss}) to conclude \Cref{eq:interpolation-semigroup-bound}. It remains to check that the families of bounded maps $(\cP_t^\theta)_{t \ge 0}$ are strongly continuous semigroups generated by the closure of $(\cL, \cD(\cL))$ on $W^{k_\theta, 1}$. We have that $\cP_0^\theta = \1$  and $\cP_t^\theta \cP_s^\theta = \cP_{t + 1}^\theta$ $\forall t, s \ge 0$ as a consequence of $\cP_t^0 |_{W^{k_\theta, 1}} = \cP_t^\theta$ $\forall t$ (this equality holds by \Cref{thm:stein-weiss}). The strong continuity follows, due to $\cP^1_t = \cP_t^\theta|_{W^{k_1, 1}}$, as for $x \in W^{k_1, 1}$
    \begin{equation*}
        \lim\limits_{t \to 0} \norm{\cP_t^\theta(x)  - x}_{W^{k_\theta, 1}} = \lim\limits_{t \to 0} \norm{\cP^1_t(x) - x}_{W^{k_\theta, 1}} \le \lim\limits_{t \to 0} \norm{\cP^1_t(x) - x}_{W^{k_1, 1}} = 0
    \end{equation*}
    where we used $W^{k_1, 1} \Subset W^{k_\theta, 1}$ and that $\cP^1_t$ is a strongly continous semigroup on $W^{k_1, 1}$. For general $x \in W^{k_\theta, 1}$, we find $(x_n)_{n \in \N} \subset W^{k_\theta, 1}$ converging to $x$ in $W^{k_\theta, 1}$ and for all $n \in \N$
    \begin{align*}
         \lim\limits_{t \to 0} \norm{\cP_t^\theta(x)  - x}_{W^{k_\theta, 1}} &\le \lim\limits_{t \to 0}\left[ (1 + M_0^{1 - \theta} M_1^\theta e^{(\omega_{k_0}(1 - \theta) + \omega_{k_1} \theta) t}) \norm{x - x_n}_{W^{k_\theta, 1}} + \norm{\cP_t^\theta(x_n)  - x_n}\right]\\
         &\le (1 + M_0^{1 - \theta} M_1^\theta)\norm{x - x_n}_{W^{k_\theta, 1}}
    \end{align*}
    which concludes the strong continuity. It remains to argue that the closure of $(\cL, \cD(\cL))$ on $W^{k_\theta, 1}$ is indeed the generator of $(\cP_t^\theta)_{t \ge 0}$. By \cite[Sec. II.2.3]{Engel.2000}, we find that the restriction of the generator $(\hat{\cL}, \cD(\hat{\cL}))$ of $(\cP_t^\theta)_{t \ge 0}$ to $W^{k_1, 1}$ is the generator of $(\cP_t^1)_{t \ge 0}$ with $(\cL, \cD(\cL))$ being a core for this restricted generator on $W^{k_1, 1}$ by assumption. This in particular means that for $\lambda > \omega_k$, $\lambda - \cL:\cD(\cL) \to W^{k_1, 1}$ has a dense range in $W^{k_1, 1}$, which further allows us to conclude that $\lambda - \cL:\cD(\cL) \to W^{k_\theta, 1}$ has a dense range in $W^{k_\theta, 1}$. Now as it is $\omega_{k_\theta}$-quasi dissipative (being the restriction of the generator of the semigroup $(\cP_t^\theta)_{t \ge 0}$) we can conclude that indeed $(\cL, \cD(\cL))$ is closable in $W^{k_\theta, 1}$ with the closure $(\hat{\cL}, \cD(\hat{\cL}))$ (c.f. \cite[Proposition 3.14]{Engel.2000}). 
\end{proof}

\begin{lem}[Approximation Lemma]\label{lem:approximation-lemma}
    Let $K \in \N$. For $i = 1, \hdots, K$ let $p_{i, 1}, p_{i, 2} \in \C[X,Y]$ polynomials of degree $d_{i, 1}$, $d_{i, 2}$ and $\{a_{i, n}\}_{n \in \N} \subset \C$ convergent sequences with limits $a_i \in \C$ such that $\{(\cA_n, \cD(\cA_n) = \cT_f)\}_{n \in \N}$ is an operator sequence with
    \begin{equation}
       \cA_n :\cT_f \to \cT_f, \quad x \mapsto \cA_n(x) \coloneqq \sum\limits_{i = 1}^K a_{i, n} A_{i, 1} \,x\, A_{i, 2} \coloneqq \sum\limits_{i = 1}^K a_{i, n} p_{i, 1}(a, a^\dagger) \,x\, p_{i, 2}(a, a^\dagger) \, . 
    \end{equation}
    If for all $k \in \R_+$ there exists $M_k, \omega_k$ such that for all $n \in \N$ the closure of $(\cA_n, \cD(\cA_n))$ generates a strongly continuous semigroup $(\cP_t^n)_{t \ge 0}$ on $W^{k, 1}$ with
    \begin{equation}\label{eq:uniform-bound-semigroups}
        \norm{\cP_t^n}_{W^{k, 1} \to W^{k, 1}} \le M_k e^{\omega_k t} \quad \forall t \in \R \, ,
    \end{equation}
    then the closure of $(\cA, \cD(\cA) = \cT_f)$, the pointwise limit of $(\cA_n, \cD(\cA_n))$, defines a strongly continuous semigroup on $W^{k, 1}$ for $k \ge 0$ as well. We further get that the semigroups generated by the closure of $(\cA_n, \cD(\cA_n))$ converge uniformly (in time) on compact intervals to the semigroup generated by the closure of $(\cA, \cD(\cA))$ and that \Cref{eq:uniform-bound-semigroups} also holds for the limiting semigroup.
\end{lem}
\begin{proof}
    Let $k \in \R_+$. To prove the lemma, we first note that $(\cA, \cD(\cA))$ is densely defined and the pointwise limit of $\{(\cA_n, \cD(\cA_n))\}_{n \in \N}$. To employ the second Trotter-Kato approximation theorem, which implies the claim (see the version in \cite[Thm.~III.4.9]{Engel.2000}), we need to show that there exists $\lambda > 0$ such that $(\lambda - \cA, \cD(\cA))$ has dense range in $W^{k, 1}$. We will do so by showing that the closure of the range contains $\cT_f$ which is a dense subset of $W^{k, 1}$. Therefore let $\lambda > \max\{\omega_k, \omega_{k + 4d}\}$ (with $\omega_\cdot$ from \Cref{eq:uniform-bound-semigroups} and $d$ the maximal degree of the polynomials but at least one, i.e.~$d = \max\limits_{i = 1, \hdots, K} \max\{d_{i, 1}, d_{i, 2}, 1\}$). Let $\xi \in \cT_f$ arbitrary, we then have that for all $n \in \N$ the operator $(\lambda - \cA_n, \cD(\cA_n))$ has dense range in $W^{k + 4d, 1}$, meaning in particular we find a sequence $\{x_{n, m}\}_{m \in \N}$ which is convergent in $W^{k + 4d, 1}$ and further
    \begin{equation}
        \lim\limits_{m \to \infty} \norm{(\lambda - \cA_n)(x_{n, m}) - \xi}_{W^{k + 4d, 1}}  = 0 \, .
    \end{equation}
    In addition, we can choose the sequence such that for all $m \in \N$, $\norm{(\lambda - \cA_n)(x_{n, m})}_{W^{k + 4d, 1}} \le \norm{\xi}_{W^{k + 4d,1}} + 1$. Due to the $\omega_{k + 4d}$-quasi dissipativity of $\cA_n$ (it is a generator of a strongly continuous semigroup with a bound given in \Cref{eq:uniform-bound-semigroups}) this immediately implies $\norm{x_{n, m}}_{W^{k + 4d, 1}} \le M_{k + 4d}\frac{\norm{\xi}_{W^{k + 4d, 1}} + 1}{\lambda - \omega_{k + 4d}} =: c_{\xi}$, i.e.~the set $\{x_{n, m}\}_{n, m \in \N}$ is bounded in $W^{k + 4d, 1}$. We now have that for $n, m \in \N$
    \begin{equation}\label{eq:dense-range-argument-inequality}
        \begin{aligned}
            \norm{(\lambda - \cA)(x_{n, m}) - \xi}_{W^{k, 1}} &\le \norm{(\lambda - \cA_n)(x_{n, m}) - \xi}_{W^{k, 1}} + \norm{(\cA - \cA_n)(x_{n, m})}_{W^{k, 1}}\\
            &\le \norm{(\lambda - \cA_n)(x_{n, m}) - \xi}_{W^{k, 1}} + \sum\limits_{i = 1}^K c_{i, k} |a_i - a_{i, n}|  \norm{x_{n, m}}_{W^{k + 4d, 1}}\\
            &\le \norm{(\lambda - \cA_n)(x_{n, m}) - \xi}_{W^{k, 1}} + \sum\limits_{i = 1}^K |a_i - a_{i, n}| c_{i, k} c_\xi\\
            &\le  \norm{(\lambda - \cA_n)(x_{n, m}) - \xi}_{W^{k, 1}} + C\sum\limits_{i = 1}^K |a_i - a_{i, n}| \, .
        \end{aligned}
    \end{equation}
    In the first line we used triangle inequality, in the second one the explicit form of $\cA$ and $\cA_n$ and then that there exists $c_{k, i} \ge 0$ such that 
    \begin{align*}
        \norm{(N + \1)^{k/4}A_{i, 1} x_{n, m} A_{i, 2} (N + \1)^{k/4}}_1\le c_{k, i} \norm{(N + \1)^d x_{n, m} (N + \1)^d}_{W^{k, 1}}
    \end{align*}
    as in the proof of \Cref{lem:infinitesimal-boundedness-W-k-1}. Lastly we used the uniform bound $c_\xi$ and set $C = \max_{i = 1, \hdots, K} c_{i,k} c_\xi$. By a proper choice of a subsequence of $\{x_{n, m}\}_{n, m \in \N}$, we get that the RHS of \Cref{eq:dense-range-argument-inequality} vanishes. Since $\{x_{n, m}\}_{n, m \in \N}$ is bounded in $W^{k + 4d, 1}$ it is in particular precompact in $W^{k, 1}$ (as of the compact embedding of the Sobolev spaces), meaning we can further choose the aforementioned sequence to be convergent in $W^{k, 1}$. Let us call it $\{y_n\}_{n \in \N} \subset \cT_f$. To summarise, for the chosen $\lambda$ and $\xi \in \cT_f$ arbitrary we have constructed a sequence $\{y_n\}_{n \in \N} \subset \cD(\cA)$ which is convergent in $W^{k, 1}$ and further $\{(\lambda - \cA)(y_n)\}_{n \in \N}$ converges to $\xi$ in $W^{k, 1}$. Hence the closure of the range of $(\lambda - \cA, \cD(\cA))$ contains $\cT_f$ a dense subset of $W^{k, 1}$, which concludes the proof. 
\end{proof}

\begin{rmk*}
    In the above lemma, it suffices to assume that the semigroups are Sobolev preserving, as one can interpolate between the sequence elements to obtain semigroups for $k \in \R_+$.
\end{rmk*}

\end{document}