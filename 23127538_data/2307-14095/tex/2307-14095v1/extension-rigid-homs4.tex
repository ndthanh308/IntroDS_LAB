\pdfoutput=1
\documentclass[a4paper,12pt,reqno,oneside]{amsart}

\renewcommand{\baselinestretch}{1.1}

% \usepackage[T1]{fontenc}
\usepackage{geometry}
\usepackage{lmodern}
\usepackage[stretch=10]{microtype}
\usepackage{graphicx}
\usepackage{tikz-cd}

\usepackage{amsmath}
\usepackage{amssymb}
\usepackage{amsthm}
\usepackage[matrix,arrow,cmtip]{xy}
\usepackage{bbm}
\usepackage[shortlabels]{enumitem}
\PassOptionsToPackage{hyphens}{url}
\usepackage[bookmarksnumbered,bookmarksdepth=2,bookmarksopen,colorlinks,linkcolor=black,citecolor=black,urlcolor=black,linktoc=all]{hyperref}
\usepackage[capitalise]{cleveref}
%\usepackage[nottoc]{tocbibind}
% \usepackage[foot]{amsaddr}

\newcommand{\Qbar}{\overline \bQ}
\newcommand{\bA}{\mathbb{A}}
\newcommand{\bC}{\mathbb{C}}
\newcommand{\bF}{\mathbb{F}}
\newcommand{\bG}{\mathbb{G}}
\newcommand{\bH}{\mathbb{H}}
\newcommand{\bK}{\mathbb{K}}
\newcommand{\bN}{\mathbb{N}}
\newcommand{\bP}{\mathbb{P}}
\newcommand{\bQ}{\mathbb{Q}}
\newcommand{\bR}{\mathbb{R}}
\newcommand{\bS}{\mathbb{S}}
\newcommand{\bZ}{\mathbb{Z}}
\newcommand{\ZZ}{\mathbb{Z}}
\newcommand{\QQ}{\mathbb{Q}}
\newcommand{\RR}{\mathbb{R}}
\newcommand{\CC}{\mathbb{C}}
\newcommand{\FF}{\mathbb{F}}
\newcommand{\GG}{\mathbb{G}}
\newcommand{\PP}{\mathbb{P}}
\newcommand{\AAA}{\mathbb{A}}

\newcommand{\Ranexp}{\bR_{\mathrm{an,exp}}}

\newcommand{\aE}{\tilde{\mathcal{E}}}

\newcommand{\cA}{\mathcal{A}}
\newcommand{\cB}{\mathcal{B}}
\newcommand{\cC}{\mathcal{C}}
\newcommand{\cE}{\mathcal{E}}
\newcommand{\cF}{\mathcal{F}}
\newcommand{\cG}{\mathcal{G}}
\newcommand{\cH}{\mathcal{H}}
\newcommand{\cI}{\mathcal{I}}
\newcommand{\cK}{\mathcal{K}}
\newcommand{\cL}{\mathcal{L}}
\newcommand{\cM}{\mathcal{M}}
\newcommand{\cN}{\mathcal{N}}
\newcommand{\cO}{\mathcal{O}}
\newcommand{\cP}{\mathcal{P}}
\newcommand{\cR}{\mathcal{R}}
\newcommand{\cS}{\mathcal{S}}
\newcommand{\cT}{\mathcal{T}}
\newcommand{\cV}{\mathcal{V}}
\newcommand{\cX}{\mathcal{X}}
\newcommand{\cY}{\mathcal{Y}}
\newcommand{\cGr}{\mathcal{G}r}

\newcommand{\fa}{\mathfrak{a}}
\newcommand{\fc}{\mathfrak{c}}
\newcommand{\fg}{\mathfrak{g}}
\newcommand{\fk}{\mathfrak{k}}
\newcommand{\fm}{\mathfrak{m}}
\newcommand{\fp}{\mathfrak{p}}
\newcommand{\fq}{\mathfrak{q}}
\newcommand{\fz}{\mathfrak{z}}
\newcommand{\fA}{\mathfrak{A}}
\newcommand{\fC}{\mathfrak{C}}
\newcommand{\fG}{\mathfrak{G}}
\newcommand{\fS}{\mathfrak{S}}
\newcommand{\fX}{\mathfrak{X}}

\newcommand{\Ag}{\mathcal{A}_g}
\newcommand{\Fg}{\mathcal{F}_g}
\newcommand{\Hg}{\mathcal{H}_g}

\newcommand{\gF}{\mathbf{F}}
\newcommand{\gG}{\mathbf{G}}
\newcommand{\gGL}{\mathbf{GL}}
\newcommand{\gGSp}{\mathbf{GSp}}
\newcommand{\gPGSp}{\mathbf{PGSp}}
\newcommand{\gPSp}{\mathbf{PSp}}
\newcommand{\gGSO}{\mathbf{GSO}}
\newcommand{\gGSU}{\mathbf{GSU}}
\newcommand{\gGU}{\mathbf{GU}}
\newcommand{\gGpU}{\mathbf{G'U}}
\newcommand{\gH}{\mathbf{H}}
\newcommand{\gK}{\mathbf{K}}
\newcommand{\gL}{\mathbf{L}}
\newcommand{\gM}{\mathbf{M}}
\newcommand{\gN}{\mathbf{N}}
\newcommand{\gO}{\mathbf{O}}
\newcommand{\gP}{\mathbf{P}}
\newcommand{\gPGL}{\mathbf{PGL}}
\newcommand{\gQ}{\mathbf{Q}}
\newcommand{\gQSp}{\mathbf{QSp}}
\newcommand{\gS}{\mathbf{S}}
\newcommand{\gSL}{\mathbf{SL}}
\newcommand{\gSO}{\mathbf{SO}}
\newcommand{\gSp}{\mathbf{Sp}}
\newcommand{\gSU}{\mathbf{SU}}
\newcommand{\gT}{\mathbf{T}}
\newcommand{\gU}{\mathbf{U}}
\newcommand{\gZ}{\mathbf{Z}}

\newcommand{\rH}{\mathrm{H}}
\newcommand{\rM}{\mathrm{M}}
\newcommand{\rB}{\mathrm{B}}

\newcommand{\rO}{\mathrm{O}}
\newcommand{\ro}{\mathrm{o}}

\DeclareMathOperator{\Ad}{Ad}
\DeclareMathOperator{\Aut}{Aut}
\DeclareMathOperator{\blockdiag}{blockdiag}
\DeclareMathOperator{\characteristic}{char}
\DeclareMathOperator{\covol}{covol}
\DeclareMathOperator{\denom}{denom}
\DeclareMathOperator{\diag}{diag}
\DeclareMathOperator{\disc}{disc}
\DeclareMathOperator{\End}{End}
\DeclareMathOperator{\Frac}{Frac}
\DeclareMathOperator{\Gal}{Gal}
\DeclareMathOperator{\GL}{GL}
\DeclareMathOperator{\Gr}{Gr}
\DeclareMathOperator{\gr}{gr}
\DeclareMathOperator{\GSp}{GSp}
\DeclareMathOperator{\Hol}{Hol}
\DeclareMathOperator{\Hom}{Hom}
\DeclareMathOperator{\im}{im}
\DeclareMathOperator{\Isog}{Isog}
\DeclareMathOperator{\Isom}{Isom}
\DeclareMathOperator{\Inn}{Inn}
\DeclareMathOperator{\Jac}{Jac}
\DeclareMathOperator{\Lie}{Lie}
\DeclareMathOperator{\MT}{MT}
\DeclareMathOperator{\Pic}{Pic}
\DeclareMathOperator{\Nm}{Nm}
\DeclareMathOperator{\Nrd}{Nrd}
\DeclareMathOperator{\NS}{NS}
\DeclareMathOperator{\Red}{Red}
\DeclareMathOperator{\Res}{Res}
\DeclareMathOperator{\rk}{rk}
\DeclareMathOperator{\Sh}{Sh}
\DeclareMathOperator{\sign}{sign}
\DeclareMathOperator{\SL}{SL}
\DeclareMathOperator{\SO}{SO}
\DeclareMathOperator{\spmap}{sp}
\DeclareMathOperator{\Sp}{Sp}
\DeclareMathOperator{\Spf}{Spf}
\DeclareMathOperator{\Spec}{Spec}
\DeclareMathOperator{\Stab}{Stab}
\DeclareMathOperator{\tr}{tr}
\DeclareMathOperator{\Tr}{Tr}
\DeclareMathOperator{\Trd}{Trd}
\DeclareMathOperator{\vol}{vol}

\newcommand{\DEG}{\mathrm{DEG}}
\newcommand{\DEGample}{\mathrm{DEG}_{\mathrm{ample}}}
\newcommand{\DEGpol}{\mathrm{DEG}_{\mathrm{pol}}}
\newcommand{\DD}{\mathrm{DD}}
\newcommand{\DDample}{\mathrm{DD}_{\mathrm{ample}}}
\newcommand{\DDpol}{\mathrm{DD}_{\mathrm{pol}}}

\renewcommand{\Im}{\mathop{\mathrm{Im}}}
\renewcommand{\Re}{\mathop{\mathrm{Re}}}

\newcommand{\an}{\mathrm{an}}
\newcommand{\ad}{\mathrm{ad}}
\newcommand{\alg}{\mathrm{alg}}
\newcommand{\can}{\mathrm{can}}
\newcommand{\comp}{\mathrm{comp}}
\newcommand{\der}{\mathrm{der}}
\newcommand{\formal}{\mathrm{for}}
\newcommand{\id}{\mathrm{id}}
\newcommand{\op}{\mathrm{op}}
\newcommand{\red}{\mathrm{red}}
\newcommand{\rig}{\mathrm{rig}}
\newcommand{\tors}{\mathrm{tors}}
\newcommand{\Zar}{\mathrm{Zar}}
\newcommand{\Quat}{\mathrm{Quat}}
\newcommand{\van}{{v\textrm{-}\mathrm{an}}}
\newcommand{\vhatan}{{\hat v\textrm{-}\mathrm{an}}}
\newcommand{\vrig}{{v\textrm{-}\mathrm{rig}}}
\newcommand{\ian}{{\iota\textrm{-}\mathrm{an}}}
\newcommand{\ihatan}{{\hat \iota\textrm{-}\mathrm{an}}}

\newcommand{\abs}[1]{\lvert #1 \rvert}
\newcommand{\Bigabs}[1]{\Bigl\lvert #1 \Bigr\rvert}
\newcommand{\ceil}[1]{\lceil #1 \rceil}
\newcommand{\floor}[1]{\lfloor #1 \rfloor}
\newcommand{\length}[1]{\lVert #1 \rVert}
\newcommand{\innerprod}[2]{\langle #1, #2 \rangle}
\newcommand{\powerseries}[2]{#1 [\![ #2 ]\!]}
\newcommand{\tatealgebra}[2]{#1 \langle #2 \rangle}

\newcommand{\bs}{\backslash}
\newcommand{\ov}{\overline}
\newcommand{\wt}{\widetilde}

\newcommand{\fullmatrix}[4]{\left( \begin{matrix} #1 & #2 \\ #3 & #4 \end{matrix} \right)}
\newcommand{\fullsmallmatrix}[4]{\bigl( \begin{smallmatrix} #1 & #2 \\ #3 & #4 \end{smallmatrix} \bigr)}
\newcommand{\diagsmallmatrix}[2]{\bigl( \begin{smallmatrix} #1 &\\& #2 \end{smallmatrix} \bigr)}

\newcommand{\defterm}[1]{\textbf{#1}}
\newcommand{\cat}[1]{\mathbf{#1}}

\newtheorem{lemma}{Lemma}[section]
\newtheorem{proposition}[lemma]{Proposition}
\newtheorem{theorem}[lemma]{Theorem}
\newtheorem{corollary}[lemma]{Corollary}
\newtheorem{conjecture}[lemma]{Conjecture}
\newtheorem{property}[lemma]{Property}
\newtheorem{properties}[lemma]{Properties}
\Crefname{conjecture}{Conjecture}{Conjectures} % Work around bug in cleveref
\newtheorem{claim}[lemma]{Claim}
\Crefname{claim}{Claim}{Claims}
\newtheorem{condition}[lemma]{Condition}
\newtheorem{hypothesis}[lemma]{Hypothesis}
\newtheorem*{lemma*}{Lemma}
\newtheorem*{proposition*}{Proposition}
\newtheorem*{theorem*}{Theorem}
\newtheorem*{corollary*}{Corollary}
\newtheorem*{claim*}{Claim}

\theoremstyle{definition}
\newtheorem*{definition}{Definition}
\newtheorem{remark}[lemma]{Remark}

\usepackage{ifthen}
\newcounter{constant}
\newcommand{\createC}[1]{\refstepcounter{constant} \label{C:#1}}
\newcommand{\newC}[1]{%
  \ifthenelse{\equal{#1}{*}} {%
      \stepcounter{constant} c_{\theconstant}%
  } {%
      \refstepcounter{constant} c_{\theconstant} \label{C:#1}%
  }%
}
\newcommand{\refC}[1]{c_{\ref*{C:#1}}}

\usepackage{color}
\newcommand{\chris}{\textcolor{red}}
\newcommand{\martin}{\textcolor{blue}}

\renewcommand{\thesubsection}{\thesection.\Alph{subsection}}


\title[Extension of relative rigid homomorphisms]{Extension of relative rigid homomorphisms from the formal multiplicative group}
\author{Martin Orr}

\address{Orr: Department of Mathematics, The University of Manchester, Alan Turing Building, Oxford Road, Manchester M13 9PL, United Kingdom}
\email{martin.orr@manchester.ac.uk}


\begin{document}

\begin{abstract}
A theorem of L\"utkebohmert states that a rigid group homomorphism from the formal multiplicative group to a smooth commutative rigid group~$G$, with relatively compact image, can be extended to a homomorphism from the full rigid multiplicative group to~$G$.
In this paper, we prove a relative version of this theorem, with any rigid space as a base.
This theorem is a step in the uniformisation of proper smooth rigid groups.
\end{abstract}

\maketitle

\section{Introduction}

Let $K$ be a non-archimedean field and let $R$ be the valuation ring of~$K$.
Throughout this paper, rigid spaces will always mean rigid spaces over~$K$ and formal schemes will always mean formal schemes over $\Spf(R)$.
Let $\GG_m(1)$ denote the formal rigid multiplicative group over~$K$, that is, the following open subgroup of the rigid $K$-group $\GG_m$:
\[ \GG_m(1) = \{ x : \abs{x} = 1 \}. \]

The aim of this paper is to prove a relative version of the following theorem of L\"utkebohmert.

\begin{proposition} \label{hom-extends-Lut}
\cite[Prop.~3.1]{Lut95}, \cite[Prop.~7.3.1]{Lut16}
% \Cref{hom-extends-conj} holds when $S = \Sp(K)$.
Let $G$ be a smooth connected commutative rigid group over~$K$.
Let $\bar\phi \colon \GG_m(1) \to G$ be a homomorphism of rigid groups.
Suppose that:
\begin{enumerate}[label=(\roman*)]
\item $\bar\phi$ factors through an admissible open subgroup $H \subset G$, which admits a smooth formal $R$-model $H_\formal$;
\item the image of $\bar\phi$ is relatively compact in a quasi-compact open subspace $G'$ of~$G$.
\end{enumerate}
Then there exists a unique homomorphism of rigid groups $\phi \colon \GG_m \to G$ whose restriction to $\GG_m(1)$ is equal to $\bar\phi$.
\end{proposition}

Our main theorem is as follows.

\newpage % COSMETIC

\begin{proposition} \label{hom-extends-thm}
Let $S$ be a rigid space over~$K$.
Let $G \to S$ be a smooth commutative rigid $S$-group with connected fibres.
Let $\bar\phi \colon \GG_m(1) \times S \to G$ be a homomorphism of rigid $S$-groups.
Suppose that $\bar\phi$ factors through an admissible open $S$-subgroup $H \subset G$ satisfying:
\begin{enumerate}[label=(\roman*)]
\item $H \cong \GG_m(1)^g \times S$;
\item There is a quasi-compact open subspace $G'$ of~$G$ such that $H$ is a relatively $S$-compact subspace of~$G'$.
\end{enumerate}
Then there exists a unique homomorphism of rigid $S$-groups $\phi \colon \GG_m \times S \to G$ whose restriction to $\GG_m(1) \times S$ is equal to $\bar\phi$.
\end{proposition}

For the definition of ``relatively $S$-compact'', see section~\ref{sec:relative-compactness}.

\Cref{hom-extends-thm} is more restricted than the na\"ive relative version of \cref{hom-extends-Lut}: \cref{hom-extends-thm}(i) requires $H$ to be a split formal torus, while \cref{hom-extends-Lut}(i) allows more general~$H$, and \cref{hom-extends-thm}(ii) requires $H$ to be relatively compact, while \cref{hom-extends-Lut}(ii) only requires the image of $\bar\phi$ to be relatively compact.
The na\"ive relative version of \cref{hom-extends-Lut} may still be true, but there are several steps in our proof of \cref{hom-extends-thm} where these additional restrictions simplify the argument.
Further work is required to determine whether these restrictions can be removed from \cref{hom-extends-thm}.
The proposition as stated is sufficient for the application to the Zilber--Pink conjecture which motivated this paper, namely \cite{DO:LGO}.

The proof of \cref{hom-extends-thm} is based on the proof of \cref{hom-extends-Lut} in \cite{Lut16}, relying on Lütkebohmert's approximation theorem for rigid analytic morphisms (which is already relative).
The most substantial difficulty in relativizing this proof appears to be establishing a relative version of \cite[Thm.~7.2.3]{Lut16} (subgroup generated by the image of a smooth formal scheme).
Under condition~(i) of \cref{hom-extends-thm}, we avoid needing to relativize this theorem by explicitly writing down a subgroup $U$ of~$H$ isomorphic to a relative ball, which contains the image of a certain morphism $\bar{u}$ (see Step~2 in the proof of \cref{hom-extends-thm}).
Meanwhile, \cref{hom-extends-thm}(ii) allows us to skip the first step in the proof of \cite[Prop.~7.3.1]{Lut16} (reducing to the case where $H$ is relatively compact)

% Also, with regard to \cref{hom-extends-thm}(ii), the proof of \cite[Prop.~7.3.1]{Lut16} begins by reducing to the case in which $H$ is relatively compact, but I don't fully understand this argument.  (Looking at \cite[Prop.~3.1]{Lut95}, this again appears to be linked to group generation.)

We will also prove the following corollary of \cref{hom-extends-thm}, which is sometimes more convenient to apply.
In particular, \cref{hom-extends-no-boundary} applies whenever $G \to S$ is the analytification of a group scheme $\fG \to \fS$ of finite type over~$K$, since a covering of $\fS$ by affine schemes~$\fS_i$ induces an admissible covering of~$S$, and the resulting rigid spaces $G|_{S_i} = (\fG|_{\fS_i})^\an$ have no boundary by \cite[Cor.~5.11]{Lut90}.

\begin{corollary} \label{hom-extends-no-boundary}
Let $S$ be a rigid space over~$K$.
Let $G \to S$ be a smooth commutative rigid $S$-group with connected fibres.
Let $\bar\phi \colon \GG_m(1)^g \times S \to G$ be a homomorphism of rigid $S$-groups.
Suppose that:
\begin{enumerate}[label=(\roman*)]
\item $\bar\phi$ is an isomorphism from $\GG_m(1)^g \times S$ to an admissible open $S$-subgroup $H \subset G$.
\item There exists an admissible covering of $S$ by affinoids $S_i$ such that, for each~$i$, $G|_{S_i}$ has no boundary, in the sense of \cite[Def.~5.9]{Lut90}.
\end{enumerate}
Then there exists a unique homomorphism of rigid $S$-groups $\phi \colon \GG_m^g \times S \to G$ whose restriction to $\GG_m(1)^g \times S$ is equal to $\bar\phi$.
\end{corollary}


\subsection{Notation}

If $S$ is an affinoid space, let $\cO(S)$ denote the ring of holomorphic functions on~$S$ and let
\[ \cO^\circ(S) = \{ f \in \cO(S) : \abs{f} \leq 1 \}, \]
where $\abs{\cdot}$ is the sup norm on $\cO(S)$.

Let $D(r)$ denote the closed disc of radius~$r$ centred at the origin (as a $K$-rigid space) and let $D = D(1)$.


\subsection*{Acknowledgements}

I am very grateful to Werner Lütkebohmert for patient answers to my questions about relativizing \cref{hom-extends-Lut}, in particular for suggesting that the argument in \cite{Lut95} should relativize, for helping me to understand the approximation theorem, and for suggesting how to prove \cref{relatively-compact-thickening}.
I am also grateful to Christopher Daw for helpful discussions.


\section{Group laws on relative rigid balls} \label{sec:group-laws-balls}

Let $S$ be an affinoid space.
Let $U$ be a rigid $S$-group whose underlying $S$-space is isomorphic to $D^g \times S$, with the identity section $S \to U$ being equal to the zero section $S \to D^g \times S$.

Then the group law $U \times_S U \to U = D^g \times S$ can be described by a $g$-tuple of holomorphic functions
\[ F \in \cO^\circ(U \times_S U) ^g = \cO^\circ(D^g \times D^g \times S)^g = \tatealgebra{\cO^\circ(S)}{X_1, \dotsc, X_g, Y_1, \dotsc, Y_g}^g. \]
Likewise the inverse $U \to U = D^g \times S$ is described by a $g$-tuple of holomorphic functions
\[ i \in \cO^\circ(U)^g = \cO^\circ(D^g \times S)^g = \tatealgebra{\cO^\circ(S)}{X_1, \dotsc, X_g}^g. \]
Because $U$ is a group over~$S$, $F$ satisfies the usual conditions to be a $g$-dimensional formal group law over $\cO^\circ(S)$ i.e.
\begin{enumerate}
\item $F(\underline X, \underline 0) = \underline X$ and $F(\underline 0, \underline Y) = \underline Y$ (the zero section is the identity section);
\item $F(\underline X, F(\underline Y, \underline Z)) = F(F(\underline X, \underline Y), \underline Z)$ (associativity);
\item $F(\underline X, i(\underline X)) = F(i(\underline X), \underline X) = \underline 0$ (inverse).
\end{enumerate}

Let $\pi \in R$ with $0 < \epsilon = \abs{\pi} < 1$.

\begin{lemma} \label{formal-group-F2-congruence}
Let $\lambda \in \ZZ_{\geq 0}$.
Let $\underline x, \underline y \in D(\epsilon^\lambda)^g$.
Then $F(\underline x, \underline y) \equiv \underline x + \underline y \bmod \pi^{2\lambda}$.
\end{lemma}

\begin{proof}
From condition~(1) for a formal group law, we obtain that
\[ F(\underline X, \underline Y) = \underline X + \underline Y + \text{higher order terms} \]
where each higher order term is divisible by at least one $X_i$ variable and by at least one $Y_j$ variable.
Hence if $\abs{\underline x}, \abs{\underline y} \leq \epsilon^\lambda$, then the higher order terms all have absolute value at most $\epsilon^{2\lambda}$, as required.
\end{proof}

\begin{lemma} \label{formal-group-Fi-congruence}
Let $R$ be an adic ring and let $I \subset R$ be an ideal of definition.
Let $F \in \powerseries{R}{X_1, \dotsc, X_g, Y_1, \dotsc, Y_g}$ be a $g$-dimensional (commutative) formal group law, with inverse $i \in \powerseries{R}{X_1, \dotsc, X_g}$.
If $\underline x, \underline y \in R^g$ with $\underline x - \underline y \in I^g$, then $F(\underline x,i(\underline y)) \in I^g$.
\end{lemma}

\begin{proof}
Fix $k \in \{ 1, \dotsc, g \}$ and $\underline x \in R^g$.
Consider $F_k(\underline x, i(\underline Y)) \in \powerseries{R}{Y_1, \dotsc, Y_g}$.
By ``Taylor expansion near $\underline Y=\underline x$'',
%(i.e.\ by a multi-dimensional ``weak inequality'' version of [ZP for $Y(1)^n$, Lemma~2.1]),
$\underline x-\underline y \in I^g$ implies that $F_k(\underline x, i(\underline y)) - F_k(\underline x, i(\underline x)) \in I^g$.
Since $F_k(\underline x, i(\underline x)) = 0$ (definition of inverse of a formal group law), this proves the lemma.
\end{proof}


\section{Relative \texorpdfstring{$S$}{S}-compactness} \label{sec:relative-compactness}

The following notion is essential to \cref{hom-extends-thm}.

\begin{definition} \cite[Def.~3.6.1]{Lut16}
Let $X \to S$ be a morphism of quasi-separated, quasi-compact rigid spaces.
An open subspace $U \subset X$ is said to be \defterm{relatively $S$-compact}, written $U \Subset_S X$,
if there exists a formal $R$-model $X_\formal \to S_\formal$ of $X \to S$ such that:
\begin{enumerate}
\item $U$ is induced by an open subscheme $U_\formal \subset X_\formal$;
\item the schematic closure of $U_0$ in $X_0$ is proper over $S_0$.
\end{enumerate}

A general subset of $X$ is said to be \defterm{relatively $S$-compact} if it is contained in an $S$-relatively compact open subspace of~$X$.
\end{definition}

\begin{remark}
If $X$, $S$, $U$ are all affinoid, then the above definition is equivalent to:
$U$ is \defterm{relatively $S$-compact} in~$X$ if there exists a closed immersion $X \to B^n(1) \times S$ which maps $U$ into $B^n(r) \times S$ for some $r<1$.
\end{remark}

For $0 < r \leq 1$, define the following rigid spaces over~$K$:
\begin{align*}
    A(r) & = \{ x : r \leq \abs{x} \leq r^{-1} \},
\\  A^{\leq}(r) & = \{ x : r \leq \abs{x} \leq 1 \},
\\  A^{\geq}(r) & = \{ x : 1 \leq \abs{x} \leq r^{-1} \}.
\end{align*}

Note that $A(r)$ is the same as what we called $\GG_m(r)$ in the introduction.
In this section, we don't care about the multiplicative group structure, so I am using the notation $A(r)$ instead.

\begin{lemma} \label{relatively-compact-thickening}
Fix $r > 1$.
Let $S$ be an affinoid space.
Let $Y \subset A(r) \times S$ be an admissible open subset such that $A(1) \times S$ is an $S$-relatively compact subset of $Y$.
Then $Y$ contains $A(r') \times S$ for some $r' < 1$.
\end{lemma}

\begin{proof}
Choose formal models $S_\formal$, $X_\formal$, $Y_\formal$, $U_\formal$ for $S$, $X := A(r) \times S$, $Y$, $U := A(1) \times S$ respectively.
By the definition of $A(1) \times S \Subset_S Y$, we can choose these formal models such that $U_\formal$ is an open formal subscheme of $Y_\formal$, and the schematic closure $\ov U_0$ of $U_0$ in $Y_0$ is proper over~$S_0$.  We can also choose the formal models such that $Y_\formal$ is an open formal subscheme of $X_\formal$.

Let $Z_0 = X_0 \setminus \ov U_0$, which is an open subscheme of~$X_0$.
Thus $Z_0$ corresponds to an open subdomain $Z = \Red^{-1}(Z_0) \subset X$.
Let
\[ Z^\leq = \{ (x,s) \in Z : \abs{x} \leq 1 \}, \; Z^\geq = \{ (x,s) \in Z : \abs{x} \geq 1 \}. \]
Let
\[ r^\leq = \sup \{ \abs{x} : (x,s) \in Z^\leq \}, \; r^\geq = \sup \{ \abs{x}^{-1} : (x,s) \in Z^\geq \}. \]
(If $Z^\leq$ or $Z^\geq$ empty, set $r^\leq=r$ or $r^\geq=r$ respectively.)

We claim that $r^\leq < 1$.
If $Z^\leq$ is empty, then this is immediate.
Otherwise, assume that $Z^\leq$ is non-empty.
Since $Z$ is the rigid generic fibre of~$Z_\formal$, it is quasi-compact.
Since $Z^\leq$ is a rational subdomain of $Z$, $Z^\leq$ is quasi-compact.
Hence, by the maximum principle, the absolute value of the first co-ordinate $\abs{x}$ attains its supremum~$r^\leq$ on~$Z^\leq$.
But $Z$ is disjoint from $U = A(1) \times S$, so $\abs{x} \neq 1$ on $Z^\leq$.
Hence $r^\leq < 1$.

Similarly, we obtain $r^\geq < 1$.

Choose $r'$ such that $\max\{ r^\leq, r^\geq \} < r' < 1$.
We claim that $A(r') \times S \subset Y$.
We shall show that $A^\leq(r') \times S \subset Y$; a very similar argument will show that $A^\geq(r') \times S \subset Y$, completing the proof of the lemma.

Indeed, if $(x,s) \in A^\leq(r') \times S$, then $r^\leq < r' \leq \abs{x}$.
Hence by the definition of~$r^\leq$, $(x,s) \not \in Z^\leq$.
Since $\abs{x} \leq 1$, it follows that $(x,s) \not \in Z$.
Hence
\[ \Red(x) \in X_0 \setminus Z_0 = \ov U_0 \subset Y_0. \]
Thus $(x,s) \in Y$, as required.
\end{proof}


\section{Approximation theorem}

The following theorem is the form of the approximation theorem which we shall use.
Fix $\pi \in K^\times$ with $\abs{\pi} < 1$.

\begin{theorem} \label{approximation:rigid}
Let $S$ be an affinoid space over~$K$.
Let $X \to S$ and $Z \to S$ be quasi-compact quasi-separated rigid $S$-spaces.
Let $U$ be an admissible open subspace of~$Z$ and let $\phi\colon U \to X$ be a morphism of rigid $S$-spaces.
Suppose that:
\begin{enumerate}[(i)]
\item $Z$ is affinoid and $U$ is a Weierstrass domain in~$Z$;
\item $\phi(U)$ is contained in an affinoid~$V$ which is relatively $S$-compact in~$X$;
\item $X \to S$ is smooth;
\item $U \Subset_S Z$.
\end{enumerate}
Let $\lambda \in \ZZ_{>0}$.
Then there exists an open subdomain $U' \subset Z$ and an $S$-morphism $\phi' \colon U' \to X$ with the following properties:
\begin{enumerate}
\item $U \Subset_S U' \Subset_S Z$;
\item $\phi'|_U \equiv \phi \bmod \pi^\lambda$ with respect to given formal $R$-models.
\end{enumerate}
\end{theorem}

\Cref{approximation:rigid} is a mild generalisation of \cite[Thm.~6.3]{Lut09} because, by replacing $X$ by $X \times_S Z$ and $S$ by $Z$, \cite[Thm.~6.3]{Lut09} implies \cref{approximation:rigid} under the additional assumption that $U \Subset Z$.

We deduce \cref{approximation:rigid} from the following theorem of Lütkebohmert.
In the following theorem statement, $S$ denotes Raynaud's rigid generic fibre of $S_\formal$, etc.

\begin{theorem} \label{approximation:formal} \cite[Thm.~7.4]{Lut95}, \cite[Thm.~3.6.7]{Lut16}
Let $S_\formal$ be an affine admissible formal scheme over $\Spf(R)$.
Let $X_\formal$ be a separated admissible formal $S_\formal$-scheme such that $X \to S$ is smooth.
Let $Z_\formal \to S_\formal$ be an admissible formal scheme and let $U_\formal$ be an open subscheme of $Z_\formal$.
Let $\phi \colon U_\formal \to X_\formal$ be a morphism of formal schemes over $S_\formal$.
Suppose that:
\begin{enumerate}[(i)]
\item $Z$ is affinoid and $U$ is a Weierstrass domain in~$Z$;
\item $\phi(U)$ is contained in an affinoid~$V$ which is relatively $S$-compact in~$X$.
\end{enumerate}

Let $\lambda \in \ZZ_{\geq 0}$.
Then there exists an admissible formal blowing-up $Z'_\formal \to Z_\formal$, which is finite over~$U_\formal$, and an open subscheme $U'_\formal \subset Z'_\formal$, such that:
\begin{enumerate}[(a)]
\item the schematic closure of $(Z'_\formal \times_{Z_\formal} U_\formal)_0$ (in $Z'_0$) is contained in $U'_0$;
\item the schematic closure $\ov U'_0$ (in $Z'_0$) is proper over $\ov U_0$ (closure in $Z_0$);
\item there exists a morphism $\phi' \colon U'_\formal \to X_\formal$ such that $\phi'|_{U_\formal}$ coincides with $\phi$ up to the level~$\lambda$.
\end{enumerate}
\end{theorem}

% \martin{I'm a little dubious about (c): what does $\phi'|_{U_\formal}$ mean?
% Should it say that $\phi'|_{Z'_\formal \times_{Z_\formal} U_\formal}$ coincides with the composition of $\phi$ with $Z'_\formal \times_{Z_\formal} U_\formal \to U_\formal$ up to level~$\lambda$?
% I believe that ``coincides up to the level~$\lambda$'' means ``becomes equal after extension of scalars to $R/(\pi^{\lambda+1})$.''}

To deduce \cref{approximation:rigid} from \cref{approximation:formal}:
Choose formal $R$-models $S_\formal$, $X_\formal$, $Z_\formal$, $U_\formal$, such that $U_\formal$ is an open subscheme of $Z_\formal$ \cite[Lemma~8.4/5]{Bos14} and such that $X \to S$, $Z \to S$, $\phi \colon U \to X$ come from morphisms of formal schemes.
Conditions (i) and~(ii) of \cref{approximation:formal} are the same as conditions (i) and~(ii) of \cref{approximation:rigid}.
Now \cref{approximation:formal} gives us a new formal model $Z'_\formal$ of $Z$ and an open subscheme $U'_\formal$.
Then the rigid generic fibre $U'$ of $U'_\formal$ is an open subdomain of~$Z$.
By \cref{approximation:formal}(b), we have $\ov U'_0$ proper over $\ov U_0$, while $\ov U_0$ is proper over $S_0$ by \cref{approximation:rigid}(iv).  Hence $\ov U'_0$ is proper over~$S_0$, so $U' \Subset_S Z$.
By \cref{approximation:formal}(a), the schematic closure of $(Z'_\formal \times_{Z_\formal} U_\formal)_0$ in $U'_0$ is equal to its schematic closure in $Z'_0$, and hence is Zariski closed in $\ov U'_0$.
Since (as we just showed), $\ov U'_0$ is proper over~$S_0$, we conclude that the schematic closure of $(Z'_\formal \times_{Z_\formal} U_\formal)_0$ in $U'_0$ is proper over~$S_0$.  Hence $U \Subset_S U'$.
Finally, \cref{approximation:formal}(c) is equivalent to \cref{approximation:rigid}(2), shifting $\lambda$ by~$1$.

% I think that \cref{approximation:formal} (even without its last sentence) is genuinely more general than \cref{approximation:rigid}, because it does not require condition~(iv) of \cref{approximation:rigid}.


\section{Main proof}

For $0 < \epsilon \leq 1$, define $\GG_m(\epsilon)$ to be the following affinoid space over~$K$:
\[ \GG_m(\epsilon) = \{ x : \epsilon \leq \abs{x} \leq \epsilon^{-1} \}. \]
This is consistent with our earlier definition of $\GG_m(1)$, but note that $\GG_m(\epsilon)$ is not closed under multiplication when $0 < \epsilon < 1$.

Before beginning the proof of \cref{hom-extends-thm}, we prove the following lemma.

\begin{lemma} \label{coeffs-congruence-lemma}
Let $S$ be an affinoid space with a formal $R$-model~$S_\formal$.
Let $G \to S$ be a smooth commutative rigid $S$-group.
Let $U \subset G$ be an admissible open $S$-subgroup whose underlying rigid $S$-space is isomorphic to $D^g \times S$, via an isomorphism $\zeta \colon U \to D^g \times S$. %, such that $U$ possesses a formal model which is a smooth formal $S_\formal$-group scheme.
Write $\zeta_1, \dotsc, \zeta_g \colon U \to D$ for the composition of $\zeta$ with the projection onto the $i$-th copy of $D$.

Let $0<\epsilon<1$ and let $\pi \in K$ with $\abs{\pi} = \epsilon$.

Let $\bar\alpha, \bar\phi \colon \GG_m(1) \times S \to G$ be morphisms of rigid $S$-spaces such that:
\begin{enumerate}[label=(\roman*)]
\item $\bar\phi$ is a homomorphism of $S$-groups;
\item $\bar\alpha$ extends to a morphism of $S$-spaces $\alpha \colon \GG_m(\epsilon^2) \times S \to G$;
\item the $S$-morphism $w_\alpha \colon \GG_m(\epsilon) \times \GG_m(\epsilon) \times S \to G$ defined by $w_\alpha(x_1, x_2) = \alpha(x_1^{-1}x_2^{-1}) \cdot \alpha(x_1) \cdot \alpha(x_2)$ has image contained in~$U$;
\item the $S$-morphism $\bar u = \bar\alpha \cdot \bar\phi^{-1} \colon \GG_m(1) \times S \to G$ has image contained in~$U$.
\end{enumerate}

Write the Laurent series $\bar u_i$ for $\zeta_i \circ \bar u \colon \GG_m(1) \times S \to D$ as
\[ \bar u_i(\xi, s) = \sum_{j \in \ZZ} \bar u_{ij}(s) \xi^j \in \tatealgebra{\cO^\circ(S)}{\xi, \xi^{-1}}. \]

Let $\lambda \in \ZZ_{\geq 1}$.
If $\im(\bar u) \subset \pi^\lambda U$ (that is, if $\zeta_i(\bar u(x)) \in D(\epsilon^\lambda) \times S$ for all $x \in \GG_m(1) \times S$ and all $i = 1, \dotsc, g$), then
\[ \abs{u_{ij}}_S \leq \epsilon^{2\min\{ \lambda, \abs{j} \}} \]
for all $i = 1, \dotsc, g$ and all $j \in \ZZ$.
\end{lemma}

\begin{proof}
Write the Laurent series $w_i$ for $\zeta_i \circ w_\alpha \colon \GG_m(\epsilon) \times \GG_m(\epsilon) \times S \to D$ as
\[ w_i(\xi_1, \xi_2, s) = \sum_{j,k \in \ZZ} w_{ijk}(s) \xi_1^j \xi_2^k \in \tatealgebra{\cO^\circ(S)}{\pi\xi_1, \pi\xi_1^{-1}, \pi\xi_2, \pi\xi_2^{-1}}, \]
where $w_{ijk} \in \cO^\circ(S)$.
Since $w_i$ maps $\GG_m(\epsilon) \times \GG_m(\epsilon) \times S$ into~$D$, we have
\begin{equation} \label{eqn:wijk-bound}
\abs{w_{ijk}}_S \leq \epsilon^{\abs{j}+\abs{k}}
\end{equation}
for all $j,k \in \ZZ$.

The $S$-group structure on~$U$ induces an $S$-group structure on $D^g \times S$.
As described in section~\ref{sec:group-laws-balls}, the $S$-group structure on $D^g \times S$ can be described by a formal group law $F \in \tatealgebra{\cO^\circ(S)}{X_1, \dotsc, X_g, Y_1, \dotsc, Y_g}^g$.
Stating this precisely using the notation of the current lemma, we have
\[ \zeta_i(x \cdot y) = F_i(\zeta_1(x), \dotsc, \zeta_g(x), \zeta_1(y), \dotsc, \zeta_g(y)) \]
for all $(x,y) \in U \times_S U$.
By \cref{formal-group-F2-congruence}, we obtain that, if $x, y \in D(\epsilon^\lambda) \times S$, then
\begin{equation} \label{eqn:p-i-congruence}
\zeta_i(x \cdot y) \equiv \zeta_i(x) + \zeta_i(y) \bmod \epsilon^{2\lambda}
\end{equation}
for each $i = 1, \dotsc, g$.

Since $\bar\phi$ is a homomorphism of commutative $S$-groups, we have
\[ w_\alpha(x_1, x_2, s) = \bar u(x_1^{-1}x_2^{-1}) \cdot \bar u(x_1) \cdot \bar u(x_2) \]
for all $(x_1, x_2, s) \in \GG_m(1) \times \GG_m(1) \times S$.
Using the hypothesis that $\im(\bar u) \subset \pi^\lambda U$ and applying \eqref{eqn:p-i-congruence}, we obtain the following congruence for all $\xi_1, \xi_2 \in \GG_m(1)$:
\begin{equation} \label{eqn:w-u-congurence}
w_i(\xi_1, \xi_2, s) \equiv \bar u_i(\xi_1^{-1}\xi_2^{-1}, s) + \bar u_i(\xi_1, s) + \bar u_i(\xi_2, s) \bmod \pi^{2\lambda}.
\end{equation}
The above is valid as a congruence of Laurent series in $\tatealgebra{\cO^\circ(S)}{\xi_1, \xi_2, \xi_1^{-1}, \xi_2^{-1}}$.

Comparing the coefficients of $\xi_1^{-j}\xi_2^{-j}$, we obtain the following congruence:
\[ w_{ijj} \equiv \bar u_{i,-j} \bmod \pi^{2\lambda}. \]
By \eqref{eqn:wijk-bound}, we have $\abs{w_{ijj}}_S \leq \epsilon^{2\abs{j}}$, so this proves the lemma.
\end{proof}

Now we prove \cref{hom-extends-thm}.

\begin{proof}[Proof of~\cref{hom-extends-thm}]
First note that it suffices to construct $\phi$ locally on an admissible covering of~$S$ and glue, so we can and do assume that $S$ is affinoid.

In the notation of the first paragraphs of \cite[Prop.~7.3.1]{Lut16}, we have $\tilde B = \tilde U = \{1\}$, $\tilde T = \tilde L = \tilde H$ (where $\tilde\cdot$ denotes reduction), $r=0$, $L(c)=H$, $t=\dim(G/S)$, $s=0$.
In particular, the step where $H$ is replaced by $L(c)$ does not change anything.

\subsubsection*{Step~1: Approximation}

We apply \cref{approximation:rigid} to our given~$S$, $X = G'$, $Z = \GG_m(a) \times S$ for some $a$ with $0<a<1$, $U = \GG_m(1) \times S$ and $\phi=\bar\phi$.
% Here~$N$ can be chosen as large as we like (we use $\GG_m(\abs{\pi}^N) \times S$ instead of $\GG_m \times S$ so that $Z$ is quasi-compact).
\Cref{approximation:rigid}(i) and~(iv) hold by our explicit description of $Z$ and $U$.
\Cref{approximation:rigid}(ii) holds with $V=H$, by hypotheses (i) and~(ii) of \cref{hom-extends-thm}.
\Cref{approximation:formal}(iii) holds because the hypotheses of \cref{hom-extends-thm} include that $G \to S$ is smooth.

By \cref{approximation:rigid}, there exist an open subdomain $U' \subset Z \subset \GG_m \times S$ and a morphism of rigid $S$-spaces $\alpha \colon U' \to G$ such that
\begin{enumerate}
\item $\GG_m(1) \times S \Subset_S U' \Subset_S \GG_m \times S$;
\item letting $\bar\alpha = \alpha|_{\GG_m(1) \times S}$, we have $\bar\alpha_0 = \bar\phi_0$.
\end{enumerate}
By \cref{relatively-compact-thickening}, $U'$ contains $\GG_m(\epsilon^2) \times S$ for some $\epsilon<1$, so we may assume that $U' = \GG_m(\epsilon^2) \times S$.
Increasing $\epsilon$ (and shrinking $\GG_m(\epsilon^2)$) if necessary, we may assume that there exists $\pi \in K^\times$ such that $\epsilon = \abs{\pi} < 1$, and that $\pi$ has a square root $\pi^{1/2} \in K$.
% (the existence of $\pi^{1/2}$ will be used in Step~6 of the proof).


\subsubsection*{Step~2: Set up of $U$}

% Let $\mu \colon H \to \GG_m(1)^g \times S$ be an isomorphism of $S$-groups, as in \cref{hom-extends-thm}(i).

Let
\[ \bar u = \bar\alpha \cdot \bar\phi^{-1} \colon \GG_m(1) \times S \to H. \]
(This means multiplication and inverse in the $S$-group law on $H$, not functional composition and inverse.)
Let $Z \subset H$ denote the preimage under the reduction map $H \to H_0$ of the image of the zero section $S_0 \to H_0$.
By (2), $\bar u_0$ is the zero homomorphism of $S_0$-group schemes $\GG_{m,S_0} \to H_0$, so $\im(\bar u) \subset Z$.

% By \cite[Thm.~1.3.2]{Berthelot}, ${]Z_0[_H} \cong D_+^g \times S$ where $D_+$ denotes the open unit disc.  (Apply \cite[Thm.~1.3.2]{Berthelot} with $P'=H_\formal$, $P=S_\formal$, $X=S_0$, $i=$ the inclusion $S_0 \to S_\formal$, $i'=$ the zero section $S_0 \to H_0$ composed with the inclusion $H_0 \to H_\formal$.)
% \martin{Under our assumption, that $H_\formal$ is a split formal $S_\formal$-torus, we don't need Berthelot here: we can see explicitly that ${]Z_0[_H} = {]1[_{\GG_m(1)^g}} \times S \cong D_+^g \times S$, via a translation which maps $1 \in \GG_m(1)^g$ to $0 \in D_+^g$.}

Let $\mu \colon H \to \AAA^g \times S$ denote the isomorphism $H \to \GG_m(1)^g \times S$ given by \cref{hom-extends-thm}(ii), composed with the translation sending $(1,\dotsc,1)$ to $(0,\dotsc,0)$.
Then $\mu$ restricts to an isomorphism $Z \to D_+^g \times S$, where $D_+$ denotes the open unit disc.

Let $\bar v$ denote the following composition:
\[ \GG_m(1) \times S  \mathrel{\overset{\bar u}{\longrightarrow}}  {Z}  \mathrel{\overset{\mu}{\longrightarrow}}  D_+^g \times S. \]
Let $p_i \colon \AAA^g \times S \to \AAA^1$ denote projection onto the $i$-th coordinate in $\AAA^g$.
Since $\GG_m(1) \times S$ is affinoid, by the maximum principle, each function $p_i \circ \bar v$ attains its maximum absolute value, which must be less than~$1$.
Thus the image of~$\bar v$ is contained in $D(r)^g \times S$ for some $r<1$, where $D(r) \subset D_+$ is the closed disc around~$0$ of radius~$r$.

% Alternatively: in Step~1, the approximation theorem gives us additional information that $u \cong 0 \colon S \to H$ mod some suitable $\pi$. We could choose that uniformiser appropriately to give r down here.

Increasing either $r$ or $\epsilon$ (and adjusting $\pi$ so that we still have $\abs{\pi} = \epsilon$), we may assume that $r=\epsilon^2 < 1$.

Let $U = \mu^{-1}(D(\epsilon)^g \times S) \subset {Z}$.
Thanks to our explicit formula for~$\mu$, we have $U = D(1,\epsilon)^g \times S \subset \GG_m(1)^g \times S \cong H$, where $D(1,\epsilon)$ denotes the closed disc of radius~$\epsilon$ and centre~$1$. Thus $U$ is an $S$-subgroup of~$H$.

Let $\zeta \colon U \to D^g \times S$ denote the following composition:
\[ U  \mathrel{\overset{\mu}{\longrightarrow}}   D(\epsilon)^g \times S  \mathrel{\overset{\sim}{\longrightarrow}}  D^g \times S, \]
where the second arrow is a rescaling of each factor.
Via $\zeta$, we can view $\AAA_{\formal}^g \times S_\formal$ as a formal model for $U$.
Explicitly, $\zeta^{-1} \colon D^g \times S \to U $ is given by
\[ \zeta^{-1}(x,s) = (1 + \pi x, s). \]

% \martin{I think this is where our assumption is very helpful: without it, the preimage of $D(\epsilon)^g \times S$ would not necessarily be an $S$-subgroup.  Instead we would need to define $U$ to be the smallest open $S$-subgroup of~$H$ containing this preimage, using a relative version of \cite[Theorem~7.2.3]{Lut16}.  Establishing that relative theorem looks like a lot of work.  We would have to hope that $U_0$ is unipotent as an $S_0$-group scheme, and that $U$ is isomorphic to a ball over~$S$.}

Since the image of $\bar v$ is contained in $D(\epsilon^2)^g \times S$, while $U$ maps to $D(\epsilon)^g \times S$, the image of~$\bar u$ reduces to the zero section in $U_0$.  Indeed, the image of~$\bar u$ is contained in~$\pi U$.


\subsubsection*{Step~3: Measuring how far $\alpha$ is from being a group homomorphism}

Define a morphism of rigid $S$-spaces $w_\alpha \colon \GG_m(\epsilon) \times \GG_m(\epsilon) \times S \to G$ by
\[ w_\alpha(x_1, x_2) = \alpha(x_1^{-1} x_2^{-1}) \alpha(x_1) \alpha(x_2). \]
Let $\bar w_\alpha$ denote the restriction of $w_\alpha$ to $\GG_m(1) \times \GG_m(1) \times S$.
Since $\bar\phi$ is a homomorphism of commutative $S$-groups, we have
\begin{equation} \label{eqn:barw-baru}
\bar w_\alpha(x_1, x_2) = \bar u(x_1^{-1} x_2^{-1}) \bar u(x_1) \bar u(x_2).
\end{equation}
Since the image of $\bar u$ is contained in $\pi U$, so is the image of $\bar w_\alpha$.

We claim that, after enlarging $\epsilon$, we may assume that $w_\alpha$ factorises through $U$.
Indeed, consider the set
\begin{multline*}
A = \{ (x_1,x_2,s) \in \GG_m(\epsilon) \times \GG_m(\epsilon) \times S :
\\ \abs{p_i(\mu(w_\alpha(x_1,x_2,s)))} \geq \epsilon \text{ for some } i=1,\dotsc,g \text{ and } \abs{x_1}, \abs{x_2} \leq 1 \}.
\end{multline*}
Assume that $A$ is non-empty.
If some point $(x_1,x_2,s) \in A$ satisfies $\abs{x_1x_2} = 1$, then it must satisfy $\abs{x_1}=\abs{x_2}=1$, i.e.\ $x_1, x_2 \in \GG_m(1)$.
Since $\bar w_\alpha$ maps $\GG_m(1) \times \GG_m(1) \times S$ into $\pi U$, it follows that $\abs{p_i(\mu(w_\alpha(x_1,x_2,s)))} \leq \epsilon^2 < \epsilon$ for all~$i$, contradicting $(x_1,x_2,s) \in A$.
Thus $\abs{x_1x_2} < 1$ on~$A$.

The set~$A$ is a finite union of affinoids (one for each~$i$).
Hence, by the maximum principle, $\abs{x_1x_2}$ attains its maximum on~$A$.
Thus there exists $\epsilon' < 1$ such that $\abs{x_1x_2} \leq \epsilon'$ on~$A$.
Now if $(x_1,x_2,s) \in \GG_m(\epsilon'^{1/3}) \times \GG_m(\epsilon'^{1/3}) \times S$, we have $\abs{x_1x_2} \geq \epsilon'^{2/3} > \epsilon'$ so $(x_1,x_2,s) \not\in A$.
But $\abs{x_1},\abs{x_2} \leq 1$, so such $(x_1,x_2,s)$ must satisfy that $\abs{p_i(\mu(w_\alpha(x_1,x_2,s)))} < \epsilon$ for all~$i$, so $w_\alpha(x_1,x_2,s) \in U$.

Thus, replacing $\epsilon$ by $\max\{ \epsilon, \epsilon'^{1/3} \}$ (which makes $\GG_m(\epsilon)$ smaller and $U$ bigger), we obtain that $w_\alpha \colon \GG_m(\epsilon) \times \GG_m(\epsilon) \times S \to G$ factors through~$U$.

\subsubsection*{Step~4: Formal power series for $\bar u$}

Let $\bar u_i \colon \GG_m(1) \times S \to D$ denote the composition
\[ \GG_m(1) \times S  \overset{\bar u}{\longrightarrow}  U  \overset{\zeta}{\longrightarrow}  D^g \times S  \overset{p_i}{\longrightarrow}  D. \]
Then $\bar u_i \in \cO^\circ(\GG_m(1) \times S) = \tatealgebra{\cO^\circ(S)}{\xi, \xi^{-1}}$.
Thus we can write
\begin{equation} \label{eqn:ui-coeffs}
\bar u_i = \sum_{j \in \ZZ} \bar u_{ij} \xi^j
\end{equation}
where $\bar u_{ij} \in \cO^\circ(S)$ and $\abs{u_{ij}}_S \to 0$ as $\abs{j} \to \infty$.

The aim of the next step of this proof is to show that the Laurent series \eqref{eqn:ui-coeffs} converge on $\GG_m(\epsilon^2) \times S$.
In other words, we will show that $\epsilon^{-2\abs{j}}\abs{u_{ij}}_S \to 0$ as $\abs{j} \to \infty$.

% \subsubsection*{Step~5: Formal power series for $\bar w$}

% Similarly to what we did with $\bar u$, define $w_i$ to be the composition
% \[ \GG_m(\epsilon) \times \GG_m(\epsilon) \times S  \mathrel{\overset{w}{\longrightarrow}}  U  \overset{q}{\longrightarrow}  D^g \times S  \to  D. \]
% We have $w_i \in \cO(\GG_m(\epsilon) \times \GG_m(\epsilon) \times S) = \tatealgebra{\cO(S)}{\pi\xi_1, \pi\xi_1^{-1}, \pi\xi_2, \pi\xi_2^{-1}}$.
% Thus we can write
% \begin{equation} \label{eqn:wi-coeffs}
% w_i = \sum_{j,k \in \ZZ} w_{ijk} \xi_1^j \xi_2^k
% \end{equation}
% where $w_{ijk} \in \cO(S)$ and $\epsilon^{-\abs{j}-\abs{k}}\abs{w_{ijk}}_S \to 0$ as $\abs{j}+\abs{k} \to \infty$.
% Furthermore, because the image of $w_i$ is in~$D$, we have $\epsilon^{-\abs{j}-\abs{k}}\abs{w_{ijk}}_S \leq 1$ for all $j,k \in \ZZ$.


% \subsubsection*{Step~6: Formal group law and congruences}

% Recall that the group law on~$U$ is induced (via~$q$) by a formal $S_\formal$-group law~$F$ on $\AAA_{\formal}^g \times S_\formal$.  The components of this formal group law have the form
% \begin{equation} \label{eqn:Fi-qi}
% F_i(\underline X, \underline Y, s) = X_i + Y_i + q_i(\underline X, \underline Y, s)
% \end{equation}
% where $q_i(\underline X, \underline Y, s) \in \tatealgebra{\cO(S)}{\underline X, \underline Y}$ with every monomial being of degree at least~$2$.

% Consequently, if $x, y \in D(\abs{\pi}^\lambda)^g \times S $, then
% \begin{equation} \label{eqn:F-congruence}
% F(x,y,s) \cong (x + y, s) \bmod \pi^{2\lambda}.
% \end{equation}

% Suppose that the image of $\bar u$ is contained in $\pi^\lambda U := q^{-1}(D(\epsilon^\lambda)^g \times S)$ for some~$\lambda \in \ZZ_{\geq 1}$ (we shall use this hypothesis inductively).
% Explicitly, in our case, $\pi^\lambda U = D(1, \epsilon^{\lambda+1})^g \times S$.

% Then, from \eqref{eqn:barw-baru} and \eqref{eqn:F-congruence}, we obtain the following congruence of Laurent series for each $i=1, \dotsc, g$:
% \begin{align*}
%     \bar w_i(\xi_1, \xi_2, s)
%   & = F \bigl( \bar u_i(\xi_1^{-1} \xi_2^{-1}), F \bigl(\bar u_i(\xi_1), \bar u_i(\xi_2) \bigr) \bigr)
% \\& \equiv \bar u_i(\xi_1^{-1} \xi_2^{-1}) + \bar u_i(\xi_1) + \bar u_i(\xi_2) \bmod \pi^{2\lambda}.
% \end{align*}
% Comparing the coefficients of $\xi_1^{-j}\xi_2^{-j}$, we obtain
% \[ w_{ijj} \cong \bar u_{i,-j} \bmod \pi^{2\lambda} \]
% for all $i = 1, \dotsc, g$ and all $j \in \ZZ$.

% Hence, from step~5, we conclude that if the image of $\bar u$ is contained in $\pi^\lambda U$, then
% \[ \epsilon^{-2\min\{\lambda, \abs{j}\}}\abs{\bar u_{ij}}_S \leq 1 \]
% for all $j \in \ZZ$.

\subsubsection*{Step~5: Induction}
We claim that
\begin{equation} \label{eqn:inductive-hypothesis}
\epsilon^{-2\min\{\lambda, \abs{j}\}}\abs{\bar u_{ij}}_S \leq 1
\end{equation}
for all $j \in \ZZ$ and all $\lambda \in \ZZ_{\geq 1}$.
We shall establish this by induction on~$\lambda$.

For the base case $\lambda=1$, we saw at the end of Step~2 that the image of $\bar u$ is contained in $\pi U$.
Hence, by \cref{coeffs-congruence-lemma}, \eqref{eqn:inductive-hypothesis} holds for $\lambda=1$.

For the inductive step, assume that \eqref{eqn:inductive-hypothesis} holds for some $\lambda \geq 1$.
We shall show that it holds for $\lambda+1$.

For $i=1,\dotsc,g$, define $u_{\lambda i}$ to be the truncated Laurent series:
\[ u_{\lambda i}(\xi) = \sum_{j=-\lambda}^\lambda \bar u_{ij} \xi^j \in \cO^\circ(S)[\xi, \xi^{-1}]. \]
This defines a holomorphic function $u_{\lambda i} \in \cO(\GG_m(\epsilon^2) \times S)$.
By the inductive hypothesis~\eqref{eqn:inductive-hypothesis}, all non-zero coefficients of $u_{\lambda i}$
 satisfy $\epsilon^{-2\abs{j}}\abs{\bar u_{ij}}_S \leq 1$.
Hence
\[ u_{\lambda i}(\GG_m(\epsilon^2) \times S) \subset D. \]
We can therefore define a morphism of rigid $S$-spaces
\[ u_\lambda = \zeta^{-1} \circ (u_{\lambda 1}, \dotsc, u_{\lambda g}, s) \colon \GG_m(\epsilon^2) \times S \to U. \]

By the inductive hypothesis~\eqref{eqn:inductive-hypothesis}, the coefficients of~$\bar u_i$ which have been removed in $u_{\lambda i}$ satisfy $\epsilon^{-2(\lambda+1)}\abs{\bar u_{ij}}_S \leq 1$, so
\begin{equation} \label{eqn:ui-congruence}
\bar u_i \equiv u_{\lambda i} \bmod \pi^{2(\lambda+1)}
\end{equation}
as Laurent series, and hence as holomorphic functions on $\GG_m(1) \times S$.
% \martin{\cite{Lut16} just has mod~$\pi^{\lambda+1}$ here, apparently deduced from mod~$\pi^{2\lambda}$.}

Define
\begin{align*}
    \beta_\lambda & = \alpha \cdot u_\lambda^{-1} \colon \GG_m(\epsilon^2) \times S \to G,
\\  \bar\beta_\lambda & = \beta_\lambda|_{\GG_m(1) \times S},
\\  \bar v_\lambda & = \bar\beta_\lambda \cdot \bar\phi^{-1} \colon \GG_m(1) \times S \to G.
\end{align*}
We can calculate $\bar v_\lambda = \bar u \cdot u_\lambda^{-1}$, so the image of $\bar v_\lambda$ is contained in~$U$.
Hence we can define $\bar v_{\lambda i} \colon \GG_m(1) \times S \to D$ to be the $i$-th projection of $\zeta \circ v_\lambda$.

Let $\iota$ denote the inverse for the formal group law~$F$.
By \eqref{eqn:ui-congruence} and \cref{formal-group-Fi-congruence}, we have 
\[ \bar v_{\lambda i}(\xi) = F_i\bigl( \bar u_1(\xi), \dotsc, u_g(\xi), \, \iota \bigl( u_{\lambda 1}(\xi), \dotsc, u_{\lambda g}(\xi) \bigr) \bigr)  \equiv   0  \bmod \pi^{2(\lambda+1)}. \]
% \martin{As before, \cite{Lut16} just has mod~$\pi^{\lambda+1}$ here.}
Thus $\bar v_\lambda$ maps $\GG_m(1) \times S$ into $\pi^{2(\lambda+1)} U$.
A fortiori, $\bar v_\lambda$ maps $\GG_m(1) \times S$ into $\pi^{(\lambda+1)} U$.

If we define $w_{\beta_\lambda} \colon \GG_m(\epsilon) \times \GG_m(\epsilon) \times S \to G$ by
\[ w_{\beta_\lambda}(x_1, x_2) = \beta_\lambda(x_1^{-1}x_2^{-1}) \cdot \beta_\lambda(x_1) \cdot \beta_\lambda(x_2), \]
then we can calculate
\[ w_{\beta_\lambda}(x_1, x_2) = w_\alpha(x_1, x_2) \cdot u_\lambda(x_1^{-1}x_2^{-1}) \cdot u_\lambda(x_1) \cdot u_\lambda(x_2). \]
Hence the image of $w_{\beta_\lambda}$ is contained in~$U$.

Therefore, we can apply \cref{coeffs-congruence-lemma} to $\bar\beta_\lambda$ instead of $\bar\alpha$ (thus replacing $\bar u$ by $\bar v_\lambda$) to deduce that
\begin{equation} \label{eqn:vij-bound}
\epsilon^{-2\min\{\lambda+1, \abs{j}\}} \abs{\bar v_{\lambda ij}}_S \leq 1
\end{equation}
for all~$j \in \ZZ$.

The components of the formal group law~$F$ have the form
\begin{equation} \label{eqn:Fi-qi}
F_i(\underline X, \underline Y, s) = X_i + Y_i + q_i(\underline X, \underline Y, s)
\end{equation}
where $q_i(\underline X, \underline Y, s) \in \tatealgebra{\cO^\circ(S)}{\underline X, \underline Y}$ with every monomial in~$q_i$ divisible by at least one~$X_{i'}$ and by at least one~$Y_{i'}$.

Since $\bar u = \bar v_\lambda \cdot u_\lambda$, we have
\begin{align} \label{eqn:ui-vi}
    \bar u_i(\xi)
  & = F_i \bigl( \bar v_{\lambda 1}(\xi), \dotsc, \bar v_{\lambda g}(\xi), u_{\lambda 1}(\xi), \dotsc, u_{\lambda g}(\xi) \bigr)
\notag
\\& = \bar v_{\lambda i}(\xi) + u_{\lambda i}(\xi) + q_i \bigl( \bar v_{\lambda 1}(\xi), \dotsc, \bar v_{\lambda g}(\xi), u_{\lambda 1}(\xi), \dotsc, u_{\lambda g}(\xi) \bigr).
\end{align}
Write
\[ q_i \bigl( \bar v_{\lambda 1}(\xi), \dotsc, \bar v_{\lambda g}(\xi), u_{\lambda 1}(\xi), \dotsc, u_{\lambda g}(\xi) \bigr) = \sum_{j \in \ZZ} q_{\lambda ij} \xi^j \]
where $q_{\lambda ij} \in \cO^\circ(S)$.
The non-zero coefficients of $u_{\lambda i}$ satisfy $\abs{\bar u_{ij}}_S \leq \epsilon^{2\abs{j}}$, and the coefficients of $\bar v_{\lambda i}$ satisfy $\abs{\bar v_{ij}}_S \leq \epsilon^{2\abs{j}}$.
Since $\tatealgebra{\cO^\circ(S)}{\pi^2\xi, \pi^{2}\xi^{-1}}$ is a ring, it follows that every monomial in $\bar v_{\lambda 1}(\xi), \dotsc, \bar v_{\lambda g}(\xi), u_{\lambda 1}(\xi), \dotsc, u_{\lambda g}(\xi)$ lies in $\tatealgebra{\cO^\circ(S)}{\pi^2\xi, \pi^{2}\xi^{-1}}$, hence that  $\abs{q_{\lambda ij}}_S \leq \epsilon^{2\abs{j}}$ for all~$j$.

Since also $\abs{\bar u_{ij}}_S \leq \epsilon^{2\abs{j}} \leq 1$ and $\abs{\bar v_{\lambda ij}}_S \leq \epsilon^{2(\lambda+1)}$ for all $i,j$, and every term of $q_i$ is divisible by some $\bar v_{\lambda i'}$, we obtain $\abs{q_{\lambda ij}}_S \leq \epsilon^{2(\lambda+1)}$.
Thus we get
\[ \abs{q_{\lambda ij}}_S \leq \epsilon^{2\min\{\lambda+1, \abs{j}\}} \]
for all $j \in \ZZ$.

Using \eqref{eqn:ui-vi}, \eqref{eqn:vij-bound}, and the fact that non-zero coefficients of $u_{\lambda i}$ (namely, $\bar u_{ij}$ for $\abs{j} \leq \lambda$) satisfy $\abs{\bar u_{ij}}_S \leq \epsilon^{2\abs{j}} = \epsilon^{2\min\{\lambda+1, \abs{j}}$, we obtain
\[ \abs{\bar u_{ij}}_S \leq \epsilon^{2\min\{\lambda+1, \abs{j}\}} \]
for all $j \in \ZZ$, that is, \eqref{eqn:inductive-hypothesis} holds for~$\lambda+1$, completing the inductive step.

\subsubsection*{Step~6: Conclusion}

We now know that \eqref{eqn:inductive-hypothesis} holds for all $\lambda \in \ZZ_{\geq1}$.
Thus
\[ \epsilon^{-2\abs{j}} \abs{\bar u_{ij}}_S \leq 1 \]
for all $i=1,\dotsc,g$ and all $j \in \ZZ$.
Hence $\bar u_i$ converges on
\[ \{ (x,s) \in \GG_m \times S : \epsilon^2 < \abs{x} < \epsilon^{-2} \}. \]
% \martin{\cite{Lut16} claims that $\bar u$ converges on $\GG_m(\epsilon^2)$, but I don't think we have shown that: to get convergence on $\abs{x}=\epsilon^{\pm2}$, we need $\epsilon^{-2\abs{j}} \abs{\bar u_{ij}}_S \to 0$ as $\abs{j} \to \infty$.}

Thus $\bar u_i$ defines a morphism of rigid spaces $\GG_m(\epsilon) \times S \to D$ for each~$i$, so $\bar u$ defines a morphism of rigid $S$-spaces $\GG_m(\epsilon) \times S \to U$.
Define
\[ \psi = \alpha \cdot \bar u^{-1} \colon \GG_m(\epsilon) \times S \to G. \]
By the definition of $\bar u$, $\psi|_{\GG_m(1) \times S} = \bar\phi$.
Since $\bar\phi$ is an $S$-group homomorphism, the identity principle implies that
\begin{equation} \label{eqn:psi-partial-hom}
\psi(\xi_1) \cdot \psi(\xi_2) = \psi(\xi_1 \cdot \xi_2)
\end{equation}
for all $\xi_1, \xi_2 \in \GG_m(\epsilon^{1/2}) \times S$.

Define $A_n \subset \GG_m$ to be the annulus
\[ A_n = \pi^{n/2}\GG_m(\epsilon^{1/2}) = \{ x \in \GG_m : \epsilon^{(n+1)/2} \leq \abs{x} \leq \epsilon^{(n-1)/2} \}. \]
The sets $A_n$ for $n \in \ZZ$ form an admissible cover of $\GG_m$, with non-empty intersections
\[ A_n \cap A_{n+1} = \{ x \in \GG_m : \epsilon^{(n+1)/2} \leq \abs{x} \leq \epsilon^{n/2} \} \]
and
\[ A_n \cap A_{n+2} = \{ x \in \GG_m : \abs{x} = \epsilon^{(n+1)/2} \}. \]

Define $\psi_n \colon A_n \times S \to G$ by
\[ \psi_n(\pi^{n/2} \xi) = \psi(\pi^{1/2})^n \psi(\xi) \]
for all $\xi \in \GG_m(\epsilon^{1/2})$.
If $(y,s) \in (A_n \cap A_{n+1}) \times S$, with $y = \pi^{n/2} \xi = \pi^{(n+1)/2}\pi^{-1/2}\xi$, then
\[ \psi_n(y) = \psi(\pi^{1/2})^n \psi(\xi) = \psi(\pi^{1/2})^n \psi(\pi^{1/2}) \psi(\pi^{-1/2}\xi) = \psi_{n+1}(y), \]
where the middle equality is \eqref{eqn:psi-partial-hom}, valid because $\pi^{1/2}, \pi^{-1/2}\xi \in \GG_m(\epsilon^{1/2})$.
Applying this twice shows that $\psi_n$ and $\psi_{n+2}$ agree on $A_n \cap A_{n+2}$.

Thus we can glue the $\psi_n$ together into a morphism of rigid $S$-spaces $\phi \colon \GG_m \times S \to G$.
As a consequence of \eqref{eqn:psi-partial-hom} and the definition of $\psi_n$, $\phi$ is an $S$-group homomorphism.

Uniqueness of $\phi$ holds by the identity principle for morphisms of rigid spaces.
\end{proof}


\section{Proof of corollary}

Now we prove \cref{hom-extends-no-boundary}.
It suffices to assume that $S$ is affinoid and that $G$ has no boundary, since we could prove the corollary on each affinoid subspace~$S_i$ from \cref{hom-extends-no-boundary}(ii), and glue the resulting homomorphisms $\phi$.

Write $\xi_1, \dotsc, \xi_g \colon \GG_m \times S \to \GG_m^g \times S$ for the inclusions of the direct factors $\GG_m$.
Write $\bar\xi_i$ for the restriction of $\xi_i$ to $\GG_m(1) \times S$.

We apply \cref{hom-extends-thm} to $\bar\phi_i := \bar\phi \circ \bar\xi_i \colon \GG_m(1) \times S \to G$.
By the hypotheses of \cref{hom-extends-no-boundary}, $\bar\phi_i$ factors through~$H$, which is isomorphic to $\GG_m(1)^g \times S$ (via $\bar\phi$).
Since $H \cong \GG_m(1)^g \times S$ is affinoid and $G$ has no boundary, there exists an open affinoid subspace $G' \subset G$ such that $H \Subset G'$.
It follows that $H \Subset_S G'$ (choose a formal $R$-model $G'_\formal \to S_\formal \to \Spf(R)$ such that $H$ is induced by an open subscheme $H_\formal \subset G'_\formal$; then $\ov H_0 \to \Spec(k)$ is proper, so $\ov H_0 \to S_0$ is proper by \cite[Cor.~II.4.8(e)]{Har75}, where $\ov H_0$ denotes the schematic closure of $\ov H_0$ in $\ov G'_0$.)
Thus $\bar\phi_i$ and~$H$ satisfy conditions (i) and~(ii) of \cref{hom-extends-thm}.

Thus, by \cref{hom-extends-thm}, for each $i = 1,\dotsc,g$, we obtain a unique homomorphism of rigid $S$-groups $\phi_i \colon \GG_m \times S \to G$ whose restriction to $\GG_m(1) \times S$ is equal to $\bar\phi_i$.
By the universal property of fibre products, there is a unique $\phi \colon \GG_m^g \times S \to G$ satisfying $\phi \circ \xi_i = \phi_i$ for each~$i$, completing the proof of \cref{hom-extends-no-boundary}.

% The hypothesis that $S$ is affinoid is present in the above theorem because restricting to an open subspace of~$S$ does not preserve the property that $G$ has no boundary.
% (We would want $G$ to ``have no boundary relatively over~$S$'', or more precisely, there exists an admissible covering of~$S$ by affinoids $S_i$ such that, for each~$i$, $G \times_S S_i$ has no boundary.  This is satisfied in the case where $G$ is the analytification of a smooth commutative group scheme of finite type over a separated $K$-scheme $S$ of finite type, by \cite[Cor.~5.11]{Lut90}.)


\bibliographystyle{amsalpha}
\bibliography{rigid}

\end{document}
