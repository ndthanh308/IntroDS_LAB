\section{Introduction}\label{sec:introduction}

A \emph{(bibliographic) citation} refers to a conceptual (directional) link that connects a research work (usually a publication) which contains a reference to (i.e., ``cites'') another work (which is being ``cited''). During the last decades, citations have become one of the most important types of bibliographic metadata~\cite{peroni-qss}. The main reason for that is that they are often considered as proxies of scientific impact, since a citation can be interpreted as an acknowledgement for the contribution of the cited work into the citing one (although this might not always be the case~\cite{YTN2017,AER2013}). As a result, they have been instrumental in scientometrics, becoming the basis for the calculation of various research impact indicators~\cite{bipdb}. Such indicators have been used to facilitate scientific knowledge discovery (e.g., they have been used by academic search engines to help researchers prioritise their reading~\cite{bip-finder}), monitor research production~\cite{observatory}, assist research assessment processes, and in many other applications.

Various sources of citation data have become available during the previous decades to address the needs of use-cases like the aforementioned ones. Apart from proprietary and restrictive sources, like Clarivate Analytics' Web of Science, Google Scholar and the Microsoft Academic Graph (MAG)~\cite{mag-qss}, due to the raised popularity of the Open Science movement, a couple of open datasets that provide citations (e.g., OpenCitations\footnote{OpenCitations: \href{https://opencitations.net}{opencitations.net}}, the OpenAIRE Graph\footnote{OpenAIRE Graph: \href{https://graph.openaire.eu}{graph.openaire.eu}}) have also become available during the last years. 
Almost all of them report citations as DOI-to-DOI pairs, failing to cover citations that involve publications  for which a DOI has not been assigned. This may not be a significant problem for many disciplines, but in Computer Science, a considerable number of conferences and workshops do not assign DOIs to their papers. In addition, in this field, conference and workshop papers are peer reviewed and, historically, serve as important contributions, carrying significant weight in research assessment processes. As a result, if they are not considered during citation analyses, this can overlook an important part of scientific production and even introduce bias. In the past, Microsoft Academic Graph (MAG) was partially covering this gap by also offering citations for papers that do not have a DOI.  However, since its discontinuation in December $2021$, this data collection is no longer maintained and updated, thus its coverage is continuously declining.

In this work, we introduce BIP! NDR, an open dataset that aims to cover this gap, improving research assessment processes and other relevant applications within the field of Computer Science. The dataset is constructed based on a workflow that identifies and retrieves Open Science publications lacking DOIs from DBLP\footnote{DBLP: \url{https://dblp.uni-trier.de/}}, the most widely known bibliographic database for publications from Computer Science, and then performs text analysis to extract citation information directly from the respective manuscripts. The current version of the dataset contains more than $510$K citations made by approximately $60$K Computer Science conference or workshop papers that,
according to DBLP, do not have a DOI. We plan to frequently update the dataset so that it can become an important resource for citations in Computer Science that are missing from the most important citation datasets. This is a valuable addition to the toolboxes of scientometricians so that they can perform more concrete analysis in the Computer Science domain. 

\noindent \textbf{Outline.} The rest of the manuscript is organized as follows: in Section~\ref{sec3-methodology} we elaborate on the technical details related to the production of the BIP! NDR dataset; in Section~\ref{sec4-dataset-structure-stats} we discuss the structure of the dataset; finally, in Section~\ref{sec5-conclusions} we conclude the work while also discussing future planned extensions.