

% \section{Notations}


\section{Preliminary}
In this section, we first introduce the problem background of the graph/hypergraph isomorphism. Then, we briefly review the most related graph Weisfeiler-Lehman kernel. The detailed descriptions of notations are in Table \ref{tab:nots}.


\begin{table}%[!htbp]
    \centering
    \label{tab:nots}
    \caption{Notations and associated descriptions in this paper.}
    \begin{tabular}{cc}
    \toprule
    Notations & Descriptions \\
    \midrule
    $G$ & A graph \\
    $V$ & Vertex set of a graph \\
    $E$ & Edge set of a graph \\
    $\ell(v)$ & Label of vertex $v$ \\
    $M_\cdot(v)$ & Multiset with respect to vertex $v$ \\
    $\mathcal{N}(v)$ & Set of neighbor vertices of vertex $v$ \\
    \midrule
    $\mathcal{G}$ & A hypergraph \\
    $\mathcal{V}$ & Vertex set of a hypergraph \\
    $\mathcal{E}$ & Hyperedge set of a hypergraph \\
    $\mathbf{H}$ & Hypergraph incidence matrix \\
    $\ell^v(v)$ & Label of vertex $v$ \\
    $\ell^e(v)$ & Label of hyperedge $e$ \\
    $\mathcal{N}_e(v)$ & Set of vertex's hyperedge neighbors of vertex $v$ \\
    $\mathcal{N}_v(e)$ & Set of hyperedge's vertex neighbors of hyperedge $e$ \\
    $M^e_\cdot(v)$ & Multiset of hyperedges with respect to vertex $v$ \\
    $M^v_\cdot(e)$ & Multiset of vertices with respect to hyperedge $e$ \\
    \midrule
    $h$ & Number of iterations \\
    $N$ & Number of hypergraphs \\
    $\phi(\cdot)$ & Feature map \\
    $c(X, \sigma)$ & Number of sub-structure $\sigma$ in structure $X$ \\
    $k_\cdot(x, x')$ & Kernel function with respect to $x$ and $x'$ \\
    \bottomrule
    \end{tabular}
\end{table}

\subsection{Graph/Hypergraph Isomorphism}

\textbf{Graph Isomorphism:} 
Given two graphs $G = \{V, E\}$ and $G' = \{V', E'\}$, the target of the graph isomorphism test (denoted by $G \cong G'$) is to find whether a bijective mapping $g := V \rightarrow V'$ exists. The mapping $g$ is called the isomorphism function, such that 
\begin{equation}
\nonumber
    (v_i, v_j) \in E \iff (g(v_i), g(v_j)) \in E' .
\end{equation}

However, the complete graph isomorphism test solution has been proven to be the NP-hard problem \cite{graph_iso_np_hard}, thus making it practically infeasible. The Weisfeiler-Lehman test \cite{graph_wl}, also known as ``naive vertex refinement'', is a typical approximate solution of the graph isomorphism test, as described in Algorithm \ref{alg:g_wl}. It starts with two input graphs $G$ and $G'$ associated with labeled vertices. Each vertice gathers its neighbor vertices' labels in each iteration to build a subtree string, which is used to relabel the vertex. For each iteration, if the statics of different types of vertex labels of the two graphs are different, the algorithm is then determined, and the two graphs are not isomorphism graphs. Otherwise, the two graphs may be isomorphisms to each other. 


\begin{algorithm}[!htbp]
	\renewcommand{\algorithmicrequire}{\textbf{Input:}}
	\renewcommand{\algorithmicensure}{\textbf{Output:}}
	\caption{Weisfeiler-Lehman test of graph isomorphism.}
	\label{alg:g_wl}
	\begin{algorithmic}
		\Require Graph ${G}=\{{V}, {E} \}$ and ${G'} = \{ {V}', {E}' \}$, vertex $v \in {V}$, and $v' \in {V}'$, vertex label map: $\ell := v/v' \rightarrow c$, hash function $f := s \rightarrow c$, $\cdot / \cdot$ operation denotes the `` or ''.
            \State \emph{1. Multiset-label determination.}
            \begin{itemize}
                \item For $i=0$, set $M_i(v) := l_0(v)=\ell(v)$.
                \item For $i>0$, assign a multiset-label $M_i(v) = \{ l_{i-1}(u) \mid u \in \mathcal{N}(v) \}$ to each node $v$ in $G$ and $G'$.
            \end{itemize}
            \State \emph{2. Sorting each multiset.}
            \begin{itemize}
                \item Sort elements in $M_i(v)$ and transfer to string $s_i(v)$. 
                \item Add $l_{i-1}(v)$ as prefix to $s_i(v)$.
            \end{itemize}
            \State \emph{3. Label compression.}
            \begin{itemize}
                \item Sort all of the string $s_i(v)$ for all $v$ in $G$ and $G'$.
                \item Map each string $s_i(v)$ to a new compressed label.
            \end{itemize}
            \State \emph{4. Relabeling.}
            \begin{itemize}
                \item Set $l_i(v) := f(s_i(v))$ for all vertices in $G$ and $G'$.
            \end{itemize}
	\end{algorithmic}  
\end{algorithm}



\textbf{Hypergraph Isomorphism:} Similarly, given two hypergraphs $\mathcal{G} = \{ \mathcal{V}, \mathcal{E} \}$ and $\mathcal{G}' = \{ \mathcal{V}', \mathcal{E}' \}$, the target of the hypergraph isomorphism test  (denoted by $\mathcal{G} \cong \mathcal{G}'$) is to find whether a bijective mapping $g := \mathcal{V} \rightarrow \mathcal{V}'$ exists. The mapping $g$ is called the isomorphism function, such that 
\begin{equation}
\nonumber
    (v_1, v_2, \cdots, v_m) \in \mathcal{E} \iff (g(v_1), g(v_2), \cdots, g(v_m)) \in \mathcal{E}' .
\end{equation}


\subsection{Graph Weisfeiler-Lehman Kernel}
Based on the Weisfeiler-Lehman test of graph isomorphism, the graph Weisfeiler-Lehman subtree kernel \cite{graph_wl_subtree} and graph Weisfeiler-Lehman edge kernel \cite{graph_wl_all} are proposed, respectively. The core motivation behind the two kernel methods is mapping the uncomparable graph structure into a feature vector in feature space, where the distance between two graphs can be calculated by the inner product of the two feature vectors as follows
\begin{equation}
    k(G, G') = \phi(G)^\top \phi(G') .
\end{equation}

\subsubsection{Graph Weisfeiler-Lehman Subtree Kernel}
Due to the vertex label in each iteration corresponding to a unique subtree structure, the simple idea behind the graph Weisfeiler-Lehman subtree kernel is to count the number of different types of labels in the graphs. Thus, the feature function $\phi$ can be defined as
\begin{equation}
\nonumber
    \phi(G) = (c(G, \sigma_{00}), \cdots, c(G, \sigma_{0|\Sigma_0|}), \cdots, c(G, \sigma_{h|\Sigma_h|})) ,
\end{equation}
where $c(G, \sigma)$ is the counting function. $\sigma_{ji}$ is the corresponding label of the $j$-th subtree in $i$-th iteration, and $\Sigma_i$ is the number of different subtree types in $i$-th iteration, with a maximum $h$ numbner of iterations. 


\subsubsection{Graph Weisfeiler-Lehman Edge Kernel}
Furthermore, the graph Weisfiler-Lehman edge kernel counts matching pairs of edges with identically labeled endpoints, which can be formulated as
\begin{equation}
\nonumber
    \phi(G) = (c(G, \sigma_{0k_0} \rightarrow \sigma_{0k_0'}), \cdots, c(G, \sigma_{hk_h} \rightarrow \sigma_{hk'_h})) ,
\end{equation}
where $k_i, k'_i \in [1, |\Sigma_i|]$. $\sigma_{ik_i} \rightarrow \sigma_{ik_i'}$ denotes an edge with two labeled vertices: $\sigma_{ik_i}$ and $\sigma_{ik'_i}$. 
