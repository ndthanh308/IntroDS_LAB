\ifCLASSOPTIONcaptionsoff
  \newpage
\fi

\begin{thebibliography}{1}

\bibitem{graph_social_network}
Myers S A, Sharma A, Gupta P, et al. Information network or social network? The structure of the Twitter follow graph[C]. Proceedings of the 23rd International Conference on World Wide Web. 2014: 493-498.

\bibitem{graph_brain}
Gao Y, Zhang Z, Lin H, et al. Hypergraph learning: Methods and practices[J]. IEEE Transactions on Pattern Analysis and Machine Intelligence, 2020, 44(5): 2548-2566.

\bibitem{graph_collabrative_network}
Newman M E J. Scientific collaboration networks. I. Network construction and fundamental results[J]. Physical review E, 2001, 64(1): 016131.

\bibitem{graph_knowledge_network}
Chen Z, Wang Y, Zhao B, et al. Knowledge graph completion: A review[J]. Ieee Access, 2020, 8: 192435-192456.

\bibitem{graph_pathways}
McDermott M J, Dwaraknath S S, Persson K A. A graph-based network for predicting chemical reaction pathways in solid-state materials synthesis[J]. Nature communications, 2021, 12(1): 3097.

\bibitem{graph_protein}
Vishveshwara S, Brinda K V, Kannan N. Protein structure: insights from graph theory[J]. Journal of Theoretical and Computational Chemistry, 2002, 1(01): 187-211.

\bibitem{hg_social_media}
Bu J, Tan S, Chen C, et al. Music recommendation by unified hypergraph: combining social media information and music content[C]. Proceedings of the 18th ACM international conference on Multimedia. 2010: 391-400.

\bibitem{hgnn}
Feng Y, You H, Zhang Z, et al. Hypergraph neural networks[C]. Proceedings of the AAAI conference on artificial intelligence. 2019, 33(01): 3558-3565.

\bibitem{hgnnp}
Gao Y, Feng Y, Ji S, et al. HGNN $^+ $: General Hypergraph Neural Networks[J]. IEEE Transactions on Pattern Analysis and Machine Intelligence, 2022.

\bibitem{dhgnn}
Jiang J, Wei Y, Feng Y, et al. Dynamic Hypergraph Neural Networks[C]. IJCAI. 2019: 2635-2641.

\bibitem{hg_coauthor}
Lung R I, Gaskó N, Suciu M A. A hypergraph model for representing scientific output[J]. Scientometrics, 2018, 117: 1361-1379.

\bibitem{hg_encoder}
Fan H, Zhang F, Wei Y, et al. Heterogeneous hypergraph variational autoencoder for link prediction[J]. IEEE Transactions on Pattern Analysis and Machine Intelligence, 2021, 44(8): 4125-4138.

\bibitem{graph_iso_np_hard}
Gärtner, T., Flach, P., \& Wrobel, S. (2003). On graph kernels: Hardness results and efficient alternatives. COLT (pp. 129–143). Springer.

\bibitem{graph_wl}
Weisfeiler B, Leman A. The reduction of a graph to canonical form and the algebra which appears therein[J]. nti, Series, 1968, 2(9): 12-16.

\bibitem{graph_k_wl}
Jin-yi Cai, Martin F¨urer, and Neil Immerman, An optimal lower bound on the number of variables for graph identification, Comb. 12 (1992), no. 4, 389–410.

\bibitem{hgnnp}
Y. Gao, Y. Feng, S. Ji and R. Ji, HGNN+: General Hypergraph Neural Networks, in IEEE Transactions on Pattern Analysis and Machine Intelligence, vol. 45, no. 3, pp. 3181-3199, 1 March 2023, doi: 10.1109/TPAMI.2022.3182052.

\bibitem{multiset}
Biggs N L. Discrete mathematics[M]. Oxford University Press, 2002.

\bibitem{radix_sort}
K. Mehlhorn. Data Structures and Efficient Algorithms. Springer, 1984.

\bibitem{graph_wl_valid}
L. Babai and L. Kucera. Canonical labelling of graphs in linear average time. In Proceedings Symposium on Foundations of Computer Science, pages 39–46, 1979.

\bibitem{graph_wl_subtree}
N. Shervashidze and K. M. Borgwardt. Fast subtree kernels on graphs. In  Proceedings of the Conference on Advances in Neural Information Processing Systems, pages 1660–1668, 2009.

\bibitem{svm_trick}
Cristianini, N., \& Shawe-Taylor, J. (2000). An introduction to support vector machines. Cambridge University Press.

\bibitem{graph_wl_all}
Shervashidze N, Schweitzer P, Van Leeuwen E J, et al. Weisfeiler-lehman graph kernels[J]. Journal of Machine Learning Research, 2011, 12(9).

\bibitem{graphlet}
Shervashidze N, Vishwanathan S V N, Petri T, et al. Efficient graphlet kernels for large graph comparison. PMLR, 2009: 488-495.

\bibitem{hg_root}
Wachman G, Khardon R. Learning from interpretations: a rooted kernel for ordered hypergraphs. Proceedings of ICML. 2007: 943-950.

\bibitem{hg_line}
Bai L, Ren P, Hancock E R. A hypergraph kernel from isomorphism tests. 2014 22nd International Conference on Pattern Recognition. IEEE, 2014: 3880-3885.

\bibitem{hg_iso}
Luks E M. Hypergraph isomorphism and structural equivalence of boolean functions[C]. Proceedings of the thirty-first annual ACM symposium on Theory of computing. 1999: 652-658.

\bibitem{SVM_lib}
Chang C C, Lin C J. LIBSVM: a library for support vector machines[J]. ACM transactions on intelligent systems and technology (TIST), 2011, 2(3): 1-27.

\bibitem{SVM}
Platt J. Probabilistic outputs for support vector machines and comparisons to regularized likelihood methods[J]. Advances in large margin classifiers, 1999, 10(3): 61-74.

\bibitem{IMDB}
Yanardag P, Vishwanathan S V N. Deep graph kernels. Proceedings of the 21th ACM SIGKDD international conference on knowledge discovery and data mining. 2015: 1365-1374.

\bibitem{MUTAG}
N. Wale, I. A. Watson, and G. Karypis. Comparison of descriptor spaces for chemical compound retrieval and classification. Knowledge and Information Systems, 14(3):347–375, 2008.

\bibitem{NCI1}
A. K. Debnath, R. L. Lopez de Compadre, G. Debnath, A. J. Shusterman, and C. Hansch. Structure-activity relationship of mutagenic aromatic and heteroaromatic nitro compounds. correlation with molecular orbital energies and hydrophobicity. J Med Chem, 34:786–797, 1991.

\bibitem{PROTEINS}
K. M. Borgwardt, C. S. Ong, S. Schönauer, S. V. N. Vishwanathan, A. J. Smola, and H.-P. Kriegel. Protein function prediction via graph kernels. In Proceedings of Intelligent Systems in Molecular Biology (ISMB), Detroit, USA, 2005.

\bibitem{oldest}
Weisfeiler, B. J., and A. A. Leman. "A reduction of a graph to a canonical form and an algebra arising during this reduction, Nauchno–Technicheskaja Informatsia, 9 (1968), 12–16."

\bibitem{efficiency_graph_kernel}
Ramon, Jan, and Thomas Gärtner. "Expressivity versus efficiency of graph kernels." Proceedings of the first international workshop on mining graphs, trees and sequences. 2003.

\bibitem{group_theory}
Eugene M. Luks, Isomorphism of graphs of bounded valence can be tested in polynomial time, J. Comput. Syst. Sci. 25 (1982), no. 1, 42–65.

\bibitem{graphlet_base}
R. Kondor, N. Shervashidze, and K. M. Borgwardt. The graphlet spectrum. In International Conference on Machine Learning, pages 529–536, 2009.

\bibitem{graph_kernel_two}
H. Kashima, K. Tsuda, and A. Inokuchi. Marginalized kernels between labeled graphs. In International Conference on Machine Learning, pages 321–328, 2003.

\bibitem{motif_graphlet}
Pržulj, Natasa, Derek G. Corneil, and Igor Jurisica. "Modeling interactome: scale-free or geometric?." Bioinformatics 20.18 (2004): 3508-3515.

\bibitem{hg_low_rank}
Babai, László, and Paolo Codenotti. "Isomorhism of hypergraphs of low rank in moderately exponential time." 2008 49th Annual IEEE Symposium on Foundations of Computer Science. IEEE, 2008.

\bibitem{CHI}
Arvind, Vikraman, et al. "Colored hypergraph isomorphism is fixed parameter tractable." Algorithmica 71 (2015): 120-138.

\end{thebibliography}



