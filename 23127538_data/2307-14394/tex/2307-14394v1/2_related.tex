\section{Related Work}
In this section, we will introduce some related graph isomorphism methods and hypergraph isomorphism methods. 
% The graph isomorphism problem has been a prominent topic in graph theory for almost four decades. Prior to the 21st century, tackling the graph isomorphism problem from a traditional perspective was a lively field, resulting in the development of various methods for testing the isomorphism between two graphs, such as the WL-Test and Subgraph Patterns. 
% With the rapid advancement of Kernel methods in the field of machine learning, many graph kernel-based methods emerged based on traditional graph isomorphism algorithms, offering a fresh perspective for extracting structural features from graphs. 
\subsection{Graph Isomorphism Methods}
%Those methods enable the extraction of meaningful feature information from the structure of graphs.

The Weisfeiler-Leman Algorithm, also known as the WL-test, was initially proposed in 1968\cite{graph_wl}. Weisfeiler and Leman made a pioneering contribution by applying the Color Refinement algorithm to the field of graph isomorphism. The fundamental concept behind this algorithm involves labeling the vertices of a graph based on their iterated degree sequence. In 1992, Babai and Mathon\cite{graph_k_wl} extended the Color Refinement process, introducing the k-dimensional Weisfeiler-Leman algorithm (k-WL). In contrast to the two-dimensional version, Babai et al. \cite{graph_k_wl} employed color tuples of vertices instead of single vertex coloring.

The Group-Theoretic Graph Isomorphism Machinery, known as Subgraph Patterns, was proposed by Luks et al.\cite{group_theory} in 1982 to test the isomorphism of graphs with bounded degrees. Luks et al. \cite{group_theory} pioneered a broader problem-solving strategy for graph isomorphism by devising a recursive mechanism that leverages the structure of permutation groups to encode graph structures. This algorithm forms the cornerstone of the group-theoretic graph isomorphism machinery, providing a solid foundation for further developments in the field. One of the most representative methods in this domain is the Graphlet method proposed by Prˇzulj et al. \cite{motif_graphlet}. Graphlets are subgraph patterns, with each graphlet representing an instance of an isomorphism type. Kondor demonstrated that vectorizing the statistical frequency of all kernel occurrences can effectively embed the structural features of graphs into a feature space.

The concept of Graph Kernel was first introduced in \cite{graph_iso_np_hard, graph_kernel_two} in 2003. Jan Ramon and Thomas Gartner et al. \cite{efficiency_graph_kernel} pioneered the Subtree Graph Kernel in 2003, which marked the emergence of subtree-based isomorphism algorithms. Jan Ramon rigorously derived and computed to demonstrate that encoding subtree structures can significantly enhance expressive power while incurring a minimal computational cost. In the realm of Graph Kernel, there are three main directions of research. Firstly, the Shortest-path graph Kernel, which is the earliest proposed graph kernel isomorphism algorithm, randomly extracts subgraph structural information through random walks. During the same period, the Weisfeiler-Lehman Subtree Kernel achieved remarkable expressive power improvement by employing the WL-test with minimal computational cost. Thirdly, the Graphlet Kernel algorithm proposed a solution to the incompleteness problem of the WL-test. It statistically counts the number of kernels in the graph through random sampling, thereby encoding the structural information of the graph data. Notably, Shervashidze's groundbreaking research in 2009 \cite{graphlet} significantly improved this method by approximating the occurrence probabilities of Graphlets through random sampling, enhancing computational efficiency. Although this approach reduced memory consumption in practical applications, it compromised the reliability and stability of the algorithm due to the random sampling technique employed.

\subsection{Hypergraph Isomorphism Methods}
The development of hypergraph-based isomorphism testing has seen vigorous growth in recent years, coinciding with the rise of hypergraph structures. As early as 2007, Wachman et al. \cite{hg_root} experimented with isomorphism algorithms on hypergraphs. Due to the inherent unreliability of random walk algorithms, the kernel function suffered from high complexity and unstable learning structures. In 2008, Laszl'o et al.\cite{hg_low_rank} investigated the computational aspects of hypergraph isomorphism and proposed algorithms that can handle hypergraphs with low-rank structures efficiently within a reasonable computational timeframe. Arvind et al. \cite{CHI} presented a fixed parameter tractable algorithm for Colored Hypergraph Isomorphism. Arvind claimed their algorithm could be seen as a generalization of Luks’s result for Hypergraph Isomorphism. In 2014, Bai et al.\cite{hg_line} tackled this issue by transforming the hypergraph into a bipartite graph, enabling the application of the WL-test to hypergraph structures. However, this transformation significantly increased the size of the hypergraph, making the algorithm excessively complex. The low computational efficiency prevented the direct application of the algorithm to hypergraph data. As stated by the authors in its Conclusion \cite{hg_line}, "Our future work is to develop a new higher-order WL isomorphism test algorithm that can be directly performed on a hypergraph." The field of hypergraph isomorphism requires a WL isomorphism algorithm that can directly operate on hypergraphs, which are the methods proposed in our paper.
