
\IEEEraisesectionheading{\section{Introduction}\label{sec:introduction}}

% 图数据广泛存在应用广泛。更多的数据是高阶关联的。图无法建模高阶关联数据,因此超图慢慢火起来。超图有很多应用

% 这些应用依赖对结构有一定的辨别能力。同构测试是这些应用的基础。图有一些结构辨别的方法。然而超图这方面很少

% 现有超图辨别方法怎么做,有什么缺点。

% 这篇文章我们针对超图同构问题,提出了一套算法。这套算法怎么做。

% 同时基于这套算法,我们进一步提出了两种核方法。怎么做的。在很多数据集上做了大量的实验。图数据集上的结果。超图数据集上的结果。总结一下我们的贡献。

% 贡献1:提出了超图WL算法,直接能用于超图上的结构辨别
% 贡献2:基于算法,提出了两个核函数,并提供了理论的证明,在图上可以退化成图WL,在超图上更好
% 贡献3:做了大量的数据集和实验,证明了我们方法快速且有效 

%围绕现实世界中的关联建模的重要性

%先讲同构这个问题的重要性
%传统方法:把图方法归类到传统方法中
% 虽然

\IEEEPARstart{N}{etwork}, as a typical irregular modeling tool, has shown its advances in many applications like social networks\cite{graph_social_network}, brain networks\cite{graph_brain}, collaborative networks\cite{graph_collabrative_network}, knowledge networks\cite{graph_knowledge_network}, chemical pathways\cite{graph_pathways}, and protein structures\cite{graph_protein}. However, real-world irregular data often comprise numerous high-order correlations that a simple graph structure cannot adequately represent. When attempting to model high-order correlations, such as the group relations in social media \cite{hg_social_media, hgnn, hgnnp} or the co-author relations in academic papers \cite{dhgnn, hg_coauthor, hg_encoder}, the intricate correlations among vertices become ambiguous, as illustrated in Figure \ref{fig:intro}. The hyperedge in hypergraphs can link more than two vertices, which endows the hypergraph with the capability to model beyond pair-wise correlation compared with the simple graph. To analyze and understand the hypergraph structure, one needs a tool to measure the similarity between different hypergraphs or sub-structures of a hypergraph, also known as the ``Isomorphism Test''.

%为了解决问题,超图被用于刻画。。。
%然而超图作为新鲜事物,存在什么什么问题 (同构性判定) 
%超图的同构判定有什么结果,介绍前期工作
%受到图同构判断的启发。。。

% 分两类,超图转图 然后再说图铜鼓   直接超图同构,再bulabula
% 将 图同构作为超图同构的一条路径
% 有很多图同构的方法,但是都无法应用用于超图。

%While graph isomorphism algorithms excel at extracting low-order structural information, they are limited in their ability to handle relationships beyond pair-to-pair connections. There exist many graph similarity measures based on graph isomorphisms like the Weisfeiler-Lehman test \cite{graph_wl} of graph isomorphism and graph kernels\cite{graph_wl_subtree, graph_wl_all, graphlet}.

Graph isomorphism algorithms, such as the Weisfeiler-Lehman test \cite{graph_wl} for graph isomorphism and graph kernels \cite{graph_wl_subtree, graph_wl_all, graphlet}, excel at extracting low-order structural information. However, they face limitations when dealing with relationships that extend beyond pair-to-pair connections. In the typical Weisfeiler-Lehman test, each vertex gathers its neighbor vertices' labels to generate the compressed vertex labels. Each compressed vertex label corresponds to a unique subtree structure. Nevertheless, the algorithm can only output whether the two graphs are identity. Furthermore, based on the algorithm, the graph kernel methods are proposed to measure the similarity of two graphs with a continuous value. The graph kernel methods design a mapping from the graph structure to the vector in the feature space and utilize the feature vector's inner product to measure the graphs' similarity. However, these graph-based similarity measurement methods, which rely on vertex-to-vertex label propagation, are not suitable for handling the complexity of such high-dimensional structures, as mentioned earlier in the case of hypergraphs.

% Figure environment removed

With the emergence of hypergraph structures, researchers have also proposed various strategies \cite{hg_iso, hg_line, hg_root} to address the problem of high-order structure similarity measurement. However, most of these methods fundamentally rely on transforming hypergraphs into graph structures. These indirect hypergraph isomorphism algorithms, to some extent, sacrifice the complex structural information inherent in hypergraphs, leading to misalignments of criteria and cumbersome computational processes.
%Recently, researchers also proposed some strategies\cite{hg_iso, hg_line, hg_root} for the hypergraph similarity measure problem. Wachman et al.
\cite{hg_root} develop a sampling-based method that draws the vertex-hyperedge sequence from the original hypergraph. However, it will bring information loss and fail in many common cases (the pyramid-like hypergraphs in Figure \ref{fig:pyramid}).
By employing an indirect approach, Bai et al. \cite{hg_line} design a transformation-based method, which transforms the hypergraph structure into a low-order graph structure with three steps. In this way, the scale of the generated graph will be extremely large and can not be employed in practice (causing the out-of-memory error in the dataset containing more than 3000 hypergraphs in Table \ref{tab:hypergraph_real}). 


To address those problems, we first extend the Weisfier-Lehman test algorithm \cite{graph_wl} from graphs to hypergraphs. The definition of the neighbor relation is the obstacle to applying the Weisfiler-Lehman test in hypergraphs. Thus, as for the complex high-order correlations, we decompose the neighbor relation into two sub-neighbor relations: vertex's hyperedge neighbor and hyperedge's vertex neighbor. Then, based on the two neighbor relations, we devise a two-stage hypergraph Weisfiler-Lehman test algorithm, which can be directly applied to hypergraphs. Second, based on the hypergraph Weisfeiler-Lehman test algorithm, we proposed a general hypergraph Weisfeiler-Lehman kernel, which maps the hypergraph structure into a vector in the feature space. With the hypergraph Weisfeiler-Lehman kernel, the distance between two hypergraphs can be computed by the inner product of the two corresponding feature vectors. Third, we implement two instances of the general hypergraph Weisfeiler-Lehman kernel: hypergraph Weisfeiler-Lehman subtree kernel and hypergraph Weisfeiler-Lehman hyperedge kernel. The hypergraph Weisfeiler-Lehman subtree kernel directly counts different types of subtree structures, and the hypergraph Weisfeiler-Lehman hyperedge kernel counts different  hyperedges. Besides, to deeply exploit the relation between the graph Weisfeiler-Lehman subtree kernel and the proposed hypergraph Weisfeiler-Lehman subtree kernel, we have proven that the hypergraph Weisfeiler-Lehman subtree kernel can be reduced to the graph Weisfeiler-Lehman subtree kernel when processing the graph structures from a mathematical perspective. To verify the effectiveness of the proposed hypergraph Weisfeiler-Lehman kernels, we conduct experiments on $19$ graph/hypergraph classification datasets, including two synthetic graph datasets, five real-world graph datasets, four synthetic hypergraph datasets, and eight real-world hypergraph datasets. Experimental results on graph datasets demonstrate that the proposed hypergraph Weisfeiler-Lehman subtree kernel can achieve the same performance as the graph Weisfeiler-Lehman subtree kernel (proven by Theorem \ref{theorem:equal}). Experimental results on hypergraph datasets show a superior performance increase compared with other kernel-based methods. We also conduct experiments to compare the runtime with other hypergraph methods, which is important in practice. Results demonstrate that our methods run faster and can achieve better performance. The main contributions of this paper are summarized as follows:

\begin{itemize}
    \item We extend the Weisfeiler-Lehman test from graphs to hypergraphs and propose the hypergraph Weisfeiler-Lehman test algorithm for the hypergraph isomorphism test problem. % 提出框架,他是对现有图方法的一个拓展
    \item We propose a general hypergraph Weisfeiler-Lehman kernel framework for hypergraph or sub-structure hypergraph similarity measure and implement two instances: hypergraph Weisfeiler-Lehman subtree kernel and hypergraph Weisfeiler-Lehman hyperedge kernel.
    \item We mathematically prove that the proposed hypergraph Weisfeiler-Lehman subtree kernel can degenerate into graph Weisfeiler-Lehman subtree kernel and achieve the same performance confronting graph structures. 
    \item Extensive experiments on $19$ graph/hypergraph classification datasets demonstrate the effectiveness of the proposed methods.
\end{itemize}


