\section{Approximation of nonlinear state-space models}
Consider the nonlinear state-space model with no interactions between states and inputs:
\begin{equation}
    x_k = T(x_{k-1},u_k).
\end{equation}
Following~\cite{korda2018linear}, we let $\ell(\R^M)=\R^{M\times\infty}$ be the space $M$-dimensional sequences. Let $\mathcal S$ be the left-shift operator on $\ell(\R^M)$, acting on a sequence $\bar u = (u_0,u_1,\cdots)$, such that $\mathcal{S}[\bar u] = (u_1, u_2,\cdots)$. Let $\chi$ be the augmented state $\chi := (x, \bar u)\in\R^N\times\ell(\R^M)$; its transition can be written as
\begin{equation}
    \chi^+ = \bar T (\chi) = (T(x,\bar u(0))+\bar u(0), \mathcal{S}\bar u).
\end{equation}
\paragraph{Space of observables.} Let us consider a space of observables $\mathcal{F}:\R^N\times \ell(\R^M) \to\C$ that provides measurements of the pair $(x, \bar u)$. In particular, we look at the space spanned by the observables $(\phi_i)_i$. That is, for each $g\in\mathcal{F}$ there exist real coefficients $(w_i)_i$ such that
\begin{equation}
    g(\cdot) = \sum_i w_i\phi_i(\cdot).
\end{equation}
We are going to select each $\phi_i$ to have the form
\begin{equation}
    \phi_i(x,\bar u) = \psi_i^x(x) + \psi_i^u(\bar u),
\end{equation}
where each $\psi_i^x,\psi_i^u:\R^N\to \C$ are nonlinear functions. 

\paragraph{Koopman operator.} Let us define $\mathcal{K}$ to be the operator that advances measurements one step forward in time.
\begin{equation}
    \mathcal{K}[g] = g\circ\bar T
\end{equation}
The nontrivial requirement for the definition of this operator is that if $g\in\mathcal{F}$ then $\mathcal{K}g\in\mathcal{F}$. That is, $T$ and $\mathcal{F}$ are such that $\mathcal{K}:\mathcal{F}\to\mathcal{F}$. If this requirement is satisfied, then there exists a real matrix $K$ with $[K]_{ij}$ such that for every $g = \sum_i w_i\phi_i$
\begin{equation}
\mathcal{K}g = \sum_i w_i \mathcal{K} \phi_i= \sum_i w_i \left(\sum_j k_{ij} \phi_j\right) = \sum_i w_i K\phi,
\end{equation}
where $\phi = (\phi_1,\phi_2,\dots)$. 

\paragraph{Justification for invariance.} The key assumption in the computation above is the relation
\begin{equation}
    \mathcal{K}\phi_i = \sum_j k_{ij} \phi_j,
\end{equation}
which is precisely the assumption that $\mathcal{K}\phi_i\in\mathcal{F}$ for all $i$. In our setting, if the collection $\{\psi^u_i\}$ is powerful enough to model nonlinear functions of finite input sequences, e.g. it 
