\section{Application Scenarios}
In this section, we present various applications of \system{} that showcase its potential for enhancing remote collaboration across diverse contexts.

\subsection{Physical Storytelling}\label{subsec:Storytelling}
\system{} enables collaborative storytelling, allowing both local and remote users to participate in creating a story together with tangible objects.
By attaching Toios to physical objects such as puppets, stuffed animals, toy figures, and LEGO blocks, both local and remote users can move the objects, crafting the story and narrative, as we do in physical space.
The local user can either observe as an audience member or actively engage with the story-creation process.
Figure \ref{fig:storytelling} illustrates a remote user physically moving a dinosaur toy on a stage to narrate a story to the local user.
This provides, for example, engaging tangible storytelling for children and their remote parents or friends. 

% % Figure environment removed

% Figure environment removed

\subsection{Tangible Remote Gaming}
\system{} facilitates remote gaming experiences.
By attaching Toios to game objects, the local user can physically interact with the remote user through the tangible game.
Figure~\ref{fig:hockey} depicts a table hockey game application that utilizes three Toios---two for the mallets and one for the puck, similar to~\cite{nakagaki2022dis, kaimoto2022sketched}. 
This application allows users to play and compete with each other in real-time, creating an engaging and immersive gaming experience.

% % Figure environment removed

% Figure environment removed

\subsection{Interior and Architectural Design}
Taking inspiration from the immersive interior and architectural design (e.g., \textit{DollhouseVR}~\cite{ibayashi2015dollhouse}), our system also facilitates the collaborative world-in-miniature exploration, in which the robot can embody the physical representation of the miniature user. 
For example, Figure~\ref{fig:interior-design} illustrates an application for collaborative interior design.
This application uses miniature furniture to facilitate discussion and decision-making between remote and local users.
The remote user is visually represented as a miniature avatar, with a Toio representing the remote user's physical body. 
The remote user can visually instruct the local user using gestures and physically push the miniature furniture to arrange the position.

% % Figure environment removed

% Figure environment removed

\subsection{Guided Instruction}
\system{} can be applied to the collaborative drawing.
By attaching a physical pen to a Toio, the remote user can move the pen and draw on a physical canvas.
Local and remote users can therefore collaborate in real-time to create drawings and illustrations together.
Figure~\ref{fig:guide} shows a remote user controlling the movement of a red pen to draw on the physical canvas. 
% \todo{We need this figure. Retake exactly the same shot with Hololens 2}




\subsection{Collaborative Modeling}
\system{} can serve as a tangible user interface for remote collaboration, allowing users to manipulate virtual objects through physical interaction with Toios.
For instance, Toios can represent tangible user interface elements such as buttons, sliders, and knobs. 
In Figure~\ref{fig:collaborative-design}, three Toios are used as a tangible UI for manipulating a 3D object. Two  of the Toios represent sliders and adjust the width and depth of the object, while the third Toio represents a knob and alters the object's height.


%  % Figure environment removed

 % Figure environment removed

\subsection{Haptic Notification}
\system{} can provide haptic notifications, enabling remote users to physically notify local users by using Toios.
By attaching Toios to the remote avatar's hand, the remote user can touch the local user and initiate communication.
In Figure ~\ref{fig:notification}, the remote user touches the local user who is reading a book to start a conversation.
% \todo{Do we have a figure for this? Something similar to SwarmHaptics or guided instruction.}

% % Figure environment removed

% Figure environment removed