\section{Introduction}
Today's mixed reality telepresence still falls short of replicating the rich tangible experiences that we naturally enjoy in our everyday lives.
In real-world collaboration, for example, we casually grasp and manipulate physical objects to facilitate discussions, employ touch for social interactions, spatially arrange physical notes for brainstorming, and guide others by holding their hands.
However, such tangible interactions are not possible with current mixed reality telepresence systems, as the virtual remote user has no way to \textit{physically} engage with the local user and environment. 

In this paper, we introduce \system{}, a mixed reality interface to achieve \textit{tangible remote collaboration} by synchronously coupling holographic telepresence with an actuated physical environment. Beyond existing holographic telepresence like \textit{Holoportation}~\cite{orts2016holoportation}, \system{} lets remote users not only visually and spatially co-present but also physically touch, grasp, manipulate, and interact with remote tangible objects, as if they were co-located in the same shared space. 
We achieve this by synchronizing the remote user's motion rendered in a mixed reality headset (Hololens 2 and Azure Kinect) with physical actuation enabled by multiple tabletop mobile robots (Sony Toio). 

% change here
% \add{
Our idea builds up on the existing \textit{physical telepresence}~\cite{leithinger2014physical} and other related approaches~\cite{lee2018physical, siu2018investigating}, but we make two key contributions beyond them.
First, we explore a \textbf{\textit{broader design space}} of tangible remote collaboration with holographic telepresence, which are not fully investigated in the prior work~\cite{siu2018investigating, lee2018physical, he2017physhare}.
For example, we showcase various interactions, such as object actuation, virtual hand physicalization, world-in-miniature exploration, shared tangible interfaces, embodied guidance, and haptic communication. 
We also demonstrate use cases and application scenarios for each interaction, such as physical storytelling, remote tangible gaming, and hands-on instruction.

Second, we contribute to a \textbf{\textit{holistic user evaluation}} to better understand how mobile robots can enhance holographic telepresence in different application scenarios.
To this end, we compare our approach (hologram + robot) with hologram-only and robots-only conditions through a within-subject user study with twelve participants. 
Both quantitative and qualitative results suggest that our system significantly enhances the level of co-presence and shared experience for mixed reality remote collaboration, compared to the other two conditions. Based on the insights, we also discuss the future of tangible remote collaboration that leverages robotic environments.
% }

\remove{Our idea builds up on the existing \textit{physical telepresence}~\cite{leithinger2014physical} and other related approaches~\cite{lee2018physical, siu2018investigating}, but we make two key contributions beyond them. First, by leveraging \textbf{\textit{multiple tabletop robots}}, rather than shape-changing displays~\cite{leithinger2014physical} or single-point actuation of X-Y plotters~\cite{lee2018physical}, our approach enables scalable, deployable, yet generalizable tangible remote collaboration.
Second, we contribute to the \textbf{\textit{demonstration and evaluation of a broader design space}} that is not fully investigated in the prior work~\cite{siu2018investigating, lee2018physical, he2017physhare}. For example, we demonstrate and evaluate various interactions, such as object actuation, virtual hand physicalization, world-in-miniature exploration, shared tangible interfaces, embodied guidance, and haptic communication. 
We also showcase various use cases and application scenarios, such as physical storytelling, remote tangible gaming, and hands-on instruction.
Collectively, our paper contributes to establishing a new approach to holographic tangible telepresence beyond the existing works.}

\remove{To evaluate our approach, we conduct a user study with twelve participants, where we compare our approach (hologram + robot) with hologram-only and robots-only conditions. Both quantitative and qualitative results suggest that our system significantly enhances the level of co-presence and shared experience for mixed reality remote collaboration, compared to the other two conditions. Based on the insights, we also discuss the future of tangible remote collaboration that leverages robotic environments.
}

Finally, this paper contributes the following:

\begin{enumerate}
\remove{\item \system{}, a system to augment holographic telepresence with multiple tabletop mobile robots that enables scalable, deployable, and generalizable tangible remote collaboration.}
\item A design space exploration and application demonstrations that showcase a set of possible interactions and use cases enabled by our system.
\item Results and insights from our user study that confirm the benefits of our approach over hologram-only and robots-only conditions. 
\end{enumerate}
\newpage