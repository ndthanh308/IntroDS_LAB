\section{\system{} Design Space}
In this section, we explore the design space of \system{} in the following four dimensions: 1) interaction techniques, 2) actuation types, 3) surface types, and 4) physical attachment (Figure~\ref{fig:design-space}). 
% \add{However, our exploration is limited to the aspects that are implementable and evaluable.
% Other potential areas, which remain unexplored, are presented in the future work section.
% }

% % The design space provides a better understanding of the potential of \system{} and how it can be used to enhance collaboration in various contexts.


\subsection{Interaction Techniques}
\subsubsection*{\textbf{Object Actuation}}\label{subsec:Storytelling}
\system{} offers various ways for remote users to interact with the local user. The object actuation enables remote users to move and manipulate objects in the local environment. For example, remote users can directly grab the Toio robot to move its location, or the attached object for more expressive engagement. 
% Figure environment removed
\remove{
Figure~\ref{fig:object-actuation} illustrates the holographic remote user moving an animal toy.
We demonstrate object actuation for different use cases, such as storytelling, gaming, or even drawing.
% Figure environment removed
}

% \add{
The object actuation can be used for different use cases, such as storytelling, gaming, and drawing.
For storytelling, \system{} allows both local and remote users to participate in creating a story together with tangible objects.
The local user can either observe as an audience member or actively engage with the story-creation process.
Figure \ref{fig:storytelling} illustrates a remote user physically moving a dinosaur toy on a stage to narrate a story to the local user.
This provides, for example, engaging tangible storytelling for children and their remote parents or friends.
% }

\subsubsection*{\textbf{Shared Tangible UI}}
Another interaction technique is the shared tangible user interface, which allows both local and remote users to manipulate virtual object properties through tangible tokens. 
Toio robots can be represented as various tangible UI elements, such as control points, buttons, sliders, and knobs, so that by controlling the same UI, local and remote users can manipulate the UI together.
For example, Figure~\ref{fig:tangible-ui} illustrates users changing the position and scale of the virtual picture by manipulating robots, which are represented as a control point of the image.
% % Figure environment removed

% Figure environment removed

In Figure~\ref{fig:collaborative-design}, three Toios are used as a tangible UI for manipulating a 3D object. Two  of the Toios represent sliders and adjust the width and depth of the object, while the third Toio represents a knob and alters the object's height.

 % Figure environment removed

\subsubsection*{\textbf{Miniature Body Interaction}}
The robot can also embody the remote user through a miniature body. 
Our system also facilitates the collaborative world-in-miniature exploration, by representing as the miniature user. 
Similar to the prior work that explores tangible world-in-miniature exploration (e.g., \textit{miniStudio}~\cite{kim2016ministudio}, \textit{Does it Feel Real?}~\cite{muender2019does}, \textit{Shoulder of Giants}~\cite{piumsomboon2019shoulder}, \textit{ASTEROIDS}~\cite{li2022asteroids}), the tangible embodiment of the miniature user facilitates rich physical affordances for the world-in-miniature interaction, while providing effective visual feedback through holographic representation.
The remote user can walk around on a real-size environment, which is captured and tracked through Azure Kinect body tracking. 
For example, the remote user can visually instruct the local user using gestures and physically move objects in the local environment by pushing them with Toios (Figure~\ref{fig:miniature-body}). 
\remove{This technique can be used for several applications, such as architecture or interior design.}
% % Figure environment removed

% Figure environment removed

% \subsection{Interior and Architectural Design}
Taking inspiration from the immersive interior and architectural design (e.g., \textit{DollhouseVR}~\cite{ibayashi2015dollhouse}), this could be used for the collaborative world-in-miniature exploration, in which the robot can embody the physical representation of the miniature user. 
For example, Figure~\ref{fig:interior-design} illustrates an application for collaborative interior design.
This application uses miniature furniture to facilitate discussion and decision-making between remote and local users.
The remote user is visually represented as a miniature avatar, with a Toio representing the remote user's physical body. 
The remote user can visually instruct the local user using gestures and physically push the miniature furniture to arrange the position.
% % Figure environment removed

% Figure environment removed

\subsubsection*{\textbf{Haptic Communication}}
Haptic communication is another interaction technique that enables the remote user to provide haptic feedback to the local user. 
There are various ways to provide haptic communication. 
For example, the user can guide the Toio robot to navigate the remote user based on the actuation, similar to \textit{dePend}~\cite{yamaoka2013depend}, as if they were holding their hands. 
This technique can be used for hands-on instruction. 
Alternatively, the remote user can physically touch the local user by moving and touching the local user's body using Toios, similar to \textit{SwarmHaptics}~\cite{kim2019swarmhaptics}.
This can be used for remote social interaction.

Figure~\ref{fig:guide} shows a remote user controlling the movement of a red pen to draw on the physical canvas.
By attaching a physical pen to a Toio, the remote user can move the pen and draw on a physical canvas.
Local and remote users can therefore collaborate in real-time to create drawings and illustrations together.

% % Figure environment removed

% Figure environment removed

This can also provide haptic notifications, enabling remote users to physically notify local users using Toios.
By attaching Toios to the remote avatar's hand, the remote user can touch the local user and initiate communication.
In Figure ~\ref{fig:notification}, the remote user touches the local user who is reading a book to start a conversation.

% Figure environment removed

\subsection{Actuation Types}
\subsubsection*{\textbf{Move Active Object}}
In \system{}, the remote users can actuate physical objects in two ways. 
First, the user can simply grasp and move the Toio robot itself. By moving the Toio, which is attached to the various object, \system{} enables the remote user to actuate physical objects (Figure~\ref{fig:storytelling}). 

One possible application is the remote gaming experience.
By attaching Toios to game objects, the local user can physically interact with the remote user through the tangible game.
Figure~\ref{fig:hockey} depicts a table hockey game application that utilizes three Toios---two for the mallets and one for the puck, similar to~\cite{nakagaki2022dis, kaimoto2022sketched}. 
This application allows users to play and compete with each other in real-time, creating an engaging and immersive gaming experience.

% % Figure environment removed

% Figure environment removed

\subsubsection*{\textbf{Move Passive Object}}
Alternatively, the user can also actuate everyday passive objects by pushing these objects with the Toio.
This allows actuating objects without attaching robots in advance.
Similar to~\cite{kennel2023interacting}, by making the robots follow the user's fingers, the remote user can physicalize their hands and fingers, so that pushing the other passive objects (Figure~\ref{fig:passive-object}).
This method allows an intuitive way of interacting with physical objects, as the remote user can use hand gestures to control objects.
In the current setup, each Toio can push an object up to 32 grams. 

% % Figure environment removed

% Figure environment removed


\subsection{Surface Types}
\subsubsection*{\textbf{Horizontal Surface}}
\system{} also supports two different surface types that the robot can move around.
The first one is the horizontal surface, such as a tabletop surface where the users sit down together to manipulate objects on the table. 

\subsubsection*{\textbf{Vertical Surface}}
Alternatively, by attaching a small magnet at the back of the Toio, Toio can move on a vertical surface such as a whiteboard or a magnetic wall.
By moving Toios on a vertical surface can be useful for applications that require standing up, such as brainstorming or presentations.
In our prototype, we attach an N35 neodymium magnet (8 mm $\times$ 3 mm, 1 mm thickness) to the bottom of the Toio robot with tape, which has a strong attraction force to be attached to the whiteboard, while weak enough to move on a wall. For the tracking of the vertical surface, we use a thinner tracking mat (Toio Developer Mat, 0.1 mm thickness) that can be attached to the whiteboard.
With the vertical surface, we can also expand the application domains, such as collaborative discussion and brainstorming with the post-it notes on a whiteboard (Figure~\ref{fig:vertical-surface}).

% % Figure environment removed

% Figure environment removed

\subsection{Attachments of the Robot}
\system{} is also designed to be versatile and adaptable to various applications by allowing the user to attach different components to the robot. These attachments provide additional functionalities and enable the robot to perform a wider range of tasks, making it suitable for a variety of applications. 

\subsubsection*{\textbf{Shape Props}}
Shape props can modify the robot's shape and physical appearance.
As illustrated in Figure~\ref{fig:storytelling}, attaching a dinosaur toy to the robot can be used to represent a dinosaur, expanding its interactive potential.
By attaching Toios to physical objects such as puppets, stuffed animals, toy figures, and LEGO blocks, both local and remote users can move the objects, crafting the story and narrative, as we do in physical space.
% By attaching a real puck to the robot, it could enhance the remote table hockey experience.

\subsubsection*{\textbf{Material Props}}
The addition of material props such as soft materials, fur, and fabric enables the local user to enhance the sensation of remote objects and users.
For example, by attaching soft materials, mobile robots can represent remote users' hands to improve haptic communication.
Also, the use of fabric materials enables the mobile robots to represent portions of the remote user's arm that are clothed.

\subsubsection*{\textbf{Functional Props}}
Attachments can supplement the robot with added functionalities.
For example, Figure~\ref{fig:drawing-attachment} illustrates remote users drawing on a transparent sheet using a robot equipped with a pen, which facilitates visual communication between users.
As shown in Figure~\ref{fig:vertical-surface}, attaching post-it notes to the mobile robots enables the remote user to highlight specific parts in the local user's environment. 
Also, by attaching magnets to the robots, users can extend their mobility from horizontal to vertical surfaces.

% Figure environment removed

% \subsubsection*{\textbf{Mobility Props}}
% Attachments can enhance the robot's mobility.
% As shown in Figure~\ref{fig:vertical-surface}, by attaching magnets to the robots, users can extend their mobility from horizontal to vertical surfaces, thus broadening the range of applications.

\subsubsection*{\textbf{Constraints}}
Mechanical constraints, such as rings and rubber bands, can be employed to restrict the movements of mobile robots as PICO~\cite{patten2007mechanical}.
This provides both the remote and local users to move the robots within a specific range of movement.
For example, by using a straight ring, the movement of mobile robots can be limited to a straight line, which could help the creation of a precise slider UI.
Also, confining all the mobile robots within a ring can help limit the area in which the remote user can influence the local environment. 
% \subsubsection*{\textbf{Deformable Objects}} 
% Using deformable objects, we can extend the range of achievable shapes.
% For example, attaching origamis to mobile robots such as the Gesundroid~\footnote{\url{https://youtu.be/kLll0hWn5sA}}, allows creating of diverse shapes beyond the capability of the mobile robots alone.


% - Mechanism Modules (fan, reel like HERMITS [45])

% - Shape-changing Modules (actuated surface like HapticBots [70])


% Also, by binding the finger to the robot, the remote user can also move the local user through direct physical guidance (Figure~\ref{fig:guide}).

% % Figure environment removed





