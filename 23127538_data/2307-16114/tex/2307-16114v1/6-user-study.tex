\newcommand{\UIApp}{Shared Tangible UI}
\newcommand{\ObjectApp}{Object Actuation}
\newcommand{\BodyApp}{Miniature Body Interaction}
\newcommand{\HandApp}{Haptic Communication}

\section{User Study}
To evaluate the effectiveness of incorporating both virtual and physical representations in holographic remote collaboration, we conducted a user study comparing our system with Hologram-Only and Robots-Only conditions across four distinct interactions. 
We gathered both quantitative and qualitative measurements for various aspects, such as social presence, system usability, and cognitive workload, with a within-subject user study.

\subsection{Method}
\subsubsection{Participants}
We recruited 12 participants (11 male, 1 female) from our local university, with an age range of 21-24 years (M = 22.1, SD = 1.16).
Participants were surveyed on their familiarity with VR/AR using a 7-point Likert scale from 1 (novice) to 7 (expert), and the average score was 2.75 (SD = 1.71).

\subsubsection{Study Setup}
We present the setup used in our study in Figure~\ref{fig:study-setup}.
One of the authors acts as a remote collaborator (referred to as "the experimenter") for each participant to reduce differences in interaction between groups.
The participant and the experimenter are situated in separate rooms and communicate remotely.
The dimensions of the participant's room were approximately 11.3 m by 5.2 m, while the experimenter's room measured approximately 7.5 m by 5.9 m.
Both the participant and experimenter were equipped with Hololens 2 headsets.
On the participant's desk, we placed the Toios and a Toio mat.
To enable the experimenter to view the participant's workspace, we used an iPad to capture the video image and transmitted it to the experimenter's display.

For audio communication, we used Discord\footnote{https://discord.com/}, a voice chat application.
To mitigate any potential interference from the Toio's sound, the participant wore noise-canceling headphones (Sony WH-1000XM4).

% % Figure environment removed

% Figure environment removed

% Figure environment removed

% % Figure environment removed

% Figure environment removed

% % Figure environment removed

\subsubsection{Study Design}
We designed our study with a within-subject design that compares the following three conditions:
\begin{description}
\item[C1. Hologram +  Robots]: Participants interacted with the remote experimenter via hologram and voice chat with using mobile robots.
\item[C2. Hologram-Only]: Participants interacted with the remote experimenter via hologram and voice chat without using mobile robots.
\item[C3. Robots-Only]: Participants interacted with the remote experimenter via voice chat without a hologram with using mobile robots.
\end{description}
To evaluate the difference in these conditions across various interactions, we used four application scenarios that best represent each interaction technique in our design space:
\begin{description}
\item[D1. Object Actuation]: We used the physical storytelling application (Figure~\ref{fig:storytelling}). Participants were instructed to create a short story with the remote experimenter by manipulating virtual or physical dinosaur toys. The remote experimenter could also move the dinosaur toys.
The fundamental elements of the stories shared similarities, including dinosaurs fighting, making up, walking around, and talking to each other. However, participants chose how the dinosaurs fight, where they make up or rest, and which directions they walk. We displayed virtual toys for C2 and used physical toys for C1 and C3.

\item[D2. Shared Tangible UI]: We used virtual image manipulation (Figure~\ref{fig:tangible-ui}). Participants were instructed to adjust the size of a virtual picture until it matched the target size printed on paper, collaborating with the remote experimenter. The virtual cubes or mobile robots were attached to the upper left and bottom right of the virtual picture, and participants and a remote experimenter could adjust the size by moving them.
We used virtual cubes for C2, and used physical cubes for C1 and C3. The virtual picture was displayed in all conditions.
\item[D3. Miniature Body Interaction]: We used the interior and architectural design application (Figure~\ref{fig:interior-design}). The remote experimenter was presented as a miniature body, similar in size to miniature furniture. Participants were instructed to move the furniture and determine furniture placement in discussion with the remote experimenter. The remote experimenter could also physically move the furniture in the conditions with a mobile robot.
\item[D4. Haptic Communication]: We used the haptic notification application (Figure~\ref{fig:notification}). Participants were instructed to read a book and engage in conversation with the remote experimenter when they were contacted. The remote experimenter initiated contact through virtual or physical touch, with mobile robots following the remote  experimenter's fingers to physically touch the participants.
\end{description}

\subsubsection{Measurements}
We measured four different aspects: 1) \textbf{Social Presence} based on the Networked Mind Measure of Social Presence questionnaire \cite{socialpresence}, 2) \textbf{Cognitive Workload} based on NASA Taskload Index (NASA-TLX)~\cite{nasa-tlx}, 3) \textbf{System Usability} based on System Usability Scale (SUS)~\cite{sus}, and 4) \textbf{Preference} based on the questionnaire in which the participants were asked which condition they preferred the most.
In addition to these measurements, we conducted an interview after the study to gather qualitative feedback from participants.

\subsubsection{Procedure}
After participants signed a consent form, we provided them with instructions on how to use the Hololens 2 and Toio robot.
Participants then conducted a task involving 12 sessions (4 applications $\times$ 3 conditions), each lasting 3 minutes.
Participants used four applications in the following order: \UIApp{}, \ObjectApp{}, \BodyApp{}, and \HandApp{}.
Participants conducted each application in all three conditions (C1, C2, and C3). The order of the three conditions was counterbalanced across participants to control for order effects.
After each application, participants answered the social presence questionnaire. 
After each condition, participants answered the SUS and NASA-TLX questionnaire to compare the three conditions.
In total, we asked the participants to complete 12 social presence questionnaires and 3 SUS and NASA TLX questionnaires. 
After the participants finished all of the sessions, we conducted a brief open-ended interview for 10-15 minutes.
The study took approximately 90 minutes in total, and each participant was compensated with 10 USD. 

\subsection{Results}
To analyze the data collected in our study, we employed a Friedman's test for each measurement.
To assess pairwise differences between conditions, we conducted multiple pair-wise comparisons using the Wilcoxon signed-rank test with Bonferroni correction.
We set the significant level at 5 \%.

\subsubsection{Social Presence}
The Social Presence Questionnaire consisted of three sub-scales: Co-Presence (CoP), Attentional Allocation (AA), and Perceived Message Understanding (PU).
Figure~\ref{fig:study-social-presence} shows the result of the social presence questionnaire for a total of 12 sessions (4 applications $\times$ 3 conditions for each).
In addition, we calculated an overall score by averaging the three sub-scales.
We checked the internal consistency with Cronbach’s alpha for each sub-scale:\ $ \alpha_{CoP} = 0.90$, $\alpha_{AA} = 0.78$, $\alpha_{PMU} = 0.93$.

For \ObjectApp{} (D1) and \UIApp{} (D2), Hologram + Robots (C1) condition had significantly higher overall social presence scores than Robots-Only (C3) condition.
For both \ObjectApp{} (D1) and \UIApp{} (D2), pairwise comparisons revealed that Hologram + Robots (C1) condition was significantly higher scores than Robots-Only (C3) condition for CoP (D1: $Z = 3.68$, $p = 0.0007 < 0.001$, D2: $Z = 3.29$, $p = 0.003 < 0.01$), PMU (D1: $Z = 2.46$, $p = 0.042 < 0.05$, D2: $Z = 2.61$, $p = 0.027 < 0.05$), and Overall (D1: $Z = 2.63$, $p = 0.025 < 0.05$, D2: $Z = 2.86$, $p = 0.013 < 0.05$).

In the interviews, participants made comments that suggested that Hologram + Robots (C1) condition resulted in a stronger sense of presence compared to Hologram-Only (C2) condition.
Specifically, one participant noted that \textit{``Hologram + Robots clearly felt the presence of the other party, whereas Hologram alone was less present.'' (P1)}, while another participant mentioned that \textit{``Hologram-only conditions were difficult to react to when the other person was out of sight'' (P2)}.
These comments suggest that combining mobile robots with holographic telepresence could help users better understand the remote user's actions and movements, even when the holographic user is out of sight.
Furthermore, for all four applications, the graph of the data suggested that Hologram + Robots (C1) had the highest scores, followed by Hologram-Only (C2) and Robots-Only (C3).

\subsubsection{Cognitive Workload}
The results for the cognitive workload are shown in Figure~\ref{fig:study-sus-nasa} (A).
% We found a significant difference between C1 and C2 ( Z=XXX, p = XX < 0.05), as well as between C1 and C3 ( Z=XXX, p = XX < 0.05). 
A lower score indicates a lower workload.
The average score for each condition was 54.0 ($SD = 19.4$) for Hologram + Robots (C1) conditions, 55.3 ($SD = 18.9$) for Hologram-Only (C2), and 51.6 ($SD = 16.9$) for Robots-Only (C3).
The Friedman test showed no significant difference ($\chi^2(2) = 0.30$, $p = 0.86$).

\subsubsection{System Usability}
The results for the system usability scale are shown in Figure~\ref{fig:study-sus-nasa} (B).
% We found a significant difference between C1 and C2 ( Z=XXX, p = XX < 0.05), as well as between C1 and C3 ($Z=XXX$, $p = XX < 0.05$). 
A higher score indicates higher usability.
The average score for each condition was 77.9 ($SD = 11.3$) for Hologram + Robots (C1) conditions, 73.3 ($SD = 14.2$) for Hologram-Only (C2), and 72.9 ($SD = 14.1$) for Robots-Only (C3).
The Friedman test showed no significant difference ($\chi^2(2) = 1.64 $, $p = 0.44$).
During the interviews, participants provided feedback on the usability.
One participant noted that \textit{``Conditions which use Toio were easy to manipulate'' (P3)}, and another participant noted that \textit{``It was easy to adjust the size of the virtual picture using Toio'' (P8)}.
Although Hologram + Robots (C1) had a higher average usability score (77.9) than the average score (68) ~\cite{sauro2011practical}, it was not significantly better than Hologram-Only (C2). 
One participant reported, \textit{``The coupling between the actual movements and the robot was slow and misaligned, which sometimes make it difficult to understand'' (P3)}. This feedback suggests that the low ability of coupling between the hologram and mobile robots may have negatively impacted usability.

\subsubsection{Preference}
The results for the preference are shown in Figure~\ref{fig:study-sus-nasa} (C).
75 \% of the participants preferred Hologram + Robots (C1) as the best, followed by Hologram-Only (C2) (17 \%) and Robots-Only (C3) (8 \%).
Chi-squared goodness of fit test revealed a significant difference from random choice ($\chi^2(2) = 9.5$, $p = 0.009 < 0.01$).
In our study, participants preferred Hologram + Robots (C1) over Hologram-Only (C2) and Robots-Only (C3).
Five out of nine participants mentioned social presence as a key factor in their preference for Hologram + Robots (C1), while the remaining four participants mentioned usability as a determining factor.
Therefore, the high social presence and usability in Hologram + Robots (C1) can enhance the overall user experience.


\subsection{Limitations and Design Implications}

\subsubsection{Precise Coupling between Holographic Users and Robot Movement}
In the applications used in the study, the coupling between the virtual body movements and mobile robots was occasionally slow and misaligned due to the Toio's maximum speed and the calibration error between the avatar and Toios.
Upon testing the start latency, the average latency was 0.483 s, 0.262 s, 0.443 s, and 0.615 s in D1, D2, D3, and D4, respectively.
This issue could potentially impact both the social presence and usability of the Hologram + Robots (C1) condition.
Employing faster mobile robots and implementing a more accurate position calibration method between the avatar and Toios could alleviate this problem.
% Utilizing a more accurate position calibration method between avatars and Toios, and enhancing the accuracy of body movement tracking could alleviate this problem.


\subsubsection{Noise of Robot Movement}
Several participants reported that the sound generated by the Toios could be distracting and interfere with their ability to concentrate on the task. 
For example, one participant commented \textit{``Toios sound was sometimes a little loud, and it was difficult to concentrate on the task.'' (P2)}, while another mentioned \textit{``I was distracted by the noise of Toios'' (P3)}.
Upon testing the noise levels generated when moving the Toio 45 cm in 4 seconds, the maximum recorded noise was 64.5 dB, 60.3 dB, 65.0 dB, and 70.0 dB in D1, D2, D3, and D4, respectively.
This issue could potentially impact the user experience and social presence.
To address this problem, we could improve the system to make Toios travel to their destination by the shortest route, reducing travel time and the duration of sound generation.

\subsubsection{Bi-Directional Collaboration between Participants}
% The setup for the experimenter and participant environments was different in order to simplify the environment. 
% This led to the experimenter only being able to see a 2D screen rather than a 3D avatar. As a result, 
% The experimenter may not have been able to accurately face their head towards the participant, which could have partially affected the social presence. 
Additionally, the collaboration in our study was between a participant and an experimenter.
To gain further insights, it may be beneficial to set up an environment where participants can collaborate with other participants without the presence of an experimenter.
This can provide insights on more realistic collaboration scenarios.

\subsubsection{Group Size}
In our study, collaboration was limited to only two people, one participant and one experimenter.
Using larger groups could potentially increase the number of interactions and affect the social presence and user experience.
However, this could also increase conflicts and misunderstandings.
Therefore, conducting studies with larger groups could help us understand how these factors influence our system.

\subsubsection{Number of Robots}
In our study, we used two Toios, but it is possible to use more.
One participant noted that \textit{``I thought it would be good if the picture application could increase the number of manipulable objects (Toio) and allow more complex UI manipulation'' (P8)}.
This comment suggests that using more Toios for UI manipulation could affect usability and user experience.
Additionally, we could use more Toios for body or hand representation, which could enhance the resolution of the remote user's movements and gestures, which could enhance the social presence.

\subsubsection{Enhancing Holographic Visualization}
In this study, we used a single Kinect camera to capture the remote user's body movements for holographic avatar generation.
Future work could expand this setup by adding more Kinect cameras to capture the user's hologram from multiple angles.
This could improve the remote user's clarity and accuracy via multi-directional coverage.
Through these improvements, the local user would better comprehend the remote user's intentions and interactions with the physical environment and overall body language.
% Other body-tracking systems could potentially be used as well, such as marker-based body-tracking and camera-based body-tracking for animated avatars.

% \subsubsection{Reducing Positional Errors}
% In our study, we observed that positional errors occurred when the remote user attempted to move the local Toio faster than its maximum speed. This would cause a disconnect between the remote user's hand/actions and the Toios location. In future work, we could replace Toios with other mobile robots capable of faster movement while maintaining accuracy, or investigate alternative moving methods that allow Toios to move more quickly without sacrificing precision.
