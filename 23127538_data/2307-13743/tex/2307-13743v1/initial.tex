\documentclass{article}
\usepackage[utf8]{inputenc}
\usepackage{multicol}
\usepackage{xcolor}
\usepackage{graphicx}
\usepackage{amsmath}
\usepackage{amsfonts}
\usepackage{float}


\definecolor{dorange}{rgb}{1,0.4,0}
\newcommand{\sdorange}[1]{\textcolor{dorange}{{#1}}}
\definecolor{olivegreen}{rgb}{0.2,.8,0.36}
\newcommand{\ardgreen}[1]{\textcolor{olivegreen}{{#1}}}
\newcommand{\yascyan}[1]{\textcolor{cyan}{{#1}}}
\newcommand{\red}[1]{\textcolor{red}{{#1}}}
\newcommand{\blue}[1]{\textcolor{blue}{{#1}}}


\title{Everpresent Lamda}
\author{sanjone }
\date{January 2022}

\begin{document}

\maketitle

\section{Questions}
\begin{itemize}
    \item What is a good time parameter? (proper time, conformal time, scale factor, spacetime volume, inverse or inverse square root of the spacetime volume, etc.). Whichever one we choose, we may want to consider equal or un-equal spacing between the steps.
    \item What starting $a$ or $z$ did previous works use and why? How far back do we want to start?
    \item Were any inflationary models for initial perturbations used when running CAMB in the 2018 paper?
    \item Why do there appear different expressions for the volume of the past lightcone in different references? (see Section 4 below) How different are the volumes given by the two expressions?
    \item Santanu had some question about $\Omega_b$ \sdorange{ My issue was not with $\Omega_b$ in particular. The issue is that we are beginning with some values of $\Omega_L$ and $H_0$. Then we are applying the standard LCDM model to get the energy density at $a=10^-8$ or so. We are evolving the Everpresent Lamda model and finding some value of $\Omega_L$, and $H_0$ at  $a=1$, which of course does not match with the initial values that we set. May be we can check it in the literature once. Its not such a big problem proably. However, we need to check it properly while doing the calculations. }
\end{itemize}

\section{Meeting 18 Jan 2020}

\begin{itemize}
    \item If I calculate the volume of the universe from $\Lamda$CDM model and calculate the $\rho_{\Lambda} = \frac{\frac{\hbar}{l_p^2}}{\sqrt{\mathcal{V}}}$, then the value of $\frac{\rho_\Lambda}{\rho_\text{cr}}$  is coming to be about $22$, which is not expected (Atleast I was not expecting it - \sdorange{Santanu}). \yascyan{In the Everpresent $\Lambda$ model, $\Lambda$ is a random number with mean value $0$ (we are assuming this)  and standard deviation $1/\sqrt{V}$ (in appropriate units). Therefore the model only says which range of values is more probable than which others, i.e. within one standard deviation is more natural or typical and outside it is less so. The model does not say that the value we observe needs to be exactly at the standard deviation of the distribution (which is $1/\sqrt{V}$). The currently observed value of $\Lambda$ therefore lies within the expected range.}
    \item \ardgreen{Plot of $\frac{\rho_\Lamda}{H^2}$ from Arad using his python code.} 
    % Figure environment removed
\end{itemize}


\section{Speeding up the Volume calculation}
The volume is given by 
\begin{equation}
\mathcal{V}(t)=\frac{4 \pi}{3} \int_{0}^{t} d t^{\prime} a\left(t^{\prime}\right)^{3}\left[\int_{t^{\prime}}^{t} d t^{\prime \prime} / a\left(t^{\prime \prime}\right)\right]^{3}
\end{equation}

Now $d\tau = \frac{dt}{a}$. Therefore, we should have 

\begin{eqnarray}
\mathcal{V}(t)=\frac{4 \pi}{3} \int_{0}^{\tau} d \tau^{\prime} a\left(t^{\prime}\right)^{4}\left( \tau - \tau' \right)^{3} \\
= \frac{4 \pi}{3} \left[ \tau^3 \int_{0}^{\tau} d \tau^{\prime} a\left(t^{\prime}\right)^{4}
- 3 \tau^2 \int_{0}^{\tau} d \tau^{\prime} a\left(t^{\prime}\right)^{4}\tau^{\prime}
+ 3 \tau \int_{0}^{\tau} d \tau^{\prime} a\left(t^{\prime}\right)^{4}\tau^{\prime 2} 
- \int_{0}^{\tau} d \tau^{\prime} a\left(t^{\prime}\right)^{4}\tau^{\prime 3}\right]
\end{eqnarray}

The value of all these 4 integrals can be stored in some variables in some steps. So if we want the value of the integration we can break it as follows

\begin{eqnarray}
\int_{0}^{\tau+\tau^{\prime}} d \tau^{\prime} a\left(t^{\prime}\right)^{4}
=\int_{0}^{\tau} d \tau^{\prime} a\left(t^{\prime}\right)^{4} + \int_{\tau}^{\tau+\tau^{\prime}} d \tau^{\prime} a\left(t^{\prime}\right)^{4}
\end{eqnarray}

The first part will be stored from previous integration. And we need to calculate the second part only. So this integration will be done in constant time or $O(1)$. As we need the values of  $\mathcal{V}(t)$ at different $t$s, lets say $n$ number of $t$s, the full process can be done in $O(n)$.  

\yascyan{\section{Volume Expressions}}

We are interested in the spacetime volume $V(t)$ of the past lightcone of a point at time $t$. This will be 

\begin{equation}
V(t)=\frac{4\pi}{3}\int_0^t dt'a(t')^3 r(t')^3,
\end{equation}
where $r(t')$ is the radial coordinate at time $t'$ for incoming null rays reaching the point at $t$, which is at the vertex of the lightcone. The expression for this is $r(t')=-\int_{t}^{t'} \frac{dt''}{a(t'')}$. If we were to instead take $r(t')=-\int_{t'}^{0} \frac{dt''}{a(t'')}$, we are considering the radius at $t=0$ of light rays that reach the observer at time $t'$.
\\ \\
\section{first two runs for the expansion history- Arad}
With the faster code for integration,  I could get one complete run in $\sim1$ hour. tolerance=1.0e-2 in rombint. However, the code hit the $H^2<0$ problem 50 and 250 times respectively before being able to run one clean run for the whole expansion history.\\
Here are two plots for $\rho_{\Lambda}/H^2$ vs time (in Mpc):

\begin{center}
    

     % Figure removed
\end{center}
\begin{center}
  % Figure removed
\end{center}
\\ \\
And here are the same functions vs redshift:\\ \\
  \begin{center}
     % Figure removed
    \end{center}
   

\begin{center}
  % Figure removed
  \end{center}
\begin{itemize}
    \item The final values for $\Omega_\Lambda$ are 0.70 and 0.64.
    \item Regarding Santanu's question on what time coordinate to use, I used equal steps in "a" to update volume, not in cosmic time. Fay made a very interesting comment on this. She said that the most natural time coordinate for us would be the volume itself, as it is directly proportional to the number of elements which are contributing to the stochastic lambda.
    \item I think "a" and cosmic time yield similar results because they have an almost liner relation; see the plot of scale factor vs time:
    \begin{center}
  % Figure removed
  \end{center}
However, volume (in $Mpc^4$) vs time behaves differently:
\begin{center}
  % Figure removed
  \end{center}
\end{itemize}

  \section{some statistics of the data- 23/03/2022}
  
  Here are the histograms for $H_0$ and $\Omega^0_\Lambda=\frac{\Lambda(a=1)}{3H_0^2}$ with different values for $\alpha$ and different steps. $da=1$ means steps of $\Delta a =10^{-4}$ were taken, and $dt=1$ means steps of 1 Mpc in time were taken.\\
  As you can see, for increasing $H_0$ to make it closer (statistically) to the observed value, we should choose $\alpha>0.015$.\\
Also there are rare cases of $\frac{\Lambda(a=1)}{3H_0^2}$ becoming as much negative as $-74$. We need to understand these cases better. In particular, why don't they terminate? (\sdorange{Please list those cases. We need to check those properly.})
  % Figure environment removed
\\ \\ \\ \\ \\ \\ \\ \\ \\ \\ \\ \\ \\ \\ \\ \\ \\ \\ \\
Next, we take a constant volume, like $10^{12}Mpc^4$, and look for the first place in an expansion history that it reaches that volume (which usually happens at $a\sim\frac{1}{2}$) and record the value of $\Lambda$. We can plot several histograms. We can either plot $\frac{\Lambda}{3H^2}$ which is the energy budget of cosmological constant at that moment, we can plot $\frac{\Lambda}{3H_0^2}$ where $H_0$ is the final hubble constant for that history, or we can plot $\frac{\Lambda}{3h_0^2}$ where $h_0$ is just a normalization constant, taken to be $70.74 km/(s.Mpc)$ and is the same for all histories. \\

\sdorange{I think only the last plot is relavent as the theory sayd that $\rho_\Lambda$ will be gaussian}

\sdorange{Also take 3-4 other Volume and plot the distribution of $\rho_\Lambda$. Variance should be inversely proportional to volume.}

Q-Q plots show that only the last one is Gaussian:

% Figure environment removed

Next, we plot the histogram of the scale factor for which a run terminates. All such plots look the same, and here is one of them:


\sdorange{I think you need to remove all the run which are getting terminated at the beginning. Only then it will provide us some relevant plots. Remove all the runs which are crashing within lets say first 10 steps and then make the plots.}

% Figure environment removed
  
\section{Number of terminations}
$\alpha=0.01$, $\delta_t=3$ :\\
---------------------------------\\
Run:                     1770 \\
Crashed:              8230   \\
Crashed after 5: 1153\\
Run/Crash5 :      1.54\\
\\
\\
$\alpha=0.01$, $\delta_t=2$ : \\
---------------------------------\\
Run:                     2823 \\
Crashed:              7147   \\
Crashed after 5: 1587\\
Run/Crash5 :      1.77\\
\\
\\
$\alpha=0.01$, $\delta_t=1$ :\\
-------------------------------\\
Run:                     4748\\
Crashed:             5252   \\
Crashed after 5: 2369\\
Run/Crash5 :      2.0\\
\\
\\
$\alpha=0.015$ $\delta_t=1$:\\
-------------------------------\\
Run:                     256\\
Crashed:              9744   \\
Crashed after 5: 4275\\
Run/Crash5 :      0.0599\\
\\
\\
$\alpha=0.02$, $\delta_t=1$ :\\
-------------------------------\\
Run:                     9\\
Crashed:              9991  \\ 
Crashed after 5:  3090\\
Run/Crash5 :      0.0029\\

  
\section{\sdorange{Our discussion on 1st April and the issue in Ever present Lambda model}}
\subsection{\sdorange{Issue with the initial volume}}
In our equation 
\begin{eqnarray}
\mathcal{V}(t) &=&\frac{4 \pi}{3} \int_{0}^{a} da^{\prime} \left(\frac{d \tau^{\prime}}{da^{\prime}}\right) a\left(t^{\prime}\right)^{4}\left( \tau - \tau' \right)^{3} \nonumber \\
&=& \frac{4 \pi}{3} \int_{0}^{10^{-8}} da^{\prime} \left(\frac{d \tau^{\prime}}{da^{\prime}}\right) a\left(t^{\prime}\right)^{4}\left( \tau - \tau' \right)^{3} + \frac{4 \pi}{3} \int_{10^{-8}}^{a} da^{\prime} \left(\frac{d \tau^{\prime}}{da^{\prime}}\right) a\left(t^{\prime}\right)^{4}\left( \tau - \tau' \right)^{3} \nonumber
\end{eqnarray}

\begin{enumerate}
\item  Without any inflationary field its not possible to calculate the first integral. Now we are taking the first integral to be $0$ which is definitely now correct. In fact due to inflation it will take very large value. 
\item As $a\longrightarrow 0$, $\tau$ will become infinite. (I think that the integration will converge because there is a $a^4$ factor. However, we need to calculate the integration analytically. )
\end{enumerate}

\subsection{\sdorange{Issue with energy conservation}}


Einstein equations and the Newtons equations are the same except in 3 cases. 1) There is a strong gravitation field, 2) Gravitation field rapidly varying (for GW) 3) Concerned speeds are close to $c$. In our universe none of these are the case. So we can check whats going on in the Newtons model, because its easier to visualize. 

Consider a comoving test particle of mass $m$ located at a position $\mathbf{r}=a(t) \mathbf{x}$ relative to a comoving observer defining the origin. Next we examine the gravitational attraction experienced by the test particle due to the ball of fluid of radius $|r|$ centred at the observer. Conservation of energy requires that
\begin{equation}
    \frac{1}{2} m \dot{\mathbf{r}}^{2}-\frac{G M m}{|\mathbf{r}|}=\frac{1}{2} m\left(\dot{a}^{2}-\frac{8}{3} \pi G \rho a^{2}\right) \mathbf{x}^{2}=-\frac{1}{2} m k c^{2} \mathbf{x}^{2}
\end{equation}
since $M=\frac{4}{3} \pi \rho|\mathbf{r}|^{3}$, the gravitational attraction being determined only by the mass within the ball, here $k$ is a constant proportional to the conserved energy, for consistency it is independent of $\mathrm{x}$. Rearranging this equation leads to the Friedmann equation:
$$
\dot{a}^{2}+k c^{2}=\frac{8 \pi G \rho a^{2}}{3}
$$
The first law of theomodynamics, $\mathrm{d} E=\mathrm{d}\left(\rho c^{2} V\right)=-p \mathrm{~d} V$ and thus
$$
\dot{\rho}=-\frac{\dot{V}}{V}\left(\rho+\frac{p}{c^{2}}\right)=-\frac{3 \dot{a}}{a}\left(\rho+\frac{p}{c^{2}}\right)
$$
since comoving volumes are proportional to $a^{3}$. So, differentiating the first equation and replacing the second we get  
$$
\ddot{a}=-\frac{4 \pi G}{3}\left(\rho+\frac{3 p}{c^{2}}\right) a
$$

----------------------

We must not add energy in one side of the equation. We need to add same amount of energy in both the sides of the equation if we want a consistent equation. Right hand side will give us a overall dark energy. However it will not satisfy the energy conservation equation. 

\section{On pressure of the dark energy- 14/04/2022}

Assuming the 1st Friedmann equation and the continuity equation for matter and radiation, we can derive the following equation:
$$\frac{\ddot{a}}{a}=-\frac{4\pi G}{3}(\rho +3p)+\frac{1}{6}(2\Lambda+\frac{\dot{\Lambda}}{H}) $$
Therefore, if we want to keep the second Friedmann equation, in order to be consistent with the perturbation theory which also uses the full Einstein equation, we have to say that the pressure of dark energy is:
$$8\pi G \rho_{\Lambda}=\Lambda, \hspace{1cm}8\pi G p_{\Lambda}=-\Lambda-\frac{\dot{\Lambda}}{3H}\\
\hspace{1cm} \Rightarrow\hspace{5mm} w=-1-\frac{\dot{\Lambda}}{3H\Lambda}$$
Here are the plots of $H$, $\ddot{a}$, and $w$ for a run that terminated at $a=0.9$: 

% Figure environment removed

I have omitted the initial 100 data point for the plot of $H$ and $\Omega_{\Lambda}$. For computing $\ddot{a}$, I used spline to make $\frac{da}{dt}$ smooth and then take derivatives. $w$ oscillates badly and its mean is NOT -1 (in this case it is -14). Since the constraints people have put on $w$ are tight, we should discuss if we really want to take this pressure expression seriously. After all, this pressure has no interpretation in the gravitational action (?).  

\section{Marginalizing over seeds- 21/04/2022}
The true physical parameters of the model (at least those relevant for supernova) are $(\alpha, \Omega_m h^2)=\theta$. There is also the auxiliary parameter $seed=s$, which we need for calculating likelihoods.
$$\begin{aligned}L\left( \Omega _{m}h^{2},\alpha \right) =L\left( \theta \right) =P\left(  D| \theta \right) =\sum _{seed}P\left(  D| \theta ,seed\right) P\left( seed\right) \\
=\dfrac{1}{N}\sum _{s}P\left(  D| \theta ,s\right) =\dfrac{1}{N}\sum _{s}L\left( \theta ,s\right) \end{aligned}$$
By $D$ I mean data, and N is the number of seeds. So the likelihood on $\theta$ is an average over $L(\theta,s)\propto \exp(-1/2\chi^2(\theta,s))$.\\
Therefore, for a correct MCMC in $\theta$ space, we have to use $$\dfrac{L\left( \theta _{i+1}\right) }{L\left( \theta i\right) }=\dfrac{\sum _{s}\exp \left( -\dfrac{1}{2}\chi ^{2}\left( \theta _{i+1},s\right) \right) }{\sum _{s'}\exp \left( -\dfrac{1}{2}\chi ^{2}\left( \theta _{i},s'\right) \right) }.$$ This suggests the following recipe: Take each step of MCMC only in $(\alpha, \Omega_m h^2)=\theta$ space; for each $\theta$, take 1000 seeds, and compute the $\chi^2$ for those 1000 seeds in \boldsymbol{parallel} in the cluster, then sum over likelihoods to find the transition probabilities. We will be sampling the posterior: $P(\theta\sdorange{|D})=\frac{L(\theta)}{Z}$.\\
One might want to work things out in 3d $(\theta,s)$ parameter space instead. 2 ways come to mind:\\
1. Change seed at each step. The ratio of transition probabilities would be:
$$\dfrac{T\left(  \theta _{i+1},s_{i+1}| \theta _{i},s_{i}\right) }{T\left(  \theta _{i},s_{i}| \theta _{i+1},s_{i+1}\right) }=\dfrac{L\left( \theta _{i+1},s_{i+1}\right) }{L\left( \theta _{i},s_{i}\right) }=\dfrac{\exp \left( -\dfrac{1}{2}\chi^{2}\left( \theta _{i+1},s_{i+1}\right) \right) }{\exp \left( -\dfrac{1}{2}\chi^{2}\left( \theta _{i},s_{i}\right) \right) }.$$
By updating $s$ at each step, we will be sampling the posterior $P_1(\theta,s|D)=\frac{1}{N}\frac{L(\theta,s)}{Z}$ ($\frac{1}{N}$ is due to the prior on seeds); summing over $seed$ will give us the desired 2d posterior: $P(\theta|D)=\sum_{s} P_1(\theta,s|D)$. The downside is that this should be done in series in the cluster which takes too long.
\\2. Keep the seed constant at each run of MCMC. So take a constant $s$ and update $\theta$. The transition probabilities are:
$$\dfrac{T\left(  \theta _{i+1},s| \theta _{i},s\right) }{T\left(  \theta _{i},s| \theta _{i+1},s\right) }=\dfrac{L\left( \theta _{i+1},s\right) }{L\left( \theta _{i},s\right) }=\dfrac{\exp \left( -\dfrac{1}{2}\chi^{2}\left( \theta _{i+1},s\right) \right) }{\exp \left( -\dfrac{1}{2}\chi ^{2}\left( \theta _{i},s\right) \right) }$$
After one run, we will sample the posterior $P_2(\theta|s,D)=\frac{L(\theta,s)}{Z_2(s)}$ for a constant $s$, which this time satisfies: $\int d\theta P_{2}\left( \theta |s,D\right) =1$; therefore $Z_{2}\left( s\right) =\int d\theta L\left( \theta ,s\right) $. So $$\Rightarrow P_{2}\left( \theta |s,D\right) =\dfrac{L\left( \theta ,s\right) }{\int d\theta 'L\left( \theta ',s\right) }$$Hence, unlike $P_1$, here $P(\theta|D)\neq\sum_{s} P_2(\theta|s,D)$. In fact there is no simple algebraic relation between our desired posterior $P$ and this $P_2$ (\sdorange{Wrong. There exists}).(Note that in the previous method, since we were also searching in seed numbers, we had $\sum_s\int d\theta P_{1}\left( \theta ,s\right) =1$ and so $Z =\frac{1}{N}\sum_s\int d\theta L\left( \theta ,s\right)$ which is s-independent.)

\sdorange{I think what you point out is correct. However, I think it can be solved easily. We can each of the  chains independently and then combine those.} 

\sdorange{$Z = \int d\theta' L(\theta)=\int d\theta' P(D|\theta)=P(D) = 1$ ... as data is given}

\sdorange{$Z = \int d\theta' L(\theta,s)=\int d\theta' P(D|\theta,s)=P(D|s)$}

\sdorange{$P(D|s)$ is the probability of data given a particular seed. When all the chains are running for equal number of chain points the number of accepted data points are giving the same thing. When you are taking all the samples of equal size we are calculating the same probability. I mean for each seed P(D|s) is just $\sum exp(-\chi^2/2)$}

\sdorange{Now $P(\theta|D) = \sum P(\theta, s |D) P(D|s) = \sum L(\theta,s) = L(\theta)$ as prior on $s$ is constant.} 

Update on method 2:\\
After finding $P_2$, we can do a summation with proper coefficients to find the real posterior $P$. Just note that:
$$P(\theta|D)=\sum_{s} P_2(\theta|s,D)P(s|D)$$
On the other hand $P(s|D)=\frac{P(D|s)\pi(s)}{\sum_{s'}P(D|s')\pi(s')}=\frac{\int d\theta P(D|s,\theta)\pi(\theta)}{\sum_{s'}\int d\theta' P(D|s',\theta')\pi(\theta ')}=\frac{\int d\theta L(\theta,s)}{\sum_{s'}\int d\theta' L(\theta',s')}$



\section{MCMC results for supernovae}
Here are the three seed numbers that give results better than $\Lambda$CDM.
\subsection{seed -10892}
% Figure environment removed

% Figure environment removed
Why is $\alpha$ vs $\Omega_m$ perfectly correlated? The reason is that if you keep the seed and $\alpha$ constant, and then multiply $\rho_m^0$ by some constant $c_0$, then all $\Lambda$, $H^2$, and $\frac{1}{\sqrt{V}}$ also get multiplied by $c_0$. In particular, this is saying that the ratio $\frac{\rho_m}{\Lambda}$ and hence $\Omega_m$ remain constant if we keep $\alpha$ and the series of random numbers fixed. Therefore, for a fixed seed number, one will be moving along some fixed curve in the $\alpha-\Omega_m$ plane.

\subsection{seed -11126}
% Figure environment removed

% Figure environment removed


\subsection{seed -11927}
% Figure environment removed

% Figure environment removed

\section{CMB Comments}

\begin{itemize}
    \item There are seeds which are coming close to LCDM ( $\chi^2$ diff about 20 ).  The seeds for which $(\Omega_m)_{min}$ is very high is not working that well.  
    \item One issue is that for most of the cases the Hubble parameter is becoming very low, sometimes even less than 50. 
    \item  The seeds are not matching with the supernova seeds. 
    \item Based on the data we can do a joint analysis. ( Supernova + CMB )
\end{itemize}


% Figure environment removed

% Figure environment removed




\end{document}