% Please make sure you insert your data according to the instructions in PoSauthmanual.pdf
\documentclass[a4paper,11pt]{article}

\usepackage{pos}
\usepackage{multirow}
\usepackage{xcolor}

\title{Search for TeV Neutrinos from Seyfert Galaxies in the Southern Sky using Starting Track Events in IceCube
}

\ShortTitle{Southern Hemisphere Seyfert Galaxy Analysis using Starting Track Events with IceCube}

% Don't change:
\author{The IceCube Collaboration \\{\normalsize \normalfont(a complete list of authors can be found at the end of the proceedings)}\\}

% Your emails:
\emailAdd{shiqiyu@icecube.wisc.edu}

\abstract{

% start of abstract
Supermassive black holes (SMBHs) power active galactic nuclei (AGN). The vicinity of the SMBH has long been proposed as the potential site of particle acceleration and neutrino production. Recently, IceCube reported evidence of neutrino emission from the Seyfert II galaxy NGC 1068. The absence of a matching flux of TeV gamma rays suggests that neutrinos are produced where gamma rays can efficiently get attenuated, for example, in the hot coronal environment near the SMBH at the core of the AGN. Here, we select the intrinsically brightest (in X-ray) Seyfert galaxies in the Southern Sky from the BAT AGN Spectroscopic Survey (BASS) and search for associated neutrinos using starting track events in IceCube. In addition to the standard power law flux assumption, we leverage a dedicated disc-corona model of neutrino production in such an environment to improve the discovery potential of the search. In this contribution, we report on the expected performance of our searches for neutrinos from these Seyfert galaxies.

% end of abstract

\vspace{4mm}
{\bfseries Corresponding authors:}
Shiqi Yu$^{1*}$, Ali Kheirandish$^{2}$, Qinrui Liu$^{3}$, Hans Niederhausen$^{1}$ \\
{$^{1}$ \itshape Michigan State University}\\[4mm]
{$^{2}$ \itshape University of Nevada, Las Vegas}\\[4mm]
{$^{3}$ \itshape Queen's University}\\[4mm]
$^*$ Presenter

\ConferenceLogo{PoS_ICRC2023_logo.pdf}

\FullConference{The 38th International Cosmic Ray Conference (ICRC2023)\\ 26 July -- 3 August, 2023\\ Nagoya, Japan}
}

\begin{document}

\maketitle

\section{Introduction}\label{sec_introduction}


\noindent The IceCube Neutrino Observatory at the South Pole recently reported evidence of TeV neutrino emission from the nearby active galaxy, NGC 1068 \cite{IceCube:2022der}. This reinforces the idea that AGN could contribute to the diffuse flux of high-energy neutrinos observed by IceCube since 2013 \cite{IceCube:2013low}. Assuming the flux of neutrinos from NGC 1068 to follow a single power-law, the inferred muon neutrino flux at neutrino energy $E_\nu=1\,\mathrm{TeV}$ is $5\times 10^{-14} \, \rm GeV^{-1}\, cm^{-2}\, s^{-1}$ with spectral index of $3.2$ \cite{IceCube:2022der}. The corresponding neutrino luminosity in the $1.5\,\mathrm{TeV}$ to $15\,\mathrm{TeV}$ energy range is much higher than the reported GeV-scale $\gamma$-ray luminosity by the {\em Fermi} Large Area Telescope (LAT)~\citep{Fermi-LAT:2019yla,Ballet:2020hze} and the neutrino flux exceeds TeV-scale gamma-ray upper limits reported by MAGIC~\cite{MAGIC:2019fvw}.\\
The cosmic-ray (CR) interactions that produce neutrinos are expected to also generate a matching $\gamma$-ray signature. The absence of TeV gamma-rays from NGC 1068, therefore, suggests that the production site of high-energy neutrinos must be opaque to high energy $\gamma$-rays. Such conditions can be found in the matter and radiation-dense region in the closest vicinity of the central SMBH, commonly referred to as AGN corona, a component of the AGN that is known for its brightness in the X-ray band \cite{Inoue_2019,Inoue:2019yfs,Murase:2019vdl,Eichmann:2022lxh}. Coronae are generally assumed to be composed of a hot, highly magnetized, and turbulent plasma in the inner region of the accretion disk that is supported by accretion dynamics and magnetic dissipation ~\cite{Miller:1999ix}. The comptonization of soft photons from the accretion disc is thought to produce the observed X-rays. If protons are accelerated in this environment, for example, by turbulent magnetic fields, the Seyfert galaxies that are intrinsically bright in X-rays may also shine bright in neutrinos \cite{Kheirandish:2021wkm}. In this work, we refer to the model presented in \cite{Murase:2019vdl, Kheirandish:2021wkm} as the \textit{disc-corona model}.\\
Assuming parameters that could explain the neutrino flux from NGC 1068, and following \cite{Kheirandish:2021wkm}, we use this model to relate the intrinsic X-ray fluxes to neutrino production for the brightest Seyfert galaxies reported by BASS in the Southern Sky ($\delta<-5\,^\circ$). We then use down-going starting track events \cite{Silva:2023icrc} recorded by IceCube to search for the corresponding neutrino emission from these objects. In other words, in this work, we study sources that may be similar to NGC 1068 as far as the production of neutrinos is concerned. Experimental results from a related IceCube study, applying the same search strategy to Northern Seyfert galaxies while using through-going neutrino-induced muon tracks in the Northern Sky, are presented in a separate contribution \cite{IceCube:2023northseyfert}. 

\section{Neutrino Dataset and Source Selection}
\label{sec_selection}
\noindent Recent progress in the background rejection and reconstruction methods for starting events in IceCube, cascades \cite{Abbasi:2023bvn} and starting tracks \cite{Silva:2021pos:estes, Silva:2023icrc}, has significantly improved the detector's sensitivity to neutrino sources in the Southern Sky. Here, we analyze starting track events recorded by IceCube during a $10.3$ year period from May 2011 to January 2022. The event selection criteria and event reconstruction methods are discussed in \cite{Silva:2021pos:estes, Silva:2023icrc}. $10,350$ events pass all selection criteria, of which $2,091$ events are down-going, i.e., originate from the Southern Sky, and therefore contribute to this analysis. 90\% of these events have estimated neutrino energies ranging from $1.4\,\mathrm{TeV}$ to $29\,\mathrm{TeV}$. The median angular resolution in the Southern Sky ranges from $\sim 2\,\mathrm{deg}$ at a neutrino energy of $1\,\mathrm{TeV}$ to $\sim 0.4\,\mathrm{deg}$ at $1\,\mathrm{PeV}$. For a detailed discussion of this neutrino dataset and its performance, see \cite{Silva:2021pos:estes, Silva:2023icrc}.
% Figure environment removed\\

\noindent BASS~\cite{Ricci:2017dhj} is an all-sky study of Swift/BAT detected X-ray AGN. We use the source classifications from the 105-month $Swift$-BAT survey ~\cite{Oh:2013wzc, Oh:2018wzc} to identify the Seyfert galaxies among the Southern sources in BASS. Combining the disc-corona neutrino flux model  \cite{Murase:2019vdl, Kheirandish:2021wkm} with the intrinsic X-ray luminosities ($2-10\,\mathrm{keV}$, absorption corrected), and distances reported by BASS, we compute the expected number of neutrinos in our dataset, accounting for the corresponding IceCube neutrino effective area. Fig. \ref{fig:selection} (left) shows the resulting expected number of neutrinos as a function of the distance of each Seyfert galaxy. We then choose the top 14 Southern Seyfert galaxies for our analysis using the neutrino expectation as a metric.
%{(\bf \color{magenta} we need to state the threshold)}. 
The neutrino fluxes predicted by the disc-corona model for our selection of Seyfert galaxies is shown Fig. \ref{fig:selection} (right). The two objects with the largest number of expected neutrinos are Centaurus A (Cen A) and Circinus Galaxy, with $2.5$ and $1.7$ neutrinos predicted,  respectively, if all of our assumptions are exactly met. In total, we expect $7.2$ neutrinos from our selection of Seyfert galaxies. The distribution of these sources in the Southern Sky and their expected number of neutrinos are shown in Fig. \ref{fig:skymap}. 

\noindent Cen A differs from the typical Seyfert galaxy in our sample in that it is known for its radio-bright, relativistic jet. While typical, "radio-quiet" Seyfert Galaxies are usually not detected in the gamma-ray bands \cite{Ackermann_2012}, Cen A is known to emit gamma-rays up to the VHE regime \cite{HESS:2018cxr}, which may originate from this jet. Similarly, the origin of the X-ray emission of Cen A is under debate, with a jet origin being one possibility \cite{Kheirandish:2021wkm}. In our modeling, we have assumed that the X-rays are coronal in nature, thus making the predicted number of neutrinos highly uncertain for this source. As the X-ray emission from Circinus Galaxy is believed to be coronal in origin, the model prediction is more robust, thus making it the most promising source in this analysis. Due to its proximity to the Milky Way, Circinus Galaxy is, in principle, visible as an extended object. Nevertheless, neutrino emission from the coronal region of its AGN would appear as a point-like source to IceCube. 

%Among all the selected sources, Centaurus A gives the highest neutrino expectation based on the intrinsic luminosity and distance reported in BASS. However, the X-ray flux measured from Centaurus A could originate from its bright jets instead of the core, which may not necessarily contribute to the coronal emission. {\em Hence, the most promising source in our catalog is the Circinus Galaxy. Consequently, to avoid biasing the stacking analyses by including Centaurus A, where the disk-corona model may not apply, it is excluded from the stacking analysis to be conservative and only included in the catalog search.}

%The dataset we used to study the selected Seyfert galaxies in the Southern sky is a sample of events collected by IceCube from 2011 to 2021 that selected by the enhanced starting track event selection (ESTES)~\citep{icrc2023:estes, Silva:2023icrc, Aartsen:2023prd}, {\em which have a state-of-the-art angular resolution and background rejection in the Southern Hemisphere}. 

\section{Analysis Methodology}\label{sec_method}
\noindent Following the analysis methods of the companion analysis in the Northern Sky \cite{IceCube:2023northseyfert}, we will perform two types of searches: a catalog and a stacking search.\\ 

% Figure environment removed

\noindent \textbf{Catalog Searches}\hspace{2em}Here, we individually analyze each of the 14 sources in our selection for an excess of point-like neutrino emission over backgrounds from atmospheric neutrinos, the diffuse astrophysical neutrino flux \cite{diffus:nue/nutau, diffuse:numu}, and neutrinos from our own galaxy \cite{Abbasi:2023bvn}. We also perform a binomial test \cite{IceCube:2018ndw}, a technique that allows us to combine, in a statistical manner, potential excess neutrino emission from multiple sources in a relatively model-independent way, i.e., without having to make any prior assumptions about the relative strength of the emission across the sources in our catalog. Here, we are primarily interested in results assuming the disc-corona flux model, but we will also perform the analysis assuming the canonical power-law flux to guard against model miss-specifications and enable easier comparisons to other IceCube works.\\

\noindent \textbf{Stacking Search}\hspace{2em}Assuming all of our model assumptions, including the intrinsic X-ray fluxes estimated in BASS, are correct, a stacking search, the joint analysis of all sources in which sources are weighted according to their expected number neutrinos, provides the most statistical power in identifying coronal neutrino emission from our set of sources. However, the increase in statistical power over the binomial search comes at the cost of increased impact of systematic errors in our model assumptions (or inputs). Because the strong X-ray emission of Cen A may be related to its jet activity rather than coronal processes, we exclude this source from our stacking search. Otherwise, this source would make the dominant contribution to this type of analysis, thereby propagating this systematic uncertainty to the other sources in the stack. The stacking search is performed only for the disc-corona model, i.e., the power-law flux is not tested in this part of our work.\\

%\noindent \textbf{Likelihood Method}\hspace{2em}
\noindent All searches rely on the standard unbinned maximum likelihood formalism \cite{Braun:2008bg, IceCube:2022der}, which considers the reconstructed event directions and energies, as well as estimated per-event angular uncertainties as the basis of the statistical method. While the generic power-law flux introduces two fit parameters, the mean number of signal events $n_s$ and spectral index $\gamma$, our disc-corona flux model (including all assumptions) fully specifies the expected neutrino observations, and the model-based search could therefore be performed without any free parameters. We acknowledge the existence of systematic uncertainties in the estimated X-ray fluxes, especially for Compton-thick AGN, the estimated source distances as well as other parameters intrinsic to the model. Changing, for example, the assumed cosmic-ray to thermal pressure ratio \cite{Kheirandish:2021wkm}, would change the normalization of the predicted neutrino flux. For each source, we therefore treat the normalization, corresponding to $n_s$, as one free parameter. In the stacking search, the total normalization is kept free, while the relative contributions of the different sources are kept fixed at the nominal predictions.\\
% Figure environment removed

%\noindent\textbf{Background Modeling}\hspace{2em}
\noindent Because we are analyzing the Southern Sky, we need to account for four types of backgrounds: atmospheric muons, atmospheric neutrinos, astrophysical neutrinos, and neutrinos from the Milky Way. The recent NGC 1068 result deployed MC simulations to model the backgrounds, because it naturally avoids signal contamination in the background data, while at the same time being able to fully sample the observable space \cite{IceCube:2022der}. Here, in the Southern Sky, systematic uncertainties in the modeling of the muon background as well as its correlation with atmospheric neutrinos, the so-called self-veto effect \cite{Arguelles:2018awr}, prevent us from using the same method. Instead, we rely on the well-tested data scrambling technique. Experimental data is randomized in the right ascension coordinate, in which the detector response, as well as atmospheric and isotropic astrophysical backgrounds, appear uniform. However, this is not true for Galactic neutrinos. In addition, the technique suffers from signal contamination, as potential point-like neutrino emission would be added to the background model, thereby reducing the sensitivity of the search. To alleviate both issues, we modify the standard scrambling method as follows: we first apply a masking technique. Events that fall into the vicinity of the brightest Seyfert galaxies (Cen A, Circinus, NGC 7582, and ESO 138-1, circle of radius 7$^{\circ}$), or the Galactic plane (Galactic latitude $\pm10^{\circ}$ ), are removed from the dataset. Scrambling the remaining data, while accounting for the missing solid angle, then gives an excellent approximation to the atmospheric and diffuse astrophysical backgrounds. Finally, we add simulated events from the Galactic plane into this hybrid dataset, using the best-fit normalization reported in \cite{Abbasi:2023bvn} for the neutrino flux expected from the diffuse galactic gamma-ray flux ($\pi^0$ template) measured by Fermi-LAT ~\cite{paper:pi0}. Following \cite{Abbasi:2023bvn}, we assume the corresponding neutrino flux to follow a power-law with a spectral index of 2.7. Using this hybrid dataset, we obtain the background pdfs required by the likelihood formalism. It also allows us to generate arbitrarily many background-only datasets but with strongly reduced signal contamination. The latter is necessary to compute background test-statistics distributions, a crucial ingredient for the likelihood ratio test. The original, unscrambled, unmasked experimental data is used to evaluate the likelihood function on the experimental data during the data fitting process.

%For this analysis, we search for potential neutrino excess from the direction of the selected sources by scrambling observed data by their reconstructed right ascension to represent the underlying isotropic backgrounds on the sky. In the Southern Hemisphere, neutrino emissions from the Galactic plane (GP) are observed by IceCube~\cite{paper:gp}. 
%\noindent As shown in Figure~\ref{fig:skymap}, some of our catalog sources are close to the GP. Therefore, we developed analysis techniques to handle the potential neutrino emissions from the GP. To avoid scrambling the potential neutrino emissions from astrophysical sources into the background representation, we mask the regions within  along the GP and 7$^{\circ}$ around the hottest four galaxies (Centaurus A, Circinus, NGC 7582, and ESO 138-1) in the selected catalog to be conservative. This helps us avoid over-estimating the background by accidentally scrambling potential neutrino emissions from the possible sources in the Southern Hemisphere. After masking and scrambling, we have the representation of isotropic backgrounds from the diffuse astrophysical flux and atmospheric muons and neutrinos. However, with the knowledge of GP's presence near the regions where some of our hottest catalog sources are located, we carefully draw MC events following the conservative $\pi^0$ template~\cite{paper:pi0}, which is inferred from the $\gamma$-ray emissions and follows a power-law in photon energy (E) of $E^{-2.7}$, to represent the potential neutrino candidates from the GP. Then, we add them to the masked-then-scrambled background events. The normalization of the $\pi^0$ template is from~\cite{paper:gp}, which is consistent with a similar GP study using ESTES (which found a post-trial p-value of 0.070 for $\pi^0$ template)~\cite{icrc2023:estes}.
% Figure environment removed

\section{Projected Analysis Performance}\label{sec_result}
\noindent We measure the performance of the analysis using the sensitivity and discovery potential as metrics. The sensitivity is defined as the median value of the upper limits that the experiment could obtain in a background-only universe. The discovery potential corresponds to the flux needed to make a $5\sigma$ discovery with $50\,\%$ probability. Because no analytic closed-form expression for the disc-corona flux model exists, we will quote these metrics in units of the number of expected events. A comparison between the sensitivity and discovery potential for individual objects in our catalog search and the expected number of events according to the disc-corona model is shown in Fig. \ref{catalog}. According to nominal model assumptions, identification of individual sources are not very likely. However, the collective neutrino emission could provide a strong signal for a stacking analysis. The sensitivity and discovery potential of the stacking search are shown in Fig. \ref{fig:stacking_result}. If all our assumptions (e.g., the intrinsic X-ray fluxes by BASS, the disc corona model with intrinsic parameters tuned to NGC 1068) are exactly met, the $3\sigma$ discovery potential corresponds to $7.0$ neutrinos (Cen A excluded), i.e., $\sim 150\,\%$  of the expected neutrino flux. %The discovery potential remains above the total number of neutrinos expected in this sample. However, the impact of statistical fluctuations in the reconstructed properties of possible signal events, for example the angular errors, induce large statistical uncertainties on the expected significance, and higher significances are possible, even if the nominal assumptions are exactly met. Having implemented the disc-corona flux model and assuming that it is correct, we improve the discovery potential in both searches over a generic search using the single power-law by \color{red}$~\sim 10\%$\color{black}.

%\begin{table}[bth!]
%\centering
%\begin{tabular}{|c|c|c|c|c|c|c|c|c}
%\hline
%& spectral model & $n_{\rm exp}$ & TS & $\hat{n}_{\rm s}$ &$\hat{\gamma}$ & $p_{\rm local}$& $n_{90\%\rm{C.L.}}$\\
%\hline\hline
%\multirow{ 2}{*}{Circinus}
%& disk-corona & 1.7 & 6.9 & 3.7& - & $4.0\times 10^{-3}\,(2.7\,\sigma)$&  10.3\\
%& power-law & - & 10.6 & 3.2 & 2.5 & $2.0\times 10^{-3}\,(2.9\,\sigma)$& 11.6 \\
%\hline
%\multirow{ 2}{*}{NGC 7582}
%& disk-corona & 0.8 & 1.4 &1.2 & - & $6.4\times 10^{-2}\,(1.5\,\sigma)$ & 4.9\\
%& power-law & -&1.8 &1.8& 4.2& $1.1\times 10^{-1}\,(1.2\,\sigma)$ & 4.8\\
%\hline
%\multirow{ 2}{*}{ESO 138-1}
%& disk-corona & 0.6 &3.1 & 1.7&-& $5.0\times 10^{-2}\,(1.6\,\sigma)$& 5.9\\
%& power-law & -&3.1 &1.9& 3.6 & $1.6\times 10^{-1}\,(1.0\,\sigma)$& 5.1\\
%\hline
%\multirow{ 2}{*}{NGC 5728}
%& disk-corona & 0.3&1.0 & 1.2&-& $1.2\times 10^{-1}\,(1.2\,\sigma)$& -& 3.7\\
%& power-law & -&1.4 & 1.4&6 & $1.6\times 10^{-1}\,(1.0\,\sigma)$& -& %\\
%\hline\hline
%Stacking & disk-corona & 4.7& 8.5 &6.8 & - & $1.8\times 10^{-3}\,(2.9\,\sigma)$ & 14.2\\
%\hline\hline
%\end{tabular}
%\caption{Results of three hottest sources in the catalog and stacking of catalog excluding Centaurus A, with expected neutrino emissions computed using disk-corona model ($n_{\rm{exp}}$); fitted test-statistic values (TS); fitted number of neutrinos ($\hat{n}_{\rm s}$) and power-law spectral index ($\hat{\gamma}$); pre-trial p-value ($p_{\rm local}$); and estimated 90\% C.L.\ upper limit on $\hat{n}_{\rm s}$ ($n_{90\%\rm{C.L.}}$) where we assume a power-law flux with a spectral index of 3.}
%\label{table:result}
%\end{table}



\section{Summary and Outlook}\label{sec3}
\noindent Motivated by the recently reported evidence for neutrino emission from the Seyfert galaxy NGC 1068 in the Northern Sky, and the fact that the majority of the bright Seyfert galaxies reside in the Southern hemisphere,  we have developed a search for neutrino emission from the intrinsically brightest-in-X-ray ($2-10\,\mathrm{keV}$) Seyfert galaxies in the Southern Sky. This analysis offers new opportunities to investigate neutrino emission from some of the primary candidates for IceCube's high-energy neutrino flux. The analysis uses starting track events in IceCube, while the selection of sources rests on the information provided by BASS. According to the disc-corona model for neutrino emission from AGN, the Circinus Galaxy is the most promising target in our search because of its high intrinsic X-ray flux that is believed to originate from the corona of the AGN. Because of the large atmospheric muon background in the Southern Sky, IceCube's performance in is reduced compared to the North, where the Earth shields the detector. Thus, a robust identification of individual sources in our catalog with IceCube is challenging because the number of neutrinos expected from each source is small. Combining possible signals using a likelihood-based stacking method is more promising. \\ 
A closely related analysis, using essentially the same methods, has recently been applied to 27 X-ray bright Seyfert galaxies in the Northern Sky (excluding $\mathrm{NGC}\,1068$) \cite{IceCube:2023northseyfert}. While the data for $\mathrm{NGC}\,4151$ and $\mathrm{CGCG}\,420-015$ appears inconsistent with background expectations at $2.7\,\sigma$ significance, the stacking search constrains the model flux at the $\sim30\,\%$ level of the model prediction in the optimistic scenario. The extent to which the properties of NGC 1068 within the disc-corona model apply to the population of Seyfert galaxies therefore remains very much an open topic \cite{IceCube:2023northseyfert}. The work presented here focuses on a distinct set of sources and is thus bound to provide important additional information about potential neutrino production mechanisms in AGN. By extending the analysis presented here to also include IceCube cascade data \cite{Abbasi:2023bvn} in the Southern Sky, we intend to double the sensitivity of this search in future work.


% Bibtex references:
\bibliographystyle{ICRC}
\bibliography{references}

% Alternatively, you can include references by hand:
%\begin{thebibliography}{99}
%\bibitem{...}
%
%\end{thebibliography}

\clearpage

%The following list of authors, affiliations and funding agencies will be updated at the day of submission. The following template is a placeholder generated via https://authorlist.icecube.wisc.edu/icecube on March 25, 2023 and will be updated.
\section*{Full Author List: IceCube Collaboration}

\scriptsize
\noindent
R. Abbasi$^{17}$,
M. Ackermann$^{63}$,
J. Adams$^{18}$,
S. K. Agarwalla$^{40,\: 64}$,
J. A. Aguilar$^{12}$,
M. Ahlers$^{22}$,
J.M. Alameddine$^{23}$,
N. M. Amin$^{44}$,
K. Andeen$^{42}$,
G. Anton$^{26}$,
C. Arg{\"u}elles$^{14}$,
Y. Ashida$^{53}$,
S. Athanasiadou$^{63}$,
S. N. Axani$^{44}$,
X. Bai$^{50}$,
A. Balagopal V.$^{40}$,
M. Baricevic$^{40}$,
S. W. Barwick$^{30}$,
V. Basu$^{40}$,
R. Bay$^{8}$,
J. J. Beatty$^{20,\: 21}$,
J. Becker Tjus$^{11,\: 65}$,
J. Beise$^{61}$,
C. Bellenghi$^{27}$,
C. Benning$^{1}$,
S. BenZvi$^{52}$,
D. Berley$^{19}$,
E. Bernardini$^{48}$,
D. Z. Besson$^{36}$,
E. Blaufuss$^{19}$,
S. Blot$^{63}$,
F. Bontempo$^{31}$,
J. Y. Book$^{14}$,
C. Boscolo Meneguolo$^{48}$,
S. B{\"o}ser$^{41}$,
O. Botner$^{61}$,
J. B{\"o}ttcher$^{1}$,
E. Bourbeau$^{22}$,
J. Braun$^{40}$,
B. Brinson$^{6}$,
J. Brostean-Kaiser$^{63}$,
R. T. Burley$^{2}$,
R. S. Busse$^{43}$,
D. Butterfield$^{40}$,
M. A. Campana$^{49}$,
K. Carloni$^{14}$,
E. G. Carnie-Bronca$^{2}$,
S. Chattopadhyay$^{40,\: 64}$,
N. Chau$^{12}$,
C. Chen$^{6}$,
Z. Chen$^{55}$,
D. Chirkin$^{40}$,
S. Choi$^{56}$,
B. A. Clark$^{19}$,
L. Classen$^{43}$,
A. Coleman$^{61}$,
G. H. Collin$^{15}$,
A. Connolly$^{20,\: 21}$,
J. M. Conrad$^{15}$,
P. Coppin$^{13}$,
P. Correa$^{13}$,
D. F. Cowen$^{59,\: 60}$,
P. Dave$^{6}$,
C. De Clercq$^{13}$,
J. J. DeLaunay$^{58}$,
D. Delgado$^{14}$,
S. Deng$^{1}$,
K. Deoskar$^{54}$,
A. Desai$^{40}$,
P. Desiati$^{40}$,
K. D. de Vries$^{13}$,
G. de Wasseige$^{37}$,
T. DeYoung$^{24}$,
A. Diaz$^{15}$,
J. C. D{\'\i}az-V{\'e}lez$^{40}$,
M. Dittmer$^{43}$,
A. Domi$^{26}$,
H. Dujmovic$^{40}$,
M. A. DuVernois$^{40}$,
T. Ehrhardt$^{41}$,
P. Eller$^{27}$,
E. Ellinger$^{62}$,
S. El Mentawi$^{1}$,
D. Els{\"a}sser$^{23}$,
R. Engel$^{31,\: 32}$,
H. Erpenbeck$^{40}$,
J. Evans$^{19}$,
P. A. Evenson$^{44}$,
K. L. Fan$^{19}$,
K. Fang$^{40}$,
K. Farrag$^{16}$,
A. R. Fazely$^{7}$,
A. Fedynitch$^{57}$,
N. Feigl$^{10}$,
S. Fiedlschuster$^{26}$,
C. Finley$^{54}$,
L. Fischer$^{63}$,
D. Fox$^{59}$,
A. Franckowiak$^{11}$,
A. Fritz$^{41}$,
P. F{\"u}rst$^{1}$,
J. Gallagher$^{39}$,
E. Ganster$^{1}$,
A. Garcia$^{14}$,
L. Gerhardt$^{9}$,
A. Ghadimi$^{58}$,
C. Glaser$^{61}$,
T. Glauch$^{27}$,
T. Gl{\"u}senkamp$^{26,\: 61}$,
N. Goehlke$^{32}$,
J. G. Gonzalez$^{44}$,
S. Goswami$^{58}$,
D. Grant$^{24}$,
S. J. Gray$^{19}$,
O. Gries$^{1}$,
S. Griffin$^{40}$,
S. Griswold$^{52}$,
K. M. Groth$^{22}$,
C. G{\"u}nther$^{1}$,
P. Gutjahr$^{23}$,
C. Haack$^{26}$,
A. Hallgren$^{61}$,
R. Halliday$^{24}$,
L. Halve$^{1}$,
F. Halzen$^{40}$,
H. Hamdaoui$^{55}$,
M. Ha Minh$^{27}$,
K. Hanson$^{40}$,
J. Hardin$^{15}$,
A. A. Harnisch$^{24}$,
P. Hatch$^{33}$,
A. Haungs$^{31}$,
K. Helbing$^{62}$,
J. Hellrung$^{11}$,
F. Henningsen$^{27}$,
L. Heuermann$^{1}$,
N. Heyer$^{61}$,
S. Hickford$^{62}$,
A. Hidvegi$^{54}$,
C. Hill$^{16}$,
G. C. Hill$^{2}$,
K. D. Hoffman$^{19}$,
S. Hori$^{40}$,
K. Hoshina$^{40,\: 66}$,
W. Hou$^{31}$,
T. Huber$^{31}$,
K. Hultqvist$^{54}$,
M. H{\"u}nnefeld$^{23}$,
R. Hussain$^{40}$,
K. Hymon$^{23}$,
S. In$^{56}$,
A. Ishihara$^{16}$,
M. Jacquart$^{40}$,
O. Janik$^{1}$,
M. Jansson$^{54}$,
G. S. Japaridze$^{5}$,
M. Jeong$^{56}$,
M. Jin$^{14}$,
B. J. P. Jones$^{4}$,
D. Kang$^{31}$,
W. Kang$^{56}$,
X. Kang$^{49}$,
A. Kappes$^{43}$,
D. Kappesser$^{41}$,
L. Kardum$^{23}$,
T. Karg$^{63}$,
M. Karl$^{27}$,
A. Karle$^{40}$,
U. Katz$^{26}$,
M. Kauer$^{40}$,
J. L. Kelley$^{40}$,
A. Khatee Zathul$^{40}$,
A. Kheirandish$^{34,\: 35}$,
J. Kiryluk$^{55}$,
S. R. Klein$^{8,\: 9}$,
A. Kochocki$^{24}$,
R. Koirala$^{44}$,
H. Kolanoski$^{10}$,
T. Kontrimas$^{27}$,
L. K{\"o}pke$^{41}$,
C. Kopper$^{26}$,
D. J. Koskinen$^{22}$,
P. Koundal$^{31}$,
M. Kovacevich$^{49}$,
M. Kowalski$^{10,\: 63}$,
T. Kozynets$^{22}$,
J. Krishnamoorthi$^{40,\: 64}$,
K. Kruiswijk$^{37}$,
E. Krupczak$^{24}$,
A. Kumar$^{63}$,
E. Kun$^{11}$,
N. Kurahashi$^{49}$,
N. Lad$^{63}$,
C. Lagunas Gualda$^{63}$,
M. Lamoureux$^{37}$,
M. J. Larson$^{19}$,
S. Latseva$^{1}$,
F. Lauber$^{62}$,
J. P. Lazar$^{14,\: 40}$,
J. W. Lee$^{56}$,
K. Leonard DeHolton$^{60}$,
A. Leszczy{\'n}ska$^{44}$,
M. Lincetto$^{11}$,
Q. R. Liu$^{40}$,
M. Liubarska$^{25}$,
E. Lohfink$^{41}$,
C. Love$^{49}$,
C. J. Lozano Mariscal$^{43}$,
L. Lu$^{40}$,
F. Lucarelli$^{28}$,
W. Luszczak$^{20,\: 21}$,
Y. Lyu$^{8,\: 9}$,
J. Madsen$^{40}$,
K. B. M. Mahn$^{24}$,
Y. Makino$^{40}$,
E. Manao$^{27}$,
S. Mancina$^{40,\: 48}$,
W. Marie Sainte$^{40}$,
I. C. Mari{\c{s}}$^{12}$,
S. Marka$^{46}$,
Z. Marka$^{46}$,
M. Marsee$^{58}$,
I. Martinez-Soler$^{14}$,
R. Maruyama$^{45}$,
F. Mayhew$^{24}$,
T. McElroy$^{25}$,
F. McNally$^{38}$,
J. V. Mead$^{22}$,
K. Meagher$^{40}$,
S. Mechbal$^{63}$,
A. Medina$^{21}$,
M. Meier$^{16}$,
Y. Merckx$^{13}$,
L. Merten$^{11}$,
J. Micallef$^{24}$,
J. Mitchell$^{7}$,
T. Montaruli$^{28}$,
R. W. Moore$^{25}$,
Y. Morii$^{16}$,
R. Morse$^{40}$,
M. Moulai$^{40}$,
T. Mukherjee$^{31}$,
R. Naab$^{63}$,
R. Nagai$^{16}$,
M. Nakos$^{40}$,
U. Naumann$^{62}$,
J. Necker$^{63}$,
A. Negi$^{4}$,
M. Neumann$^{43}$,
H. Niederhausen$^{24}$,
M. U. Nisa$^{24}$,
A. Noell$^{1}$,
A. Novikov$^{44}$,
S. C. Nowicki$^{24}$,
A. Obertacke Pollmann$^{16}$,
V. O'Dell$^{40}$,
M. Oehler$^{31}$,
B. Oeyen$^{29}$,
A. Olivas$^{19}$,
R. {\O}rs{\o}e$^{27}$,
J. Osborn$^{40}$,
E. O'Sullivan$^{61}$,
H. Pandya$^{44}$,
N. Park$^{33}$,
G. K. Parker$^{4}$,
E. N. Paudel$^{44}$,
L. Paul$^{42,\: 50}$,
C. P{\'e}rez de los Heros$^{61}$,
J. Peterson$^{40}$,
S. Philippen$^{1}$,
A. Pizzuto$^{40}$,
M. Plum$^{50}$,
A. Pont{\'e}n$^{61}$,
Y. Popovych$^{41}$,
M. Prado Rodriguez$^{40}$,
B. Pries$^{24}$,
R. Procter-Murphy$^{19}$,
G. T. Przybylski$^{9}$,
C. Raab$^{37}$,
J. Rack-Helleis$^{41}$,
K. Rawlins$^{3}$,
Z. Rechav$^{40}$,
A. Rehman$^{44}$,
P. Reichherzer$^{11}$,
G. Renzi$^{12}$,
E. Resconi$^{27}$,
S. Reusch$^{63}$,
W. Rhode$^{23}$,
B. Riedel$^{40}$,
A. Rifaie$^{1}$,
E. J. Roberts$^{2}$,
S. Robertson$^{8,\: 9}$,
S. Rodan$^{56}$,
G. Roellinghoff$^{56}$,
M. Rongen$^{26}$,
C. Rott$^{53,\: 56}$,
T. Ruhe$^{23}$,
L. Ruohan$^{27}$,
D. Ryckbosch$^{29}$,
I. Safa$^{14,\: 40}$,
J. Saffer$^{32}$,
D. Salazar-Gallegos$^{24}$,
P. Sampathkumar$^{31}$,
S. E. Sanchez Herrera$^{24}$,
A. Sandrock$^{62}$,
M. Santander$^{58}$,
S. Sarkar$^{25}$,
S. Sarkar$^{47}$,
J. Savelberg$^{1}$,
P. Savina$^{40}$,
M. Schaufel$^{1}$,
H. Schieler$^{31}$,
S. Schindler$^{26}$,
L. Schlickmann$^{1}$,
B. Schl{\"u}ter$^{43}$,
F. Schl{\"u}ter$^{12}$,
N. Schmeisser$^{62}$,
T. Schmidt$^{19}$,
J. Schneider$^{26}$,
F. G. Schr{\"o}der$^{31,\: 44}$,
L. Schumacher$^{26}$,
G. Schwefer$^{1}$,
S. Sclafani$^{19}$,
D. Seckel$^{44}$,
M. Seikh$^{36}$,
S. Seunarine$^{51}$,
R. Shah$^{49}$,
A. Sharma$^{61}$,
S. Shefali$^{32}$,
N. Shimizu$^{16}$,
M. Silva$^{40}$,
B. Skrzypek$^{14}$,
B. Smithers$^{4}$,
R. Snihur$^{40}$,
J. Soedingrekso$^{23}$,
A. S{\o}gaard$^{22}$,
D. Soldin$^{32}$,
P. Soldin$^{1}$,
G. Sommani$^{11}$,
C. Spannfellner$^{27}$,
G. M. Spiczak$^{51}$,
C. Spiering$^{63}$,
M. Stamatikos$^{21}$,
T. Stanev$^{44}$,
T. Stezelberger$^{9}$,
T. St{\"u}rwald$^{62}$,
T. Stuttard$^{22}$,
G. W. Sullivan$^{19}$,
I. Taboada$^{6}$,
S. Ter-Antonyan$^{7}$,
M. Thiesmeyer$^{1}$,
W. G. Thompson$^{14}$,
J. Thwaites$^{40}$,
S. Tilav$^{44}$,
K. Tollefson$^{24}$,
C. T{\"o}nnis$^{56}$,
S. Toscano$^{12}$,
D. Tosi$^{40}$,
A. Trettin$^{63}$,
C. F. Tung$^{6}$,
R. Turcotte$^{31}$,
J. P. Twagirayezu$^{24}$,
B. Ty$^{40}$,
M. A. Unland Elorrieta$^{43}$,
A. K. Upadhyay$^{40,\: 64}$,
K. Upshaw$^{7}$,
N. Valtonen-Mattila$^{61}$,
J. Vandenbroucke$^{40}$,
N. van Eijndhoven$^{13}$,
D. Vannerom$^{15}$,
J. van Santen$^{63}$,
J. Vara$^{43}$,
J. Veitch-Michaelis$^{40}$,
M. Venugopal$^{31}$,
M. Vereecken$^{37}$,
S. Verpoest$^{44}$,
D. Veske$^{46}$,
A. Vijai$^{19}$,
C. Walck$^{54}$,
C. Weaver$^{24}$,
P. Weigel$^{15}$,
A. Weindl$^{31}$,
J. Weldert$^{60}$,
C. Wendt$^{40}$,
J. Werthebach$^{23}$,
M. Weyrauch$^{31}$,
N. Whitehorn$^{24}$,
C. H. Wiebusch$^{1}$,
N. Willey$^{24}$,
D. R. Williams$^{58}$,
L. Witthaus$^{23}$,
A. Wolf$^{1}$,
M. Wolf$^{27}$,
G. Wrede$^{26}$,
X. W. Xu$^{7}$,
J. P. Yanez$^{25}$,
E. Yildizci$^{40}$,
S. Yoshida$^{16}$,
R. Young$^{36}$,
F. Yu$^{14}$,
S. Yu$^{24}$,
T. Yuan$^{40}$,
Z. Zhang$^{55}$,
P. Zhelnin$^{14}$,
M. Zimmerman$^{40}$\\
\\
$^{1}$ III. Physikalisches Institut, RWTH Aachen University, D-52056 Aachen, Germany \\
$^{2}$ Department of Physics, University of Adelaide, Adelaide, 5005, Australia \\
$^{3}$ Dept. of Physics and Astronomy, University of Alaska Anchorage, 3211 Providence Dr., Anchorage, AK 99508, USA \\
$^{4}$ Dept. of Physics, University of Texas at Arlington, 502 Yates St., Science Hall Rm 108, Box 19059, Arlington, TX 76019, USA \\
$^{5}$ CTSPS, Clark-Atlanta University, Atlanta, GA 30314, USA \\
$^{6}$ School of Physics and Center for Relativistic Astrophysics, Georgia Institute of Technology, Atlanta, GA 30332, USA \\
$^{7}$ Dept. of Physics, Southern University, Baton Rouge, LA 70813, USA \\
$^{8}$ Dept. of Physics, University of California, Berkeley, CA 94720, USA \\
$^{9}$ Lawrence Berkeley National Laboratory, Berkeley, CA 94720, USA \\
$^{10}$ Institut f{\"u}r Physik, Humboldt-Universit{\"a}t zu Berlin, D-12489 Berlin, Germany \\
$^{11}$ Fakult{\"a}t f{\"u}r Physik {\&} Astronomie, Ruhr-Universit{\"a}t Bochum, D-44780 Bochum, Germany \\
$^{12}$ Universit{\'e} Libre de Bruxelles, Science Faculty CP230, B-1050 Brussels, Belgium \\
$^{13}$ Vrije Universiteit Brussel (VUB), Dienst ELEM, B-1050 Brussels, Belgium \\
$^{14}$ Department of Physics and Laboratory for Particle Physics and Cosmology, Harvard University, Cambridge, MA 02138, USA \\
$^{15}$ Dept. of Physics, Massachusetts Institute of Technology, Cambridge, MA 02139, USA \\
$^{16}$ Dept. of Physics and The International Center for Hadron Astrophysics, Chiba University, Chiba 263-8522, Japan \\
$^{17}$ Department of Physics, Loyola University Chicago, Chicago, IL 60660, USA \\
$^{18}$ Dept. of Physics and Astronomy, University of Canterbury, Private Bag 4800, Christchurch, New Zealand \\
$^{19}$ Dept. of Physics, University of Maryland, College Park, MD 20742, USA \\
$^{20}$ Dept. of Astronomy, Ohio State University, Columbus, OH 43210, USA \\
$^{21}$ Dept. of Physics and Center for Cosmology and Astro-Particle Physics, Ohio State University, Columbus, OH 43210, USA \\
$^{22}$ Niels Bohr Institute, University of Copenhagen, DK-2100 Copenhagen, Denmark \\
$^{23}$ Dept. of Physics, TU Dortmund University, D-44221 Dortmund, Germany \\
$^{24}$ Dept. of Physics and Astronomy, Michigan State University, East Lansing, MI 48824, USA \\
$^{25}$ Dept. of Physics, University of Alberta, Edmonton, Alberta, Canada T6G 2E1 \\
$^{26}$ Erlangen Centre for Astroparticle Physics, Friedrich-Alexander-Universit{\"a}t Erlangen-N{\"u}rnberg, D-91058 Erlangen, Germany \\
$^{27}$ Technical University of Munich, TUM School of Natural Sciences, Department of Physics, D-85748 Garching bei M{\"u}nchen, Germany \\
$^{28}$ D{\'e}partement de physique nucl{\'e}aire et corpusculaire, Universit{\'e} de Gen{\`e}ve, CH-1211 Gen{\`e}ve, Switzerland \\
$^{29}$ Dept. of Physics and Astronomy, University of Gent, B-9000 Gent, Belgium \\
$^{30}$ Dept. of Physics and Astronomy, University of California, Irvine, CA 92697, USA \\
$^{31}$ Karlsruhe Institute of Technology, Institute for Astroparticle Physics, D-76021 Karlsruhe, Germany  \\
$^{32}$ Karlsruhe Institute of Technology, Institute of Experimental Particle Physics, D-76021 Karlsruhe, Germany  \\
$^{33}$ Dept. of Physics, Engineering Physics, and Astronomy, Queen's University, Kingston, ON K7L 3N6, Canada \\
$^{34}$ Department of Physics {\&} Astronomy, University of Nevada, Las Vegas, NV, 89154, USA \\
$^{35}$ Nevada Center for Astrophysics, University of Nevada, Las Vegas, NV 89154, USA \\
$^{36}$ Dept. of Physics and Astronomy, University of Kansas, Lawrence, KS 66045, USA \\
$^{37}$ Centre for Cosmology, Particle Physics and Phenomenology - CP3, Universit{\'e} catholique de Louvain, Louvain-la-Neuve, Belgium \\
$^{38}$ Department of Physics, Mercer University, Macon, GA 31207-0001, USA \\
$^{39}$ Dept. of Astronomy, University of Wisconsin{\textendash}Madison, Madison, WI 53706, USA \\
$^{40}$ Dept. of Physics and Wisconsin IceCube Particle Astrophysics Center, University of Wisconsin{\textendash}Madison, Madison, WI 53706, USA \\
$^{41}$ Institute of Physics, University of Mainz, Staudinger Weg 7, D-55099 Mainz, Germany \\
$^{42}$ Department of Physics, Marquette University, Milwaukee, WI, 53201, USA \\
$^{43}$ Institut f{\"u}r Kernphysik, Westf{\"a}lische Wilhelms-Universit{\"a}t M{\"u}nster, D-48149 M{\"u}nster, Germany \\
$^{44}$ Bartol Research Institute and Dept. of Physics and Astronomy, University of Delaware, Newark, DE 19716, USA \\
$^{45}$ Dept. of Physics, Yale University, New Haven, CT 06520, USA \\
$^{46}$ Columbia Astrophysics and Nevis Laboratories, Columbia University, New York, NY 10027, USA \\
$^{47}$ Dept. of Physics, University of Oxford, Parks Road, Oxford OX1 3PU, United Kingdom\\
$^{48}$ Dipartimento di Fisica e Astronomia Galileo Galilei, Universit{\`a} Degli Studi di Padova, 35122 Padova PD, Italy \\
$^{49}$ Dept. of Physics, Drexel University, 3141 Chestnut Street, Philadelphia, PA 19104, USA \\
$^{50}$ Physics Department, South Dakota School of Mines and Technology, Rapid City, SD 57701, USA \\
$^{51}$ Dept. of Physics, University of Wisconsin, River Falls, WI 54022, USA \\
$^{52}$ Dept. of Physics and Astronomy, University of Rochester, Rochester, NY 14627, USA \\
$^{53}$ Department of Physics and Astronomy, University of Utah, Salt Lake City, UT 84112, USA \\
$^{54}$ Oskar Klein Centre and Dept. of Physics, Stockholm University, SE-10691 Stockholm, Sweden \\
$^{55}$ Dept. of Physics and Astronomy, Stony Brook University, Stony Brook, NY 11794-3800, USA \\
$^{56}$ Dept. of Physics, Sungkyunkwan University, Suwon 16419, Korea \\
$^{57}$ Institute of Physics, Academia Sinica, Taipei, 11529, Taiwan \\
$^{58}$ Dept. of Physics and Astronomy, University of Alabama, Tuscaloosa, AL 35487, USA \\
$^{59}$ Dept. of Astronomy and Astrophysics, Pennsylvania State University, University Park, PA 16802, USA \\
$^{60}$ Dept. of Physics, Pennsylvania State University, University Park, PA 16802, USA \\
$^{61}$ Dept. of Physics and Astronomy, Uppsala University, Box 516, S-75120 Uppsala, Sweden \\
$^{62}$ Dept. of Physics, University of Wuppertal, D-42119 Wuppertal, Germany \\
$^{63}$ Deutsches Elektronen-Synchrotron DESY, Platanenallee 6, 15738 Zeuthen, Germany  \\
$^{64}$ Institute of Physics, Sachivalaya Marg, Sainik School Post, Bhubaneswar 751005, India \\
$^{65}$ Department of Space, Earth and Environment, Chalmers University of Technology, 412 96 Gothenburg, Sweden \\
$^{66}$ Earthquake Research Institute, University of Tokyo, Bunkyo, Tokyo 113-0032, Japan \\

\subsection*{Acknowledgements}

\noindent
The authors gratefully acknowledge the support from the following agencies and institutions:
USA {\textendash} U.S. National Science Foundation-Office of Polar Programs,
U.S. National Science Foundation-Physics Division,
U.S. National Science Foundation-EPSCoR,
Wisconsin Alumni Research Foundation,
Center for High Throughput Computing (CHTC) at the University of Wisconsin{\textendash}Madison,
Open Science Grid (OSG),
Advanced Cyberinfrastructure Coordination Ecosystem: Services {\&} Support (ACCESS),
Frontera computing project at the Texas Advanced Computing Center,
U.S. Department of Energy-National Energy Research Scientific Computing Center,
Particle astrophysics research computing center at the University of Maryland,
Institute for Cyber-Enabled Research at Michigan State University,
and Astroparticle physics computational facility at Marquette University;
Belgium {\textendash} Funds for Scientific Research (FRS-FNRS and FWO),
FWO Odysseus and Big Science programmes,
and Belgian Federal Science Policy Office (Belspo);
Germany {\textendash} Bundesministerium f{\"u}r Bildung und Forschung (BMBF),
Deutsche Forschungsgemeinschaft (DFG),
Helmholtz Alliance for Astroparticle Physics (HAP),
Initiative and Networking Fund of the Helmholtz Association,
Deutsches Elektronen Synchrotron (DESY),
and High Performance Computing cluster of the RWTH Aachen;
Sweden {\textendash} Swedish Research Council,
Swedish Polar Research Secretariat,
Swedish National Infrastructure for Computing (SNIC),
and Knut and Alice Wallenberg Foundation;
European Union {\textendash} EGI Advanced Computing for research;
Australia {\textendash} Australian Research Council;
Canada {\textendash} Natural Sciences and Engineering Research Council of Canada,
Calcul Qu{\'e}bec, Compute Ontario, Canada Foundation for Innovation, WestGrid, and Compute Canada;
Denmark {\textendash} Villum Fonden, Carlsberg Foundation, and European Commission;
New Zealand {\textendash} Marsden Fund;
Japan {\textendash} Japan Society for Promotion of Science (JSPS)
and Institute for Global Prominent Research (IGPR) of Chiba University;
Korea {\textendash} National Research Foundation of Korea (NRF);
Switzerland {\textendash} Swiss National Science Foundation (SNSF);
United Kingdom {\textendash} Department of Physics, University of Oxford.

\end{document}
