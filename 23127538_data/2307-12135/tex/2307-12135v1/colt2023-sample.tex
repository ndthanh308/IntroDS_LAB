\documentclass[final]{colt2023}

\title[The Sample Complexity of Multi-Distribution Learning]{Open Problem: The Sample Complexity of Multi-Distribution Learning for VC Classes}
\usepackage{times}
\usepackage[numbers]{natbib}
\usepackage{mathtools}
\usepackage{thm-restate}
\usepackage{algorithm}
\usepackage{algorithmic}
\usepackage[suppress]{color-edits}
\usepackage{booktabs}
\usepackage{multirow}
\addauthor{ez}{blue}
\addauthor{nh}{purple}

\newcommand\calF{\mathcal{F}}
\newcommand\calG{\mathcal{G}}
\newcommand\calM{\mathcal{M}}
\newcommand\calV{\mathcal{V}}
\newcommand\calU{\mathcal{U}}
\newcommand\calW{\mathcal{W}}
\newcommand\calP{\mathcal{P}}
\newcommand\calD{\mathbb{D}}
%%%%%%%%%%%%%%%%%
%% macros introduced by Luke 
\newcommand\mydef[1]{{\bf\em #1}}
%%%%%%%%%%%%%%%%%

\newcommand{\numviparams}{{| \lambda |}}
\newcommand{\scoreaccvars}[1]{s_1^{#1}, \ldots, s_{\numviparams}^{#1}}
\newcommand{\scoreaccvar}[2]{s_{#1}^{#2}}
\newcommand{\isdeterm}[1]{\text{Deterministic}({#1})}


\newcommand{\expect}[1]{\mathbb{E}\left[{#1}\right]}
\newcommand{\var}[1]{\mathbb{V}\left[ {#1} \right]}
\newcommand{\expectdist}[2]{\mathbb{E}_{#1}\left[ {#2} \right]}
\newcommand{\vardist}[2]{\mathbb{V}_{#1}\left[ {#2} \right]}
\newcommand{\cov}[2]{\mathbb{C}\text{ov}[{#1}][{#2}]}
\newcommand{\covv}[1]{\mathbb{C}\text{ov}[{#1}]}
\newcommand{\corr}[1]{\mathbb{C}\text{orr}[{#1}]}

\newcommand{\fix}[1]{\mathit{fix}\left({#1}\right)}
\newcommand{\sbr}[1]{\left\llbracket {#1} \right\rrbracket}
\newcommand{\ctxtype}[3]{{#1} \cong_\text{ctx} {#2} : {#3}}
\newcommand{\bigstep}[3]{{#1} \Downarrow_{#2} {#3}}


% PCF types
\newcommand{\bool}{\mathit{bool}}
\newcommand{\nat}{\mathit{nat}}

\newcommand{\ctx}[1]{\mathcal{C}\left[ {#1}\right] }
\newcommand{\pcft}[1]{\text{PCF}_{#1}}

\newcommand{\nfl}{\mathbb{N}_\bot}
\newcommand{\bfl}{\mathbb{B}_\bot}

% PCF constructs
\newcommand{\succc}[1]{\mathbf{succ}({#1})}
\newcommand{\succcn}[2]{\mathbf{succ}^{#1}({#2})}
\newcommand{\zero}{\mathbf{0}}
\newcommand{\zerotest}[1]{\mathbf{zero}\left({#1}\right)}
\newcommand{\pred}[1]{\mathbf{pred}\left( {#1} \right)}
\newcommand{\predn}[2]{\mathbf{pred}^{#1}\left( {#2} \right)}
\def\solvable{\#}

\newcommand{\true}{\mathbf{true}}
\newcommand{\false}{\mathbf{false}}
\newcommand{\pcffix}[1]{\mathbf{fix}\left({#1}\right)}
\newcommand{\pcffn}[3]{\mathbf{fn}~{#1}:{#2}\mathpunct{.}{#3}}
\newcommand{\pairtype}[2]{{#1} * {#2}}
\newcommand{\pairexp}[2]{\mathbf{pair}({#1}, {#2})}
\newcommand{\leftexp}[1]{\mathbf{left}({#1})}
\newcommand{\rightexp}[1]{\mathbf{right}({#1})}

\newcommand{\RationalPos}{\mathbb{Q}^{+}}

\newcommand{\meas}[1]{\mathbb{M}\left( {#1} \right) }
\newcommand{\integ}[1]{\sbr{#1}_I}

\newcommand{\notbigstep}[2]{{#1}~\cancel{\Downarrow}_{#2}}
\newcommand{\subtrace}[3]{{#1}^{{#2} \ldots {#3}}}
\newcommand{\supp}[1]{\textsf{supp}\left({#1}\right)}
\newcommand{\dom}[1]{\textsf{Dom}\left({#1}\right)}
\newcommand{\suppk}[2]{\textsf{Supp}^{#1}\left({#2}\right)}
\newcommand{\tracespace}{\bigcup_{n \in \mathbb{N}}[0, 1]^n}
\newcommand{\generictracespace}{\mathbb{T}}
\newcommand{\nnreals}{\mathbb{R}_{\geq 0}}
\newcommand{\posreals}{\mathbb{R}_{> 0}}
\newcommand{\reals}{\mathbb{R}}

\newcommand{\unrollkM}[2]{\textsf{unroll}_{#1}\left({#2}\right)}
\newcommand{\nphmcint}[5]{\Psi_\textsf{NP}\left({#1}, {#2}, {#3}, {#4}, {#5}\right)}

%SPCF constructs
\newcommand{\spcfvalues}{\Lambda^0_v}

\newcommand{\prevalueM}[1]{\textsf{value}^{-1}_{#1}(\spcfvalues{})}
\newcommand{\num}[1]{\underline{#1}}

% \theoremstyle{definition}
% \newtheorem{thm}{Theorem}
% \newtheorem{lem}{Lemma}
% \newtheorem{defn}{Definition}
% \newtheorem{conj}{Conjecture}
% \newtheorem{prop}{Proposition}

%\theoremstyle{definition}
%\newtheorem{defn}{Definition}[section]
%\newtheorem{example}[defn]{Example}
%
%
%\theoremstyle{plain}
%\newtheorem{thm}{Theorem}[section]
%\newtheorem{lem}[thm]{Lemma}
%\newtheorem{cor}[thm]{Corollary}
%\newtheorem{conj}[thm]{Conjecture}
%\newtheorem{prop}[thm]{Proposition}
%\newtheorem{remark}[thm]{Remark}

%% Proofs
%\let\oldproof\proof
%\renewcommand{\proof}{\color{blue}\oldproof}


\definecolor{codegreen}{rgb}{0,0.6,0}
\definecolor{codegray}{rgb}{0.5,0.5,0.5}
\definecolor{codepurple}{rgb}{0.58,0,0.82}
\definecolor{backcolour}{rgb}{0.95,0.95,0.92}

\lstdefinestyle{myStyle}{
    belowcaptionskip=1\baselineskip,
    breaklines=true,
    frame=none,
    basicstyle=\footnotesize\ttfamily,
    keywordstyle=\bfseries\color{green!40!black},
    commentstyle=\itshape\color{purple!40!black},
    identifierstyle=\color{blue},
    backgroundcolor=\color{gray!10!white},
    %backgroundcolor=\color{backcolour}, 
    numberstyle=\tiny\color{codegray},
    stringstyle=\color{codepurple},
    breakatwhitespace=false,                          
    keepspaces=true,                 
    numbers=left,       
    numbersep=5pt,                  
    showspaces=false,                
    showstringspaces=false,
    showtabs=false,                  
    tabsize=2,
}

% argmin/argmax
\DeclareMathOperator*{\argmax}{arg\,max}
\DeclareMathOperator*{\argmin}{arg\,min}

% Concatenation of lists
\newcommand\doubleplus{+\kern-1.3ex+\kern0.8ex}

% Program configurations
\newcommand{\tuple}[1]{\ensuremath{\langle #1 \rangle}}
% Rule based definitions
\newcommand{\Rule}[4][]{\ensuremath{\inferrule*[lab={\hypertarget{#2}{(\TirName{#2})}},#1]{#3}{#4}}}

% Calligraphic symbols
\newcommand{\calI}{{\mathcal I}} 
\newcommand{\calT}{{\mathcal T}}

%  Macro for new Y operator.
\newcommand{\yBounded}[3]{\mu^{#1}_{#2}\rvert_{#3}}

%%%%%%%%%%%%%%%%%
 
%%%%%%%%%%%%%%%%%

\newcommand{\expv}{\mathbb{E}}

\newcommand{\combTr}[2]{\left[\begin{matrix}
		#1\\
		#2
	\end{matrix} \right]}

\newcommand{\exType}[2]{\left\{\begin{matrix}
		#1\\
		#2
	\end{matrix} \right\}}
\newcommand{\myint}[1]{ [#1]}
\newcommand{\Uniform}{\ensuremath{\mathrm{Uniform}}}
\newcommand{\Normal}{\ensuremath{\mathrm{normal}}}
\DeclareMathOperator{\abs}{abs}
\DeclareMathOperator{\pdf}{pdf}

\newcommand{\intConf}[1]{\lceil#1\rceil}
\newcommand{\tr}{\boldsymbol{t}}

\newcommand{\sample}{\tt{sample}}
%\newcommand{\fix}{\texttt{fix}}
%\newcommand{\num}[1]{\underline{#1}}
\newcommand{\myif}{\texttt{if}}
\newcommand{\mylet}{\texttt{let} \, }
\newcommand{\myin}{\, \texttt{in} \,}
\newcommand{\mythen}{\, \texttt{then} \,}
\newcommand{\myelse}{\, \texttt{else} \,}
\newcommand{\score}{\tt{score}}
\newcommand{\tick}{\tt{tick}}

\newcommand{\term}{\tt{term}}
\newcommand{\pv}{\mathbf{v}}
\newcommand{\rv}{\mathbf{r}}

\newcommand{\interval}{\mathfrak{I}}

\newcommand{\typeReal}{\textbf{\textsf{R}}}

\newcommand{\symbolInt}{\myint{\cdot}}

\newcommand{\LambdaInterval}{\Lambda_{\interval}}
\newcommand{\LambdaSymbolic}{\Lambda_{\text{sym}}}

\newcommand{\toIntervalTerm}[1]{#1^{2\interval}}

%Others
\newcommand{\Sset}{\mathbb{S}}
\newcommand{\Iset}{\mathbb{I}}
\newcommand{\Rset}{\mathbb{R}}
\newcommand{\Nset}{\mathbb{N}}
\newcommand{\Zset}{\mathbb{Z}}

\newcommand{\Term}{\mathbb{T}}
\newcommand{\prob}{\mathbb{P}}
\newcommand{\expt}{\mathbb{E}}


\newcommand{\Leb}{\tt{Leb}}
\newcommand{\Red}{\tt{Red}}
\newcommand{\cost}{\text{cost}}

%\newcommand{\intervalab}[2]{\underline{[#1,#2]}}
\newcommand{\intervalab}{\underline{[a,b]}}
\newcommand{\interI}{\mathcal{I}}
\newcommand{\trans}{\mathcal{T}}

\newcommand{\iv}{\mathbb{I}}

% Programming language constructs
\newcommand{\lit}[1]{\underline{#1}}
\newcommand{\letIn}[1]{\mathsf{let}\,{#1}\,\mathsf{in}\,}
\newcommand{\fixLam}[2]{\mu {#1} {#2}.}
\newcommand{\ifElse}[3]{\mathsf{if} (#1 \le \num{0}) \, {#2} \,\mathsf{else}\, {#3}}

%%Basic notions
\newcommand{\pspace}{(\Omega,\mathcal{F},\probm)}
\newcommand{\probm}{\mathbb{P}}
\newcommand{\condexpv}[2]{{\expt}{\left[{#1} \mid {#2}\right]}}

\newcommand{\stdConf}[1]{(#1)}
%\newcommand{\intConf}[1]{\lceil#1\rceil}
%\newcommand{\intConf}[1]{(#1)}
%\newcommand{\symConf}[1]{\langle\!\langle  #1 \rangle\!\rangle}
%\newcommand\symPath[1]{(#1)}
\newcommand{\symPath}[1]{\langle\!\langle  #1 \rangle\!\rangle}
\newcommand\symConf[1]{(#1)}

\newcommand{\ifSimple}[3]{\mathsf{if}(#1, #2, #3)}
%\newcommand{\ifElse}[3]{\mathsf{if} (#1 \le 0) \, \allowbreak {#2} \, \allowbreak \mathsf{else}\, {#3}}
%\newcommand{\ifElse}[3]{\ifSimple{#1}{#2}{#3}}

%\newcommand{\trace}{\mathsf{s}}
%
%\newcommand\defn[1]{{\bf \em #1}}
\newcommand{\traces}{\mathbb{T}}
%
%\newcommand{\stdConf}[1]{(#1)}
%%\newcommand{\intConf}[1]{\lceil#1\rceil}
%\newcommand{\intConf}[1]{(#1)}
%%\newcommand{\symConf}[1]{\langle\!\langle  #1 \rangle\!\rangle}
%%\newcommand\symPath[1]{(#1)}
%\newcommand{\symPath}[1]{\langle\!\langle  #1 \rangle\!\rangle}
%\newcommand\symConf[1]{(#1)}

\newcommand{\valueSem}[1]{\mathsf{val}_{#1}} % value (semantics)
\newcommand{\weightSem}[1]{\mathsf{wt}_{#1}} % weight (semantics)
\newcommand{\measureSem}[1]{\llbracket #1 \rrbracket}
\newcommand{\posterior}{\mathsf{posterior}}


%%%%%%%%%
% 
%%%%%%%%
\newcommand{\loc}{\ell}
\newcommand{\locs}{\mathit{L}}
\newcommand{\blocs}{\mathit{L}_{\mathrm{b}}}

\newcommand{\iflocs}{\mathit{L}_{\mathrm{if}}}
\newcommand{\looplocs}{\mathit{L}_{\mathrm{while}}}

\newcommand{\alocs}{\mathit{L}_{\mathrm{a}}}
\newcommand{\wlocs}{\mathit{L}_{\mathrm{w}}}
\newcommand{\rlocs}{\mathit{L}_{\mathrm{r}}}
\newcommand{\Alocs}[1]{\mathit{L}_{\mathrm{A}}^{\mathsf{#1}}}
\newcommand{\Dlocs}{\mathit{L}_{\mathrm{nd}}}
\newcommand{\transitions}{{\rightarrow}}

%%% 
\newcommand{\plocs}{\mathit{L}_{\mathrm{p}}}
\newcommand{\tlocs}{\mathit{L}_{\mathrm{t}}}

\newcommand{\lin}{\loc_\mathrm{init}}
\newcommand{\lout}{\loc_\mathrm{out}}
\newcommand{\val}[1]{\mbox{\sl Val}_{#1}}

\newcommand{\pvars}{V_\mathrm{p}}
\newcommand{\rvars}{V_{\mathrm{r}}}
\newcommand{\pre}{\mathrm{pre}}

\newcommand{\sle}{\sqsubseteq}
\newcommand{\sge}{\sqsupseteq}

\newcommand{\lfp}{\mathrm{lfp}}
\newcommand{\gfp}{\mathrm{gfp}}

\newcommand{\rdvarjdis}{\mathcal D}
\newcommand{\sampset}{\textit{supp}}

\newcommand{\upd}{\mbox{\sl upd}}
\newcommand{\wet}{\mbox{\sl wt}}
\newcommand{\transset}{\mathfrak T}
\newcommand{\valin}{\pv_{\mathrm{init}}}
\newcommand{\ret}{\mbox{\sl ret}}

\newcommand{\win}{w_{\mathrm{init}}}

\newcommand{\sampdpd}{\overline{\Upsilon}}

\newcommand{\outmap}{\text{O}}
\newcommand{\sat}[1]{\langle #1 \rangle}
\newcommand{\monoid}{\mbox{\sl Monoid}}
\newcommand{\handelmanformat}{(\dagger)}

\newcommand{\trunc}{\mathcal{B}}

\newcommand{\ewt}{\mbox{\sl ewt}}
\newcommand{\statemap}{\text{St}}

\newcommand{\valrd}{{\mathbf{r}}}
\newcommand{\frmloc}{\ell^{\mathrm{src}}}
\newcommand{\toloc}{\ell^{\mathrm{dst}}}

\newcommand{\monomials}{\mathbf{M}}
\coltauthor{%
 \Name{Pranjal Awasthi} \Email{pranjalawasthi@google.com}\\
 \addr Google Research, Mountain View, CA, USA
 \AND
 \Name{Nika Haghtalab} \Email{nika@berkeley.edu}\\
 \addr University of California, Berkeley, CA, USA
 \AND
 \Name{Eric Zhao} \Email{eric.zh@berkeley.edu}\\
 \addr University of California, Berkeley, CA, USA
}

\begin{document}

\maketitle

\begin{abstract}
	Multi-distribution learning is a natural generalization of PAC learning to settings with multiple data distributions.
	There remains a significant gap between the known upper and lower bounds for PAC-learnable classes.
	In particular, though we understand the sample complexity of learning a VC dimension $d$ class on $k$ distributions to be $O(\epsilon^{-2} \ln(k) (d + k) + \min \bset{\epsilon^{-1} d k,  \epsilon^{-4} \ln(k) d })$, the best lower bound is ${\Omega}(\epsilon^{-2}(d + k \ln(k)))$.
	We discuss recent progress on this problem and some hurdles that are fundamental to the use of game dynamics in statistical learning. 
	
\end{abstract}

\begin{keywords}
	PAC learning, multi-distribution learning, distributional robustness, learning in games.
\end{keywords}

\section{Introduction}
The pervasive need for robustness, fairness, and multi-agent welfare in  learning processes has led to the development of learning paradigms whose performance hold under multiple distributions and scenarios.
\emph{Multi-distribution learning}, or MDL, is a setting introduced by~\cite{haghtalabOnDemandSamplingLearning2022} to address these needs and unify several existing frameworks and applications, such as notions of \emph{min-max} fairness \cite{mohri_agnostic_2019,Abernethy2022}, \emph{group distributionally robust} optimization \cite{sagawa_distributionally_2020}, and collaborative learning \cite{blumCollaborativePACLearning2017}.
MDL is a generalization of the agnostic learning paradigms~\citep{valiant_theory_1984,blumer1989learnability} to multiple data distributions. In this setting, given a set of distributions $\dists = \bset{\dist_1, \dots, \dist_k}$ supported on $\features \times \labels$, loss function $\loss$, and a hypothesis class $\hyps$, %
the goal of MDL is to find a (possibly randomized) hypothesis $\hyp$ where
\begin{align}
	\label{eq:optimal}
	\smash{\max_{\dist \in \dists} \risk_{\dist}(\hyp) \leq \epsilon + \min_{\hyp^* \in \hyps} \max_{\dist \in \dists} \risk_{\dist}(\hyp^*),\; \text{where}\; \risk_\dist(\hyp) \asseq \EEs{(x,y) \sim \dist}{\loss(\hyp, (x, y))}.}
\end{align}
Such an $\hyp$ is called an \emph{$\epsilon$-optimal solution} to the MDL problem $(\dists, \hyps)$ and we denote
$\opt \asseq \min_{\hyp^* \in \hyps} \max_{\dist \in \dists} \risk_{\dist}(\hyp^*)$.
Our open problem concerns the sample complexity of MDL.


\paragraph{Problem Statement.}
Consider an example oracle 
$\oracle_i$ for each distribution $\dist_i \in \dists$, which once queried returns an independent sample $(x,y)\sim \dist_i$.
The optimal sample complexity of MDL is the smallest total number of queries issued to examples oracles, in a possibly adaptive fashion, that is sufficient for learning an $\epsilon$-optimal solution.
Formally, 
a multi-distribution learning algorithm at each iteration $t = 1,2, \dots$, chooses an index $\tsv{i}{t} \in [k]$, queries $\oracle_{\tsv{i}{t}}$ to sample an instance $(\tsv{x}{t}, \tsv{y}{t})$ and, upon termination, returns a (possibly randomized) solution $\hyp$.
We use the shorthands $\tsv{z}{t} = (\tsv{x}{t}, \tsv{y}{t}, \tsv{i}{t})$, $\cZ = \features \times \labels \times [k]$, and $\cZ^*$ to denote a sequence $\tsv{z}{1},\tsv{z}{2}, \dots$ of any size.
\begin{definition}[Multi-Distribution Learnability]
We say a hypothesis class $\hyps$ is multi-distribution learnable with sample complexity $m_\hyps: (0, 1)^2 \times \naturals \to \naturals$ if there exists functions $\cA_s: \cZ^* \to [k]$ and $\cA_\hyp: \cZ^* \to \simplex(\labels)^\features$ where the following holds: for every $(\epsilon, \delta) \in (0, 1)$, $k \in \naturals$, and set of $k$ distributions $\dists$ over $\features \times \labels$, by letting $\tsv{i}{t} = \cA_s(\tsv{z}{1}, \dots, \tsv{z}{t-1})$ for $t \in [m_\hyps(\epsilon, \delta, k)]$, with probability at least $1 - \delta$, the solution $\hyp = \cA_\hyp(\tsv{z}{1}, \dots, \tsv{z}{m})$ is $\epsilon$-optimal, i.e., satisfying \eqref{eq:optimal}.
\end{definition}
\begin{problem}
\label{prob:main}
What is the optimal sample complexity of MDL? 
Are hypothesis classes  $\hyps$ with VC dimension $d$  multi-distribution learnable with a sample complexity of $O\left( \epsilon^{-2}(\ln(k) d + k \ln(k / \delta)\right))$?

\end{problem}

Recalling that the sample complexity of agnostic learning is $m_\hyps(\epsilon, \delta, 1) \in \Theta(\epsilon^{-2}(d + \ln(1/ \delta)))$ \cite{shai}, one hopes to avoid paying the $\Omega\paraflat{k \cdot m_\hyps(\epsilon, \delta/k, 1)}$ samples necessary to independently learn each of the $k$ data distributions. This is why our conjectured sample complexity avoids a dependence on $dk$ and has an optimal $\epsilon^{-2}$ dependence.
Existing results, however, have fallen short of meeting both of these requirements and traded off lack of dependence on $dk$ with the optimal dependence on $\epsilon$, as shown in rows 1 and 2 of Table~\ref{tab:bounds}.
On the other hand, the optimal sample complexity of MDL has been rightly characterized for finite hypothesis classes in row 3 (and more generally those of finite Littlestone dimension or Bregman diameter~\citep{haghtalabOnDemandSamplingLearning2022}) and obtains optimal $\epsilon^{-2}\ln(|\hyps|)$ dependence.
The best lower bound, row 4, leaves a logarithmic gap with the conjectured upper bound.
Near-optimal bounds are known for \emph{realizable} settings where $\opt \mkern-4mu = \mkern-4mu 0$ (row 5) and \emph{personalized} settings where one can produce a different hypothesis for each distribution (row 6).
\begin{table}[htbp]
	\centering
	\caption{Best known bounds on the sample complexity of MDL for hypothesis classes with VC dimension $d$. $\tilde{O}$ hides double-log factors and an additive factor of $\epsilon^{-2} k \ln(k/\delta)$.}
	\label{tab:bounds}
	\begin{tabular}{llll}
		\toprule
		&\textbf{Bound}                                                                                           & \textbf{Assumption} & \textbf{Citation}                                 \\
		\midrule
		1.&$\tilde{O}(\epsilon^{-2} \ln(k) d 
 + \epsilon^{-1} \red{dk} \log(d/\epsilon)) $ & N/A                 & \cite{haghtalabOnDemandSamplingLearning2022}     \\
		2.&$\tilde{O}(\red{\epsilon^{-4}} \ln(k)(d + \ln(1/\delta\epsilon))$             & N/A                 & (Theorem~\ref{theorem:diff})                        \\
		3.&$\tilde{O}(\epsilon^{-2}\ln(\red{\setsize{\hyps}}))$                                               & N/A                 & \cite{haghtalabOnDemandSamplingLearning2022}     \\
		4.&$\Omega(\epsilon^{-2}(d + k \ln(\min\bset{d, k}/\delta)))$                                               & N/A                 & \cite{haghtalabOnDemandSamplingLearning2022}     \\
		\midrule
		5.&	${O}(\ln(k) \epsilon^{-1} (d \ln(1/\epsilon)  + k  \ln(k / \delta) ))$ & $\opt = 0$          & \cite{chenTightBoundsCollaborative2018,nguyenImprovedAlgorithmsCollaborative2018} \\
	6.&	$\tilde{O}(\ln(k) \epsilon^{-2} (d \ln(d/\epsilon)  + k  \ln(k / \delta) ))$                              & Personalized       & (Theorem~\ref{theorem:personal})                        \\
		\bottomrule
	\end{tabular}
\end{table}

\paragraph{Broad Applications.}
One of the motivating application of MDL is \emph{collaborative learning}, where multiple stakeholders (representing $\dist_i$) collaborate in training a model that provides high performance for each stakeholder~\cite{blumCollaborativePACLearning2017,nguyenImprovedAlgorithmsCollaborative2018,chenTightBoundsCollaborative2018,blum_one_2021}.
The sample complexity of MDL thus quantifies the value of collaboration in learning: whereas our conjectured upper bound would imply that collaboration reduces the amount of data needed by a $\ln(k) / k$ factor, existing bounds only imply a $\min \bset{\ln(k) / k \epsilon^2, \epsilon}$ factor reduction.

Another application of MDL is to Group \emph{distributionally robust optimization} (DRO) which concerns learning a model with performance guarantees for  many deployment environments \cite{sagawa_distributionally_2020,sagawa_investigation_2020}.
MDL sample complexity bounds quantify the cost of obtaining this robustness, a question of growing interest and which has been studied in terms of finite-sum convergence \cite{carmonhausler,asi2021} and sample complexity \cite{haghtalabOnDemandSamplingLearning2022}.
Our conjectured upper bound would extend these favorable results to VC classes by only increasing the sample complexity logarithmically.

MDL also captures notions of min-max fairness in learning, which concerns prioritizing the well-being of the worst-off subgroup and has applications in federated learning \cite{mohri_agnostic_2019} and equity \cite{Abernethy2022}.
Min-max fair learning has mainly been studied in settings with presampled datasets, where an inevitable sample complexity lower bound of $\Omega(dk/\epsilon^2)$ arises as one cannot adaptively choose distributions to sample from.
The sample complexity of MDL thus captures how min-max fairness can be attained at less cost by adapting one's data collection strategy on the fly.

\section{Overview of Current Approaches}
Multi-distribution learning can be formulated as the zero-sum game between a ``learner'' who chooses hypotheses $\hyp \in \hyps$ and an ``adversary'' whose chooses indices $i \in [k]$, with the payoff function $\risk_{D_i}(h)$.
Importantly, for any mixed-strategy $\epsilon$-min-max equilibrium $(\rhyp, \rloss) \in \simplex(\hyps) \times \simplex_k$, the randomized map $\rhyp$ is a  $2 \epsilon$-optimal solution.
All existing multi-distribution learning algorithms can be expressed as finding a $\epsilon$-equilibrium using no-regret dynamics (see \cite{haghtalabOnDemandSamplingLearning2022} for an overview).

\paragraph{Game dynamics.}
Formally, a game dynamic is a $T$-iteration process where, at each $t \in [T]$, a learner chooses hypothesis $\smash{\tsv{\hyp}{t} \in \hyps}$ with a no-regret algorithm and an adversary chooses a distribution $\smash{\tsv{i}{t} \in [k]}$ with a (semi-)bandit algorithm.
The learner estimates its current cost function $\smash{\hyp \mapsto \risk_{\dist_{\tsv{i}{t}}}(\hyp)}$ by sampling $N_{\mathrm{learn}}$ datapoints from $\smash{\oracle_{\tsv{i}{t}}}$, while the adversary estimates its cost function $\smash{i \mapsto - \risk_{\dist_{i}}(\tsv{\hyp}{t})}$ by, for $N_{\mathrm{adv}}$ choices of $i \in [k]$, sampling a datapoint from each $\oracle_i$.
The random mapping $\rhyp$ where $\rhyp(x) = \text{Uniform}(\tsv{\hyp}{1}(x), \dots, \tsv{\hyp}{t}(x))$ is a $2 \epsilon$-optimal solution.

\paragraph{Different instantiations.}
Every result in Table~\ref{tab:bounds} can be obtained by instantiating this game dynamics template.
Row 3 can be obtained 
by setting $N_{\mathrm{learn}}=N_{\mathrm{adv}}=1$, $T \propto \epsilon^{-2}(\ln(\setsize{\hyps}) + k \ln(k/\delta))$, having the learner choose $\tsv{\hyp}{t}$ with Hedge and the adversary choose $\tsv{i}{t}$ with Exp3 \cite{haghtalabOnDemandSamplingLearning2022}.
Row 1 can be obtained with the same algorithm but first creating an offline $\epsilon$-covering 
 of the class $\hyps$ on each data distribution $D_i \in \dists$, using $O(d/\epsilon)$ samples per distribution.
Row 2
can be obtained by setting $N_{\mathrm{adv}}=k$, $N_{\mathrm{learn}}\propto \epsilon^{-2}(d + \ln(1/\delta\epsilon))$, $T \propto \epsilon^{-2} \ln(k/\delta)$, having the learner choose $\tsv{\hyp}{t}$ to be the (approximate) risk minimizer of the current cost function and the adversary choose $\tsv{i}{t}$ with Hedge (Theorem~\ref{theorem:diff});
in contrast to the prior upper bound, this  bound uses an algorithm that iterates fewer times but samples more at each iteration.





\paragraph{Personalization.}
We can pinpoint the challenge of negotiating trade-offs between different data distributions as the primary difficulty of handling infinite classes.
Consider the personalized setting where, during inference time, $\cA_\hyp(\tsv{z}{1}, \dots, \tsv{z}{m})$ can return a different hypothesis $h_i$ for each distribution $\dist_i$.
This assumes away the difficulty of combining hypotheses that are each near-optimal for different distributions.
The conjectured sample complexity bound of $\tilde{O}(\ln(k) \epsilon^{-2} (d \ln(d/\epsilon)  + k  \ln(k / \delta) ))$ can be obtained in the personalized setting (Row 6 of Table~\ref{tab:bounds}) by running the Row 1 algorithm $\ln(k)$ times, at each round limiting the adversary to playing within a small region of the simplex $\Delta_k$ that we can efficiently cover $\hyps$ on (Theorem~\ref{theorem:personal}).

\subsection{Existing Challenges}
\paragraph{Adaptive coverings.}
A potential approach to closing the gap with the conjectured sample complexity bound is to find a method of adaptively covering the hypothesis class $\hyps$.
Whereas Row 1 was obtained by taking a naive offline $\epsilon$-covering of $\hyps$ on all $k$ distributions, Row 2 was obtained by an algorithm that (implicitly) $\epsilon$-covers the class $\hyps$ on $O(\ln(k)\epsilon^{-2})$ adaptive choices of $D_i \in \dists$.
It is unclear whether a covering of lower resolution can be used, or if it is possible to only cover $\hyps$ on $O(\ln(k))$ choices of distributions $D_i \in \dists$.
 We also note that it is not the size of the $\epsilon$-covering of $k$ distributions, i.e., $k{\epsilon}^{-O(d)}$, that is the bottleneck, but rather the number of samples needed %
 to create such a cover. %
In contrast, the personalized algorithm decided in an online fashion what distributions need to be covered and it only covers $\hyps$ on $O(\ln(k))$ choice of (mixture) distributions from $\dists$.

\paragraph{Agnostic-to-realizable.}
Another potential tool is an agnostic-to-realizable reduction \cite{hopkins22}, since nearly-optimal sample complexity bounds are known for realizable settings where $\opt = 0$ \cite{blumCollaborativePACLearning2017,chenTightBoundsCollaborative2018,nguyenImprovedAlgorithmsCollaborative2018}.
This technique has had success in related problems, such as the closely related adversarial PAC learning problem \cite{montasserVC2019}.
Unfortunately, because multi-distribution learning involves online decision-making---determining which example oracles to call---the usual reduction of testing all possible labelings of observed datapoints is intractable.

\paragraph{Bounding regret.}
Game dynamics algorithms rely on the learner achieving a low regret on the sequence of distributions chosen by the adversary.
However, with VC classes,  even when all distributions share a Bayes classifier, an oblivious adversary can force the learner to suffer regret linear in $k$.
It is therefore necessary to reason about the adversary's behavior to bound the regret of the learner.
This is atypical; game dynamics proofs usually bound each player's regret independently.
\begin{restatable}{proposition}{difficult}
	\label{proposition:difficult}
	Consider an algorithm $\cA$ that, given distributions $D_1, \dots, D_T$, draws only $N$ datapoints in total and returns a sequence of hypotheses $\hyp_1, \dots, \hyp_k$ where each $\hyp_t$ is trained only on datapoints sampled from $D_1, \dots, D_t$.
	There exists a sequence $D_1, \dots, D_T$ with only $k$ distinct members, where $\smash{\mathbb{E}[T^{-1} \sum_{t \in [T]} \err_{D_{t}}(\hyp_t)] - \min_{h^* \in \cH} T^{-1} \sum_{t \in [T]} \err_{D_{t}}(h^*)
			\in \Omega \paraflat{\sqrt{\vcd k / N}}.}$
\end{restatable}

\section{Intermediate Open Problems}
\paragraph{Lower Bounds.}
We believe a $\ln(k) \vcd$ factor is missing from the best known sample complexity lower bound of $\Theta(\epsilon^{-2}(\vcd + k \ln(\min\bset{k, \vcd} / \delta)))$.
The absence of a $\ln(k) \vcd$ term would be significant as it would imply that, when VC dimension dominates sample complexity, handling more data distributions comes effectively for free.
Interestingly, this $\ln(k)$ factor does not appear in the upper bound when the complexity of $\hyps$ is characterized by Littlestone dimension, perhaps due to the stronger compression guarantees for online-learnable classes.
A $\ln(k) \vcd$ term would also shed light on compression schemes for VC classes \cite{littlestone1986relating}; a lower bound of $\Theta(\ln(k) \vcd + k)$ would lend evidence against the existence of $O(\text{VC}(\hyps))$-size compression schemes.
\begin{problem}
\label{problem:lowerbound}
Is the sample complexity of multi-distribution learning in $\Omega(\log(k) \vcd)$?
\end{problem}

\paragraph{Proper learning.}
All existing multi-distribution learning algorithms with fast sample complexity rates produce either a randomized hypothesis $\hyp \in \simplex(\hyps)$ or an improper hypothesis resulting from taking a majority vote.
An open question is whether improperness is necessary for fast rates.
\begin{problem}
What is the sample complexity of proper multi-distribution learning?
\end{problem}
\paragraph{Oracle-efficient learning.}
For oracle-efficient algorithms, that is an algorithm only accessing $\hyps$ through an ERM oracle \cite{dudik2020}, only the sample complexity bound from Row 2 in Table~\ref{tab:bounds} is known.
An open question is whether there exists a statistical-computational trade-off for MDL. 
\begin{problem}
What is the sample complexity of oracle-efficient multi-distribution learning?
\end{problem}

\bibliographystyle{alpha}
% \bibliography{references}
\begin{thebibliography}{10}

\bibitem{eilers2021product}Eilers, M., Meier, S. \& Müller, P. Product Programs in the Wild: Retrofitting Program Verifiers to Check Information Flow Security. {\em Computer Aided Verification (CAV)}. (2021)
\bibitem{tiwari2009complete}Tiwari, M., Wassel, H., Mazloom, B., Mysore, S., Chong, F. \& Sherwood, T. Complete information flow tracking from the gates up. {\em Proceedings Of The 14th International Conference On Architectural Support For Programming Languages And Operating Systems}. pp. 109-120 (2009)

\bibitem{tiwari2009execution}Tiwari, M., Li, X., Wassel, H., Chong, F. \& Sherwood, T. Execution leases: A hardware-supported mechanism for enforcing strong non-interference. {\em Proceedings Of The 42nd Annual IEEE/ACM International Symposium On Microarchitecture}. pp. 493-504 (2009)

\bibitem{jin2012proof}Jin, Y. \& Makris, Y. Proof carrying-based information flow tracking for data secrecy protection and hardware trust. {\em 2012 IEEE 30th VLSI Test Symposium (VTS)}. pp. 252-257 (2012)

\bibitem{li2011caisson}Li, X., Tiwari, M., Oberg, J., Kashyap, V., Chong, F., Sherwood, T. \& Hardekopf, B. Caisson: A Hardware Description Language for Secure Information Flow. {\em Proceedings Of The 32Nd ACM SIGPLAN Conference On Programming Language Design And Implementation}. pp. 109-120 (2011), http://doi.acm.org/10.1145/1993498.1993512

\bibitem{li2014sapper}Li, X., Kashyap, V., Oberg, J., Tiwari, M., Rajarathinam, V., Kastner, R., Sherwood, T., Hardekopf, B. \& Chong, F. Sapper: A Language for Hardware-level Security Policy Enforcement. {\em Proceedings Of The 19th International Conference On Architectural Support For Programming Languages And Operating Systems}. pp. 97-112 (2014), http://doi.acm.org/10.1145/2541940.2541947

\bibitem{zhang2015secverilog}Zhang, D., Wang, Y., Suh, G. \& Myers, A. A Hardware Design Language for Timing-Sensitive Information-Flow Security. {\em Proceedings Of The Twentieth International Conference On Architectural Support For Programming Languages And Operating Systems}. pp. 503-516 (2015), http://doi.acm.org/10.1145/2694344.2694372

\bibitem{bidmeshki2015vericoq}Bidmeshki, M. \& Makris, Y. VeriCoq: A Verilog-to-Coq converter for proof-carrying hardware automation. {\em 2015 IEEE International Symposium On Circuits And Systems (ISCAS)}. pp. 29-32 (2015)

\bibitem{hu2016detecting}Hu, W., Mao, B., Oberg, J. \& Kastner, R. Detecting hardware trojans with gate-level information-flow tracking. {\em Computer}. \textbf{49}, 44-52 (2016)

\bibitem{kong2017using}Kong, S., Shen, Y. \& Zhou, H. Using security invariant to verify confidentiality in hardware design. {\em Proceedings Of The On Great Lakes Symposium On VLSI 2017}. pp. 487-490 (2017)

\bibitem{ardeshiricham2017register}Ardeshiricham, A., Hu, W., Marxen, J. \& Kastner, R. Register transfer level information flow tracking for provably secure hardware design. {\em Proceedings Of The Conference On Design, Automation \& Test In Europe (DATE)}. pp. 1695-1700 (2017), http://dl.acm.org/citation.cfm?id=3130379.3130775

\bibitem{ardeshiricham2017clepsydra}Ardeshiricham, A., Hu, W. \& Kastner, R. Clepsydra: Modeling timing flows in hardware designs. {\em 2017 IEEE/ACM International Conference On Computer-Aided Design (ICCAD)}. pp. 147-154 (2017)

\bibitem{deng2017secchisel}Deng, S., Gümüşoğlu, D., Xiong, W., Gener, Y., Demir, O. \& Szefer, J. SecChisel: language and tool for practical and scalable security verification of security-aware hardware architectures. {\em Cryptology EPrint Archive}. (2017)

\bibitem{bidmeshki2017information}Bidmeshki, M., Antonopoulos, A. \& Makris, Y. Information flow tracking in analog/mixed-signal designs through proof-carrying hardware IP. {\em Design, Automation \& Test In Europe Conference \& Exhibition (DATE), 2017}. pp. 1703-1708 (2017)

\bibitem{boraten2018securing}Boraten, T. \& Kodi, A. Securing NoCs against timing attacks with non-interference based adaptive routing. {\em 2018 Twelfth IEEE/ACM International Symposium On Networks-on-Chip (NOCS)}. pp. 1-8 (2018)

\bibitem{pilato2018tainthls}Pilato, C., Wu, K., Garg, S., Karri, R. \& Regazzoni, F. Tainthls: High-level synthesis for dynamic information flow tracking. {\em IEEE Transactions On Computer-Aided Design Of Integrated Circuits And Systems}. \textbf{38}, 798-808 (2018)

\bibitem{zagieboylo2019using}Zagieboylo, D., Suh, G. \& Myers, A. Using information flow to design an ISA that controls timing channels. {\em 2019 IEEE 32nd Computer Security Foundations Symposium (CSF)}. pp. 272-27215 (2019)

\bibitem{pieper2020dynamic}Pieper, P., Herdt, V., Große, D. \& Drechsler, R. Dynamic information flow tracking for embedded binaries using SystemC-based virtual prototypes. {\em 2020 57th ACM/IEEE Design Automation Conference (DAC)}. pp. 1-6 (2020)

\bibitem{restuccia2021aker}Restuccia, F., Meza, A. \& Kastner, R. AKER: A design and verification framework for safe and secure soc access control. {\em IEEE/ACM International Conference On Computer Aided Design (ICCAD)}. (2021), https://par.nsf.gov/servlets/purl/10298115

\bibitem{restuccia2022framework}Restuccia, F., Meza, A., Kastner, R. \& Oberg, J. A Framework for Design, Verification, and Management of SoC Access Control Systems. {\em IEEE Transactions On Computers}. (2022), https://kastner.ucsd.edu/wp-content/uploads/2022/10/admin/tcomputer-aker22.pdf

\bibitem{cherupalli2017software}Cherupalli, H., Duwe, H., Ye, W., Kumar, R. \& Sartori, J. Software-based gate-level information flow security for IoT systems. {\em 50th IEEE/ACM International Symposium On Microarchitecture}. (2017), https://dl.acm.org/doi/pdf/10.1145/3123939.3123955

\bibitem{fadiheh2023exhaustive}Fadiheh, M., Wezel, A., Muller, J., Bormann, J., Ray, S., Fung, J., Mitra, S., Stoffel, D. \& Kunz, W. An Exhaustive Approach to Detecting Transient Execution Side Channels in RTL Designs of Processors. {\em IEEE Transactions On Computers}. \textbf{72}, 222-235 (2023,1)

\bibitem{wu2022exert}Wu, J., Fowze, F. \& Forte, D. EXERT: EXhaustive Integrity Analysis for Information Flow Security. {\em Asian Hardware Oriented Security And Trust Symposium (AsianHOST)}. (2022), https://dforte.ece.ufl.edu/wp-content/uploads/sites/65/2022/09/EXERT%5C_AsianHost.pdf

\bibitem{athalye2022knox}Athalye, A., Kaashoek, M. \& Zeldovich, N. Verifying Hardware Security Modules with Information-Preserving Refinement. {\em OSDI}. (2022)

\bibitem{fowze2022eisec}Fowze, F., Choudhury, M. \& Forte, D. EISec: Exhaustive Information Flow Security of Hardware Intellectual Property Utilizing Symbolic Execution. {\em Asian Hardware Oriented Security And Trust Symposium (AsianHOST)}. (2022)

\bibitem{athalye2019notary}Athalye, A., Belay, A., Kaashoek, M., Morris, R. \& Zeldovich, N. Notary: A Device for Secure Transaction Approval. {\em 27th Symposium On Operating Systems Principles (SOSP)}. (2019), https://doi.org/10.1145/3341301.3359661

\bibitem{meza2023hyperflowgraph}Meza, A. \& Kastner, R. Information Flow Coverage Metrics for Hardware Security Verification.  (2023), arXiv 2304.08263

\bibitem{ryan2023countering}Ryan, K. \& Sturton, C. Countering the Path Explosion Problem in the Symbolic Execution of Hardware Designs.  (2023), arXiv 2304.05445 

\bibitem{dorsey2020intel}Dorsey, V. \& Morhardt, C. Intel Security Development Lifecycle. (Intel,2020)

\bibitem{he2015model}He, S., Roe, N., Wood, E., Nachtigal, N. \& Helms, J. Model of the Product Development Lifecycle. (Sandia National Laboratories,2015)

\bibitem{YangSP2016}Yang, K., Hicks, M., Dong, Q., Austin, T. \& Sylvester, D. A2: Analog Malicious Hardware. {\em 2016 IEEE Symposium On Security And Privacy (SP)}. pp. 18-37 (2016)

\bibitem{or1200}. OpenRISC 1200 Implementation. , https://github.com/openrisc/or1200

\bibitem{msp430}. openMSP430. , https://opencores.org/projects/openmsp430

\bibitem{farzana2019soc}Farzana, N., Rahman, F., Tehranipoor, M. \& Farahmandi, F. SoC Security Verification using Property Checking. {\em 2019 IEEE International Test Conference (ITC)}. pp. 1-10 (2019)

\bibitem{TrustHub2}Farzana, N., Farahmandi, F. \& Tehranipoor, M. SoC Security Properties and Rules. {\em IACR Cryptol. EPrint Arch.}. \textbf{2021} pp. 1014 (2021)

\bibitem{hicks2015specs}Hicks, M., Sturton, C., King, S. \& Smith, J. SPECS: A Lightweight Runtime Mechanism for Protecting Software from Security-Critical Processor Bugs. {\em ASPLOS}. pp. 517-529 (2015)

\bibitem{bilzor2011security}Bilzor, M., Huffmire, T., Irvine, C. \& Levin, T. Security Checkers: Detecting processor malicious inclusions at runtime. {\em HOST}. (2011)

\bibitem{zhang2017scifinder}Zhang, R., Stanley, N., Griggs, C., Chi, A. \& Sturton, C. Identifying Security Critical Properties for the Dynamic Verification of a Processor. {\em ASPLOS}. pp. 541-554 (2017)

\bibitem{zhang2020transys}Zhang, R. \& Sturton, C. Transys: Leveraging Common Security Properties Across Hardware Designs. {\em Proceedings Of The Symposium On Security And Privacy (S\&P)}. (2020)

\bibitem{trippel2020ICAS}Trippel, T., Shin, K., Bush, K. \& Hicks, M. ICAS: an Extensible Framework for Estimating the Susceptibility of IC Layouts to Additive Trojans. {\em 2020 IEEE Symposium On Security And Privacy (SP)}. pp. 1742-1759 (2020)

\bibitem{Deutschbein2022JCEN}Deutschbein, C., Meza, A., Restuccia, F., Kastner, R. \& Sturton, C. Isadora: Automated Information Flow Property Generation for Hardware Security Verification. {\em Journal Of Cryptographic Engineering (JCEN)}. (2022)

\bibitem{zhang2021sidechannel}Zhang, T., Park, J., Tehranipoor, M. \& Farahmandi, F. PSC-TG: RTL Power Side-Channel Leakage Assessment with Test Pattern Generation. {\em 2021 58th ACM/IEEE Design Automation Conference (DAC)}. pp. 709-714 (2021)

\bibitem{torlak2014rosette}Torlak, E. \& Bodik, R. A Lightweight Symbolic Virtual Machine for Solver-Aided Host Languages. {\em Proceedings Of The 35th ACM SIGPLAN Conference On Programming Language Design And Implementation}. pp. 530-541 (2014), https://doi.org/10.1145/2594291.2594340

\bibitem{cha2012mayhem}Cha, S., Avgerinos, T., Rebert, A. \& Brumley, D. Unleashing Mayhem on Binary Code. {\em Proceedings Of The 2012 IEEE Symposium On Security And Privacy}. pp. 380-394 (2012)

\bibitem{bao2021symbolic}Bao, Q., Wang, Z., Li, X., Larus, J. \& Wu, D. Abacus: Precise side-channel analysis. {\em International Conference On Software Engineering (ICSE)}. pp. 797-809 (2021)

\bibitem{wang2017cached}Wang, S., Wang, P., Liu, X., Zhang, D. \& Wu, D. CacheD: Identifying cache-based timing channels in production software. {\em USENIX Security Symposium}. pp. 235-252 (2017)

\bibitem{wang2019identifying}Wang, S., Bao, Y., Liu, X., Wang, P., Zhang, D. \& Wu, D. Identifying Cache-Based Side Channels through Secret-Augmented Abstract Interpretation. {\em 28th USENIX Security Symposium (USENIX Security 19)}. pp. 657-674 (2019,8), https://www.usenix.org/conference/usenixsecurity19/presentation/wang-shuai

\bibitem{brotzman2019casym}Brotzman, R., Liu, S., Zhang, D., Tan, G. \& Kandemir, M. Casym: Cache aware symbolic execution for side channel detection and mitigation. {\em Symposium On Security And Privacy (SP)}. (2019)

\bibitem{guarnier2020spectector}Guarnieri, M., Köpf, B., Morales, J., Reineke, J. \& Sánchez, A. Spectector: Principled Detection of Speculative Information Flows. {\em 2020 IEEE Symposium On Security And Privacy (SP)}. pp. 1-19 (2020)

\bibitem{avgerinos2014automatic}Avgerinos, T., Cha, S., Rebert, A., Schwartz, E., Woo, M. \& Brumley, D. Automatic exploit generation. {\em Communications Of The ACM}. \textbf{57}, 74-84 (2014)

\bibitem{avgerinos2011automatic}Avgerinos, T., Hao, B. \& Brumley, D. Automatic exploit generation. {\em Network And Distributed System Security Symposium (NDSS)}. (2011)

\bibitem{renzelmann2012symdrive}Renzelmann, M., Kadav, A. \& Swift, M. SymDrive: Testing Drivers without Devices. {\em 10th USENIX Symposium On Operating Systems Design And Implementation}. (2012), https://www.usenix.org/conference/osdi12/technical-sessions/presentation/renzelmann

\bibitem{zhang2018end}Zhang, R., Deutschbein, C., Huang, P. \& Sturton, C. End-to-End Automated Exploit Generation for Validating the Security of Processor Designs. {\em Proceedings Of The International Symposium On Microarchitecture (MICRO)}. (2018)

\bibitem{Shen2018SymbolicEB}Shen, L., Mu, D., Cao, G., Qin, M., Blackstone, J. \& Kastner, R. Symbolic execution based test-patterns generation algorithm for hardware Trojan detection. {\em Comput. Secur.}. \textbf{78} pp. 267-280 (2018)

\bibitem{clarkson2010hyperproperties}Clarkson, M. \& Schneider, F. Hyperproperties. {\em J. Comput. Secur.}. \textbf{18}, 1157-1210 (2010,9), http://dl.acm.org/citation.cfm?id=1891823.1891830

\bibitem{Kozyri2022expressing}Kozyri, E., Chong, S. \& Myers, A. Expressing Information Flow Properties. {\em Foundations And Trends® In Privacy And Security}. \textbf{3}, 1-102 (2022), http://dx.doi.org/10.1561/3300000008

\bibitem{meza2022safety}Meza, A., Restuccia, F., Kastner, R. \& Oberg, J. Safety verification of third-party hardware modules via information flow tracking. {\em 1st Real-Time Intelligent Edge Computing Workshop (RAGE)}. (2022), https://kastner.ucsd.edu/wp-content/uploads/2022/08/admin/rage22-safety.pdf

\bibitem{Deutschbein2021Isadora}Deutschbein, C., Meza, A., Restuccia, F., Kastner, R. \& Sturton, C. Isadora: Automated Information Flow Property Generation for Hardware Designs. {\em Proceedings Of The Workshop On Attacks And Solutions In Hardware Security (ASHES)}. (2021)

\bibitem{ferraiuolo2017secverilog}Ferraiuolo, A., Xu, R., Zhang, D., Myers, A. \& Suh, G. Verification of a Practical Hardware Security Architecture Through Static Information Flow Analysis. {\em Proceedings Of The Twenty-Second International Conference On Architectural Support For Programming Languages And Operating Systems}. pp. 555-568 (2017), http://doi.acm.org/10.1145/3037697.3037739

\bibitem{ardeshiricham2019verisketch}Ardeshiricham, A., Takashima, Y., Gao, S. \& Kastner, R. VeriSketch: Synthesizing Secure Hardware Designs with Timing-Sensitive Information Flow Properties. {\em Proceedings Of The 2019 ACM SIGSAC Conference On Computer And Communications Security}. pp. 1623-1638 (2019)


\end{thebibliography}


\newpage
\appendix
\begin{comment}
\section{System Architecture}
\label{appendix:architecture}
\system has a novel modularized system architecture with three key components: 
\emph{StreamManager}, 
\emph{TxnManager} and \emph{TxnScheduler}. 
These components are instantiated in each thread locally.
The execution outline of \system is presented in Algorithm~\ref{alg:algo}.
Transactional stream processing is continuous and potentially never ends (Line 1$\sim$8).
The dependency resolution and execution of state transactions are separated into two non-overlapping phases by punctuations~\cite{Tucker:2003:EPS:776752.776780} (Line 2 and 5), which guarantees that no subsequent input event will have a smaller timestamp. 
Effectively, a batch of state transactions is collected during the first phase, and processed during the second phase.

In the first phase (i.e., stream processing phase), 
the \emph{StreamManager} conducts preprocessing for every input event ($e$). Similar to some prior works~\cite{tstream}, state transactions may be issued but not immediately processed during preprocessing (Line 3).
The \emph{pre\_processing} and \emph{post\_processing} functions are exposed as APIs to users.
The \emph{TxnManager} handles dependency resolution (Line 4) among state transactions and insert decomposed operations to construct a \tpg. We discuss the detailed two-phase \tpg construction process in Section~\ref{subsec:construction}.

In the second phase  (i.e., transaction processing phase), 
the \emph{TxnManager} is first involved again to refine (Line 6) the constructed \tpg with further dependency resolution.
The \emph{TxnScheduler} 
schedules operations for concurrent execution based on the constructed \tpg according to the three dimensions of scheduling decisions (Line 7). 
In particular, a scheduling decision model $M$ is instantiated based on the constructed \tpg (Line 14).
\textbf{\circled{1}} Guided by $M$, execution threads adopt an exploration strategy (Section~\ref{subsec:explore}) to explore the constructed \tpg for operations available to be scheduled constrained by dependencies. 
\textbf{\circled{2}} 
During exploration, one or multiple operations may be treated as the 
% basic 
unit of scheduling (Section~\ref{subsec:granularity}). 
Subsequently, \textbf{\circled{3}} every thread executes operation(s) in the unit of scheduling with various abort handling mechanisms (Section~\ref{subsec:abort_handling}).
Only when state transactions are processed (i.e., committed or aborted) can the associated input events be postprocessed (Line 8) by the \emph{StreamManager} based on transaction processing results.
\end{comment}

\begin{comment}
\begin{algorithm}
\footnotesize
    \KwData{$e$ \tcp{Input event}}
    \KwData{$txn_{ts}$ \tcp{State transaction}}
    \KwData{$G$ \tcp{The currently constructed TPG}}
    \While{!finish processing of input streams}{
        \eIf(\tcp*[h]{Phase 1}){\text{$e$ is not a $punctuation$}}{
                $txn_{ts}$ $\gets$ PRE\_Processing($e$)\;
                \textbf{TPG\_Construction}($G$, $txn_{ts}$)\; 
          }(\tcp*[h]{Phase 2}){
                \textbf{TPG\_Refinement}($G$)\; 
                \textbf{TXN\_Scheduling}($G$)\; 
                POST\_Processing()\;
          }
    }
    
    \SetKwFunction{FMain}{TPG\_Construction}
    \SetKwProg{Fn}{Function}{:}{}
    \Fn{\FMain{$G$, $txn_{ts}$}}{
        $O_{1..k}$ $\gets$ \textbf{Partition} $txn_{ts}$\;
        \ForEach{\text{operation $O_{i}$ $\in$ $O_{1..k}$}}{
            \textbf{Identify} its \ld\;
            $G$ $\gets$ $G$ + $O_{i}$ \;
        }
    }
    \SetKwFunction{FMain}{TPG\_Refinement}
    \SetKwProg{Fn}{Function}{:}{}
    \Fn{\FMain{$G$}}{
        \ForEach{\text{vertex $e_{i}$ $\in$ $G$}}{
            \textbf{Identify} its \td, \pd\;
        }
    }
    
    \SetKwFunction{FMain}{TXN\_Scheduling}
    \SetKwProg{Fn}{Function}{:}{}
    \Fn{\FMain{$G$}}{
        $M$ $\gets$ Instantiated with $G$;\tcp{A decision model}
        \While{!finish scheduling of $G$
        }{
          \textbf{\circled{2}} $Scheduling Unit$ $\gets$ \textbf{\circled{1}} \emph{Explore}($G$, $M$)\; 
            \textbf{\circled{3}} \emph{Execute with Abort Handling} ($Scheduling Unit$)\; 
        }
    }
  \caption{Execution Outline of \system}
  \label{alg:algo}
\end{algorithm}
\end{comment}

\end{document}
