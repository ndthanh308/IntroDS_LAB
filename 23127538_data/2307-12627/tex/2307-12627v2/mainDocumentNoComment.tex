%\documentclass{article}
%\documentclass[preprint,authoryear]{elsarticle}
\documentclass[11pt,a4paper]{article}
\usepackage{authblk}

%\usepackage{lineno}
%\modulolinenumbers[5]
\usepackage{comment}
\usepackage[english]{babel}
\usepackage[T1]{fontenc}
\usepackage[utf8]{inputenc}
\usepackage{soul}
\usepackage{color}
\usepackage{amsmath,amsfonts,amsthm,amssymb}
\usepackage{mathtools}
\usepackage{bbold}
\usepackage{palatino}
\usepackage{eurosym}
\usepackage{enumerate}
\usepackage{ulem}
\usepackage[toc,page]{appendix}
\usepackage[inline]{enumitem}
\usepackage[%  
colorlinks=true,
pdfborder={0 0 0},
linkcolor=red
]{hyperref}
%\setlength{\oddsidemargin}{0mm}
%\setlength{\evensidemargin}{0mm}
%\setlength{\topmargin}{3mm}
%\setlength{\textheight}{22cm}
%\setlength{\textwidth}{16cm}
\usepackage[top=2.5cm, left=2.5cm, right=2.5cm, bottom=2.5cm]{geometry}
%\usepackage[style=authoryear]{biblatex} 
\usepackage[square,numbers]{natbib}
\usepackage[etex=true,export]{adjustbox}
\numberwithin{equation}{section}
%\bibliographystyle{abbrvnat}
\usepackage{xr}
\usepackage{import}
%\usepackage{example}

%\newcommand*{\addFileDependency}[1]{% argument=file name and extension
%	\typeout{(#1)}% latexmk will find this if $recorder=0
%	% however, in that case, it will ignore #1 if it is a .aux or 
%	% .pdf file etc and it exists! If it doesn't exist, it will appear 
%	% in the list of dependents regardless)
%	%
%	% Write the following if you want it to appear in \listfiles 
%	% --- although not really necessary and latexmk doesn't use this
%	%
%	\@addtofilelist{#1}
%	%
%	% latexmk will find this message if #1 doesn't exist (yet)
%	\IfFileExists{#1}{}{\typeout{No file #1.}}
%}\makeatother
%
%\newcommand*{\myexternaldocument}[1]{%
%	\externaldocument{#1}%
%	\addFileDependency{#1.tex}%
%	\addFileDependency{#1.aux}%
%}
%\externaldocument{supplementaryNoComment}
%\journal{XXX}
%\setcitestyle{authoryear,open={((},close={))}}
%\usepackage{natbib}
%\bibliographystyle{abbrvnat}
%\setcitestyle{authoryear,open={((},close={))}}
%TC:incbib
%command for Chloe comments
\newcommand{\el}[1]{{\color{blue} \textit{#1}}}
\newcommand{\ca}[1]{{\color{magenta} \textit{#1}}}




\title{An individual-based model to explore the impact of psychological stress on immune infiltration into tumour spheroids}
\author[1,2]{Emma Leschiera\thanks{Corresponding author. Email: emma.leschiera@devinci.fr}}

\author[3]{Gheed Al-Hity}
%\ead{gheed.alhity@kcl.ac.uk}
\author[3]{Melanie S. Flint}

\author[4]{Chandrasekhar Venkataraman}
%\ead{luis.almeida@sorbonne-universite.fr}


%\ead{chloe.audebert@sorbonne-universite.fr}


%\ead{m.flint@brighton.ac.uk}


%\ead{emma.leschiera@inria.fr}


\author[5]{Tommaso Lorenzi} %\ead{tommaso.lorenzi@polito.it}
\author[6]{Luis Almeida}
\author[6,7]{Chloe Audebert}
%\ead{c.venkataraman@sussex.ac.uk}
\affil[1]{\small Léonard de Vinci Pôle Universitaire, Research Center, 92 916 Paris La Défense, France}
\affil[2]{\small Univ. Bordeaux, CNRS, INRIA, Bordeaux INP, IMB, UMR 5251, F-33400 Talence, France}
\affil[3]{School of Applied Sciences, University of Brighton, Centre for Stress and Age-related Diseases, Moulsecoomb, Brighton, BN2, 4GJ, UK.}
\affil[4]{School of Mathematical and Physical Sciences, University of Sussex, Department of Mathematics, Falmer, Brighton, BN1 9QH, UK.}

\affil[5]{Department of Mathematical Sciences ``G. L. Lagrange'', Politecnico di Torino, 10129 Torino, Italy.}
\affil[6]{Sorbonne Universit\'e, CNRS, Universit\'e de Paris, Laboratoire Jacques-Louis Lions UMR 7598, 75005 Paris, France.}
\affil[7]{Sorbonne Universit\'e, CNRS, Institut de biologie Paris-Seine (IBPS), Laboratoire de Biologie Computationnelle et Quantitative UMR 7238, 75005 Paris, France.}

\renewcommand\Authands{ and }

\date{\vspace{-5ex}}

%\author{ {\sc Lu\'is Almeida$^1$, Chlo\'e Audebert$^{1,2}$, Emma Leschiera$^1$, Tommaso Lorenzi$^3$ and Shensi Shen$^4$}~\\[3pt]
	%$^1$Sorbonne Universit\'e, CNRS, Universit\'e de Paris, Inria, Laboratoire Jacques-Louis Lions UMR 7598, 75005 Paris, France.\\[3pt]
	%$^2$Sorbonne Universit\'e, CNRS, Institut de biologie Paris-Seine (IBPS), Laboratoire de Biologie Computationnelle et Quantitative UMR 7238, 75005 Paris, France.\\[3pt]
	%$^3$Department of Mathematical Sciences ``G. L. Lagrange'', Dipartimento di Eccellenza 2018-2022, Politecnico di Torino, 10129 Torino, Italy. \\[3pt]
	%$^4$Singapore-Sichuan Frontier Medical Center, West China Hospital Institute of Thoracic Cancer. \\[3pt]
	\begin{document}
		%\begin{frontmatter}
		\maketitle
		\begin{abstract}
			In recent \textit{in vitro} experiments on co-culture between breast tumour spheroids and
			activated immune cells, it was observed that the introduction of the stress hormone cortisol resulted in a decreased immune
			cell infiltration into the spheroids. Moreover, the presence of cortisol deregulated the normal levels of the pro- and anti-inflammatory cytokines
			IFN-$\gamma$ and IL-10.
			%In this work, we present a mathematical framework to describe the interactions between tumour cells and CTLs, and investigate how psychological  stress may affect immune infiltration. 
			We present an individual-based model to explore the interaction dynamics between tumour and immune cells under psychological  stress conditions. With our model, we explore the processes underlying the emergence of different levels of immune infiltration, with particular focus on the biological mechanisms regulated by IFN-$\gamma$ and IL-10. %In our framework, we assume thaCTLs are exposed to psychological stress, which in turn deregulates the normal levels of IFN-$\gamma$ and IL-10 in the tumour micro-environment. 
			%Using this model, we are able to qualitatively reproduce the results of the \textit{in vitro} experiments, as well as investigate which biological mechanisms are more likely to be affected by stress and impact immune infiltration. 
			%Computational simulations of our model indicate that the motility of T cells, their capability to infiltrate through tumour cells, their growth rate, and the interplay between these mechanisms can affect tumour-immune dynamics and the level of infiltration of CTLs into the tumour. 
			The set-up of numerical simulations is defined to mimic the scenarios considered in the experimental study.  Similarly to the experimental quantitative analysis, we compute a score that quantifies the level of immune cell infiltration into the tumour. %The infiltration score obtained through numerical simulations is in good quantitative agreement with the experimental one.
			%Here, we present a spatially explicit stochastic individual-based model that takes into account ITH, and effectively captures the way it affects the anti-tumour immune response. 
			%Moreover, the ability of the model to qualitatively reproduce experimental observations of the study is demonstrated.  
			The results of numerical simulations indicate that the motility of immune cells, their capability to infiltrate through tumour cells, their growth rate and the interplay between these cell parameters can affect the level of immune cell infiltration in different ways. %On the basis of an algorithm developed in the biological study, we define a score to quantify the effects of psychological stresson immune infiltration in a controlled manner. 
			Ultimately, numerical simulations of this model
			support a deeper understanding  of the impact of biological stress-induced mechanisms on immune infiltration.
		\end{abstract}
		
		\begin{comment}
			\begin{highlights}
				\item A novel spatial individual-based model describing interactions between tumour cells and cytotoxic T cells.
				\item The model takes into account the effects of pro-inflammatory and anti-inflammatory cytokines.
				\item  The motility of T cells, their capability to infiltrate through
				tumour cells, their growth rate, and the interplay between these mechanisms can affect the level of infiltration of CTLs into the tumour.
				\item Two- and three dimensional numerical simulations show a good qualitative match with experimental results.
			\end{highlights}
		\end{comment}
		
		\textit{Keywords: }
		Numerical simulations; Immune infiltration; Psychological stress; Individual-based models; Tumour-Immune interactions
		
		
		
		
		%\linenumbers
		
		\section{Introduction} \label{Introduction}
		The ability of psychological stress to induce immune suppression is widely recognised \citep{coe2007psychosocial,morey2015current,seiler2020impact}, but the mechanisms underlying the effects of psychological stress on the adaptive immune response during tumour progression are not completely understood. 
		
		There has been increasing interest in detailing the mechanistic role that psychological stress may play in the context of initiation and progression of cancer. In particular, it has been reported that psychological stress positively influences carcinogenesis through mechanisms that promote proliferation, angiogenesis and metastasis, as well as mechanisms that protect tumour cells from apoptosis ~\citep{lee2009surgical,nilsson2007stress}. The negative role played by psychological stress on the immune system has also been documented. %In breast cancer, stress leads to a transient increase immunosuppressive macrophage populations and may promote breast cancer progression through macrophage M2 polarization (Qin et al., 2015). However, the \textit{in vivo} mechanisms by which stress alters immune-effector (e.g., CD8+) and immune-suppressive CTLs (e.g., Tregs), and thereby disease progression and recurrence, remain elusive.
		Using a pre-clinical mouse model, in~\citep{budiu2017restraint} the authors have shown that psychological stress has a negative impact on T cell numbers and activation, as evidenced by a decrease in the numbers of CD8+ and CD3+CD69+ T cells.  %The administration of beta-blockers made it possible to reverse the negative effect of  stress on the adaptive immune system and slowed down tumour progression. 
		
		In~\citep{al2021integrated}, the authors developed a 3D \textit{in vitro} model to explore the effects of
		the stress hormone cortisol on
		immune cell infiltration into tumour spheroids. %Prior to testing the effect of cortisol on the \textit{in vitro} co-culture model, the authors verified that this model was able to reproduce the outcomes of a 66CL4 syngeneic \textit{in vivo} mouse model. 
		Using two independent image-based algorithms, they quantified the effects of cortisol on immune infiltration, which was assessed by counting the number of immune cells within the
		tumour spheroid boundary. %, demonstrating their qualitatively, and quantitatively similar ability to measure trafficking. 
		The results from
		this model recapitulated the conclusions of~\citep{budiu2017restraint}, by showing that cortisol triggered a reduction in immune infiltration levels. 
		
		The mixture of cytokines produced in the tumour-microenvironment plays a key role in tumour progression~\citep{dranoff2004cytokines}. Pro-inflammatory cytokines that are released in response to infection can inhibit tumour development and progression. Alternatively, tumour cells can produce anti-inflammatory cytokines that promote growth, attenuate apoptosis and facilitate metastasis. %In stressed conditions, the production of pro- and anti-inflammatory cytokines can be altered. For example, in~\citep{budiu2017restraint} the authors found that stress increased the levels of inhibitory cytokines, such as the pro-tumourigenic chemokine CXCL10 and granzyme B, contributing to suppression of anti-tumour immune response.
		In the experiments reported in~\citep{al2021integrated}, cortisol downregulated  IFN-$\gamma$ and upregulated IL-10. 
		IFN-$\gamma$ is a pro-inflammatory cytokine that stimulates immune response, through T cell trafficking in the tumour-microenvironment and infiltration~\citep{castro2018interferon,schiltz2002effects}, whereas IL-10 is an anti-inflammatory cytokine that inhibits immune response by reducing T cell proliferation~\citep{alhakeem2018chronic,couper200810}. %A more detailed understanding of cytokine–tumour-cell-immune cell interactions provides new opportunities for improving cancer immunotherapy. %Finally, the administration of three glucocorticoid receptor antagonists reversed the effects of cortisol and significantly increased immune infiltration.
		
		From a biological and medical perspective, it is difficult to investigate the connection between psychological stress, immune infiltration and the underlying molecular and cellular processes. The challenge lies in
		integrating theoretical and empirical knowledge to achieve a deeper understanding of the mechanisms
		and factors that contribute to inhibition of the anti-tumour immune response. In this context, mathematical models provide easy and cheap tools towards identifying dependencies between different biological phenomena and how
		these may affect the efficacy of the immune response on much shorter timescales  than laborious and expensive experiments. %A number of mathematical models have been developed to investigate the dynamics of tumour development under the action of adaptive immune response. 
		Different aspects of the interaction dynamics between immune and tumour cells have been studied using different deterministic continuum models formulated as ordinary differential
		equations (ODEs)~\citep{de2003mathematical,kirschner1998modeling,kuznetsov1994nonlinear,luksza2017neoantigen}, integro-differential equations (IDEs)~\citep{delitala2013recognition,kolev2013numerical,lorenzi2015mathematical} and partial differential equations (PDEs)~\citep{atsou2020size, atsou2021size,matzavinos2004travelling,matzavinos2004mathematical}. These models usually describe the evolution of tumour and immune cell densities that depend on one or more independent variables, usually time and/or space. Such models are defined on the
		basis of cell population-level phenomenological assumptions, which may limit the amount of biological detail incorporated into the model. By using computational models, such as cellular-automaton (CA) models~\citep{bouchnita2017hybrid,ghaffarizadeh2018physicell}, hybrid PDE-CA models~\citep{almeida2022hybrid,leschiera2021mathematical} and individual-based models (IBMs)~\citep{christophe2015biased,kather2017silico,macfarlane2018modelling, macfarlane2019stochastic}, a mathematical representation
		of biological phenomena that are challenging to include in purely continuum models can be achieved. In fact, these models can be posed on a spatial domain, where cells spatially interact with each other according to a defined set of rules, which can collectively generate global emergent behaviours of tumour-immune cell competition. 
		
		In~\citep{leschiera2021mathematical}, we proposed an IBM that describes the earliest stages of tumour-immune competition. In this model, we included cytotoxic T lymphocytes (CTLs) and tumour cells, which interact in a two-dimensional domain under a set of rules describing cell division, migration via chemotaxis, cytotoxic killing of tumour cells by CTLs and immune evasion. %The model takes into account antigen expression and presentation processes and the ability of T cells to target specific antigens. 
		However, the model in~\citep{leschiera2021mathematical} does not consider the role played by psychological stress
		in immune infiltration and the influence of pro- and anti-inflammatory cytokines on tumour progression.   These aspects are addressed in the present work.
		
		In light of these considerations, and motivated by \textit{in vitro} experimental observations in co-culture between cancer
		spheroids and immune cells~\citep{al2021integrated}, in this paper we develop an IBM to study the effect of psychological stress on immune infiltration. %The mathematical framework presented here is developed in collaboration with the biologists who designed and conducted the \textit{in vitro} experiment presented in~\citep{al2021integrated}. 
		%Compared to \textit{in vivo} studies, which are performed on a whole living organism, \textit{in vitro} experiments are performed in a controlled environment, allowing for a more detailed examination of biological effects. For this reason, mathematical models can be more easily calibrated on \textit{in vitro} studies.
		The model builds on our previous work~\citep{leschiera2021mathematical} and is calibrated to qualitatively reproduce, \textit{in silico}, the experimental results presented in~\citep{al2021integrated}.  As mentioned earlier, in this study the authors found that the introduction of cortisol in the co-culture resulted in a decrease in immune cell infiltration into tumour spheroids, as well as in the alteration of IFN-$\gamma$ and IL-10 levels. In our model,
		we assume that cells are exposed to psychological stress, and that this deregulates IFN-$\gamma$ and IL-10 levels. We explore the processes underlying the emergence of different levels of immune infiltration, with particular focus on biological mechanisms regulated by IFN-$\gamma$ and IL-10.  
		%Using a method similar to the one employed in~\citep{al2021integrated}, we define a score to quantify the effects of psychological stresson immune infiltration in a controlled manner. The results of numerical simulations of this model are able to qualitatively reproduce the results of \textit{in vitro} experiments presented in ~\citep{al2021integrated}, and contribute to a better understanding  of the impact of biological stress-induced mechanisms on immune infiltration.
		Based on one of the two image-based algorithms developed in~\citep{al2021integrated} to quantify immune infiltration, in our numerical simulations we compute a score to quantify the effects of psychological stress on immune infiltration.
		% This model is able to qualitatively reproduce the \textit{in vitro} experiments presented in ~\citep{al2021integrated}. A better understanding of the impacts of biological stress-induced mechanisms on immune infiltration in tumour micro-environment is reached.
		
		
		
		%The paper is organised as follows. In Sec.~\ref{Description of the experiments}, we summarise the \textit{in vitro} experiments and the main experimental observations that motivated our modelling assumptions.  The IBM is presented in Sec.~\ref{The mathematical model}.   Results of preliminary computational simulations are summarised in Sec.~\ref{ch4:Numerical simulations and preliminary results}. Full details of model implementation and model parameterisation are provided in Sup.Mat.\ref{Details of computational model} and Sup.Mat.\ref{appendix5:graph}, respectively. In Sec.~\ref{ch4:Main results}, we present the main computational results of our study and we discuss them in view of the experimental results presented in~\citep{al2021integrated}. Sec. \ref{ch4:Discussion, conclusions and research perspectives} concludes the paper and presents some research perspectives.
		\section{Methods}
		\subsection{Summary of the experimental protocol employed in~\citep{al2021integrated}}
		A summary of the experimental protocol employed in~\citep{al2021integrated} is provided below (further details can be found in~\citep{al2021integrated}). The mathematical model and numerical simulations will then be implemented accordingly, in order to facilitate comparison with the experiments. A schematic summarising the experimental procedure employed in~\citep{al2021integrated} is displayed in Fig.\ref{ch4:experimentalStrategies}.
		
		%\sout{To address the question of the role played by psychological  stress in T cell infiltration, we took advantage of a recently developed 3D co-culture \textit{in vitro} model. This model explored the effect of the gluco-corticoid stress hormone, cortisol on 66CL4 co-culture between cancer spheroids and activated immune cells (splenocytes). A schematic view of the experimental process considered in~\citep{al2021integrated} is given in Fig.\ref{ch4:experimentalStrategies}. We now describe the different steps that have been followed to develop the co-culture model and the main experimental observations that motivated the assumptions of our mathematical model.}
		
		% Figure environment removed
		
		\paragraph{Growth of the spheroids} Spheroids from a murine triple
		negative 66CL4 breast cancer cell line were generated. It took 4 days for the spheroids to fully form, during which their area increased over time. %Spheroids were first grown for 4 days (\textit{i.e.} from day -4 to day 0), during which their area increased over time. %The seeding density over the
		%7 days of experiment duration was optimised . 
		The cell line
		was seeded at different densities. %, in order to attribute the changes in the spheroids to the infiltration of immune cells and not to the spheroids themselves. 
		The seeding density with the
		largest area, and with the least variation over
		4 days of growth post culturing cells into spheroids, was chosen (\textit{cf.} Group-\textbf{1} in Fig.\ref{ch4:experimentalStrategies}). This was done to ensure that the size of the spheroids remained stable  and that changes in the spheroids were due to the infiltration of immune cells. 
		
		\paragraph{Introduction of splenocytes} After full spheroid generation, immune cells (splenocytes), containing activated T lymphocytes, were co-cultured with spheroids. %On day 0, there were 5 times as many immune cells as tumour cells at time -7. 
		Cortisol was added to the co-culture and spheroids were later split into different groups. The groups relevant to our study are: Group-\textbf{1} containing spheroids only, Group-\textbf{2} comprising spheroids and splenocytes, and Group-\textbf{3} containing spheroids, splenocytes and cortisol. %Spheroids from each group were treated daily for 4 days (\textit{i.e.} from day 0 to day 4) and the culture medium was collected for further analysis.
		\paragraph{Trafficking index measuring infiltration levels of immune cells} To test whether cortisol caused a reduction in immune infiltration in the co-culture, each group was imaged daily for 4 days.  From two imaged-based algorithms, two trafficking indices   were computed every day to measure immune cell infiltration into the spheroids \citep{al2021integrated} (\textit{i.e.} to quantify the number of immune cells within the  boundary of the tumour spheroids). Below we report on the implementation of the trafficking index that inspired the development of the score proposed in the present study to measure infiltration levels. Further details can be found in \citep{al2021integrated}.
		
		The \textit{classification-based trafficking index} (TIC) is an algorithm %based on a k-means classifier~\citep{forgy1965cluster}. %\sout{The assumption underlying the classification is that the color of a pixel in a given image characterise the cell type that occupies the region, respectively immune and tumour cells. }The TIC is a two-step approach 
		in which first a machine learning algorithm is used to classify each pixel in the image into different groups obtaining three classes: background, tumour cell or immune cell. 
		The TIC is then based on the number of pixels classified as immune cells that are completely surrounded by pixels classified as tumour cells, divided by the total count of pixels surrounded by tumour cells. The resulting statistic yields a number in the interval $[0,1]$, with a larger TIC indicating a greater level of trafficking.  
		
		\paragraph{Investigating the effects of cortisol on immune infiltration} Over 4 days, the co-cultures in Group-\textbf{2} and Group-\textbf{3} were
		imaged and the TIC was computed from the corresponding images.
		
		It was found that, compared to the group with untreated spheroids and splenocytes only (\textit{i.e.} Group-\textbf{2}), the introduction of cortisol significantly reduced immune cell infiltration into the spheroids in Group-\textbf{3}. %Moreover, the concentration of two cytokines in the culture medium was measured, namely: IFN-$\gamma$ and IL-10. 
		Moreover, cortisol significantly reduced IFN-$\gamma$ levels and increased IL-10 levels.
		\label{Description of the experiments}
		
		\subsection{Modelling framework}
		Building upon our previous work~\citep{leschiera2021mathematical}, to reproduce the \textit{in vitro} results %of the co-culture between tumour spheroids and activated immune cells 
		presented in~\citep{al2021integrated}, we consider two cell types: tumour cells and immune cells. Although activated immune cells have been considered in the experiments, here we focus on CTLs only. We use a Cellular Potts (CP) approach and the CompuCell3D open-source software~\citep{izaguirre2004compucell} to model the interactions between these two cell types.
		
		Adhesive interactions between cells may affect the physical capability of CTLs to infiltrate through tumour cells. In this context, the choice of a CP model is of particular interest, since adhesive interactions between neighboring cells are represented through specific parameters, which describe the net adhesion/repulsion between cell membranes~\citep{izaguirre2004compucell} (see Supp.Mat.\ref{Details of computational model} for a detailed description of the implementation of our CP model). %These parameters may affect the physical capability of CTLs to infiltrate through tumour cells. More details on the role of one of these parameters are given later on.
		
		Moreover, the CompuCell3D software easily allows for the visualisation of the results of numerical simulations. To carry out numerical simulations of the model, we consider both a 2D square spatial domain and a 3D cubic spatial domain. %Our model is posed on a spatial domain partitioned into square elements of side $\Delta x$. ,
		%The 3D domain models the scenario of a 3D \textit{in vitro} co-culture assay, while the 2D domain corresponds to a 2D cross-section of it. 
		%\ca{maybe no need to say it here : Next, exploiting the good quantitative agreement between the results of numerical simulations of the 2D and 3D models, we carry out the numerical simulations of the 2D model only, since they require computational times much smaller than those that would be required by the numerical exploration of the corresponding 3D model.}
		
		At each time step, the states of the cells
		are updated according to the rules described below.
		\paragraph{Growth of tumour cells}
		We denote by $N_T(t)$ the number of tumour cells in the system at time $t$ %=h\Delta t$, with $h\in \mathbb{N_0}$, 
		and we label each cell by an index $n=1,\dots, N_T(t)$.
		
		In the experimental setup described in Sec.~\ref{Description of the experiments}, spheroids are cultured for a sufficient time until they attain a stable size. To reproduce such dynamics, we let tumour cells proliferate until a maximum tumour size is attained, which corresponds to the carrying capacity of the population. With the selected parameter values, this process takes approximately 7 days. 
		
		At the initial time of simulations, we assume a certain number of tumour cells to % be already present and we let them 
		be tightly packed in a circular configuration positioned at the centre of the domain. %reproducing a 2D cross-section of the geometry of the spheroids (in the 2D domain) or the 3D spheroid culture (in the 3D domain). 
		At each time-step, tumour cells grow at a rate drawn from a uniform distribution. %the parameters of the bounds of the uniform distribution are chosen to match the mean duration of the spheroids cycle length\ca{it make sens to talk about spheroids cycle?}. 
		Mitosis occurs when tumour cells grow to a critical size and then
		divide. We refer the reader to our previous paper~\citep{leschiera2021mathematical} for a detailed description of the modelling strategy employed to represent cell division. 
		
		Tumour cells can
		die due to intra-population competition (\textit{i.e.} competition between tumour cells for limited space and resources), at a rate proportional to the total number of tumour cells. If tumour cells exhaust their lifespan (which is drawn from a uniform distribution when cells are
		created) then they die. Dead tumour cells are removed from the domain. %This modelling rule allows us to obtain a logistic growth in the number of tumour cells.  Following the experiments, the rate of tumour cell death due to intra-tumour competition is chosen so that the carrying capacity, \textit{i.e.} the saturation value reached by the number of tumour cells due to intra-population competition, is reached after 7 days of growth, corresponding to day 0 in the experiments. Note that, following the experiments, we verify that the number of tumour cells in the absence of CTLs remains approximately constant for the 4 days after reaching the saturation value (\textit{i.e.} from day 0 to day 4).
		\paragraph{Introduction of CTLs}
		We denote by $N_C(t)$ the number of CTLs in the system at time $t$, and we label each of them by an index $m= 1,...,N_{C}(t)$.
		
		
		When introduced, CTLs are randomly distributed at the border of the spatial domain. %In agreement with the experiments, the number of CTLs introduced in the domain at day 0 corresponds to 5 times the number of tumour cells introduced at day -7. 
		Once
			in the domain, CTLs grow and divide through mitosis according to rules similar to those used for tumour cells. CTLs die due to intra-population competition (\textit{i.e.} competition between CTLs for limited space and resources), at a rate proportional to the total number of CTLs.  A CTL can also die due to natural death when it reaches the end of its intrinsic lifespan, which is drawn from a uniform distribution when the cell is created.%If a CTL exhausts its lifespan (which is drawn from a uniform distribution when the cell is created) then it dies and is removed from the domain.
		
		We assume that tumour cells at the border of the tumour (the region where cytokines and immune cells are more abundant \citep{boissonnas2007vivo}) secrete a chemoattractant. We assume that this chemoattractant represents  a mixture of different chemokines (e.g. CXCL9/10/11 \citep{galon2019approaches,gorbachev2007cxc}) produced by tumour cells and other cells in the tumour micro-environment (e.g. monocytes, endothelial cells, fibroblasts \citep{tokunaga2018cxcl9}), and that it triggers
			the movement of CTLs towards tumour cells.  A detailed description of the chemoattractant dynamics is given in Sup.Mat.\ref{Details of computational model}. CTLs are assumed to move up the gradient of the chemoattractant towards tumour cells.
		
		According to the experiments, CTLs are activated against tumour cells. Therefore, we suppose that, upon contact, CTLs can induce tumour cell death with a certain probability. We refer to this probability as the “immune success rate”. If tumour cells satisfy the conditions to be eliminated then they
		die. %Building on the modelling strategies developed in~\citep{leschiera2021mathematical}, we assume that an elimination event keeps a CTL engaged for 6 hours and only after this time the CTL can induce another tumour cell to die.\\
		\paragraph{Infiltration score} Similarly to the TIC proposed in~\citep{al2021integrated} to measure immune infiltration, in our work we define an `infiltration score'. This score allows us to quantify the level of CTL infiltration into the tumour. % and is defined in a numerically equivalent way of the TIC algorithm developed in~\citep{al2021integrated}. 
		Provided that there are tumour cells in the domain, we define the infiltration score as the number of CTLs surrounded by tumour cells, divided by the number of tumour cells and CTLs surrounded by tumour cells; that is, 
		\begin{equation}
			I(t) := \frac{\sum_{m=1}^{{N_C(t)}}\delta_{{m\in N_{CS}(t)}}}{\sum_{m=1}^{{N_C(t)}}\delta_{{m\in N_{CS}(t)}}+\sum_{n=1}^{N_T(t)}\delta_{{n\in N_{TS}(t)}}}.
			\label{infiltration score}
		\end{equation}
		In \eqref{infiltration score}, $\delta_{{m\in N_{CS}(t)}}=1$ if $m\in N_{CS}(t)$, and $\delta_{{m\in N_{CS}(t)}}=0$ otherwise, where $N_{CS}(t)$ denotes the set of indices of CTLs surrounded by tumour cells at time $t$.  Function $\delta_{{n\in N_{TS}(t)}}$ is defined in a similar way, where $N_{TS}(t)$ denote the set of indices of tumour cells surrounded by tumour cells at time $t$. In Compucell3D, these terms are handled by using specific functions which track neighbors of every cell (further details can
		be found in Sup.Mat.\ref{Details of computational model}).
		Note that $0\leq I(t) \leq 1$.
		
		\paragraph{Investigating the effects of psychological  stress on immune infiltration}
		Over 4 days, CTLs move via
		chemotaxis towards the tumour and infiltrate into it.
		%Fig.\ref{ch4:modellingStrategies} describes the strategies adopted in our model to study the effects of psychological stresson immune infiltration and qualitatively reproduce the experimental results obtained in~\citep{al2021integrated}.  %In this study, to test whether stress decreased infiltration levels in the co-culture model, cortisol was added to the co-culture and the results were compared to those obtained for the group of spheroids and
		%activated splenocytes only. Cortisol significantly decreased the levels of IFN-$\gamma$ in the
		%co-culture media and significantly increased the levels of IL-10 (\textit{cf.} step \textbf{(1)} of Fig.\ref{ch4:modellingStrategies}). 
		
		In the experiments reported in  \citep{al2021integrated}, the spatio-temporal dynamics of cortisol,  IFN-$\gamma$ and IL-10   are not measured. The measurements  focus on the level of immune infiltration into tumour spheroids, as well as IFN-$\gamma$ and IL-10 amounts after 4 days of co-culture. %However, the experimental results showed that the introduction of cortisol in the co-culture  decreased  immune infiltration into tumour spheroids and altered IFN-$\gamma$ and IL-10 levels.
			Therefore, to reduce the number of parameters and the complexity of the model, we do not explicitely model the dynamics of cortisol, IFN-$\gamma$ or IL-10. Through numerical simulations, we investigate  the effects of three parameters associated to IFN-$\gamma$ and IL-10, which we expect to play a key role in determining the infiltration of CTLs into the tumour. These three parameters are: the secretion rate of the chemoattractant by tumour cells, the
		"tumour cell-CTL adhesion strength" and the growth
		rate of CTLs. Below we detail how these three parameters are associated to IFN-$\gamma$ and IL-10. 
		
		
		It has been shown that IFN-$\gamma$ induces the stimulation of various chemokines (\textit{e.g.} CXCL9/10/11) which drive the chemotactic movement of CTLs towards the tumour \citep{castro2018interferon,galon2019approaches, gorbachev2007cxc}. 
		Therefore, in our study, the role of IFN-$\gamma$ is investigated by varying the secretion rate of the chemoattractant by tumour cells.
		
		
		Moreover, IFN-$\gamma$ induces the expression of cellular adhesion molecules (\textit{e.g.} E-cadherin or ICAM-1), which enhance the infiltration of CTLs into the tumour \citep{harjunpaa2019cell, jorgovanovic2020roles}. Therefore, the role of IFN-$\gamma$ is also investigated by varying the "tumour cell-CTL adhesion strength" (TC-CTL adhesion strength). %In particular, in our model, the parameter modelling the adhesion strength between tumour cells and CTLs can directly affect the physical capability of CTLs to infiltrate through tumour cells. %In particular, a high adhesivity between tumour cells and CTLs leads to scenarios in which CTLs display a high capability to infiltrate through tumour cells, while a low adhesivity between the two cell types lead to scenarios in which CTLs tend to accumulate at the margin of the tumour, without infiltrating it. 
		This parameter refers to the CP parameter associated to the adhesion between tumour cell and CTL membranes. In our model, the TC-CTL adhesion strength regulates CTL ability to infiltrate through tumour cells. In particular, high values of this parameter facilitate the infiltration of CTLs through tumour cells, while low values lead the CTLs to accumulate at the margin of the tumour, without infiltrating into it. More details on the calibration of this parameter are provided in Sup.Mat.\ref{appendix5:graph}.
		
		
		Finally, IL-10 is an immunoregulatory cytokine that can attenuate inflammatory responses by suppressing CTL production and proliferation~\citep{alhakeem2018chronic,couper200810}. 
		Therefore, the effect of IL-10 is investigated by varying the growth rate of CTLs.
		
		%Varying these three parameters affects both the number of CTLs at the end of simulations, as well as their movement towards and within the tumour. 
		In this work, we explore different scenarios. We suppose that in non-stressed conditions IFN-$\gamma$ levels are high, IL-10 levels are low, and CTLs infiltrate into the tumour. In stressed conditions instead, we suppose that IFN-$\gamma$ levels decrease and  IL-10 levels increase, leading to a decreased CTL infiltration. 
		By considering a range of values of these three parameters, we explore their impact on tumour-immune dynamics independently and together, assessing their influence on immune infiltration in a controlled manner.
		\begin{comment}
			
		\end{comment}
		\label{The mathematical model}
		\section{Preliminary results in 2D and 3D}
		\label{ch4:Numerical simulations and preliminary results}
		In this section, the results of preliminary numerical simulations of the model in 2D and 3D are presented, which will be used to guide the simulations leading to the main results presented in Sec.~\ref{ch4:Main results}. 
		
		Given the stochastic nature of the model, all the results we present in this section and in Sec.~\ref{ch4:Main results} are obtained by averaging over $5$ simulations, which were %in order to capture the inherent stochasticity in the model and the Compucell3D software, 
		carried out using the parameter values reported in Tab.\ref{ch4:table1} and Tab.\ref{ch4:table2}. It should be noted that the standard deviation between these 5 simulations is relatively small which leads us not to increase the number of simulations (and their relative computational cost). A lower number of replicates would  not allow to check the robustness of the results. Full details of model implementation and model parameterisation are provided in Sup.Mat.\ref{Details of computational model} and Sup.Mat.\ref{appendix5:graph}, respectively. Files to run a simulation example of the model with Compucell3D \cite{izaguirre2004compucell} are available at \url{https://plmlab.math.cnrs.fr/leschiera/roleofstress}.
		
		
		\subsection{Tumour development in the absence of CTLs}
		\label{ch4:preliminary scenario}
		We first establish a preliminary scenario where tumour cells grow, divide and die according to the rules described in Sec.~\ref{The mathematical model}, in the absence of CTLs.  At the initial time of simulations, 35 tumour cells are placed in the domain. % and part of the parameters related to tumour cells are chosen to qualitatively reproduce the growth of the spheroids obtained in~\citep{al2021integrated} and described in Sec.~\ref{Description of the experiments}.
		More details about the definition of the model initial conditions are given in Sup.Mat.\ref{appendix5:graph}. We carry out numerical simulations for $11$ days (which we count from day -7 to day 4).
		%At the initial time point of the simulation (\textit{i.e.} at day -7 of the experiments), a small number of tumour cells is placed in the domain. 
		The plots in Fig.\ref{ch4:tumour_no_T_cells}\textbf{(a1)-(b1)} show the time evolution of the tumour cell number in 2D and 3D, while Fig.\ref{ch4:tumour_no_T_cells}\textbf{(a2)-(a3)} and Fig.\ref{ch4:tumour_no_T_cells}\textbf{(b2)-(b3)} display samples of initial and final spatial distributions of tumour cells 2D and 3D, respectively. The tumour growth is of logistic type, as expected due to the rules that govern tumour cell division and death. In more detail, as shown by Fig.\ref{ch4:tumour_no_T_cells}\textbf{(a1)-(b1)}, the number of tumour cells increases from day -7 to day 0. Around day 0, it reaches the carrying capacity. %(\textit{i.e.} the saturation value attained due to intra-population competition). 
		The number of tumour cells at carrying capacity is similar to the seeding density chosen in ~\citep{al2021integrated} (\textit{cf.} Fig.1 for 66CL4 spheroids in~\citep{al2021integrated}). From day 0 to day 4, the tumour cell number fluctuates around the carrying capacity. 
		
		These simulations allowed us to calibrate the model parameters related to tumour cells in order to qualitatively reproduce the growth of the spheroids obtained in the experiments. The other simulations were carried out keeping the values of these parameters fixed and equal to those used for these simulations.
		
		% Figure environment removed
		
		
		
		%In the following subsection, we establish a control scenario in which CTLs are introduced into the domain and interact with tumour cells in non-stressed conditions.  
		
		\subsection{Control scenario: CTL infiltration in non-stressed conditions}
		\label{Control scenario: CTL infiltration in non-stressed conditions}
		In the experimental results presented by~\citep{al2021integrated}, in the absence of cortisol, immune cells are able to infiltrate the tumour spheroids. Here we verify the ability of our model to reproduce such dynamics by exploring the infiltration of CTLs into the tumour over $4$ days. For these simulations, the initial number of tumour cells is set at carrying capacity, whereas 
		150 CTLs are introduced in the domain. The values of the parameters related to CTLs are chosen so as to qualitatively reproduce the interaction dynamics between spheroids and immune cells in non-stressed conditions presented in~\citep{al2021integrated}.
		The parameters related to IFN-$\gamma$ and IL-10 levels are set to baseline values (\textit{i.e.} non-stressed conditions). In particular, we let tumour cells secrete the chemoattractant at a high rate, CTLs grow at their normal rate and display a high capability to infiltrate the tumour cell population  (\textit{i.e.} we consider a sufficiently high value for the TC-CTL adhesion strength). In order to gain a deeper understanding of the effects produced by the three aforementioned parameters on immune infiltration, %(\textit{i.e.} the secretion rate of the chemoattractant, the "tumour cell-CTL adhesion strength" and the growth rate of CTLs),
		for the moment we simplify our model by assuming that CTLs are not able to eliminate tumour cells (\textit{i.e.} the immune success rate is set equal to 0). The full model with an immune success rate greater than 0 will be considered in Sec.~\ref{Increasing the immune success rate has an impact on the infiltration score only when the TC-CTL adhesion strength is high}.
		
		
		The plots in Figs.\ref{ch4:baselineTOT}\textbf{(a1)}-\textbf{(b1)} show the time evolution of the number of tumour cells  and CTLs in 2D and 3D, while Figs.\ref{ch4:baselineTOT}\textbf{(a2)}-\textbf{(a3)} and Figs.\ref{ch4:baselineTOT}\textbf{(b2)}-\textbf{(b3)} display samples of initial and final spatial distributions of tumour cells and CTLs in 2D and 3D, respectively.
		The choice of parameter values corresponding to these figures results in the infiltration of CTLs into the tumour.
		The plots in Figs.\ref{ch4:baselineTOT}\textbf{(c)}-\textbf{(d)} display, respectively, the corresponding average value of the infiltration score, computed via~\eqref{infiltration score}, and the average number of infiltrated CTLs over 4 days. Both in 2D and 3D, as soon as CTLs are introduced in the domain, they move towards the tumour and infiltrate it. Fig.\ref{ch4:baselineTOT}\textbf{(c)} indicates that the infiltration score increases over time, both in 2D and 3D. In 2D its value tends to saturate between day 3 and day 4. Moreover, the value of the infiltration score in the 3D setting is larger than in the 2D case. %This is likely to be due to the larger number of tumour cells on the surface of the tumour which are not surrounded by other tumour cells. These cells are not taken into account in the infiltration score defined via~\eqref{infiltration score}. 
		Note that, in 2D, the mean value of the infiltration score obtained at day 4 of simulations is similar to the mean value of the TIC computed in~\citep{al2021integrated}, when cortisol was not introduced in the co-culture (\textit{cf.} Fig.5 in~\citep{al2021integrated}). 
		Fig.\ref{ch4:baselineTOT}\textbf{(d)} demonstrates that, in 2D, most of the CTLs infiltrate the tumour already at day 1, as the average number of infiltrated CTLs increases only slightly between day 1 and 4. On the other hand, in 3D, CTLs seem to be slightly slower in moving towards the tumour. However, the average number of infiltrated CTLs at the end of numerical simulations is similar in the two settings.  Finally, as shown by Figs.\ref{ch4:baselineTOT}\textbf{(a1)-(b1)}, and as expected on the basis of the rules that govern tumour cell and CTL growth and death, both in 2D and 3D, over time the number of tumour cells fluctuates around the carrying capacity, while CTL number increases until it reaches a saturation value. This result indicates that the changes in the tumour surface and volume observed in Fig.\ref{ch4:baselineTOT}\textbf{(a3)} and Fig.\ref{ch4:baselineTOT}\textbf{(b3)} are due to the infiltration of CTLs into the tumour.
		% Figure environment removed
		
		
		%\subsection{Investigation of the effects of psychological stresson immune infiltration}
		
		
		\section{Main results}
		\label{ch4:Main results}
		%In the previous section we presented the results of preliminary simulations corresponding to scenarios of tumour growth and tumour-immune interaction in non-stressed conditions. 
		In this section we explore the effects of psychological stress on immune infiltration. %In particular, following the experimental results presented in~\citep{al2021integrated}, we study the impact that IFN-$\gamma$ and IL-10 may have on immune infiltration.  %In~\citep{al2021integrated}, the authors demonstrated that, under stress, the levels of IFN-$\gamma$ in the co-culture media decreased, while the levels of IL-10 increased. 
		To do so, first we decrease the secretion rate of the chemoattractant and the TC-CTL adhesion strength. These two parameters are associated with decreased levels of IFN-$\gamma$. Next, for each scenario considered, we decrease the growth rate of CTLs, which is associated with increased levels of IL-10. The initial number and position of tumour cells and CTLs is kept equal to that used in the control scenario.
		
		%Exploiting the good agreement between the results of 2D and 3D simulations presented in the previous subsections, 
		Conducting baseline numerical simulations in 3D provided valuable insights into potential disparities between calculating the infiltration score on 2D and 3D images. Nevertheless, since in the experiments reported in \citep{al2021integrated} the infiltration score is computed on 2D images, we now carry out 2D simulations only, also because they require much less computational time than the corresponding 3D simulations.
		
		
		In this section, we report on results obtained by varying the values of the three aforementioned parameters %(\textit{i.e.} the secretion rate of the chemoattractant, the TC-CTL adhesion strength and the growth rate of CTLs), 
		while the other parameters are kept equal to the values used in the previous section. For each scenario, the infiltration score is computed via \eqref{infiltration score}. % and listed in Tab.\ref{ch4:table1} and Tab.\ref{ch4:table2}.
		%The obtained cell dynamics are compared to those observed in the control scenario summarised in Sec. \ref{Control scenario: CTL infiltration in non-stressed conditions}. %The values of the parameters are chosen so as to qualitatively reproduce essential aspects of the experimental results obtained by~\citep{al2021integrated}.
		
		\subsection{Decreasing the secretion rate of the chemoattractant and the TC-CTL adhesion strength reduces the infiltration of CTLs into the tumour}
		\label{Decreasing the secretion rate of the chemoattractant reduces the infiltration of CTLs into the tumour}
		To investigate how immune infiltration is affected by IFN-$\gamma$ levels in the domain, we start
		by comparing the control scenario of Sec. \ref{Control scenario: CTL infiltration in non-stressed conditions} with  scenarios  in which the values of the secretion rate of the chemoattractant and of the TC-CTL adhesion strength are reduced (\textit{cf.} Tab.\ref{ch4:table1} and Tab.\ref{ch4:table2}). 
		
		
		Fig.\ref{ch4:mainres1}\textbf{(a)} displays the average value of the infiltration score at different times of the simulations, for high, intermediate and low values of the secretion rate of the chemoattractant and the TC-CTL adhesion strength. %Moreover, Figure ~\ref{ch4:mainres1}\textbf{(b)}-\textbf{(f)} show the final spatial distributions of tumour cells and CTLs for two decreasing values of the level of CTL infiltration [Fig.\ref{ch4:mainres1}\textbf{(c)}-\textbf{(d)}] and two decreasing values of the secretion rate of the chemoattractant [Fig.\ref{ch4:mainres1}\textbf{(e)}-\textbf{(f)}], and compare them to the spatial distribution of tumour cells and CTLs obtained in the control scenario [Fig.\ref{ch4:mainres1}\textbf{(b)}]. 
		This figure shows that both parameters affect the infiltration of CTLs into the tumour, as the infiltration score decreases as soon as one of the two parameters is reduced. In addition, when the
		TC-CTL adhesion strength is sufficiently high, decreasing the secretion rate of the chemoattractant considerably reduces the infiltration score. On the other hand, for sufficiently  low values of the
		TC-CTL adhesion strength, decreasing the secretion rate of the chemoattractant does not have an impact on the infiltration score, as its value is already small. Taken together, these results suggest that the secretion rate of the chemoattractant has an impact on T cell infiltration only when CTLs display a sufficiently high capability to infiltrate through tumour cells. %On the other hand, if CTLs have a low capability to infiltrate through tumour cells, independently on the dynamics of the chemoattractant, they will not be    able infiltrate into the tumour.
		
		Then, we analyse the spatial cell distributions observed at the end of simulations. Figs.\ref{ch4:mainres1}\textbf{(c)}-\textbf{(d)} show samples of the final spatial distributions of tumour cells and CTLs for intermediate and low values of the TC-CTL adhesion strength. Figs.\ref{ch4:mainres1}\textbf{(e)}-\textbf{(f)} show similar plots for intermediate and low values of the secretion rate of the chemoattractant. 
		These plots are to be compared with the one in Fig.\ref{ch4:mainres1}\textbf{(b)}, which displays the final spatial distributions of tumour cells and CTLs obtained in the control scenario. In particular, Figs.\ref{ch4:mainres1}\textbf{(b)}-\textbf{(d)} show that decreasing  the TC-CTL adhesion strength leads to scenarios in which CTLs accumulate around the tumour, because  the secretion rate of the chemoattractant is high, but they do not infiltrate into it. The calculation of the infiltration score defined via \eqref{infiltration score} only takes into account CTLs infiltrated into the tumour but not the ones surrounding it. Therefore, the infiltration score decreases. On the other hand, Figs.\ref{ch4:mainres1}\textbf{(b)}-\textbf{(e)}-\textbf{(f)} indicate that decreasing the secretion rate of the chemoattractant leads to scenarios in which CTLs away from the tumour are not sensitive to the gradient of the chemoattractant and, therefore, do not move towards the tumour. The more CTLs are not sensitive to the chemoattractant and do not infiltrate the tumour, the more the infiltration score decreases.
		% Figure environment removed
		
		Taken together, these results qualitatively reproduce key experimental findings presented in~\citep{al2021integrated}, which indicated that cortisol reduced IFN-$\gamma$ levels and led also immune infiltration to reduce. The modelling assumption underlying these computational results may provide the following theoretical explanation for such behaviour. Since IFN-$\gamma$ may affect both CTL movement and infiltration capabilities, deregulation of IFN-$\gamma$ levels inhibits CTL ability to migrate towards the tumour and to infiltrate through tumour cells. The interplay between these mechanisms results in a progressive reduction of immune infiltration levels.
		
		\subsection{Decreasing the growth rate of CTLs reduces the number of infiltrated CTLs}
		We further investigate the effects of psychological stress on immune infiltration by exploring the role played by IL-10. For these simulations, we consider the same parameter values used in the previous subsection but we reduce the value of the growth rate of CTLs (\textit{cf.} Tab.\ref{ch4:table1} and Tab.\ref{ch4:table2}).
		
		Figs.\ref{ch4:mainres2}\textbf{(a)}-\textbf{(b)} show a comparison between the infiltration score obtained when the effects of IL-10 are not considered (\textit{i.e.} for a normal value of the CTL growth rate), and the one obtained when the effects of IL-10 are considered (\textit{i.e.} when the CTL growth rate is reduced). Figs.\ref{ch4:mainres2}\textbf{(c)}-\textbf{(d)} also compare the number of tumour cells and CTLs at the end of numerical simulations (\textit{i.e.} at day 4 of the experiments) for the two scenarios considered. 
		Comparing the results of Fig.\ref{ch4:mainres2}\textbf{(a)} with those displayed in Fig.\ref{ch4:mainres2}\textbf{(b)}, we see that, similarly to the results observed in the previous subsection, decreasing the growth rate of CTLs reduces the infiltration score only when the TC-CTL adhesion strength is sufficiently high. %Moreover, for intermediate values of the TC-CTL adhesion strength, decreasing  the growth rate of CTLs slightly reduces the infiltration score. 
		However, when  the TC-CTL adhesion strength is sufficiently low, decreasing the growth rate of CTLs does not have an impact on the infiltration score, as its value is already small.  As shown by Figs.\ref{ch4:mainres2}\textbf{(c)}-\textbf{(d)}, decreasing  the growth rate of CTLs leads to a decreased number of CTLs at the end of simulations, while the final number of tumour cells remains similar in the two scenarios.
		
		If we assume that high levels of IL-10 inhibit CTL growth, the outputs of our model indicate that, as expected, decreasing the proliferation rate of CTLs diminishes the number of CTLs in the domain. Moreover, if CTLs display a sufficiently high capability to infiltrate through tumour cells, we observe a reduction in the number of infiltrated CTLs (\textit{i.e.} the infiltration score decreases). On the other hand, if CTLs have a low capability to infiltrate through tumour cells, decreasing the proliferation rate of CTLs does not affect the infiltration score, as the number of infiltrated CTLs is already low. This suggests that high levels of IL-10 decrease immune infiltration only when CTLs display a sufficiently high capability to infiltrate through tumour cells, that is, when  IFN-$\gamma$ levels are sufficiently high.
		
		% Figure environment removed
		\subsection{Relationship between IFN-$\gamma$ and IL-10 levels and infiltration score}
		The results presented in the previous subsections summarise how scenarios corresponding to different levels of CTL infiltration into the tumour can emerge under sample combinations of the values of the secretion rate of the chemoattractant, the TC-CTL adhesion strength and the CTL growth rate.  We now undertake a more comprehensive investigation of the relationship between these parameters and the infiltration score.
			
			In order to do this, we perform numerical simulations holding all parameters constant but considering a broader range of values of the secretion rate of the chemoattractant by tumour cells, the tumour cell-CTL adhesion strength and the growth rate of CTLs.  For each pair of parameters,  the third parameter is set to its baseline value. For each scenario considered,  we determine the final value of the infiltration score computed via \eqref{infiltration score}. The results obtained are summarised in the heat maps of Fig.\ref{ch4:heatmap}. 
		
		As shown by the green regions on the bottom side of the two heat maps of Fig.\ref{ch4:heatmap}\textbf{(a)-(b)}, for sufficiently small values of the tumour cell-CTL adhesion strength, the immunoscore is relatively low (<0.1) independently of the value of the chemoattractant secretion rate and the CTL growth rate. This is due to the fact that, independently of their number or their sensitivity to the chemoattractant, CTLs accumulate around the tumour, but they might not infiltrate into it. On the other hand, when low values of the chemoattractant secretion rate  and the growth rate of CTLs are considered, but the baseline (\textit{i.e.} high) value of the tumour cell-CTL adhesion strength is considered (\textit{cf.} Fig.\ref{ch4:heatmap}\textbf{(c)}), the infiltration score is larger than in the two previous scenarios. 
		
		The orange-red regions of the heat maps of Fig.\ref{ch4:heatmap} indicate that there are several possible parameter ranges giving rise to an intermediate infiltration score (between 0.1 and 0.2). The first and second ones correspond to intermediate values of the tumour cell-CTL adhesion strength along with either intermediate to low  values of the chemoattractant secretion rate (\textit{cf.} Fig.\ref{ch4:heatmap}\textbf{(a)}) or normal to low  values of the CTL growth rate (\textit{cf.} Fig.\ref{ch4:heatmap}\textbf{(b)}). The third parameter range corresponds to low to normal values of the CTL growth rate along with intermediate to low values of the chemoattractant secretion rate (\textit{cf.} Fig.\ref{ch4:heatmap}\textbf{(c)}). 
		
		Finally, as shown by the light blue regions on the top-left side of Fig.\ref{ch4:heatmap}\textbf{(a)-(b)}, for high values of the three parameters, which correspond to the baseline parameters chosen in the control scenario, the value of the immunoscore is relatively high (>0.2). Fig.\ref{ch4:heatmap}{\textbf{(c)} shows that a relatively high immunoscore can be obtained also for lower values of the CTL growth rate (resp. chemoattractant secretion rate), provided that the chemoattractant secretion rate (resp. CTL growth rate) is large enough. Moreover, these results also show that increasing the growth rate of CTLs or the chemoattractant secretion rate to values higher than those considered in the control scenario does not always increase the infiltration score. 
			% Figure environment removed
			\subsection{Increasing the immune success rate has an impact on the infiltration score only when the TC-CTL adhesion strength is sufficiently large}
			\label{Increasing the immune success rate has an impact on the infiltration score only when the TC-CTL adhesion strength is high}
			So far, we have investigated with our model the effects of psychological stress on immune infiltration in the case where CTLs are not able to eliminate tumour cells.  However, in~\citep{al2021integrated} is reported that immune cells are activated against the spheroids, although the cytotoxic effect of immune cells on tumour cells is not particularly pronounced. %Therefore, it is reasonable to assume that, upon contact, the CTLs can eliminate the tumour cells. However, in the results obtained in~\citep{al2021integrated}, the cytotoxic effect of immune cells is not particularly observed. 
			Motivated by these considerations, now we investigate tumour-immune dynamics and the effects of psychological stress on immune infiltration in the case where CTLs can eliminate tumour cells with a small probability.
			
			%Figs.\ref{ch4:mainres3}\textbf{(a)}-\textbf{(b)} compare the infiltration score obtained with the parameter values considered in Sec.~\ref{Decreasing the secretion rate of the chemoattractant reduces the infiltration of CTLs into the tumour}, with the \ca{one} \sout{infiltration score} obtained when the same parameter values are considered and CTLs can eliminate tumour cells with a small probability (\textit{i.e.} we only increase the value of the immune success rate).
			Figs.\ref{ch4:mainres3}\textbf{(a)}-\textbf{(b)} show a comparison between the infiltration score obtained with the parameter values considered in Sec.~\ref{Decreasing the secretion rate of the chemoattractant reduces the infiltration of CTLs into the tumour}, assuming that CTLs are able, or not able, to eliminate tumour cells (\textit{i.e} the immune success rate is either zero or different from zero - \textit{cf.} Tab.\ref{ch4:table2}).
			Figs.\ref{ch4:mainres3}\textbf{(c)}-\textbf{(d)} show a comparison between the number of tumour cells and CTLs at the end of simulations (corresponding to day 4 of the experiments) for the two scenarios considered. Comparing the results of Fig.\ref{ch4:mainres3}\textbf{(a)} with those displayed in Fig.\ref{ch4:mainres3}\textbf{(b)}, we see that, when  the TC-CTL adhesion strength is sufficiently high, increasing the immune success rate decreases the infiltration score. This is probably due to the fact that, % when CTLs cannot infiltrate through tumour cells, they almost do not come into contact with tumour cells. Since tumour cells can be killed only if they are in contact with CTLs, this decreases their likelihood of being killed. On the other hand, 
			when CTLs can infiltrate through tumour cells, they are more likely to come into contact with tumour cells, thus increasing the chance for CTLs to eliminate them. Since dead tumour cells are cleared from the domain, this in turn diminishes the number of CTLs surrounded by tumour cells, leading to a reduced infiltration score. 
			However, when the TC-CTL adhesion strength is sufficiently low, increasing the immune success rate does not have an impact on the infiltration score. In fact, in this scenario, CTLs accumulate around the tumour, decreasing their probability to come into contact with tumour cells. This reduces their chance to eliminate tumour cells. Analogous considerations hold for the case in which lower growth rates of CTLs  are considered (results not shown).
			
			As shown by Figs.\ref{ch4:mainres3}\textbf{(c)}-\textbf{(d)}, increasing the immune success rate leads to a slightly decreased number of tumour cells at the end of simulations only when sufficiently large values of
			the TC-CTL adhesion strength and the secretion rate of the chemoattractant are considered. On the other hand, for intermediate and sufficiently small values of these two parameters, increasing the immune success rate does
			not have an impact on the final number of tumour cells. %Finally, independently from the value of the two parameters, the number of CTLs at the end of simulations remains similar in both scenarios. \ca{normal puisque on change pas l'IL-10 non?}
			
			% Figure environment removed
			\begin{comment}
				\section{Extension of the results in 3D}
				\label{Extension of the results in 3D}
				The results presented in the previous sections investigated the effects of psychological stresson immune infiltration in 2D. As previously said, one of the advantages of using software such as Compucell3D is that it allows you to switch from 2D to 3D simulations in a simple way, without considerably changing the numerical code. Therefore, in our next simulations,
				we attempt to verify that our model reproduces similar results in 3D. To this end, we consider the same agents, simulated on a 3D domain, with their behaviour defined by the same set of rules considered previously. Moreover, we consider the same parameter settings used for the numerical simulations in 2D, but we adjust the rate of death of tumour cells and CTLs due to intra-population competition, to obtain a  number of cells at the end of numerical simulations similar to the 2D scenario (see~\ref{appendix5:graph} for further details). The results obtained are summarised by the plots in Figs.\ref{ch4:tumour_no_T_cells_3D} and~\ref{ch4:tumour_T_cells_3D}. Due to the high computational cost in simulating the three dimensional version of our model, all quantities we present in this section are obtained by averaging over the results of 4
				simulations.
				
				The growth of the tumour cells in the absence of CTLs in the 3D setting is illustrated in Fig.\ref{ch4:tumour_no_T_cells_3D},  in which we plot the time evolution of the tumour cell number (Fig.\ref{ch4:tumour_no_T_cells_3D}\textbf{(a)}), as well as an example of the spatial
				distribution of tumour cells at different times of the simulation (Fig.\ref{ch4:tumour_no_T_cells_3D}\textbf{(b)}). The growth of the tumour is similar to that obtained in the 2D setting (\textit{cf.} Fig.\ref{ch4:tumour_no_T_cells}\textbf{(a)}): the number of tumour cells reaches the carrying capacity around day 0 and then remains stable around its value until the end of the simulation.
				
				We then introduce CTLs in the domain, and investigate the effects of psychological stresson immune infiltration in our 3D setting. First, our results are able to reproduce the three dimensional analogue of the control scenario considered in Fig.\ref{ch4:baselineTOT}. As shown by Fig.\ref{ch4:tumour_T_cells_3D}\textbf{(a)}, similarly to the 2D case, over time the number of tumour cells tends to stay constant around the value of carrying capacity,
				while the number of CTLs increases until it reaches a saturation value. Figs.\ref{ch4:tumour_T_cells_3D}\textbf{(b)-(c1)} show that, in non-stressed conditions, CTLs rapidly move towards the tumour and infiltrate it. Interestingly, the value of the infiltration score obtained in the 3D setting is larger than that obtained in 2D. This is probably due to the  larger number of tumour cells on the surface of the tumour and not surrounded by tumour cells, which are therefore not counted in our infiltration score defined in \eqref{infiltration score}. However, Fig.\ref{ch4:tumour_T_cells_3D}\textbf{(c2)} shows that the average number of infiltrated CTLs is similar in the two settings. Finally, when decreasing values of the TC-CTL adhesion strength and the secretion rate of the chemoattractant are considered (\textit{cf.} Fig.\ref{ch4:tumour_T_cells_3D}\textbf{(d1)-(d2)-(e1)-(e2)}, the results obtained are similar to those obtained in Fig.\ref{ch4:mainres1}. In particular, either CTLs away from the tumour are not
				sensitive to the gradient of the chemoattractant and, therefore, do not move towards the tumour (\textit{cf.} Fig.\ref{ch4:tumour_T_cells_3D}\textbf{(d1)}), or CTLs accumulate around the tumour because the secretion rate of the chemoattractant
				is high, but as the TC-CTL adhesion strength is low, they do not infiltrate into it (\textit{cf.} Fig.\ref{ch4:tumour_T_cells_3D}\textbf{(d2)}). Again, in both scenarios the average number of infiltrated CTLs is similar in 2D and 3D. Overall, these results demonstrate that, in the scenarios considered here, which are analogous to those considered for the corresponding two-dimensional model when the effects of IFN-$\gamma$ are investigated, decreasing the secretion rate of the chemoattractant and the TC-CTL adhesion strength reduces the infiltration of
				CTLs into the tumour. We expect to obtain similar results when the effects of IL-10 are considered and the value of the growth rate of CTLs is decreased.
				
			\end{comment}
			
			
			\section{Discussion, conclusions and research perspectives}
			\label{ch4:Discussion, conclusions and research perspectives}
			%\subsection{Discussion and conclusions}
			The \textit{in vitro} co-culture experiments presented in~\citep{al2021integrated} are performed in an isolated and relatively homogeneous environment and involve only a few constituents: tumour spheroids, activated immune cells, culture medium and cortisol. Furthermore, each experiment has clear observables, namely the confocal images of the co-culture, the trafficking indices  and the levels of IFN-$\gamma$ and IL-10, which make these experiments highly suitable to be studied through a mathematical model.
			
			
			In this paper, we have presented an IBM to describe the interaction dynamics between CTLs and tumour cells, to reproduce qualitative aspects presented in \citep{al2021integrated} and evaluate immune cell trafficking into tumour cells under normal and stressed conditions. %The IBM allows an intuitive and flexible description of the system at hand, and can be used as \textit{in silico} laboratories to highlight stylized facts, and uncover mechanisms that underlie emergent features of tumour-immune interactions. 
			%Distinct mathematical strategies were used to model different biological mechanisms, to enable a better understanding of the effect of each mechanism on immune infiltration.
			In particular, on the basis of the experiments presented in~\citep{al2021integrated}, we have investigated in a causal, systematic
			manner the way in which IFN-$\gamma$ and IL-10 may impact on the infiltration of CTLs into tumour cells. %The roles played by  these cytokines have been studied in a causal, systematic manner. We explored different scenarios: first we supposed that, in normal conditions, the levels of
			%IFN-$\gamma$ were high, the levels of IL-10 were low and CTLs had a high capability to infiltrate through tumour cells. Then we supposed that, in stressed conditions, the levels of IFN-$\gamma$ decreased, the levels of IL-10 increased and CTLs had a lower capability to infiltrate through tumour cells. By varying the secretion rate of the chemoattractant, the growth rate of CTLs and the TC-CTL adhesion strength, we studied the impact of psychological stresson immune infiltration. Building an infiltration score on the basis of the TIC algorithm developed in~\citep{al2021integrated}, we quantified the effects of psychological stresson immune infiltration in a controlled manner.
			
			The results of numerical simulations qualitatively reproduce, both in 2D and 3D,
			the growth of the tumour spheroids prior the introduction of immune cells and the tumour-immune dynamics in non-stressed conditions. The tumour growth is of logistic type. %When CTLs are introduced, the number of tumour cells is set at carrying capacity to ensure that changes in the tumour surface and volume are due to the infiltration of CTLs.  
			In the control scenario, \textit{i.e.} the scenario in which the secretion rate of the chemoattractant, the TC-CTL adhesion strength and the CTLs growth rate are set at their baseline values, CTLs are able to infiltrate into the tumour.  %This choice of parameter values reflect the idea that in absence of stress, the level of IFN-$\gamma$ is considered to be high, while the level of IL-10 is considered to be low. 
			
			We then have investigated the effects of psychological stress on immune infiltration. First, the results of our model support the idea that reducing the secretion rate of the chemoattractant and the
			TC-CTL adhesion strength, which are associated to a decrease in IFN-$\gamma$ levels, reduces the infiltration of CTLs into the tumour. These results also suggest that the secretion rate of the chemoattractant is more likely to have an impact on T cell infiltration when CTLs display a sufficiently high capability to infiltrate through tumour cells. 
			We have also studied the effects of psychological stress on immune infiltration by reducing the growth rate of CTLs, which is associated to increased IL-10 levels. Decreasing the growth rate of CTLs reduces the number of CTLs in the domain. This leads to a significant reduction in the infiltration score only when the TC-CTL adhesion strength is sufficiently large. The sensitivity analysis of these three paramaters has allowed us to undertake a more comprehensive investigation of the relationship between them and the value of the infiltration score.
			Finally, we have performed numerical simulations by letting CTLs eliminate tumour cells with a small probability - \textit{i.e.} when the immune success rate is greater than 0. In the scenario in which CTLs are able to infiltrate into the tumour, increasing the immune success rate leads to a reduced infiltration score, as tumour cells in contact with CTLs are eliminated. This in turn leads to a slightly decreased number of tumour cells at the end of simulations. 
			
			%Exploiting the features of Compucell3D and as an illustrative example, we have also reported on the results of numerical simulations of a three-dimensional version of our model (\textit{cf.} Figs.\ref{ch4:tumour_no_T_cells_3D} and~\ref{ch4:tumour_T_cells_3D}). These results demonstrate that key aspects of the dynamics observed in 2D carry through when three spatial dimensions are considered, thus conferring additional robustness to the conclusions of our work.
			
			
			
			In summary, the results of numerical simulations of our model indicate that the interplay between IFN-$\gamma$ and IL-10 plays a key role in determining the effects of psychological stress on immune infiltration reported in \citep{al2021integrated}, as both cytokines contribute to regulate immune infiltration in opposite ways.
			Moreover, our results shed light on the impact of three biological stress-induced mechanisms on immune infiltration. In particular, they support the idea that a high infiltration score can be obtained only when the secretion rate of the chemoattractant and the TC-CTL adhesion strength are large, provided that the growth of CTLs is not inhibited. On the other hand, reducing the value of these parameters can lead to a reduced immune infiltration in different ways. For example, we found that the parameter having the strongest impact on immune infiltration is the TC-CTL adhesion strength, which is associated with the physical capability of CTLs to infiltrate through tumour cells.  In this regard, the development of abnormal structural features that inhibit the ability of CTLs to penetrate tumour sites is a hallmark of cancer progression~\citep{galon2019approaches}. Evidence is emerging that glucocorticoids act on adhesion of immune cells by inhibiting adhesion molecules (integrins and selectins) \citep{cronstein1992mechanism,kalfeist2022impact}. The deregulation of adhesion molecules may act as barriers to T cell migration and infiltration. In this context, the results of this study
			support the idea that new glucocorticoid receptor antagonists should be developed to target cell adhesion molecules in order to enhance immune infiltration.
			
			The results of numerical simulations also support the idea that an efficient anti-tumour immune response can occur only in highly infiltrated tumours. This is a key result because it indicates that therapeutic strategies promoting the infiltration of CTLs into tumours may be a promising approach against cancer. In particular, our findings suggest that a synergistic
			effect can be achieved by combining glucocorticoid receptor antagonists, which facilitate CTL infiltration, with immune checkpoint therapies, which enhance the
			effectiveness of \textit{in situ} anti-tumour immune response~\citep{galon2019approaches}. %This is a key result because it supports the idea that therapeutic strategies promoting the infiltration of CTLs into tumours may be a promising therapy approach in the war against cancer. explanation for the inability of CTLs to penetrate tumour sites could be the presence of physical barriers.
			%In fact, recent experimental observations indicate that a ‘sufficient’ T cell infiltration in tumour sites is critical for the response to immunotherapies. On the basis of this assumption, it can be envisaged that strategies that facilitate the infiltration of CTLs to tumours could overcome tumour resistance to immunotherapiy strategies.%When the TC-CTL adhesion strength is low, independently of the value of the secretion rate of the chemoattractant and the growth rate of CTLs,  we obtain a low infiltration score. On the other hand, when the TC-CTL adhesion strength is high, varying the value of the other two  parameters has a greater impact on immune infiltration, and leads to a different scenarios with different infiltration scores. ciao
			
			%\subsection{Research perspectives}
			%In the future, better quantitative comparison with experiments will allow for systematic
			%choice of parameters and validation of the mechanisms we propose here. 
			The current version of our model can be developed further in several ways. Firstly, due to the high computational cost in simulating the three dimensional version of our model, we carried out 3D simulations only to part of our study. However, by running the simulations on high performance computers, this limitation may be addressed in the future and a larger spectrum of parameter values could be tested. 
			
			We managed to calibrate some parameters of the model (see Tab.\ref{ch4:table1} and Tab.\ref{ch4:table2}) from the literature and define them on the basis of precise biological assumptions. However, there are some parameters (\textit{e.g} parameters related to the dynamics of the chemoattractant, the death rate of tumour and CTLs due to intra-population competition) whose values were simply chosen with an exploratory aim and to qualitatively reproduce essential aspects of the experimental results obtained in~\citep{al2021integrated}. In order to minimise the impact of this limitation on the conclusions of our study, first we selected a baseline set of parameters that allowed to reproduce the growth of the spheroids and CTL infiltration as obtained in~\citep{al2021integrated}. Then, we carried out simulations by keeping all parameter fixed and changing only the values of our three parameters of interest, and then comparing the simulation results so obtained.
			
			To keep the model as simple as possible, we chose to include only mechanisms that were necessary to reproduce part of the experimental results presented in~\citep{al2021integrated}. If experimental measurements were available for cortisol, IFN-$\gamma$ and IL-10, we could calibrate the parameters related to their concentration dynamics, and then update the model in order to explicitly incorporate the dynamical modelling of these quantities  using PDEs.
			Also, our current model does not consider the effects of tumour necrosis and hypoxia or CTL exhaustion. These mechanisms can actively contribute to deregulate the normal levels of pro- and anti-inflammatory cytokines, resulting in more aggressive tumours and impaired immune response~\citep{galon2019approaches,balkwill2009tumour,jiang2015t,wherry2011t}. %However, if appropriate future experimental evidence emerges for this system, such as time series data for proliferation rates or cell density, then this could be used to determine whether an additional mechanism, like stress-induced adaptation of the survival potential, should be added to the basic framework established here. 
			
			From a biological point of view, a natural development of this work would consist in studying the effects of therapeutic strategies which counteract the negative impact of psychological stress on immune infiltration. In fact, in~\citep{al2021integrated} it was found that the administration of glucocorticoid receptor antagonists reversed the effects of cortisol and significantly enhanced immune infiltration in tumour spheroids. The effects of therapeutic strategies could be incorporated into our model by, for example, including a detailed metabolic network at the sub-cellular level that directly influences the dynamics at the cellular level, such as CTL growth and movement. 
			In this regard, we could also investigate the delivery schedule of therapeutic agents (\textit{i.e.} time and dosage) that may make it possible to maximise the number of infiltrated CTLs at the end of the treatment. %Other important biological factors not currently taken into account by the model include cytokines other than IFN-$\gamma$ and IL-10. The effects on immune infiltration of the complex crosstalk between cytokines in the tumour microenvironment under psychological stress could be further studied through suitable extensions of the model. %In its present form, our modelling framework qualitatively reproduces key aspects of the results reported in the experimental study~\citep{al2021integrated}. However, due to the high computational expense in simulating large number of cells, at this stage, the model has not been calibrated using any quantitative type of data. Hence, it cannot be employed to generate predictions that can directly be used in the clinic. By fitting its parameters to a specific type of quantitative clinical data, our model could, in principle, be used to assess different levels of infiltration as potential biomarkers for comparing and predicting outcomes in tumour immunotherapy treatments. Integrating the model with tumour biopsies from patients could offer insight into potential outcomes of treatments. \ca{je retirerais cette partie sur la comparaison quantitative.}
			%\ca{tu pourrais dire que l'impact de d'autre cytokyne pourrait etre étudié si on sait sur quoi elles agissent.}
			
			Despite its relative simplicity, our model provides a novel \textit{in silico} framework to investigate the impact of biological mechanisms linked to psychological stress on immune infiltration, and may be a promising tool to easily and cheaply explore therapeutic strategies designed to increase immune infiltration and improve the overall anti-tumour immune response. %Finally, the effects of changing the spatial domain from 2D to 3D would need to be considered. It is probable that the change of dimension would alter the time it takes to run simulations of our mathematical model. However, in this way, a more realistic scenario, closer to the 3D co-culture model, could be explored.
			\section*{Funding}
			E.L. has received funding from the European Research Council (ERC) under the European Union’s Horizon 2020 research and innovation programme (grant agreement No 740623). T.L. gratefully acknowledges support from the the PRIN 2020 project (No. 2020JLWP23) “Integrated Mathematical Approaches to Socio-Epidemiological Dynamics” (CUP: E15F21005420006). T.L. gratefully acknowledge support of the Institut Henri Poincar\'e (UAR 839 CNRS-Sorbonne Universit\'e), and LabEx CARMIN (ANR-10-LABX-59-01). L.A., T.L. and E.L. gratefully acknowledge support from the CNRS International Research Project ‘Mod\'elisation de la biom\'ecanique cellulaire et tissulaire’ (MOCETIBI).
			%\appendix
	\newpage
	\begin{appendices}	
			\section{Details of the individual-based model}
		\label{Details of computational model}
		The individual-based model (IBM) has been numerically simulated using the multicellular modelling environment CompuCell3D~\citep{izaguirre2004compucell}. This software is an open source solver, which uses a Cellular Potts (CP) model~\citep{graner1992simulation} (also known as Glazier-Graner-Hogeweg model). In CP models, biological cells are treated as discrete entities represented as a set of lattice sites, defined as pixels in 2D (or voxels in 3D), each with characteristic values of area and perimeter (or volume and surface in 3D), and intrinsic motility on a regular lattice. Interaction descriptions and dynamics between cells are modelled by means of the effective energy of the system. This determines many characteristics such as cell size, motility, adhesion strength, and the reaction to gradients of chemotactic fields. During a simulation, each cell will attempt to extend its boundaries, through a series of index-copy attempts, in order to minimise the effective energy. The success of the index copy attempt depends on rules which take into account energy changes. 
		
		Files to run a simulation example of the model with Compucell3D \cite{izaguirre2004compucell} are available at \url{https://plmlab.math.cnrs.fr/leschiera/roleofstress}.
		
		\subsection{Cell types}
		In CP models, cells are uniquely identified with an index $\sigma_i$ on each lattice site $i$, with $i$ a vector of integers occupying lattice site $i$. Each cell in the model has a type $\tau(\sigma_i)$, which determines its properties, and the processes and interactions in which it participates. Note
		that, technically, the extracellular medium is also considered as a cell of type medium. In our model, we define 3 cell types: medium, tumour cell and CTL.
		\subsection{Cellular dynamics}
		The effective energy is the basis for operation of all CP models, including CompuCell3D~\citep{izaguirre2004compucell}, because it determines the interactions between cells (including the extracellular medium). Configurations evolve to minimise the effective energy $H$ of the system, defined in a two-dimensional system as
		
		\begin{equation}
			\footnotesize
			\begin{aligned}
				H&=\underbrace{\sum_{i,j}J(\tau(\sigma_i),\tau(\sigma_j))(1-\delta(\sigma_i,\sigma_j))}_{\text{boundary energy}}+\underbrace{\sum_\sigma\left[\lambda_{area}(\sigma)(a(\sigma)-A_t(\sigma))^2\right]}_{\text{area constraint}}+\underbrace{\sum_\sigma\left[\lambda_{per}(\sigma)(p(\sigma)-P_t(\sigma))^2\right]}_{\text{perimeter constraint}}.
				\label{Hsum}
			\end{aligned}
		\end{equation}
		The most important component of the effective energy equation is the boundary energy, which governs the adhesion of cells. The boundary energy $J(\tau(\sigma_i),\tau(\sigma_j))$ describes the contact energy between two cells $\sigma_i$ and $\sigma_j$ of types $\tau(\sigma_i)$ and $\tau(\sigma_j)$. It is calculated by summing over all neighbouring pixels $i$ and $j$ that form the boundary between two cells. Moreover, $\delta(\sigma_i,\sigma_j)=1$ if $\sigma_i=\sigma_j$, and $\delta(\sigma_i,\sigma_j)=0$ otherwise. Thanks to the term $(1-\delta(\sigma_i,\sigma_j))$, the boundary energy contribution is considered only between lattice sites belonging to two different cells. When considering a two-dimensional domain, the second and third terms represent, respectively, a cell-area and cell-perimeter constraint. In particular, $a(\sigma)$ and $p(\sigma)$ are the surface area and perimeter of the cell $\sigma$, $A_t(\sigma)$ and $P_t(\sigma)$ are the cell’s target surface area and perimeter, and $\lambda_{area}(\sigma)$ and $\lambda_{per}(\sigma)$ are an area and perimeter constraint coefficients. Note that in 3D these two terms  represent, respectively, a cell-volume and cell-surface constraint and, therefore, they might assume a different value. %The final term, $H_{chem}$ models chemotaxis-directed cell motility (see Eq. \eqref{chem} below).\\ 
		
		The cell configuration evolves through lattice-site copy attempts. To begin an index-copy attempt, the algorithm randomly selects a lattice site to be a target pixel $i$, and a neighbouring lattice site to be a source pixel $i^\prime$. If the source and target pixels belong to the same cell (\textit{i.e.} if $\sigma_i =\sigma_{i^\prime}$, they do not need to attempt an lattice-site copy and thus the effective energy will not be calculated. Otherwise, an attempt will be made to switch the target pixel as the source pixel, thereby increasing the surface area of the source cell and decreasing the surface area of the target cell.  \\
		The algorithm computes $\Delta H=H-H^\prime$, with $H$ being the effective energy of the system and $H^\prime$ being the effective energy if the copy occurs. Then, it sets $\sigma_i =\sigma_{i^\prime}$ with probability $P(\sigma_i\rightarrow\sigma_{i^\prime})$ given by
		\begin{equation}
			P(\sigma_i\rightarrow\sigma_{i^\prime})= \begin{cases}\quad 1 \; \;  \quad : \quad \Delta H \le 0 \\
				\exp^{-\frac{\Delta H}{T_m}} \; : \quad \Delta H > 0.
			\end{cases}
			\label{Boltzmann}
		\end{equation}
		The change in effective energy $\Delta H$ provides a measure of the energy cost of such a copy and the parameter~$T_m$ determines the level of stochasticity of accepted copy attempts. The
		unit of simulation time is the Monte Carlo step (MCS).
		\subsection{Subcellular dynamics and chemotaxis}
		In our model we simulate CTL chemotaxis toward tumour cells, defined as the cell motion induced by the presence of a chemical gradient. In CompuCell3D~\citep{izaguirre2004compucell}, chemotaxis is obtained biasing the cell’s motion up or down a field gradient by adding a term $\Delta H_{chem}$ in the calculated effective-energy change $\Delta H $ used in the acceptance function \eqref{Boltzmann}. For a field $c_i$: 
		\begin{equation}
			\Delta H_{chem}=-\lambda_{chem}(\phi_i-\phi_{i^\prime})),
			\label{chem}
		\end{equation}
		where $\phi_i$ is the chemical field at the index-copy target pixel $i$, $\phi_{i^\prime}$ the field at the index-copy source pixel $i^\prime$, and $\lambda_{chem}\geq 0$ the strength of chemotaxis. \\
		The change in concentration of the chemical field $\phi$ is obtained by solving a reaction-diffusion equation of the following general form: 
		\begin{equation}
			\frac{\partial \phi}{\partial t}= D \Delta\phi-\gamma\phi+S  
		\end{equation}
		where $D$, $\gamma$ and $S$ denote the diffusion constant, decay constant and secretion rates of the field, respectively. These three parameters may vary with position and cell-lattice configuration, and thus be a function of cell~$\sigma$ and pixel $i$.
		
		In CompuCell3D, this general form of PDEs can be solved using a number of different PDE solvers. More details about the different PDE solvers can be found in the CompuCell3D Reference Manual.
		
		In the main body of the paper, the dynamic of the concentration of the chemoattractant secreted by tumour cells $\phi$ is governed by the following reaction-diffusion equation:
		\begin{equation}
			\frac{\partial \phi}{\partial t}= D \Delta \phi- \gamma \phi+ \alpha\sum_{n=1}^{N_T(t)}\delta_{{n\in N_{TB}(t)}}.
			\label{inf gamma}
		\end{equation}
		In Eq.~\eqref{inf gamma}, $D$ is the chemoattractant diffusion constant, $\gamma$ is the rate of natural decay and $\alpha$ is the secretion rate. Moreover, $\delta_{{n\in N_{TB}(t)}}=1$ if $n\in N_{TB}(t)$, and $\delta_{{n\in N_{TB}(t)}}=0$ otherwise, where $N_{TB}(t)$ denotes the set of indices of tumour cells in contact with the surrounding medium at time $t$. This terms, handled in Compucell3D by the DiffusionSolverFE, takes into account the fact that only the tumour cells at the border of the tumour secrete the chemoattractant. We complement Eq.~\eqref{inf gamma} with zero-flux boundary conditions and an initial concentration $\phi_{init}$ at time 0 of the experiments (\textit{i.e.} when CTLs are introduced) which is set to be zero
		everywhere in the domain but at the border of the tumour (\textit{cf.} Tab.\ref{ch4:table2}).
		\subsection{Infiltration score}
		%The chemotactic movement of CTLs is modulated by a parameter $\lambda_{chem}>0$, which determines the strength of chemotaxis (further details of the model are provided in~\ref{Details of computational model}).
		Building on the TIC proposed in \cite{al2021integrated} to measure immune infiltration levels, in our model at each time-step we compute the ‘infiltration score’ via Eq.~\eqref{infiltration score}. This score allows us to quantify the level of CTL infiltration
		into the tumour and is defined as the number of CTLs surrounded by tumour cells, divided by the number
		of tumour cells and CTLs surrounded by tumour cells. Below we detail how the infiltration score is implemented in CompuCell3D.   
		
		In  CompuCell3D, the function $get\_cell\_neighbor\_data\_list(cell)$ allows to access a list of each cell neighbors. The neighbour of a cell is defined as an adjacent cell that shares a surface area with the cell in question. 
		In our model, for each cell we loop over all its neighbors and we compute its common surface area with medium, tumour cells and CTLs using the function $neighbor\_list.common\_surface\_area\_by\_type(cell)$. We then assume that $\delta_{{m\in N_{CS}(t)}}=1$ if the common surface area between the $m^{th}$ CTL and CTLs and medium surrounding it is strictly lower than 4 and that $\delta_{{n\in N_{TS}(t)}}=1$ if the common surface area between the $n^{th}$ tumour cell and CTLs and medium surrounding it is strictly lower than 6. More details of these functions can be found in the CompuCell3D manuals available at: \url{https://compucell3d.org/Manuals}.
		\section{Initial conditions and values of model parameters}
		\label{appendix5:graph}
		
		The IBM is based on the mathematical model developed in our previous work~\cite{leschiera2021mathematical}, and has been calibrated to qualitatively reproduce the experimental results presented in~\cite{al2021integrated}. 
		
		\subsection{Set-up of simulations}
		To carry out numerical simulations in Sec.~\ref{ch4:Numerical simulations and preliminary results}, we used a CP approach both on a 2D spatial domain with a total of $400\times 400$ lattice sites and on a 3D spatial domain with a total of $100\times 100 \times 100$ lattice sites. The numerical simulations we present in Sec.~\ref{ch4:Main results} were carried out on the 2D domain only. In both cases, simulations were performed using the software CompuCell3D~\cite{izaguirre2004compucell} on a standard workstation (Intel i7 Processor, 4 cores, 16 GB RAM, macOS 11.2.2). 
		
		%First, we let tumour cells proliferate for 11 days (\textit{i.e.} from day -7 to day 4), verifying that they reach their carrying capacity at day 0. 
		%CTLs are then introduced in the domain at day 0. %At the beginning of simulations there is a total of $400$ tumour cells, tightly packed in a circular configuration positioned at the centre of the domain, reproducing the geometry of a solid tumour.\\
		
		At the initial time point of simulations (\textit{i.e.} on day -7), 35 tumour cells are placed in the centre of the domain (\textit{cf.} Fig.\ref{ch4:tumour_no_T_cells}). First we let tumour cells grow in the absence of CTLs for 11 days, carrying out numerical simulations for $33000$ time-steps. On day 0, the number of tumour cells is set at carrying capacity (\textit{i.e.} 950 cells). This is done to ensure that the changes in the tumour surface and volume are due to the infiltration of CTLs into the tumour. We then randomly introduce 150 CTLs at the border of the domain and we let them grow, move and interact with tumour cells for 4 days, carrying out numerical simulations for $1200$ time-steps (\textit{cf.} Figs.\ref{ch4:baselineTOT}-\ref{ch4:mainres3}). 
		
		In the next subsection we describe the way in which additional components of the model were calibrated leading to the parameter values reported in Tab.\ref{ch4:table1} and Tab.\ref{ch4:table2}, to qualitatively reproduce the behaviour of the experimental results presented in~\cite{al2021integrated}. 
		
		\subsection{Model calibration through parameter exploration} The model is calibrated to qualitatively reproduce the experimental results reported in~\cite{al2021integrated}. 
		Due to computational cost, it was not feasible to start with the actual number of cells present within a cell culture (which can reach the order of magnitude $10^5$) or to simulate the same number of  cells as found in a real tumour. With our model, we instead focused on qualitatively capturing the change in immune infiltration levels while varying a certain set of parameters. 
		
		Some parameters of the model (see Tab.\ref{ch4:table1} and Tab.\ref{ch4:table2}) are estimated from the literature
		and defined on the basis of precise biological assumptions. Other model parameters that could not be based on a literature source, such as the cell death rates due to intra-population competition, are adjusted to qualitatively reproduce the growth of the spheroids and CTLs in non-stressed conditions presented in \cite{al2021integrated}. Finally, there are some parameters, such as the TC-CTL adhesion strength, the immune success
		rate and the secretion rate of the chemoattractant, whose values were simply chosen and varied with an exploratory
		aim to qualitatively reproduce essential aspects of the experimental results obtained in~\cite{al2021integrated}.
		
		\subsection{Parameters in the 2D system}
		In the 2D system, the value of the rate of death due to competition
		between tumour cells is chosen so that the number of tumour cells reaches its carrying capacity after 7 days of proliferation. The number of CTLs introduced in the domain on day 0 and the value of the rate of death due to competition
		between CTLs are chosen so that the value of the infiltration score computed at the end of the 2D simulations in the control scenario is similar to the value of the TIC algorithm obtained on day 4 in~\cite{al2021integrated}, when cortisol is not introduced in the
		co-culture. When modelling tumour cell and CTL growth, at each time-step, we let cells grow at a random rate drawn from a uniform distribution; the parameters of the bounds of the uniform distribution are chosen to match the mean duration of a tumour cell and CTL cycle length.
		
		The ratio between the energy at the interface between tumour cells and CTLs and the energy at the interface between tumour cells (\textit{i.e.} the values of parameters $J_{CT}$ and $J_{TT}$ in Eq. \eqref{Hsum}) allows us to consider a wide range
		of biological scenarios corresponding to different degrees of immune infiltration. In particular, if $J_{CT} < J_{TT}$, then CTLs infiltrate through tumour cells, whereas if $J_{CT} > J_{TT}$, then CTLs accumulate at the margin of the tumour, without infiltrating it.  Therefore, to obtain different degrees of immune infiltration, we fix the value of $J_{TT}$ and we vary the value of $J_{CT}$. In the body of the paper we refer to the parameter $J_{CT}$ as “tumour cell-CTL adhesion strength” (TC-CTL adhesion strength). In the control scenario we suppose that CTLs have a high capability of infiltrate through tumour cells. Therefore, we suppose that $J_{CT} < J_{TT}$. In stressed conditions instead, we suppose that CTLs have a lower capability to infiltrate through tumour cells. Therefore, we increase the value of $J_{CT}$ to a value equal to or greater than that chosen for  $J_{TT}$.
		
		\subsection{Parameters in the 3D system}
		The numerical simulations shown in Sec.~\ref{ch4:Numerical simulations and preliminary results} attempted to verify that our model produces similar results
		both in 2D and 3D. Therefore, in the 3D system we make use of the same values selected in the two-dimensional case (see previous subsection). Due to the slightly different number of cells obtained at the end of the numerical simulations in 3D, we simply adjust the death rate of tumour cells and CTLs due to intra-population competition, in order to obtain a number of cells at the end of the numerical simulations similar to that of the 2D scenario.
		\begin{table}[hp!]
			\small
			\caption{Parameter values used to implement the CP model. Energies, temperature and constrains are dimensionless parameters.}
			\centering
			\begin{tabular}{p{1.3cm} p{1.2cm} p{6.3cm} p{4.5cm} p{1cm}}
				\hline
				Phenotype & Symbol & Description & Value & Ref. \\
				\hline
				\textbf{Domain} & Pixel & Lattice site in 2D & $1$ Pixel $=3\times3\; \mu m^2$ &  \\ 
				& Voxel & Lattice site in 3D & $1$ Voxel $=3\times3\times3\; \mu m^3$ &  \\ 
				& $\Delta t$ &Time-step&  1 MCS = 0.5 min & \\
				\\
				\textbf{CC3D} & $J_{MT}$ & Contact energy tumour cells-medium & 50 &  ~\cite{leschiera2021mathematical}\\ 
				& $J_{MC}$ & Contact energy CTLs-medium & 50 & ~\cite{leschiera2021mathematical} \\ 
				& $J_{CT}$ & Contact energy CTLs-tumour cells & high adh.: 5; intermediate adh.: 50; low adh.: 95  &  \\ 
				& $J_{TT}$ & Contact energy tumour cells-tumour cells & 50 & ~\cite{leschiera2021mathematical} \\ 
				& $J_{CC}$ & Contact energy CTLs-CTLs & 1000 & ~\cite{leschiera2021mathematical} \\
				& $d_{T}$ & Tumour cell diameter  & 20-40 $(\mu m)$ &~\cite{gong2017computational}\\
				& $d_{C}$ & CTL diameter & 12 $(\mu m)$ & ~\cite{gao20162}\\
				& $A_0$ & Initial area constrain (2D) & $\mathcal{U}_{[25,55]}$ - tumour cells \textit{(pixels)} $\mathcal{U}_{[20,25]}$ - CTLs \textit{(pixels)} &  \\
				& $V_0$ & Initial volume constrain (3D) & $\mathcal{U}_{[25,55]}$ - tumour cells  \textit{(voxels)} $\mathcal{U}_{[20,25]}$ - CTLs  \textit{(voxels)} &  \\
				& $P_t$ & Perimeter constrain (2D) & $4\sqrt{A_t}+0.5$ \textit{(pixels)} &  \\
				& $S_t$ & Surface constrain (3D) & $6V_t^{\frac{2}{3}}$ \textit{(voxels)} &  \\
				& $\lambda_{area}$ & Tumour cell and CTL area constrain (2D)  & $10$ & \\
				& $\lambda_{per}$ & Tumour cell and CTL perimeter constrain (2D)  & $10$ & \\
				& $\lambda_{vol}$ & Tumour cell and CTL volume constrain (3D)  & $20$ & \\
				& $\lambda_{surf}$ & Tumour cell and CTL surface constrain (3D)  & $20$ & \\
				& $T_m$ & Fluctuation amplitude parameter & $10$ & ~\cite{leschiera2021mathematical}\\
				& $\lambda_{chem}$ & Strength and direction of chemotaxis  & $50$ & ~\cite{leschiera2021mathematical}\\
				\hline
			\end{tabular}
			\label{ch4:table1}
		\end{table}
		
		\begin{table}[hp!]
			\small
			\caption{Parameter values used in numerical simulations.}
			\centering
			\begin{adjustbox}{angle=0}
				\begin{tabular}{p{1.8cm} p{4.3cm} p{6.6cm} p{1cm}}
					\hline
					Phenotype  & Description & Value & Ref. \\
					\hline
					\textbf{Tumour} & Initial number  &$N_T(0) =36$ &  \\ 
					& Index identifier& $n=1, \dots, N_T(t)$&  \\
					
					&  Lifespan   & $\mathcal{U}_{[3,7]}$ (\textit{days}) & ~\cite{gong2017computational}\\
					
					& Growth rate  &  $\mathcal{U}_{[0.015,0.019]}$ (\textit{pixel or voxel/MCS})  &  ~\cite{al2021integrated}
					\\
					
					& Mean cycle time  & $12$ (\textit{hours}) & ~\cite{al2021integrated}  
					\\
					& Rate of death due to intra-pop. competition   & (2D) $4.6\times 10^{-7}$ (\textit{1/MCS}) {\color{white}-- - - - - - - - - - - - } (3D) $4.4\times10^{-7}$ (\textit{1/MCS}) &  estim. estim. 
					\\\\
					\textbf{CTLs}  &Initial number & $N_C(0)=150 $&  \\
					& Index identifier  & $m=1, \dots, N_C(t)$&  \\
					& Growth rate  & normal: $\mathcal{U}_{[0.0038,0.0042]}$ (\textit{pixel or voxel/MCS})   decreased:  $\frac{1}{2}\mathcal{U}_{[0.0038,0.0042]}$ (\textit{pixel/MCS})& ~\cite{al2021integrated}
					\\
					& Mean cycle time  & 8-10 (\textit{hours}) & ~\cite{al2021integrated,gong2017computational}  
					\\
					& Rate of death due to intra-pop. competition   & (2D) $1.2\times 10^{-6}$ (\textit{1/MCS})  {\color{white}-- -  - - - - - - - - - -} (3D) $1.3\times 10^{-6}$ (\textit{1/MCS}) &  estim.  estim.
					
					\\
					& Lifespan & $\mathcal{U}_{[2.5,3.5]}$ (\textit{days})&  ~\cite{gong2017computational}\\
					& Engagement time & 6 (\textit{hours})  & ~\cite{christophe2015biased} \\
					& Immune success rate  & Figs~\ref{ch4:tumour_no_T_cells}-\ref{ch4:mainres2}:	$0$; Fig~\ref{ch4:mainres3}: $0.00005$ &   \\
					\\
					\textbf{Chemoattr.} 
					& Concentration &  $\phi\geq 0 $ (\textit{mol/pixel or voxel}) &  \\
					& Diffusion & $ D=2$ (\textit{pixel$^2$ or voxel$^3$/MCS}) &  ~\cite{leschiera2021mathematical}\\
					& Secretion &  high: $\alpha=30$; intermediate: $\alpha=10$; low: $\alpha=3$  (\textit{mol/MCS/pixel or voxel})\\
					& Decay & $\gamma=7\times 10^{-4} $ (\textit{1/MCS}) & ~\cite{leschiera2021mathematical} \\
					& Initial concentration  & (2D) $\phi^{init}=0.5(280-\sqrt{(x-200)^2+(y-200)^2})$  (3D) $\phi^{init}=0.5(280-\sqrt{(x-50)^2+(y-50)^2 +(z-50)^2})$&  \\
					
					\hline
				\end{tabular}
			\end{adjustbox}
			\label{ch4:table2}
		\end{table}
	\end{appendices}	 
	%\include{supplementaryNoComment}
	\bibliographystyle{vancouver}
	\bibliography{mainDocumentNoComment.bib}
		\end{document}
