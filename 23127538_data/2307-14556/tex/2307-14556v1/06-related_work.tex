\section{Related Work}
\label{work}
Godefroid et al. \cite{godefroid2017learn} proposed a two layer sequence-to-sequence RNN to
train a PDF-object test case generator with the goal to improve code coverage. They observed a
tension between generating valid test cases and actually inserting errors because the models
reproduce the input well. Therefore they proposed an algorithm called 'SampleFuzz' that inserts
the lowest predicted probability if the highest probability is above a set threshold.
In contrast to their work we researched a different input format namely HTML-tags that
are more structure reliant than PDF-objects. Furthermore, we provided a detailed analysis
of the effects changes in the model architecture have on the output. Finally, we proposed
to improve the code coverage of a generator model by training a DDQN agent to inject HTML tags.

B\"{o}ttinger et al.\cite{bottinger2018deep} researched the application of deep Q-learning on
mutation based fuzzers for PDF files. They used an existing corpus of PDF files to start
with and evaluated two different training approaches. First based on code coverage and
a second based on the standard state-action-reward memory approach.
Whereas they worked on a mutation based approach, we proposed to apply a DDQN agent
on an existing deep learning generator to improve the code coverage of it. Furthermore,
we saw an improvement on the baseline and the generator alone by utilizing a default
replay memory with the state-action-reward approach.

Sablotny et al.\cite{sablotny2018rnnfuzz} proposed a model architecture based on stacked 
Recurrent Neural Networks to generate HTML test cases. Their results indicated that it is
possible to augment a generation based fuzzer with RNNs and increase the
performance of the underlying fuzzer. In contrast to this earlier work we used a TCN based approach
and highlighted that our design is also able to discover areas that were not covered
by the underlying fuzzer. Furthermore, the earlier work did not add a second model to guide the generator
model to increase the code coverage.

% \begin{enumerate}
%     \item Learn and Fuzz paper
%     \item RL fuzz paper
%     \item CHECK FOR NEWER REFS!!!
% \end{enumerate}