%-------------------------------------------------------------------------------
\section{Assumptions and limitations}
\label{section:limitations}
%-------------------------------------------------------------------------------

\subsubsection{NF limitations}
To allow ESE, NFs must fit within some limitations, much like the ones enumerated in \cite{pix}:
i) there must be a clean separation between stateful and stateless operations, a constraint put in practice by only allowing state to persist within a set of well-defined data structures;
ii) loops must be statically bounded;
and iii) no pointer arithmetic is allowed outside the data-structures. These constraints are already enforced for safety reasons in commonly used packet processing framework like eBPF\footnote{NFs developed in eBPF store their state in kernel-maintained maps~\cite{ebpf-maps}.}~\cite{ebpf}, a widely used framework in both academia an industry~\cite{katran,xdp,cilium,crab,hxdp,xdp-netdev,hda}.

\subsubsection{RSS limitations}
For \maestro to consider other hash function besides the standard toeplitz-based one, they would have to be formulated as an SMT problem and added to \librs. This requires having their implementation openly disclosed.

In practice, a more limiting factor is packet field selection: shared-nothing approaches can only be applied if state is sharded using RSS-compatible packet fields. DPDK's API~\cite{dpdk-rss-fields} reference includes all possible field combinations that RSS can use (\eg IPv4/IPv6 TCP/UDP flow tuples), but each NIC only implements a subset of them~\cite{e810,x710}.

\subsubsection{Attacking state sharding}
We mentioned earlier that it would be possible to ``fill-up'' a single core with fewer flows in a shared-nothing parallel NF than would otherwise be needed in the sequential or lock-based parallel versions.
This could potentially be used as a DoS attack vector, reducing the cost for an attacker to block new flows from being admitted.
RSS++ flow redistribution addresses this for well-behaved traffic, but an attacker can subvert this by specifically using flows that induce exact RSS hash collisions.
Colliding flows end up on the same entry within the RSS indirection table and thus cannot be split apart.

Though out-of-scope for this paper, \maestro provides some defense from such attacks due to the randomization used to generate RSS keys.
Even assuming the attacker has access to the NF source code and understands how it can be sharded across cores, different random RSS keys that comply with the sharding constraints will still distribute different flows in a different way.
Without access to the actual key generated in \librs, the attacker would have a harder time reverse-engineering a set of co-located flows, mitigating their ability to induce the kind of persistent skew needed in a successful attack.

%-------------------------------------------------------------------------------