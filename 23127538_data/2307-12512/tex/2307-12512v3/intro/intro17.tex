\section{Introduction} \label{sec:introduction}

Extended Reality (XR), broadly encompassing virtual, augmented, and mixed reality technologies, can potentially revolutionize fields such as education, healthcare, and gaming~\cite{thomas2012survey,alizadehsalehi2020bim,xi2022challenges}. The primary ethos for XR is to provide immersive, interactive, and realistic experiences for users. A key component of delivering this user experience is to transfer the physical world into the virtual space. For example, our everyday spaces and objects can be transformed into video game assets (like tennis racquets, swords, or chess pieces) for interactive gaming applications.\footnote{\label{fn:demo} in a demo, we transform mugs and a desk into our lab to a life-size chess board (link:\href{https://bit.ly/3q7DKKy}{https://bit.ly/3q7DKKy})} To enable these applications, we find a common thread --- any XR system should localize and track objects in an environment. Specifically, this object-tracking system needs to satisfy three key requirements to realize XR applications:

% Figure environment removed


\noindent \textbf{R1. Ease of anchor deployment:} Any asset localization system must have low deployment efforts, which can potentially be embedded within common electronics like TVs or soundbars. This single module should be smaller than $1$ m.~\footnote{Most consumer electronics like TVs or soundbars are around $1$ m in length.}  

\noindent \textbf{R2. Accurate and reliable:} Assets must be localized to an accuracy within a few centimeters in room-scale scenarios. We place a stringent requirement of a few centimeters of accuracy to provide a glitch-free user experience. Providing immersive XR experiences consequently means small user or object tracking errors are more obvious and severely impede the adoption of XR~\cite{wang2007design}. Specifically, the localization system must be reliable during movement, under occlusions, and consistently track assets within an accuracy of a few cm.   

\noindent \textbf{R3. Multi-asset low latency localization:} Finally, an XR system needs to localize multiple objects in an environment in real time. In dynamic scenarios, this can mean we must localize tens of objects with a $60$--$80$ Hz update rate as people naturally perceive their surroundings at 60--75 Hz~\cite{deering1998limits}, and delays in updates of object locations in a dynamic scenario can break away from an immersive experience.  

\begin{table*}[]
    \begin{tabular}{|l|c|c|c|c|c|c|c|}
        \hline
         & \textbf{Visual} & \textbf{Acoustic} & \textbf{Radar} & \textbf{RFID} &  \textbf{Single anchor} & \textbf{\name} \\ \hline
        \textbf{R1: Ease of anchor deployment} & \checkmark & \checkmark & \checkmark & $\times$  & \checkmark & \checkmark \\ \hline
        \textbf{R2: Accuracy and reliability} & $\times$ & \checkmark & $\times$ & \checkmark  & $\times$ & \checkmark \\ \hline
        \textbf{R3: Multi-asset and low latency} & \checkmark & $\times$ & \checkmark  & \checkmark & \checkmark & \checkmark \\ \hline
    \end{tabular}
    \caption{Existing technologies do not satisfy the 3 key requirements for an XR localization/tracking system. \label{tab:related-works}}
\end{table*}

However, none of the existing asset localization systems meet these three key requirements to deliver XR applications in everyday scenarios (see Table~\ref{tab:related-works}). 
Camera and visual sensors are susceptible to poor lighting and visual occlusions, consequently failing to provide reliable localization (\textbf{R2}). Additionally, deploying a camera-based system can be privacy invasive~\cite{russell2013people} in home and public settings.
Acoustic systems~\cite{liu2020indoor} provide accurate localization but are difficult to localize multi-asset with low latency simultaneously (\textbf{R3}).
Radar systems~\cite{kong2022m3track,mukherjee2022scalable,xue2021mmmesh} can provide low-latency object tracking from a single module but fail to track occluded objects or those which have small radar cross-sections (RCS). 
Some RFID systems have succeeded in realizing low latency \cite{turbotrack,mobitagbot,tagoram}. Their asymmetric architecture (cost-effective tags and expensive readers) better suits large-scale deployments in retails and industrial sectors. However, long-range RFID systems ($>$ 6m) are expensive and bulky to integrate into consumer electronics, precluding wide-scale deployments (\textbf{R1}).


Altenatively, many single RF module localization solutions~\cite{chen2019m,ge2021single,giorgetti2009single,grosswindhager2018salma,groth2021calibration,kotaru2017position,li2020multipath,meissner2012multipath,soltanaghaei2018multipath,wang2019efficient,wang2019high,zhang2022toward} leveraging WiFi/BLE or ultra-wideband (UWB) are easy to deploy because of transceivers which can be inexpensively deployed in consumer electronics. However, they fail to provide the necessary cm-level accuracy.
None of the existing systems simultaneously satisfy all three stringent requirements to enable XR applications, and prior art will be more carefully considered in Sec.~\ref{sec:related}. 


To address the need for XR-compliant localization, we develop \name, which consists of two parts --- a localization tag, attachable to objects of interest, and a single localization module to furnish few-cm level locations from a single vantage point. The localization module is less than $1$ m and can be easily incorporated within everyday electronics such as televisions or soundbars (satisfying \textbf{R1}). It leverages the tag's single UWB transmission for a few cm accurate localization. An accompanying MAC protocol also supports the localization of multiple tags at an update rate of $100$ Hz (satisfying \textbf{R3}). An example deployment of \name is showcased in Fig.~\ref{fig:muloc_prototype_tv}, where beverage cups are attached with off-the-shelf UWB tags. \name is leveraged to transform an office space into a life-sized chess board, with these cups taking the place of chess pieces and localized with cm-level accuracy. A video demo of this case study is also included as well$^{\ref{fn:demo}}$. However, to simultaneously meet all the aforementioned requirements, we need to solve four key challenges:    


\noindent \textbf{1. Geometric dilution of precision:} In most UWB localization systems, three or more UWB anchors need to be placed in diverse locations in a room to localize the UWB tag, increasing deployment efforts and breaking away from \textbf{R1}. Alternatively, we can place these UWB anchors within a single localization module constrained to a $1$ m space. However, reducing the spatial diversity can worsen the localization accuracy by $10\times$. This accuracy degradation is called `geometric dilution of precision'~\cite{spilker1996global} (GDOP). A potential strategy to overcome GDOP is to borrow techniques from RFID-systems~\cite{tagoram, turbotrack, mobitagbot} that achieve real-time cm-scale accuracy from a single RFID reader. However, we observe UWB systems provide $15 \times$ worse measurement accuracy compared to RFID systems~\cite{tagoram}, owing to an RFID system sharing the same clock at the transmitter and receiver (mono-static architecture). Hence a direct consequence of GDOP is a \name's reduced resilience to measurement noise, which precludes us from directly borrowing techniques from RFID-based systems.

To reduce our measurement noise, we could increase transmit power to improve signal quality, increase transmission length for better averaging, or choose better hardware with lower noise floors. However, these solutions come at the cost of increased battery consumption at the tag, increased localization latency, or expensive tag design, respectively. Alternatively, \name makes a key observation when looking at the phase difference of the received UWB signal measurements (PDoA) between a pair of anchors ---  PDoA measurement quality can be improved proportionally to the distance between the pair of anchors. This simple observation forms the cornerstone of \name's design and allows us to satisfy the first requirement \textbf{R1}. 

\noindent \textbf{2. Ambiguous location predictions:} However, this improved PDoA measurement quality comes at a detrimental cost --- increasing the anchor spacing creates multiple ambiguous location predictions as phase measurements wrap around at $2\pi$. The changes in these ambiguities mirror the changes in the true location of the tag, and they do not affect tracking systems~\cite{wang2014rf, cao2021itracku}, which leverage phases to provide cm-level tracking accuracy for handwriting recognition. However, incorrectly choosing an ambiguous absolute location can degrade the accuracy by several tens of centimeters and may create glitches within the XR system. 

To predict accurate locations despite phase wrap-around, \name leverages a simple observation --- unlike phase measurements, time of arrival measurements do not suffer from ambiguity. Specifically, the time difference of arrival (TDoA) between a pair of anchors, although inaccurate in furnishing cm-level localization, can help to detect and filter out ambiguities. By cleverly fusing these time-difference and phase-difference measurements, \name can provide cm-level accurate locations from a single UWB transmission and satisfy the second requirement \textbf{R2}. 

\noindent \textbf{3. Measurement bias-aware localization:} However, as we push the envelope on cm-accurate location predictions, we find that hardware biases can corrupt our location estimates and degrade our location accuracy by over $2 \times$. Specifically, through empirical measurements, and as observed in previous studies~\cite{de2019range}, UWB modules~\cite{decawave-dw1000} suffer from a distance-sensitive measurement bias. We model, estimate, and calibrate for these biases via a three-point calibration procedure. We fuse the time and phase measurements with a corrected PDoA and TDoA measurement model by leveraging a particle filter to provide cm-accurate and low-latency location estimates, satisfying \textbf{R2}.

\noindent \textbf{4. High update rate multi-tag operation:} In addition to providing low-latency localization, \name must furnish locations for multiple objects in the environment. Often, the UWB transmissions for localization from multiple tags in an environment can cause packet collisions at \name's module. The collision causes localization failure $25\%$ of the time. We leverage a low-power wireless side channel to alleviate packet collisions to design a power-efficient medium access control (MAC) protocol. Specifically, \name deploys a LoRa-based MAC to support consistent localization for tens of tags at over $80$ Hz, satisfying \textbf{R3}. 

\name brings together these key techniques to build a $1$ m sized module, consisting of $6$ Decawave DW1000~\cite{evb1000} UWB modules for localization, along with a Semtech LoRa SX1272~\cite{sx1272mb2das} to furnish a side-channel for the MAC protocol. Additionally, we prototype a simple UWB + LoRa Tag using the Decawave EVB1000 and a LoRa SX1272. Through extensive evaluations, we find that \name satisfies all the three stringent requirements with
\begin{enumerate}
    \item Static localization error with median and 90th percentile accuracy of $1.5$ cm and $5.5$ cm, an improvement of $9.5 \times$ and $5.2 \times$ from state-of-art systems~\cite{zhao2021uloc}. 
    \item Dynamic localization error with median and 90th percentile accuracy of $2.4$ cm and $5.3$ cm, an improvement of $11 \times$ and $8 \times$ from state-of-art systems~\cite{zhao2021uloc}. 
    \item Localization failure rate of $0.5 \%$ when using the MAC protocol as compared to a failure rate of $~25\%$ without a MAC protocol, a $50\times$ improvement, for $10$ tags operating simultaneously at $100$ Hz location update rate.  
    \item Location compute latency of $1$ ms, allowing for real-time localization ($60$ Hz) of 16 tags. 
\end{enumerate}

