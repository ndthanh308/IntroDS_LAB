\section{Related Works}
\label{sec:related}
Providing indoor location information for people and various in-animate objects is a well-studied problem. This section will broadly cover various techniques leveraged to address this problem. We will find that none of the existing techniques meets the stringent requirements we set up earlier in Sec.~\ref{sec:introduction}. Recall that we seek to provide easy-to-deploy (\textbf{R1}), few-cm accurate localization (\textbf{R2}) in dynamic scenarios for multiple people or objects of interest (\textbf{R3}). A few key technologies which can be considered are:

\noindent \textbf{Visual sensing:} Under this broad umbrella, we have many distinct technologies. Existing VR systems utilize external IR-based sensors~\cite{borges2018htc} or specialized cameras~\cite{vicon} to furnish accurate ground truth locations. There are also works that deploy a single Lidar~\cite{hasan2022lidar} for person tracking or utilize headset-mounted cameras~\cite{monica2022evaluation}. However, these systems are sensitive to visual occlusions, hindering a user experience. Recent works~\cite{zhong2021towards, sahin2023hoot, luo2022novel} which leverage machine learning to track objects despite occlusions. 
Alternatively, other studies ~\cite{guo2022multi, li2020pose, sensys2020litag,sensys2021curvelight,imwut2023spectral} seek to deploy multiple cameras, let tag equips with a camera, or utilize special light sources to be robust to occlusions.
However, no studies have simultaneously solved all the problems of ease of anchor deployment (\textbf{R1}), accuracy (\textbf{R2}), and the risk of security and privacy~\cite{vigdor2019somebody}.
Moving away from deploying privacy-invasive cameras, other works~\cite{miller2022cappella} seek to use the cameras on-board VR setups fused with occlusion-resilient radio-frequency (RF) signals like ultra-wideband.
These systems have a low deployment cost but do not achieve a few-cm level accuracy. 

\noindent \textbf{Acoustic sensing:}  Alternative to these systems, various acoustic localization systems~\cite{famili2022pilot, merenda2022rfid, murakami20193,cao2020earphonetrak,sensys2020fmtrack,sensys2020symphony} take advantage of the lower speed of sound (~$340$ m/s) for fine-grained localization and meet the required localization accuracy. However, acoustic sensing has a few fundamental drawbacks~\cite{li2022experience}. 
First, acoustic systems~\cite{liu2020indoor} are difficult to provide both multi-asset and low latency localization simultaneously because of narrow bandwidth, deviating from \textbf{R3}.
Second, they hinder music and audio playback, precluding immersive  XR applications. Third, acoustic signals that employ ultrasound (> $20$ kHz) for sensing have considerable audio leakage in the audible frequency range, affecting user experience. 

\noindent \textbf{Radar-based sensing:} Mm-wave radars near the $60$ GHz and $77$ GHz bands have gained recent interest. Many works~\cite{xue2021mmmesh, kong2022m3track} have looked at furnishing human pose with these radars from a single radar. Recent work~\cite{mukherjee2022scalable} has shown that the human body can act as a strong blockage at these frequencies. These blockages can hinder tracking multiple people and objects in an environment and affect user experience. Additionally, tracking and identifying smaller assets in an environment can be challenging as radar reflections depend on an object's radar cross-sectional area (RCS). Alternatively, many works~\cite{soltanaghaei2021millimetro} propose placing retro-reflective tags on objects with small RCS to guarantee their detection; however, these systems suffer from poor localization accuracy. 

\noindent \textbf{RF-based sensing:} 
The robustness of sub-6 GHz RF-signals to occlusions~\cite{slezak2022measurement} and low privacy risk makes it a promising technology to consider. The common mode of operation is for multiple RF radios to jointly localize an active RF transmitter or a passive RF reflector (tags). Many works have looked at leveraging WiFi~\cite{mostafa2022survey, spotfi, arraytrack, vasisht2016decimeter}, LoRa~\cite{percom2019lorain}, or BLE~\cite{ayyalasomayajula2018bloc} to achieve robust user localization. However, these systems fail to provide the required localization accuracy due to bandwidth limitations. 

RFID has a strong asymmetry in the reader-tag relationship, and the transmitter and receiver share the same clock, which allows for highly accurate phase acquisition.
According to \cite{mobitagbot,tagoram,turbotrack}, RFID systems do not have carrier and sampling frequency offset and enjoy a phase measurement accuracy of $0.085^\circ$~\cite{tagoram}, 15$\times$ better than the UWB, which provides an accuracy of ~$1.4^\circ$.
Using the highly accurate phase, \cite{tagoram,turbotrack,wang2014rf,ma2017minding,ipsn2018omnitrack} has succeeded with tracking or localization at the few cm levels.
However, due to the asymmetric nature, RFID readers whose range is several meters are not suitable for embedding into consumer electronics (\textbf{R1}) because of their power-hungriness and expensiveness (ex. ImpinjJ Speedway R420 costs \$1666).
The main target of RFID is industrial or retail store settings where thousands of tags must be deployed inexpensively, and readers' one-time cost is justifiable.
For instance, \cite{mobitagbot} looks at item ordering in manufacturing lines, retail stores, or libraries. \cite{tagoram,turbotrack} examine industrial robotics or baggage handling tasks.



Unlike RFID, Ultra-wideband provides a more symmetric architecture where localization modules can cost $\$10-100$. Consequently, we have seen their increased adoption in smartphones and smart tags. It provides over $500$ MHz of bandwidth and a time resolution of $1$ ns, providing localization accuracy to a few tens of centimeters.
Many current UWB-localization schemes leverage the accurate time-resolution for Two-Way Ranging (TWR)~\cite{zwirello2012uwb,ledergerber2019ultra,garcia2015robust,alarifi2016ultra,bonnin2019uwb, kempke2016harmonium,9289338,iball} and localize objects via trilateration. However, these multiple-packet exchanges increase localization latency and prevent real-time tracking of multiple objects of interest (\textbf{R3}). Many works instead leverage the TDoA or PDoA of the UWB signal to multiple time-synchronized anchors~\cite{chen2020joint, vecchia2019talla, tiemann2016atlas, grobetawindhager2019snaploc,cao2021itracku}, or AoA measurements~\cite{zhao2021uloc, heydariaan2020anguloc, decawave-pdoa-kit} at multiple anchors to furnish locations using a single packet. Some works~\cite{yang2022vuloc} employ alternative transmission schemes to TWR to reduce the packet overhead. However, these systems only meet the necessary localization accuracy when the UWB anchors are placed in diverse locations, increasing deployment efforts and deviating from \textbf{R1}. 


As discussed in Sec.~\ref{sec:back}, few-cm accurate localization is challenging due to geometric dilution of precision. To circumvent this problem, three common techniques are leveraged. First, by leveraging reflected paths in the environment, many systems~\cite{kotaru2017position, chen2019m, li2020multipath, zhang2022toward, soltanaghaei2018multipath, meissner2012multipath, grosswindhager2018salma} create additional ``virtual'' radios in the environment. These ``virtual'' radios provide the needed spatial diversity to localize an object of interest. However, multipath is often unreliable~\cite{almers2006keyhole} in many environments and can lead to localization failure and poor user experience. Second, many works~\cite{wang2019high, ge2021single, wang2019efficient, rostami2022enabling} look at fusing TWR, TDOA, and AoA information to provide single anchor localization solutions. However, some systems cannot furnish the few-cm accurate localization requirement or rely on TWR measurements, increasing the system's latency. Finally, some works develop switched beam antennas~\cite{groth2021calibration, giorgetti2009single}, which selectively sense signals approaching the anchor from different directions. However, these systems lack the required angular resolution to provide localization accuracy of a few cm. 

