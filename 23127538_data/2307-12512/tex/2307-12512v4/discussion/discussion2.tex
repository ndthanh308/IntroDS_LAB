\section{Discussion and Future work}

\name overcomes the fundamental challenges arising from geometric dilution of precision to deliver cm-level accurate localization by developing an easy-to-deploy and low-latency localization module. Through this development, we are one step closer to achieving immersive XR experiences. However, a few limitations and possibilities of future work can be explored to build upon \name.

\noindent \textbf{Extensions to 3D:} \name focuses on localizing people and assets on a 2D floor plane, which is required in various XR applications. However, these ideas can be extended to the 3D domain by incorporating a vertical array of antennas in conjunction with the current horizontal linear array. These 3D-compliant antenna arrays can be retrofitted with television screens or paintings to allow cm-accurate 3D localization. 

\noindent \textbf{Improving power efficiency of \name's localization module:} Various works~\cite{biri2020socitrack} have noted the $10 \times$ higher power consumption of UWB reception than transmission. Keeping this in mind, we designed a system that requires only a single transmission from the tag for localization to ensure long battery life. However, the $6$ receivers on \name's wall-powered localization module are power inefficient. To rectify this, antenna switching schemes~\cite{gu2021tyrloc}  can be employed, or multiple antennas can be combined to connect to a single receiver~\cite {cao2021itracku} to reduce the number of receivers. However, unlike \name's system, these alternatives will not be FiRa compliant~\cite{coppens2022overview}.

\noindent \textbf{Miniaturized tag design: } We prototype our tag from off-the-shelf EVB1000~\cite{evb1000} and LoRa~\cite{sx1272mb2das} evaluation boards. Future work can look towards miniaturizing these tag designs. Since these radios we employ are centered at 3.4 GHz and 930 MHz, it allows us to place these radio modules in close proximity with limited RF interference. 

\section{Acknowledgement}

We thank Neil Smith at UCSD, Kazuhiro Kizaki at Osaka University, and the members of WCSNG at UCSD for their help and feedback.
