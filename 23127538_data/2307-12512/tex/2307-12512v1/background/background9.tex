\section{Why is this problem hard?} \label{sec:back}

% Figure environment removed

We have established the need for localizing users and objects within a few centimeters of a single vantage point. In this section, we will find that restricting our sensing to within a space of $1$ m reduces our geometric diversity leading to localization errors of many $10$'s of centimeters. This phenomenon is commonly referred to as geometric dilution of precision. We will explore the use of three common UWB measurements -- two-way-ranging (TWR), time-difference-of-arrival (TDoA), and angle-of-arrival (AoA) -- and find systems that rely on these measurements fail to furnish the required accuracy. Additionally, we'll explore fusing and jointly optimizing for these measurements to improve localization accuracy. However, even this measurement fusion is insufficient. To test this hypothesis, we build a simple simulation environment described below. 

\noindent\textbf{Simulation environment:} We perform extensive simulation in a $3 \times 3$ environment, a standard room size, to find the best case localization accuracy. We use $6$ UWB transceivers, placed either diversely in the environment (red diamonds in Fig.~\ref{fig:back} (a)) or in a limited space near the bottom wall (see Fig.~\ref{fig:back} (b, c, d)). Next, we divide this space into a $1$ mm grid and place tags in each position to measure the location accuracy. The pixels of the `heatmaps' represent these tag locations, and the pixel color intensity quantifies the median localization accuracy across $100$ simulated trials.   

\noindent\textbf{Simulating TWR:} Many UWB radios measure the time of flight (ToF) of the signal between the transmitter and receiver with up to a resolution of $15.6$ ps~\cite{decawave-pdoa-kit}. The ToF is measured via multiple packet exchanges, taking at least $0.3$ ms~\cite{corbalan2020ultra}. And clock drifts at the receiver during this TWR event can lead to a ToF measurement deviation of $150$ ps for a $0.5$ ppm clock crystal. Hence, we characterize our simulated TWR measurements with a zero-mean Gaussian with a standard deviation of $150$ ps. 

\noindent\textbf{Simulating TDoA:} Instead of an absolute time of flight measurement, we can measure the difference in the time of arrivals across a pair of synchronized receivers. However, TDoA measurements depend on the receivers' clock synchronization accuracy. Our measurements, independently verified by Decawave~\cite{tdoa_an}, show clock-sync errors in best-case wired synchronization can cause a TDoA measurement deviation of $140$ ps. Hence, our simulated TDoAs are Gaussian distributed with a standard deviation of $140$ ps.  

\noindent\textbf{Simulating AoA:} Some UWB systems~\cite{zhao2021uloc, heydariaan2020anguloc} alternatively measure the angle of arrival of a signal between a pair of receivers placed half-wavelength apart (see close pairs of red diamonds in Fig.~\ref{fig:back}(c)). We can measure AoA with noise deviation of $1.5^\circ$, as independently verified in~\cite{zhao2021uloc, heydariaan2020anguloc}. Consequently, we simulate our AoA measurements as zero-mean Gaussian with $1.5^\circ$ standard deviation.

\subsection{Quantifying localization errors}

TWR, or distance measurements between a tag and multiple receivers placed diversely in an environment, can be used to trilaterate the tag's position to achieve a few cm-level accuracies. From Fig.~\ref{fig:back}(e), we find that the median localization error is $2.9$ cm. Additionally, this error is consistent (with a variation of a few centimeters) across the space (see heatmap in Fig.~\ref{fig:back}(a)). However, when we place all the receivers within a $1$ m linear form factor to satisfy \textbf{R1}, we find that the accuracy degrades by over $8 \times$ as compared to the diverse antenna placement. Additionally, we observe a non-uniform performance with errors as large as $1$ m. To meet \textbf{R2}, we have made our localization system too erroneous to be usable. 

The fundamental reason for the performance degradation is the reduced geometric diversity when the antennas are closer. With the antennas placed around the environment, trilateration is more resilient to errors in distance measurements. We quantify the localization errors by leveraging TDoA or AoA measurements and summarize the results in Fig.~\ref{fig:back}(e). We find under low geometric diversity, the median localization errors can be close to $54.4$ cm and $40.9$ cm for TDoA and AoA measurements, respectively.   

\subsection{Fusing all measurements}

Similar to many robotics applications~\cite{alatise2020review}, we can use TWR, AoA, and TDoA measurements to provide higher accuracy. This fusion is done by jointly optimizing the error function from TWR, AoA, and TDoA~\cite{liu2013joint} measurements. Specifically, in Fig.~\ref{fig:back}(c), we measure $6$ TWR measurements from each receiver (red diamonds), $3$ AoA measurements from each closely-spaced pair of UWB receivers, and $3$ TDoA measurements between one antenna from each of these paired groups. The measurement-fusion efforts provide median localization of $23.3$ cm. However, it still fails to meet our criteria of a few-cm error in localization. 

None of the existing states of art systems can surmount the challenge of localizing from a single vantage point and deliver the stringent requirements set forth by our application use case. In \name, we develop the algorithm (Sec.~\ref{sec:design}) and prototype a system (Sec.~\ref{sec:implementation} and ~\ref{sec:des-mac}) to achieve this small-form-factor, high accuracy (median accuracy of $3.3$ cm as seen from Fig.\ref{fig:back}(d) and Fig.~\ref{fig:back}(e)), and multi-asset localization system, for use within VR systems and immersive audio applications. In the following section, we will delineate the key ideas which allow \name to circumvent the challenges posed by geometric dilution of precision. 
