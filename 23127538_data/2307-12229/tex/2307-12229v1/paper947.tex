% This is samplepaper.tex, a sample chapter demonstrating the
% LLNCS macro package for Springer Computer Science proceedings;
% Version 2.20 of 2017/10/04
%
\documentclass[runningheads]{llncs}
%
\usepackage{graphicx}
\usepackage{amsmath}
\usepackage{amssymb}
\usepackage{booktabs}
\usepackage{rotating}
\usepackage{makecell}
\usepackage{xspace}
\usepackage{xcolor}
\usepackage[export]{adjustbox}
\usepackage{multirow}
\usepackage{float}
\usepackage{pifont}
\usepackage{caption}
\usepackage{subcaption}
\usepackage{multirow}
% Used for displaying a sample figure. If possible, figure files should
% be included in EPS format.
%
% If you use the hyperref package, please uncomment the following line
% to display URLs in blue roman font according to Springer's eBook style:
% \renewcommand\UrlFont{\color{blue}\rmfamily}
\usepackage[pagebackref,breaklinks,colorlinks]{hyperref}

\usepackage[capitalize]{cleveref}
\crefname{section}{Sec.}{Secs.}
\Crefname{section}{Section}{Sections}
\Crefname{table}{Table}{Tables}
\crefname{table}{Tab.}{Tabs.}

\makeatletter
\DeclareRobustCommand\onedot{\futurelet\@let@token\@onedot}
\def\@onedot{\ifx\@let@token.\else.\null\fi\xspace}
\def\iid{\emph{i.i.d}\onedot} \def\IID{\emph{I.I.D}\onedot}
\def\eg{\emph{e.g}\onedot} \def\Eg{\emph{E.g}\onedot}
\def\ie{\emph{i.e}\onedot} \def\Ie{\emph{I.e}\onedot}
\def\cf{\emph{c.f}\onedot} \def\Cf{\emph{C.f}\onedot}
\def\etc{\emph{etc}\onedot} \def\vs{\emph{vs}\onedot}
\def\wrt{w.r.t\onedot} \def\dof{d.o.f\onedot}
\def\aka{\emph{a.k.a}\onedot}
\def\etal{\emph{et al}\onedot}
\makeatother

\newcommand*\samethanks[1][\value{footnote}]{\footnotemark[#1]}

\begin{document}
%
\title{EchoGLAD: Hierarchical Graph Neural Networks for Left Ventricle Landmark Detection on Echocardiograms}
%
\titlerunning{Graph Neural Networks for Left Ventricle Landmark Detection}
% If the paper title is too long for the running head, you can set
% an abbreviated paper title here
%
\author{Masoud Mokhtari\inst{1}\orcidID{0000-0001-9471-5573} \and
Mobina Mahdavi \inst{1} \and
Hooman Vaseli \inst{1}\orcidID{0000-0002-8259-9488} \and
Christina Luong \inst{2} \and
Purang Abolmaesumi\inst{1}\thanks{Co-Corresponding Authors}\and
Teresa S. M. Tsang\inst{2} \and
Renjie Liao\inst{1}\samethanks}
% index{Mokhtari, Masoud}
% index{Mahdavi, Mobina}
% index{Vaseli, Hooman}
% index{Luong, Christina}
% index{Abolmaesumi, Purang}
% index{Tsang, Teresa}
% index{Liao, Renjie}
%
\authorrunning{M. Mokhtari et al.}
% First names are abbreviated in the running head.
% If there are more than two authors, 'et al.' is used.
%
\institute{Electrical and Computer Engineering, University of British Columbia,
Vancouver, BC, Canada \\
\email{\{masoud, mobina, hoomanv, purang, rjliao\}@ece.ubc.ca} \and
Vancouver General Hospital, Vancouver, BC, Canada\\
\email{\{christina.luong, t.tsang\}@ubc.ca}}
%
\maketitle              % typeset the header of the contribution
%
\begin{abstract}
The functional assessment of the left ventricle chamber of the heart requires detecting four landmark locations and measuring the internal dimension of the left ventricle and the approximate mass of the surrounding muscle.
The key challenge of automating this task with machine learning is the sparsity of clinical labels, i.e., only a few landmark pixels in a high-dimensional image are annotated, leading many prior works to heavily rely on isotropic label smoothing. 
However, such a label smoothing strategy ignores the anatomical information of the image and induces some bias.
To address this challenge, we introduce an \textbf{echo}cardiogram-based, hierarchical \textbf{g}raph neural network (GNN) for left ventricle \textbf{la}ndmark \textbf{d}etection (EchoGLAD). 
Our main contributions are: 1) a hierarchical graph representation learning framework for multi-resolution landmark detection via GNNs; 2) induced hierarchical supervision at different levels of granularity using a multi-level loss. 
We evaluate our model on a public and a private dataset under the in-distribution (ID) and out-of-distribution (OOD) settings. 
For the ID setting, we achieve the state-of-the-art mean absolute errors (MAEs) of 1.46~mm and 1.86~mm on the two datasets. 
Our model also shows better OOD generalization than prior works with a testing MAE of 4.3~mm. 
    
\keywords{Graph Neural Networks  \and Landmark Detection \and Ultrasound.}
\end{abstract}
%
%
%

\section{Introduction}
\label{sec:intro}

% Figure environment removed

Left Ventricular Hypertrophy (LVH), one of the leading predictors of adverse cardiovascular outcomes, is the condition where heart’s mass abnormally increases secondary to anatomical changes in the Left Ventricle (LV)~\cite{GRADMAN2006326}. These anatomical changes include an increase in the septal and LV wall thickness, and the enlargement of the LV chamber. More specifically, Inter-Ventricular Septal (IVS), LV Posterior Wall (LVPW) and LV Internal Diameter (LVID) are assessed to investigate LVH and the risk of heart failure~\cite{McFarland1978}. As shown in Figure \ref{fig: plax} (a), four landmarks on a parasternal long axis (PLAX) echo frame can characterize IVS, LVPW and LVID, and allow cardiac function assessment. To automate this, machine learning-based (ML) landmark detection methods have gained traction. 

It is difficult for such ML models to achieve high accuracy due to the sparsity of positive training signals (four or six) pertaining to the correct pixel locations. In an attempt to address this, previous works use 2D Gaussian distributions to smooth the ground truth landmarks of the LV~\cite{jamie,jafari2021u,lin2021reciprocal}. However, as shown in Figure \ref{fig: plax} (b), for LV landmark detection where landmarks are located at the wall boundaries (as illustrated by the dashed line), we argue that an isotropic Gaussian label smoothing approach confuses the model by being agnostic to the structural information of the echo frame and penalizing the model similarly whether the predictions are perpendicular or along the LV walls.

In this work, to address the challenge brought by sparse annotations and label smoothing, we propose a hierarchical framework based on Graph Neural Networks (GNNs) \cite{scarselli2008graph} to detect LV landmarks in ultrasound images. 
As shown in Figure~\ref{fig: overall_arch}, our framework learns useful representations on a hierarchical grid graph built from the input echo image and performs multi-level prediction tasks. 

Our contributions are summarized below.
\begin{itemize}
    \item[$\bullet$] We propose a novel GNN framework for LV landmark detection, performing message passing over hierarchical graphs constructed from an input echo;
    \item[$\bullet$] We introduce a hierarchical supervision that is automatically induced from sparse annotations to alleviate the issue of label smoothing;
    \item[$\bullet$] We evaluate our model on two LV landmark datasets and show that it not only achieves state-of-the-art mean absolute errors (MAEs) (1.46 mm and 1.86 mm across three LV measurements) but also outperforms other methods in out-of-distribution (OOD) testing (achieving 4.3~mm).    
\end{itemize}

% Figure environment removed

\section{Related Work}

Various convolution-based LV landmark detection works have been proposed. Sofka \etal~\cite{sofka2017fully} use Fully Convolutional Networks to generate prediction heatmaps followed by a center of mass layer to produce the coordinates of the landmark locations. Another work~\cite{lin2021reciprocal} uses a modified U-Net \cite{unet} model to produce a segmentation map followed by a focal loss to penalize pixel predictions in close proximity of the ground truth landmark locations modulated by a Gaussian distribution. Jafari \etal~\cite{jafari2021u} use a similar U-Net model with Bayesian neural networks~\cite{bnn} to estimate the uncertainty in model predictions and reject samples that exhibit high uncertainties. Gilbert \etal~\cite{gilbert2019automated} smooth ground truth labels by placing 2D Gaussian heatmaps around landmark locations at angles that are statistically obtained from training data. Lastly, Duffy \etal~\cite{echonetlvh} use atrous convolutions \cite{atrous} to make predictions for LVID, IVS and LVPW measurements.

Other related works focus on the detection of cephalometric landmarks from X-ray images. These works are highly transferable to the task of LV landmark detection as they must also detect a sparse number of landmarks. McCouat \etal~\cite{contour} is one of these works that abstains from using Gaussian label smoothing, but still relies on one-hot labels and treats landmark detection as a pixel-wise classification task. Chen \etal~\cite{pyramid} is another cephalometric landmark detection work that creates a feature pyramid from the intermediate layers of a ResNet~\cite{resnet}.

Our approach is different from prior works in that it aims to avoid the issue shown in \cref{fig: plax} (b) and the sparse annotations problem by the introduction of simpler auxiliary tasks to guide the main pixel-level task, so that the ML model learns the location of the landmarks without relying on Gaussian label smoothing. 
It further improves the representation learning via efficient message-passing~\cite{scarselli2008graph,messagepassing} of GNNs among pixels and patches at different levels without having as high a computational complexity as transformers~\cite{vision_transformer,swin}.
Lastly, while GNNs have never been applied to the task of LV landmark detection, they have been used for landmark detection in other domains. Li \etal~\cite{li2020structured} and Lin \etal~\cite{lin2021structure} perform face landmark detection via modeling the landmarks with a graph and performing a cascaded regression of the locations. 
These methods, however, do not leverage hierarchical graphs and hierarchical supervision and instead rely on initial average landmark locations, which is not an applicable approach to echo, where the anatomy of the depicted heart can vary significantly. Additionally, Mokhtari \etal~\cite{mokhtari2022} use GNNs for the task of EF prediction from echo cine series. However, their work focuses on regression tasks.  


\section{Method}
\label{sec: method}
\subsection{Problem Setup}
We consider the following supervised setting for LV wall landmark detection. We have a dataset $D = \{X, Y\}$, where $|D| = n$ is the number of $\{x^i, y^i\}$ pairs such that $x^i \in X$, $y^i \in Y$, and $i \in [1, n]$. Each $x^i \in \mathbb{R}^{H\times W}$ is an echo image of the heart, where H and W are height and width of the image, respectively, and each $y^i$ is the set of four point coordinates $[(h^i_1, w^i_1), (h^i_2, w^i_2), (h^i_3, w^i_3), (h^i_4, w^i_4)]$ indicating the landmark locations in $x^i$. Our goal is to learn a function ${f}: \mathbb{R}^{H \times W} \mapsto 
\mathbb{R}^{4 \times 2}$ that predicts the four landmark coordinates for each input image. \textit{A figure in the supp. material further clarifies how the model generates landmark location heatmaps on different scales (Fig. 2).}


\subsection{Model Overview}
\label{sec: model_arch}
As shown in Figure~\ref{fig: overall_arch}, each input echo frame is represented by a hierarchical grid graph where each sub-graph corresponds to the input echo frame at a different resolution. The model produces heatmaps over both the main pixel-level task as well as the coarse auxiliary tasks. While the pixel-level heatmap prediction is of main interest, we use a hierarchical multi-level loss approach where the model's prediction over auxiliary tasks is used during training to optimize the model through comparisons to coarser versions of the ground truth. The intuition behind such an approach is that the model learns nuances in the data by performing landmark detection on the easier auxiliary tasks and uses this established reasoning when performing the difficult pixel-level task.

\subsection{Hierarchical Graph Construction}

\label{sec: graph_creation}
To learn representations that better capture the dependencies among pixels and patches, we introduce a hierarchical grid graph along with multi-level prediction tasks. 
As an example, the simplest task consists of a grid graph with only four nodes, where each node corresponds to four equally-sized patches in the original echo image. 
In the main task (the one that is at the bottom in Figure \ref{fig: overall_arch} and is the most difficult), the number of nodes is equal to the total number of pixels.

More formally, let us denote a graph as $G = (V, E)$, where $V$ is the set of nodes, and $E$ is the set of edges in the graph such that if $v_i, v_j \in V$ and there is an edge from $v_i$ to $v_j$, then $e_{i,j} \in E$.
To build hierarchical task representations, for each image $x \in X$ and the ground truth $y \in Y$, $K$ different auxiliary graphs $G_k(V_k, E_k)$ are constructed using the following steps for each $k \in [1,K]$:
\begin{enumerate}
    \item $2^k \times 2^k = 4^k$ nodes are added to $V_k$ to represent each patch in the image. Note that the larger values of $k$ correspond to graphs of finer resolution, while the smaller values of $k$ correspond to coarser graphs.
    \item Grid-like, undirected edges are added such that $e_{m-1, q}, e_{m+1, q}, e_{m, q-1}, e_{m, q+1} \in E_k$ for each $m, q \in [1 \dots 2^k]$ if these neighbouring nodes exist in the graph (border nodes will not have four neighbouring nodes).
    \item A patch feature embedding $z^k_j$, where $j \in [1 \dots 4^k]$ is generated and associated with that patch (node) $v_j \in V_k$. The patch feature construction technique is described in Section~\ref{sec: feat_gen}.
    \item Binary node labels $\hat{y}_k \in \{0, 1\}^{4^k \times 4}$ are generated such that $\hat{y}_{kj} = 1$ if at least one of the ground truth landmarks in $y$ is contained in the patch associated with node $v_j \in V_k$. Note that for each auxiliary graph, four different one-hot labels are predicted, which correspond to each of the four landmarks required to characterize LV measurements.
\end{enumerate}
The main graph, $G_\text{main}$, has a grid structure and contains $H \times W$ nodes regardless of the value of $K$, where each node corresponds to a pixel in the image. Additionally, to allow the model to propagate information across levels, we add inter-graph edges such that each node in a graph is connected to four nodes in the corresponding region in the next finer graph as depicted in \cref{fig: overall_arch}.

\subsection{Node Feature Construction}
\label{sec: feat_gen}
The graph representation described in Section~\ref{sec: graph_creation} is not complete without proper node features, denoted by $z \in \mathbb{R}^{|V|\times d}$, characterizing patches or pixels of the image. To achieve this, the grey-scale image is initially expanded in the channel dimension using a CNN. 
The features are then fed into a U-Net where the decoder part is used to obtain node features such that deeper layer embeddings correspond to the node features for the finer graphs.
This means that the main pixel-level graph would have the features of the last layer of the network. 
\textit{A figure clarifying node feature construction is provided in the supp. material (Fig. 1).}

\subsection{Hierarchical Message Passing}
\label{sec: intercom}
We now introduce how we perform message passing on our constructed hierarchical graph using GNNs to learn node representations for predicting landmarks. 

The whole hierarchical graph created for each sample, \ie, the main graph, auxiliary graphs, and cross-level edges, are collectively denoted as $G^i$, where $i\in [1, \dots, n]$. 
Each $G^i$ is fed into GNN layers followed by an MLP: 
\begin{align}
    h_{\text{nodes}}^{l+1} &= \text{ReLU}(\text{GNN}_l(G^i), h_{\text{nodes}}^l), \quad l \in [0, \dots, L] \\
    h_{\text{out}} &= \sigma(\text{MLP}(h_{\text{nodes}^{L+1}})) ,
\end{align}
where $\sigma$ is the Sigmoid function, $h_{\text{nodes}}^l \in \mathbb{R}^{|V_{G^i}|\times d}$ is the set of d-dimensional embeddings for all nodes in the graph at layer $l$, and $h_\text{out} \in [0,1]^{|V_{G^i}|\times 4}$ is the four-channel prediction for each node with each channel corresponding to a heatmap for each of the pixel landmarks.
The initial node features $h_{\text{nodes}}^1$ are set to the features $z$ described in Sections~\ref{sec: graph_creation} and \ref{sec: feat_gen}. The coordinates $(x_{\text{out}}^p, y_{\text{out}}^p)$ for each landmark location $p \in [1,2,3,4]$ are obtained by taking the expected value of individual heatmaps $h_\text{out}^p$ along the $x$ and $y$ directions such that:
\begin{align}
    x_{\text{out}}^p = \sum_{s=1}^{|V_{G^i}|} \text{softmax}(h_\text{out}^p)_s * \text{loc}_x(s) \label{eq: soft1},
\end{align}
where similar operations are performed in the y direction for $y_{\text{out}}^p$. Here, we vectorize the 2D heatmap into a single vector and then feed it to the softmax. 
$\text{loc}_x$ and $\text{loc}_y$ return the $x$ and $y$ positions of a node in the image. It must be noted that unlike some prior works such as Duffy \etal \cite{echonetlvh} that use post-processing steps such as imposing thresholds on the heatmap values, our work directly uses the output heatmaps to find the final predictions.


\subsection{Training and Objective Functions}\label{sec: objectives}
To train the network, we leverage two types of objective functions. 1) \emph{Weighted Binary Cross Entropy (BCE):} Since the number of landmark locations is much smaller than non-landmark locations, we use a weighted BCE loss; 2) 
 \emph{L2 regression of landmark coordinates:} We add a regression objective which is the L2 loss between the predicted coordinates and the ground truth labels.  

\section{Experiments}
\subsection{Datasets}
\label{sec: dataset}
\textbf{Internal Dataset:}
Our private dataset contains 29,867 PLAX echo frames, split in a patient-exclusive manner with 23824, 3004, and 3039 frames for training, validation, and testing, respectively.
\textbf{External Dataset:}
The public Unity Imaging Collaborative (UIC) \cite{uic} LV landmark dataset consists of a combination of 3822 end-systolic and end-diastolic PLAX echo frames acquired from seven British echocardiography labs. The provided splits contain 1613, 298, and 1911 training, validation, and testing samples, respectively. For both datasets, we down-sample the frames to a fixed size of $224 \times 224$.
\subsection{Implementation Details}
\label{sec: imp}
Our model creates $K$=7 auxiliary graphs. For the node features, the initial single-layer CNN uses a kernel size of 3 and zero-padding to output features with a dimension of $224\times224\times4$  ($C$=4). The U-Net's encoder contains $7$ layers with $128\times128, 64\times64, 32\times32, 16\times16, 8\times8, 4\times4$, and $2\times2$ spatial dimensions, and $8, 16, 32, 64, 128, 256$, and $512$ number of channels, respectively. Three Graph Convolutional Network (GCN)\cite{gcn} layers ($L=3$) with a hidden node dimension of 128 are used. To optimize the model, we use the Adam optimizer~\cite{adam} with an initial learning rate of 0.001, $\beta$ of (0.9, 0.999) and a weight decay of 0.0001, and for the weighted BCE loss, we use a weight of 9000. The model is implemented using PyTorch~\cite{pytorch} and Pytorch Geometric \cite{pyg} and is trained on two 32-GB Nvidia Titan GPUs. Our code-base is publicly available at \url{https://github.com/MasoudMo/echoglad}.

\subsection{Results}
\label{sec: quant_results}
We evaluate models using Mean Absolute Error (MAE) in mm, and Mean Percent Error (MPE) in percents, which is formulated as $\text{MPE} = 100\times\frac{|L_{\text{pred}} - L_{\text{true}}|}{L_{\text{true}}}$, where $L_{\text{pred}}$ and $ L_{\text{true}}$ are the prediction and ground truth values for every measurement. We also report the Success Detection Rate (SDR) for LVID for 2 and 6 mm thresholds. This rate shows the percentage of samples where the absolute error between ground truth and LVID predictions is below the specific threshold. These thresholds are chosen based on the healthy ranges for IVS (0.6-1.1cm), LVID (2.0-5.6cm), and LVPW (0.6-0.1cm). Hence, the 2 mm threshold provides a stringent evaluation of the models, while the 6 mm threshold facilitates the assessment of out-of-distribution performance.


\textbf{In-Distribution (ID) Quantitative Results.}
In \cref{tab:results1}, we compare the performance of our model with previous works in the ID setting where the training and test sets come from the same distribution (\eg, the same clinical setting), we separately train and test the models on the private and the public dataset. \textit{The results for the public dataset are provided in the supp. material (Table 1).}

\textbf{Out-of-Distribution (OOD) Quantitative Results.}
To investigate the generalization ability of our model compared to previous works, we train all models on the private dataset (which consists of a larger number of samples compared to UIC), and test the trained models on the public UIC dataset as shown in \cref{tab:results3}. 
Based on our visual assessment, the UIC dataset looks very different compared to the private dataset, thus serving as an OOD test-bed. 

\textbf{Qualitative Results.} \textit{Failure cases are shown in supp. material (Fig. 3).}

\begin{table}
% \scriptsize
\caption{\textbf{Quantitative results} on the private test set for models trained on the private training set. We see that our model has the best average performance over the three measurements, which shows the superiority of our model in the in-distribution setting for high-data regime.}
\label{tab:results1}
\begin{center}
\begin{tabular}{l|ccc|ccc|cc}

\multicolumn{1}{c|}{Model} &
  \multicolumn{3}{c|}{MAE {[}mm{]} $\downarrow$} &
  \multicolumn{3}{c|}{MPE {[}\%{]} $\downarrow$} &
  \multicolumn{2}{c}{SDR{[}\%{]} of LVID $<$ $\uparrow$} \\ 
%   \cline{2-7} 
\multicolumn{1}{c|}{} &
  \multicolumn{1}{c|}{LVID} &
  \multicolumn{1}{c|}{IVS} &
  LVPW &
  \multicolumn{1}{c|}{LVID} &
  \multicolumn{1}{c|}{IVS} &
  LVPW &
  \multicolumn{1}{c|}{2.0 mm} &
  6.0 mm \\ \midrule\midrule
Gilbert \etal \cite{gilbert2019automated}&
  \multicolumn{1}{c|}{2.9} &
  \multicolumn{1}{c|}{1.4} &
  1.4 &
  \multicolumn{1}{c|}{6.5} &
  \multicolumn{1}{c|}{14.5} &
  15.2 &
  \multicolumn{1}{c|}{48.1} &
  88.9\\ 
Lin \etal \cite{lin2021reciprocal} &
  \multicolumn{1}{c|}{9.4} &
  \multicolumn{1}{c|}{11.2} &
  9.0 &
  \multicolumn{1}{c|}{21.2} &
  \multicolumn{1}{c|}{116.5} &
  92.9 &
  \multicolumn{1}{c|}{26.0} &
  49.1 \\ 
McCouat \etal \cite{contour} &
    \multicolumn{1}{c|}{\textbf{2.2}} &
    \multicolumn{1}{c|}{1.3} &
    1.4 &
    \multicolumn{1}{c|}{\textbf{4.8}} &
    \multicolumn{1}{c|}{13.5} &
    15.1 &
    \multicolumn{1}{c|}{58.3} &
    93.9  \\ 
Chen \etal \cite{pyramid} &
    \multicolumn{1}{c|}{2.3} &
    \multicolumn{1}{c|}{1.2} &
    1.2 &
    \multicolumn{1}{c|}{5.2} &
    \multicolumn{1}{c|}{12.6} &
    13.8 &
    \multicolumn{1}{c|}{60.4} &
    92.6 \\ 
Duffy \etal \cite{echonetlvh} &
    \multicolumn{1}{c|}{2.5} &
    \multicolumn{1}{c|}{1.2} &
    1.2 &
    \multicolumn{1}{c|}{5.4} &
    \multicolumn{1}{c|}{13.2} &
    13.5 &
    \multicolumn{1}{c|}{52.1} &
    93.0 \\ 
Ours &
\multicolumn{1}{c|}{\textbf{2.2}} &
\multicolumn{1}{c|}{\textbf{1.1}} &
\textbf{1.1} &
\multicolumn{1}{c|}{\textbf{4.8}} &
\multicolumn{1}{c|}{\textbf{11.2}} &
\textbf{12.2} &
    \multicolumn{1}{c|}{\textbf{62.4}} &
    \textbf{94.4}
\\
\end{tabular}
\end{center}
\end{table}


\begin{table}
% \scriptsize
\caption{\textbf{Quantitative results} on the public UIC test set for models trained on the private training set. This table shows the out-of-distribution performance of the models when trained on a larger dataset and tested on a smaller external dataset. We can see that in this case, our model outperforms previous works by a large margin, which attests to the generalizability of our framework.}
\label{tab:results3}
\begin{center}
\begin{tabular}{l|ccc|ccc|cc}

\multicolumn{1}{c|}{Model} &
  \multicolumn{3}{c|}{MAE {[}mm{]} $\downarrow$} &
  \multicolumn{3}{c|}{MPE {[}\%{]} $\downarrow$} &
  \multicolumn{2}{c}{SDR{[}\%{]} of LVID $<$ $\uparrow$} \\ 
%   \cline{2-7} 
\multicolumn{1}{c|}{} &
  \multicolumn{1}{c|}{LVID} &
  \multicolumn{1}{c|}{IVS} &
  LVPW &
  \multicolumn{1}{c|}{LVID} &
  \multicolumn{1}{c|}{IVS} &
  LVPW &
  \multicolumn{1}{c|}{2.0 mm} &
  6.0 mm \\ \midrule\midrule
Gilbert \etal \cite{gilbert2019automated}&
  \multicolumn{1}{c|}{9.5} &
  \multicolumn{1}{c|}{4.8} &
  4.1 &
  \multicolumn{1}{c|}{23.5} &
  \multicolumn{1}{c|}{32.3} &
  26.8 &
\multicolumn{1}{c|}{22.5} &
52.2 \\ 
Lin \etal \cite{lin2021reciprocal} &
  \multicolumn{1}{c|}{51.5} &
  \multicolumn{1}{c|}{51.7} &
  41.3 &
  \multicolumn{1}{c|}{121.0} &
  \multicolumn{1}{c|}{375.8} &
  298.0 &
\multicolumn{1}{c|}{11.3} &
24.6 \\ 
McCouat \etal \cite{contour} &
    \multicolumn{1}{c|}{5.9} &
    \multicolumn{1}{c|}{3.6} &
    4.4 &
    \multicolumn{1}{c|}{18.5} &
    \multicolumn{1}{c|}{30.5} &
    36.4 &
\multicolumn{1}{c|}{34.6} &
72.3 \\ 
Chen \etal \cite{pyramid} &
    \multicolumn{1}{c|}{7.4} &
    \multicolumn{1}{c|}{5.3} &
    6.9 &
    \multicolumn{1}{c|}{22.5} &
    \multicolumn{1}{c|}{49.4} &
    62.4 &
\multicolumn{1}{c|}{28.9} &
65.3 \\ 
Duffy \etal \cite{echonetlvh} &
    \multicolumn{1}{c|}{13.7} &
    \multicolumn{1}{c|}{4.1} &
    5.5 &
    \multicolumn{1}{c|}{36.8} &
    \multicolumn{1}{c|}{36.4} &
    45.4 &
\multicolumn{1}{c|}{6.2} &
20.6 \\ 
Ours &
\multicolumn{1}{c|}{\textbf{5.8}} &
\multicolumn{1}{c|}{\textbf{2.8}} &
\textbf{4.3} &
\multicolumn{1}{c|}{\textbf{18.4}} &
\multicolumn{1}{c|}{\textbf{23.8}} &
\textbf{34.6} &
\multicolumn{1}{c|}{\textbf{35.8}} &
\textbf{\textbf{74.9}}
 \\
\end{tabular}
\end{center}
\end{table}

\begin{table}
  \caption{\textbf{Ablation results} on the validation set of our private dataset. Vanilla U-Net uses a simple U-Net model, while U-Net Main Graph only uses the pixel-level graph (no aux. graphs). Main Model is our proposed approach. Lastly, Single-Scale Loss has the same framework as the Main Model but only computes the loss for the model's predictions on the main graph (no multi-scale loss).}
  \label{tab:ablation1}
\begin{center}
\begin{tabular}{l|ccc}

\multicolumn{1}{c|}{Model} &
  \multicolumn{3}{c}{MPE {[}\%{]}} \\ 
%   \cline{2-7} 
\multicolumn{1}{c|}{} &
  \multicolumn{1}{c|}{LVID} &
  \multicolumn{1}{c|}{IVS} &
  LVPW \\ \midrule\midrule
Vanilla U-Net &
  \multicolumn{1}{c|}{5.31} &
  \multicolumn{1}{c|}{13.17} &
  13.47 \\ 
U-Net Main Graph &
  \multicolumn{1}{c|}{4.98} &
  \multicolumn{1}{c|}{11.67} &
  12.78 \\ 
Single-Scale Loss &
  \multicolumn{1}{c|}{5.41} &
  \multicolumn{1}{c|}{12.37} &
  12.8 \\ 
Main Model &
  \multicolumn{1}{c|}{\textbf{4.91}} &
  \multicolumn{1}{c|}{\textbf{11.45}} &
  \textbf{12.36} \\ 

\end{tabular}
\end{center}
\end{table}

\textbf{Ablation Studies.}
In Table \ref{tab:ablation1}, we show the benefits of a hierarchical graph representation with a multi-scale objective for the task of LV landmark detection. \emph{We provide a qualitative view of the ablation study in supp. material (Fig. 4).}

\section{Conclusion and Future Work}
In this work, we introduce a novel hierarchical GNN for LV landmark detection. The model performs better than the state-of-the-art on most measurements without relying on label smoothing. We attribute this gain in performance to two main contributions. First, our choice of representing each frame with a hierarchical graph has facilitated direct interaction between pixels at differing scales. This approach is effective in capturing the nuanced dependencies amongst the landmarks, bolstering the model's performance. Secondly, the implementation of a multi-scale objective function as a supervisory mechanism has enabled the model to construct a superior inductive bias. This approach allows the model to leverage simpler tasks to optimize its performance in the more challenging pixel-level landmark detection task.

For future work, we believe that the scalability of the framework for higher-resolution images must be studied. Additionally, extension of the model to video data can be considered since the concept of intra-scale and inter-scale edges connecting nodes could be extrapolated to include temporal edges linking similar spatial locations across frames. Such an approach could greatly enhance the model's performance in unlabeled frames, mainly through the enforcement of consistency in predictions from frame to frame. 

\bibliographystyle{splncs04}
\bibliography{ref}

\documentclass[10pt, a4paper,  aps, pra, showpacs, longbibliography, nofootinbib,superscriptaddress]{revtex4-2}
\pdfoutput=1
%\usepackage[utf8x]{inputenc}
\usepackage{ucs}
\usepackage{amsmath}
\usepackage{amsfonts}
\usepackage{amssymb}
\usepackage{mathtools}
\usepackage{makeidx}
\usepackage{cellspace,booktabs}
\usepackage{natbib}
\usepackage{lipsum}
\usepackage{bm}
\usepackage{bbm}
\usepackage{relsize}
%\usepackage{bibunits}
%begin hyperref setup
\usepackage{hyperref}
\hypersetup{colorlinks = true, linkcolor=magenta,citecolor=blue, urlcolor=magenta, bookmarksnumbered =  true}
%End hyperref setup
%\usepackage{subcaption}
%\usepackage[format=hang,justification=justified,singlelinecheck=false]{caption}

\usepackage[usenames,dvipsnames]{color}
\definecolor{light-gray}{gray}{0.55}

\usepackage{microtype}

\usepackage{graphicx}


\renewcommand{\dag}{^{\dagger}}
\newcommand{\ssm}{\rm\scriptscriptstyle}
\newcommand{\exv}[1]{ \langle #1 \rangle }

\newcommand{\bra}[1]{ \langle #1 \rvert }
\newcommand{\ket}[1]{ \lvert #1 \rangle}
\newcommand{\braket}[2]{\langle #1 \vert #2 \rangle }
\newcommand{\innerbraket}[3]{\langle #1 \vert #2 \vert #3 \rangle }
\newcommand{\tr}[2][]{\text{Tr}_{ #1 } ( #2 )}
\newcommand{\up}{\uparrow}
\newcommand{\down}{\downarrow}

%\newcommand{\overlr}[1]{\overleftarrow{\overrightarrow{#1}}}
\newcommand{\overlr}[1]{\overset{\leftrightarrows}{#1}}
\renewcommand{\overleftarrow}[1]{\overset{\leftarrow}{#1}}
\renewcommand{\overrightarrow}[1]{\overset{\rightarrow}{#1}}


\newcommand{\pfrac}[2]{\frac{\partial #1}{\partial #2}}
\newcommand{\intinf}{\int_{-\infty}^{\infty}}

\DeclareRobustCommand{\EG}[1]{{\color{blue}#1}}
\DeclareRobustCommand{\CKA}[1]{{\color{red}{#1}}}



\DeclareRobustCommand\VC[1]{{\color{ForestGreen}VC: #1 }}

\DeclareRobustCommand\AV[1]{{\color{Magenta}AV: #1 }}
\begin{document}
%TC:ignore
\widetext

\date{\today}
\author{Agnes Valenti}
\affiliation{Institute for Theoretical Physics, ETH Zurich, CH-8093, Switzerland}
\author{Vladimir Calvera}
\affiliation{Department of Physics, Stanford University, Stanford, CA 94305, USA}
\author{Steven A. Kivelson}
\affiliation{Department of Physics, Stanford University, Stanford, CA 94305, USA}
\author{Erez Berg}
\affiliation{Department of Condensed Matter Physics, Weizmann Institute of Science, Rehovot 76001, Israel}
\author{Sebastian D. Huber}
\affiliation{Institute for Theoretical Physics, ETH Zurich, CH-8093, Switzerland}

\title{Supplemental Material for ``Nematic metal in a multi-valley electron gas: \\ Variational Monte Carlo analysis and application to AlAs''}

\maketitle

\tableofcontents


\setcounter{equation}{0}
\setcounter{figure}{0}
\setcounter{page}{1}
\makeatletter
\renewcommand{\theequation}{S\arabic{equation}}
\renewcommand{\thefigure}{S\arabic{figure}}
\renewcommand{\bibnumfmt}[1]{[S#1]}
\renewcommand{\citenumfont}[1]{S#1}


\section{Approximations in the model Hamiltonian}
In our VMC simulations, we consider the Hamiltonian
\begin{align}
\label{eq:H}
H&=-\sum \limits_{i} \frac{1}{2m^{*}}\big(\eta^{\tau_i /2} \partial_{i,x}^2+ \eta^{-\tau_i /2} \partial_{i,y}^2 \big) + \sum \limits_{i<j} V(|{\bf r}_i- {\bf r}_j|), \\
V(|{\bf r}_i-{\bf r}_j|)&=\frac{1}{(2\pi)^2}\int {\rm d} {\bf q}\, {\rm e}^{i{\bf q} |{\bf r}_i-{\bf r}_j|}v({\bf q}), \\
v({\bf q})&= \frac{e^2}{2 \epsilon_0 \epsilon} \frac{\tanh (d |{\bf q}|)}{|{\bf q}|}.
\end{align}
where $m^{*}$ is the effective mass, $\eta$ denotes the anisotropy, the sum runs over all particles $i$ and $d $ corresponds to the gate-distance. 
In this section, we discuss neglected terms and made approximations in comparison to experiments on an AlAs quantum well \cite{hossain2021spontaneous}.


\subsection{Finite thickness}
A source that may affect the regions of stability in the phase diagram is the finite thickness of the AlAs quantum well. In~\cite{de2005effects} it has been shown, that while the finite thickness significantly alters the spin susceptibility, it only results in a slight shift of the phase boundaries.   

We neglect its effect here, but note that it still may be non-negligible as the results in~\cite{de2005effects} are obtained via perturbation theory and the estimation of phase boundaries is sensitive to small energy differences. In principle, finite-size thickness could be directly taken into account by performing simulations in a slab of finite thickness. This would, however, require increased computational effort. An estimation of the effects while keeping the simulations strictly two-dimensional can also be obtained via a simple approximation, using a device-specific form factor $F(q)$ that modifies the interaction $v(q)$ in Fourier space~\cite{de2005effects} to $\tilde{v}(q)=v(q)F(q)$, where $F(q=0)=1$ and $F(q\gg 1/w)\sim C/(wq)$ (here, $w$ is the width of the quantum well). For an AlAs quantum well, this form factor is given in~\cite{de2005effects, gold1987electronic}, with $C\approx 3$. 


\subsection{Valley-dependent interaction terms}
In this subsection, we consider the contribution of inter-valley scattering terms that we neglected in treating the valleys as separate isospin flavours. In particular, let us consider the interactions within the full model, that takes into account the complete Brillouin zone with valleys around $X=(2\pi/a,0)$ and $Y=(0,2\pi/a)$ where $a_{\text{AlAs}}= 566 \rm{pm}$ the AlAs lattice constant. Here, we took into account that AlAs has a ``zinc-blend'' structure. Then, the inter-valley terms of the form
\begin{align}
v_{\ssm i-v}=\frac{1}{2L^2}\sum_{{\bf q}\neq 0} \bigg[ \tilde{v}({\bf q}) \sum_{\bf k} a^{\dagger}_{{\bf k} + {\bf q}} a_{\bf k} \sum_{\bf k'} a^{\dagger}_{{\bf k'-q}} a_{\bf k'}\bigg], \\
{\bf k} \in \text{valley $X$}, \ \ {\bf k+q} \in \text{valley $Y$}, \nonumber \\
{\bf k'} \in \text{valley $Y$}, \ \ {\bf k'-q} \in \text{valley $X$} \nonumber
\end{align}
are neglected when valleys are treated as isospin flavours.

The strength of this interaction term goes as $\tilde{v}({\bf q})$ with $q \sim \sqrt{2} \frac{2\pi}{a}$, where $a$ is the lattice constant ($q$ needs to connect between the $X$  and the $Y$ point of the Brillouin zone). Since for $q$ much larger than $1/w$ and $1/d$, $\tilde{v}({\bf q})\propto  3/(wq^2)$, this contribution is very small in comparison to intra-valley terms that involve a momentum transfer of $q \sim k_F$. Concretely, at a density $n=10^{11} \rm{cm}^{-2}$ (corresponding to $r_s \approx 15.4$, with $k_F \approx \sqrt{2\pi \cdot 10^{15}} \rm{m}^{-1}$), and using $w=20\,\rm{nm}$~\cite{hossain2021spontaneous}, the valley- (and spin-) dependent interactions are smaller by a factor of
\begin{align}
%\gamma \approx 
\frac{\tilde{v}(2\sqrt{2}\pi/a)}{\tilde{v}(k_F)} \approx \frac{3 k_F a^2}{8\pi^2 w}
%\frac{\tanh(d \sqrt{2} \frac{2\pi}{a})}{\tanh(d k_F)} 
\sim \mathcal{O}(10^{-4}).
\end{align}
This is small compared to the typical energy differences between states of different spin or valley polarization found in our calculations, which are of the order of $10^{-2}$ of the total energy (see Fig. 2 of the main text). We therefore conclude that the inter-valley scattering terms can be neglected.

This estimation is not significantly affected by the considered metal-gate screening, since
\begin{align}
 \frac{\tanh (d2\sqrt{2} \pi/a)}{\tanh (d k_F )}\approx 1,
\end{align}
for $d=100$ nm.
% However, considering that the VMC-obtained energy differences are also of a similar order of magnitude (divided by the correlation energy), the valley dependent terms might have an influence on the phase boundaries. In order to determine which polarization they favour (sign of the interaction), we consider a simplified  picture.

% Concretely, we consider one electron in valley $X$ and one electron in valley $Y$. Neglecting other states, the inter-valley exchange interaction between these particles is given by~\cite{auerbach1998interacting}
% \begin{align}
% v_{\tau, \tau'}=J^{F}\sum_{\sigma \sigma'}c^{\dagger}_{\sigma \tau} c^{\dagger}_{\sigma' \tau'} c_{\sigma' \tau} c_{\sigma \tau'}.
% \end{align}
% The operator $c^{\dagger}_{\sigma \tau}$ ($c_{\sigma \tau}$) creates (annihilates) a state at valley $\tau$ with spin $\sigma$.
% Consequently, this interaction acts in the spin-space of the two occupied valleys at $X$ and $Y$. The coupling $J_F$ can be proven to be positive and real~\cite{auerbach1998interacting}.
% Defining the spin one half operators
% \begin{align}
% {\bf S}_{\tau}:=\frac{1}{2}\sum_{\sigma \sigma'} c^{\dagger}_{\sigma \tau} \vec{\sigma}_{\sigma \sigma'} c_{\sigma' \tau}
% \end{align}
% with $\vec{\sigma}$ the vector of Pauli matrices, we obtain the exchange interaction
% \begin{align}
% v_{\tau,\tau'}=-2J^{F}\big( {\bf S}_{\tau} \cdot {\bf S}_{\tau'}+\frac{1}{4}n_{\tau} n_{\tau'}\big),
% \label{eq:vintervalley}
% \end{align}
% with the valley occupation $n_{\tau}=\sum_{\sigma} c^{\dagger}_{\sigma \tau} c_{\sigma \tau}$.

% The first term in Eq.~(\ref{eq:vintervalley}) corresponds to ferromagnetic coupling (also denoted as Hund's coupling) that favours parallel alignment of spins in the two valleys. As a consequence, we expect the energy difference between the symmetric state and the spin-polarized state to be reduced - an existence of a spin-polarized phase even in the isotropic case, where the symmetric state is strongly favoured, is however questionable. The second term results in a lower energy, when two valleys are populated in comparison to one. Thus, it favorized valley-{\em unpolarized} states. This aligns with the comparison of our numerical VMC results with the experimental observations: In the experiment, the phase boundary between the symmetric state and VP is found at $r_s \approx 20$. In our simulations, we already find a transition at $r_s \approx 11$. We thus expect the discussed inter-valley contribution to shift this transition to larger $r_s$, closer to the experimental results.


\subsection{Screening}
In modifying the gate distance $d$ in between $d=70$ and $d=300$ nm, we only found phase boundary shifts of similar order of magnitude as their error bars. However, we note that e.g. a single-gate screened potential or no screening at all could lead to more significant effects.

The evaluation and implementation of the dual-gate screened potential is further detailed in Sec.~\ref{dual-gateV}.

\subsection{Electron-phonon interaction}

We now discuss the effect of electron-phonon coupling, neglected throughout
most of this work. Coupling to the lattice favors the valley-polarized
state over the spin-polarized state, since the valley order parameter
couples linearly to an orthorombic lattice distortion. However, since
the 2DEG is embedded in a three-dimensional material, the energy gain
due to the distortion may expected to be small. Here, we show that
for parameters relevant to the experiment in Ref. \cite{hossain2021spontaneous}, the energy
gain due to the lattice distortion is negligible compared to the energy
differences we find in the purely electronic model (where only Coulomb
interactions are taken into account).

We add to the electronic Hamiltonian of Eq. (\ref{eq:H}) the following terms:
\begin{equation}
\Delta H = H_{el-ph}+H_{ph}.
\end{equation}
The electron-phonon coupling, $H_{el-ph}$, is written as
\begin{equation}
H_{el-ph}=\int d^{2}r\,\frac{1}{2}E_2 \left(n_{X}-n_{Y}\right)\left(\epsilon_{xx}-\epsilon_{yy}\right),
\end{equation}
where $n_{X},$ $n_{Y}$ are the two-dimensional densities of electrons
in the two valleys, $E_2$ is the shear tetragonal deformation potential, and $\epsilon_{ij}$ is the strain tensor.
The phonon Hamiltonian includes the elastic energy (we shall ignore
the phonon dynamics for simplicity):
\begin{equation}
H_{ph}=\int d^{3}r\:\frac{1}{2}\left[C_{11}\left(\epsilon_{xx}^{2}+\epsilon_{yy}^{2}\right)+2C_{12}\epsilon_{xx}\epsilon_{yy}\right],
\end{equation}
where $C_{11}$ and $C_{12}$ are elastic constants (we omitted the
third elastic constant characterizing a cubic system, $C_{44}$, since
it plays no role in our discussion). We start by assuming that the
2DEG has an effective thickness $w$, and that the strain is essentially
uniform across the thickness. (In reality, the thickness of the semiconductor
is much larger than the thickness of the 2DEG, as we shall discuss
below). The energy of the fully valley polarized state is then found
to be
\begin{equation}
E=\int d^{2}r\,\left\{\frac{1}{2}E_2\, n(\epsilon_{xx}-\epsilon_{yy})+\frac{w}{2}\left[C_{11}\left(\epsilon_{xx}^{2}+\epsilon_{yy}^{2}\right)+2C_{12}\epsilon_{xx}\epsilon_{yy}\right]\right\}.
\end{equation}
Here, $n$ is the two-dimensional electron density. 

Minimizing this expression over $\epsilon_{xx}$ and $\epsilon_{yy}$,
we find that the energy gain per unit area due to coupling to the
lattice is
\begin{equation}
E_{el-ph}=-\frac{E_2^{2}n^{2}}{4w\left(C_{11}-C_{12}\right)}.
\end{equation}
The energy gain per electron is $E_{el-ph}/n$. For AlAs, $C_{11}-C_{12}\approx 63\,{\rm GPa}$
\cite{adachi1985gaas} and $E_2 \approx 5.8\,\rm{eV}$ \cite{Charbonneau1991MeasurementElasticAlAs,Gunawan2006ValleySusceptibility}.
Taking $n=10^{11}\,{\rm cm}^{-2}$,
we obtain $E_{el-ph}/n\sim -10^{-3}\,{\rm meV}$ per electron.
This is significantly smaller than the typical energy differences we find in
the electron-only model between the valley and spin polarized states
(see Fig. 2 of the main text). We conclude that the effects of coupling
to the lattice can be ignored.

In fact, the thickness of the semiconductor in the experiment is much
larger than $w$. The lattice distortion is not uniform across the
thickness of the semiconductor. Moreover, depending on the thickness,
it may be favorable to form domains of opposite valley polarization,
such that the distortion decreases in the bulk as a function of distance
from the 2DEG. In any case, the fact that the semiconductor is thicker
than $w$ only decreases the energy gain due to the lattice distortion,
compared to the naive estimate given above. 


\section{Optimization}

We use the variational principle
\begin{align}
E[\Lambda]:=\frac{\langle \Psi_{\Lambda} | \hat{H} | \Psi_{\Lambda} \rangle}{\langle \Psi_{\Lambda} | \Psi_{\Lambda} \rangle} \geq E_g,
\end{align}
where $E_g$ is the ground-state energy and $\Psi_{\Lambda}$ the trial wave-function with optimizable parameters ${\Lambda}$. By minimizing the expectation value of the Hamiltonian with respect to a given trial wave-function, one obtains a ground-state approximation.
Keeping the spin polarization fixed, the complexity in the evaluation of the above expectation value lies in a high-dimensional integral over all particle positions:
\begin{align}
\frac{\langle \Psi_{\Lambda} | \hat{H} | \Psi_{\Lambda} \rangle}{\langle \Psi_{\Lambda} | \Psi_{\Lambda} \rangle} = \frac{\int d{\bf r} d{\bf r' } \Psi_{\Lambda}^* ({\bf r}) \hat{H} \Psi_{\Lambda} ({\bf r'})}{\int d{\bf r} |\Psi_{\Lambda} ({\bf r})|^2 }= \frac{\int d{\bf r} |\Psi_{\Lambda} ({\bf r})|^2 H_L({\bf r}) }{\int d{\bf r} |\Psi_{\Lambda}({\bf r})|^2}.
\label{eq:Hexp}
\end{align}
Here, ${\bf r}=({\bf r}_1, ...{\bf r}_N)$, where ${\bf r}_i$ denotes the coordinates of particle $i$. 
We defined  the `local energy' $H_L({\bf r})=\Psi_{\Lambda}({\bf r})^{-1} \hat{H} \Psi_{\Lambda}({\bf r})$. The resulting integral can be efficiently estimated using Monte Carlo sampling: Instead of integrating over the complete Hilbert space, samples are drawn from the probability distribution $p\propto|\Psi_{\Lambda} ({\bf r})|^2$  using the Metropolis-Hastings algorithm. Then,
\begin{align}
\frac{\int d{\bf r} |\Psi_{\Lambda} ({\bf r})|^2 H_L({\bf r}) }{\int d{\bf r} |\Psi_{\Lambda}({\bf r})|^2} = \frac{1}{N_s} \sum_{{\bf r}_i \sim p}H_L({\bf r}_i)+\xi
\end{align}
and $\{ {\bf r}_i \}$ are the $N_s$ samples obtained via Metropolis Monte Carlo. The variance of the Gaussian-distributed statistical error $\xi$ with zero mean scales as $1/\sqrt{N_s}$, but vanishes when the Hamiltonian is evaluated and the trial wave-function corresponds to an eigenstate. Concretely, we can write this zero-variance property as \cite{casula2005new}
\begin{align}
\sigma^2=\sum_{{\bf r}_i \sim p}\bigg[H_L({\bf r}_i)-\bar{H}_L\bigg]^2 \geq 0, \\
\bar{H}_L=\frac{1}{N_s} \sum_{{\bf r}_i \sim p}H_L({\bf r}_i)
\end{align}
The equality holds when the wave-function corresponds to the ground-state (or an eigenstate), because then
\begin{align}
H_L({\bf r})=\Psi_{\Lambda}({\bf r})^{-1} \hat{H} \Psi_{\Lambda}({\bf r})=E_g, \ \ \ \bar{H}_L=E_g.
\end{align}
This zero-variance property allows for accurate estimation of the ground-state wave-function when an appropriate trial wave-function is used.

For minization of the variational energy $E[\Lambda]$, we use the stochastic reconfiguration technique introduced in
\cite{sorella1998green}, which can be understood as effective second order approximation to imaginary time evolution. 
The parameters $\Lambda$ of the trial wave-function are updated in every iteration as
\begin{align}
\Lambda \to \Lambda - \gamma S^{-1} F_{\Lambda},
\end{align}
where $\gamma$ is the learning rate and the force $F$ is given by
\begin{align}
F_k=2\bigg( \frac{\langle \partial_{\Lambda_k} \Psi_{\Lambda} | \hat{H}| \Psi_{\Lambda} \rangle}{\langle \Psi_{\Lambda}|\Psi_{\Lambda} \rangle}-E[\Lambda]\frac{\langle \partial_{\Lambda_k} \Psi_{\Lambda}| \Psi_{\Lambda} \rangle}{\langle \Psi_{\Lambda}|\Psi_{\Lambda} \rangle}\bigg).
\end{align}
The derivative with respect to the $k$-th variational parameter is denoted by $\partial_{\Lambda_k}$.

Second order effects are included by the covariance matrix \cite{park2020geometry}
\begin{align}
S_{k,k'}=\frac{\langle \partial_{\Lambda_k} \Psi_{\Lambda}| \partial_{\Lambda_{k'}}\Psi_{\Lambda} \rangle}{\langle \Psi_{\Lambda}|\Psi_{\Lambda} \rangle}-\frac{\langle \partial_{\Lambda_k} \Psi_{\Lambda}| \Psi_{\Lambda} \rangle  \langle \Psi_{\Lambda}| \partial_{\Lambda_{k'}}\Psi_{\Lambda} \rangle}{\langle \Psi_{\Lambda}|\Psi_{\Lambda} \rangle \langle \Psi_{\Lambda}|\Psi_{\Lambda} \rangle}
\end{align}
We employ the explicit regularization $S=S+\epsilon \mathbbm{1}$ in order to ensure invertibility. We use $\epsilon=10^{-3}$. At iteration $n_{\ssm it}$, we use the learning rate $\gamma={\rm max}(\gamma_0 \cdot 0.997^{n_{it}},\gamma_{\ssm min})$. The inital learning rate $\gamma_0$ is chosen (depending on density and system size) in the interval $[0.008,0.04]$ and $\gamma_{\ssm min} \in [0.0005,0.001]$.

\section{Trial wave-function}
The chosen parametrization of the trial wave-function, i.e. the concrete use of the parameters $\Lambda$, is of crucial importance to the accuracy of the results. The goal lies in representing the many-body correlated ground state wave-function as precisely as possible. At the same time, the trial wave function should also allow for efficient evaluation in order to keep the computational cost feasible.
In addition, antisymmetry has to be ensured for fermionic systems. An established route in ensuring antisymmetry lies in separating the wave-function into a product of a determinantal part, and a Jastrow factor
\begin{align}
\Psi_{\Lambda}({\bf r}_1, ...,{\bf r}_N)=\Psi_J({\bf r}_1, ...,{\bf r}_N)\cdot \Psi_D ({\bf r}_1, ...,{\bf r}_N), \label{eq:J_PsiD}
\end{align}
where the Jastrow factor $\Psi_J$ is chosen to be real,  positive and symmetric, such that the nodes and antisymmetry are fully determined by the determinantal part of the wave-function $\Psi_D$.
In particular, we employ a Slater-Jastrow-Backflow wave-function as elaborated in more detail below.

Before specifying more concretely the choice of Jastrow factor $J$ and determinantal part $\Psi_D$, we note here that in principle such a separation is not necessary. In particular, it has been shown that any totally antisymmetric wave-function can be represented as a single generalized determinant \cite{pfau2020ab, foulkesvariational}. The challenge lies then in finding the appropriate multi-electron orbitals. In \cite{pfau2020ab}, the authors have capitalized on deep neural networks as general function approximators \cite{bengio2017deep} to obtain these multi-electron orbitals. The resulting neural-network ansatz {\it Fermi Net} yields more accurate results for small atoms and molecules \cite{pfau2020ab} than any other wave-function to-date. However, a large computational cost is required to move to larger lattice sizes. 
For the isotropic electron gas, conventional trial wave-functions have been outperformed using a neural-network ansatz for systems with $27$ and $54$ electrons \cite{cassella2023discovering, wilson2022wave, li2022ab}.
The most scalable and accurate neural-network ansatz for periodic systems to-date constructed in \cite{pescia2023message} using message-passing neural networkds yields results for the $3D$ homogeneous electron gas up to $120$ electrons. While these consititute very promising results, many variational parameters and thus a comparably large computational cost is associated with neural-network quantum states for fermionic, continuous systems to-date. 
 
For the purpose of gaining physical insights about phases of multi-valley anisotropic systems, we are interested in probing the phase diagram as a function of $r_s$ and anisotropy on a dense grid. Thus, we require many simulations and use wave-functions of the product form \ref{eq:J_PsiD}, that are less accurate than the above mentioned wave-functions but only require little computational effort. 

We explain the form of our implemented trial wave-function and the use of the parameters $\Lambda$ in detail below.

\subsection{The Jastrow factor}
The Jastrow factor \cite{jastrow1955many} improves the many-body wave-function by effectively keeping electrons apart and creating a correlation hole.
We write the Jastrow prefactor of the wave-function as
\begin{align}
\Psi_J=e^{-J({\bf r}_1, ...{\bf r}_N)},
\end{align}
where the Jastrow factor $J$ is real and symmetric with respect to permutation of the particle positions $({\bf r}_1, ...{\bf r}_N)$.
While this non-negative bosonic prefactor \cite{holzmann2016theory} can in principle depend on all electron positions, in practice it is systematically constructed in a many-body expansion \cite{kim2018qmcpack}
\begin{align}
J=\frac{1}{2}\sum_{i,j} u_2 ({\bf r}_{i}, {\bf r}_{j}) +\frac{1}{6}\sum_{i,j, k} u_3 ({\bf r}_{i}, {\bf r}_{j }, {\bf r}_{k }) + ...,
\end{align}
with each $n$-body term being $u_n$ symmetric in the particle positions. Here, ${\bf }_{i}$ denotes the position of particle $i$ and the function $u_2$ ($u_3$) can depend on the isospins $\alpha_i, \alpha_j$ (and $\alpha_k$) of the particles $i$, $j$ (and $k$).

The two-body function $u_2$ has been approximated in early work as the random-phase-approximation (RPA) correlation function \cite{ceperley1978ground, kwon1993effects}. More expressive power is contained in the approximation of the two-body term as a (iso)spin-dependent liquid-like factor
\begin{align}
u_2({\bf r}_{i},{\bf r}_{j}) &=u_{\alpha_i, \alpha_j}(r_{ij}), \label{eq:u2}\\
r_{ij} &=|{\bf r}_{i}-{\bf r}_{i}|,
\end{align}
in combination with a polynomial expansion of $u_{\alpha_i, \alpha_i}(r_{ij})$ \cite{drummond2004jastrow, drummond2009phase} or a cubic Bspline interpolation as implemented in {\it qmcpack} \cite{kim2018qmcpack}:
\begin{align}
u_{\alpha_i \alpha_j}(r_{ij})=\sum \limits_{m=0}^M p_m B_3 \big( \frac{r_{ij}}{r_C /M}-m \big), \label{eq:ualphabeta}
\end{align}
where the cardinal cubic B-spline function $B_3(x)$ is centered at $x=-1$ and zero everywhere except on the interval $x\in [-3,1)$. The optimizable parameters are given by the $M$ control points $p_m$.
In order to comply with periodicity, the above parametrization is combined with cutting the Jastrow factor at the Wigner-Seitz radius $r_C$. Continuity of the wave-function and its first- and second derivatives is ensured by setting the last parameters $p_m$ to zero. The expressivity of the parametrization can be increased by increasing the number of control points $M$.

We note two issues with the above two-body Jastrow factor: First, a circular cutoff at the Wigner-Seitz radius limits the representability to short-range correlations, as the edges of the simulation cell are not included.
Second, the resulting correlation hole is isotropic. However, our Hamiltonian is anisotropic.

Both long-range correlations and anisotropic effects are typically encaptured by adding long-range periodic terms \cite{drummond2004jastrow}. In our case, we expect that the anisotropy in the kinetic energy of the Hamiltonian will have a relevant effect also on the short-range behaviour of the correlations. 
We thus propose the following two-body Jastrow factor to encapture both anisotropy and longer-ranged correlations
\begin{align}
u_2({\bf r}_{i},{\bf r}_{j})=u_{\alpha_i, \alpha_j}(r_{ij})+\nu_{\alpha_i, \alpha_j}({\bf r}_{i},{\bf r}_{j}).
\end{align}
Here, we separated the Jastrow factor into the short-ranged isotropic term $u$~(\ref{eq:ualphabeta}), and a second term $\nu$ allowing for anisotropy and correlations throughout the whole simulation cell. Any choice of $\nu$ that involves a cutoff and short-range anisotropy will either be restricted to very short-range correlations or lead to discontinuities as a consequence of periodic boundary conditions.
We thus resort to a different choice for the anisotropic term $\nu$. We combine anisotropy, cubic Bspline interpolation and a periodic ansatz presented in Ref.~\cite{whitehead2016jastrow}. In particular, we construct $\nu$ out of building blocks that already fulfill periodic boundary conditions \cite{whitehead2016jastrow}
\begin{align}
f_x({x}_{ij})=|{x}_{ij}|\big(1-2\frac{|{x}_{ij}|^3}{L_x^3}\big), \ \ 
f_y({y}_{ij})=|{y}_{ij}|\big(1-2\frac{|{y}_{ij}|^3}{L_y^3}\big),
\end{align}
with ${x}_{ij}=r^x_{i}-r^x_{j}$ being the x-component of the distance between particle $i$ and particle $j$. Respectively, $y_{ij}$ denotes the $y$-component. The length of the simulation cell in $x$ ($y$)-direction is given by $L_x$ ($L_y$), such that ${x}_{ij}\in (-L_x/2, L_x/2]$.  Here, we use periodic boundary conditions to define the distance within the Wigner-Seitz unit cell. The function $f_x$ is symmetric under $x\to -x$ and satisfies periodic boundary conditions at the edge of the simulation cell. More concretely
\begin{align}
f_x(L_x/2)=f_x(-L_x/2)\neq 0, \\
f'_x(L_x/2)=0, \\
f''_x(L_x/2)\neq 0. 
\end{align}
and vice versa for $f_y$. Thus, any function made out of $f_x$ and $f_y$ as building blocks automatically satisfies periodic boundary conditions.
In Ref.~\cite{whitehead2016jastrow} it was shown, that constructing a Jastrow factor out of a polynomial expansion of these building blocks can achieve lower energies for the homogenous electron gas than a combination of more conventional short-range and long-range Jastrow terms.

We make use of these building blocks $f_x$ and $f_y$ to design a Jastrow factor that can not only represent short-and long range correlations, but also anisotropy on all length scales. In addition, we find that numerical stability is improved using a combination with cubic Bspline interpolation instead of a polynomial expansion. We arrive at the Jastrow term 
\begin{align}\label{eq:parametrizationNugg}
\nu_{\alpha_i, \alpha_j}({\bf r}_{i},{\bf r}_{j}) &=A_1^{\alpha_i, \alpha_j}(g_1({\bf r}_{i},{\bf r}_{j}))+A_2^{\alpha_i, \alpha_j}(g_2({\bf r}_{i},{\bf r}_{j})), \\
g_1({\bf r}_{i},{\bf r}_{j})&=\sqrt{\lambda f_x^2 + (1-\lambda) f_y^2}, \label{eq:g1}\\
g_2({\bf r}_{i},{\bf r}_{j})&=\sqrt{(1-\lambda) f_x^2 + \lambda f_y^2}. \label{eq:g2}
\end{align}
Here, $A_1$ ($A_2$) corresponds to a cubic Bspline interpolation as in Eq.~(\ref{eq:ualphabeta}), defined on the image $g_1 \in [0,L_x/2]$ ($g_2 \in [0,L_y/2]$). No cutoff needs to be imposed here, and correlations in the whole simulation cell can be represented.
We used here that at small radius $r = \sqrt{x^2+y^2}$, $f_x(x)=|x|+\mathcal{O}(|x|^4)$, such that $(f^2_x(x)+f^2_y(y))^{1/2}=r+\mathcal{O}(|r|^4)$. The optimizable parameter $\lambda \in [0,1]$ therefore tunes the anisotropy: When $\lambda=0.5$, then the function $\nu$ is isotropic and $g_1({\bf r})=g_2({\bf r})=r+\mathcal{O}(|r|^4)$ for small $r$.

The used Jastrow factor components are isospin-dependent. With $\alpha=1, 2$ corresponding to spin-up and spin-down of valley $X$ and $\alpha=3, 4$ of valley $Y$, respectively, we captialize on the symmetries of the model by setting
\begin{align}
u_{11}=u_{22}=u_{33}&=u_{44},  \ \ \text{intra-valley, intra-spin} \label{eq:usym1}\\
u_{12}&=u_{34},  \ \ \text{intra-valley, inter-spin} \\
u_{13}=u_{14}=u_{23}&=u_{24}, \ \  \text{inter-valley.}
\end{align}
For the anisotropic $\nu$-term, we can make use of $90 \deg$ rotation between the two valleys and the parametrization of anisotropy detailed above, using $\lambda$. In particular, when denoting the parameter $\lambda$ explicitly in the function name by writing $\nu^{\lambda}$, we set
\begin{align}
\nu^{\lambda}_{11}=\nu_{22}^{\lambda}=\nu_{33}^{1-\lambda}=\nu_{44}^{1-\lambda}&, \ \  \text{intra-valley, intra-spin}\\
\nu_{12}^{\lambda}=\nu_{34}^{1-\lambda}&,  \ \ \text{intra-valley, inter-spin} \\
\nu_{13}^{\lambda}=\nu_{14}^{\lambda}=\nu_{23}^{\lambda}=\nu_{24}^{\lambda}&, \ \ \text{inter-valley}. \label{eq:vsymn}
\end{align}
Here, $\nu^{1-\lambda}$ corresponds to replacing the parameter $\lambda$ by $(1-\lambda)$ in the definition of $g_1$ and $g_2$ in Eqs.~(\ref{eq:g1})-(\ref{eq:g2}) and thereby effectively swapping $x$ and $y$.


% Figure environment removed

Our Jastrow factor uses Bspline interpolation with $8$-$10$ control points (depending on density and anisotropy). Further increasing the number of segments did not result in lower energies. Figure~\ref{fig:JastrowFactor} shows the Jastrow factor between two electrons of different isospin optimized for the VP and SP state. The deviation from spherical symmetry can be observed in form of anisotropy, implying $C_2$-symmetry (intra-valley correlation, left hand side of Fig.~\ref{fig:JastrowFactor}) and $C_4$-symmetry (inter-valley correlation, right hand side of Fig.~\ref{fig:JastrowFactor}).



\subsection{The determinantal part}
The nodal structure is solely determined by the determinantal part of the wave-function.
Thus, the parametrization of the determinantal part should either already correspond to the correct nodal structure, or allow for flexibility such that optimization yields an accurate approximation thereof.


As a natural first step of {\it guessing} the nodal structure, we set the determinantal part to a single Slater determinant
\begin{align}\label{eq:DeterminatalPartDef}
\Psi_D({\bf r}_1,...{\bf r}_N) &=\prod \limits_{\alpha}\det \big( {\bf D}^{\alpha}\big), \nonumber \\
{\bf D}^{\alpha} &=\big( D_{ij}^{\alpha} \big)_{\{i,j\}}, \nonumber \\
D_{ij}^{\alpha} &= \phi_j^{\alpha}({\bf r}_i).
\end{align}
Here, we used a common trick to reduce computational cost and factorized the wave-function into a product of Slater-determinants ${\bf D}^{\alpha}$ per isospin sector $\alpha=(\sigma, \tau)$. Although the anti-symmetry of the wave-function is lost and we thus wrote down an unphysical state, this apparent issue dissolves when computing any observable diagonal in isospin basis. 

The single-particle orbitals $\phi^{\alpha}_{j}({\bf r}_i)$  of the involved Slater-determinants can then be computed by e.g. a mean-field approach. For closed-shell systems, single-determinant wave-functions often provide good approximations of the nodal surface. There exist several common approaches to improve upon such a wave-function. For small systems, a multideterminant wave-function can provide excellent results. However, the number of required determinants quickly becomes excessively high for large system sizes \cite{rios2006inhomogeneous}.
Instead, we here make use of Backflow transformations \cite{lee1981green, kwon1993effects, kwon1998effects}.

Formally justified in the context of Fermi liquid theory and the homogeneous electron gas \cite{kwon1993effects}, backflow transformation are introduced by evaluating the single-particle orbitals $\phi_j^{\alpha}({\bf r}_i)$ at a set of collective coordinates $({\bf x}_i, ...{\bf x}_j)$. More concretely, the determinal part of the wave-function with backflow transformations is then given by \cite{rios2006inhomogeneous}
\begin{align}
\Psi^{BT}_D({\bf r}_1,...{\bf r}_N) =\Psi_D({\bf x}_1,...{\bf x}_N),
\end{align}
where the new coordinates are given by
\begin{align}
{\bf x}_i={\bf r}_i+{\bf \xi}_i ({\bf r}_1,...{\bf r}_N).
\end{align}

The backflow displacement is typically parametrized using two-body interparticle distances \cite{lee1981green, kwon1993effects, schmidt1981structure, rios2006inhomogeneous}
\begin{align}
\xi_i^{\alpha_i \alpha_j} =\sum \limits_{j \neq i}^{N} \eta^{\alpha_i \alpha_j}_{ij} {\bf r}_{ij}, \\  {\bf r}_{ij} = {\bf r}_{i} -{\bf r}_{j} \label{eq:backflow}
\end{align}
where the number of electrons is given by $N$ and $\eta_{ij}^{\alpha_i \alpha_j}=\eta^{\alpha_i \alpha_j}(r_{ij})$ is a function that depends on the distance between two electrons, and their isospins $\alpha_i$, $\alpha_j$. We capitalize on the symmetries of the model with respect to the isospin in the same way as for the Jastrow factor, see Eqs.~(\ref{eq:usym1})-(\ref{eq:vsymn}). If the considered system includes nuclei, additional electron-nucleus and electron-electron-nucleus terms are typically added. Here, we consider all-electron systems such that the term Eq.~\ref{eq:backflow} is sufficient. 

We parametrize the two-electron function $\eta_{ij}$ using the very generic Bspline interpolation form implemented in {\it qmcpack} \cite{kim2018qmcpack}
\begin{align}
\eta^{\alpha_i \alpha_j}(r_{ij})=\sum \limits_{m=0}^M p_m B_3 \big( \frac{r_{ij}}{r_C /M}-m \big), \label{eq:backflowParam}
\end{align}
where the cardinal cubic B-spline function $B_3(x)$ is centered at $x=-1$ and defined on the interval $x\in [-3,1)$. The optimizable parameters are given by the $M$ control points $p_m$. Throughout or simulations, we use $8$ control points.
While the function $\eta$ is expected to decay as $r_{ij}^{-5/2}$ (for the 2D isotropic electron gas) \cite{kwon1993effects}, the above parametrization implies cutting the backflow function smoothly at the cutoff radius $r_C$. The advantages of this cutoff lie in computational efficiency and compatibility with periodic boundary conditions.
We however note here, that one could in principle employ a similar parametrization as we detail above for the Jastrow factor \cite{whitehead2016jastrow} in order to maintain computational efficiency while at the same time allowing to take advantage of the whole simulation cell. However, as the long-range effects of the backflow transformation are not as relevant as the correlations represented by the Jastrow factor, we keep the simple form \ref{eq:backflowParam}.

\subsubsection{Effective anisotropy}
Beyond the evaluation of the single-particle orbitals at generalized coordinates, the choice and filling of orbitals represents another important degree of freedom in the parametrization. Since we study fluid phases in a periodic simulation cell, we use plane-wave orbitals. The filling of these orbitals is, unlike the isotropic case, not protected by any symmetry. In particular, the effective shape of the Fermi surface may be renormalized by interactions. Since our model is based on a parabolic dispersion relation and the effective mass approximation, we parametrize this interaction-driven reshaping of the Fermi surface with a single parameter $\tilde{\eta}_{\ssm FS}$, that defines an ellipse in the same way as the bare anisotropy $\eta$. We do not optimize $\tilde{\eta}_{\ssm FS}$ using stochastic reconfiguration together with the other parameters as it determines the filling of the orbitals. Instead, we scan through a range of $\tilde{\eta}_{\ssm FS}$ and pick the solution with lowest energy.  In order to keep the computational cost low, the optimal effective anisotropy for each density and anisotropy is found using a Slater-Jastrow wave-function. Selected tests throughout the phase diagram were performed to confirm that this value is not changed (within the precision of the discrete $\tilde{\eta}_{\ssm FS}$-grid) when Backflow transformations are added.

In all our simulations, the effective anisotropy $\tilde{\eta}_{\ssm FS}$ is assumed to be the same in both valleys. In the case of the considered state with partial polarization (3 of 4 Fermi pockets filled, with the same density each), we note that no symmetry formally justifies this choice. 
However, the energy as a function of effective anisotropy has a plateau around the minimum for all states except the SVP (see Fig.~3 of the main part), where the change in energy is small in comparison to the energy difference to states with other isospin polarization. In addition, this plateau is in the same regime of $\tilde{\eta}_{\ssm FS}$'s for all states except the SVP. Thus, we do not expect a significant effect by allowing for different effective anisotropies in both valleys.

%{\it Trial moves.} When variational Monte Carlo is used for a spin system on a lattice, a local trial move can simply consist of flipping one spin. Here, the situation is more involved since one has to account for the continuous nature of the problem. In particular, particles can move around freely in the simulation cell. For the spatial part of the sample, we thus consider a local single-particle update drawn from the distribution
%\begin{align}
%T({\bf r}_0, ..., {\bf r}_i \to {\bf r}'_i, ...{\bf r}_{N})=\frac{1}{4\pi \tau} \exp \bigg( -\frac{({\bf r}'_i -{\bf r}_i)^2 }{4\tau}\bigg),
%\end{align}
%where the `timestep' $\tau$ is chosen such that the correlation length of the created samples in minimal - as a rule of thumb, this amounts to setting $\tau$ such that the Metropolis Monte Carlo acceptance probability is around $0.5$ \cite{casinomanual}.
%We use the above single-particle trial moves but note that more efficient sampling can be achieved by more involved sampling schemes, such as e.g. including a drift in the determination of a trial move \cite{kim2018qmcpack}.

\section{Evaluation of the Hamiltonian}
The ground-state energy is estimated by minimizing the variational energy, which requires evaluating the expectation of the Hamiltonian.
We detail below the evaluation of the considered Hamiltonian
\begin{align}
H=-\sum \limits_{i} \frac{1}{2m^{*}}\big(\eta^{\tau_i/2} \partial_{i,x}^2+ \eta^{-\tau_i/2} \partial_{i,y}^2 \big) + \sum \limits_{i<j} V(|{\bf r}_i- {\bf r}_j|),
\end{align}
consisting of an anisotropic valley-dependent kinetic energy and a dual-gate screened potential. The above Hamiltonian is defined in free space (in the thermodynamic limit). We further explain below the evaluation of the Hamiltonian with simulations performed in a finite simulation cell with periodic boundary conditions.

\subsection{Kinetic energy}
We perform simulations where each electron has a fixed isospin, since the Hamiltonian does not mix isospins. This approach has the advantage of reduced computational cost, as we do not need to keep track of a spinor part of the wave-function. However, the resulting wave-function is evidently unphysical. In particular, in the construction of the trial wave-function it is only possible to anti-symmetrize within the same isospin sector when keeping the isospin of each particle fixed. This apparent unphysical behaviour of the wave-function can be resolved when considering expectation values: For isospin-independent operators, the expectation value is the same as  the expectation value of the fully antisymmetrized version of this wave-function. However, the kinetic part of the Hamiltonian
\begin{align}
H_{\ssm kin}=-\sum \limits_{i} \frac{1}{2m^{*}}\big(\eta^{\tau_i/2} \partial_{i,x}^2+ \eta^{-\tau_i/2} \partial_{i,y}^2 \big)
\label{eq:Hkin}
\end{align}
is valley-dependent. We show below that this dependence does not pose an obstacly to the evaluation of the Hamiltonian with a trial wave function that is not antisymmetric with respect to exchange between isospin sectors.
Since all parts of the Hamiltonian are spin-independent we will for readability only focus on the valley degree of freedom. In particular, we define the valley eigenstates $|+1\rangle$ (electron in valley $Y$) and $|-1\rangle$ (electron in valley $X$). These are eigenstates to the operator $\hat{\tau}^z$:
\begin{align}
\hat{\tau}|\bar{\tau}\rangle=\bar{\tau}|\bar{\tau}\rangle, \ \ \bar{\tau}\in \{ +1, -1 \}.
\end{align}
Since we consider many-particle states with fixed isospin polarization, we introduce $N_{\ssm X}$ ($N_{\ssm Y}$) as the number of particles in eigenstate $|+1\rangle$ (eigenstate $|-1\rangle$).


Let us now re-write the kinetic part of the Hamiltonian~(\ref{eq:Hkin}) in more abstract terms. 
We can write it as a sum of the $x$- and $y$ part
\begin{align}
H_{\ssm kin}=H_{{\ssm kin},x}+H_{{\ssm kin}, y}
\end{align}
Both parts are of the same structure. In the following, we will just consider $H_{{\ssm kin},x}$ for simplicity, but the same arguments hold for the $y$-part.
Generally, $H_{{\ssm kin},x}$ consists of single-particle operators of the form
$\hat{O}_i f(\hat{\tau}_i^z)$, where $\hat{O}_i$ is an isospin-independent operator $\partial_{i,x}^2$ acting on particle $i$. $f(\hat{\tau}_i^z)$ is a function that involves the operator $\hat{\tau}_i^z$, corresponding to $\hat{\tau}^z$ acting on particle $i$. Here, $f(\hat{\tau}_i^z)=\eta^{ \hat{\tau}_i^z}$.
Then,
\begin{align}
H_{{\ssm kin},x}=\sum_n \hat{O}_n f(\hat{\tau}_n^z).
\label{eq:Hkin_f}
\end{align}

Now, we consider the spacial part of our trial wave-function $\phi (r_1, \tau_1, ...,r_n, \tau_n)$. By fixing the valley degree of freedom for each particle we do not have to keep track of the spinor part of the wave-function during the simulation. Including this part however explicitly results in the complete form of the trial wave-function
\begin{align}
\Psi_{\Lambda}(r_1, ... r_N)=\phi (r_1, \bar{\tau}_1, ...,r_n, \bar{\tau}_n) \zeta (\bar{\tau}_1, ...\bar{\tau}_N), \\
\zeta (\bar{\tau}_1, ... \bar{\tau}_N)=|\bar{\tau}_1\rangle ... |\bar{\tau}_N \rangle, \\
\bar{\tau}_1= ... =\bar{\tau}_{N_{\ssm X}}=+1, \ \ \bar{\tau}_{N_{\ssm X}+1}= ... =\bar{\tau}_{N_{\ssm X}+N_{\ssm Y}}=-1
\end{align}
We have used the notation $\bar{\tau}$ to distinguish the isospin {\em eigenstate} (label) from the position {\em variable}. Further, we will assume that $\Psi_{\Lambda}$ is antisymmetric with respect to exchange within the same isospin sector, but not between sectors.
Antisymmetry between sectors can be explicitly enforced:
\begin{align}
\Psi_{\Lambda}^{AS}(r_1, ... r_N)=\frac{1}{\mathcal{\sqrt{N}}}\sum_i (-1)^{{\rm sgn} (P_i)} \phi (r_{P_i(1)}, \bar{\tau}_{P_i(1)}, ...,r_{P_i(N)}, \bar{\tau}_{P_i(N)}) \zeta (\tau_{P_i(1)}, ...\tau_{P_i(N)}).
\label{eq:psiAS}
\end{align}
The sum goes over all $\mathcal{N}$ permutations $P_i$ that exchange particles between isospin sectors. 

We now want to show that
\begin{align}
\frac{\langle \Psi_{\Lambda} | H_{{\ssm kin},x} | \Psi_{\Lambda} \rangle}{\langle \Psi_{\Lambda}| \Psi_{\Lambda}\rangle}\overset{!}{=}\frac{\langle \Psi_{\Lambda}^{AS} | H_{{\ssm kin},x } | \Psi_{\Lambda}^{AS} \rangle}{\langle \Psi_{\Lambda}^{AS}| \Psi_{\Lambda}^{AS}\rangle}.
\end{align}
Using $\langle \Psi_{\Lambda}^{AS}| \Psi_{\Lambda}^{AS}\rangle=\langle \Psi_{\Lambda}| \Psi_{\Lambda}\rangle$, the above equation simplifies to
\begin{align}
\langle \Psi_{\Lambda} | H_{{\ssm kin},x} | \Psi_{\Lambda} \rangle \overset{!}{=}\langle \Psi_{\Lambda}^{AS} | H_{{\ssm kin},x } | \Psi_{\Lambda}^{AS} \rangle 
\label{eq:eqshow}
\end{align}
Re-writing the right hand side of the above equation using Eqs.~(\ref{eq:Hkin_f}) and (\ref{eq:psiAS}) yields
\begin{align}
\langle \Psi_{\Lambda}^{AS} | H_{{\ssm kin},x } | \Psi_{\Lambda}^{AS} \rangle =&\frac{1}{\mathcal{N}}\int dr_1 ... d r_N \sum_i (-1)^{{\rm sgn} (P_i)}\phi (r_{P_i (1)}, \bar{\tau}_{P_i(1)}, ...,r_{P_i (N)}, \bar{\tau}_{P_i (N)}) \zeta (\bar{\tau}_{P_i(N)}, ...\bar{\tau}_{P_i(N)} ) \nonumber \\ 
&\times\sum_n \hat{O}_n f(\hat{\tau}_n^z)  \sum_j (-1)^{{\rm sgn} (P_j)}\phi (r_{P_j (1)}, \bar{\tau}_{P_j(1)}, ...,r_{P_j (N)}, \bar{\tau}_{P_j (N)}) \zeta (\bar{\tau}_{P_j(N)}, ...\bar{\tau}_{P_j(N)})  =(*)
\end{align}

Using the orthogonality of the spinor basisfunctions $\zeta$ and the fact that they are $z$-eigenstates such that $\hat{\tau}_n^z\zeta(\bar{\tau}_1 ...\bar{\tau}_N)= \bar{\tau}_n\zeta(\bar{\tau}_1 ...\bar{\tau}_N)$, we obtain
\begin{align}
(*) =&\frac{1}{\mathcal{N}}\sum_i\int dr_1 ... d r_N \phi (r_{P_i (1)}, \bar{\tau}_{P_i(1)}, ...,r_{P_i (N)}, \bar{\tau}_{P_i (N)}) \zeta (\bar{\tau}_{P_i(N)}, ...\bar{\tau}_{P_i(N)} ) \nonumber \\ 
&\times\sum_n \hat{O}_n f({\bar{\tau}}_n)  \phi (r_{P_i (1)}, \bar{\tau}_{P_i(1)}, ...,r_{P_i (N)}, \bar{\tau}_{P_i (N)}) \zeta (\bar{\tau}_{P_i(N)}, ...\bar{\tau}_{P_i(N)}).
\end{align}
Using commutativity of addition and the integration order we arrive at
\begin{align}
(*) =&\frac{1}{\mathcal{N}}\sum_i\int dr_{P_i (1)} ... d r_{P_{i}(N)} \phi (r_{P_i (1)}, \bar{\tau}_{P_i(1)}, ...,r_{P_i (N)}, \bar{\tau}_{P_i (N)}) \zeta (\bar{\tau}_{P_i(N)}, ...\bar{\tau}_{P_i(N)} ) \nonumber \\ 
&\times \big( \hat{O}_{P_i(1)} f({\bar{\tau}}_{P_i(1)}) + ...+  \hat{O}_{P_i(N)} f({\bar{\tau}}_{P_i(N)})\big)  \phi (r_{P_i (1)}, \bar{\tau}_{P_i(1)}, ...,r_{P_i (N)}, \bar{\tau}_{P_i (N)}) \zeta (\bar{\tau}_{P_i(N)}, ...\bar{\tau}_{P_i(N)}) \\
=&\int dr_{1} ... d r_{N} \phi (r_{1}, \bar{\tau}_{1}, ...,r_{N}, \bar{\tau}_{N}) \zeta (\bar{\tau}_{N}, ...\bar{\tau}_{N} )  \big( \hat{O}_{1} f({\bar{\tau}}_{1}) + ...+  \hat{O}_{N} f({\bar{\tau}}_{N})\big)  \phi (r_{1}, \bar{\tau}_{1}, ...,r_{N}, \bar{\tau}_{N}) \zeta (\bar{\tau}_{N}, ...\bar{\tau}_{N})
\end{align}
where we have renamed integration variables in the last step and thus demonstrated the equality~(\ref{eq:eqshow}).
It directly follows that
\begin{align}
\frac{\langle \Psi_{\Lambda} | H_{{\ssm kin}} | \Psi_{\Lambda} \rangle}{\langle \Psi_{\Lambda}| \Psi_{\Lambda}\rangle}=\frac{\langle \Psi_{\Lambda}^{AS} | H_{\ssm kin } | \Psi_{\Lambda}^{AS} \rangle}{\langle \Psi_{\Lambda}^{AS}| \Psi_{\Lambda}^{AS}\rangle}.
\end{align}


%Maybe short: Evaluation as explained in Casino manual.


\subsection{Dual-gate screened interaction}
\label{dual-gateV}

In a simulation cell with periodic boundary conditions, the interaction between electrons has to be evaluated as a sum over all image charges:
\begin{align}
V &=\sum_{\bf L}  \sum_{i<j} V(|{\bf r}_i-{\bf r}_j + {\bf L}|)+ V_{\ssm Mad}+ V_{\ssm b}, \label{eq:Vsum} \\
V_{\ssm Mad} &=N\sum_{\bf L \neq 0} V(|{\bf L}|)
\end{align} 
The sum runs over all lattice vectors ${\bf L}$ of the simulation cell. The Madelung energy $V_{\ssm Mad}$ is a constant contribution that arises from interactions between particles and their own images. Furthermore, we consider all-electron systems. The contribution of the positive background is given by $V_{\ssm b}$, and corresponds to
\begin{align}
V_{\ssm b}=-\frac{1}{2}v({\bf q}=0)n^2 L^2,
\end{align}
where $n$ is the density, $L^2$ the area of the simulation cell in two dimensions and $v({{\bf q}=0})$ the $q=0$ component of the Fourier transform of the potential $V$.

While the above expression~(\ref{eq:Vsum}) is diagonal in real-space and can thus in principle be directly computed, the sum over all lattice vectors does not converge when long-ranged interactions such as the Coulomb potential are considered. The divergence cancels out with the (also diverging contribution) of the positive background, which cannot be translated into a real-space cutoff in the sum over lattice vectors.
This challenge is typically solved by breaking the interaction in a part that is short-ranged in real space and a part that is short-ranged in reciprocal space, for instance using Ewald summation \cite{sangster1976interionic, toukmaji1996ewald}.

For comparison with experimental observations however, implementation of a potential that is externally screened  may yield a more realistic comparison. Considering a two-dimensional electron system, if the potential is externally screened by two metal gates the sum over lattice vectors converges due to the exponential decay in real-space of the resulting potential \cite{throckmorton2012fermions}. Thus, it can directly be computed in real space. In addition, the positive background can be directly substracted since $v({\bf q}=0)$  is finite. Below, we derive the real-space form of the dual-gate screened potential and explain the concrete implementation in our simulations.

Concretely, we consider a two-dimensional electron system with two metal gates above and below, separated by distance $2d$. The gates induce screening of the Coulomb interaction within the electron system. The form of the obtained screened potential is well-known in Fourier space
%V(|{\bf r}_i-{\bf r}_j|)=\int {\rm d} {\bf q}\, {\rm e}^{{\rm i}{\bf qr}}v(|{\bf q}|), \\
\begin{align}
v({\bf q})=  \frac{e^2}{2 \epsilon_0 \epsilon} \frac{\tanh (d |{\bf q}|)}{|{\bf q}|},
\end{align}
Performing VMC calculations however requires access to the potential in real space. The Fourier transform
\begin{align}
V(r)=\frac{1}{(2\pi)^2}\int_{\mathbb{R}^2} {\rm d} {\bf q}\, {\rm e}^{{\rm i}{\bf qr}}v(|{\bf q}|)
\end{align}
is not analytically solvable. Instead, we directly compute the potential $V(r)$ in real space using the method of images. Concretely, we will repeat here the derivation given in \cite{throckmorton2012fermions}.

Two metal gates introduce infinite ``columns'' of image charges above and below the sample.
Concreteley, the resulting interaction potential between two electrons is given by \cite{throckmorton2012fermions, goodwin2020critical}
\begin{align}
V(r)=\frac{e^2}{4\pi \epsilon \epsilon_0} \sum_{n=-\infty}^{n=\infty}\frac{(-1)^n}{\sqrt{r^2+(2dn)^2}}.
\end{align}
Now, we proceed to compute the above series: We re-rewrite the sum by using the identity
\begin{align}
\frac{1}{r}=\frac{2}{\sqrt{\pi}}\int_0^{\infty} du e^{-r^2 u^2}.
\end{align}
Then, we obtain
\begin{align}
V(r)&=\frac{2}{\sqrt{\pi}}\frac{e^2}{4\pi \epsilon \epsilon_0 2d} \sum_{n=-\infty}^{n=\infty}\int_0^{\infty} du(-1)^n e^{-(r/(2d))^2u^2 -n^2u^2} \\
&=\frac{2}{\sqrt{\pi}}\frac{e^2}{4\pi \epsilon \epsilon_0 2d} \int_0^{\infty} du e^{-(r/(2d))^2u^2} \sum_{n=-\infty}^{n=\infty}(-1)^n e^{ -n^2u^2} \\
&= \frac{2}{\sqrt{\pi}}\frac{e^2}{4\pi \epsilon \epsilon_0 2d} \int_0^{\infty} du e^{-(r/(2d))^2u^2} \theta_4 (0, e^{-u^2}),
\end{align}
where we used the Jacobi theta function
\begin{align}
\theta_4(z,q)=\sum_{n=-\infty}^{\infty} (-1)^n q^{n^2} e^{2niz}.
\end{align}
Making use of the identity \cite{wolframurl} 
\begin{align}
\theta_4(z,q)=\frac{2\sqrt{\pi}}{\sqrt{-\ln q}}e^{(4z^2+\pi^2)/4\ln q} \sum_{k=0}^{\infty} e^{k(k+1)\pi^2/\ln q} \cosh \bigg( \frac{(2k+1)\pi z}{\ln q}\bigg)
\end{align}
we insert
\begin{align}
\theta_4(0,e^{-u^2})=\frac{2\sqrt{\pi}}{u} \sum_{k=0}^{\infty} e^{-(k+1/2)^2\pi^2/u^2} 
\end{align}
and obtain \cite{throckmorton2012fermions}
\begin{align}
V(r)&=4\frac{e^2}{4\pi \epsilon \epsilon_0 2d}\sum_{k=0}^{\infty}  \int_0^{\infty} du \frac{1}{u} e^{-(r/(2d))^2u^2} e^{-(k+1/2)^2\pi^2/u^2}  \\
&=4\frac{e^2}{4\pi \epsilon \epsilon_0 2d}\sum_{k=0}^{\infty} K_0 \bigg( (2k+1) \pi \frac{r}{2d}\bigg),
\label{eq:Vrscreened}
\end{align}
where we have evaluated the integral using modified Bessel functions of the second kind $K_n(x)$. The behaviour of the potential at large distances can be understood using the large-$x$ limit of $K_n(x)$:
\begin{align}
K_n(x) \approx \sqrt{\frac{\pi}{2x}}e^{-x}.
\label{eq:Blarge}
\end{align}
Thus, at large distance $r$
\begin{align}
V(r) \approx 4\frac{e^2}{4\pi \epsilon \epsilon_0 2d}\sum_{k=0}^{\infty}\sqrt{\frac{1}{(2k+1) r/d}}e^{-(2k+1)\pi r/(2d)} \sim \frac{1}{\sqrt{r}} e^{-\pi r/(2d)},
\end{align}
since the most dominant contribution for $r\gg d$ comes from the $k=0$ contribution and all other contributions are exponentially smaller. The decay of the dual-gate screened potential is thus approximately exponential in real space at large distances (with sub-leading pre-factor $1/\sqrt{r}$).
For numerical evaluation, we make use of the exact formula~(\ref{eq:Vrscreened}). In particular, due to the long-range behaviour of the modified Bessel functions~(\ref{eq:Blarge}), the sum can be truncated at $k\sim \mathcal{O}(d/r)$ and numerically computed. For very small $r$ when the evaluation becomes infeasible, the effect of screening is at the same time negligible and the bare Coulomb interaction can be used instead. 
In order to keep the computational cost low and avoid the evaluation of the sum~(\ref{eq:Vrscreened}) during runtime, we pre-evaluate the screened potential on a dense grid and load the stored values during the VMC simulations. In particular, we use an interval $r\in(\epsilon, l)$ where $\epsilon$ is very small and sufficiently $l$ large such that $V(l) \approx 0$. The potential is then evaluated at arbitrary $r$ by piecewise linear interpolation, if $r$ is in the pre-evaluated interval. For $r\geq l$, the potential is simply set to zero. For $r<\epsilon$, it is fitted to an $1/r$ behaviour chosen such that the overall potential is continuous (almost exactly corresponding to the bare Coulomb interaction). Here, we use $\epsilon \sim 2\times 10^{-3}a$, where $a$ is the lattice constant of AlAs. The length $l$ is an integer multiple of the simulation cell length, chosen such that $V(l)<10^{-10}$ meV.  

Further, for typical densities and particle numbers the screening length will be larger than one simulation cell. Thus, we evaluate the potential by summing over image cells. This sum given in Eq.~(\ref{eq:Vsum}) quickly converges due to the exponential decay of the interaction. Thus, we can truncate it after a finite number of terms. Throughout most of the simulations, we used the distance $d=100$ nm. For this value, the number of image cells required for evaluation of the potential ranges between $\mathcal{O}(1)$ to $\mathcal{O}(10)$ for the simulated densities and particle numbers. 

Due to the exponential decay of the potential in real space, the Madelung energy $V_{\ssm Mad}$ becomes negligible already at finite, sufficiently large system sizes. We thus directly set it to zero during the simulations such that there is no necessity to account for it in the finite-size scaling analysis.

We have varied the screening length in between $d=70$ and $d=300$ nm (for selected densities within the anisotropy $\eta=5.79$) and found that the effect on the estimated phase boundaries is of similar order as ambiguities in the phase boundary estimation resulting from phenomenological fitting functions. It is however possible that a comparably much enhanced screening or no screening at all could have a more enhanced effect on phase diagram.

We note however, that the effect of the varied screening length on the finite-size scaling - thus, on the finite-size contributions - is striking: The exponent (as explained in the next section) of the finite-size fit strongly varies with screening length.



\section{Finite-size scaling}

The goal of realistic numerical many-body approaches such as quantum Monte Carlo methods lies in producing reliable ground state properties. In the vast majority of use-cases such as the simulation of (here 2D) bulk systems this corresponds to systems close to the thermodynamic limit. However, calculations are necessarily performed with a finite number of electrons: 
For a given density $n$, the simulations are we use a finite number of electrons $N$ in a simulation cell of area $L^2=n/N$. Thus, the arising finite-size errors have to be accounted for.
Despite the existence of sophisticated methods for the correction of finite-size errors \cite{drummond2008finite, kwee2008finite, holzmann2016theory}, a largely successful practice in numerical studies of condensed matter consists of extrapolating finite-size energies to infinite system size, using an assumed relationship between energy and particle number.
Within QMC simulations, this relation typically takes into account scaling behaviour arising from finite-size corrections on the kinetic energy as well as resulting from a compression of the exchange-correlation hole into the simulation cell.
For a Fermi fluid within the isotropic 2D electron gas without external screening, careful consideration of the contributions to the scaling behaviour has resulted in the two-parameter form
\begin{align}
e_{\inf}=e(N)+a \Delta t_{\ssm HF}(N)+cN^{-\gamma},
\end{align}
introduced in \cite{ceperley1980ground}. Here, $e_{\inf}$ corresponds to the energy in the thermodynamic limit and is used as a fitting parameter together with $a$ and $c$. The finite-size energies are given by $e(N)$. The exponent $\gamma$ has been discussed in literature, the first result $\gamma=3/2$ \cite{ceperley1980ground} has been later improved to the more accurate $\gamma=5/4$ in \cite{drummond2008finite}.
$\Delta t_{\ssm HF}(N)$ corresponds to the difference in the infinite and finite-size kinetic energy within Hartree-Fock, and captures the finite-size effects that arise from discrete filling of the Fermi surface in metallic states. This effect can also directly be alleviated by considering offsets $(\Delta_x,\Delta_y)$ in the (2D) $k$-grid, which can be understood as twists in the simulation cell boundary conditions
\begin{align}
\Psi(x+nL,y+mL)=e^{{\ssm} i(n\theta_x+m\theta_y)}\Psi(x,y),
\end{align}
with $\theta_x=\Delta_x L$ and $\theta_y=\Delta_y L$. Pure periodic boundary conditions correspond to the choice $(\Delta_x,\Delta_y)=(0,0)$.
A smarter choice of the twist or averaging over random offsets (``twist averaging'' \cite{lin2001twist}) can be used to achieve $\Delta t_{\ssm HF}(N)=0$ directly.
Computational cost can be saved by considering particular twists. Concretely, we here implement the ``special twist'' method introduced in \cite{dagrada2016exact}: Simulations for a given state and electron number are performed with the same offset, which is chosen such that $\Delta t_{\ssm HF}(N)=0$. This condition does not uniquely define the special twist, leaving an ambiguity in the choice. However, it has been shown that different choices in the special twist lead to very similar results, as long as the condition $\Delta t_{\ssm HF}(N)=0$ is fulfilled \cite{dagrada2016exact}. We here choose offsets along the diagonal $(k_x,k_y)=(\alpha,\alpha)$ in order two treat both valleys on equal footing. Concretely, for a given density, number of particles, anisotropy and Fermi sea filling the diagonal $(\alpha,\alpha)$ is scanned and the condition $\Delta t_{\ssm HF}(N)=0$ checked for each value of $\alpha$. If a value of $\alpha$ can be found where the condition is fulfilled (for a chosen particle number), it is used for a simulation of the considered trial wave-function. We note that a special twist does not exist for all particle numbers and Fermi sea fillings, in particular when the effective anisotropy of the Fermi surface varies strongly from the bare anisotropy. However, in all our simulations we found a sufficient amount of special twists to perform finite-size extrapolation as further detailed below.  
% We note however, that a different choice (e.g. along the $x$-direction) leads to very similar results [].

% Figure environment removed

We consider an anisotropic electron gas with dual-gate screened interaction and thus the exponent $\gamma=5/4$ derived for an unscreened potential does not apply here.
For very large lattice sizes, the exponential decay of the screened interactions is expected to result in all correlations decaying to zero within one simulation cell and thus a negligible finite-size error. For the considered lattice sizes however, the correlations surpass the size of the simulation cell and thus the regime of trivial behaviour in the finite-size error is not reached.
An accurate finite-size scaling behaviour requires careful consideration of the contributing corrections \cite{chiesa2006finite, drummond2008finite, holzmann2016theory}. We detail below that a rough estimation of the leading-order corrections is not sufficient for a well-behaved and reliable fit for the considered model, and instead extrapolate to the thermodynamic limit by including $\gamma$ as (anisotropy- and density-dependent) parameter that is fitted together with $e_{\inf}$ and $c$. The density-dependence of the exponent is effectively a result of finite-size-scaling contributions of different order with density-dependent prefactors. Despite the strongly phenomenological choice, the chosen fitting function yields a largely accurate fit when $r_s \lesssim 27$ (i.e. when the is system expected to be in a metallic phase, following the experimental predictions \cite{hossain2021spontaneous}) for the simulated system sizes, as shown exemplary in Fig.~\ref{fig:scaling} for $\eta=3$, $r_s \approx 15.4$. We note that we observe a more significant mismatch between simulated finite-size behavior and assumed phenomenological fit for the largest simulated $r_s \approx 30$, inducing a larger error in the finite-size scaling. We leave this observation to analyze in future studies. Further, we justify the phenomenological choice with our interest only in energy differences, which are less sensitive to the choice of finite-size extrapolation than absolute energies (with a sufficient amount of simulated particle numbers).

We detail below, the approximate estimation of contributions of the leading-order finite-size corrections and argue that they do not suffice for a reliable extrapolation.
As an instructing starting point for an estimate of the finite-size error on the potential $V_N$, we can write the electron-electron potential energy per particle for a simulation cell of area $L\times L$ containing $N$ electrons in Fourier space \cite{chiesa2006finite}
\begin{align}
V_N=\frac{1}{L^2}\sum \limits_{{\bf k}\neq 0} v({\bf k}) (\rho_{\bf k}\rho_{-\bf k}/N-1),
\label{eq:U_N}
\end{align}
where $\rho_{\bf k}:=\sum_j^N \exp (i {\bf k \cdot r}_j)$, and $v_k$ corresponds to the Fourier component of the dual-gate screened potential
\begin{align}
v({\bf k})=\frac{ e^2}{2 \epsilon_0 \epsilon}\frac{\tanh (d k)}{k},
\end{align}
with $k=|{\bf k}|$. The electron charge is given by $e$ and the metal-gate distance by $d$.
A finite-size error in Eq.~\ref{eq:U_N} is induced by the discrete $k$-mesh. As the system size increases, the mesh gets finer until the sum in (\ref{eq:U_N}) eventually converges to an inegral.

One can thus write the error on the potential energy per particle as the difference between integral and discrete sum
\begin{align}
\Delta V_N =\frac{1}{4\pi^2}\int v({\bf k}) (S({\bf k})-1) d{\bf k} - \frac{1}{L^2}\sum \limits_{\bf k \neq 0} v({\bf k}) (S_N({\bf k})-1),
\end{align}
where we used the static structure factor $S_N({\bf k})=\langle \rho_{\bf k} \rho_{-{\bf k}} \rangle/N$, and $S({\bf k})$ is the structure factor in the thermodynamic limit.

The leading order contribution is given by
\begin{align}
\Delta_1 =-\frac{1}{4\pi^2}\int v({\bf k}) d{\bf k}  + \frac{1}{L^2}\sum \limits_{\bf k \neq 0} v({\bf k}) .
\end{align}
 which is an integration error that arises due to the omission of the ${\bf k}=0$ area element from the sum. To leading order, the scaling of the error with number of particles can by estimated by considering this missing contribution, i.e. $\int_{\mathcal D} v_k d{\bf k}$, where $\mathcal{D}$ is a domain with area $4\pi^2/L^2$ \cite{chiesa2006finite}. However, $\Delta_1$ corresponds to the Madelung constant that we directly set to zero during our simulations, see Sec.~\ref{dual-gateV}. Thus, we do not need to account for the finite-size correction from the Madelung constant.


 
 The next-leading order correction comes from the discretization of $\int v({\bf k}) (S({\bf k})-1) d{\bf k}$ \cite{chiesa2006finite}. Approximating $S({\bf k})\approx S_N({\bf k})$ \footnote{This approximation is justified within the random-phase approximation. This becomes apparent when decomposing the potential into a long-range and short-range part. The long-range part that exhibits finite-size errors decays fast in reciprocal space. As a consequence, only the behaviour of $S({\bf k})$ at small $k$ is relevant. In the limit of $k\to 0$, the random phase approximation becomes exact}, one can use the same estimation as above for the integration error
\begin{align}
 \Delta_2 = \frac{1}{4\pi^2}\int v({\bf k}) S({\bf k})d{\bf k}  - \frac{1}{L^2}\sum \limits_{\bf k \neq 0} v_k S({\bf k})\propto \int_{\mathcal{D}}S({\bf k})v({\bf k}) d{\bf k}.
 \label{eq:Delta2}
 \end{align}
We note here that this estimation can not be applied for a quatitative computation of the finite-size correction, as it effectively implies neglecting a term of the same order of magnitude \cite{drummond2008finite}. We still apply the estimation, as we are only interested in the scaling behaviour. However we note that the scaling behaviour may also be affected by the missing contribution, as one would need to repeat the calculation in  \cite{drummond2008finite} for an externally screened potential.
 
Due to the validity of the random-phase approximation at small $k$, $S({\bf k})$ can be calculated analytically. For the isotropic 2D electron gas, $S({\bf k}) \propto k^{3/2}$ and $\Delta_2 \propto N^{-5/4}$. 
We here apply the random-phase approximation to compute the static structure factor $S({\bf k})$ in presence of a dual-gate screened interaction. We follow the derivation given in \cite{giuliani2005quantum} for the unscreened Coulomb potential and detail below only the differing steps using a dual-gate screened potential. Concretely, we make use of the relationship \cite{giuliani2005quantum}
\begin{align}
S({\bf k})=-\frac{\hbar}{\pi n}\int_0^{\infty} \mathcal{I}m \chi_{nn}( k,\omega) d\omega,
\label{eq:Sq}
\end{align}
where $\chi_{n n}(k,\omega)$ is the density-density response function. We use the approximation of the density-density response function obtained within the random-phase approximation
\begin{align}
\chi^{RPA}_{nn}(k,\omega)=\frac{\chi_0 (k,\omega)}{1-v_k \chi_0 (k,\omega)}.
\label{eq:chiRPA}
\end{align}
We introduced the notation $v_k :=v({\bf k})$ in order to simplify the consistency between vector and scalar objects.
Here, $\chi_0 (k, \omega)$ is the Lindhard function (density-density response function of independent electrons). 

In order to compute the imaginary part of the density-density response function, it is useful to search for its poles. We note here that the poles have also a direct physical meaning: The poles in the lower half of the complex frequency plane correspond to the frequencies of collective modes, giving rise to sharp peaks (resonances) in the spectral function. At long wavelength (small $k$), it is well-known that the spectrum of the density fluctuations within the random-phase approximation is dominanted by a collective excitation known as the {\it plasmon} \cite{giuliani2005quantum}.

We now turn to the calculation of the poles of $\chi^{RPA}_{nn}$.
They arise from zeros of the denominator of Eq.~(\ref{eq:chiRPA}), since the Lindhard function has no poles (only a branch cut along the real axis). Thus, the poles are given as solutions $\Omega_p$ of the equation
\begin{align}
1-v_k \chi_0 (k,\Omega_p)=0.
\label{eq:plasmonEq}
\end{align}
We use the small-$k$ expansion of the Lindhard function in two dimensions \cite{giuliani2005quantum} 
\begin{align}
\chi_0(k,\omega )\approx \frac{n k^2}{m \omega^2}[1+\frac{3}{4} \frac{k^2 v_F^2}{\omega^2}],
\end{align}
where $v_F$ is the Fermi velocity.
Then, the (real) solution to Eq.~(\ref{eq:plasmonEq}) is given by
\begin{align}
\Omega_p^2=\sqrt{\bigg(\frac{v_k n k^2}{2m}\bigg)^2+\frac{3}{4}k^4 v_F^2 \frac{v_k n}{m v_F}}+\frac{v_k n k^2}{2 m}.
\end{align}

We can then approximate the behaviour of the imaginary part of $\chi_{nn}^{RPA}(k,\omega)$ in the vicinity of the plasmon frequency $\Omega_p$ for $k\to 0$ as \cite{giuliani2005quantum}
\begin{align}
-\frac{1}{\pi}\mathcal{I}m \chi_{nn}^{RPA} (k,\omega) \approx \frac{\Omega_p(k)}{2 v_k} \delta (\omega-\Omega_p(k))
\end{align}
Inserting into Eq.~(\ref{eq:Sq}) yields
\begin{align}
S^{RPA}({\bf k})= \frac{\hbar}{n} \frac{\Omega_p (k)}{2 v_k}.
\end{align}
For the dual-gate screened interaction, the structure factor then approximately scales as $S({\bf k}) \propto k^{3/2} \tanh^{-1/2} (dk)$ in the small-$k$ limit.

We now want to determine the scaling of the correction $\Delta_2$ using the random-phase approximation of the structure factor. Inserting into Eq.~(\ref{eq:Delta2}), we obtain
\begin{align}
\Delta_2\propto  \int_0^{\frac{2\pi}{L}} dq \ 2\pi q S(q) v_q  \propto \int_0^{2\pi \sqrt{\frac{n}{N}}} q^{3/2}\tanh^{\frac{1}{2}}(dq). 
\end{align}
While this integral is not straightforward to evaluate, one can instead compute the integral numerically for a given density $n$ and gate-distance $d$ for a dense grid of values $1/N \in [0,1/N_{\ssm min}]$. We find that matching the obtained function with a fit $g(N)$ expanding in integer powers of $1/\sqrt{N}$ yields excellent agreement.
While we expect the anisotropy of the wave-function in principle also to enter in the finite-size correction, we did not include it in the above consideration due to the phenomenological observation that the scaling behaviour in our simulations is not very sensitive to the considered anisotropy.


Approximative scaling of the correction $\Delta_2$ is however not sufficient in order to obtain a reliable finite-size scaling. In particular, when employing the fit
\begin{align}
E_N=E_{\inf} - c  g(N),
\end{align}
the numerically obtained finite-size energies are not matched very well. In addition, we note that using an externally screened potential requires more careful consideration of the contributions to the finite-size error. More concretely, corrections from the discretized potential are much smaller than in the unscreened case: The potential is not divergent at small $k$, but finite. Thus, more subtle effects can become relevant such as further sub-leading corrections, dependence on the choice of trial wave-function \cite{holzmann2016theory} as well as higher-order corrections to the kinetic energy \cite{drummond2008finite}.


As we are further interested in energy differences, which are less sensitive to the choice of finite-size extrapolation than absolute energies (with a sufficient amount of simulated particle numbers), we choose to resort to the phenomenological fit $E_N=E_{\inf}-c N^{-\gamma}$ instead of more careful consideration of the finite-size corrections. The maximal particle numbers simulated are chosen with respect to the considered polarization, density and anisotropy. In particular, as shown in Fig.~\ref{fig:scaling}, the two states with lowest energy are simulated up to higher particle numbers (here $N=176$ and $N=162$ for the symmetric and VP state): Their energies are more relevant for the determination of the ground-state energy.
In principle, it is feasible to go simulate higher particle numbers $N>200$. Since we are constructing a phase diagram as a function of $r_s$ and $\eta$, and scan through a range of effective anisotropies at each point in the phase diagram, we perform our simulations with $N\leq 180$ to keep the overall computational cost manageable.

\section{Error bars}
In this section, we discuss the origin of the error bars on the phase boundaries (Fig. 1 in the main text) and on the extrapolated energies in the thermodynamic limit.

% Figure environment removed

\paragraph{Error bar on extrapolated energies.} The precision of the variational energy, extrapolated to infinite particle number is given by error propagation of the statistical sampling error on the finite-size errors through the fit $e(N)=e_{\inf}-a \Delta t_{\ssm HF}(N)+cN^{-\gamma}$. Ambiguities in the fitting parameters contribute via the covariance matrix of the fit, but the phenomenological choice of scaling behaviour introduces an error that is here not accounted for.

\paragraph{Error bar on phase boundaries.}
Simulations are performed on a discrete grid in the $r_s$-and $\eta$-plane: We simulate the anisotropies $\eta \in \{1,3,5.2,5.79,9,12\}$ and the Wigner-seitz radii $r_s \in \{10.9, 12.6, 15.4, 21.8, 30.8\}$ (corresponding to the densities $n \in \{2.0, 1.5, 1.0, 0.5, 0.25\}\times 10^{11}$ cm$^{-2}$. The extrapolated energies (for each simulated isospin polarization) are fitted on horizontal and vertical lines in the plane, i.e. along $r_s$ and along $\eta$.
In particular, we use the parameterization of the correlation energy suggested by Rapisarda and Senatore \cite{rapisarda1996diffusion} to perfom a fit of the energies as a function of $r_s$. Albeit derived for an unscreened potential, we find the parametrization to yield an excellent fit to our energies. As a function of $\eta$, we employ a generic polynomial fit. Error propagation is used to obtain uncertainties in the fitting parameters of the vertical and horizontal fit. The parameters are then varied within this range, and the minimal and maximal position of the respective phase boundary define the error on its estimation. For a fit along a horizontal line, this error corresponds to an uncertainty in $r_s$. Vice versa, a fit on a vertical line yields phase boundary uncertainties in $\eta$. Since the order of the used polynomial on vertical lines is somewhat arbitrary, we use polynomials with order $2,3,4$, determine the uncertainty of fitting parameters for each choice of polynomial and determine the error on the phase boundary using the minimal and maximal estimated phase boundary of {\em all} performed polynomial fits. Figure~\ref{fig:phasediagram2D_deg} demonstrates the dependence of the phase boundaries on the choice of fitting polynomial: The error bars in all $3$ subfigures are determined as detailed above. The colors, however, are obtained by first performing a fit in $r_s$ and evaluating this fit on a dense grid of $500$ values in $r_s$. Then, the so-obtained energies are interpolated with a polynom of degree $2$ (left of Fig.~\ref{fig:phasediagram2D_deg}), degree $3$ (middle) and $4$ (right) in $\eta$-direction. Evaluating the polynomial fit again with $500$ anisotropies yields energies on a $500\times 500$ grid. These energies can be used to color the phase diagram, thereby providing a guide to the eyes.

An additional error arises from the discretization in the scanned effective Fermi surface anisotropies $\tilde{\eta}_{\ssm FS}$. Figure~4 in the main manuscript shows that this error is rather small for the VP, SP and symmetric state as these showcase approximately an energy plateau around the minimum. However, the SVP state shows a stronger dependence on the effective anisotropy. Thus, more accurate results can be obtained by using a finer $\tilde{\eta}_{\ssm FS}$-grid.




\section{Hartree-Fock}
We perform Hartree-Fock calculations directly in the thermodynamic limit. Hartree-Fock can be understood as a variational method, where the trial wave-function corresponds to a ground state as a noninteracting Hamiltonian. Here, we consider only states with fixed isospin-polarization. Then, the mean-field trial wave-function $\Psi_M$ is a Slater-determinant and the optimizable degree of freedom corresponds to the orbital filling, i.e. the shape of the effective Fermi surface.
As effects beyond a parabolic dispersion are neglected in our system, we parametrize the effective Fermi surface as an ellipse with anisotropy $\tilde{\eta}_{\ssm HF}$. Then, we use the variational principle
\begin{align}
\frac{\langle \Psi_M |H| \Psi_M \rangle}{\langle \Psi_M | \Psi_M \rangle}\geq E_g
\label{eq:HFvar}
\end{align}
to obtain a ground-state approximation.
We explain in the following the concrete steps of our algorithm.
First, we determine the above expectation value~(\ref{eq:HFvar}) given a state $\Psi_M$ with effective anisotropy $\tilde{\eta}_{\ssm HF}$ and a given isospin-polarization $(n_1, n_2, n_3, n_4)$. We denote the density in isospin flavour $\alpha$ as $n_{\alpha}$, and $\sum_{\alpha} n_{\alpha}=n$, where $n$ is the total density of the system. We adopt the notation, that the states $i=1$ and $i=2$ correspond to the spin-up, spin-down states of valley $\tau=1$, and $i=3$, $i=4$ to the valley $\tau=-1$. %In order to improve the readability of the following calculation, we will introduce $Tau :=\tau/2$.


\subsection{Kinetic energy}
The kinetic energy for a state with isospin polarization $(n_1, n_2, n_3, n_4)$ is given by
\begin{align}
E_{\ssm kin}=\sum_{{\bf k}, \alpha} \hbar^2\frac{ \eta^{\tau_{\alpha}/2} k_x^2+ \eta^{-\tau_{\alpha}/2} k_y^2}{2 m^{*}}n_{{\bf k},\alpha}.
\end{align}
Here, the isospin flavour is denoted with index $\alpha$. In addition, $n_{{\bf k},\alpha}=1$ if orbital ${\bf k}, \alpha$ is filled (and $0$ otherwise).
In the thermodynamic limit, we can replace the sum with an integral $\sum_{\bf {k}}\to L^2/(2\pi)^2 \int d {\bf k}$ and write
\begin{align}
E_{\ssm kin}=\frac{L^2}{(2\pi)^2}\sum_{\alpha}\int d {\bf k} \hbar^2\frac{ \eta^{\tau_{\alpha}/2} k_x^2+ \eta^{-\tau_{\alpha}/2} k_y^2}{2 m^{*}}\Theta((k^{\alpha}_F)^2-\tilde{\eta}_{\ssm HF}^{\tau_{\alpha}/2}k_x^2-\tilde{\eta}_{\ssm HF}^{-\tau_{\alpha}/2}k_y^2),
\end{align}
where the Fermi wave-vector for the isospin flavour $\alpha$ is given as $k_F^{\alpha}=\sqrt{4\pi n_{\alpha}}$.
With the substitution 
\begin{align}
\tilde{k_x}=\tilde{\eta}_{\ssm HF}^{\tau_{\alpha}/4}k_x, \ \
\tilde{k_y}=\tilde{\eta}_{\ssm HF}^{-\tau_{\alpha}/4}k_y, \ \ 
\end{align}
we obtain
\begin{align}
E_{\ssm kin}&=\frac{L^2}{(2\pi)^2}\sum_{\alpha}\int d {\bf k} \hbar^2\frac{ (\eta/\tilde{\eta}_{\ssm HF})^{\tau_{\alpha}/2} \tilde{k}_x^2+ (\eta/\tilde{\eta}_{\ssm HF})^{-\tau_{\alpha}/2} \tilde{k}_y^2}{2 m^{*}}\Theta((k^{\alpha}_F)^2-\tilde{k}_x^2-\tilde{k}_y^2) \\
&=\frac{L^2}{(2\pi)^2}\sum_{\alpha}\int_0^{k_F^{\alpha}} d\tilde{k} \tilde{k}^3 \int_0^{2\pi}  d\theta   \hbar^2\frac{ (\eta/\tilde{\eta}_{\ssm HF})^{\tau_{\alpha}/2} \cos(\theta)^2+ (\eta/\tilde{\eta}_{\ssm HF})^{-\tau_{\alpha}/2} \sin(\theta)^2}{2 m^{*}} \\
&=\frac{L^2}{(2\pi)^2}\pi \hbar^2\frac{ (\eta/\tilde{\eta}_{\ssm HF})^{\tau_{\alpha}/2} + (\eta/\tilde{\eta}_{\ssm HF})^{-\tau_{\alpha}/2} }{2 m^{*}}\sum_{\alpha}\frac{1}{4}(k^{\alpha}_F)^4 \\
&=L^2\pi\hbar^2\frac{ (\eta/\tilde{\eta}_{\ssm HF})^{\tau_{\alpha}/2} + (\eta/\tilde{\eta}_{\ssm HF})^{-\tau_{\alpha}/2} }{2 m^{*}}\sum_{\alpha}n_{\alpha}^2.
\end{align}
Thus,
\begin{align}
\frac{E_{\ssm kin}}{N}=\frac{\pi\hbar^2}{n}\frac{ (\eta/\tilde{\eta}_{\ssm HF})^{\tau_{\alpha}/2} + (\eta/\tilde{\eta}_{\ssm HF})^{-\tau_{\alpha}/2} }{2 m^{*}}\sum_{\alpha}n_{\alpha}^2.
\end{align}

\subsection{Exchange energy}
We adapt the derivation in \cite{giuliani2005quantum} of the exchange energy of an isotropic, two-dimensional electron gas for our case. The exchange energy is given by %Eq. (1.92) QTEL
\begin{align}
E_x=-\frac{1}{2L^2}\sum_{{\bf {q}}\neq 0} v_{\bf {q}} \sum_{{\bf k}\alpha}n_{{\bf k}+{\bf q} \alpha}n_{{\bf k} \alpha},
\end{align}
where $\alpha$ corresponds to the isospin flavour (spin $\sigma_{\alpha}$ and valley $\tau_{\alpha}$).


We proceed by computing the exchange energy.
Replacing the summation by integration (thermodynamic limit, $\sum_{{\bf k}}\to L^2/(2\pi)^2 \int d {\bf k}$) and using
\begin{align}
n_{{\bf k}\alpha}=\Theta((k^{\alpha}_F)^2-\tilde{\eta}_{\ssm HF}^{\tau_{\alpha}/2}k_x^2-\tilde{\eta}_{\ssm HF}^{-\tau_{\alpha}/2}k_y^2),
\end{align}
we obtain
\begin{align}
E_x= -\frac{L^2}{2(2\pi)^{4}}\sum_{\alpha}\int d {\bf q} \int d {\bf k} v_{{\bf q}} &\Theta((k^{\alpha}_F)^2-\tilde{\eta}_{\ssm HF}^{\tau_{\alpha}/2}(k_x+q_x)^2-\tilde{\eta}_{\ssm HF}^{-\tau_{\alpha}/2}(k_y+q_y)^2)    \nonumber\\
&\times\Theta((k^{\alpha}_F)^2-\tilde{\eta}_{\ssm HF}^{\tau_{\alpha}/2}k_x^2-\tilde{\eta}_{\ssm HF}^{-\tau_{\alpha}/2}k_y^2).
\end{align}
We can now make the following change of integration variables:
\begin{align}
\tilde{k}_x=\tilde{\eta}_{\ssm HF}^{\tau_{\alpha}/4}k_x, \ \
\tilde{k}_y=\tilde{\eta}_{\ssm HF}^{-\tau_{\alpha}/4}k_y, \ \ 
\tilde{q}_x=\tilde{\eta}_{\ssm HF}^{\tau_{\alpha}/4}q_x, \ \
\tilde{q}_y=\tilde{\eta}_{\ssm HF}^{-\tau_{\alpha}/4}q_y.
\end{align}
With $d\tilde{k}_x d\tilde{k}_y=dk_x dk_y$ (and likewise for $d {\bf q}$), we arrive at
\begin{align}
E_x= &-\frac{L^2}{2(2\pi)^{4}}\sum_{\alpha}\int d {\bf \tilde{q}} v_{{\bf q}}\int d {\bf \tilde{k}}  \Theta((k^{\alpha}_F)^2-(\tilde{k}_x+\tilde{q}_x)^2-(\tilde{k}_y+\tilde{q}_y)^2)    \nonumber\\
&\times\Theta((k^{\alpha}_F)^2-\tilde{k}_x^2-\tilde{k}_y^2),
\end{align}
where 
\begin{align}
v({\bf q})=v(\tilde{\eta}_{\ssm HF}^{-\tau_{\alpha}/4}\tilde{q}_x,\tilde{\eta}_{\ssm HF}^{\tau_{\alpha}/4}\tilde{q}_y).
\end{align}
We use that \cite{giuliani2005quantum} %(1.93) QTEL, careful sign mistake for 2D
\begin{align}
\frac{1}{N_{\alpha}}\sum_{{\bf k}} n_{{\bf k}+{\bf q} \alpha} n_{{\bf k}\alpha}=1-\frac{2}{\pi}\bigg[\arcsin (\frac{q}{2k^{\alpha}_F})+\frac{q}{2k^{\alpha}_F}\sqrt{1-\big(\frac{q}{2k^{\alpha}_F}\big)^2} \ \bigg]
\end{align}
for $q<2k^{\alpha}_F$ and $0$ otherwise. Here, $q=|{\bf q}|$.

% Figure environment removed

Then, we obtain
\begin{align}
E_x=-\frac{1}{2(2\pi)^{2}}\sum_{\alpha}\int_0^{2k^{\alpha}_F} d \tilde{q} \tilde{q} \int_0^{2\pi} d\theta v_{{\bf q}} N_{\alpha} \bigg(1-\frac{2}{\pi}\bigg[\arcsin (\frac{\tilde{q}}{2k^{\alpha}_F})+\frac{\tilde{q}}{2k^{\alpha}_F}\sqrt{1-\big(\frac{\tilde{q}}{2k^{\alpha}_F}\big)^2}\bigg] \bigg) .
\end{align}
Substituting $z=\tilde{q}/(2k^{\alpha}_F)$ and using that
\begin{align}
v({\bf q}) &=\frac{e^2}{2\epsilon_0\epsilon}\frac{\tanh(d\sqrt{q_x^2+ q_y^2})}{\sqrt{q_x^2+q_y^2}} \\
&=\frac{e^2}{2\epsilon_0\epsilon}\frac{\tanh(d\sqrt{\tilde{\eta}_{\ssm HF}^{-\tau_{\alpha}/2}\tilde{q}_x^2+\tilde{\eta}_{\ssm HF}^{\tau_{\alpha}/2}\tilde{q}_y^2})}{\sqrt{\tilde{\eta}_{\ssm HF}^{-\tau_{\alpha}/2}\tilde{q}_x^2+\tilde{\eta}_{\ssm HF}^{\tau_{\alpha}/2}\tilde{q}_y^2}} \\
&=\frac{e^2}{4 \epsilon_0\epsilon zk_F^{\alpha}}\frac{\tanh(2dzk_F^{\alpha}\sqrt{\tilde{\eta}_{\ssm HF}^{-\tau_{\alpha}/2}\cos(\theta)^2+\tilde{\eta}_{\ssm HF}^{\tau_{\alpha}/2}\sin(\theta)^2})}{\sqrt{\tilde{\eta}_{\ssm HF}^{-\tau_{\alpha}/2}\cos(\theta)^2+\tilde{\eta}_{\ssm HF}^{\tau_{\alpha}/2}\sin(\theta)^2}}.
\end{align}
Inserting into the exchange energy and dividing by the particle number, we obtain
\begin{align}
\frac{E_x}{N} = &-\sum_{\alpha}\frac{e^2 k_F^{\alpha}}{(2\pi)^{2}{2\epsilon_0\epsilon}}\frac{n_{\alpha}}{n}\int_0^{1} d z \int_0^{2\pi} d\theta  \frac{\tanh(2dzk_F^{\alpha}\sqrt{\tilde{\eta}_{\ssm HF}^{-\tau_{\alpha}/2}\cos(\theta)^2+\tilde{\eta}_{\ssm HF}^{\tau_{\alpha}/2}\sin(\theta)^2})}{\sqrt{\tilde{\eta}_{\ssm HF}^{-\tau_{\alpha}/2}\cos(\theta)^2+\tilde{\eta}_{\ssm HF}^{\tau_{\alpha}/2}\sin(\theta)^2}}  \\
 &\times \bigg(1-\frac{2}{\pi}\bigg[\arcsin (z)+z\sqrt{1-\big(z\big)^2}\bigg]\bigg) .
\end{align}
The above integral can be easily computed numerically for a given isospin polarization $(n_1, n_2, n_3, n_4)$. 




\subsection{Hartree-Fock: Results}
We minimize $E/N=E_{\ssm kin}/N+E_x/N$ for the optimal effective anisotropy $\tilde{\eta}_{\ssm HF}$, for different isospin polarizations as a function of density $n$.


% Figure environment removed

Figure~\ref{fig:HF} shows the Hartree-Fock energies as a function of $r_s$ for states of different isospin polarization. We observe several attributes of the phase diagram that conflict with the experimental observations on AlAs: First, the valley-polarized, spin-unpolarized state (VP) and the spin-polarized, valley-unpolarized state (SP) are degenerate and thus mean-field cannot explain the experimental results on AlAs. 
The accidental degeneracy between VP and SP is a result of the form of the exchange energy: Only intra-valley terms contribute.
Second, the polarization transitions occur at much higher densities than observed in the experiment (VP in experiment at $r_s \approx 20$). Third, a phase with a partially polarized ground state with $3$ filled Fermi pockets exists within Hartree-Fock, which is also not observed in the experiment. The phase diagram as a function of anisotropy is given in Figure~\ref{fig:HFeta}. Here, we observe that the phase boundaries are rather insensitive to a change in anisotropy within Hartree-Fock, as opposed to VMC.

Above, we considered states with fixed isospin polarization. A  complete analysis that accounts for effects beyond polarized phases such as inter-valley coherence, requires including the possibility of isospin mixing. We will leave this analysis for future studies.


\bibliographystyle{unsrt}
\bibliography{papers.bib}

%TC:endignore


\end{document}


\end{document}
