% % This is samplepaper.tex, a sample chapter demonstrating the
% % LLNCS macro package for Springer Computer Science proceedings;
% % Version 2.20 of 2017/10/04
% %
% \documentclass[runningheads]{llncs}
% %
% \usepackage{graphicx}
% \usepackage{amsmath}
% \usepackage{amssymb}
% \usepackage{booktabs}
% \usepackage{rotating}
% \usepackage{makecell}
% \usepackage{xspace}
% \usepackage{xcolor}
% \usepackage[export]{adjustbox}
% \usepackage{multirow}
% \usepackage{float}
% \usepackage{pifont}
% \usepackage{caption}
% \usepackage{subcaption}
% \usepackage{multirow}
% % It is strongly recommended to use hyperref, especially for the review version.
% % hyperref with option pagebackref eases the reviewers' job.
% % Please disable hyperref *only* if you encounter grave issues, e.g. with the
% % file validation for the camera-ready version.
% %
% % If you comment hyperref and then uncomment it, you should delete
% % ReviewTempalte.aux before re-running LaTeX.
% % (Or just hit 'q' on the first LaTeX run, let it finish, and you
% %  should be clear).
% \usepackage[pagebackref,breaklinks,colorlinks]{hyperref}


% \newcommand{\masoud}[1]{{\color{green}{[Masoud: #1]}}}
% \newcommand{\placeholder}[1]{{\color{red}{[X]}}}
% \newcommand{\neda}[1]{{\color{blue}{[Neda: #1]}}}

% \makeatletter
% \DeclareRobustCommand\onedot{\futurelet\@let@token\@onedot}
% \def\@onedot{\ifx\@let@token.\else.\null\fi\xspace}
% \def\iid{\emph{i.i.d}\onedot} \def\IID{\emph{I.I.D}\onedot}
% \def\eg{\emph{e.g}\onedot} \def\Eg{\emph{E.g}\onedot}
% \def\ie{\emph{i.e}\onedot} \def\Ie{\emph{I.e}\onedot}
% \def\cf{\emph{c.f}\onedot} \def\Cf{\emph{C.f}\onedot}
% \def\etc{\emph{etc}\onedot} \def\vs{\emph{vs}\onedot}
% \def\wrt{w.r.t\onedot} \def\dof{d.o.f\onedot}
% \def\aka{\emph{a.k.a}\onedot}
% \def\etal{\emph{et al}\onedot}
% \makeatother

% \newcommand*\samethanks[1][\value{footnote}]{\footnotemark[#1]}

% % Support for easy cross-referencing
% \usepackage[capitalize]{cleveref}
% \crefname{section}{Sec.}{Secs.}
% \Crefname{section}{Section}{Sections}
% \Crefname{table}{Table}{Tables}
% \crefname{table}{Tab.}{Tabs.}
% \crefname{subtable}{Fig.}{Figures}

% \begin{document}

\title{Supplementary Material}
%
% If the paper title is too long for the running head, you can set
% an abbreviated paper title here
%
\author{Masoud Mokhtari et al.}
%
\authorrunning{M. Mokhtari et al.}
% First names are abbreviated in the running head.
% If there are more than two authors, 'et al.' is used.
%
\institute{Electrical and Computer Engineering, University of British Columbia,
Vancouver, BC, Canada}
%
\maketitle              % typeset the header of the contribution

% Figure environment removed

\begin{table}[h!]
% \scriptsize
\caption{\textbf{Quantitative results} on the public UIC test set for models trained on the UIC training set. Although the number of training samples is much lower for UIC compared to our private dataset, we see that our model still outperforms previous works on average over the three measurements, which showcases the accuracy of our model in the low-data regime and in-distribution settings. Lin \etal is excluded since they require video inputs.}
\label{tab:results2}
\begin{center}
\begin{tabular}{l|ccc|ccc|cc}

\multicolumn{1}{c|}{Model} &
  \multicolumn{3}{c|}{MAE {[}mm{]} $\downarrow$} &
  \multicolumn{3}{c|}{MPE {[}\%{]} $\downarrow$} &
  \multicolumn{2}{c}{SDR{[}\%{]} of LVID $<$ $\uparrow$} \\ 
%   \cline{2-7} 
\multicolumn{1}{c|}{} &
  \multicolumn{1}{c|}{LVID} &
  \multicolumn{1}{c|}{IVS} &
  LVPW &
  \multicolumn{1}{c|}{LVID} &
  \multicolumn{1}{c|}{IVS} &
  LVPW &
  \multicolumn{1}{c|}{2.0 mm} &
  6.0 mm \\ \midrule\midrule
Gilbert \etal &
  \multicolumn{1}{c|}{5.2} &
  \multicolumn{1}{c|}{2.5} &
  3.1 &
  \multicolumn{1}{c|}{12.2} &
  \multicolumn{1}{c|}{19.0} &
  22.7 &
    \multicolumn{1}{c|}{32.2} &
    70.0 \\ 
McCouat \etal &
    \multicolumn{1}{c|}{2.5} &
    \multicolumn{1}{c|}{1.6} &
    2.4 &
    \multicolumn{1}{c|}{7.5} &
    \multicolumn{1}{c|}{14.8} &
    19.9 &
    \multicolumn{1}{c|}{56.4} &
    91.7 \\ 
Chen \etal &
    \multicolumn{1}{c|}{2.3} &
    \multicolumn{1}{c|}{\textbf{1.5}} &
    2.3 &
    \multicolumn{1}{c|}{7.1} &
    \multicolumn{1}{c|}{\textbf{12.5}} &
    21.4 &
    \multicolumn{1}{c|}{57.3} &
    94.6 \\ 
Yao \etal &
    \multicolumn{1}{c|}{15.4} &
    \multicolumn{1}{c|}{8.8} &
    9.2 &
    \multicolumn{1}{c|}{44.8} &
    \multicolumn{1}{c|}{78.5} &
    80.5 &
    \multicolumn{1}{c|}{7.5} &
    24.6 \\ 
Duffy \etal &
    \multicolumn{1}{c|}{8.7} &
    \multicolumn{1}{c|}{3.4} &
    3.8 &
    \multicolumn{1}{c|}{24.8} &
    \multicolumn{1}{c|}{34.8} &
    34.1 &
    \multicolumn{1}{c|}{13.7} &
    42.4 \\ 
Ours &
\multicolumn{1}{c|}{\textbf{2.2}} &
\multicolumn{1}{c|}{\textbf{1.5}} &
\textbf{1.9} &
\multicolumn{1}{c|}{\textbf{6.2}} &
\multicolumn{1}{c|}{14.0} &
\textbf{16.9} &
    \multicolumn{1}{c|}{\textbf{58.9}} &
    \textbf{94.9}
 \\
\end{tabular}
\end{center}
\end{table}

% Figure environment removed

% Figure environment removed

% \begin{table}[h!]
%     \begin{subtable}[h]{0.45\textwidth}
%         \centering
%     \settowidth\rotheadsize{Example 1}
%     \begin{tabular}{@{\hspace{0.5mm}}c@{\hspace{0.5mm}}c@{\hspace{0.5mm}}c@{\hspace{0.5mm}}c}
%         % Figure removed & 
%         % Figure removed &
%         \\
%         {\footnotesize } 
%         {\footnotesize $112\times112$} 
%         & {\footnotesize $56\times56$} 
%         \\ \addlinespace[0.5mm]
%         % Figure removed & 
%         % Figure removed &
%         \\
%         {\footnotesize } 
%         {\footnotesize $28\times28$}
%         & {\footnotesize $1\times1$}
%     \end{tabular}
%     % \vspace{-0.3cm}
%     \caption*{Fig. 2: \textbf{Hierarchical predictions} - An example of the model's prediction for an input echo. We show the model's prediction for the case where only three auxiliary graphs are used. We see that the model is learning the LV landmarks on different resolutions to achieve high accuracy for the pixel-level task. We show zoomed-in versions of the higher resolution task to enable comparison.} 
%     % \vspace{-0.4cm}
%     \label{fig: qual_results_sucess_ood} 
%     \end{subtable}
%      \hfill
%     \begin{subtable}[h]{0.48\textwidth}
%         \centering
%             \settowidth\rotheadsize{Example 1}
%     \begin{tabular}{@{\hspace{0mm}}c@{\hspace{0.5mm}}c@{\hspace{0.5mm}}c@{\hspace{0.5mm}}c}
%         \rothead{\centering Ground Truth} &
%         % Figure removed & 
%         % Figure removed &            
%         \\ \addlinespace[0.5mm]
%         \rothead{\centering Predictions} &
%         % Figure removed & 
%         % Figure removed &
%         \\ 
%         {\footnotesize  } 
%         & {\footnotesize Failure 1} 
%         & {\footnotesize Failure 2}
%         \\
%     \end{tabular}
%        \caption*{Fig. 3:\textbf{ Qualitative visualization} of our model on two failure cases from the test set of our private dataset. 
%     The Failure 1 example is a low-quality PLAX image that also corresponds to a patient with severe LVH, a scenario that happens rarely in our dataset. 
%     The Failure 2 example belongs to a case with a low quality of PLAX with unclear boundaries for the walls and the chambers of the LV.}
%        \label{fig:proto_vis}
%     \end{subtable}
%      \label{tab:temps}
% \end{table}

% Figure environment removed
\null
\vfill

% \end{document}
