% !TeX spellcheck = en_GB

\documentclass[proc]{edpsmath}
\usepackage[utf8]{inputenc}
\usepackage[T1]{fontenc}

\usepackage{amsmath,amssymb,amsthm}
\usepackage{graphicx}
\usepackage{hyperref}
\usepackage{physics}
\usepackage{mathtools}
\usepackage{float}

\newcommand{\bR}{\mathbb{R}}
\newcommand{\bb}{\mathbf{b}}
\newcommand{\bA}{\mathbf{A}}
\newcommand{\be}{\mathbf{e}}
\newcommand{\vp}{v_\parallel}
\newcommand{\bx}{\mathbf{x}}
\newcommand{\bv}{\mathbf{v}}
\newcommand{\bu}{\mathbf{u}}
\newcommand{\bB}{\mathbf{B}}
\newcommand{\Dt}{\Delta t \;}

\newcommand{\dbx}{\dot{\mathbf{x}}}
\newcommand{\diff}[2]{\frac{\text{d} #1}{\text{d} #2}}
\newcommand{\pa}[2]{\frac{\partial #1}{\partial #2}}
\newcommand{\ap}{a_\parallel}
\newcommand{\mean}[1]{\langle #1 \rangle}
\newcommand{\Bap}{B_\parallel^\ast}
\newcommand{\bracket}[2]{\left\{ #1, #2 \right\}}
\newcommand{\gcbracket}[2]{\left\{ #1, #2 \right\}_{\text{g.c.}}}
\newcommand{\dt}{\frac{\text{d}}{\text{d} t}}

\renewcommand{\d}{\;\text{d}}

\setlength{\parindent}{0pt}

%TIKZ
\usepackage{tikz}
\usetikzlibrary{cd}
\tikzcdset{arrow style=tikz
	, diagrams={>={Stealth[length=1.2mm, width=1mm]}}
}
\usetikzlibrary{positioning}
\usetikzlibrary{arrows.meta}
\usetikzlibrary{decorations.pathreplacing}
\usetikzlibrary{shapes.symbols}
\usetikzlibrary{shapes.geometric}


\begin{document}

\title{Splitting Scheme for Gyro-kinetic Equations with Semi-Lagrangian and Arakawa Substeps} % Freddie approved

\author{Dominik Bell}\address{Max-Planck-Institut für Plasmaphysik, Garching, Germany; \email{dominik.bell@ipp.mpg.de\ \&\ martin.campos-pinto@ipp.mpg.de\ \&\ frederik.schnack@ipp.mpg.de\ \&\ sonnen@ipp.mpg.de}}\secondaddress{Technische Universität München, Zentrum Mathematik, Garching, Germany}
\author{Martin Campos Pinto}\sameaddress{1}
\author{Davor Kumozec}\address{Faculty of Sciences, University of Novi Sad, Serbia; \email{davor.kumozec@dmi.uns.ac.rs}}
\author{Frederik Schnack}\sameaddress{1}
\author{Emily Bourne}\address{CEA\unskip, IRFM\unskip, Saint-Paul-les-Durance\unskip, F-13108\unskip, France; \email{emily.bourne@epfl.ch}}
\author{Eric Sonnendrücker}\sameaddress{1,2}


\begin{abstract}
	The gyro-kinetic model is an approximation of the Vlasov-Maxwell system in a strongly magnetized magnetic field. We propose a new algorithm for solving it combining the Semi-Lagrangian (SL) method and the Arakawa (AKW) scheme with a time-integrator. Both methods are successfully used in practice for different kinds of applications, in our case, we combine them by first decomposing the problem into a fast (parallel) and a slow (perpendicular) dynamical system. The SL approach and the AKW scheme will be used to solve respectively the fast and the slow subsystems. Compared to the scheme in \cite{pygyro_code}, where the entire model is solved using only the SL method, our goal is to replace the method used in the slow subsystem by the AKW scheme, in order to improve the conservation of the physical constants.
\end{abstract}

\begin{resume}
	Le modèle gyro-cinétique est une approximation du système de Vlasov-Maxwell dans un champ magnétique fortement magnétisé. Nous proposons un nouvel algorithme pour le résoudre en combinant la méthode Semi-Lagrangienne (SL) et le schéma d'Arakawa (AKW) avec un intégrateur temporel. Les deux méthodes sont utilisées avec succès dans la pratique pour différents types d'applications. Dans notre cas, nous les combinons en décomposant d'abord le problème en un système dynamique rapide (parallèle) et un système dynamique lent (perpendiculaire). L'approche SL et le schéma AKW seront utilisés pour résoudre respectivement les sous-systèmes rapide et lent. Par rapport au schéma de \cite{pygyro_code}, où le modèle entier est résolu en utilisant uniquement la méthode SL, notre objectif est de remplacer la méthode utilisée dans le sous-système lent par le schéma AKW, afin d'améliorer la conservation des constantes physiques.
\end{resume}

\maketitle
\newpage
\tableofcontents

\section{Introduction}
Current quantum hardware is unable to carry out universal quantum computations due to the buildup of errors that occur during the computation. 
The magnitude of the individual error is currently above the value that the Threshold Theorem requires in order to kick-start quantum error correction and fault-tolerant quantum computation~\cite[Section 10.6]{nielsen_chuang_2010}. 
Although the experimentally achieved fidelity rates are promising and the error bounds are inching closer to the required threshold, we will have to work for the foreseeable future with quantum hardware with errors that build-up during the computation.  This implies that we can only do a limited number of steps before the output of the computation has become completely uncorrelated with the intended one.

For fault-tolerant quantum computing, we repeat four steps: 
1) We apply a number of single and two-qubit quantum gates, in parallel whenever possible; 
2) We perform a syndrome measurement on a subset of the qubits; 
3) We perform fast classical computations to determine which errors have occurred and how to correct them; 
and, 4) We apply correction terms based on the classical computations.
We then repeat these four steps with a next sequence of gates. 
These four steps are essential to fault-tolerant quantum computing. 


The starting point of this work is to use the four steps outlined above, not to carry out error correction and fault-tolerant computation, but to enhance short, constant-depth, {\em uncorrected} quantum circuits that perform single qubit gates and {\em nearest-neighbor} two qubit gates. 
Since in the long run we will have to implement error-correction and fault-tolerant computation anyhow, and this is done by such a four-step process, why not make other use of this architecture? Moreover, on some of the quantum hardware platforms, these operations are already in place.
Embracing this idea we naturally arrive at the question: what is the computational power of \textit{low-depth} quantum-classical circuits organized as in the four steps outlined above? 
We thus investigate circuits that execute a small, ideally constant, number of stages, where at each stage we may apply, in parallel, single qubit gates and {\em nearest-neighbor} two qubit gates, followed by measurements, followed by low-depth classical computations of which the outcome can control quantum gates in later stages. 
It is not clear, at first, whether such circuits, especially with constant depth, can do anything remotely useful. 
But we will see that this is indeed the case: many quantum computations can be done by such circuits in constant depth. 
By parallelizing quantum computations in this way, we improve the overall computational capabilities of these circuits, as we do not incur errors on qubits that are idle, simply because qubits are not idle for a very long time. 
Furthermore, reducing the depth of quantum circuits, at the cost of increasing width, allows the circuit to be run faster even if errors occur.

The first usage of such a four-step layout, not to do error correction, but to perform computations, can be found in the paradigm of measurement-based quantum computing~\cite{gottesman1999demonstrating,raussendorf2001one,jozsa2006introduction,clark2007generalised}: 
A universal form of quantum computing where a quantum state is prepared and operations are performed by measuring qubits in different bases, depending on previous measurements and intermediate measurements.

\citeauthor{PhamSvore2013} were the first to formalize the four-step protocol for performing computations~\cite{PhamSvore2013}. They included specific hardware topologies by considering two-dimensional graphs for imposing constraints on qubit interactions. In their model, they develop circuits for particularly useful multi-qubit gates, including specifying costs in the width, number of qubits, depth, number of concurrent time steps, size, and total number of non-Identity operations.
As a result, they find an algorithm that factors integers in polylogarithmic depth.
\citeauthor{Browne:2011} showed that the main tool in the work by \citeauthor{PhamSvore2013}, the fan-out gate, can also be replaced by additional log-depth classical computations in the measurement-based quantum computing setting~\cite{Browne:2011}.

More recently, \citeauthor{Cirac:2021} introduced a scheme to implement unitary operations involving quantum circuits combined with Local Operations and Classical Communication ($\mathsf{LOCC}$) channels: $\mathsf{LOCC}$-assisted quantum circuits~\cite{Cirac:2021}. Similarly to the four-step scheme we just described, they allow for a short depth circuit to be run on the qubits, followed by one round of $\mathsf{LOCC}$, in which ancilla qubits are measured and local unitaries are applied based on the measurement outcomes. They show that in this model any 1D transitionally invariant matrix-product state (MPS) with fixed bond dimension is in the same phase of matter as the trivial state. Similar ideas can be found in~\cite{TVV_NonAbelianTopologicalOrder_2022, tantivasadakarn2021long}.

In this work, we introduce a new model, called \textit{Local Alternating Quantum-Classical Computations} ($\LAQCC$). In this model we alternate between running quantum circuits (constrained by locality), ending in the measurement of a subset of qubits, and fast classical computations based on the measurement results. The outcome of the classical computations are then used to control future quantum circuits. We allow for flexibility in this model, by giving different constraints to the power of both the quantum circuits and the classical circuits as well as the number of alternations between them. 
Most attention will be given to $\LAQCC$ containing quantum circuits of constant depth, classical circuits of logarithmic depth and at most a constant number of alternations between them. 
Any circuit constructed in this model is considered to be of constant depth. 
We restrict ourselves to logarithmic depth classical computations, as this is the first natural and non-trivial extension beyond constant-depth classical computations. 
Constant-depth classical computations do however also have an equivalent constant-depth quantum implementation.

The definition of $\LAQCC$ sharpens the original definition of \citeauthor{PhamSvore2013} by adding constraints to the intermediate classical computations. This allows us to bound the power of $\LAQCC$ from above. 

The main result of \citeauthor{Cirac:2021}, that 1D translational invariant MPS with fixed bond dimension can be prepared by $\mathsf{LOCC}$-assisted circuits, relies on local symmetries of the MPS. These symmetries allow them to prepare local states (on a constant number of qubits) and glue them together by doing one round of the appropriate entangling measurement and corrections, after which they run a round of local unitaries to get the desired result. This general scheme for preparing states that exhibit an MPS description with the appropriate local symmetries requires only geometrically local unitaries and one round of measurement and corrections an therefore is accessible in $\LAQCC$. Studying different local symmetries, known as Symmetry Protected Topological (SPT) phases of matter, to find measurement-based constant depth circuits for states is a broad ongoing field of research~\cite{TVV_NonAbelianTopologicalOrder_2022, tantivasadakarn2021long, smith2023deterministic}. 
All these schemes have a $\LAQCC$ implementation.

%$\LAQCC$-circuits also exist for general schemes of preparing local states, based on the local tensors, and gluing them together using one round of entangled measurement and corrections, based on the local symmetry. 
%The main result of \citeauthor{Cirac:2021}, that 1D translational invariant MPS with fixed bond dimension can be prepared by $\mathsf{LOCC}$-assisted circuits, relies heavily on local symmetries of the MPS and as a result also has an equivalent $\LAQCC$ implementation. 
%The corrections applied after the measurement round are local unitaries depending on the local symmetries of the MPS. 

 

%This general scheme of preparing local states, based on the local tensors, and gluing it together by doing one round of entangled measurement and corrections, based on the local symmetry, is accessible in $\LAQCC$.
Note however that \citeauthor{Cirac:2021} also suggest a circuit for the $W$-state.
This circuit uses sequentially and dependent measurement-based corrections of the ancilla qubits. 
These dependent measurements translate to sequential alternations between the quantum and classical circuits and therefore increase the total depth to linear depth, exceeding the constant-depth constraints imposed by $\LAQCC$-circuits. 

We study the power of the $\LAQCC$ model with respect to state preparation, showing that even with only constant quantum-depth and logarithmic classical depth it remains possible to prepare states with long-range entanglement.
Another surprising result is that it is unlikely that $\LAQCC$ circuits are classically simulatable. We show that any instantaneous quantum polynomial-time (IQP) circuit~\cite{Bremner2010,Shepherd2009} has an $\LAQCC$ implementation.
Classical simulation of IQP circuits implies the collapse of the polynomial hierarchy to the third level, which is not believed to be true~\cite{Bremner2017}. Therefore, we expect that $\LAQCC$ circuits are unlikely to be classically simulatable. We bound the power of $\LAQCC$ by showing that it is contained in $\QNC^1$, the class of polynomial-size, log-depth circuits.

Next, we also study the power that intermediate classical calculations can add to quantum computations, by considering a new model that alternates between polynomially many polynomial-depth quantum circuits and unbounded classical computations
We study this model by doing a complexity theoretical analysis, where we draw inspiration from the notions of complexity given by \citeauthor{RosenthalYuen:2022}, \citeauthor{MetgerYuen:2023}, and \citeauthor{Aaronson:2004}.
All three complexity notions are based on the notion of state preparation, instead of more traditional definition of complexity such as the decidability of a computational problem. 
The first two consider classes based on sequences of quantum states preparable by a polynomial-sized quantum circuit, where the circuits are uniformly generated by a computational class, for instance, the class $\mathsf{PSPACE}$, which results in the complexity class $\mathsf{StatePSPACE}$~\cite{RosenthalYuen:2022,MetgerYuen:2023}.
The third notion considers a relative complexity, where the complexity is measured between two given states, and is measured by the number of gates, from a given gate-set, required to transform one state in another state~\cite{Aaronson:2004}. 
For our definition of state preparation complexity, we drop the uniformity constraint from~\cite{RosenthalYuen:2022,MetgerYuen:2023} and define a class as $\mathsf{StateX}$, which refers to states preparable by circuits of type $\mathsf{X}$. 
As an example, if $\mathsf{X} = \QNC^0$, this results in the class $\mathsf{StateQNC^0}$, which is the set of states preparable from the $\ket{0}^n$ state by poly-size constant-depth circuits. 
This notion is similar to the relative complexity from~\cite{Aaronson:2004}, where one state is the  $\ket{0}^n$ state and instead of counting the number of gates we consider the set of states preparable by a fixed number of gates. Using this notion of complexity we show that any state preparable by an $\LAQCC^*$ circuit is also preparable by a $\mathsf{PostQPoly}$ circuit, the class of circuits of polynomial depth with an additional post-selection gate. 

All Clifford circuits have a constant-depth $\LAQCC$ implementation, implying that any stabilizer state can be implemented by a constant-depth $\LAQCC$ circuit, see Section~\ref{sec:clifford_circuits} for a proof of this statement. 
Efficient circuits for stabilizer states have been known already through measurement-based quantum computing. Therefore this paper focuses on the preparation of non-stabilizer states, and as a surprising result we find novel constant-depth protocols for four very natural classes of non-stabilizer states.
Despite the extensive research into these four classes of non-stabilizer states and the many applications of them, no efficient constant- or low-depth state preparation protocols are known yet. We specifically consider these four classes as they are all often used as initial states in other algorithms.

The first state is a uniform superposition over an arbitrary number of states. 
This state finds applications in many quantum algorithms, as they often start with a uniform superposition over multiple states. 
This superposition is often achieved by applying Hadamard gates to every qubit due to its simplicity to prepare. 
Yet, the analysis of many algorithms, such as Shor's algorithm~\cite{Shor:1997}, would benefit from a different initial superposition. 
The circuit to prepare the uniform superposition over an arbitrary number of states uses an exact version of Grover search as a subroutine, that turns a probabilistic circuit, with a known constant probability of success, into a deterministic circuit. 
We use the circuit for preparing a uniform superposition over an arbitrary number of states as a subroutine in the next two quantum state preparation protocols. 

The second state is the $W$-state, the uniform superposition over all computational basis states of Hamming-weight~$1$, a natural long-ranged entangled state that displays a fundamentally nonequivalent type of entanglement from the Greenberger–Horne–Zeilinger state~\cite{WState:2000}, for which $\LAQCC$-type constant-depth circuits were previously known~\cite{PhamSvore2013, Cirac:2021}. 
The $W$-state is often used as benchmark for new quantum hardware~\cite{Haffner2005,Neeley2010,GarciaPerez:2021}. 
A novel way to prepare the $W$-state therefore gives a new way to benchmark different quantum devices with each other. 
A circuit for preparing the $W$-state was given in~\cite{Cirac:2021}, but this implementation requires sequentially alternating measurements followed by local unitaries, which in the $\LAQCC$ model is not considered to be of constant depth. 
We improve this protocol by giving an $\LAQCC$ implementation of the $W$-state, based on a compress-uncompress method that links the one-hot and binary encoding of integers.

The third state considered is the Dicke state, a generalization of the $W$-state, a superposition over all computational basis states with Hamming-weight $k$~\cite{Dicke:1954}. 
Dicke states have relevance in various practical settings.
For instance, for quantum game theory~\cite{zdemir2007}, quantum storage~\cite{Bacon_Compress:2006,Plesch:2010}, quantum error correction~\cite{ouyang2014permutation}, quantum metrology~\cite{toth2012multipartite}, and quantum networking~\cite{prevedel2009experimental}. 
Dicke states have been used as a starting state for variational optimization algorithms, most notably Quantum Alternating Operator Ansatz (QAOA)~\cite{Hadfield2019}, to find solutions to problems such as Maximum k-vertex Cover~\cite{Brandhofer2022,cook2020quantum}.
The ground states of physical Hamiltonians describing one-dimensional chains tend to show a resemblance to Dicke states such as states resulting from the Bethe ansatz, making them an ideal starting state when investigating the ground state behavior of these Hamiltonians~\cite{TDL_BetheAnsatzDerivation:2010,B_ExcitedStateQuantumPhaseTransitions:2013,DickeTransitions:2021}. 
For instance, the algorithm by \citeauthor{van2021preparing}, who give an algorithm to prepare the Bethe ansatz eigenstates of the spin-1/2 XXZ spin chain, starts by first preparing a Dicke state~\cite{van2021preparing}. 
A Dicke-state preparation protocol based on the compress-uncompress methodology used in the $W$-state furthermore finds applications in entanglement distillation, where the entanglement of a large state is concentrated on only a few qubits. 
Efficient deterministic circuits for preparing Dicke states have been proposed by \citeauthor{bartschi2019deterministic}~\cite{bartschi2019deterministic, bartschi2022deterministic_short_depth}. 
They provide a quantum circuit of depth $\mathO(k \log(\frac{n}{k}))$, allowing arbitrary connectivity, to prepare a Dicke state, which they conjecture to be optimal when $k$ is constant. 
In this work, we provide a constant-depth $\LAQCC$ circuit below their conjectured bound already for constant $k$. 
However, this does not directly disprove their conjecture, as we allow for intermediate measurements and classical computations. 
More significantly, we even construct constant-depth $\LAQCC$ circuits for $k = \mathO(\sqrt{n})$ greatly improving their bound.
This construction extends the compress-uncompress method for the $W$-state combined with additional subroutines. 

We continue with a log-depth state preparation protocol for the Dicke-state for arbitrary $k$. 
This protocol implements an efficient transformation between the factoradic number representation and the combinatorial number representation of a positive integer. 
The combinatorial number representation relates directly to the Dicke state. 
The provided efficient transformation between number representation systems might be of independent interest. 

We conclude by modifying our protocol for preparing a Dicke-state to a protocol that prepares quantum many-body scar states in constant-depth. 
These states have low entanglement and longer coherence times than states with similar energy density.
These characteristics make many-body scar states interesting to analyze and relevant within physics.
Many-body scar states appear for instance in the AKLT model~\cite{AKLT:1987,MRBAR:2018,MRB:2018} and different spin models~\cite{SI:2019,MOBFR:2020}.
Known methods for preparing these states have polynomial-depth~\cite{Gustafson:2023}, whereas our circuit has constant depth. 

% We conclude by studying the power that intermediate classical calculations can add to quantum computations. 
% In this study, we define a new model that relaxes constant-depth quantum circuits to polynomial depth quantum circuits, log-depth classical calculations to unbounded classical computations and a constant number of alternations to a polynomial number of alternations. 
% We call this model $\LAQCC^*$. 
% We study this model by doing a complexity theoretical analysis, where we draw inspiration from the notions of complexity given by \citeauthor{RosenthalYuen:2022}, \citeauthor{MetgerYuen:2023}, and \citeauthor{Aaronson:2004}.
% All three complexity notions are based on the notion of state preparation, instead of more traditional definition of complexity such as the decidability of a computational problem. 
% The first two consider classes based on sequences of quantum states preparable by a polynomial-sized quantum circuit, where the circuits are uniformly generated by a computational class, for instance, the class $\mathsf{PSPACE}$, which results in the complexity class $\mathsf{StatePSPACE}$~\cite{RosenthalYuen:2022,MetgerYuen:2023}.
% The third notion considers a relative complexity, where the complexity is measured between two given states, and is measured by the number of gates, from a given gate-set, required to transform one state in another state~\cite{Aaronson:2004}. 
% For our definition of state preparation complexity, we drop the uniformity constraint from~\cite{RosenthalYuen:2022,MetgerYuen:2023} and define a class as $\mathsf{StateX}$, which refers to states preparable by circuits of type $\mathsf{X}$. 
% As an example, if $\mathsf{X} = \QNC^0$, this results in the class $\mathsf{StateQNC^0}$, which is the set of states preparable from the $\ket{0}^n$ state by poly-size constant-depth circuits. 
% This notion is similar to the relative complexity from~\cite{Aaronson:2004}, where one state is the  $\ket{0}^n$ state and instead of counting the number of gates we consider the set of states preparable by a fixed number of gates. Using this notion of complexity we show that any state preparable by an $\LAQCC^*$ circuit is also preparable by a $\mathsf{PostQPoly}$ circuit, the class of circuits of polynomial depth with an additional post-selection gate. 

\paragraph{Summary of results}
\begin{itemize}
    \item We give a new definition of a computational model that captures the power of the four step process: applying a constant number of layers of one- and two-qubit gates; performing a syndrome measurement; perform a fast classical computation determining corrections; apply corrections. We call this model \emph{Local Alternating Quantum Classical Computations}, or $\LAQCC$ for short. In this model we bound the allowed quantum operations, intermediate classical calculations, and number of rounds separately. In Section~\ref{sec:LAQCC_model} we define this model and give a list of operations based on results from literature contained in this computational model. In some of these operations we explicitly use that we allow for multiple, but at most constant, rounds  of corrections.
    \item  We show show that there exist $\LAQCC$ circuits that can not be weakly simulated in Section~\ref{sec:IQP_in_LAQCC}. We further show that for every $\LAQCC$ circuit there exists a $\QNC^1$ circuit simulating it perfectly, in Section~\ref{sec:LAQCC_in_QNC1}.
    \item We introduce a new type computational complexity for preparing states and show that the extension of $\LAQCC$ where we allow a polynomial number of rounds and unbounded classical computation, is contained in $\mathsf{PostQPoly}$, the class of polynomial circuits with post-selection, in Section~\ref{sec:Complexity results}.
    \item We show a protocol to prepare the uniform superposition state of size $q$ in $\LAQCC$ using $\mathO(\ceil{\log_2(q)}^2)$ qubits in Section~\ref{sec:superposition_modulo_q}. 
    \item We show a protocol to prepare the $W_n$ state in $\LAQCC$ using $\mathO(n\log(n))$ qubits in Section~\ref{sec:W_state_in_LAQCC}.
    \item We show two ways of preparing the Dicke-$(n,k)$ state. The first method is in $\LAQCC$, works up to $k = \mathO(\sqrt{n})$, uses $\mathO(n^2\log(n))$ qubits, and is found in Section~\ref{sec:dicke:small_k}. The second method is in $\LAQCC\text{-}\mathsf{LOG}$ (an extension of $\LAQCC$ allowing for logarithmic number of alterations instead of constant), works for any $k$, uses $\mathO(\text{poly}(n))$ qubits, and is found in Section~\ref{sec:Dicke_in_LAQCC_LOG}. 
    \item We extend on our $\LAQCC$ method of generating Dicke-$(n,k)$ states for $k = \mathO(\sqrt{n})$ and show a protocol to generate many-body scar states for a particular Hamiltonian in $\LAQCC$ (Section~\ref{sec:many_body_scar}). 
\end{itemize}
Summarized in a table, we provide the following state generation protocols:
\begin{table}[htb]
\centering
\begin{tabular}{l|l|l|l}
\textbf{State description} & \textbf{Width} & \textbf{Depth} & \textbf{Implementation}\\
\hline 
Uniform superposition mod $q$: $\frac{1}{\sqrt{q}} \sum_{i = 0}^{q-1}\ket{i}$ & $\mathO(\ceil{\log^2 q})$ & $\mathO(1)$ & Section~\ref{sec:superposition_modulo_q}\\

$W$-state: $\frac{1}{\sqrt{n}}\sum_{i = 0}^{n-1}\ket{e_i}$ & $\mathO(n \log n)$ & $\mathO(1)$ & Section~\ref{sec:W_state_in_LAQCC}\\

Dicke-$(n,k)$, $k = \mathO(\sqrt{n})$: $\binom{n}{k}^{-1/2}\sum_{x \in \{0,1\}^n: |x| = k} \ket{x}$ &  $\mathO(n^2\log n)$ & $\mathO(1)$ 
&Section~\ref{sec:dicke:small_k}\\

Dicke-$(n,k)$: $\binom{n}{k}^{-1/2}\sum_{x \in \{0,1\}^n: |x| = k} \ket{x}$ & $\mathO(\text{poly}(n))$ & $\mathO(\log n)$ &Section~\ref{sec:Dicke_in_LAQCC_LOG}\\

QMBS: $\ket{S_k} = \frac{1}{k! \sqrt{\mathcal N(n,k)}}(Q^\dagger)^k \ket{\Omega}$ &  $\mathO(n^2\log n)$ & $\mathO(1)$  &  Section~\ref{sec:many_body_scar}
\end{tabular}
\caption{Summary of state preparation protocols given in this paper.}
\label{tab:sate_prep}
\end{table}
In the entry for the quantum many-body scar state $Q$ denotes the raising operator and $\mathcal N(n,k)=\binom{n-k-1}{k}$. 
Section~\ref{sec:many_body_scar} will provide more details on the variables and the implementation. 

\paragraph{Organization of the paper}
\noindent We first introduce relevant preliminaries in Section~\ref{sec:preliminaries}. 
In Section~\ref{sec:LAQCC_model} we formally define the class of Local Alternating Quantum-Classical Computations ($\LAQCC$). We also show that any Clifford circuit can be implemented in constant depth $\LAQCC$ (a result based on a result from measurement-based quantum computing~\cite{jozsa2006introduction}). 
This result allows us to give many useful multi-qubit gates and routines in Section~\ref{sec:gates_created_in_LAQCC}. 
Beyond that we show that constant depth $\LAQCC$ circuits are contained in $\QNC^1$ and that any $\mathsf{IQP}$ circuit has an $\LAQCC$ implementation.
We conclude this section with an analysis of a more powerful instantiation of $\LAQCC$ and show an inclusion with respect to the class $\mathsf{PostQPoly}$, which is the class of circuits of polynomial depth with one additional post-selection gate. 
In Section~\ref{sec:state_prep_in_LAQCC} we give $\LAQCC$ circuit implementations for preparing the uniform superposition over an arbitrary number of states, the $W$-state and the Dicke state up to $k = \mathO(\sqrt{n})$. We furthermore give a log-depth circuit implementation for preparing the Dicke state for any $k$. We conclude by showing a $\LAQCC$ circuit for generating many body scar states of a particular type of Hamiltonian.



% !TeX spellcheck = en_GB


\section{The Gyro-kinetic Model}
\label{sec:gk-model}

In a Tokamak, the dynamics of particles consists of a slow motion along the magnetic field lines superimposed with a fast gyration around the magnetic field lines. This fast motion can be averaged out to reduce the dimension of phase space in order to make the models computationally more tractable, while keeping most of the important physics. The resulting models are called gyro-kinetic and will be described in the following.\\

The model we will look at is defined by the Lie-transformed, low-frequency particle Lagrangian $L$, where we follow the derivation in \cite{Bottino_Sonnendrucker_2015}, \cite{emily} and \cite{Latu_2017}. Given a static magnetic field $\bB$, its intensity $B = \norm{\bB}$ and its direction $\bb = \bB / B$, the particle charge $q \not= 0$ and the particle mass $m >0$,  
%and denoting $\bA$ as the vector-valued background potential
we are able to write the Lagrangian $L$ as
\begin{equation}\label{Lagrangian}
	L(t, \bx, \vp, \mu, \dot{\bx}, \dot{v}_\parallel, \dot{\mu}) = \left(\nabla \times q\bB + m \vp \bb \right)\cdot\dbx + \frac{m}{q} \mu \dot{\theta} - H(\bx, \vp),
\end{equation}
where $\bx \in \Omega \subseteq \bR^3$ is the position of the gyro-centre, $\vp \in \bR$ is the velocity parallel to the magnetic field lines, $\mu$ is the modified magnetic moment and $\theta$ the angle of cylindrical coordinates. The Hamiltonian $H$ will be introduced shortly.
Looking at the equation of motion of $\theta$, i.e. its Euler-Lagrange equation
\begin{equation}
	0 = \frac{\dd}{\dd t} \left( \frac{\partial L}{\partial \dot \theta} \right) - \left(
		\frac{\partial L}{\partial \theta}\right) =  \frac{\dd}{\dd t} \frac{m}{q} \mu,
\end{equation}
we can immediately conclude that $\mu$ is an exact invariant of the system, i.e.
\begin{equation}
	\diff{}{t} \mu = 0,
\end{equation}
and thus the phase-space is only $4$-dimensional.
The gyro-kinetic equation describing the gyro-centre distribution function $f=f(t,\bx, \vp)$ is of the form
\begin{equation}\label{drift-kinetic model}
	\pa{f}{t} + \bu \cdot \nabla f + \ap \pa{f}{\vp} = 0,
\end{equation}
which describes the positions of a collection of identical particles and whose exact solution is constant along the trajectories $(\bx(t), \vp(t))$ in the phase-space, i.e. 
\begin{equation}
	\frac{\dd}{\dd t} f(t,\bx(t), \vp(t)) = 0.
\end{equation}
In order to determine the equations of motion for ${\dd \bx}/{\dd t} = \bu$ and ${\dd \vp}/{\dd t} = \ap$, we look at the remaining Euler-Lagrange equations. Simplifying notations by defining
\begin{equation}
	\bB^\ast  \coloneqq \bB + \frac{m}{q}\vp \nabla \times \bb , \qquad
	\Bap  \coloneqq \bb \cdot \bB^\ast = B + \frac{m \vp}{q B} \bb \cdot \left( \nabla \times \bB \right),
\end{equation}
one can derive the characteristic trajectories (see \cite{Bottino_Sonnendrucker_2015} for a detailed derivation) from the remaining Euler-Lagrange equations of \eqref{Lagrangian}, which yield
\begin{subequations}
	\begin{align}
		\bu  & = \frac{1}{\Bap} \left( \frac{1}{m} \pa{H}{\vp} \bB^\ast + \frac{1}{q} \bb \times \nabla H \right) \label{eom for bu}, \\
		\ap & = \frac{1}{\Bap} \left( -\frac{1}{m} \bB^\ast \cdot  \nabla H  \right). \label{eom for ap}
	\end{align}
\end{subequations}
As noted in \cite{Latu_2017}, the phase space is divergence-free, i.e.
\begin{equation}
	\nabla \cdot \bu + \pa{\ap}{\vp} = 0,
\end{equation}
thus we can rewrite \eqref{drift-kinetic model} in conservative form
\begin{equation}\label{conservation1}
	\pa{}{t}\left(\Bap f\right) + \nabla\cdot\left( \Bap \bu f \right) + \pa{}{\vp} \left( \Bap \ap f \right) = 0.
\end{equation}

The equations of motion form a Hamiltonian system (see \cite{idomura2008conservative} for details) with the electrostatic gyro-centre Hamiltonian in the zero-Larmor-radius limit
\begin{equation}
	H(t, \bx, \vp) = \frac{1}{2} m \vp^2 + \mu B(\bx) + q \phi (t,\bx).
\end{equation}
Furthermore, we need a bracket for the Hamiltonian system which is the guiding-centre Poisson bracket
\begin{subequations}
	\begin{align}
		\gcbracket{F}{G} & = \frac{\bB}{m \Bap} \cdot \left( \left(\nabla F\right) \pa{G}{\vp} - \left(\nabla G\right) \pa{F}{\vp} \right) \label{fast-poisson} \\
		& \qquad + \frac{\vp}{q \Bap} \left( \nabla \times \bb \right) \cdot \left( \left(\nabla F\right) \pa{G}{\vp} - \left(\nabla G\right) \pa{G}{\vp} \right) \\
		& \qquad - \frac{1}{q \Bap} \bb \cdot \left[\left(\nabla F\right) \times \left(\nabla G\right)\right].
	\end{align}
\end{subequations}
We now split the Poisson bracket in parts containing $\bB$ (i.e. \eqref{fast-poisson}) which we expect to have fast dynamics, and other terms which will be called the slow subsystem.
%The part of the bracket responsible for the fast subsystem reads
%\begin{equation}
%	\bracket{F}{G}_{\text{fast}} = \frac{\bB}{m \Bap} \cdot \left( \left(\nabla F\right) \pa{G}{\vp} - \left(\nabla G\right) \pa{F}{\vp} \right)
%\end{equation}
%while for the slow subsystem it reads
%\begin{equation}
%	\begin{aligned}
%		\bracket{F}{G}_{\text{slow}} & = \frac{\vp}{q \Bap} \left( \nabla \times \bb \right) \cdot \left( \left(\nabla F\right) \pa{G}{\vp} - \left(\nabla G\right) \pa{G}{\vp} \right) \label{gc_lg} \\
%		& \qquad - \frac{1}{q \Bap} \bb \cdot \left[\left(\nabla F\right) \times \left(\nabla G\right)\right].
%	\end{aligned}
%\end{equation}
Thus the equations of motion for the fast and slow subsystems, with variables $\bu = \bu_f + \bu_s$ and $a_{\parallel} = a_{\parallel, f} + a_{\parallel, s}$, reads:
\begin{subequations}
	\begin{align}
		(\text{fast}) = & \left\{ \begin{aligned}
			\bu_f & = \frac{1}{\Bap} \frac{1}{m} \pa{H}{\vp} \bB, \\
			a_{\parallel, f} & = -\frac{1}{\Bap} \frac{1}{m} \bB \cdot \left( \nabla H \right),
		\end{aligned} \right. \label{split step fast}\\
		(\text{slow}) = &  \left\{ \begin{aligned}
			\bu_s & = \frac{1}{\Bap} \frac{1}{q} \left( \pa{H}{\vp} \vp \nabla \times \bb + \bb \times \left(\nabla H\right) \right), \\
			a_{\parallel, s} & = - \frac{a}{\Bap} \frac{1}{q} \vp \left( \nabla \times \bb \right) \cdot \left( \nabla H \right).
		\end{aligned} \right. \label{split step slow}
	\end{align}
\end{subequations}
This splitting is of particular interest when introducing a phase-space discretization, where the fast system trajectories may travel across many cells, which introduces a higher CFL number and thus needs a time-integration that handles this well, while the treatment of the slow subsystem naturally is less restrictive.

For a constant and uniform background magnetic field $\bB = B \hat \be_z$, where $\hat \be_z$ is the unit-vector pointing in the $z$-variable direction, the model simplifies to drift-kinetic equations and the subsystems become
\begin{subequations}
	\begin{align}
		(\text{fast}) = & \left\{ \begin{aligned}
			\bu_f & = {\vp} \bb, \\
			a_{\parallel, f} & = -\frac{q}{m} \bb \cdot \nabla \phi,
		\end{aligned} \right. \label{split step fast const B} \\
		(\text{slow}) = &  \left\{ \begin{aligned}
			\bu_s & = \frac{\bb \times \nabla \phi}{B}, \\
			a_{\parallel, s} & = 0.
		\end{aligned} \right. \label{split step slow const B}
	\end{align}
\end{subequations}
Continuing this simplification, we are able to rewrite the space variables to cylindrical coordinates as are introduced in \cite{Latu_2017}. Thus, we look for the distribution function $f = f(t, r, \theta, z, v_\parallel)$ satisfying
%TODO add sources?
\begin{equation}
	\partial_t f + \{\phi, f\} + v_\parallel \nabla_\parallel f - \nabla_\parallel \phi \partial_{v_\parallel}f = 0, \label{eq:GK_model}
\end{equation}
where $\nabla_\parallel = \bb \cdot \nabla$ and the bracket is transformed to polar coordinates, which reads, given the toroidal magnetic field $B_0$,
\begin{equation}
	\{\phi, f \} = \frac{1}{rB_0}\partial_r \phi\partial_\theta f -\frac{1}{rB_0}\partial_\theta \phi\partial_r f. \label{eq:bracket_in_polar}
\end{equation}
Since the plasma is quasi-neutral this equation is complemented by solving a elliptic quasi-neutrality (QN) equation for the self-consistent potential $\phi = \phi(t, r, \phi, z)$, i.e. solving for a given temperature profile $T_e$
\begin{equation}
	- \left[\partial_r^2 \phi + \left( \frac{1}{r} + \frac{\partial_r n_0}{n_0}\right)\partial_r \phi + \frac{1}{r^2} \partial_\theta^2 \phi \right] + \frac{1}{T_e} \phi = \int_{-\infty}^{\infty} (f - f_\text{eq}) \ \dd v_\parallel, \label{eq:qn}
\end{equation}
where the given (radial symmetric) equilibrium function $f_\text{eq}$ is a Gaussian and $n_0$ is a radial profile, which is the integral over $\vp$ of the equilibrium function. The initial distribution function $f(t=0, r, \theta, z, v_\parallel)$ is defined to be the equilibrium with a perturbation in some modes, which will result in an observable turbulence behaviour of the system after a certain amount of time. For more details and the exact definitions and constants, we refer to Section 4 in \cite{Latu_2017}. Normally these equations are defined on an infinitely long cylinder of some radius, but in order to discretize them, we cut the domain down to a finite cylinder with a hole in the middle and reduce the velocity space, such that   $(r, \theta, z, \vp) \in [r_\text{min}, r_\text{max}]\times[0, 2\pi]\times[0, 2\pi R_0] \times [-v_\text{max}, v_\text{max}]$, with parameters that will be made more precise in Section \ref{sec:num_exp}. To complete this system of equations, we briefly want to discuss the boundary conditions of this reduction. The distribution function $f$ is periodic along $\theta$, $z$, while having homogeneous Dirichlet boundaries in the $\vp$-direction. In the radial direction $r$, we assume the values are given by an outside equilibrium function. For the potential $\phi$, we assume periodic boundary conditions along $\theta$ and $z$. In $r$-direction, we decompose $\phi$ into Fourier modes at $r_\text{min}$, taking homogeneous Neumann boundary conditions for the zeroth mode and homogeneous Dirichlet boundary conditions for the others. At $r_\text{max}$, we just take plain homogeneous Dirichlet boundary conditions. \\


%Does this apply? Can you explain it to me?
%\subsubsection{Conserved Quantities}
Lastly, we want to take a look at physical properties of these equations. Since we have a transport equation in conservative form \eqref{conservation1} that conserves arbitrary functions of $f$  along non-linear characteristic trajectories, we can write the Casimir equation
\begin{equation}
	\dt C(f) + \bracket{C(f)}{H} =0,
\end{equation}
where the Casimir invariant $\int C( f ) $ is conserved. Therefore, the system has an infinite number of conserved quantities such as the particle number, the $L^1$-norm $\int f $, or the $L^2$-norm $\int f^2 $. In addition, the gyro-kinetic equations conserve the total energy $H = E_k + E_p$:
\begin{subequations}
	\begin{align}
		E_k &= \int f(\bx, \vp) \left[\frac{1}{2} m \vp^2 + \mu B(\bx)\right] \d \bx \d \vp \, , \\
		E_p &= \int q \mean{\phi}_\alpha (t,\bx) f(\bx, \vp) \d \bx \d \vp \, ,
	\end{align}
\end{subequations}
where $E_k$ is the kinetic energy and $E_f$ is the potential energy. For more details we refer the reader to \cite{idomura2008conservative}.

% !TeX spellcheck = en_GB

\section{Discretization via Operator Splitting}
\label{sec:splitting_discretization}

Operator splitting has proven to be an effective method for the time-integration of ordinary differential equations (ODEs), whose vector-field can be written as a sum of simpler terms. This is of special interest for geometric integration, that is, if the underlying problem has geometric properties that should be conserved by the integrator and can be enforced more easily for each split step. Usually, the combined computational cost for the integration of the split steps is lower compared to integrating the full ODE, albeit at the expense of introducing an additional approximation error. A broad overview of these methods is given in \cite{mclachlan_quispel_2002}, also containing the second-order Strang and first-order Lie splitting, that we will use in the following.\\

In the SL scheme from \cite{pygyro_code}, the (Hamiltonian) operator splitting is applied on the full screw-pinch model equation, that has a separable Hamiltonian
\begin{equation}
	H = \frac{1}{2} m \vp^2 + q \phi
\end{equation}
and describes the time-evolution of the distribution function $f$, i.e.
\begin{equation}
 \partial_t f + \{\phi, f \} + v_\parallel \nabla_\parallel f - \nabla_\parallel \phi\,\, \partial_{v_\parallel} f = 0,
\end{equation}
We can split this by first applying the above discussed splitting of the Poisson bracket into fast and slow parts as described above, and then splitting the Hamiltonian into its two parts in the fast subsystem, yielding
\begin{subequations}
	\begin{align}
		\partial_t f + v_\parallel \nabla_\parallel f & = 0, && \text{(Advection on flux surface),} & \label{eq:adv_flux} \\
		\partial_t f + \nabla_\parallel \phi\,\, \partial_{v_{\parallel}} f & = 0, && (v\text{-parallel advection),} & \label{eq:adv_par} \\
		\partial_t f + \{\phi, f\} & = 0, && \text{(Advection on poloidal plane),} \label{eq:adv_poloidal} &
	\end{align}
\end{subequations}
where $\{\phi,f\}$ is the Poisson bracket in polar coordinates defined in \eqref{eq:bracket_in_polar}. Splitting the Hamiltonian immediately and using the whole gyro-centre bracket would yield only 2 equations, namely \eqref{eq:adv_par} and \eqref{eq:adv_poloidal} would be kept together.\\

As described in \cite{emily} and \cite{Latu_2017}, we then obtain solution operators to each sub-system by the SL method, which will shortly be introduced in the next chapter. For now, we denote them by $A,B$ and $C$ solving the equations \eqref{eq:adv_flux}, \eqref{eq:adv_par} and \eqref{eq:adv_poloidal} for a given time-step and potential $\phi$ respectively.
For a time step $\Delta \tau \in \bR$, the first-order Lie splitting of $f^n = f(t = n \Delta \tau)$ reads
\begin{equation}
	f^{n+1/2} = C\left(\frac{\Delta \tau}{2}\right) B\left(\frac{\Delta \tau}{2}\right) A\left(\frac{\Delta \tau}{2}\right) f^n, \label{eq:Lie}
\end{equation}
while the second-order Strang splitting is given by
\begin{equation}
	f^{n+1} = A\left(\frac{\Delta \tau}{2}\right) B\left(\frac{\Delta \tau}{2}\right) C\left(\Delta \tau\right) B\left(\frac{\Delta \tau}{2}\right) A\left(\frac{\Delta \tau}{2}\right) f^n. \label{eq:Strang}
\end{equation}
So far, we assumed that we have a good approximation of the potential $\phi$ at hand. In practice, this is obtained by solving the QN equation i.e. \eqref{eq:qn} by a spectral Finite Element method (FEM) combined with some Finite Difference (FD) approximations. Since this is not the main part of our work, we refer to the sources above for more details.

In total, the iterative predictor-corrector solution procedure $f^n \rightarrow f^{n+1}$ can be described as follows:
\begin{enumerate}
	\item Given $f^n$, obtain $\phi$ from solving \eqref{eq:qn}.
	\item Given $\phi$, obtain $f^{n+1/2}$ by applying \eqref{eq:Lie}. (predictor step)
	\item Given $f^{n+1/2}$, obtain $\phi$ from solving \eqref{eq:qn} again.
	\item Given $\phi$, obtain $f^{n+1}$ by applying \eqref{eq:Strang}. (corrector step)
\end{enumerate}

This approach has shown to be quite successful, which inter alia is due to the unconditional stability of the SL scheme. Nonetheless, the method lacks structure preserving properties, like conservation of energy and $L^2$-norm. This gives rise to the main idea of this project, which exchanges the non-restrictive time-stepping of the SL method for the structure preservation of an AKW FD scheme, at least for the slow-time subsystem, where a time-step restriction is supposed to not be too computationally expensive. In other words, we solve the poloidal advection equation i.e. \eqref{eq:adv_poloidal} by applying the AKW scheme combined with a suitable time-integration. Since this advection equation essentially consists of a Poisson bracket, which is rich in geometric structure - exactly what the AKW scheme was designed for - we aim at improving the overall conservation properties of the full simulation.







\subsection{Semi-Lagrangian Scheme for the Fast Time Subsystem}
%TODO: What are the conservation properties of the SL scheme
%TODO: Are these good sources?
Recapitulating the semi-Lagrangian method, mostly referring to the overview given in \cite{campospinto}, we will start from the general formulation, where the goal is to solve an advection problem of the form
\begin{equation}
    \partial_t f(t,\mathbf{x})+v(t,\mathbf{x})\cdot \nabla f(t,\mathbf{x})=0, \qquad t\in[0,T], \quad \mathbf{x}\in\mathbb{R}^d, \label{eq:adv_toy}
\end{equation}
where $v$ is a velocity field $\mathbb{R}^d\longrightarrow\mathbb{R}^d$, $T > 0$ is the final time and initial conditions are given by $f_0(\mathbf{x})=f(t = 0,\mathbf{x})$. Assuming that $v$ is a given and smooth vector-field, we can use the method of characteristics, i.e. obtaining trajectories $X(t)=X(t;s,x)$ that are solutions to the ODE
\begin{equation}
    X'(t)=v(t,X(t)), \qquad X(s)=x, \qquad t\in[0,T],
\end{equation}
for $\bx \in \bR^d$ and $s \in [0,T]$. It can be shown that the flow $F_{s,t}:x\longrightarrow X(t)$ is invertible and satisfies $(F_{s,t})^{-1}=F_{t,s}$. Thus, we can derive the analytical solution to equation \eqref{eq:adv_toy} as
\begin{equation}
    f(t,\mathbf{x})=f_0((F_{0,t})^{-1}(\bx)),\qquad  t\in[0,T], \quad \mathbf{x}\in\mathbb{R}^d.
\end{equation}
This implies, given two consecutive time-steps $t_n$ and $t_{n+1}$, we can define the backwards flow
\begin{equation}
    B^{n,n+1}=(F_{t_n,t_{n+1}})^{-1},
\end{equation}
in order to advance the solution $f^n$ from time $t_n$ to time $t_{n+1}$, i.e. $f^{n+1} = f^n \circ B^{n, n+1}$. So far, the derivation has been completely analytical. In practice however, we have to introduce approximation errors for discretizing the distribution $f$, e.g. a Spline interpolation on a grid, and for the backwards flow $B$, which generally depends on the discretization of the vector-field $v$.\\

Now we are in the position to apply this methodology to the flux surface and $\vp$-advection, which is, again, discussed in \cite{emily} and \cite{Latu_2017}. For the flux surface advection equation \eqref{eq:adv_flux}, i.e.
%TODO: Do we have to recall that equation here? or is ref enough
\begin{equation}
 \partial_t f + v_\parallel \nabla_\parallel f = 0,
\end{equation}
where we remind that $\nabla_\parallel = \bb \cdot \nabla$. This is a one-dimensional constant velocity advection with velocity $\vp \bb$ on the flux surface $(\theta, z)$ for each given $r$. Thus, we can construct an analytical two-dimensional SL operator, that uses the exact trajectory as the velocity is not related to the flux surface. On the other hand, the $v$-parallel advection operator defined by equation \eqref{eq:adv_par}, i.e.
\begin{equation}
    \partial_t f + \nabla_\parallel \phi\,\, \partial_{v_{\parallel}} f = 0,
\end{equation}
 contains the parallel gradient of $\phi$ which only depends on the spatial coordinates and is therefore constant along $v_\parallel$. As a result, the trajectory used by this one-dimensional SL method can be accurately defined, while the parallel gradient of $\phi$ is computed using a field-aligned FD method. \\
 
When implementing the method, we work on a four dimensional computational grid on which the point values of the distribution functions and potentials for the different time-stages are known or calculated.
Both advection steps use special interpolation techniques, since we will not end up exactly on grid points when tracing back the characteristics, such that they are at least of order three. Additionally, we have to take account for boundary conditions, when the characteristics move outside the computational domain, extending it by extrapolation for instance. \\







\subsection{Arakawa Scheme for the Slow Time Subsystem}
\label{sec:AKW}
Introducing the Arakawa scheme, we mainly reference the original article \cite{Arakawa_1966}, where one is interested in the spatial two-dimensional discretization of the differential equation
\begin{equation}
	\partial_t f + \{\phi, f\} = 0, \label{eq:br_ode}
\end{equation}
where $\phi$ is a given potential and $\{\phi,f\}$ is a Poisson bracket of the form
\begin{equation}\label{eq:poisson_bracket}
	\{\phi,f\} = -  \left(\partial_x\phi\right) \left(\partial_y f\right) + \left(\partial_y\phi\right) \left(\partial_x f\right) \, .
\end{equation}

The main aim of this scheme is the conservation of the following properties
\begin{subequations}\label{conservation-properties}
	\begin{align}
		\text{Mass} : && \dt \int f(t) \d x \text{d} y & = 0 & \Leftrightarrow && \int\bracket{\phi}{f} \d x \text{d} y & = 0 \, , \label{eq:consv_1}\\
		L^2\text{-norm :} && \dt \int f^2(t) \d x \text{d} y & = 0 & \Leftrightarrow && \int f \, \bracket{\phi}{f} \d x \text{d} y & = 0 \, , \\
		\text{Total energy :} && \dt \int \phi \, f(t) \d x \text{d} y & = 0 & \Leftrightarrow && \int \phi \bracket{\phi}{f} \d x \text{d} y & = 0 \, ,
	\end{align}
\end{subequations}
which is deeply embedded in its construction. Albeit we are interested in the application of the scheme in polar coordinates, i.e. a change of coordinates in the Poisson bracket c.f. equation \eqref{eq:bracket_in_polar}, we will for simplicity start with the construction in Cartesian coordinates since conceptionally it does make no difference.



\subsubsection{Construction of the Discrete Bracket}
\label{sec:const_stenc}
Given a two-dimensional grid $(x_i, y_j)$ for $0 \le i \le N_x, 0 \le j \le N_y$ with a uniform grid size $d >0$, we simplify notation by writing $g_{i,j} = g(x_i, y_j)$ for any function $g$ evaluated at the point $(x_i, y_j)$ and write the collection of point-values as discrete function $g_h$. Denoting the Poisson bracket by 
\begin{equation} \label{eq:J-bracket}
	J(f, g) = \{f, g\}
\end{equation} and following \cite{Arakawa_1966}, we can approximate it at any point $(x_i, y_j)$ as a certain linear combination of the following nine-point stencils, that amounts to
\begin{equation}
	J(f,g)=J_1(f_h, g_h)+\mathcal{O}(d^2), \quad \text{ where } \quad J_1 = \frac{1}{3}(J_1^{++}+J_1^{+\times}+J_1^{\times+}),
\end{equation}
with the stencils defined as
\begin{equation}
	\begin{aligned}
    J_1^{++}& =\frac{1}{4d^2}\left[(f_{i+1,j}-f_{i-1,j})(g_{i,j+1}-g_{i,j-1}) \right. \\
	& \hspace{11mm}  \left.-(f_{i,j+1}-f_{i,j-1})(g_{i+1,j}-g_{i-1,j})\right], \label{eq:stencil1}
	\end{aligned}
\end{equation}
as well as
\begin{equation}
    \begin{aligned}
    	J_1^{+\times} & = \frac{1}{4d^2} \left[ f_{i+1,j} (g_{i+1,j+1} - g_{i+1,j-1}) - f_{i-1,j} (g_{i-1,j+1} - g_{i-1,j-1}) \right. \\
    	& \hspace{11mm} \left. - f_{i,j+1} (g_{i+1,j+1} - g_{i-1,j+1}) + f_{i,j-1}(g_{i+1,j-1} - g_{i-1,j-1}) \right],
    \end{aligned}
\end{equation}
and
\begin{equation}
    \begin{aligned}
    	J_1^{\times+} & = \frac{1}{4d^2} \left[ f_{i+1,j+1} (g_{i,j+1} - g_{i+1,j}) - f_{i-1,j-1} (g_{i-1,j} - g_{i,j-1}) \right. \\
    	& \hspace{11mm} \left. - f_{i-1,j+1} (g_{i,j+1} - g_{i-1,j}) + f_{i+1,j-1} (g_{i+1,j} - g_{i,j-1}) \right].
    \end{aligned}
\end{equation}
This approximation is proven to be the only FD second order approximation of the analytical Poisson bracket that conserves mass, $L^2$-norm and energy for an isotropic mesh. The $+$ and $\times$ notation comes from the patterns done by the FD stencils on the grid, as is visualized in Figure \ref{fig:stencils}. So for example in $J_1^{+\times}$, we choose the $+$-pattern, which corresponds to two central FD schemes, for the discretization of $\partial_x f_{i,j}$ and $\partial_y f_{i,j}$ and the same with the $\times$-pattern for $g_{i,j}$, then multiply the two together getting a discretization of the bracket \eqref{eq:poisson_bracket}.


% Figure environment removed

In order to continue these considerations to fourth order, we introduce $J_2(f_h, g_h)=\frac{1}{3}(J_2^{\times \times} + J_2^{\times+} + J_2^{+\times})$, where we now use the extended twelve point-stencils with the additional
four points $(i+2,j)$, $(i-2,j)$, $(i,j+2)$ and $(i,j-2)$, visualized in Figure \ref{fig:stencils}, such that
\begin{equation}
\begin{aligned}
	J_2^{\times\times} &= \frac{1}{8d^2} \left[(f_{i+1,j+1} - f_{i-1,j-1}) (g_{i-1,j+1} - g_{i+1,j-1}) \right. \\
	& \hspace{11mm} - \left. (f_{i-1,j+1} - f_{i+1,j-1}) (g_{i+1,j+1} - g_{i-1,j-1}) \right] \, ,
\end{aligned}
\end{equation}
as well as
\begin{equation}
	\begin{aligned}
		J_2^{\times+} & = \frac{1}{8d^2} \left[ f_{i+1,j+1} (g_{i,j+2} - g_{i+2,j}) - f_{i-1,j-1} (g_{i-2,j} - g_{i,j-2}) \right. \\
		& \hspace{11mm} \left. - f_{i-1,j+1} (g_{i,j+2} - g_{i-2,j}) + f_{i+1,j-1} (g_{i+2,j} - g_{i,j-2})\right] \, ,
	\end{aligned}
\end{equation}
and
\begin{equation}
	\begin{aligned}
		J_2^{+\times} & = \frac{1}{8d^2} \left[f_{i+2,j} (g_{i+1,j+1} - g_{i+1,j-1}) - f_{i-2,j} (g_{i-1,j+1} - g_{i-1,j-1}) \right. \\
		& \hspace{11mm} \left. - f_{i,j+2} (g_{i+1,j+1} - g_{i-1,j+1}) + f_{i,j-2} (g_{i+1,j-1} - g_{i-1,j-1})\right] \, . \label{eq:stencil2}
	\end{aligned}
\end{equation}
In total, we can now approximate the Poisson bracket $J$ at any point on the grid by
\begin{equation}
	J(f,g) = 2J_1(f_h, g_h)-J_2(f_h, g_h) + \mathcal{O}(d^4),
\end{equation}
up to fourth order while conserving all the desired quantities above as is shown in \cite{Arakawa_1966}. This leads us to define the discrete Poisson bracket $J_h$ on any point of the grid as
\begin{equation} \label{order4-disc-bracket}
	J_{h}(f_h, g_h) = 2J_1(f_h, g_h) - J_2(f_h, g_h),
\end{equation}
thus being able to spatially discretize equation \eqref{eq:br_ode} as
\begin{equation}
	\partial_t f_h = - J_h(\phi_h, f_h),
\end{equation}
such that the right-hand-side is approximated up to order four. By construction, this skew-symmetric discrete bracket satisfies the algebraic properties
\begin{subequations}
	\begin{align}
		\sum_{i,j} J_{h, (i,j)}(\phi_h, f_h) d^2 &= 0,  \\
		\sum_{i,j} f_{i,j} J_{h, (i,j)}(\phi_h, f_h) d^2 &= 0, \\
		\sum_{i,j} \phi_{i,j} J_{h, (i,j)}(\phi_h, f_h) d^2&= 0, 
	\end{align}
\end{subequations}
as is calculated in \cite{Arakawa_1966}, which leads to 
\begin{subequations}
	\begin{align}
		\partial_t \sum_{i,j}  f_{i,j} d^2&= 0,  \\
		\partial_t \sum_{i,j}  f_{i,j}^2 d^2 &= 0, \\
		\partial_t \sum_{i,j}  (\phi f)_{i,j} d^2 &= 0,
	\end{align}
\end{subequations}
where $\partial_t \phi = 0$. This the discrete analogous of the conservation properties of the equations \eqref{conservation-properties}.
% \begin{subequations}
% 	\begin{align}
% 		\frac{\dd}{\dd t} \int f_h \ \dd x \dd y &= \frac{\d}{\d t} \sum_{i,j} f_{i,j} \ d^2  = - \int J_h(\phi, f) \ \dd x \dd y = -\sum_{i,j} J_{h, (i,j)}(\phi, f) \ d^2 = 0, \label{eq:consv_1} \\
% 		\frac{\dd}{\dd t} \int f^2 \ \dd x \dd y &= - \int f J_h(\phi, f) \ \dd x \dd y \approx -\sum_{i,j} f_{i,j} J_{h, (i,j)}(\phi, f) \ d^2 = 0, \\
% 		\frac{\dd}{\dd t} \int \phi f \ \dd x \dd y &= - \int \phi J_h(\phi, f) \ \dd x \dd y \approx -\sum_{i,j} \phi_{i,j} J_{h, (i,j)}(\phi, f) \ d^2 = 0. \\
% 	\end{align}
% \end{subequations}





\subsubsection{Boundary Conditions} \label{sec:BC}

For points close to or on the boundary of the computational grid, the stencil might use points outside the domain that need to be defined. This is a problem that has to be tackled by introducing boundary conditions (BC) or locally reformulating the stencils. The stencils involve two functions in two spatial variables each, where each direction can be treated individually as we will see shortly and where the motivation comes from the physical properties described in Section \ref{sec:gk-model}. Importantly, the choice of boundary conditions also influences the conservation of mass, $L^2$-norm and energy.

The easiest option is taking periodic BC, as defined in \cite{Arakawa_1966}, where $x_{N_x + 1} = x_1, x_{N_x + 2} = x_2, x_{0} = x_{N_x}, x_{-1} = x_{N_x-1}$, etc. and the same for the $y$-direction. Looking at the stencils \eqref{eq:stencil1}-\eqref{eq:stencil2}, this leaves us only with interior points, where the function-values are known. For our model, this BC is used when looking at point sequences along the $\theta$ direction for both functions as physically this variable is periodic as described in Section \ref{sec:gk-model}.

Secondly, we can define homogeneous Dirichlet BC, i.e. the function values on or outside  the boundary of the grid are equal zero. This means for the stencils \eqref{eq:stencil1}-\eqref{eq:stencil2}, whenever an index is smaller two or greater $N_{x/y}-1$, the corresponding function value is set equal zero. Thus, we end up with reduced stencils, if the point is close to a grid boundary. We will apply this technique to the $r$-direction of the potential $\phi$, which is supposed to be zero outside the interior as described in Section \ref{sec:gk-model}.

Finally, we want to introduce extrapolatory BC; knowing the function values on the boundary and outside the grid, we can directly impose them in the stencils. To make that more precise, we assume $f$ outside the interior of the domain in one variable direction $x$ is known and described by the equilibrium function $f_\text{eq}$. Note that $f_\text{eq}$ still has to fulfil the BC in the other variable direction as we use its values on the boundary. In other words, we directly define the stencils \eqref{eq:stencil1}-\eqref{eq:stencil2}, with the outside values
\begin{equation}
	f_{i, j} = f_\text{eq}(x_i, y_i) \qquad \text{for} \quad  i< 1, i > N_x,  1 \le j \le N_y,
\end{equation}
 where the points in $x$-direction are linearly extended by $d$, e.g. $x_{-1} = x_1 - 2d$. This is exactly the situation of the radial direction of the distribution function $f$ in our model, with an equilibrium function $f_\text{eq}$ that is periodic in $\theta$, c.f. Section \ref{sec:gk-model}.

In \cite{crouseilles2018exponential}, these different kinds of BC with respect to conservation their properties are discussed in more detail, albeit only for the second-order scheme. They show that only the purely periodic BC have perfect conservation properties, for the others one still conserves mass, $L^2$-norm and energy well, but not up to machine precision as there is a small error introduced at the boundary. By their numerical experiments, the AKW scheme works best when combining theses different BC, where they conclude the same combination as we proposed above.





\subsubsection{Transformation to Polar Coordinates}\label{sec:consv-props}
\label{sec:polar}
The original AKW scheme \cite{Arakawa_1966} and all our considerations so far were formulated in Cartesian coordinates on an isotropic mesh, i.e. $\Delta x = \Delta y$. When going to an anisotropic mesh, the convergence order of the stencils still holds true, as we calculate in Appendix \ref{app:Order_of_Arakawa_stencil}. However, the gyro-kinetic model and \texttt{PyGyro} code in \cite{pygyro_code} to which we want to apply the Arakawa scheme, are also formulated in polar coordinates $(r,\theta, z, v_\parallel)$. Transforming the previous part to polar coordinates, we first introduce a polar grid $(r_i, \theta_j)$ for $0 \le i \le N_r, 0 \le j \le N_\theta$ with grid-increments $\Delta r >0$ and $\Delta \theta > 0$. The polar bracket $\{\cdot, \cdot\}^p$ of our model is given by \eqref{eq:bracket_in_polar}, dropping the constant $B_0$, which is different to the 
bracket $\{\cdot, \cdot\}^c$ in Cartesian coordinates from \eqref{eq:poisson_bracket} by a factor of $r$. 
This is due to a change of metric coming from the coordinate transformation from Cartesian to polar coordinates, i.e.
\begin{align}
	x  = r \cos(\theta), \qquad y  = r \sin(\theta),
\end{align}
which gets more clear when looking at the quantities under the integral: 
\begin{subequations}
	\begin{align}
		\int\bracket{\phi}{f}^c \d x \d y &= \int\bracket{\phi}{f}^p r \d r \d \theta, \\
		\Longleftrightarrow   \int  \left[ -\left(\partial_x\phi\right) \left(\partial_y f\right) + \left(\partial_y\phi\right) \left(\partial_x f\right) \right]  \d x \d y &=  \int \left[- \frac{1}{r} \left(\partial_\theta\phi\right) \left(\partial_r f\right) + \frac{1}{r} \left(\partial_r\phi\right) \left(\partial_\theta f\right)\right] r \d r \d \theta, \\
		\Longleftrightarrow  \int J^c(f, \phi)  \d x \d y &= \int J^p(f, \phi) r \d r \d \theta,
	\end{align}
\end{subequations}
where $J^c$ is defined in \eqref{eq:J-bracket} and 
\[ J^p(f,g) = \left[- \frac{1}{r} \left(\partial_\theta\phi\right) \left(\partial_r f\right) + \frac{1}{r} \left(\partial_r\phi\right) \left(\partial_\theta f\right)\right] .\] 
Analogous to the previous Section \ref{sec:const_stenc}, we define the discrete bracket at any point of the polar grid as
\begin{equation}
	J^p_{h, (i,j)}(f_h, g_h) = \frac{1}{r_i} J^c_{h, (i,j)}(f_h, g_h),
\end{equation}
from which we obtain the discrete conserved quantities 
\begin{align} \label{eq:pol_cons_quant}
	 \sum_{i,j}  f_{i,j} r_i \Delta r \Delta \theta = 0,  \qquad \sum_{i,j}  f_{i,j}^2 r_i \Delta r \Delta \theta  = 0, \qquad \sum_{i,j}  (\phi f)_{i,j} r_i \Delta r \Delta \theta  = 0,
\end{align}
for mass, $L^2$-norm and energy respectively. As before, their conservation is equivalent to the equations 
\begin{subequations}\label{eq:alg_prop}
	\begin{align} 
		\sum_{i,j} J^p_{h, (i,j)}(\phi_h, f_h) r_i \Delta r \Delta \theta &= 0,  \\
		\sum_{i,j} f_{i,j} J^p_{h, (i,j)}(\phi_h, f_h) r_i \Delta r \Delta \theta &= 0, \\
		\sum_{i,j} \phi_{i,j} J^p_{h, (i,j)}(\phi_h, f_h) r_i \Delta r \Delta \theta &= 0. 
	\end{align}
\end{subequations}

% !TeX spellcheck = en_GB

\section{Numerical Experiments}
\label{sec:num_exp}


While the Arakawa method was implemented in \textit{Python} in order to integrate it seamlessly into the \texttt{PyGyro} code, efforts were made to maximize performance in order to do large scale simulations with more than 300 million degrees of freedom in feasible times. This section is devoted to discuss the implementation of the Arakawa scheme, its integration to the \texttt{PyGyro} code, and further numerical experiments and verifications.




\subsection{Implementation of the Discrete Bracket}

Since the potential $\phi$ is constant while performing the poloidal step, which is defined on a polar domain similar to Section \ref{sec:polar}, and in order to create a computationally efficient scheme, we choose to implement the discrete bracket as a \texttt{scipy} sparse matrix $\mathbb{J}_\phi$ of size $(N_r N_\theta)^2$, that is constructed only dependent on the point-values of $\phi$, mapping the point-values of the current distribution function $f$, such that
\begin{equation}
	\mathbb{J}_\phi f_h = J^p_h(\phi_h, f_h) \approx \{\phi, f\}^p.
\end{equation}
After every time step, we update the non-zero entries of this matrix in-place using the new values of $\phi$. These entries are computed in a \textit{Fortran} routine which was generated using \texttt{pyccel} (see \cite{pyccel}) achieving near-native \textit{Fortran}-performance with much less development time. An explicit time integrator then uses matrix-vector multiplication which is computed in \textit{C} thanks to the usage of \texttt{scipy}. An implicit time stepping makes use of the implemented sparse solvers, also provided by \texttt{scipy}.\\

In order to test if the conserved quantities in equation \eqref{eq:pol_cons_quant} hold, we can directly calculate the equivalent algebraic conditions from equation \eqref{eq:alg_prop}, which should only depend on the definition of $J_h$ and are independent of the actual point values in $f_h$ and $\phi_h$ as long as they satisfy the boundary conditions.  

This is interesting with respect to the discussion of the conservation properties depending on the different BC in \cite{crouseilles2018exponential} and Section \ref{sec:BC}, we therefore implemented all BC discussed in Section \ref{sec:BC}, with additional possible periodic and Dirichlet BC in radial direction of the distribution function $f$. 
\begin{table}[h]
	\begin{tabular}{| l | l | l | l | l | l | l|l|l|}
		\hline
		BC	& Order & Mass & $L^2$-norm & Energy & Order & Mass & $L^2$-norm & Energy \\
		\hline
		Periodic& 2 &  1.47e-14 & 2.62e-14 & 1.50e-14 & 4 & 3.93e-14 & 5.57e-14 & 7.44e-14  \\ \hline
		Dirichlet& 2 &  1.35e-13 & 9.09e-13 & 8.67e-13 & 4 & 4.12e-13 & 4.37e-11 & 3.40e-12 \\ \hline
		Extrapolation& 2 &  8.53e-14 & 1.09e-11 & 4.57e-12 & 4 & 2.56e-13 & 2.55e-11 & 1.63e-11 \\
		\hline
	\end{tabular}
	\medskip
	\caption{Algebraic conservation properties, i.e. equation \eqref{eq:alg_prop}, for vectors of size $N_rN_\theta$, with $N_r = N_\theta = 64$, where $f_h \in \bR^{N_rN_\theta}$ and $\phi_h \in \bR^{N_rN_\theta}$ have uniformly distributed values between $-100$ and $100$, while satisfying the BC.}
	\label{tab:alg_prop}
\end{table}

The results of this first test can be found in Table \ref{tab:alg_prop}. Even though the values are not on machine-precision, as was predicted in \cite{crouseilles2018exponential} and which is due the imperfect conservation at the non-periodic boundaries, they are still very small, and we expect good conservation properties from using this scheme in the full code. We also observed, that these values increase proportionally to the norm of $\phi$, which may should be normalized in the indicator. 


\newpage

\subsection{Poloidal Advection Test-Case}

In order to further verify the AKW scheme and its properties, we look at a test-case similar to Section 3.5.2 in \cite{emily}, that is, we consider a poloidal advection on a polar domain with $r \in [1, 20]$ and $\theta \in [0, 2\pi]$. The equilibrium distribution function $f_\text{eq}$ reads
\begin{equation}
	f_\text{eq}(r, v_\parallel) = \frac{n_0(r)}{\sqrt{2\pi T_i(r)}} \exp\left( - \frac{v_\parallel^2}{2 T_i(r)} \right)
\end{equation}
with radial profiles
\begin{equation}
	\mathcal{P}(r) = C_{\mathcal{P}} \exp\left[ - \kappa_\mathcal{P} \delta r_\mathcal{P} \tanh \left( \frac{r - r_p}{\delta r_\mathcal{P}} \right) \right]
\end{equation}
for $\mathcal{P} \in \{T_i, n_0\}$ and with constants
\begin{align}
	C_{T_i} & = 1 & C_{n_0} & = \left(r_\text{max} - r_\text{min}\right) \left[\int_{r_\text{min}}^{r_\text{max}} \exp\left[ - \kappa_{n_0} \delta r_{n_0} \tanh \left( \frac{r - r_p}{\delta r_\mathcal{P}} \right) \right] \d r \right]^{-1}
\end{align}
and parameters
\begin{align*}
	B_0 & = 0 \, , & R_0 & = 239.8081535 \, , & r_\text{min} & = 0.1 \, , & r_\text{max} & = 14.5 \, , & r_p & = \frac{r_\text{max} - r_\text{min}}{2} \, , \\
	\epsilon & = 10^{-6} \, , & \kappa_{n_0} & = 0.055 \, , & \kappa_{T_i} & = 0.27586 \, , & \delta r_{T_i} & = \frac{\delta r_{n_0}}{2} = 1.45 \, , & \delta r & = \frac{4 \delta r_{n_0}}{\delta r_{T_i}} \, .
\end{align*}

Initial potential $\phi$ and initial distribution function $f$ are given by
\begin{subequations}
	\begin{align}
		\phi(r, \theta) & = -5 r^2 + \sin(\theta), \\
		f(t = 0, r, \theta) & = f_\text{eq}(r, \vp=0) + B(r, \theta), \\
		B(r, \theta) & = \begin{cases}
			\cos(\frac{\pi}{8} \sqrt{(r - 7)^2 + 2(\theta - \pi)^2}), & \text{if } \sqrt{(r - 7)^2 + 2(\theta - \pi)^2} \le 4, \\
			0, & \text{else}.
		\end{cases},
	\end{align}
\end{subequations}
where the radius-dependent equilibrium function $f_\text{eq}(r)$ is the same as in \cite{Latu_2017} with a fixed $\vp = 0$. For this system, the exact trajectories are given by
\begin{subequations}
	\begin{align}
		\theta(t) &= \theta_0 - 10 t, \\
		r(t) &= \sqrt{ r_0^2 + \frac15 \left( \sin\left(\theta_0 - 10 t\right) - \sin\left(\theta_0 \right)\right)},
	\end{align}
\end{subequations}
for initial points $(r_0, \theta_0)$. Following these characteristics with the initial distribution yields an exact solution to our problem, that can be used to calculate the error of our numerical scheme. The initial distribution is of Gaussian-like shape, as displayed in Figure \ref{fig:init_f}, and gets advected around the center while keeping its shape.

\begin{center}
	\begin{minipage}[h]{0.5\textwidth}
		\centering
		% Figure environment removed
	\end{minipage}\begin{minipage}[h]{0.5\textwidth}
		\centering
		% Figure environment removed
	\end{minipage}
\end{center}
\vspace{0.3cm}

We use the AKW scheme as introduced in Section \ref{sec:AKW} with extrapolatory BC, given by the equilibrium function, and different mesh sizes. The time integration is done by an explicit Runge-Kutta scheme of order $4$, where we take the step-size $\Delta t = 0.001 / N_r$ and perform a total number of $N = T_\text{end} / dt$ steps to the final time $T_\text{end} = 0.02$. This way, the time-discretization is fine enough, such that the dominant error is coming from the spatial discretization of the bracket only. The numerical results are listed in Table \ref{tab:num_exp}.
\begin{table}[h]
    \begin{tabular}{| l | l | l | l | l | l | l|l|l|}
    \hline
     $N_r$& $N_\theta$ & $L^2$-error & order & conserved mass & conserved $L^2$-norm & conserved energy \\
	\hline
	%8 & 8 & 8.07e-02 & &1.40e-06 & 2.24e-06 & 7.42e-09\\ \hline order:3.39
	16 & 16 & 7.72e-03 &  &9.20e-10 & 1.43e-09 & 9.56e-12\\ \hline
	32 & 32 & 5.62e-04 & 3.78 &8.06e-10 & 1.25e-09 & 5.79e-12\\ \hline
	64 & 64 & 3.96e-05 & 3.83 &1.01e-09 & 1.57e-09 & 6.49e-12\\ \hline
	128 & 128 & 4.74e-06 & 3.06 &1.04e-09 & 1.61e-09 & 5.82e-12\\ \hline
    \end{tabular}
	\medskip
	\caption{Spatial discretization and conservation errors for different mesh sizes $(N_r, N_\theta)$. The error of the solution with respect to the exact solution is calculated in the $L^2$-norm. The conservation errors are relative errors of the initial discrete quantities compared to their values at final time.}
	\label{tab:num_exp}
\end{table}
The error is not up to the desired order of $4$, this could be due to conflicting BC as discussed in \cite{crouseilles2018exponential}, but it is not far off and converges towards it for smaller time-steps. The conserved properties look similar to the ones in Table \ref{tab:alg_prop}, albeit being a few orders of magnitude higher which seems to be due to the bigger norm of $\phi$ and imperfect BC.  

As another test, we solve the system over a longer period of time and keep track of the mass, the $L^2$-norm and the energy. Taking $N_r = N_\theta = 64$, time-step $\Delta t = 0.01$ and perform $N = 500$ steps, yields relative errors of the conserved quantities and values of algebraic indicators as shown in Figure \ref{fig:cons}. %More analyis...

% Figure environment removed
The long time conservation seems to be in accordance with the results in Tables \ref{tab:num_exp} and \ref{tab:alg_prop}, albeit the algebraic indicators being larger than before due to the larger norm of $\phi$, as was observed before. 


\subsection{Full Gyro-kinetic Simulations}

The \texttt{PyGyro} code is a \textit{Python 3} library for gyro-kinetic simulations leveraging the acceleration provided by the modules \texttt{Pythran} (see \cite{Pythran}), \texttt{Numba} (see \cite{Numba}), or \texttt{Pyccel} (see \cite{pyccel}). It is highly parallelized using \texttt{MPI} and thus suitable for running even large-scale simulations on computing clusters. In the following, we will look at results from simulations of the full gyro-kinetic model of \cite{emily}, and then compare the new combination of Semi-Lagrangian and the Arakawa method to the results purely using the Semi-Lagrangian scheme. We will pay special attention to the energy conservation in the poloidal step. Since turbulences mostly occur in this substep, this is also where preserving energy is most crucial.

The parameters used are the same as those in the previous section. The grid size is as follows:
\begin{align*}
	& N_r = 128, && N_\theta = 256, &&& N_z = 128, &&&& N_{v_\parallel} = 72.
\end{align*}
The simulation also uses the following parameters for the initial perturbation:
\begin{align*}
	\iota & = 0, & m & \in \left\{5, 15\right\}, & n & = 1.
\end{align*}
The equilibrium function $f_\text{eq}$ as well as the radial profiles $\{T_i , T_e, n_0\}$ are defined in in the previous section.


Firstly we compare the Arakawa method to the second-order Semi-Lagrangian scheme by looking at the $L^2$-norm of the electric potential $\phi$, c.f. Figure \ref{fig:l2phi}. We observe the expected behaviour for both schemes where $\norm{\phi}_2$ follows the analytical growth rate of $\norm{\phi}_2 = 4\cdot 10^{-5} \times \exp\left(0.00354 t\right)$ very well in the linear regime (until $t \sim 3000$) (shown on the left in a semi-logarithmic plot). After that the $L^2$-norm saturates and settles at a value of around $8.0$. Notably, the two simulations agree very well even in the non-linear regime which is a good consistency check.

% Figure environment removed

Furthermore, we plot both the full distribution function and additionally its difference to the equilibrium distribution function in Figure \ref{fig:akwfullndiff}, up to $t=5000$ in steps of $1000$. Shown are slices at $z=0$ and $v_\parallel \simeq 0$ in a polar projection with variables $r$ and $\theta$. We see that it behaves as expected with turbulences forming after the linear regime is over, appearing in the plot for $t=4000$. Interesting to note is the fact that the difference to the equilibrium function seems to increase even for low-valued regions of the domain, although only by a relative difference of half a percentage point.
This effect also occurs in the simulation using the SL scheme, to an even larger extend, so it is not an artifact of the Arakawa scheme. 
%We can see that the same effect occurs in the simulation using the SL scheme, displayed in Figure \ref{fig:slfdiff}, so it is not an artifact of the Arakawa scheme. 
One point of discussion is, if this error is due to the boundary conditions, but compared to extrapolation BC, we see no improvement when switching to Dirichlet BC, in this case constant extrapolation by the boundary value, which seem to be the only other plausible option.

% % Figure environment removed

% % Figure environment removed

Our main point of comparison between the Arakawa method and the Semi-Lagrangian scheme is the conserved quantities in \eqref{conservation-properties}: the mass and $L^2$-norm of the distribution function, and the potential energy, as well as the kinetic energy
\begin{equation}
	E_\text{kin} = \frac{m}{2} \int \left(f(t, r, \theta, z, v_\parallel) - f_\text{eq}(r, v_\parallel)\right) v^2_\parallel \d r \d \theta \d z \d v_\parallel \, .
\end{equation} \\

From a simulation with the above grid size and time step-size $\Delta t = 1$, we investigate the conservation properties for the four above mentioned quantities by computing them before and after the poloidal advection step and plotting the absolute and relative errors in Figure \ref{fig:absnrelerrlog} on a semi-logarithmic scale.
\newpage

% Figure environment removed

\newpage

	% Figure environment removed

\newpage

We can clearly see that the Arakawa scheme preserves the conserved quantities much better than the Semi-Lagrangian scheme, although the error is of order of machine precision only in the linear phase. After that, the error grows but remains multiple orders of magnitude smaller than for the semi-Lagrangian scheme. The kinetic energy is also much better preserved which was to be expected, since the behaviour of conservation acts similarly to the mass of the distribution function but with $\vp^2$ weights on each $\vp$-slice, keeping in mind that $\vp$ is just a parameter in each poloidal advection step. The additional error that is introduced by the imperfect boundary conditions is responsible for the conservation properties being bigger than the machine precision; this is especially true for later times when the distribution function moves further away from the equilibrium that is assumed to hold outside the domain. \\

% % Figure environment removed

Lastly, we compare the two time-integrators in Figure \ref{fig:abserrlogakw} as described in the text: The second order Crank-Nicolson method and the fourth-order Runge-Kutta scheme, both for the full model simulation. For the RK4 scheme we have a CFL condition which reads
\begin{equation}
	CFL = \max\left(\mathbb{J}_\phi\right) \frac{\Dt}{\Delta x}
\end{equation}
This becomes restrictive only in the non-linear phase later on in the simulation. One can thus use the explicit RK4 with large time step-size $\Dt$, and either use the implicit CN2 method or the RK4 for the later time domain. By observing the accuracy in the discrete conservation of continuous invariants, we see that the difference between the two is negligible for early times, but the RK4 performs better for the very late phase.

% Figure environment removed


\newpage

\section{Conclusion and Future Work}
In this work, I design corruption-robust algorithms for the Lipschitz contextual search problem. I present the \emph{agnostic checking} technique and demonstrate its effectiveness in designing corruption-robust algorithms. There are several open problems for future research. First, in the algorithm I propose for pricing loss, the schedule for agnostic checks is fixed upfront. Can the learner design an adaptive checking schedule for the pricing loss? Second, this work assumes the learner has knowledge of the Lipschitz constant $L$. Can the learner design efficient no-regret algorithms without knowledge of $L$? 


\begin{acknowledgement}
	\textbf{Acknowledgements}\\
	The authors would like to thank Emmanuel Franck, Hélène Hivert, Guillaume Latu, Hélène Leman, Bertrand Maury, Michel Mehrenberger, and Laurent Navoret for organizing the CEMRACS conference 2022 and for the wonderful opportunity to come to Marseille and do research. We express special thanks to Michel Mehrenberger and Virginie Grandgirard for the daily supervision and general shaping of the project, and to Xue Hong for discussions on the theoretical side. Dominik Bell and Frederik Schnack thank Eric Sonnendrücker for the opportunity to participate in the CEMRACS and the fruitful discussions after the conference to give this project the final details. They also thank Pierre Navaro for great discussions on and off topic.
\end{acknowledgement}

\bibliographystyle{ieeetr}
\bibliography{literature}

\appendix
\addtocontents{toc}{\protect\setcounter{tocdepth}{0}}
% !TeX spellcheck = en_GB

\section{Order of the Arakawa Stencils}\label{app:Order_of_Arakawa_stencil}

For isotropic grids, the approximation properties of the Arakawa scheme has been shown in \cite{Arakawa_1966}. The following \textit{Mathematica} code calculates the approximation power of the discrete stencils on anisotropic meshes. First, we define the series expansion of $f$ and $\phi$ up to order 4:
\begin{subequations}
	\begin{align}
		\mathrm{F}(i, j) & = \text{Series}[f(x + i \Delta x , y+ j \Delta y),\{\Delta x,0,3\},\{\Delta y,0,3 \}] \,, \\
		\mathrm{Phi}(i, j) & = \text{Series}[\phi(x + i \Delta x , y+ j \Delta y),\{\Delta x,0,3\},\{\Delta y,0,3\}] \,.
	\end{align}
\end{subequations}


\subsubsection*{Second Order Scheme (Nine-Point-Stencils)}

Implementing the stencils from section \ref{sec:const_stenc}:
\begin{subequations}
	\begin{align}
		& \begin{aligned}
			\text{J1pp} & = \frac{1}{4 \Delta x \Delta y} \left[ (\text{F}(1,0)-\text{F}(-1,0)) (\text{Phi}(0,1)-\text{Phi}(0,-1)) \right. \\
		& \qquad \hphantom{\frac{1}{4 \Delta x \Delta y}} \left. - (\text{F}(0,1)-\text{F}(0,-1)) (\text{Phi}(1,0)-\text{Phi}(-1,0)) \right] ,
	    \end{aligned} \\
		& \begin{aligned}
			\text{J1px} & = \frac{1}{4 \Delta x \Delta y} \left[ -\text{F}(-1,0) (\text{Phi}(-1,1) - \text{Phi}(-1,-1)) + \text{F}(0,-1) (\text{Phi}(1,-1)-\text{Phi}(-1,-1)) \right. \\
			& \qquad \hphantom{\frac{1}{4 \Delta x \Delta y}} \left. -\text{F}(0,1) (\text{Phi}(1,1)-\text{Phi}(-1,1))+\text{F}(1,0) (\text{Phi}(1,1)-\text{Phi}(1,-1))\right],
		\end{aligned} \\
		& \begin{aligned}
			\text{J1xp} & = \frac{1}{4 \Delta x\Delta y} \left[- \text{F}(-1,1) (\text{Phi}(0,1)-\text{Phi}(-1,0))-\text{F}(-1,-1)
			(\text{Phi}(-1,0)-\text{Phi}(0,-1)) \right. \\
			& \qquad \hphantom{\frac{1}{4 \Delta x \Delta y}} \left. + \text{F}(1,-1) (\text{Phi}(1,0)-\text{Phi}(0,-1))+\text{F}(1,1) (\text{Phi}(0,1)-\text{Phi}(1,0)) \right],
		\end{aligned} \\
		& \text{J1} = \frac13 \left(\text{J1pp} + \text{J1px} + \text{J1xp}\right),
	\end{align}
\end{subequations}
we can look at the series expansion of $J_1$:
\begin{subequations}
	\begin{align}
		\mathrm{J1} = & f^{(1,0)}  \phi ^{(0,1)} - f^{(0,1)}  \phi ^{(1,0)} \\
	    & + \frac{1}{6}  \Delta x^2 \left(  f^{(2,0)} \phi^{(1,1)} -  f^{(1,1)} \phi^{(2,0)} -  f^{(2,1)} \phi^{(1,0)} +  f^{(1,0)} \phi ^{(2,1)} + f^{(3,0)} \phi^{(0,1)} -  f^{(0,1)} \phi^{(3,0)} \right) \\
		&+ \frac{1}{6} { \Delta y}^2  \left(f^{(1,0)} \phi^{(0,3)} - f^{(0,3)} \phi^{(1,0)} + f^{(1,1)} \phi^{(0,2)} - f^{(0,2)} \phi^{(1,1)} + f^{(1,2)} \phi^{(0,1)} - f^{(0,1)} \phi^{(1,2)} \right)  \\
	    & + \mathcal{O}(\Delta x^k \Delta y^l \ | \ k+l > 2)
	\end{align}
\end{subequations}
which shows us that the approximation is of order $2$.



\subsubsection*{Fourth Order Scheme (Thirteen-Point-Stencils)}

Similar to above, we define the stencils:
\begin{subequations}
	\begin{align}
		& \begin{aligned}
			\text{J2xp} & = \frac{1}{8 \Delta x \Delta y} \left[ - \text{F}(-1,1)(\text{Phi}(0,2)-\text{Phi}(-2,0)) - \text{F}(-1,-1)(\text{Phi}(-2,0)-\text{Phi}(0,-2)) \right. \\
			& \quad \hphantom{= \frac{1}{8 \Delta x \Delta y}} \left. + \text{F}(1,-1) (\text{Phi}(2,0)-\text{Phi}(0,-2)) + \text{F}(1,1)(\text{Phi}(0,2)-\text{Phi}(2,0)) \right] \, ,
		\end{aligned} \\
		& \begin{aligned}
			\text{J2px} & = \frac{1}{8 \Delta x \Delta y} \left[ - \text{F}(-2,0) (\text{Phi}(-1,1)-\text{Phi}(-1,-1)) + \text{F}(0,-2)(\text{Phi}(1,-1)-\text{Phi}(-1,-1)) \right. \\
			& \quad \hphantom{= \frac{1}{8 \Delta x \Delta y}} \left. - \text{F}(0,2) (\text{Phi}(1,1)-\text{Phi}(-1,1)) + \text{F}(2,0)(\text{Phi}(1,1)-\text{Phi}(1,-1)) \right] \, ,
	    \end{aligned} \\
		& \begin{aligned}
			\text{J2xx} & = \frac{1}{8 \Delta x \Delta y} \left[ (\text{F}(1,1)-\text{F}(-1,-1))(\text{Phi}(-1,1)-\text{Phi}(1,-1)) \right. \\
			& \quad \hphantom{= \frac{1}{8 \Delta x \Delta y}} \left. - (\text{F}(-1,1)-\text{F}(1,-1))(\text{Phi}(1,1)-\text{Phi}(-1,-1)) \right] \, , 
		\end{aligned} \\
		& \text{J2} = \frac{1}{3} (\text{J2px} + \text{J2xp} + \text{J2xx}) \, , \\
        & \text{Jh} = 2\text{J1} - \text{J2} \, ,
	\end{align}
\end{subequations}
and look at the series expansion of $J_h$:
\begin{align}
    \mathrm{Jh} & = f^{(1,0)}(x,y) \phi^{(0,1)}(x,y)-f^{(0,1)}(x,y) \phi ^{(1,0)}(x,y) \\
    &\quad -\frac{1}{12} \Delta x^4 \left( \phi^{(2,1)}(x,y)  f^{(3,0)}(x,y)- f^{(2,1)}(x,y)6\phi^{(3,0)}(x,y)\right)\\
    &\quad +\frac{1}{12} \Delta x^2 \Delta y^2 \left(\phi^{(1,2)}(x,y)  f^{(2,1)}(x,y)-f^{(1,2)}(x,y)  \phi^{(2,1)}(x,y)\right.\\
    &\quad -\frac{1}{36} \Delta x^2 \Delta y^2   \left(  f^{(3,0)}(x,y) \phi^{(0,3)}(x,y)+ f^{(0,3)}(x,y) \phi^{(3,0)}(x,y)\right) \\
    &\quad -\frac{1}{12} \Delta y^4 \left(f^{(1,2)}(x,y) \phi^{(0,3)}(x,y)-f^{(0,3)}(x,y) \phi^{(1,2)}(x,y)\right)\\
    &\quad + \mathcal{O}(\Delta x^k \Delta y^l \ | \ k+l > 4) \, ,
\end{align}
which shows the approximation order of $4$.








\end{document}
