% !TeX spellcheck = en_GB

\section{Order of the Arakawa Stencils}\label{app:Order_of_Arakawa_stencil}

For isotropic grids, the approximation properties of the Arakawa scheme has been shown in \cite{Arakawa_1966}. The following \textit{Mathematica} code calculates the approximation power of the discrete stencils on anisotropic meshes. First, we define the series expansion of $f$ and $\phi$ up to order 4:
\begin{subequations}
	\begin{align}
		\mathrm{F}(i, j) & = \text{Series}[f(x + i \Delta x , y+ j \Delta y),\{\Delta x,0,3\},\{\Delta y,0,3 \}] \,, \\
		\mathrm{Phi}(i, j) & = \text{Series}[\phi(x + i \Delta x , y+ j \Delta y),\{\Delta x,0,3\},\{\Delta y,0,3\}] \,.
	\end{align}
\end{subequations}


\subsubsection*{Second Order Scheme (Nine-Point-Stencils)}

Implementing the stencils from section \ref{sec:const_stenc}:
\begin{subequations}
	\begin{align}
		& \begin{aligned}
			\text{J1pp} & = \frac{1}{4 \Delta x \Delta y} \left[ (\text{F}(1,0)-\text{F}(-1,0)) (\text{Phi}(0,1)-\text{Phi}(0,-1)) \right. \\
		& \qquad \hphantom{\frac{1}{4 \Delta x \Delta y}} \left. - (\text{F}(0,1)-\text{F}(0,-1)) (\text{Phi}(1,0)-\text{Phi}(-1,0)) \right] ,
	    \end{aligned} \\
		& \begin{aligned}
			\text{J1px} & = \frac{1}{4 \Delta x \Delta y} \left[ -\text{F}(-1,0) (\text{Phi}(-1,1) - \text{Phi}(-1,-1)) + \text{F}(0,-1) (\text{Phi}(1,-1)-\text{Phi}(-1,-1)) \right. \\
			& \qquad \hphantom{\frac{1}{4 \Delta x \Delta y}} \left. -\text{F}(0,1) (\text{Phi}(1,1)-\text{Phi}(-1,1))+\text{F}(1,0) (\text{Phi}(1,1)-\text{Phi}(1,-1))\right],
		\end{aligned} \\
		& \begin{aligned}
			\text{J1xp} & = \frac{1}{4 \Delta x\Delta y} \left[- \text{F}(-1,1) (\text{Phi}(0,1)-\text{Phi}(-1,0))-\text{F}(-1,-1)
			(\text{Phi}(-1,0)-\text{Phi}(0,-1)) \right. \\
			& \qquad \hphantom{\frac{1}{4 \Delta x \Delta y}} \left. + \text{F}(1,-1) (\text{Phi}(1,0)-\text{Phi}(0,-1))+\text{F}(1,1) (\text{Phi}(0,1)-\text{Phi}(1,0)) \right],
		\end{aligned} \\
		& \text{J1} = \frac13 \left(\text{J1pp} + \text{J1px} + \text{J1xp}\right),
	\end{align}
\end{subequations}
we can look at the series expansion of $J_1$:
\begin{subequations}
	\begin{align}
		\mathrm{J1} = & f^{(1,0)}  \phi ^{(0,1)} - f^{(0,1)}  \phi ^{(1,0)} \\
	    & + \frac{1}{6}  \Delta x^2 \left(  f^{(2,0)} \phi^{(1,1)} -  f^{(1,1)} \phi^{(2,0)} -  f^{(2,1)} \phi^{(1,0)} +  f^{(1,0)} \phi ^{(2,1)} + f^{(3,0)} \phi^{(0,1)} -  f^{(0,1)} \phi^{(3,0)} \right) \\
		&+ \frac{1}{6} { \Delta y}^2  \left(f^{(1,0)} \phi^{(0,3)} - f^{(0,3)} \phi^{(1,0)} + f^{(1,1)} \phi^{(0,2)} - f^{(0,2)} \phi^{(1,1)} + f^{(1,2)} \phi^{(0,1)} - f^{(0,1)} \phi^{(1,2)} \right)  \\
	    & + \mathcal{O}(\Delta x^k \Delta y^l \ | \ k+l > 2)
	\end{align}
\end{subequations}
which shows us that the approximation is of order $2$.



\subsubsection*{Fourth Order Scheme (Thirteen-Point-Stencils)}

Similar to above, we define the stencils:
\begin{subequations}
	\begin{align}
		& \begin{aligned}
			\text{J2xp} & = \frac{1}{8 \Delta x \Delta y} \left[ - \text{F}(-1,1)(\text{Phi}(0,2)-\text{Phi}(-2,0)) - \text{F}(-1,-1)(\text{Phi}(-2,0)-\text{Phi}(0,-2)) \right. \\
			& \quad \hphantom{= \frac{1}{8 \Delta x \Delta y}} \left. + \text{F}(1,-1) (\text{Phi}(2,0)-\text{Phi}(0,-2)) + \text{F}(1,1)(\text{Phi}(0,2)-\text{Phi}(2,0)) \right] \, ,
		\end{aligned} \\
		& \begin{aligned}
			\text{J2px} & = \frac{1}{8 \Delta x \Delta y} \left[ - \text{F}(-2,0) (\text{Phi}(-1,1)-\text{Phi}(-1,-1)) + \text{F}(0,-2)(\text{Phi}(1,-1)-\text{Phi}(-1,-1)) \right. \\
			& \quad \hphantom{= \frac{1}{8 \Delta x \Delta y}} \left. - \text{F}(0,2) (\text{Phi}(1,1)-\text{Phi}(-1,1)) + \text{F}(2,0)(\text{Phi}(1,1)-\text{Phi}(1,-1)) \right] \, ,
	    \end{aligned} \\
		& \begin{aligned}
			\text{J2xx} & = \frac{1}{8 \Delta x \Delta y} \left[ (\text{F}(1,1)-\text{F}(-1,-1))(\text{Phi}(-1,1)-\text{Phi}(1,-1)) \right. \\
			& \quad \hphantom{= \frac{1}{8 \Delta x \Delta y}} \left. - (\text{F}(-1,1)-\text{F}(1,-1))(\text{Phi}(1,1)-\text{Phi}(-1,-1)) \right] \, , 
		\end{aligned} \\
		& \text{J2} = \frac{1}{3} (\text{J2px} + \text{J2xp} + \text{J2xx}) \, , \\
        & \text{Jh} = 2\text{J1} - \text{J2} \, ,
	\end{align}
\end{subequations}
and look at the series expansion of $J_h$:
\begin{align}
    \mathrm{Jh} & = f^{(1,0)}(x,y) \phi^{(0,1)}(x,y)-f^{(0,1)}(x,y) \phi ^{(1,0)}(x,y) \\
    &\quad -\frac{1}{12} \Delta x^4 \left( \phi^{(2,1)}(x,y)  f^{(3,0)}(x,y)- f^{(2,1)}(x,y)6\phi^{(3,0)}(x,y)\right)\\
    &\quad +\frac{1}{12} \Delta x^2 \Delta y^2 \left(\phi^{(1,2)}(x,y)  f^{(2,1)}(x,y)-f^{(1,2)}(x,y)  \phi^{(2,1)}(x,y)\right.\\
    &\quad -\frac{1}{36} \Delta x^2 \Delta y^2   \left(  f^{(3,0)}(x,y) \phi^{(0,3)}(x,y)+ f^{(0,3)}(x,y) \phi^{(3,0)}(x,y)\right) \\
    &\quad -\frac{1}{12} \Delta y^4 \left(f^{(1,2)}(x,y) \phi^{(0,3)}(x,y)-f^{(0,3)}(x,y) \phi^{(1,2)}(x,y)\right)\\
    &\quad + \mathcal{O}(\Delta x^k \Delta y^l \ | \ k+l > 4) \, ,
\end{align}
which shows the approximation order of $4$.






