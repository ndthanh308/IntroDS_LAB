% !TeX spellcheck = en_GB

\section{Introduction}

Gyro-kinetic models have become a very popular choice for simulating plasmas since they reduce the phase space compared to the Vlasov equation and are thus more computationally viable while still describing the important physics. This reduction is done by averaging out the fast motion of particles around the magnetic field lines, called gyration.

In the works \cite{idomura2007} and \cite{idomura2008conservative} the authors use a Morinishi finite differences (FD) scheme \cite{Morinishi1998FullyCH} for the advection equations, which conserves the mass and $L^2$-norm of the distribution function. In \cite{crouseilles2018exponential}, the gyro-kinetic equation is split; the linear part of the advection is solved using Fourier techniques while the non-linear part is discretized using the FD Arakawa scheme from \cite{Arakawa_1966} of order 2, which aims at preserving the kinetic energy and the square vorticity. The transport in time then uses an exponential integrator. In \cite{Latu_2017} the model is split into a fast and a slow subsystem for all of which a backward Semi-Lagrangian scheme is used.

This paper follows the latter method but replaces the Semi-Lagrangian scheme in the slow subsystem (which does not involve the magnetic field) by an Arakawa scheme of order 4. The motivation behind this is to improve conservation properties for the most turbulent substep at the expense of either using a costly implicit time integrator, or an explicit one which is constrained by a CFL condition. For this method we test different orders and boundary conditions. This spatial discretization will be combined with either an implicit Crank-Nicolson integrator or an explicit Runge-Kutta scheme of order 4. The implementation proceeds on top of the \texttt{PyGyro} code \cite{pygyro_code} which is a \textit{Python} implementation of the gyro-kinetic model presented in \cite{Latu_2017} with the aim to replicate test cases of the \texttt{GYSELA} code from \cite{GRANDGIRARD2006395} and more recently \cite{Grandgirard_CPC2016}.

After an introduction of the gyro-kinetic equation in Section \ref{sec:gk-model}, we will describe the splitting ansatz of dividing the model in a slow and a fast subsystem, as well as the two aforementioned schemes in Section \ref{sec:splitting_discretization}. In section \ref{sec:num_exp}, numerical experiments will be presented, first as a verification of the Arakawa scheme in a 2-dimensional test case, and after that results for the full gyro-kinetic model. Concluding with Section \ref{sec:conclusion}, we discuss the benefits of this approach and perspectives.

