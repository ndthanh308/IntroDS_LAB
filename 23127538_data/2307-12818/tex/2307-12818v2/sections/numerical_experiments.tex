% !TeX spellcheck = en_GB

\section{Numerical Experiments}
\label{sec:num_exp}


While the Arakawa method was implemented in \textit{Python} in order to integrate it seamlessly into the \texttt{PyGyro} code, efforts were made to maximize performance in order to do large scale simulations with more than 300 million degrees of freedom in feasible times. This section is devoted to discuss the implementation of the Arakawa scheme, its integration to the \texttt{PyGyro} code, and further numerical experiments and verifications.




\subsection{Implementation of the Discrete Bracket}

Since the potential $\phi$ is constant while performing the poloidal step, which is defined on a polar domain similar to Section \ref{sec:polar}, and in order to create a computationally efficient scheme, we choose to implement the discrete bracket as a \texttt{scipy} sparse matrix $\mathbb{J}_\phi$ of size $(N_r N_\theta)^2$, that is constructed only dependent on the point-values of $\phi$, mapping the point-values of the current distribution function $f$, such that
\begin{equation}
	\mathbb{J}_\phi f_h = J^p_h(\phi_h, f_h) \approx \{\phi, f\}^p.
\end{equation}
After every time step, we update the non-zero entries of this matrix in-place using the new values of $\phi$. These entries are computed in a \textit{Fortran} routine which was generated using \texttt{pyccel} (see \cite{pyccel}) achieving near-native \textit{Fortran}-performance with much less development time. An explicit time integrator then uses matrix-vector multiplication which is computed in \textit{C} thanks to the usage of \texttt{scipy}. An implicit time stepping makes use of the implemented sparse solvers, also provided by \texttt{scipy}.\\

In order to test if the conserved quantities in equation \eqref{eq:pol_cons_quant} hold, we can directly calculate the equivalent algebraic conditions from equation \eqref{eq:alg_prop}, which should only depend on the definition of $J_h$ and are independent of the actual point values in $f_h$ and $\phi_h$ as long as they satisfy the boundary conditions.  

This is interesting with respect to the discussion of the conservation properties depending on the different BC in \cite{crouseilles2018exponential} and Section \ref{sec:BC}, we therefore implemented all BC discussed in Section \ref{sec:BC}, with additional possible periodic and Dirichlet BC in radial direction of the distribution function $f$. 
\begin{table}[h]
	\begin{tabular}{| l | l | l | l | l | l | l|l|l|}
		\hline
		BC	& Order & Mass & $L^2$-norm & Energy & Order & Mass & $L^2$-norm & Energy \\
		\hline
		Periodic& 2 &  1.47e-14 & 2.62e-14 & 1.50e-14 & 4 & 3.93e-14 & 5.57e-14 & 7.44e-14  \\ \hline
		Dirichlet& 2 &  1.35e-13 & 9.09e-13 & 8.67e-13 & 4 & 4.12e-13 & 4.37e-11 & 3.40e-12 \\ \hline
		Extrapolation& 2 &  8.53e-14 & 1.09e-11 & 4.57e-12 & 4 & 2.56e-13 & 2.55e-11 & 1.63e-11 \\
		\hline
	\end{tabular}
	\medskip
	\caption{Algebraic conservation properties, i.e. equation \eqref{eq:alg_prop}, for vectors of size $N_rN_\theta$, with $N_r = N_\theta = 64$, where $f_h \in \bR^{N_rN_\theta}$ and $\phi_h \in \bR^{N_rN_\theta}$ have uniformly distributed values between $-100$ and $100$, while satisfying the BC.}
	\label{tab:alg_prop}
\end{table}

The results of this first test can be found in Table \ref{tab:alg_prop}. Even though the values are not on machine-precision, as was predicted in \cite{crouseilles2018exponential} and which is due the imperfect conservation at the non-periodic boundaries, they are still very small, and we expect good conservation properties from using this scheme in the full code. We also observed, that these values increase proportionally to the norm of $\phi$, which may should be normalized in the indicator. 


\newpage

\subsection{Poloidal Advection Test-Case}

In order to further verify the AKW scheme and its properties, we look at a test-case similar to Section 3.5.2 in \cite{emily}, that is, we consider a poloidal advection on a polar domain with $r \in [1, 20]$ and $\theta \in [0, 2\pi]$. The equilibrium distribution function $f_\text{eq}$ reads
\begin{equation}
	f_\text{eq}(r, v_\parallel) = \frac{n_0(r)}{\sqrt{2\pi T_i(r)}} \exp\left( - \frac{v_\parallel^2}{2 T_i(r)} \right)
\end{equation}
with radial profiles
\begin{equation}
	\mathcal{P}(r) = C_{\mathcal{P}} \exp\left[ - \kappa_\mathcal{P} \delta r_\mathcal{P} \tanh \left( \frac{r - r_p}{\delta r_\mathcal{P}} \right) \right]
\end{equation}
for $\mathcal{P} \in \{T_i, n_0\}$ and with constants
\begin{align}
	C_{T_i} & = 1 & C_{n_0} & = \left(r_\text{max} - r_\text{min}\right) \left[\int_{r_\text{min}}^{r_\text{max}} \exp\left[ - \kappa_{n_0} \delta r_{n_0} \tanh \left( \frac{r - r_p}{\delta r_\mathcal{P}} \right) \right] \d r \right]^{-1}
\end{align}
and parameters
\begin{align*}
	B_0 & = 0 \, , & R_0 & = 239.8081535 \, , & r_\text{min} & = 0.1 \, , & r_\text{max} & = 14.5 \, , & r_p & = \frac{r_\text{max} - r_\text{min}}{2} \, , \\
	\epsilon & = 10^{-6} \, , & \kappa_{n_0} & = 0.055 \, , & \kappa_{T_i} & = 0.27586 \, , & \delta r_{T_i} & = \frac{\delta r_{n_0}}{2} = 1.45 \, , & \delta r & = \frac{4 \delta r_{n_0}}{\delta r_{T_i}} \, .
\end{align*}

Initial potential $\phi$ and initial distribution function $f$ are given by
\begin{subequations}
	\begin{align}
		\phi(r, \theta) & = -5 r^2 + \sin(\theta), \\
		f(t = 0, r, \theta) & = f_\text{eq}(r, \vp=0) + B(r, \theta), \\
		B(r, \theta) & = \begin{cases}
			\cos(\frac{\pi}{8} \sqrt{(r - 7)^2 + 2(\theta - \pi)^2}), & \text{if } \sqrt{(r - 7)^2 + 2(\theta - \pi)^2} \le 4, \\
			0, & \text{else}.
		\end{cases},
	\end{align}
\end{subequations}
where the radius-dependent equilibrium function $f_\text{eq}(r)$ is the same as in \cite{Latu_2017} with a fixed $\vp = 0$. For this system, the exact trajectories are given by
\begin{subequations}
	\begin{align}
		\theta(t) &= \theta_0 - 10 t, \\
		r(t) &= \sqrt{ r_0^2 + \frac15 \left( \sin\left(\theta_0 - 10 t\right) - \sin\left(\theta_0 \right)\right)},
	\end{align}
\end{subequations}
for initial points $(r_0, \theta_0)$. Following these characteristics with the initial distribution yields an exact solution to our problem, that can be used to calculate the error of our numerical scheme. The initial distribution is of Gaussian-like shape, as displayed in Figure \ref{fig:init_f}, and gets advected around the center while keeping its shape.

\begin{center}
	\begin{minipage}[h]{0.5\textwidth}
		\centering
		% Figure environment removed
	\end{minipage}\begin{minipage}[h]{0.5\textwidth}
		\centering
		% Figure environment removed
	\end{minipage}
\end{center}
\vspace{0.3cm}

We use the AKW scheme as introduced in Section \ref{sec:AKW} with extrapolatory BC, given by the equilibrium function, and different mesh sizes. The time integration is done by an explicit Runge-Kutta scheme of order $4$, where we take the step-size $\Delta t = 0.001 / N_r$ and perform a total number of $N = T_\text{end} / dt$ steps to the final time $T_\text{end} = 0.02$. This way, the time-discretization is fine enough, such that the dominant error is coming from the spatial discretization of the bracket only. The numerical results are listed in Table \ref{tab:num_exp}.
\begin{table}[h]
    \begin{tabular}{| l | l | l | l | l | l | l|l|l|}
    \hline
     $N_r$& $N_\theta$ & $L^2$-error & order & conserved mass & conserved $L^2$-norm & conserved energy \\
	\hline
	%8 & 8 & 8.07e-02 & &1.40e-06 & 2.24e-06 & 7.42e-09\\ \hline order:3.39
	16 & 16 & 7.72e-03 &  &9.20e-10 & 1.43e-09 & 9.56e-12\\ \hline
	32 & 32 & 5.62e-04 & 3.78 &8.06e-10 & 1.25e-09 & 5.79e-12\\ \hline
	64 & 64 & 3.96e-05 & 3.83 &1.01e-09 & 1.57e-09 & 6.49e-12\\ \hline
	128 & 128 & 4.74e-06 & 3.06 &1.04e-09 & 1.61e-09 & 5.82e-12\\ \hline
    \end{tabular}
	\medskip
	\caption{Spatial discretization and conservation errors for different mesh sizes $(N_r, N_\theta)$. The error of the solution with respect to the exact solution is calculated in the $L^2$-norm. The conservation errors are relative errors of the initial discrete quantities compared to their values at final time.}
	\label{tab:num_exp}
\end{table}
The error is not up to the desired order of $4$, this could be due to conflicting BC as discussed in \cite{crouseilles2018exponential}, but it is not far off and converges towards it for smaller time-steps. The conserved properties look similar to the ones in Table \ref{tab:alg_prop}, albeit being a few orders of magnitude higher which seems to be due to the bigger norm of $\phi$ and imperfect BC.  

As another test, we solve the system over a longer period of time and keep track of the mass, the $L^2$-norm and the energy. Taking $N_r = N_\theta = 64$, time-step $\Delta t = 0.01$ and perform $N = 500$ steps, yields relative errors of the conserved quantities and values of algebraic indicators as shown in Figure \ref{fig:cons}. %More analyis...

% Figure environment removed
The long time conservation seems to be in accordance with the results in Tables \ref{tab:num_exp} and \ref{tab:alg_prop}, albeit the algebraic indicators being larger than before due to the larger norm of $\phi$, as was observed before. 


\subsection{Full Gyro-kinetic Simulations}

The \texttt{PyGyro} code is a \textit{Python 3} library for gyro-kinetic simulations leveraging the acceleration provided by the modules \texttt{Pythran} (see \cite{Pythran}), \texttt{Numba} (see \cite{Numba}), or \texttt{Pyccel} (see \cite{pyccel}). It is highly parallelized using \texttt{MPI} and thus suitable for running even large-scale simulations on computing clusters. In the following, we will look at results from simulations of the full gyro-kinetic model of \cite{emily}, and then compare the new combination of Semi-Lagrangian and the Arakawa method to the results purely using the Semi-Lagrangian scheme. We will pay special attention to the energy conservation in the poloidal step. Since turbulences mostly occur in this substep, this is also where preserving energy is most crucial.

The parameters used are the same as those in the previous section. The grid size is as follows:
\begin{align*}
	& N_r = 128, && N_\theta = 256, &&& N_z = 128, &&&& N_{v_\parallel} = 72.
\end{align*}
The simulation also uses the following parameters for the initial perturbation:
\begin{align*}
	\iota & = 0, & m & \in \left\{5, 15\right\}, & n & = 1.
\end{align*}
The equilibrium function $f_\text{eq}$ as well as the radial profiles $\{T_i , T_e, n_0\}$ are defined in in the previous section.


Firstly we compare the Arakawa method to the second-order Semi-Lagrangian scheme by looking at the $L^2$-norm of the electric potential $\phi$, c.f. Figure \ref{fig:l2phi}. We observe the expected behaviour for both schemes where $\norm{\phi}_2$ follows the analytical growth rate of $\norm{\phi}_2 = 4\cdot 10^{-5} \times \exp\left(0.00354 t\right)$ very well in the linear regime (until $t \sim 3000$) (shown on the left in a semi-logarithmic plot). After that the $L^2$-norm saturates and settles at a value of around $8.0$. Notably, the two simulations agree very well even in the non-linear regime which is a good consistency check.

% Figure environment removed

Furthermore, we plot both the full distribution function and additionally its difference to the equilibrium distribution function in Figure \ref{fig:akwfullndiff}, up to $t=5000$ in steps of $1000$. Shown are slices at $z=0$ and $v_\parallel \simeq 0$ in a polar projection with variables $r$ and $\theta$. We see that it behaves as expected with turbulences forming after the linear regime is over, appearing in the plot for $t=4000$. Interesting to note is the fact that the difference to the equilibrium function seems to increase even for low-valued regions of the domain, although only by a relative difference of half a percentage point.
This effect also occurs in the simulation using the SL scheme, to an even larger extend, so it is not an artifact of the Arakawa scheme. 
%We can see that the same effect occurs in the simulation using the SL scheme, displayed in Figure \ref{fig:slfdiff}, so it is not an artifact of the Arakawa scheme. 
One point of discussion is, if this error is due to the boundary conditions, but compared to extrapolation BC, we see no improvement when switching to Dirichlet BC, in this case constant extrapolation by the boundary value, which seem to be the only other plausible option.

% % Figure environment removed

% % Figure environment removed

Our main point of comparison between the Arakawa method and the Semi-Lagrangian scheme is the conserved quantities in \eqref{conservation-properties}: the mass and $L^2$-norm of the distribution function, and the potential energy, as well as the kinetic energy
\begin{equation}
	E_\text{kin} = \frac{m}{2} \int \left(f(t, r, \theta, z, v_\parallel) - f_\text{eq}(r, v_\parallel)\right) v^2_\parallel \d r \d \theta \d z \d v_\parallel \, .
\end{equation} \\

From a simulation with the above grid size and time step-size $\Delta t = 1$, we investigate the conservation properties for the four above mentioned quantities by computing them before and after the poloidal advection step and plotting the absolute and relative errors in Figure \ref{fig:absnrelerrlog} on a semi-logarithmic scale.
\newpage

% Figure environment removed

\newpage

	% Figure environment removed

\newpage

We can clearly see that the Arakawa scheme preserves the conserved quantities much better than the Semi-Lagrangian scheme, although the error is of order of machine precision only in the linear phase. After that, the error grows but remains multiple orders of magnitude smaller than for the semi-Lagrangian scheme. The kinetic energy is also much better preserved which was to be expected, since the behaviour of conservation acts similarly to the mass of the distribution function but with $\vp^2$ weights on each $\vp$-slice, keeping in mind that $\vp$ is just a parameter in each poloidal advection step. The additional error that is introduced by the imperfect boundary conditions is responsible for the conservation properties being bigger than the machine precision; this is especially true for later times when the distribution function moves further away from the equilibrium that is assumed to hold outside the domain. \\

% % Figure environment removed

Lastly, we compare the two time-integrators in Figure \ref{fig:abserrlogakw} as described in the text: The second order Crank-Nicolson method and the fourth-order Runge-Kutta scheme, both for the full model simulation. For the RK4 scheme we have a CFL condition which reads
\begin{equation}
	CFL = \max\left(\mathbb{J}_\phi\right) \frac{\Dt}{\Delta x}
\end{equation}
This becomes restrictive only in the non-linear phase later on in the simulation. One can thus use the explicit RK4 with large time step-size $\Dt$, and either use the implicit CN2 method or the RK4 for the later time domain. By observing the accuracy in the discrete conservation of continuous invariants, we see that the difference between the two is negligible for early times, but the RK4 performs better for the very late phase.

% Figure environment removed


\newpage