% !TeX spellcheck = en_GB

\section{Conclusion}
\label{sec:conclusion}


In this work we studied a splitting for the gyro-kinetic equation in fast and slow subsystems where demonstrated that the Arakawa scheme yields superior conservation properties when implemented as the spatial discretization of the Poisson bracket defining the poloidal split-step, compared to a full Semi-Lagrangian scheme. This conservation concerns the mass and $L^2$-norm of the distribution function as well as the physical energies $E_\text{kin}$ and $E_\text{pot}$. The improvement was measured by computing the error occurring during the poloidal advection substep of the system. Although the mass and $L^2$-norm of the distribution function, as well as the potential energy are preserved up to machine precision only in the linear phase, the error is for all the quantities multiple orders of magnitude lower compared to the Semi-Lagrangian implementation. The usage of the more physical boundary conditions of extrapolating the distribution function with its equilibrium values outside the domain, has shown to be the right choice, albeit introducing a non-negligible error at later times. Manipulating the Arakawa scheme, to avoid imperfect boundary conditions and fall better in the framework of this problem, is left for future work. 
The comparison between the time-integrators, namely the second order Crank-Nicolson scheme and the 4th order Runge-Kutta method, showed that their behaviour is identical except for very late times, where strong turbulences occur.



