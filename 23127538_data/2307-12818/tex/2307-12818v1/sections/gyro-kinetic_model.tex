% !TeX spellcheck = en_GB


\section{The Gyro-kinetic Model}
\label{sec:gk-model}

In a Tokamak, the dynamics of particles consists of a slow motion along the magnetic field lines superimposed with a fast gyration around the magnetic field lines. This fast motion can be averaged out to reduce the dimension of phase space in order to make the models computationally more tractable, while keeping most of the important physics. The resulting models are called gyro-kinetic and will be described in the following.\\

The model we will look at is defined by the Lie-transformed, low-frequency particle Lagrangian $L$, where we follow the derivation in \cite{Bottino_Sonnendrucker_2015}, \cite{emily} and \cite{Latu_2017}. Given a static magnetic field $\bB$, its intensity $B = \norm{\bB}$ and its direction $\bb = \bB / B$, the particle charge $q \not= 0$ and the particle mass $m >0$,  
%and denoting $\bA$ as the vector-valued background potential
we are able to write the Lagrangian $L$ as
\begin{equation}\label{Lagrangian}
	L(t, \bx, \vp, \mu, \dot{\bx}, \dot{v}_\parallel, \dot{\mu}) = \left(\nabla \times q\bB + m \vp \bb \right)\cdot\dbx + \frac{m}{q} \mu \dot{\theta} - H(\bx, \vp),
\end{equation}
where $\bx \in \Omega \subseteq \bR^3$ is the position of the gyro-centre, $\vp \in \bR$ is the velocity parallel to the magnetic field lines, $\mu$ is the modified magnetic moment and $\theta$ the angle of cylindrical coordinates. The Hamiltonian $H$ will be introduced shortly.
Looking at the equation of motion of $\theta$, i.e. its Euler-Lagrange equation
\begin{equation}
	0 = \frac{\dd}{\dd t} \left( \frac{\partial L}{\partial \dot \theta} \right) - \left(
		\frac{\partial L}{\partial \theta}\right) =  \frac{\dd}{\dd t} \frac{m}{q} \mu,
\end{equation}
we can immediately conclude that $\mu$ is an exact invariant of the system, i.e.
\begin{equation}
	\diff{}{t} \mu = 0,
\end{equation}
and thus the phase-space is only $4$-dimensional.
The gyro-kinetic equation describing the gyro-centre distribution function $f=f(t,\bx, \vp)$ is of the form
\begin{equation}\label{drift-kinetic model}
	\pa{f}{t} + \bu \cdot \nabla f + \ap \pa{f}{\vp} = 0,
\end{equation}
which describes the positions of a collection of identical particles and whose exact solution is constant along the trajectories $(\bx(t), \vp(t))$ in the phase-space, i.e. 
\begin{equation}
	\frac{\dd}{\dd t} f(t,\bx(t), \vp(t)) = 0.
\end{equation}
In order to determine the equations of motion for ${\dd \bx}/{\dd t} = \bu$ and ${\dd \vp}/{\dd t} = \ap$, we look at the remaining Euler-Lagrange equations. Simplifying notations by defining
\begin{equation}
	\bB^\ast  \coloneqq \bB + \frac{m}{q}\vp \nabla \times \bb , \qquad
	\Bap  \coloneqq \bb \cdot \bB^\ast = B + \frac{m \vp}{q B} \bb \cdot \left( \nabla \times \bB \right),
\end{equation}
one can derive the characteristic trajectories (see \cite{Bottino_Sonnendrucker_2015} for a detailed derivation) from the remaining Euler-Lagrange equations of \eqref{Lagrangian}, which yield
\begin{subequations}
	\begin{align}
		\bu  & = \frac{1}{\Bap} \left( \frac{1}{m} \pa{H}{\vp} \bB^\ast + \frac{1}{q} \bb \times \nabla H \right) \label{eom for bu}, \\
		\ap & = \frac{1}{\Bap} \left( -\frac{1}{m} \bB^\ast \cdot  \nabla H  \right). \label{eom for ap}
	\end{align}
\end{subequations}
As noted in \cite{Latu_2017}, the phase space is divergence-free, i.e.
\begin{equation}
	\nabla \cdot \bu + \pa{\ap}{\vp} = 0,
\end{equation}
thus we can rewrite \eqref{drift-kinetic model} in conservative form
\begin{equation}\label{conservation1}
	\pa{}{t}\left(\Bap f\right) + \nabla\cdot\left( \Bap \bu f \right) + \pa{}{\vp} \left( \Bap \ap f \right) = 0.
\end{equation}

The equations of motion form a Hamiltonian system (see \cite{idomura2008conservative} for details) with the electrostatic gyro-centre Hamiltonian in the zero-Larmor-radius limit
\begin{equation}
	H(t, \bx, \vp) = \frac{1}{2} m \vp^2 + \mu B(\bx) + q \phi (t,\bx).
\end{equation}
Furthermore, we need a bracket for the Hamiltonian system which is the guiding-centre Poisson bracket
\begin{subequations}
	\begin{align}
		\gcbracket{F}{G} & = \frac{\bB}{m \Bap} \cdot \left( \left(\nabla F\right) \pa{G}{\vp} - \left(\nabla G\right) \pa{F}{\vp} \right) \label{fast-poisson} \\
		& \qquad + \frac{\vp}{q \Bap} \left( \nabla \times \bb \right) \cdot \left( \left(\nabla F\right) \pa{G}{\vp} - \left(\nabla G\right) \pa{G}{\vp} \right) \\
		& \qquad - \frac{1}{q \Bap} \bb \cdot \left[\left(\nabla F\right) \times \left(\nabla G\right)\right].
	\end{align}
\end{subequations}
We now split the Poisson bracket in parts containing $\bB$ (i.e. \eqref{fast-poisson}) which we expect to have fast dynamics, and other terms which will be called the slow subsystem.
%The part of the bracket responsible for the fast subsystem reads
%\begin{equation}
%	\bracket{F}{G}_{\text{fast}} = \frac{\bB}{m \Bap} \cdot \left( \left(\nabla F\right) \pa{G}{\vp} - \left(\nabla G\right) \pa{F}{\vp} \right)
%\end{equation}
%while for the slow subsystem it reads
%\begin{equation}
%	\begin{aligned}
%		\bracket{F}{G}_{\text{slow}} & = \frac{\vp}{q \Bap} \left( \nabla \times \bb \right) \cdot \left( \left(\nabla F\right) \pa{G}{\vp} - \left(\nabla G\right) \pa{G}{\vp} \right) \label{gc_lg} \\
%		& \qquad - \frac{1}{q \Bap} \bb \cdot \left[\left(\nabla F\right) \times \left(\nabla G\right)\right].
%	\end{aligned}
%\end{equation}
Thus the equations of motion for the fast and slow subsystems, with variables $\bu = \bu_f + \bu_s$ and $a_{\parallel} = a_{\parallel, f} + a_{\parallel, s}$, reads:
\begin{subequations}
	\begin{align}
		(\text{fast}) = & \left\{ \begin{aligned}
			\bu_f & = \frac{1}{\Bap} \frac{1}{m} \pa{H}{\vp} \bB, \\
			a_{\parallel, f} & = -\frac{1}{\Bap} \frac{1}{m} \bB \cdot \left( \nabla H \right),
		\end{aligned} \right. \label{split step fast}\\
		(\text{slow}) = &  \left\{ \begin{aligned}
			\bu_s & = \frac{1}{\Bap} \frac{1}{q} \left( \pa{H}{\vp} \vp \nabla \times \bb + \bb \times \left(\nabla H\right) \right), \\
			a_{\parallel, s} & = - \frac{a}{\Bap} \frac{1}{q} \vp \left( \nabla \times \bb \right) \cdot \left( \nabla H \right).
		\end{aligned} \right. \label{split step slow}
	\end{align}
\end{subequations}
This splitting is of particular interest when introducing a phase-space discretization, where the fast system trajectories may travel across many cells, which introduces a higher CFL number and thus needs a time-integration that handles this well, while the treatment of the slow subsystem naturally is less restrictive.

For a constant and uniform background magnetic field $\bB = B \hat \be_z$, where $\hat \be_z$ is the unit-vector pointing in the $z$-variable direction, the model simplifies to drift-kinetic equations and the subsystems become
\begin{subequations}
	\begin{align}
		(\text{fast}) = & \left\{ \begin{aligned}
			\bu_f & = {\vp} \bb, \\
			a_{\parallel, f} & = -\frac{q}{m} \bb \cdot \nabla \phi,
		\end{aligned} \right. \label{split step fast const B} \\
		(\text{slow}) = &  \left\{ \begin{aligned}
			\bu_s & = \frac{\bb \times \nabla \phi}{B}, \\
			a_{\parallel, s} & = 0.
		\end{aligned} \right. \label{split step slow const B}
	\end{align}
\end{subequations}
Continuing this simplification, we are able to rewrite the space variables to cylindrical coordinates as are introduced in \cite{Latu_2017}. Thus, we look for the distribution function $f = f(t, r, \theta, z, v_\parallel)$ satisfying
%TODO add sources?
\begin{equation}
	\partial_t f + \{\phi, f\} + v_\parallel \nabla_\parallel f - \nabla_\parallel \phi \partial_{v_\parallel}f = 0, \label{eq:GK_model}
\end{equation}
where $\nabla_\parallel = \bb \cdot \nabla$ and the bracket is transformed to polar coordinates, which reads, given the toroidal magnetic field $B_0$,
\begin{equation}
	\{\phi, f \} = \frac{1}{rB_0}\partial_r \phi\partial_\theta f -\frac{1}{rB_0}\partial_\theta \phi\partial_r f. \label{eq:bracket_in_polar}
\end{equation}
Since the plasma is quasi-neutral this equation is complemented by solving a elliptic quasi-neutrality (QN) equation for the self-consistent potential $\phi = \phi(t, r, \phi, z)$, i.e. solving for a given temperature profile $T_e$
\begin{equation}
	- \left[\partial_r^2 \phi + \left( \frac{1}{r} + \frac{\partial_r n_0}{n_0}\right)\partial_r \phi + \frac{1}{r^2} \partial_\theta^2 \phi \right] + \frac{1}{T_e} \phi = \int_{-\infty}^{\infty} (f - f_\text{eq}) \ \dd v_\parallel, \label{eq:qn}
\end{equation}
where the given (radial symmetric) equilibrium function $f_\text{eq}$ is a Gaussian and $n_0$ is a radial profile, which is the integral over $\vp$ of the equilibrium function. The initial distribution function $f(t=0, r, \theta, z, v_\parallel)$ is defined to be the equilibrium with a perturbation in some modes, which will result in an observable turbulence behaviour of the system after a certain amount of time. For more details and the exact definitions and constants, we refer to Section 4 in \cite{Latu_2017}. Normally these equations are defined on an infinitely long cylinder of some radius, but in order to discretize them, we cut the domain down to a finite cylinder with a hole in the middle and reduce the velocity space, such that   $(r, \theta, z, \vp) \in [r_\text{min}, r_\text{max}]\times[0, 2\pi]\times[0, 2\pi R_0] \times [-v_\text{max}, v_\text{max}]$, with parameters that will be made more precise in Section \ref{sec:num_exp}. To complete this system of equations, we briefly want to discuss the boundary conditions of this reduction. The distribution function $f$ is periodic along $\theta$, $z$, while having homogeneous Dirichlet boundaries in the $\vp$-direction. In the radial direction $r$, we assume the values are given by an outside equilibrium function. For the potential $\phi$, we assume periodic boundary conditions along $\theta$ and $z$. In $r$-direction, we decompose $\phi$ into Fourier modes at $r_\text{min}$, taking homogeneous Neumann boundary conditions for the zeroth mode and homogeneous Dirichlet boundary conditions for the others. At $r_\text{max}$, we just take plain homogeneous Dirichlet boundary conditions. \\


%Does this apply? Can you explain it to me?
%\subsubsection{Conserved Quantities}
Lastly, we want to take a look at physical properties of these equations. Since we have a transport equation in conservative form \eqref{conservation1} that conserves arbitrary functions of $f$  along non-linear characteristic trajectories, we can write the Casimir equation
\begin{equation}
	\dt C(f) + \bracket{C(f)}{H} =0,
\end{equation}
where the Casimir invariant $\int C( f ) $ is conserved. Therefore, the system has an infinite number of conserved quantities such as the particle number, the $L^1$-norm $\int f $, or the $L^2$-norm $\int f^2 $. In addition, the gyro-kinetic equations conserve the total energy $H = E_k + E_p$:
\begin{subequations}
	\begin{align}
		E_k &= \int f(\bx, \vp) \left[\frac{1}{2} m \vp^2 + \mu B(\bx)\right] \d \bx \d \vp \, , \\
		E_p &= \int q \mean{\phi}_\alpha (t,\bx) f(\bx, \vp) \d \bx \d \vp \, ,
	\end{align}
\end{subequations}
where $E_k$ is the kinetic energy and $E_f$ is the potential energy. For more details we refer the reader to \cite{idomura2008conservative}.